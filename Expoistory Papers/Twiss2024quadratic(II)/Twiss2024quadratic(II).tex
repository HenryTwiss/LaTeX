\documentclass[12pt,reqno,oneside]{amsart}
\usepackage{import}
%===============================%
%  Packages and basic settings  %
%===============================%
\usepackage[headheight=15pt,rmargin=0.5in,lmargin=0.5in,tmargin=0.75in,bmargin=0.75in]{geometry}
\usepackage{imakeidx}
\usepackage{framed}
\usepackage{amssymb}
\usepackage{amsmath}
\usepackage{mathrsfs}
\usepackage{enumitem}
\usepackage{hyperref}
\usepackage{appendix}
\usepackage[capitalise,noabbrev]{cleveref}
\usepackage{tikz}
\usepackage{tikz-cd}
\usepackage{nomencl}\makenomenclature
\usetikzlibrary{braids,arrows,decorations.markings,calc}

%====================================%
%  Theorems, environments & cleveref  %
%====================================%
\newtheorem{theorem}{Theorem}[section]
\newtheorem{proposition}{Proposition}[section]
\newtheorem{corollary}{Corollary}[section]
\newtheorem{lemma}{Lemma}[section]
\newtheorem{conjecture}{Conjecture}[section]
\newtheorem{remark}{Remark}[section]

\newenvironment{stabular}[2][1]
  {\def\arraystretch{#1}\tabular{#2}}
  {\endtabular}

%==================================%
%  Custom commands & environments  %
%==================================%
\newcommand{\legendre}[2]{\left(\frac{#1}{#2}\right)}
\newcommand{\dlegendre}[2]{\displaystyle{\left(\frac{#1}{#2}\right)}}
\newcommand{\tlegendre}[2]{\textstyle{\left(\frac{#1}{#2}\right)}}
\newcommand{\psum}{\sideset{}{'}\sum}
\newcommand{\asum}{\sideset{}{^{\ast}}\sum}
\newcommand{\tmod}[1]{\ \left(\text{mod }#1\right)}
\newcommand{\xto}[1]{\xrightarrow{#1}}
\newcommand{\xfrom}[1]{\xleftarrow{#1}}
\newcommand{\normal}{\mathrel{\unlhd}}
\newcommand{\mf}{\mathfrak}
\newcommand{\mc}{\mathcal}
\newcommand{\ms}{\mathscr}

\newcommand{\Mat}{\mathrm{Mat}}
\newcommand{\GL}{\mathrm{GL}}
\newcommand{\SL}{\mathrm{SL}}
\newcommand{\PSL}{\mathrm{PSL}}
\renewcommand{\O}{\mathrm{O}}
\newcommand{\SO}{\mathrm{SO}}
\newcommand{\U}{\mathrm{U}}
\newcommand{\Sp}{\mathrm{Sp}}

\newcommand{\N}{\mathbb{N}}
\newcommand{\Z}{\mathbb{Z}}
\newcommand{\Q}{\mathbb{Q}}
\newcommand{\R}{\mathbb{R}}
\newcommand{\C}{\mathbb{C}}
\newcommand{\F}{\mathbb{F}}
\renewcommand{\H}{\mathbb{H}}
\renewcommand{\P}{\mathbb{P}}

\renewcommand{\a}{\alpha}
\renewcommand{\b}{\beta}
\newcommand{\g}{\gamma}
\renewcommand{\d}{\delta}
\newcommand{\z}{\zeta}
\renewcommand{\t}{\theta}
\renewcommand{\i}{\iota}
\renewcommand{\k}{\kappa}
\renewcommand{\l}{\lambda}
\newcommand{\s}{\sigma}
\newcommand{\w}{\omega}

\newcommand{\G}{\Gamma}
\newcommand{\D}{\Delta}
\renewcommand{\L}{\Lambda}
\newcommand{\W}{\Omega}

\newcommand{\e}{\varepsilon}
\newcommand{\vt}{\vartheta}
\newcommand{\vphi}{\varphi}
\newcommand{\emt}{\varnothing}

\newcommand{\x}{\times}
\newcommand{\ox}{\otimes}
\newcommand{\op}{\oplus}
\newcommand{\bigox}{\bigotimes}
\newcommand{\bigop}{\bigoplus}
\newcommand{\del}{\partial}
\newcommand{\<}{\langle}
\renewcommand{\>}{\rangle}
\newcommand{\lf}{\lfloor}
\newcommand{\rf}{\rfloor}
\newcommand{\wtilde}{\widetilde}
\newcommand{\what}{\widehat}
\newcommand{\conj}{\overline}
\newcommand{\cchi}{\conj{\chi}}

\DeclareMathOperator{\id}{\textrm{id}}
\DeclareMathOperator{\sgn}{\mathrm{sgn}}
\DeclareMathOperator{\im}{\mathrm{im}}
\DeclareMathOperator{\rk}{\mathrm{rk}}
\DeclareMathOperator{\tr}{\mathrm{trace}}
\DeclareMathOperator{\nm}{\mathrm{norm}}
\DeclareMathOperator{\ord}{\mathrm{ord}}
\DeclareMathOperator{\Hom}{\mathrm{Hom}}
\DeclareMathOperator{\End}{\mathrm{End}}
\DeclareMathOperator{\Aut}{\mathrm{Aut}}
\DeclareMathOperator{\Tor}{\mathrm{Tor}}
\DeclareMathOperator{\Ann}{\mathrm{Ann}}
\DeclareMathOperator{\Gal}{\mathrm{Gal}}
\DeclareMathOperator{\Trace}{\mathrm{Trace}}
\DeclareMathOperator{\Norm}{\mathrm{Norm}}
\DeclareMathOperator{\Span}{\mathrm{Span}}
\DeclareMathOperator*{\Res}{\mathrm{Res}}
\DeclareMathOperator{\Vol}{\mathrm{Vol}}
\DeclareMathOperator{\Li}{\mathrm{Li}}
\renewcommand{\Re}{\mathrm{Re}}
\renewcommand{\Im}{\mathrm{Im}}

\newcommand{\GH}{\G\backslash\H}
\newcommand{\GG}{\G_{\infty}\backslash\G}

\newenvironment{psmallmatrix}
  {\left(\begin{smallmatrix}}
  {\end{smallmatrix}\right)}

%============%
%  Comments  %
%============%
\newcommand{\todo}[1]{\textcolor{red}{\sf Todo: [#1]}}

%===================%
%  Label reminders  %
%===================%
% [label=(\roman*)]
% [label=(\alph*)]
% [label=(\arabic{enumi})]

%==================%
%  Other settings  %
%==================%
\pgfdeclarelayer{background}
\pgfsetlayers{background,main}
\tikzset{->-/.style={decoration={
  markings,
  mark=at position .5 with {\arrow{>}}},postaction={decorate}}}

%=================%
%  Title & Index  %
%=================%
\title{A quadratic double Dirichlet series II: the number field case}
\author{Henry Twiss}
\date{2024}
\makeindex

\begin{document}

\begin{abstract}
    We construct a quadratic double Dirichlet series $Z(s,w)$ built from single variable quadratic Dirichlet $L$-functions $L(s,\chi)$ over $\Q$. We prove that $Z(s,w)$ admits meromorphic continuation to the $(s,w)$-plane and satisfies a group of functional equations.
\end{abstract}

\maketitle

\section{Preliminaries}
    We present an overview of quadratic Dirichlet $L$-functions over $\Q$. We begin with the Riemann zeta-function. The zeta function $\z(s)$ is defined as the Dirichlet series or Euler product
    \[
        \z(s) = \sum_{m \ge 1}\frac{1}{m^{s}} = \prod_{\text{$p$ prime}}\left(1-\frac{1}{p^{s}}\right)^{-1},
    \]
    for $\Re(s) > 1$. The second equality is an analytic reformulation of the fundamental theorem of arithmetic. The Riemann zeta function also admits meromorphic continuation to $\C$ with a simple pole at $s = 1$ of residue $1$. The functional equation is
    \[
        \pi^{-\frac{s}{2}}\G\left(\frac{s}{2}\right)\z(s) = \pi^{-\frac{1-s}{2}}\G\left(\frac{1-s}{2}\right)\z(1-s).
    \]
    Now we recall characters on $\Z$. They are multiplicative functions $\chi:\Z \to \C$. They form a group under multiplication. The two flavors we will care about are:
    
    \begin{itemize}
        \item Dirichlet characters: multiplicative functions $\chi_{d}:\Z \to \C$ modulo $d \ge 1$ (in that they are $d$-periodic) and such that $\chi_{d}(m) = 0$ if $(m,d) > 1$.
        \item Hilbert characters: The group of characters generated by those that appear in the sign change of reciprocity statements.
    \end{itemize}
    
    The image of a Dirichlet character always lands in the roots of unity. If $\chi$ is a Dirichlet character then its conjugate $\conj{\chi}$ is also a Dirichlet character. Moreover, $\conj{\chi}$ is the multiplicative inverse to $\chi$ and the Dirichlet characters modulo $m$ form a group under multiplication. This group is always finite and its order is $\phi(d) = |(\Z/d\Z)^{\ast}|$. Dirichlet characters also satisfy orthogonality relations:

    \begin{theorem}[Orthogonality relations]
        \phantom{ }
        \begin{enumerate}[label=(\roman*)]
          \item For any two Dirichlet characters $\chi$ and $\psi$ modulo $d$,
          \[
            \frac{1}{\phi(d)}\psum_{a \tmod{d}}\chi(a)\conj{\psi}(a) = \d_{\chi,\psi}.
          \]
          \item For any $a,b \in (\Z/d\Z)^{\ast}$,
          \[
            \frac{1}{\phi(d)}\sum_{\chi \tmod{d}}\chi(a)\cchi(b) = \d_{a,b}.
          \]
        \end{enumerate}
    \end{theorem}

    The Dirichlet characters that are of interest to us are those given by the quadratic residue symbol on $\Z$. First let us recall this symbol. For any odd prime $p$ and any $m \ge 1$, we define the quadratic residue symbol $\tlegendre{m}{p}$ by
    \[
        \legendre{m}{p} \equiv m^{\frac{p-1}{2}} \tmod{p} = \begin{cases} 1 & \text{if $x^{2} \equiv m \tmod{p}$ is solvable}, \\ -1 & \text{if $x^{2} \equiv m \tmod{p}$ is not solvable}, \\ 0 & \text{if $m \equiv 0 \tmod{p}$}. \end{cases}
    \]
    This symbol only depends upon $m$ modulo $p$ and is multiplicative in $m$. We can extend the quadratic residue symbol multiplicatively in the denomator. If $d = p_{1}^{e_{1}}p_{2}^{e_{2}} \cdots p_{k}^{e_{k}}$ is the prime factorization of $d$, then we define
    \[
        \legendre{m}{d} = \prod_{1 \le i \le k}\legendre{m}{p_{i}}^{e_{i}}.
    \]
    So the quadratic residue symbol now makes sense for any odd $d \ge 1$. We can extend this symbol further and allow $d \ge 1$ to be even. To this end, we define
    \[
        \legendre{m}{2} = \begin{cases} 1 & \text{if $m \equiv 1,7 \tmod{8}$}, \\ -1 & \text{if $m \equiv 3,5 \tmod{8}$}, \\ 0 & \text{if $m \equiv 0 \tmod{2}$}, \end{cases}
    \]
    and extend $\tlegendre{m}{d}$ multiplatively in $d$ when $d$ is even. Now the quadratic residue symbol makes sense for any $m,d \ge 1$. Moreover, it is multiplicative in both $m$ and $d$ but no longer depends upon only $m$ modulo $d$ (it also depends upon $m$ modulo $8$). In particular,
    \[
        \legendre{-1}{d} = \begin{cases} 1 & \text{$d \equiv 1 \tmod{4}$}, \\ -1 & \text{$d \equiv 3 \tmod{4}$}, \\ 0 & \text{$d \equiv 0 \tmod{2}$}, \end{cases} \quad \text{and} \quad \legendre{2}{d} = \begin{cases} 1 & \text{$d \equiv 1,7 \tmod{8}$}, \\ -1 & \text{$d \equiv 3,5 \tmod{8}$}, \\ 0 & \text{$d \equiv 0 \tmod{2}$}, \end{cases}
    \]
    and if $d \not\equiv 0 \tmod{2}$, we can compactly write
    \[
        \legendre{-1}{d} = (-1)^{\frac{d-1}{2}} = \begin{cases} 1 & \text{$d \equiv 1 \tmod{4}$}, \\ -1 & \text{$d \equiv 3 \tmod{4}$}, \end{cases} \quad \text{and} \quad \legendre{2}{d} = (-1)^{\frac{d^{2}-1}{8}} = \begin{cases} 1 & \text{$d \equiv 1,7 \tmod{8}$}, \\ -1 & \text{$d \equiv 3,5 \tmod{8}$}. \end{cases}
    \]
     The quadratic residue symbol also admits the following reciprocity law:

    \begin{theorem}[Quadratic reciprocity]
        If $d,m \ge 1$ are relatively prime, then
        \[
            \legendre{d}{m} = (-1)^{\frac{d^{(2)}-1}{2}\frac{m^{(2)}-1}{2}}\legendre{m}{d},
        \]
        where $d^{(2)}$ and $m^{(2)}$ are the parts of $d$ and $m$ relatively prime to $2$ respectively.
    \end{theorem}

    We can now define the quadratic Dirichlet characters. For any odd square-free $d \ge 1$, define the quadratic Dirichlet character $\chi_{d}$ by the following quadratic residue symbol:
    \[
        \chi_{d}(m) = \begin{cases} \legendre{d}{m} & \text{if $d \equiv 1 \tmod{4}$}, \\ \legendre{4d}{m} & \text{if $d \equiv 2,3 \tmod{4}$}. \end{cases}
    \]
    This quadratic Dirichlet character is attached to the quadratic extension $\Q(\sqrt{d})$. We extend $\chi_{d}$ multiplicatively in the denominator so that $\chi_{d}$ makes sense for any odd $d \ge 1$. In particular, $\chi_{d}(m) = \pm1$ provided $d$ and $m$ are relatively prime and $\chi_{d}(m) = 0$ if $(m,d) > 1$. Quadratic reciprocity implies that $\chi_{d}$ is a Dirichlet character modulo $d$ if $d \equiv 1 \tmod{4}$ and is a Dirichlet character modulo $4d$ if $d \equiv 2,3 \tmod{4}$. Indeed, if $d \equiv 1 \tmod{4}$ then $d^{(2)} = d$ and the sign is always $1$. If $d \equiv 3 \tmod{4}$, then $d^{(2)} = d$ and the sign is $\tlegendre{-1}{m}$ which is a character modulo $4$. If $d \equiv 2 \tmod{4}$, then $d^{(2)} \equiv 1,3 \tmod{4}$ and we are reduced to one of the previous two cases.

    We now discuss the Hilbert characters. We will only need four of them: the quadratic Dirichlet characters modulo $8$. We define them as follows:
    \begin{gather*}
        \psi_{1}(m) = \begin{cases} 1 &\text{if $m \not\equiv 0 \tmod{2}$}, \\ 0 &\text{if $m \equiv 0 \tmod{2}$}, \end{cases} \quad \psi_{-1}(m) = \begin{cases} 1 &\text{if $m \equiv 1 \tmod{4}$}, \\ -1 &\text{if $m \equiv 3 \tmod{4}$}, \\ 0 &\text{if $m \equiv 0 \tmod{2}$}, \end{cases} \\ \psi_{2}(m) = \begin{cases} 1 &\text{if $m \equiv 1,7 \tmod{8}$}, \\ -1 &\text{if $m \equiv 3,5 \tmod{8}$}, \\ 0 &\text{if $m \equiv 0 \tmod{2}$}, \end{cases} \quad \psi_{-2}(m) = \begin{cases} 1 &\text{if $m \equiv 1,3 \tmod{8}$}, \\ -1 &\text{if $m \equiv 5,7 \tmod{8}$}, \\ 0 &\text{if $m \equiv 0 \tmod{2}$}. \end{cases}
    \end{gather*}
    We can write $\psi_{-1}$ and $\psi_{2}$ in terms of Legendre symbols:
    \[
        \psi_{-1}(m) = \legendre{-1}{m} \quad \text{and} \quad \psi_{2}(m) = \legendre{m}{2}.
    \]
    Moreover, these characters satisfy the relations
    \[
        \psi_{-2}(m) = \psi_{-1}(m)\psi_{2}(m), \quad \psi_{1}(m) = \psi_{-1}(m)\psi_{-1}(m), \quad \text{and} \quad \psi_{-1}(m) = \psi_{2}(m)\psi_{-2}(m).
    \]
    Notice that if $d \equiv 1,2,5 \tmod{8}$, then $d^{(2)} \equiv 1 \tmod{4}$ so that the sign in the statement of quadratic recipricty is $1$. If $d \equiv 3,6,7 \tmod{8}$ (recall that we are assuming $d$ is square free so that $d \not\equiv 0,4 \tmod{8}$) then $d^{(2)} \equiv 3 \tmod{4}$ and the sign is $(-1)^{\frac{m^{(2)}-1}{2}}$. This fact together with the relations for the quadratic characters modulo $8$ imply
    \[
        \chi_{d}(m) = \begin{cases} \chi_{m}(d) & \text{if $d \equiv 1 \tmod{4}$}, \\ \chi_{-1}(m)\chi_{m}(d) & \text{if $d \equiv 3 \tmod{4}$}, \\ \chi_{2}(m)\chi_{m}\left(\frac{d}{2}\right) & \text{if $d \equiv 2 \tmod{8}$}, \\ \chi_{-2}(m)\chi_{m}\left(\frac{d}{2}\right) & \text{if $d \equiv 6 \tmod{8}$}. \end{cases}
    \]
    With the Dirichlet characters and Hilbert characters introduced, we are ready to discuss the $L$-functions associated to quadratic Dirichlet characters. We define the $L$-function $L(s,\chi_{d})$ attached to $\chi_{d}$ by a Dirichlet series or Euler product:
    \[
        L(s,\chi_{d}) = \sum_{m \ge 1}\frac{\chi_{d}(m)}{|m|^{s}} = \prod_{p \text{ prime}}\left(1-\frac{\chi_{d}(P)}{|P|^{s}}\right)^{-1}.
    \]
    By definition of the quadratic Dirichlet character, $L(s,\chi_{d}) \ll \z(s)$ for $\Re(s) > 1$ so that $L(s,\chi_{d})$ is locally absolutely uniformly convergent in this region. $L(s,\chi_{d})$ also admits meromorphic continuation to $\C$ with a simple pole at $s = 1$ if $d$ is a perfect square. For square-free $d$, the completed $L$-function is defined as
    \[
        L^{\ast}(s,\chi_{d}) = \begin{cases} \pi^{-\frac{s}{2}}\G\left(\frac{s}{2}\right)L(s,\chi_{d}) & \text{if $\chi_{d}$ is even}, \\ \pi^{-\frac{s}{2}}\G\left(\frac{s+1}{2}\right)L(s,\chi_{d}) & \text{if $\chi_{d}$ is odd}, \end{cases}
    \]
    and satisfies the functional equation
    \[
        L^{\ast}(s,\chi_{d}) = \begin{cases} \e_{\chi}q^{\frac{1}{2}-s}L^{\ast}(1-s,\chi_{d}) & \text{if $\chi_{d}$ is even}, \\ -\e_{\chi}q^{\frac{1}{2}-s}L^{\ast}(1-s,\chi_{d}) & \text{if $\chi_{d}$ is odd}. \end{cases}
    \]
    Note that the gamma factors depend upon the partiy of $\chi_{d}$. This the root cause of an important technical issue later when deriving functional equations for the quadratic double Dirichlet series.
\section*{The Quadratic Double Dirichlet Series}
\section*{The Interchange}
\section*{Weighting Terms}
\section*{Functional Equations}
\section*{Meromorphic Continuation}
\section*{Poles and Residues}

\begin{thebibliography}{99}
    \bibitem{R}
    Rosen, M. (2002). Number theory in function fields (Vol. 210). Springer Science \& Business Media.

    \bibitem{H}
    Hormander, L. (1973). An introduction to complex analysis in several variables. Elsevier.

    \bibitem{CG}
    Chinta, G., \& Gunnells, P. E. (2007). Weyl group multiple Dirichlet series constructed from quadratic characters. Inventiones mathematicae, 167, 327-353.

    \bibitem{S}
    Stanley, R. (2023). Enumerative Combinatorics: Volume 2. Cambridge University Press.
\end{thebibliography}

\end{document}