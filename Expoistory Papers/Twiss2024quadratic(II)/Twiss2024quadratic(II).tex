\documentclass[12pt,reqno,oneside]{amsart}
\usepackage{import}
%===============================%
%  Packages and basic settings  %
%===============================%
\usepackage[headheight=15pt,rmargin=0.5in,lmargin=0.5in,tmargin=0.75in,bmargin=0.75in]{geometry}
\usepackage{fancyhdr}
\usepackage{imakeidx}
\usepackage{framed}
\usepackage{amssymb}
\usepackage{amsmath}
\usepackage{mathrsfs}
\usepackage{enumitem}
\usepackage{multirow}
\usepackage{hyperref}
\usepackage[capitalise,noabbrev]{cleveref}
\usepackage{appendix}
\usepackage[hyperref,amsthm,amsmath,thref,framed,thmmarks]{ntheorem}
\usepackage{tikz}
\usepackage{tikz-cd}
\usepackage{nomencl}\makenomenclature
\usetikzlibrary{braids,arrows,decorations.markings,calc}

%=======================%
%  Book style settings  %
%=======================%
\pagestyle{fancy}
\fancyhf{}
\fancyhead[L]{\nouppercase{\leftmark}}
\fancyfoot[C]{\thepage}
\setlength\parindent{0pt}
\raggedbottom

%====================================%
%  Theorems, environments & cleveref  %
%====================================%
\theoremstyle{plain}\newtheorem{theorem}{Theorem}[section]
\theoremstyle{nonumberplain}\renewtheorem{theorem*}{Theorem}
\theoremstyle{plain}\newtheorem{proposition}[theorem]{Proposition}
\theoremstyle{nonumberplain}\renewtheorem{proposition*}{Proposition}
\theoremstyle{plain}\newtheorem{corollary}[theorem]{Corollary}
\theoremstyle{nonumberplain}\renewtheorem{corollary*}{Corollary}
\theoremstyle{plain}\newtheorem{lemma}[theorem]{Lemma}
\theoremstyle{nonumberplain}\renewtheorem{lemma*}{Lemma}
\theoremstyle{plain}\newtheorem{conjecture}[theorem]{Conjecture}
\theoremstyle{nonumberplain}\renewtheorem{conjecture*}{Conjecture}
\theoremstyle{plain}\newtheorem{remark}[theorem]{Remark}
\theoremstyle{nonumberplain}\renewtheorem{remark*}{Remark}
\theoremstyle{plain}\newtheorem{problem}[theorem]{Open Problem}
\theoremstyle{nonumberplain}\renewtheorem{problem*}{Open Problem}
\theoremstyle{plain}\newtheorem{heuristic}[theorem]{Heuristic}
\theoremstyle{nonumberplain}\renewtheorem{heuristic*}{Heuristic}
\crefname{conjecture}{Conjecture}{Conjectures}

\newenvironment{stabular}[2][1]
  {\def\arraystretch{#1}\tabular{#2}}
  {\endtabular}

%==================================%
%  Custom commands & environments  %
%==================================%
\newcommand{\legendre}[2]{\left(\frac{#1}{#2}\right)}
\newcommand{\dlegendre}[2]{\displaystyle{\left(\frac{#1}{#2}\right)}}
\newcommand{\tlegendre}[2]{\textstyle{\left(\frac{#1}{#2}\right)}}
\newcommand{\psum}{\sideset{}{'}\sum}
\newcommand{\asum}{\sideset{}{^{\ast}}\sum}
\newcommand{\tmod}[1]{\ (\mathrm{mod}\text{ }#1)}
\renewcommand{\bmod}[1]{\ \left(\mathrm{mod}\text{ }#1\right)}
\newcommand{\xto}[1]{\xrightarrow{#1}}
\newcommand{\xfrom}[1]{\xleftarrow{#1}}
\newcommand{\normal}{\mathrel{\unlhd}}
\newcommand{\mf}{\mathfrak}
\newcommand{\mc}{\mathcal}
\newcommand{\ms}{\mathscr}

\newcommand{\Mat}{\mathrm{Mat}}
\newcommand{\GL}{\mathrm{GL}}
\newcommand{\SL}{\mathrm{SL}}
\newcommand{\PSL}{\mathrm{PSL}}
\renewcommand{\O}{\mathrm{O}}
\newcommand{\SO}{\mathrm{SO}}
\newcommand{\U}{\mathrm{U}}
\newcommand{\Sp}{\mathrm{Sp}}

\newcommand{\N}{\mathbb{N}}
\newcommand{\Z}{\mathbb{Z}}
\newcommand{\Q}{\mathbb{Q}}
\newcommand{\R}{\mathbb{R}}
\newcommand{\C}{\mathbb{C}}
\newcommand{\F}{\mathbb{F}}
\renewcommand{\H}{\mathbb{H}}
\renewcommand{\P}{\mathbb{P}}

\renewcommand{\a}{\alpha}
\renewcommand{\b}{\beta}
\newcommand{\g}{\gamma}
\renewcommand{\d}{\delta}
\newcommand{\z}{\zeta}
\renewcommand{\t}{\theta}
\renewcommand{\i}{\iota}
\renewcommand{\k}{\kappa}
\renewcommand{\l}{\lambda}
\newcommand{\s}{\sigma}
\newcommand{\w}{\omega}

\newcommand{\G}{\Gamma}
\newcommand{\D}{\Delta}
\renewcommand{\L}{\Lambda}
\newcommand{\W}{\Omega}
\newcommand{\scL}{\mathscr{L}}

\newcommand{\e}{\varepsilon}
\newcommand{\vt}{\vartheta}
\newcommand{\vphi}{\varphi}
\newcommand{\emt}{\varnothing}

\newcommand{\x}{\times}
\newcommand{\ox}{\otimes}
\newcommand{\op}{\oplus}
\newcommand{\bigox}{\bigotimes}
\newcommand{\bigop}{\bigoplus}
\newcommand{\del}{\partial}
\newcommand{\<}{\langle}
\renewcommand{\>}{\rangle}
\newcommand{\lf}{\lfloor}
\newcommand{\rf}{\rfloor}
\newcommand{\wtilde}{\widetilde}
\newcommand{\what}{\widehat}
\newcommand{\conj}{\overline}
\newcommand{\cchi}{\conj{\chi}}

\DeclareMathOperator{\id}{\textrm{id}}
\DeclareMathOperator{\sgn}{\mathrm{sgn}}
\DeclareMathOperator{\im}{\mathrm{im}}
\DeclareMathOperator{\rk}{\mathrm{rk}}
\DeclareMathOperator{\adj}{\mathrm{adj}}
\DeclareMathOperator{\tr}{\mathrm{trace}}
\DeclareMathOperator{\nm}{\mathrm{norm}}
\DeclareMathOperator{\disc}{\mathrm{disc}}
\DeclareMathOperator{\ord}{\mathrm{ord}}
\DeclareMathOperator{\sym}{\mathrm{sym}}
\DeclareMathOperator{\ext}{\mathrm{ext}}
\DeclareMathOperator{\Hom}{\mathrm{Hom}}
\DeclareMathOperator{\End}{\mathrm{End}}
\DeclareMathOperator{\Aut}{\mathrm{Aut}}
\DeclareMathOperator{\Tor}{\mathrm{Tor}}
\DeclareMathOperator{\Ann}{\mathrm{Ann}}
\DeclareMathOperator{\Gal}{\mathrm{Gal}}
\DeclareMathOperator{\Trace}{\mathrm{Tr}}
\DeclareMathOperator{\Norm}{\mathrm{N}}
\DeclareMathOperator{\Cl}{\mathrm{Cl}}
\DeclareMathOperator{\Span}{\mathrm{Span}}
\DeclareMathOperator*{\Res}{\mathrm{Res}}
\DeclareMathOperator{\Vol}{\mathrm{Vol}}
\DeclareMathOperator{\Li}{\mathrm{Li}}
\DeclareMathOperator{\Supp}{\mathrm{Supp}}
\renewcommand{\Re}{\mathrm{Re}}
\renewcommand{\Im}{\mathrm{Im}}
\DeclareMathOperator{\Ph}{\mathrm{Ph}}
\DeclareMathOperator{\SC}{\mathrm{SC}}


\newcommand{\GH}{\G\backslash\H}
\newcommand{\GG}{\G_{\infty}\backslash\G}

\newenvironment{psmallmatrix}
  {\left(\begin{smallmatrix}}
  {\end{smallmatrix}\right)}

\newcommand{\smc}[1]{
    \mathchoice
    {{\scriptstyle\mathcal{#1}}}
    {{\scriptstyle\mathcal{#1}}}
    {{\scriptscriptstyle\mathcal{#1}}}
    {\scalebox{0.7}{$\scriptscriptstyle\mathcal{#1}$}}
}

%============%
%  Comments  %
%============%
\newcommand{\todo}[1]{\textcolor{red}{\sf Todo: [#1]}}

%===================%
%  Label reminders  %
%===================%
% [label=(\roman*)]
% [label=(\alph*)]
% [label=(\arabic{enumi})]

%==================%
%  Other settings  %
%==================%
\pgfdeclarelayer{background}
\pgfsetlayers{background,main}
\tikzset{->-/.style={decoration={
  markings,
  mark=at position .5 with {\arrow{>}}},postaction={decorate}}}

%=================%
%  Title & Index  %
%=================%
\title{A quadratic double Dirichlet series II: the number field case}
\author{Henry Twiss}
\date{2024}
\makeindex

\begin{document}

\begin{abstract}
    We construct a quadratic double Dirichlet series $Z(s,w)$ built from single variable quadratic Dirichlet $L$-functions $L(s,\chi)$ over $\Q$. We prove that $Z(s,w)$ admits meromorphic continuation to the $(s,w)$-plane and satisfies a group of functional equations.
\end{abstract}

\maketitle

\section{Preliminaries}
    We present an overview of quadratic Dirichlet $L$-functions over $\Q$. We begin with the Riemann zeta-function. The zeta function $\z(s)$ is defined as the Dirichlet series or Euler product
    \[
        \z(s) = \sum_{m \ge 1}\frac{1}{m^{s}} = \prod_{\text{$p$ prime}}\left(1-\frac{1}{p^{s}}\right)^{-1},
    \]
    for $\Re(s) > 1$. The second equality is an analytic reformulation of the fundamental theorem of arithmetic. The Riemann zeta function also admits meromorphic continuation to $\C$ with a simple pole at $s = 1$ of residue $1$. The functional equation is
    \[
        \pi^{-\frac{s}{2}}\G\left(\frac{s}{2}\right)\z(s) = \pi^{-\frac{1-s}{2}}\G\left(\frac{1-s}{2}\right)\z(1-s).
    \]
    Now we recall characters on $\Z$. They are multiplicative functions $\chi:\Z \to \C$. The two flavors of characters of interest to use are:
    
    \begin{itemize}
        \item Dirichlet characters: multiplicative functions $\chi_{m}:\Z \to \C$ modulo $m \ge 1$ (in that they are $m$-periodic) and such that $\chi_{m}(d) = 0$ if $(d,m) > 1$.
        \item Hilbert symbols: Dirichlet characters modulo $1$.
    \end{itemize}
    
    In either case, the image always lands in the roots of unity. If $\chi$ is a Dirichlet character then its conjugate $\conj{\chi}$ is also a Dirichlet character. Moreover, $\conj{\chi}$ is the multiplicative inverse to $\chi$ and the Dirichlet characters modulo $m$ form a group under multiplication. This group is always finite and its order is $\phi(m) = |(\Z/m\Z)^{\ast}|$. Dirichlet characters also satisfy orthogonality relations:

    \begin{theorem}[Orthogonality relations]
        \phantom{ }
        \begin{enumerate}[label=(\roman*)]
          \item For any two Dirichlet characters $\chi$ and $\psi$ modulo $m$,
          \[
            \frac{1}{\phi(m)}\psum_{a \tmod{m}}\chi(a)\conj{\psi}(a) = \d_{\chi,\psi}.
          \]
          \item For any $a,b \in (\Z/m\Z)^{\ast}$,
          \[
            \frac{1}{\phi(m)}\sum_{\chi \tmod{m}}\chi(a)\cchi(b) = \d_{a,b}.
          \]
        \end{enumerate}
    \end{theorem}

    The Dirichlet characters that are of interest to us are those given by the quadratic residue symbol on $\F_{q}[t]$. First let us recall this symbol. For any irreducible $p \in \F_{q}[t]$ and any $m \in \F_{q}[t]$, we define the quadratic residue symbol $\tlegendre{m}{p}$ by
    \[
        \legendre{m}{p} \equiv m^{\frac{|p|-1}{2}} \tmod{p} \in \{0,1,-1\}.
    \]
    In other words, $\tlegendre{m}{p} = 0,1,-1$ accoding to whether $p \mid m$, $m$ is a square modulo $p$, or $m$ is not a square modulo $p$. This symbol is multiplicative in $m$. Moreover, if $b \in \F_{q}^{\ast}$, we have
    \[
        \legendre{b}{p} = \sgn(b)^{\deg(p)}.
    \]
    where $\sgn(b) = \pm1$ depending on if $b \in (\F_{q}^{\x})^{2}$ or not. For $m \in \F_{q}[t]$ we define $\sgn(m) = \sgn(b_{n})$ if $m(t) = b_{n}t^{n}+b_{n-1}t^{n+1}+\cdots+b_{0}$ (with $b_{n} \neq 0$). We can extend the quadratic residue symbol multiplicatively in the denomator. If $d = bp_{1}^{e_{1}}p_{2}^{e_{2}} \cdots p_{k}^{e_{k}}$ is the prime factorization of $d$ (with $b \in \F_{q}^{\ast}$), then we define
    \[
        \legendre{m}{d} = \prod_{1 \le i \le k}\legendre{m}{p_{i}}^{e_{i}}.
    \]
    So the quadratic residue symbol now makes sense for any non-zero $d \in \F_{q}[t]$. The quadratic residue symbol also has the following reciprocity property:

    \begin{theorem}[Quadratic reciprocity]
        If $d,m \in \F_{q}[t]$ are monic, relatively prime, and non-zero, then
        \[
            \legendre{d}{m} = (-1)^{\frac{q-1}{2}\deg(d)\deg(m)}\legendre{m}{d}.
        \]
    \end{theorem}

    Note that if $q \equiv 1 \tmod{4}$, the sign in the statement of quadratic reciprocity is always $1$ so that reciprocity is perfect. We can now define the quadratic Dirichlet characters. For any non-zero $d \in \F_{q}[t]$, define the quadratic Dirichlet character $\chi_{d}$ by the following quadratic residue symbol:
    \[
        \chi_{d}(m) = \legendre{d}{m} = m^{\frac{|d|-1}{2}} \pmod{d},
    \]
    for any non-zero monic $m \in \F_{q}[t]$. Then $\chi_{d}(m) \in \{\pm 1\}$ provided $d$ and $m$ are relatively prime and $\chi_{d}(m) = 0$ if $(m,d) > 1$. Note that by quadratic reciprocity, $\chi_{d}$ is a Dirichlet character modulo $d$ if $d$ is monic. Since the quadratic residue symbols are multiplicative, $\chi_{d}$ is multiplicative in $d$. Therefore, factoring out a constant if necessary, we may always force $d$ to be monic. Moreover, for $b \in \F_{q}^{\x}$, we see that $\chi_{b}$ is a Hilbert symbol:
    \[
        \chi_{b}(m) = \legendre{b}{m} = \sgn(b)^{\deg(m)},
    \]
    where $m \in \F_{q}[t]$ is a non-zero monic. The only Hilbet symbols we will need are those given by the quadratic reisdue symbol. There are only two of them: one nontrivial and one trivial. The nontrivial Hilbert symbol is $\chi_{\t}$ where $\t \in \F^{\x}-(\F^{\x})^{2}$:
    \[
        \chi_{\t}(m) = (-1)^{\deg(m)},
    \]
    where $m \in \F_{q}[t]$ is a non-zero monic. Note that $\conj{\chi_{\t}} = \chi_{\t}$. The other Hilbert symbol is the trivial character $\chi_{\t}^{2} = \chi_{\t\t} = \chi_{1}$. In general, we denote a Hilbert symbol by $\chi_{a}$ where $a \in \{1,\t\}$.
    
    With the Dirichlet characters and Hilbert symbols introduced, we are ready to discuss the $L$-functions associated to quadratic Dirichlet characters. We define the $L$-function $L(s,\chi_{d})$ attached to $\chi_{d}$ by a Dirichlet series or Euler product:
    \[
        L(s,\chi_{d}) = \sum_{\text{$m$ monic}}\frac{\chi_{d}(m)}{|m|^{s}} = \prod_{\text{$P$ monic irr}}\left(1-\frac{\chi_{d}(P)}{|P|^{s}}\right)^{-1}.
    \]
    By definition of the quadratic Dirichlet character, $L(s,\chi_{d}) \ll \z(s)$ for $\Re(s) > 1$ so that $L(s,\chi_{d})$ is locally absolutely uniformly convergent in this region. $L(s,\chi_{d})$ also admits meromorphic continuation to $\C$ with a simple pole at $s = 1$ if $d$ is a perfect square and is analytic otherwise (see \cite{rosen2002number} for a proof). Moreover, $L(s,\chi_{d})$ is a polynomial in $q^{-s}$ of degree at most $\deg(d)-1$. The completed $L$-function is defined as follows:
    \[
        L^{\ast}(s,\chi_{d}) = \begin{cases} \frac{1}{1-q^{-s}}L(s,\chi_{d}) & \text{if $\deg(d)$ is even}, \\ L(s,\chi_{d}) & \text{if $\deg(d)$ is odd}, \end{cases}
    \]
    and satisfies the functional equation
    \[
        L^{\ast}(s,\chi_{d}) = \begin{cases} q^{2s-1}|d|^{\frac{1}{2}-s}L^{\ast}(1-s,\chi_{d}) & \text{if $\deg(d)$ is even}, \\ q^{2s-1}(q|d|)^{\frac{1}{2}-s}L^{\ast}(1-s,\chi_{d}) & \text{if $\deg(d)$ is odd}. \end{cases}
    \]
    Note that in the case $\deg(d)$ is even, the conductor is $|d|$ and in the case $\deg(d)$ is odd, the conductor is $q|d|$. In other words, the gamma factors depend upon the degree of $d$. This will cause a small but important technical issue later when we want to derive functional equations for the quadratic double Dirichlet series.
\section*{The Quadratic Double Dirichlet Series}
\section*{The Interchange}
\section*{Weighting Terms}
\section*{Funcational Equations}
\section*{Meromorphic Continuation}
\section*{Poles and Residues}

\begin{thebibliography}{99}
    \bibitem{R}
    Rosen, M. (2002). Number theory in function fields (Vol. 210). Springer Science \& Business Media.

    \bibitem{H}
    Hormander, L. (1973). An introduction to complex analysis in several variables. Elsevier.

    \bibitem{CG}
    Chinta, G., \& Gunnells, P. E. (2007). Weyl group multiple Dirichlet series constructed from quadratic characters. Inventiones mathematicae, 167, 327-353.

    \bibitem{S}
    Stanley, R. (2023). Enumerative Combinatorics: Volume 2. Cambridge University Press.
\end{thebibliography}

\end{document}