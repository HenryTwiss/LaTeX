\documentclass[12pt,reqno,oneside]{amsart}
\usepackage{import}
%===============================%
%  Packages and basic settings  %
%===============================%
\usepackage[headheight=15pt,rmargin=0.5in,lmargin=0.5in,tmargin=0.75in,bmargin=0.75in]{geometry}
\usepackage{imakeidx}
\usepackage{framed}
\usepackage{amssymb}
\usepackage{amsmath}
\usepackage{mathrsfs}
\usepackage{enumitem}
\usepackage{hyperref}
\usepackage{appendix}
\usepackage[capitalise,noabbrev]{cleveref}
\usepackage{tikz}
\usepackage{tikz-cd}
\usepackage{nomencl}\makenomenclature
\usetikzlibrary{braids,arrows,decorations.markings,calc}

%====================================%
%  Theorems, environments & cleveref  %
%====================================%
\newtheorem{theorem}{Theorem}[section]
\newtheorem{proposition}{Proposition}[section]
\newtheorem{corollary}{Corollary}[section]
\newtheorem{lemma}{Lemma}[section]
\newtheorem{conjecture}{Conjecture}[section]
\newtheorem{remark}{Remark}[section]

\newenvironment{stabular}[2][1]
  {\def\arraystretch{#1}\tabular{#2}}
  {\endtabular}

%==================================%
%  Custom commands & environments  %
%==================================%
\newcommand{\legendre}[2]{\left(\frac{#1}{#2}\right)}
\newcommand{\dlegendre}[2]{\displaystyle{\left(\frac{#1}{#2}\right)}}
\newcommand{\tlegendre}[2]{\textstyle{\left(\frac{#1}{#2}\right)}}
\newcommand{\psum}{\sideset{}{'}\sum}
\newcommand{\asum}{\sideset{}{^{\ast}}\sum}
\newcommand{\tmod}[1]{\ \left(\text{mod }#1\right)}
\newcommand{\xto}[1]{\xrightarrow{#1}}
\newcommand{\xfrom}[1]{\xleftarrow{#1}}
\newcommand{\normal}{\mathrel{\unlhd}}
\newcommand{\mf}{\mathfrak}
\newcommand{\mc}{\mathcal}
\newcommand{\ms}{\mathscr}

\newcommand{\Mat}{\mathrm{Mat}}
\newcommand{\GL}{\mathrm{GL}}
\newcommand{\SL}{\mathrm{SL}}
\newcommand{\PSL}{\mathrm{PSL}}
\renewcommand{\O}{\mathrm{O}}
\newcommand{\SO}{\mathrm{SO}}
\newcommand{\U}{\mathrm{U}}
\newcommand{\Sp}{\mathrm{Sp}}

\newcommand{\N}{\mathbb{N}}
\newcommand{\Z}{\mathbb{Z}}
\newcommand{\Q}{\mathbb{Q}}
\newcommand{\R}{\mathbb{R}}
\newcommand{\C}{\mathbb{C}}
\newcommand{\F}{\mathbb{F}}
\renewcommand{\H}{\mathbb{H}}
\renewcommand{\P}{\mathbb{P}}

\renewcommand{\a}{\alpha}
\renewcommand{\b}{\beta}
\newcommand{\g}{\gamma}
\renewcommand{\d}{\delta}
\newcommand{\z}{\zeta}
\renewcommand{\t}{\theta}
\renewcommand{\i}{\iota}
\renewcommand{\k}{\kappa}
\renewcommand{\l}{\lambda}
\newcommand{\s}{\sigma}
\newcommand{\w}{\omega}

\newcommand{\G}{\Gamma}
\newcommand{\D}{\Delta}
\renewcommand{\L}{\Lambda}
\newcommand{\W}{\Omega}

\newcommand{\e}{\varepsilon}
\newcommand{\vt}{\vartheta}
\newcommand{\vphi}{\varphi}
\newcommand{\emt}{\varnothing}

\newcommand{\x}{\times}
\newcommand{\ox}{\otimes}
\newcommand{\op}{\oplus}
\newcommand{\bigox}{\bigotimes}
\newcommand{\bigop}{\bigoplus}
\newcommand{\del}{\partial}
\newcommand{\<}{\langle}
\renewcommand{\>}{\rangle}
\newcommand{\lf}{\lfloor}
\newcommand{\rf}{\rfloor}
\newcommand{\wtilde}{\widetilde}
\newcommand{\what}{\widehat}
\newcommand{\conj}{\overline}
\newcommand{\cchi}{\conj{\chi}}

\DeclareMathOperator{\id}{\textrm{id}}
\DeclareMathOperator{\sgn}{\mathrm{sgn}}
\DeclareMathOperator{\im}{\mathrm{im}}
\DeclareMathOperator{\rk}{\mathrm{rk}}
\DeclareMathOperator{\tr}{\mathrm{trace}}
\DeclareMathOperator{\nm}{\mathrm{norm}}
\DeclareMathOperator{\ord}{\mathrm{ord}}
\DeclareMathOperator{\Hom}{\mathrm{Hom}}
\DeclareMathOperator{\End}{\mathrm{End}}
\DeclareMathOperator{\Aut}{\mathrm{Aut}}
\DeclareMathOperator{\Tor}{\mathrm{Tor}}
\DeclareMathOperator{\Ann}{\mathrm{Ann}}
\DeclareMathOperator{\Gal}{\mathrm{Gal}}
\DeclareMathOperator{\Trace}{\mathrm{Trace}}
\DeclareMathOperator{\Norm}{\mathrm{Norm}}
\DeclareMathOperator{\Span}{\mathrm{Span}}
\DeclareMathOperator*{\Res}{\mathrm{Res}}
\DeclareMathOperator{\Vol}{\mathrm{Vol}}
\DeclareMathOperator{\Li}{\mathrm{Li}}
\renewcommand{\Re}{\mathrm{Re}}
\renewcommand{\Im}{\mathrm{Im}}

\newcommand{\GH}{\G\backslash\H}
\newcommand{\GG}{\G_{\infty}\backslash\G}

\newenvironment{psmallmatrix}
  {\left(\begin{smallmatrix}}
  {\end{smallmatrix}\right)}

%============%
%  Comments  %
%============%
\newcommand{\todo}[1]{\textcolor{red}{\sf Todo: [#1]}}

%===================%
%  Label reminders  %
%===================%
% [label=(\roman*)]
% [label=(\alph*)]
% [label=(\arabic{enumi})]

%==================%
%  Other settings  %
%==================%
\pgfdeclarelayer{background}
\pgfsetlayers{background,main}
\tikzset{->-/.style={decoration={
  markings,
  mark=at position .5 with {\arrow{>}}},postaction={decorate}}}

%=================%
%  Title & Index  %
%=================%
\title{A quadratic double Dirichlet series II: the number field case}
\author{Henry Twiss}
\date{2024}
\makeindex

\begin{document}

\begin{abstract}
    We construct a quadratic double Dirichlet series $Z(s,w)$ built from single variable quadratic Dirichlet $L$-functions $L(s,\chi)$ over $\Q$. We prove that $Z(s,w)$ admits meromorphic continuation to the $(s,w)$-plane and satisfies a group of functional equations.
\end{abstract}

\maketitle

\section{Preliminaries}
    We present an overview of quadratic Dirichlet $L$-functions over $\Q$. We begin with the Riemann zeta-function. The zeta function $\z(s)$ is defined as the Dirichlet series or Euler product
    \[
        \z(s) = \sum_{m \ge 1}\frac{1}{m^{s}} = \prod_{\text{$p$ prime}}\left(1-\frac{1}{p^{s}}\right)^{-1},
    \]
    for $\Re(s) > 1$. The second equality is an analytic reformulation of the fundamental theorem of arithmetic. The Riemann zeta function also admits meromorphic continuation to $\C$ with a simple pole at $s = 1$ of residue $1$. The functional equation is
    \[
        \pi^{-\frac{s}{2}}\G\left(\frac{s}{2}\right)\z(s) = \pi^{-\frac{1-s}{2}}\G\left(\frac{1-s}{2}\right)\z(1-s).
    \]
    Now we recall characters on $\Z$. They are multiplicative functions $\chi:\Z \to \C$ and form a group under multiplication. The two flavors we will care about are:
    
    \begin{itemize}
        \item Dirichlet characters: multiplicative functions $\chi_{d}:\Z \to \C$ modulo $d \ge 1$ (in that they are $d$-periodic) and such that $\chi_{d}(m) = 0$ if $(m,d) > 1$.
        \item Hilbert characters: The group of characters generated by those that appear in the sign change of reciprocity statements.
    \end{itemize}
    
    The image of a Dirichlet character always lands in the roots of unity. Moreover, $\conj{\chi}$ is the multiplicative inverse to $\chi$ and the Dirichlet characters modulo $d$ form a subgroup under multiplication. This group is always finite and its order is $\phi(d) = |(\F_{q}[t]/d\F_{q}[t])^{\x}|$. The Dirichlet characters that are of interest to us are those given by the quadratic residue symbol on $\Z$. First let us recall this symbol. For any odd prime $p$ and any $d \in \Z$, we define the quadratic residue symbol $\tlegendre{d}{p}$ by
    \[
        \legendre{d}{p} \equiv d^{\frac{p-1}{2}} \tmod{p} = \begin{cases} 1 & \text{if $x^{2} \equiv d \tmod{p}$ is solvable}, \\ -1 & \text{if $x^{2} \equiv d \tmod{p}$ is not solvable}, \\ 0 & \text{if $d \equiv 0 \tmod{p}$}. \end{cases}
    \]
    This symbol only depends upon $d$ modulo $p$ and is multiplicative in $d$. We can extend the quadratic residue symbol multiplicatively in the denominator. First we define
    \[
        \legendre{d}{-1} = \begin{cases} 1 & \text{if $d \ge 0$}, \\ -1 & \text{if $d < 0$}, \end{cases} \quad \text{and} \quad \legendre{d}{2} = \begin{cases} 1 & \text{if $d \equiv 1,7 \tmod{8}$}, \\ -1 & \text{if $d \equiv 3,5 \tmod{8}$}, \\ 0 & \text{if $d \equiv 0 \tmod{2}$}. \end{cases}
    \]
    If $m = up_{1}^{e_{1}}p_{2}^{e_{2}} \cdots p_{k}^{e_{k}}$ is the prime factorization of $m$ (with $u = \pm1$), then we define
    \[
        \legendre{d}{m} = \legendre{d}{u}\prod_{1 \le i \le k}\legendre{d}{p_{i}}^{e_{i}}.
    \]
    The quadratic residue symbol now makes sense for any $m \in \Z$ and is multiplicative in both $d$ and $m$. The quadratic residue symbol also admits the following reciprocity law:

    \begin{theorem}[Quadratic reciprocity]
        If $d,m \in \Z$, then
        \[
            \legendre{d}{m} = (-1)^{\frac{d^{(2)}-1}{2}\frac{m^{(2)}-1}{2}}\legendre{m}{|d|},
        \]
        where $d^{(2)}$ and $m^{(2)}$ are the parts of $d$ and $m$ relatively prime to $2$ respectively.
    \end{theorem}

    Moreover, we have the additional relations
    \[
        \legendre{-1}{m} = (-1)^{\frac{m^{(2)}-1}{2}} \quad \text{and} \quad \legendre{2}{m} = (-1)^{\frac{m^{2}-1}{8}},
    \]
    and if $m \not\equiv 0 \tmod{2}$, we can write
    \[
        \legendre{-1}{m} = (-1)^{\frac{m-1}{2}} = \begin{cases} 1 & \text{$m \equiv 1 \tmod{4}$}, \\ -1 & \text{$m \equiv 3 \tmod{4}$}, \end{cases} \quad \text{and} \quad \legendre{2}{m} = (-1)^{\frac{m^{2}-1}{8}} = \begin{cases} 1 & \text{$m \equiv 1,7 \tmod{8}$}, \\ -1 & \text{$m \equiv 3,5 \tmod{8}$}. \end{cases}
    \]

    We can now define the quadratic Dirichlet characters. For any square-free $d \in \Z$, define the quadratic Dirichlet character $\chi_{d}$ by the following quadratic residue symbol:
    \[
        \chi_{d}(m) = \begin{cases} \legendre{d}{m} & \text{if $d \equiv 1 \tmod{4}$}, \\ \legendre{4d}{m} & \text{if $d \equiv 2,3 \tmod{4}$}. \end{cases}
    \]
    This quadratic Dirichlet character is attached to the quadratic extension $\Q(\sqrt{d})$. We extend $\chi_{d}$ multiplicatively in the denominator so that $\chi_{d}$ makes sense for any odd $d$. In particular, $\chi_{d}(m) = \pm1$ provided $d$ and $m$ are relatively prime and $\chi_{d}(m) = 0$ if $(m,d) > 1$. Quadratic reciprocity implies that $\chi_{d}$ is a Dirichlet character modulo $|d|$ if $d \equiv 1 \tmod{4}$ and is a Dirichlet character modulo $|4d|$ if $d \equiv 2,3 \tmod{4}$. Indeed, if $d \equiv 1 \tmod{4}$ then $d^{(2)} = d$ and the sign is always $1$. If $d \equiv 3 \tmod{4}$, then $d^{(2)} = d$ and the sign is $\tlegendre{-1}{m}$ which is a character modulo $4$. If $d \equiv 2 \tmod{4}$, then $d^{(2)} \equiv 1,3 \tmod{4}$ and we are reduced to one of the previous two cases. We will also set
    \[
        q(d) = \begin{cases} |d| & \text{if $d \equiv 1 \tmod{4}$}, \\ |4d| & \text{if $d \equiv 2,3 \tmod{4}$}, \end{cases} \quad \text{and} \quad \e_{\chi_{d}} = \frac{\tau(\chi_{d})}{\sqrt{q(d)}} = \begin{cases} 1 & \text{if $d \equiv 1 \tmod{4}$}, \\ 1+i & \text{if $d \equiv 2,3 \tmod{4}$}, \end{cases}
    \]
    where $\tau(\chi_{d})$ is the Gauss sum attached to $\chi_{d}$. We will also require an associated character. For each $\chi_{m}$ (here we are purposely interchanging the roles of $d$ and $m$ to keep consistency with the notation when discussing the quadratic double Dirichlet series later), we define $\wtilde{\chi}_{m}$ by
    \[
        \wtilde{\chi}_{m}(d) = (-1)^{\frac{m^{(2)}-1}{2}\frac{d^{(2)}-1}{2}}\chi_{m}(d).
    \]
    By quadratic reciprocity, $\wtilde{\chi}_{m}$ is a quadratic Dirichlet character of the same modulus as $\chi_{m}$ and is multiplicative in $m$. We now discuss the Hilbert characters. We will only need four of them: the quadratic Dirichlet characters modulo $8$. They are given as follows:
    \begin{gather*}
        \chi_{1}(m) = \begin{cases} 1 &\text{if $m \not\equiv 0 \tmod{2}$}, \\ 0 &\text{if $m \equiv 0 \tmod{2}$}, \end{cases} \quad \chi_{-1}(m) = \begin{cases} 1 &\text{if $m \equiv 1 \tmod{4}$}, \\ -1 &\text{if $m \equiv 3 \tmod{4}$}, \\ 0 &\text{if $m \equiv 0 \tmod{2}$}, \end{cases} \\ \chi_{2}(m) = \begin{cases} 1 &\text{if $m \equiv 1,7 \tmod{8}$}, \\ -1 &\text{if $m \equiv 3,5 \tmod{8}$}, \\ 0 &\text{if $m \equiv 0 \tmod{2}$}, \end{cases} \quad \chi_{-2}(m) = \begin{cases} 1 &\text{if $m \equiv 1,3 \tmod{8}$}, \\ -1 &\text{if $m \equiv 5,7 \tmod{8}$}, \\ 0 &\text{if $m \equiv 0 \tmod{2}$}. \end{cases}
    \end{gather*}
    In general, we will denote a Hilbert character by $\chi_{a}$ with $a \in \{\pm1,\pm2\}$. The Hilbert characters also satisfy an important orthogonality property:

    \begin{theorem}[Orthogonality of Hilbert characters]
        If $d,m \in \Z$ are odd, then
        \[
            \frac{1}{4}\sum_{a \in \{\pm1,\pm2\}}\chi_{a}(dm) = \begin{cases} 1 & \text{if $d \equiv m \tmod{8}$}, \\ 0 & \text{if $d \not\equiv m \tmod{8}$}. \end{cases}
        \]
    \end{theorem}

    Also, we have the identities
    \[
        \wtilde{\chi}_{a}(m) = \chi_{a}(m), \quad \chi_{-1}(m) = \legendre{-1}{m}, \quad \text{and} \quad \chi_{2}(m) = \legendre{2}{m},
    \]
    and the relations
    \[
        \chi_{-2}(m) = \chi_{-1}(m)\chi_{2}(m), \quad \chi_{1}(m) = \chi_{-1}(m)\chi_{-1}(m), \quad \text{and} \quad \chi_{-1}(m) = \chi_{2}(m)\chi_{-2}(m).
    \]
    We now return to $\chi_{d}$ for square-free $d$. If $d \equiv 1,2,5 \tmod{8}$, then $d^{(2)} \equiv 1 \tmod{4}$ so that the sign in the statement of quadratic reciprocity is $1$. If $d \equiv 3,6,7 \tmod{8}$, then $d^{(2)} \equiv 3 \tmod{4}$ and the sign is $(-1)^{\frac{m^{(2)}-1}{2}}$. This fact together with the relations for the quadratic characters modulo $8$ imply
    \[
        \chi_{d}(m) = \begin{cases} \chi_{m}(d) & \text{if $d \equiv 1 \tmod{4}$}, \\ \chi_{-1}(m)\chi_{m}(d) & \text{if $d \equiv 3 \tmod{4}$}, \\ \chi_{2}(m)\chi_{m}\left(\frac{d}{2}\right) & \text{if $d \equiv 2 \tmod{8}$}, \\ \chi_{-2}(m)\chi_{m}\left(\frac{d}{2}\right) & \text{if $d \equiv 6 \tmod{8}$}. \end{cases}
    \]
    With the Dirichlet and Hilbert characters introduced, we are ready to discuss the $L$-functions associated to quadratic Dirichlet characters. We define the $L$-function $L(s,\chi_{d})$ attached to $\chi_{d}$ for square-free $d$, by a Dirichlet series or Euler product:
    \[
        L(s,\chi_{d}) = \sum_{m \ge 1}\frac{\chi_{d}(m)}{m^{s}} = \prod_{p \text{ prime}}\left(1-\frac{\chi_{d}(p)}{p^{s}}\right)^{-1}.
    \]
    By definition of the quadratic Dirichlet character, $L(s,\chi_{d}) \ll \z(s)$ for $\Re(s) > 1$ so that $L(s,\chi_{d})$ is locally absolutely uniformly convergent in this region. $L(s,\chi_{d})$ also admits analytic continuation to $\C$. The completed $L$-function $L^{\ast}(s,\chi_{d})$ is defined as
    \[
        L^{\ast}(s,\chi_{d}) = \begin{cases} \pi^{-\frac{s}{2}}\G\left(\frac{s}{2}\right)L(s,\chi_{d}) & \text{if $d > 0$}, \\ \pi^{-\frac{s}{2}}\G\left(\frac{s+1}{2}\right)L(s,\chi_{d}) & \text{if $d < 0$}, \end{cases}
    \]
    and satisfies the functional equation
    \[
        L^{\ast}(s,\chi_{d}) = \begin{cases} q(d)^{\frac{1}{2}-s}L^{\ast}(1-s,\chi_{d}) & \text{if $d > 0$ and $d \equiv 1 \tmod{4}$}, \\ (1+i)q(d)^{\frac{1}{2}-s}L^{\ast}(1-s,\chi_{d}) & \text{if $d > 0$ and $d \equiv 2,3 \tmod{4}$}, \\ -q(d)^{\frac{1}{2}-s}L^{\ast}(1-s,\chi_{d}) & \text{if $d < 0$ and $d \equiv 1 \tmod{4}$} \\ (1+i)q(d)^{\frac{1}{2}-s}L^{\ast}(1-s,\chi_{d}) & \text{if $d < 0$ and $d \equiv 2,3 \tmod{4}$}. \end{cases}
    \]
    Note that the gamma factors depend upon the partiy of $\chi_{d}$. This the root cause of an important technical issue later when deriving functional equations for the quadratic double Dirichlet series. Analogously, the Dirichlet $L$-function $L(w,\wtilde{\chi}_{m})$ attached to $\wtilde{\chi}_{m}$ for square-free $m$ is defined by a Dirichlet series or Euler product:
    \[
        L(w,\wtilde{\chi}_{m}) = \sum_{d \ge 1}\frac{\wtilde{\chi}_{m}(d)}{d^{w}} = \prod_{p \text{ prime}}\left(1-\frac{\wtilde{\chi}_{m}(p)}{p^{w}}\right)^{-1}.
    \]
    As for $L(s,\chi_{d})$, $L(w,\wtilde{\chi}_{m}) \ll \z(w)$ for $\Re(w) > 1$ so that $L(w,\wtilde{\chi}_{m})$ is locally absolutely uniformly convergent in this region. Moreover, $L(w,\wtilde{\chi}_{m})$ admits analytic continuation to $\C$ and the completed $L$-function $L^{\ast}(w,\wtilde{\chi}_{m})$ is defined as
    \[
        L^{\ast}(w,\wtilde{\chi}_{m}) = \begin{cases} \pi^{-\frac{w}{2}}\G\left(\frac{w}{2}\right)L(w,\wtilde{\chi}_{m}) & \text{if $m \equiv 1,2,5 \tmod{8}$}, \\ \pi^{-\frac{w}{2}}\G\left(\frac{w+1}{2}\right)L(w,\wtilde{\chi}_{m}) & \text{if $m \equiv 3,6,7 \tmod{8}$}, \end{cases}
    \]
    and satisfies the functional equation
    \[
        L^{\ast}(w,\wtilde{\chi}_{m}) = \begin{cases} \e_{\wtilde{\chi}_{m}}q(m)^{\frac{1}{2}-w}L^{\ast}(1-w,\wtilde{\chi}_{m}) & \text{if $m \equiv 1,2,5 \tmod{8}$}, \\ -\e_{\wtilde{\chi}_{m}}q(m)^{\frac{1}{2}-w}L^{\ast}(1-w,\wtilde{\chi}_{m}) & \text{if $m \equiv 3,6,7 \tmod{8}$}. \end{cases}
    \]

    \begin{remark}
        The definitions for $L(s,\chi_{d})$, $L^{\ast}(s,\chi_{d})$, $L(w,\wtilde{\chi}_{m})$, and $L^{\ast}(w,\wtilde{\chi}_{m})$ work perfectly well even when $d$ and $m$ are not square-free (however the functional equations do not hold). We purposely do not define these $L$-functions, yet, for $d$ and $m$ not necessarily square-free.
    \end{remark}
\section*{The Quadratic Double Dirichlet Series}
    We will now define the quadratic double Dirichlet series $Z(s,w)$. For any integer $d \ge 1$, write $d = d_{0}d_{1}^{2}$ where $d_{0}$ is square-free. Equivalently, $d_{0}$ is the square-free part of $d$ and $\frac{d}{d_{0}}$ is a perfect square. The \textbf{quadratic double Dirichlet series} $Z(s,w)$ is defined as
    \[
        Z(s,w) = \sum_{\text{$d$ odd}}\frac{L^{(2)}(s,\chi_{d_{0}})Q_{d_{0}d_{1}^{2}}(s)}{d^{w}},
    \]
    where $Q_{d_{0}d_{1}^{2}}(s)$ is the \textbf{correction polynomial} defined by
    \[
        Q_{d_{0}d_{1}^{2}}(s) = \sum_{e_{1}e_{2} \mid d_{1}}\mu(e_{1})\chi_{d_{0}}(e_{1})e_{1}^{-s}e_{2}^{1-s} = \sum_{e_{1}e_{2}e_{3} = d_{1}}\mu(e_{1})\chi_{d_{0}}(e_{1})e_{1}^{-s}e_{2}^{1-s},
    \]
    and $\mu$ is the usual M\"obius function. For $\Re(s) > 1$, there is the trivial estimate
    \[
        Q_{d_{0}d_{1}^{2}}(s) \ll \sum_{e_{1}e_{2} \mid d_{1}}1 \ll \s_{0}(d_{1})^{2} \ll_{\e} d_{1}^{2\e} \ll_{\e} d^{\e},
    \]
     for any $\e > 0$. As $L(s,\chi_{d_{0}}) \ll 1$ for $\Re(s) > 1$, $Z(s,w)$ is locally absolutely uniformly convergent in the region $\L = \{(s,w) \in \C^{2}:\Re(s) > 1, \Re(w) > 1\}$. It will also be necessary to consider quadratic double Dirichlet series twisted by a pair of Hilbert characters $\chi_{a_{1}}$ and $\chi_{a_{2}}$. The \textbf{quadratic double Dirichlet series} $Z_{a_{1},a_{2}}(s,w)$ twisted by $\chi_{a_{1}}$ and $\chi_{a_{2}}$ is defined as
    \[
        Z_{a_{1},a_{2}}(s,w) = \sum_{\text{$d$ odd}}\frac{L^{(2)}(s,\chi_{a_{1}d_{0}})\chi_{a_{2}}(d)Q_{d_{0}d_{1}^{2}}(s,\chi_{a_{1}})}{d^{w}},
    \]
    where $Q_{d_{0}d_{1}^{2}}(s,\chi_{a_{1}})$ is the \textbf{correction polynomial} twisted by $\chi_{a_{1}}$ defined by
    \[
        Q_{d_{0}d_{1}^{2}}(s,\chi_{a_{1}}) = \sum_{e_{1}e_{2} \mid d_{1}}\mu(e_{1})\chi_{a_{1}d_{0}}(e_{1})e_{1}^{-s}e_{2}^{1-2s} = \sum_{e_{1}e_{2}e_{3} = d_{1}}\mu(e_{1})\chi_{a_{1}d_{0}}(e_{1})e_{1}^{-s}e_{2}^{1-2s},
    \]
    and $\mu$ is the usual M\"obius function. By definition of the Hilbert characters, we have the analogous bound $Q_{d_{0}d_{1}^{2}}(s,\chi_{a_{1}}) \ll d_{\e}$ so that $Z_{a_{1},a_{2}}(s,w)$ converges locally absolutely uniformly in the same region as $Z(s,w)$ does. In particular, $Z(s,w) = Z_{1,1}(s,w)$. We will also require quadratic double Dirichlet series correpsonding to the characters $\wtilde{\chi}_{m}$. Analogously writing $m = m_{0}m_{1}^{2}$, the \textbf{quadratic double Dirichlet series} $\wtilde{Z}(s,w)$ is defined as
    \[
        \wtilde{Z}(w,s) = \sum_{\text{$m$ odd}}\frac{L^{(2)}(w,\wtilde{\chi}_{m_{0}})Q_{m_{0}m_{1}^{2}}(w)}{m^{s}},
    \]
    where $Q_{d_{0}d_{1}^{2}}(w)$ is the \textbf{correction polynomial} defined by
    \[
        Q_{m_{0}m_{1}^{2}}(w) = \sum_{e_{1}e_{2} \mid m_{1}}\mu(e_{1})\chi_{m_{0}}(e_{1})e_{1}^{-w}e_{2}^{1-w} = \sum_{e_{1}e_{2}e_{3} = m_{1}}\mu(e_{1})\chi_{m_{0}}(e_{1})e_{1}^{-w}e_{2}^{1-w},
    \]
    and $\mu$ is the usual M\"obius function. We have the analogous estimate $Q_{m_{0}m_{1}^{2}}(w) \ll_{\e} m^{\e}$ and as $L(w,\wtilde{\chi}_{m_{0}}) \ll 1$ for $\Re(s) > 1$, $\wtilde{Z}(w,s)$ is locally absolutely uniformly convergent in the same region as $Z(s,w)$. We also need to consider twists by a pair of Hilbert characters $\wtilde{\chi}_{a_{2}}$ and $\wtilde{\chi}_{a_{1}}$. The \textbf{quadratic double Dirichlet series} $\wtilde{Z}_{a_{2},a_{1}}(w,s)$ twisted by $\wtilde{\chi}_{a_{2}}$ and $\wtilde{\chi}_{a_{1}}$ is defined as
    \[
        \wtilde{Z}_{a_{2},a_{1}}(w,s) = \sum_{\text{$m$ odd}}\frac{L^{(2)}(w,\wtilde{\chi}_{a_{2}}m_{0})\wtilde{\chi}_{a_{1}}(m)Q_{m_{0}m_{1}}^{2}(w,\wtilde{\chi}_{a_{2}})}{m^{s}},
    \]
    where $Q_{m_{0}m_{1}^{2}}(w,\wtilde{\chi}_{a_{2}})$ is the \textbf{correction polynomial} twisted by $\wtilde{\chi}_{a_{2}}$ defined by
    \[
        Q_{m_{0}m_{1}^{2}}(w,\wtilde{\chi}_{a_{2}}) = \sum_{e_{1}e_{2} \mid m_{1}}\mu(e_{1})\wtilde{\chi}_{a_{2}m_{0}}(e_{1})e_{1}^{-w}e_{2}^{1-2w} = \sum_{e_{1}e_{2}e_{3} = m_{1}}\mu(e_{1})\wtilde{\chi}_{a_{2}m_{0}}(e_{1})e_{1}^{-w}e_{2}^{1-2w}.
    \]
    and $\mu$ is the usual M\"obius function. By definition of the Hilbert characters, we have the analogous bound $Q_{m_{0}m_{1}^{2}}(w,\wtilde{\chi}_{a_{2}}) \ll_{\e} m^{\e}$ so that $\wtilde{Z}_{a_{2},a_{1}}(w,s)$ converges locally absolutely uniformly in the same region as $\wtilde{Z}(w,s)$ does. In particular, $\wtilde{Z}(w,s) = \wtilde{Z}_{1,1}(w,s)$.
\section*{The Interchange}
    As defined, $Z_{a_{1},a_{2}}(s,w)$ is a sum of $L$-functions, and hence Euler products, in $s$. We will prove an interchange formula for $Z_{a_{1},a_{2}}(s,w)$ which will show that it can be expressed as a sum of $L$-functions in $w$. That is, we want the variables $s$ and $w$ to change places. Precisely:

    \begin{theorem}[Interchange]
        Wherever $Z_{a_{1},a_{2}}(s,w)$ converges locally absolutely uniformly,
        \[
            Z_{a_{1},a_{2}}(s,w) = \sum_{\text{$d$ odd}}\frac{L^{(2)}(s,\chi_{a_{1}d_{0}})\chi_{a_{2}}(d)Q_{d_{0}d_{1}^{2}}(s,\chi_{a_{1}})}{d^{w}} = \sum_{\text{$m$ odd}}\frac{L^{(2)}(w,\wtilde{\chi}_{a_{2}m_{0}})\wtilde{\chi}_{a_{1}}(m)Q_{m_{0}m_{1}^{2}}(w,\wtilde{\chi}_{a_{2}})}{m^{s}}.
        \]
    \end{theorem}
    \begin{proof}
        Only the second equality needs to be proved. To do this, first expand the $L$-function $L^{(2)}(s,\chi_{a_{1}d_{0}})$ and polynomial $Q_{d_{0}d_{1}^{2}}(s,\chi_{a_{1}})$ to get
        \begin{align*}
            Z(s,w) &= \sum_{\text{$d$ odd}}\frac{L^{(2)}(s,\chi_{a_{1}d_{0}})\chi_{a_{2}}(d)Q_{d_{0}d_{1}^{2}}(s,\chi_{a_{1}})}{d^{w}} \\
            &= \sum_{\text{$d$ odd}}\left(\sum_{\text{$m$ odd}}\chi_{a_{1}d_{0}}(m)m^{-s}\right)\left(\sum_{e_{1}e_{2} \mid d_{1}}\mu(e_{1})\chi_{a_{1}d_{0}}(e_{1})e_{1}^{-s}e_{2}^{1-2s}\right)\chi_{a_{2}}(d)d^{-w} \\
            &= \sum_{\text{$m,d$ odd}}\sum_{e_{1}e_{2} \mid d_{1}}\mu(e_{1})\chi_{a_{2}}(d)\chi_{a_{1}d_{0}}(me_{1})e_{1}^{-s}e_{2}^{1-2s}m^{-s}d^{-w}.
        \end{align*}
        Now $\chi_{a_{1}d_{0}}(me_{1}) = 0$ unless $(d_{0},me_{1}) = 1$. We make this restriction on the sum giving
        \[
            \sum_{\text{$m,d$ odd}}\sum_{\substack{e_{1}e_{2} \mid d_{1} \\ (d_{0},me_{1}) = 1}}\mu(e_{1})\chi_{a_{2}}(d)\chi_{a_{1}d_{0}}(me_{1})e_{1}^{-s}e_{2}^{1-2s}m^{-s}d^{-w}.
        \]
        Making the change of variables $me_{1} \to m$ yields
        \[
            \sum_{\text{$d$ odd}}\sum_{\substack{\text{$m$ odd} \\ e_{1} \mid m}}\sum_{\substack{e_{1}e_{2} \mid d_{1} \\ (d_{0},m) = 1}}\mu(e_{1})\chi_{a_{2}}(d)\chi_{a_{1}d_{0}}(m)e_{2}^{1-2s}m^{-s}d^{-w}.
        \]
        For fixed $d = d_{0}d_{1}^{2}$ and $e_{2}$, the subsum over $m$ and $e_{1}$ is
        \[
            \sum_{\substack{\text{$m$ odd} \\ e_{1} \mid m}}\sum_{\substack{e_{1} \mid \frac{d_{1}}{e_{2}} \\ (d_{0},m) = 1}}\mu(e_{1})\chi_{a_{1}d_{0}}(m)m^{-s} = \sum_{\substack{\text{$m$ odd} \\ (d_{0},m) = 1}}\chi_{a_{1}d_{0}}(m)m^{-s}\left(\sum_{e_{1} \mid \left(\frac{d_{1}}{e_{2}},m\right)}\mu(e_{1})\right).
        \]
        The inner sum over $e_{1}$ of the M\"obius function vanishes unless $\left(\frac{d_{1}}{e_{2}},m\right) = 1$ in which case it is $1$. Therefore the triple sum above becomes
        \[
            \sum_{\text{$m,d$ odd}}\sum_{\substack{e_{2} \mid d_{1} \\ \left(\frac{d_{0}d_{1}}{e_{2}},m\right) = 1}}\chi_{a_{2}}(d)\chi_{a_{1}d_{0}}(m)e_{2}^{1-2s}m^{-s}d^{-w}.
        \]
        Making the change of variables $d \to de_{2}^{2}$, the condition $\left(\frac{d_{0}d_{1}}{e_{2}},m\right) = 1$ becomes $(d_{0}d_{1},m) = 1$ which is equivalent to $(d,m) = 1$. Moreover, $\chi_{a_{2}}(de_{2}^{2}) = \chi_{a_{2}}(d)$. Altogether, we obtain
        \[
            \sum_{\substack{\text{$m,d$ odd} \\ (d,m) = 1}}\sum_{e_{2}}\chi_{a_{2}}(d)\chi_{a_{1}d_{0}}(m)e_{2}^{1-2s-2w}m^{-s}d^{-w}.
        \]
        Writing $m = m_{0}m_{1}^{2}$ analogously as for $d$, quadratic reciprocity implies $\chi_{d_{0}}(m) = \wtilde{\chi}_{m}(d_{0}) = \wtilde{\chi}_{m_{0}}(d)$ where the last equality holds because $(d,m) = 1$ and both $d_{0}$ and $m_{0}$ differ from $d$ and $m$ respectively by perfect squares. As $\chi_{a_{1}}(m) = \wtilde{\chi}_{a_{1}}(m)$ and $\chi_{a_{2}}(d) = \wtilde{\chi}_{a_{2}}(d)$, the previous fact implies $\chi_{a_{2}}(d)\chi_{a_{1}d_{0}}(m) = \wtilde{\chi}_{a_{1}}(m)\wtilde{\chi}_{a_{2}m_{0}}(d)$ and so our expression becomes
        \[
            \sum_{\substack{\text{$m,d$ odd} \\ (d,m) = 1}}\sum_{e_{2}}\wtilde{\chi}_{a_{1}}(m)\wtilde{\chi}_{a_{2}m_{0}}(d)e_{2}^{1-2s-2w}m^{-s}d^{-w}.
        \]
        But now we can reverse the argument with the roles of $d$, $m$, $\chi_{a_{1}}$, and $\chi_{a_{2}}$ interchanged respectively, but with $\wtilde{\chi}_{a_{1}}$ and $\wtilde{\chi}_{a_{2}}$, to obtain
        \[
            Z(s,w) = \sum_{\text{$m$ odd}}\frac{L^{(2)}(w,\wtilde{\chi}_{a_{2}m_{0}})\wtilde{\chi}_{a_{1}}(m)Q_{m_{0}m_{1}^{2}}(w,\wtilde{\chi}_{a_{2}})}{m^{s}}.
        \]
    \end{proof}

    Note that the interchange is not completely symmetric because of the characters $\wtilde{\chi}_{a_{2}m_{0}}$, $\wtilde{\chi}_{a_{1}}$, and $\wtilde{\chi}_{a_{2}}$ in the second expression for $Z_{a_{1},a_{2}}(s,w)$. This is due to the fact that reciprocity is not perfect. In even more general settings the correction polynomials in $w$ need not be equal to those in $s$.

    \begin{remark}\label{rem:symmetry_of_Double_Dirichlet_series}
        When $a_{1} = a_{2} = 1$, the interchange implies
        \[
            Z(s,w) = \wtilde{Z}(w,s).
        \]
        More generally, the interchange implies the following relations for twisted quadratic double Dirichlet series:
        \[
            Z_{a_{1},a_{2}}(s,w) = \wtilde{Z}_{a_{2},a_{1}}(w,s),
        \]
        for $a_{1},a_{2} \in \{\pm1,\pm2\}$.
    \end{remark}
\section*{Weighting Terms}
    We will now study the coefficients of $Z_{a_{1},a_{2}}(s,w)$ expanded in $s$ and $w$. Expanding $L^{(2)}(s,\chi_{a_{1}d_{0}})Q_{d_{0}d_{1}^{2}}(s,\chi_{a_{1}})$ in the numerator of $Z_{a_{1},a_{2}}(s,w)$, we can write
    \[
        Z_{a_{1},a_{2}}(s,w) = \sum_{\text{$d$ odd}}\frac{L^{(2)}(s,\chi_{a_{1}d_{0}})\chi_{a_{2}}(d)Q_{d_{0}d_{1}^{2}}(s,\chi_{a_{1}})}{d^{w}} = \sum_{\text{$m,d$ odd}}\frac{\chi_{a_{1}d_{0}}(\what{m})\chi_{a_{2}}(d)a(m,d)}{m^{s}d^{w}},
    \]
    where $\what{m}$ is the part of $m$ relatively prime to $d_{0}$ and the \textbf{weighting coefficient} $a(m,d)$ is given by
    \[
        a(m,d) = \sum_{\substack{e_{1}e_{2}^{2}e_{3} = m \\ e_{1}e_{2} \mid d_{1} \\ (d_{0},e_{1}e_{3}) = 1}}\mu(e_{1})e_{2}.
    \]
    To see this, the coefficient of $m^{-s}d^{-w}$ in the definition of $Z_{a_{1},a_{2}}(s,w)$ is
    \begin{align*}
        \chi_{a_{2}}(d)\sum_{\substack{e_{1}e_{2}^{2}e_{3} = m \\ e_{1}e_{2} \mid d_{1}}}\mu(e_{1})\chi_{a_{1}d_{0}}(e_{1}e_{3})e_{2} &= \chi_{a_{2}}(d)\sum_{\substack{e_{1}e_{2}^{2}e_{3} = m \\ e_{1}e_{2} \mid d_{1} \\ (d_{0},e_{1}e_{3}) = 1}}\mu(e_{1})\chi_{a_{1}d_{0}}(e_{1}e_{3})e_{2} \\
        &= \chi_{a_{1}d_{0}}(\what{m})\chi_{a_{2}}(d)\sum_{\substack{e_{1}e_{2}^{2}e_{3} = m \\ e_{1}e_{2} \mid d_{1} \\ (d_{0},e_{1}e_{3}) = 1}}\mu(e_{1})e_{2} \\
        &= \chi_{a_{1}d_{0}}(\what{m})\chi_{a_{2}}(d)a(m,d),
    \end{align*}
    where the first equality holds because $\chi_{d_{0}}(e_{1}e_{3}) = 0$ unless $(d_{0},e_{1}e_{3}) = 1$ and the second equality holds because if $(d_{0},e_{1}e_{3}) = 1$, $\what{m}$ differs from $e_{1}e_{3}$ by a perfect square (the divisors of which belong to $(d_{0},e_{2})$) and so $\chi_{d_{0}}(e_{1}e_{3}) = \chi_{d_{0}}(\what{m})$. For completeness, we extend the definition of $a(m,d)$ to all $m,d \ge 1$. In particular, $a(m,d)$ makes sense when $m$ or $d$ may be even.
    
    \begin{remark}\label{rem:weighting_coefficient_remark}
        Also, $a(m,d) = 0$ unless $m = e_{1}e_{2}^{2}e_{3}$ with $(d_{0},e_{1}e_{3}) = 1$ and $e_{1}e_{2}^{2} \mid d_{1}$.
    \end{remark}

    We will define $L(s,\chi_{a_{1}d})$ to be the Dirichlet series given by
    \[
        L(s,\chi_{a_{1}d}) = L(s,\chi_{a_{1}d_{0}})Q_{d_{0}d_{1}^{2}}(s,\chi_{a_{1}}) = \sum_{m \ge 1}\frac{\chi_{a_{1}d_{0}}(\what{m})a(m,d)}{m^{s}}.
    \]
    In particular, $L(s,\chi_{d})$ now makes sense for $d$ not necessarily square-free and this definition agrees with the former when $d$ is square-free. Moreover, we have the representation
    \[
        Z_{a_{1},a_{2}}(s,w) = \sum_{\text{$d$ odd}}\frac{\chi_{a_{2}}(d)L^{(2)}(s,\chi_{a_{1}d})}{d^{w}}.
    \]
    If we perform the same procedure but with the interchange, we get
    \[
        \wtilde{Z}_{a_{2},a_{1}}(w,s) = \sum_{\text{$m$ odd}}\frac{L^{(2)}(w,\wtilde{\chi}_{a_{2}m_{0}})\wtilde{\chi}_{a_{1}}(m)Q_{m_{0}m_{1}^{2}}(w,\wtilde{\chi}_{a_{2}})}{m^{s}} = \sum_{\text{$m,d$ odd}}\frac{\wtilde{\chi}_{a_{2}m_{0}}(\what{d})\wtilde{\chi}_{a_{1}}(m)a(d,m)}{m^{s}d^{w}},
    \]
    where $\what{d}$ is the part of $d$ relatively prime to $m_{0}$. Analogously, we define $L(w,\wtilde{\chi}_{a_{2}m})$ to be the Dirichlet series given by
    \[
        L(w,\wtilde{\chi}_{a_{2}m}) = L(w,\wtilde{\chi}_{a_{2}m_{0}})Q_{m_{0}m_{1}^{2}}(w,\wtilde{\chi}_{a_{2}}) = \sum_{d \ge 1}\frac{\wtilde{\chi}_{a_{2}m_{0}}(\what{d})a(d,m)}{d^{w}},
    \]
    so that
    \[
        \wtilde{Z}_{a_{2},a_{1}}(w,s) = \sum_{\text{$m$ odd}}\frac{\wtilde{\chi}_{a_{1}}(m)L^{(2)}(w,\wtilde{\chi}_{a_{2}m})}{m^{s}}.
    \]
    We now investigate the structure of the weighting coefficients $a(m,d)$. Their structure controls the majority of the information about both the quadratic double Dirichlet series and the correction polynomials. We first show that the weighting coefficients posess a multiplicativity property:

    \begin{proposition}\label{prop:multiplicativity_of_weighting_coefficients}
        We have $a(m,1) = a(1,d) = 1$ and
        \[
            a(m,d) = \prod_{\substack{p^{\a} \mid\mid m \\ p^{\b} \mid\mid d}}a(p^{\a},p^{\b}).
        \]
    \end{proposition}
    \begin{proof}
        From the definition of the weighting coefficients, $a(m,1) = a(1,d) = 1$. We will prove multiplicativity in $m$ and then in $d$. Letting $m = m'p^{\a}$, we must show
        \[
            a(m,d) = a(m',d)a(p^{\a},d).
        \]
        To acomplish this, for $e_{1}e_{2}^{2}e_{3} = m$, let $e_{1} = c_{1}d_{1}$, $e_{2} = c_{2}d_{2}$, and $e_{3} = c_{3}d_{3}$ with $c_{1},c_{2},c_{3} \mid m'$ and $d_{1},d_{2},d_{3} \mid p^{\a}$. Because $(m',p^{\a}) = 1$, as $e_{1}e_{2}^{2}e_{3}$ runs over decompositions of $m$, $c_{1}c_{2}^{2}c_{3}$ and $d_{1}d_{2}^{2}d_{3}$ run over decompositions of $m'$ and $p^{\a}$ respectively. Moreover, as $e_{1}e_{2}$ runs over the divisors of $d_{1}$ so does $c_{1}d_{1}c_{2}d_{2}$. These facts combined with multiplicativity of the M\"obius function gives
        \begin{align*}
            a(m,d) &= \sum_{\substack{e_{1}e_{2}^{2}e_{3} = m \\ e_{1}e_{2} \mid d_{1} \\ (d_{0},e_{1}e_{3}) = 1}}\mu(e_{1})e_{2} \\
            &= \sum_{\substack{c_{1}c_{2}^{2}c_{3} = m' \\ d_{1}d_{2}^{2}d_{3} = p^{\b} \\ c_{1}d_{1}c_{2}d_{2} \mid d_{1} \\ (d_{0},c_{1}d_{1}c_{3}d_{3}) = 1}}\mu(c_{1})(d_{1})|c_{2}|d_{2} \\
            &= \left(\sum_{\substack{c_{1}c_{2}^{2}c_{3} = m' \\ c_{1}c_{2} \mid d_{1} \\ (d_{0},c_{1}c_{3}) = 1}}\mu(c_{1})|c_{2}|\right)\left(\sum_{\substack{d_{1}d_{2}^{2}d_{3} = p^{\a} \\ d_{1}d_{2} \mid d_{1} \\ (d_{0},d_{1}d_{3}) = 1}}\mu(d_{1})d_{2}\right) \\
            &= a(m',d)a(p^{\a},d),
        \end{align*}
        as desired. Now we prove multiplicativity in $d$. Since we have already proven multiplicativity in $m$, we may assume $m = p^{\a}$. Letting $d = d'p^{\b}$, we must show
        \[
            a(p^{\a},d) = a(p^{\a},p^{\b}).
        \]
        As $e_{1}e_{2}^{2}e_{3} = p^{\a}$, the $e_{i}$ are powers of $p$ for $1 \le i \le 3$. It follows that $e_{1}e_{2} \mid d_{1}$ is equivalent to $e_{1}e_{2} \mid p^{\b}$. Moreover, $(d_{0},e_{1}e_{2}) = 1$ is equivalent to $(1,e_{1}e_{2}) = 1$ or $(p,e_{1}e_{2}) = 1$ depending on of $\b$ is even or odd. These facts imply the desired identity.
    \end{proof}

    The correction polynomials $Q_{d_{0}d_{1}^{2}}(s,\chi_{a_{1}})$ are tightly connected to the weighting coefficients $a(m,d)$. In particular, $Q_{d_{0}d_{1}^{2}}(s,\chi_{a_{1}})$ is a Dirichlet polynomial whose coefficients are essentially given by the weighting coefficients. We first prove this relationship when $d$ is an odd prime power:

    \begin{lemma}\label{lem:prime_correction_odd}
        For any prime $p$ and $\a \ge 1$, we have
        \[
            Q_{p^{2\a+1}}(s) = \sum_{k \le 2\a}\frac{a(p^{k},p^{2\a+1})}{p^{ks}}.
        \]
        Moreover, the same holds for $Q_{p^{2\a+1}}(w)$.
    \end{lemma}
    \begin{proof}
        Expanding the correction polynomial in $p^{-s}$ yields
        \[
            Q_{p^{2\a+1}}(s) = \sum_{e_{1}e_{2} \mid p^{\a}}\mu(e_{1})\chi_{p}(e_{1})e_{1}^{-s}e_{2}^{1-2s} = \sum_{k \le 2\a}\frac{b(p^{k},p^{2\a+1})}{p^{ks}}.
        \]
        where
        \[
            b(p^{k},p^{2\a+1}) = \sum_{e_{1}e_{2}^{2} = p^{k}}\mu(e_{1})\chi_{p}(e_{1})e_{2}.
        \]
        The proof will be finished if we can show $b(p^{k},p^{2\a+1}) = a(p^{k},p^{2\a+1})$. To see this, first observe $\mu(e_{1})\chi_{p}(e_{1}) = 0$ unless $e_{1} = 1$ in which case it is $1$. So $b(p^{k},p^{2\a+1}) = 0$ if $k$ is odd and $p^{\frac{k}{2}}$ if $k$ is even. Compactly stated,
        \[
            b(p^{k},p^{2\a+1}) = \begin{cases} p^{\frac{k}{2}} & \text{if $k$ is even}, \\ 0 & \text{if $k$ is odd}. \end{cases}
        \]
        On the other hand, $k \le \a$ so that
        \[
            a(p^{k},p^{2\a+1}) = \sum_{\substack{e_{1}e_{2}^{2}e_{3} = p^{k} \\ e_{1}e_{2} \mid p^{\a} \\ (p,e_{1}e_{3}) = 1}}\mu(e_{1})e_{2} = \sum_{\substack{e_{1}e_{2}^{2} \mid p^{k} \\ (p,e_{1}e_{3}) = 1}}\mu(e_{1})e_{2} = \sum_{e_{2}^{2} = p^{k}}e_{2} =  \begin{cases} p^{\frac{k}{2}} & \text{if $k$ is even}, \\ 0 & \text{if $k$ is odd}. \end{cases}
        \]
        This finishes the proof. Clearly the same holds for $Q_{p^{2\a+1}}(w)$.
    \end{proof}

    There is an analogous statement when $d$ is an even prime power up to a square-free factor and relatively prime factor:
    
    \begin{lemma}\label{lem:prime_correction_even}
        For any square-free integer $d_{0} \ge 1$, $a_{1} \in \{\pm1,\pm2\}$, prime $p$ not dividing $d_{0}$, and $\b \ge 1$, we have
        \[
            Q_{d_{0}p^{2\b}}(s,\chi_{a_{1}}) = (1-\chi_{a_{1}d_{0}}(p)p^{-s})\sum_{k \le 2\b}\frac{\chi_{a_{1}d_{0}}(p^{k})a(p^{k},p^{2\b})}{p^{ks}}.
        \]
        Moreover, the same holds for $Q_{m_{0}p^{2\b}}(w,\wtilde{\chi}_{a_{2}})$.
    \end{lemma}
    \begin{proof}
        Expand the correction polynomial in $p^{-s}$ to get
        \[
            Q_{d_{0}p^{2\b}}(s,\chi_{a_{1}}) = \sum_{e_{1}e_{2} \mid p^{\a}}\mu(e_{1})\chi_{a_{1}d_{0}}(e_{1})e_{1}^{-s}e_{2}^{1-2s} = \sum_{k \le 2\b}\frac{b(p^{k},p^{2\b})}{p^{ks}}.
        \]
        where
        \[
            b(p^{k},p^{2\b}) = \sum_{e_{1}e_{2}^{2} = p^{k}}\mu(e_{1})\chi_{a_{1}d_{0}}(e_{1})e_{2}.
        \]
        It suffices to show $b(p^{k},p^{2\b}) = \chi_{a_{1}d_{0}}(p^{k})\left(a(p^{k},p^{2\b})-a(p^{k-1},p^{2\b})\right)$. On the one hand, $\mu(e_{1}) = 0$ unless $e_{1} = 1,p$ in which case $\mu(e_{1}) = \pm1$ accordingly. So
        \[
            b(p^{k},p^{2\b}) = \sum_{e_{1}e_{2}^{2} = p^{k}}\mu(e_{1})\chi_{a_{1}d_{0}}(e_{1})e_{2} = \begin{cases} \chi_{a_{1}d_{0}}(p^{k})p^{\frac{k}{2}} & \text{if $k$ is even}, \\ -\chi_{a_{1}d_{0}}(p^{k})p^{\frac{k-1}{2}} & \text{if $k$ is odd}, \end{cases}
        \]
        where we have used the identity $\chi_{a_{1}d_{0}}(e_{1}) = \chi_{a_{1}d_{0}}(p^{k})$ which holds because this quadratic Dirichlet character only depends upon the parity of $k$. On the other hand, as in the proof of \cref{lem:prime_correction_odd} 
        \[
            a(p^{k},p^{2\b}) = \begin{cases} p^{\frac{k}{2}} & \text{if $k$ is even}, \\ 0 & \text{if $k$ is odd}. \end{cases}
        \]
        But then
        \[
            \chi_{a_{1}d_{0}}(p^{k})\left(a(p^{k},p^{2\b})-a(p^{k-1},p^{2\b})\right) = \begin{cases} \chi_{a_{1}d_{0}}(p^{k})p^{\frac{k}{2}} & \text{if $k$ is even}, \\ -\chi_{a_{1}d_{0}}(p^{k})p^{\frac{k-1}{2}} & \text{if $k$ is odd}, \end{cases}
        \]
        which completes the proof. Clearly the same holds for $Q_{m_{0}p^{2\b}}(w,\wtilde{\chi}_{a_{2}})$.
    \end{proof}

    \cref{lem:prime_correction_odd,lem:prime_correction_even} together show that $Q_{d_{0}d_{1}^{2}}(s,\chi_{a_{1}})$ is a Dirichlet polynomial whose coefficients are essentially given by the weighting coefficients $a(m,d)$ when $d$ is an prime power. The proof of these lemmas also give the value of $a(p^{k},p^{l})$ and we collect this as a corollary:

    \begin{corollary}\label{cor:evaulation_of_weighting_coefficients_at_primes}
        For any prime $p$,
        \[
            a(p^{k},p^{l}) = \begin{cases} \min\left(p^{\frac{k}{2}},p^{\frac{l}{2}}\right) & \text{if $\min(k,l)$ is even}, \\ 0 & \text{otherwise}. \end{cases}
        \]
    \end{corollary}

    If we combine \cref{prop:multiplicativity_of_weighting_coefficients,cor:evaulation_of_weighting_coefficients_at_primes} we can compute $a(m,d)$ in general:

    \begin{corollary}\label{cor:evaulation_of_weighting_coefficients_in_general}
        For any integers $m,d \ge 1$,
        \[
            a(m,d) = \begin{cases} (m,d)^{\frac{1}{2}} & \text{if $(m,d)$ is a perfect square}, \\ 0 & \text{otherwise}. \end{cases}
         \]
    \end{corollary}
    
    As an immediate consquence of \cref{cor:evaulation_of_weighting_coefficients_in_general}, $a(m,d)$ is symmetric in $m$ and $d$. As the weighting coefficients are multiplicative, $Q_{d_{0}d_{1}^{2}}(s,\chi_{a_{1}})$ will posess an Euler product. To state the Euler product explicitely, we write $d = d_{0}d_{1}^{2}d_{2}^{2}$ with $d_{0}$ square-free and, $d_{2}$ relatively prime to $d_{0}d_{1}$, and such that every prime divisor of $d_{1}$ divides $d_{0}$. In other words, $d_{0}$ is the square-free part of $d$, $d_{1}$ is the square part of $d$ whose prime factors divide $d$ to odd power, and $d_{2}$ is the square part of $d$ whose prime factors divide $d$ to even power. We have the following Euler product:

    \begin{theorem}\label{thm:correction_polynomial_Euler_product}
        Let $d = d_{0}d_{1}^{2}d_{2}^{2}$ be the square decomposition of $d$ stratified by even and odd powers. Then for any $a_{1} \in \{\pm1,\pm2\}$,
        \[
            Q_{d_{0}d_{1}^{2}d_{2}^{2}}(s,\chi_{a_{1}}) = \prod_{p^{\a} \mid\mid d_{1}}Q_{p^{2\a+1}}(s) \cdot \prod_{p^{\b} \mid\mid d_{2}}Q_{d_{0}p^{2\b}}(s,\chi_{a_{1}}).
        \]
        Moreover, the same holds for $Q_{m_{0}m_{1}^{2}m_{2}^{2}}(w,\wtilde{\chi}
        _{a_{2}})$.
    \end{theorem}
    \begin{proof}
        Recall that
        \[
            L(s,\chi_{a_{1}d}) = L(s,\chi_{a_{1}d_{0}})Q_{d_{0}d_{1}^{2}d_{2}^{2}}(s,\chi_{a_{1}}) = \sum_{m \ge 1}\frac{\chi_{a_{1}d_{0}}(\what{m})a(m,d)}{m^{s}}.
        \]
        We will now derive an alterate expression for $L(s,\chi_{a_{1}d})$. By \cref{prop:multiplicativity_of_weighting_coefficients}, the coefficients of $L(s,\chi_{a_{1}d})$ are multiplicative. Therefore $L(s,\chi_{a_{1}d})$ admits the Euler product
        \[
            L(s,\chi_{a_{1}d}) = \prod_{\text{$p$ prime}}\left(\sum_{k \ge 0}\frac{\chi_{a_{1}d_{0}}(\what{p^{k}})a(p^{k},d)}{p^{ks}}\right).
        \]
        Decomposing the product according to primes dividing $d = d_{0}d_{1}^{2}d_{2}^{2}$, we get
        \begin{align*}
            & L(s,\chi_{a_{1}d}) \\
            &= \prod_{\text{$p$ prime}}\left(\sum_{k \ge 0}\frac{\chi_{a_{1}d_{0}}(\what{p^{k}})a(p^{k},d)}{p^{ks}}\right) \\
            &= \prod_{p \nmid d}\left(\sum_{k \ge 0}\frac{\chi_{a_{1}d_{0}}(\what{p^{k}})a(p^{k},1)}{p^{ks}}\right)\prod_{p^{\a} \mid\mid d_{1}}\left(\sum_{k \ge 0}\frac{\chi_{a_{1}d_{0}}(\what{p^{k}})a(p^{k},p^{2\a+1})}{p^{ks}}\right) \cdot \prod_{p^{\b} \mid\mid d_{2}}\left(\sum_{k \ge 0}\frac{\chi_{a_{1}d_{0}}(\what{p^{k}})a(p^{k},p^{\b})}{p^{ks}}\right) \\
            &= \prod_{p \nmid d}\left(\sum_{k \ge 0}\frac{\chi_{a_{1}d_{0}}(\what{p^{k}})}{p^{ks}}\right)\prod_{p^{\a} \mid\mid d_{1}}\left(\sum_{k \ge 0}\frac{\chi_{a_{1}d_{0}}(\what{p^{k}})a(p^{k},p^{2\a+1})}{p^{ks}}\right) \cdot \prod_{p^{\b} \mid\mid d_{2}}\left(\sum_{k \ge 0}\frac{\chi_{a_{1}d_{0}}(\what{p^{k}})a(p^{k},p^{\b})}{p^{ks}}\right) \\
            &= \prod_{p \nmid d}\left(\sum_{k \ge 0}\frac{\chi_{a_{1}d_{0}}(\what{p^{k}})}{p^{ks}}\right)\prod_{p^{\a} \mid\mid d_{1}}\left(\sum_{k \ge 0}\frac{a(p^{k},p^{2\a+1})}{p^{ks}}\right) \cdot \prod_{p^{\b} \mid\mid d_{2}}\left(\sum_{k \ge 0}\frac{\chi_{a_{1}d_{0}}(p^{k})a(p^{k},p^{\b})}{p^{ks}}\right).
        \end{align*}
        Including the factors corresponding to primes $p \mid d_{2}$ into the first product, we must multiply the last factor by the inverse of $\sum_{k \ge 0}\chi_{a_{1}d_{0}}(p)p^{-ks} = (1-\chi_{a_{1}d_{0}}(p)p^{-s})^{-1}$ obtaining
        \[
            \prod_{p \nmid d_{0}}\left(\sum_{k \ge 0}\frac{\chi_{a_{1}d_{0}}(\what{p^{k}})}{p^{ks}}\right)\prod_{p^{\a} \mid\mid d_{1}}\left(\sum_{k \ge 0}\frac{a(p^{k},p^{2\a+1})}{p^{ks}}\right) \cdot \prod_{p^{\b} \mid\mid d_{2}}\left((1-\chi_{a_{1}d_{0}}(p)p^{-s})\sum_{k \ge 0}\frac{\chi_{a_{1}d_{0}}(p^{k})a(p^{k},p^{\b})}{p^{ks}}\right),
        \]
        as every prime divisor of $d_{1}$ divdes $d_{0}$. The first product is $L(s,\chi_{a_{1}d_{0}})$. For the second and third products, \cref{rem:weighting_coefficient_remark} implies that the sums run up to $k \le 2\a$ and $k \le 2\b$ respectively. Therefore they are $Q_{p^{2\a+1}}(s)$ and $Q_{d_{0}p^{2\b}}(s,\chi_{a_{1}})$ respectively. It follows that
        \[
            L(s,\chi_{a_{1}d}) = L(s,\chi_{a_{1}d_{0}}) \cdot \prod_{p^{\a} \mid\mid d_{1}}Q_{p^{2\a+1}}(s) \cdot \prod_{p^{\b} \mid\mid d_{2}}Q_{d_{0}p^{2\b}}(s,\chi_{a_{1}}).
        \]
        This is our alternate expression for $L(s,\chi_{a_{1}d})$ and equating the two results in
        \[
            L(s,\chi_{a_{1}d_{0}})Q_{d_{0}d_{1}^{2}d_{2}^{2}}(s,\chi_{a_{1}}) = L(s,\chi_{a_{1}d_{0}}) \cdot \prod_{p^{\a} \mid\mid d_{1}}Q_{p^{2\a+1}}(s) \cdot \prod_{p^{\b} \mid\mid d_{2}}Q_{d_{0}p^{2\b}}(s,\chi_{a_{1}}),
        \]
        from which the proof is complete since $L(s,\chi_{a_{1}d_{0}}) \neq 0$ for $\Re(s) > 1$ (so that we may divide by $L(s,\chi_{a_{1}d_{0}})$). Clearly the same holds for $Q_{m_{0}m_{1}^{2}m_{2}^{2}}(w,\wtilde{\chi}_{a_{2}})$.
    \end{proof}

    Observe that for $d = d_{0}d_{1}^{2}d_{2}^{2}$, the prime factors that divide $d_{1}d_{2}$ are exactly those factors that divide $d$ to power larger than $1$. Thus, from \cref{thm:correction_polynomial_Euler_product} the Euler product for $Q_{d_{0}d_{1}^{2}d_{2}^{2}}(s,\chi_{a_{1}})$ is supported on exactly the primes dividing $d$ to order larger than $1$ and also depends upon the character $\chi_{a_{1}d_{0}}$.
\section*{Functional Equations}
    We can now derive functional equations for $Z_{a_{1},a_{2}}(s,w)$. These functional equations will be induced from the functional equations for $L(s,\chi_{a_{1}d})$ and $L(s,\wtilde{\chi}_{a_{2}m})$. To prove these latter functional equations, we require a functional equation for the correction polynomials:

    \begin{theorem}\label{thm:functional_equation_correction_polynomials}
        $Q_{d_{0}d_{1}^{2}}(s,\chi_{a_{1}})$ admits the functional equation.
        \[
            Q_{d_{0}d_{1}^{2}}(s,\chi_{a_{1}}) = d_{1}^{1-2s}Q_{d_{0}d_{1}^{2}}(1-s,\chi_{a_{1}}).
        \]
        Moreover, the same holds for $Q_{m_{0}m_{1}^{2}}(w,\wtilde{\chi}_{a_{2}})$.
    \end{theorem}
    \begin{proof}
        The stragety is to interchange $e_{2}$ and $e_{3}$ in the sum defining $Q_{d_{0}d_{1}^{2}}(s,\chi_{a_{1}})$:
        \begin{align*}
            d_{1}^{1-2s}Q_{d_{0}d_{1}^{2}}(1-s) &= d_{1}^{1-2s}\sum_{e_{1}e_{2}e_{3} = d_{1}}\mu(e_{1})\chi_{a_{1}d_{0}}(e_{1})e_{1}^{s-1}e_{2}^{2s-1} \\
            &= \sum_{e_{1}e_{2}e_{3} = d_{1}}\mu(e_{1})\chi_{a_{1}d_{0}}(e_{1})e_{1}^{s-1}\left(\frac{d_{1}}{e_{2}}\right)^{1-2s} \\
            &= \sum_{e_{1}e_{2}e_{3} = d_{1}}\mu(e_{1})\chi_{a_{1}d_{0}}(e_{1})e_{1}^{s-1}(e_{1}e_{3})^{1-2s} \\
            &= \sum_{e_{1}e_{2}e_{3} = d_{1}}\mu(e_{1})\chi_{a_{1}d_{0}}(e_{1})e_{1}^{-s}e_{3}^{1-2s} \\
            &= Q_{d_{0}d_{1}^{2}}(s,\chi_{a_{1}}).
        \end{align*}
        Clearly the same holds for $Q_{m_{0}m_{1}^{2}}(w,\wtilde{\chi}_{a_{2}})$.
    \end{proof}

    We will define the completed $L$-function $L^{\ast}(s,\chi_{a_{1}d})$ by
    \[
        L^{\ast}(s,\chi_{a_{1}d}) = L^{\ast}(s,\chi_{a_{1}d_{0}})Q_{d_{0}d_{1}^{2}}(s,\chi_{a_{1}}).
    \]
    In particular, $L^{\ast}(s,\chi_{d})$ makes sense even when $d$ is not square-free and agrees with the previous definition when $d$ is square-free.
    Combining \cref{thm:functional_equation_correction_polynomials}, the functional equation for $L^{\ast}(s,\chi_{a_{1}d_{0}})$, and that $d \equiv d_{0} \tmod{4}$, we obtain a functional equation for $L^{\ast}(s,\chi_{a_{1}d})$:
    \[
        L^{\ast}(s,\chi_{a_{1}d}) = \begin{cases} \e_{\chi_{a_{1}d_{0}}}q(d)^{\frac{1}{2}-s}L^{\ast}(1-s,\chi_{a_{1}d}) & \text{if $a_{1}d > 0$}, \\ -\e_{\chi_{a_{1}d_{0}}}q(d)^{\frac{1}{2}-s}L^{\ast}(1-s,\chi_{a_{1}d}) & \text{if $a_{1}d < 0$}. \end{cases}
    \]
    Analogously, define the completed $L$-function $L^{\ast}(w,\wtilde{\chi}_{a_{2}m})$ by
    \[
        L^{\ast}(w,\wtilde{\chi}_{a_{2}m}) = L^{\ast}(w,\wtilde{\chi}_{a_{2}m_{0}})Q_{m_{0}m_{1}^{2}}(w,\wtilde{\chi}_{a_{2}}).
    \]
    Then, as before, we have a functional equation for $L^{\ast}(w,\wtilde{\chi}_{a_{2}m})$:
    \[
        L^{\ast}(w,\wtilde{\chi}_{a_{2}m}) = \begin{cases} \e_{\wtilde{\chi}_{a_{2}m_{0}}}q(m)^{\frac{1}{2}-w}L^{\ast}(1-w,\wtilde{\chi}_{a_{2}m}) & \text{if $a_{2}m \equiv 1,2,5 \tmod{8}$}, \\ -\e_{\wtilde{\chi}_{a_{2}m_{0}}}q(m)^{\frac{1}{2}-w}L^{\ast}(1-w,\wtilde{\chi}_{a_{2}m}) & \text{if $a_{2}m \equiv 3,6,7 \tmod{8}$}. \end{cases}
    \]
    The functional equations for $L^{\ast}(s,\chi_{a_{1}d})$ and $L^{\ast}(w,\wtilde{\chi}_{a_{2}m})$ will induce functional equations for $Z_{a_{1},a_{2}}(s,w)$. However, there is an obstruction caused by the gamma factors. Indeed, the gamma factors for $L^{\ast}(s,\chi_{a_{1}d})$ and $L^{\ast}(w,\wtilde{\chi}_{a_{2}m})$ depend $a_{1}d$ and $a_{2}m$ modulo $8$ respectively. To induce functional equations we need these gamma factors to be constant. Orthogonality of the Hilbert characters will allow us to get past this issue. For $b \in \{1,3,5,7\}$, define $Z_{a_{1},a_{2}}^{b}(s,w)$ and $\wtilde{Z}_{a_{2},a_{1}}^{b}(w,s)$ by
    \[
        Z_{a_{1},a_{2}}^{b}(s,w) = \frac{1}{4}\sum_{a \in \{\pm1,\pm2\}}\chi_{a}(b)Z_{a_{1},aa_{2}}(s,w) \quad \text{and} \quad \wtilde{Z}_{a_{2},a_{1}}^{b}(w,s) = \frac{1}{4}\sum_{a \in \{\pm1,\pm2\}}\wtilde{\chi}_{a}(b)\wtilde{Z}_{a_{2},aa_{1}}(w,s).
    \]
    In terms of the representations
    \[
        Z_{a_{1},a_{2}}(s,w) = \sum_{\text{$d$ odd}}\frac{\chi_{a_{2}}(d)L^{(2)}(s,\chi_{a_{1}d})}{d^{w}} \quad \text{and} \quad \wtilde{Z}_{a_{2},a_{1}}(w,s) = \sum_{\text{$m$ odd}}\frac{\wtilde{\chi}_{a_{1}}(m)L^{(2)}(w,\wtilde{\chi}_{a_{2}m})}{m^{s}},
    \]
    and orthogonality of the Hilbert characters, $Z_{a_{1},a_{2}}^{b}(s,w)$ and $\wtilde{Z}_{a_{2},a_{1}}^{b}(w,s)$ are the subseries containing only those $d$ and $m$ equivalent to $b$ modulo $8$ respectively. Then $Z_{a_{1},a_{2}}^{b}(s,w)$ and $\wtilde{Z}_{a_{2},a_{1}}^{b}(w,s)$ are sums of $L$-functions with a fixed gamma factor and so we can obtain functional equations. The fact that $Z_{a_{1},a_{2}}(s,w)$ and $\wtilde{Z}_{a_{2},a_{1}}(w,s)$ are linear combinations of these series will induce function equations. Precisely, we have the following statement:

    \begin{theorem}\label{thm:double_Dirichlet_series_functional_equation}
        $Z_{a_{1},a_{2}}(s,w)$ and $\wtilde{Z}_{a_{2},a_{1}}(w,s)$ admit the functional equations
        \[
            Z_{a_{1},a_{2}}(s,w) = \chi_{a_{1}}(-1)\e_{\chi_{a_{1}d_{0}}}\pi^{s-\frac{1}{2}}\frac{\G\left(\frac{(1-s)+\d_{a_{1}}}{2}\right)}{\G\left(\frac{s+\d_{a_{1}}}{2}\right)}\frac{(1-\chi_{a_{1}d}(2)2^{-s})}{(1-\chi_{a_{1}d}(2)2^{-(1-s)})}Z_{a_{1},a_{2}}\left(1-s,s+w-\frac{1}{2}\right),
        \]
        where $\d_{\a_{1}} = \frac{\chi_{a_{1}}(1)-\chi_{a_{1}}(-1)}{2}$, and
        \begin{align*}
            &\wtilde{Z}_{a_{2},a_{1}}(w,s) \\
            &= \frac{1}{4}\left(\e_{\chi_{a_{2}m_{0}}}\pi^{w-\frac{1}{2}}\frac{\G\left(\frac{1-w}{2}\right)}{\G\left(\frac{w}{2}\right)}\frac{(1-\chi_{a_{2}m}(2)2^{-w})}{(1-\chi_{a_{2}m}(2)2^{-(1-w)})}\right)\sum_{\substack{a \in \{\pm1,\pm2\} \\ a_{2}b \equiv 1,2,5 \tmod{8}}}\wtilde{\chi}_{a}(b)\wtilde{Z}_{a_{2},aa_{1}}\left(1-w,s+w-\frac{1}{2}\right) \\
            &- \frac{1}{4}\left(\e_{\chi_{a_{2}m_{0}}}\pi^{w-\frac{1}{2}}\frac{\G\left(\frac{(1-w)+1}{2}\right)}{\G\left(\frac{w+1}{2}\right)}\frac{(1-\chi_{a_{2}m}(2)2^{-w})}{(1-\chi_{a_{2}m}(2)2^{-(1-w)})}\right)\sum_{\substack{a \in \{\pm1,\pm2\} \\ a_{2}b \equiv 3,6,7 \tmod{8}}}\wtilde{\chi}_{a}(b)\wtilde{Z}_{a_{2},aa_{1}}\left(1-w,s+w-\frac{1}{2}\right).
        \end{align*}
    \end{theorem}
    \begin{proof}
        The functional equation for $Z_{a_{1},a_{2}}(s,w)$ follows immeditely from the functional equation for the $L$-function attached to a quadratic Dirichlet character and that $a_{1}d > 0$ or $a_{1}d < 0$ according to if $a_{1} > 0$ or $a_{1} < 0$. For the functional equation for $\wtilde{Z}_{a_{2},a_{1}}(w,s)$, the functional equation for the $L$-function attached to a quadratic Dirichlet character implies
        \[
            \wtilde{Z}_{a_{2},a_{1}}^{b}(w,s) = \e_{\chi_{a_{2}m_{0}}}\pi^{w-\frac{1}{2}}\frac{\G\left(\frac{1-w}{2}\right)}{\G\left(\frac{w}{2}\right)}\frac{(1-\chi_{a_{2}m}(2)2^{-w})}{(1-\chi_{a_{2}m}(2)2^{-(1-w)})}\wtilde{Z}_{a_{2},a_{1}}^{b}\left(1-w,s+w-\frac{1}{2}\right),
        \]
        if $a_{2}b \equiv 1,2,5 \tmod{8}$, and
        \[
            \wtilde{Z}_{a_{2},a_{1}}^{b}(w,s) = -\e_{\chi_{a_{2}m_{0}}}\pi^{w-\frac{1}{2}}\frac{\G\left(\frac{(1-w)+1}{2}\right)}{\G\left(\frac{w+1}{2}\right)}\frac{(1-\chi_{a_{2}m}(2)2^{-w})}{(1-\chi_{a_{2}m}(2)2^{-(1-w)})}\wtilde{Z}_{a_{2},a_{1}}^{b}\left(1-w,s+w-\frac{1}{2}\right),
        \]
        if $a_{2}b \equiv 3,6,7 \tmod{8}$. Since $\wtilde{Z}_{a_{2},a_{1}}(w,s) = \sum_{b \in \{1,3,5,7\}}\wtilde{Z}_{a_{2},a_{1}}^{b}(w,s)$, the functional equations just stated imply
        \begin{align*}
            &\wtilde{Z}_{a_{2},a_{1}}(w,s) \\
            &= \left(\e_{\chi_{a_{2}m_{0}}}\pi^{w-\frac{1}{2}}\frac{\G\left(\frac{1-w}{2}\right)}{\G\left(\frac{w}{2}\right)}\frac{(1-\chi_{a_{2}m}(2)2^{-w})}{(1-\chi_{a_{2}m}(2)2^{-(1-w)})}\right)\sum_{a_{2}b \equiv 1,2,5 \tmod{8}}\wtilde{Z}_{a_{2},a_{1}}^{b}\left(1-w,s+w-\frac{1}{2}\right) \\
            &- \left(\e_{\chi_{a_{2}m_{0}}}\pi^{w-\frac{1}{2}}\frac{\G\left(\frac{(1-w)+1}{2}\right)}{\G\left(\frac{w+1}{2}\right)}\frac{(1-\chi_{a_{2}m}(2)2^{-w})}{(1-\chi_{a_{2}m}(2)2^{-(1-w)})}\right)\sum_{a_{2}b \equiv 3,6,7 \tmod{8}}\wtilde{Z}_{a_{2},a_{1}}^{b}\left(1-w,s+w-\frac{1}{2}\right).
        \end{align*}
        The desired functional equation for $\wtilde{Z}_{a_{2},a_{1}}(w,s)$ follows by expressing $\wtilde{Z}_{a_{2},a_{1}}^{b}(w,s)$ in terms of $\wtilde{Z}_{a_{2},aa_{1}}(w,s)$ for $a \in \{\pm1,\pm2\}$.
    \end{proof}

    These functional equations are quite unruly and it is often far more simple to compactify them in terms of vectors. Define $\mathbf{Z}(s,w)$ and $\wtilde{\mathbf{Z}}(w,s)$ by
    \[
        \mathbf{Z}(s,w) = (Z_{a_{1},a_{2}}(s,w))_{a_{1},a_{2} \in \{\pm 1, \pm 2\}} \quad \text{and} \quad \wtilde{\mathbf{Z}}(w,s) = (\wtilde{Z}_{a_{2},a_{1}}(w,s))_{a_{1},a_{2} \in \{\pm 1, \pm 2\}},
    \]
    with lexicographical ordering. Also set
    \[
        \g^{a_{1}}(s) = \chi_{a_{1}}(-1)\e_{\chi_{a_{1}}},
    \]
    and
    \[
        \g^{a_{2}b}(w) = \todo{xxx}.
    \]
  
\section*{Meromorphic Continuation}
\section*{Poles and Residues}

\begin{thebibliography}{99}
    \bibitem{R}
    Rosen, M. (2002). Number theory in function fields (Vol. 210). Springer Science \& Business Media.

    \bibitem{H}
    Hormander, L. (1973). An introduction to complex analysis in several variables. Elsevier.

    \bibitem{CG}
    Chinta, G., \& Gunnells, P. E. (2007). Weyl group multiple Dirichlet series constructed from quadratic characters. Inventiones mathematicae, 167, 327-353.

    \bibitem{S}
    Stanley, R. (2023). Enumerative Combinatorics: Volume 2. Cambridge University Press.
\end{thebibliography}

\end{document}