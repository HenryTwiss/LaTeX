\documentclass[12pt,reqno,oneside]{amsart}
\usepackage{import}
%===============================%
%  Packages and basic settings  %
%===============================%
\usepackage[headheight=15pt,rmargin=0.5in,lmargin=0.5in,tmargin=0.75in,bmargin=0.75in]{geometry}
\usepackage{imakeidx}
\usepackage{framed}
\usepackage{amssymb}
\usepackage{amsmath}
\usepackage{mathrsfs}
\usepackage{enumitem}
\usepackage{hyperref}
\usepackage{appendix}
\usepackage[capitalise,noabbrev]{cleveref}
\usepackage{tikz}
\usepackage{tikz-cd}
\usepackage{nomencl}\makenomenclature
\usetikzlibrary{braids,arrows,decorations.markings,calc}

%====================================%
%  Theorems, environments & cleveref  %
%====================================%
\newtheorem{theorem}{Theorem}[section]
\newtheorem{proposition}{Proposition}[section]
\newtheorem{corollary}{Corollary}[section]
\newtheorem{lemma}{Lemma}[section]
\newtheorem{conjecture}{Conjecture}[section]
\newtheorem{remark}{Remark}[section]

\newenvironment{stabular}[2][1]
  {\def\arraystretch{#1}\tabular{#2}}
  {\endtabular}

%==================================%
%  Custom commands & environments  %
%==================================%
\newcommand{\legendre}[2]{\left(\frac{#1}{#2}\right)}
\newcommand{\dlegendre}[2]{\displaystyle{\left(\frac{#1}{#2}\right)}}
\newcommand{\tlegendre}[2]{\textstyle{\left(\frac{#1}{#2}\right)}}
\newcommand{\psum}{\sideset{}{'}\sum}
\newcommand{\asum}{\sideset{}{^{\ast}}\sum}
\newcommand{\tmod}[1]{\ \left(\text{mod }#1\right)}
\newcommand{\xto}[1]{\xrightarrow{#1}}
\newcommand{\xfrom}[1]{\xleftarrow{#1}}
\newcommand{\normal}{\mathrel{\unlhd}}
\newcommand{\mf}{\mathfrak}
\newcommand{\mc}{\mathcal}
\newcommand{\ms}{\mathscr}

\newcommand{\Mat}{\mathrm{Mat}}
\newcommand{\GL}{\mathrm{GL}}
\newcommand{\SL}{\mathrm{SL}}
\newcommand{\PSL}{\mathrm{PSL}}
\renewcommand{\O}{\mathrm{O}}
\newcommand{\SO}{\mathrm{SO}}
\newcommand{\U}{\mathrm{U}}
\newcommand{\Sp}{\mathrm{Sp}}

\newcommand{\N}{\mathbb{N}}
\newcommand{\Z}{\mathbb{Z}}
\newcommand{\Q}{\mathbb{Q}}
\newcommand{\R}{\mathbb{R}}
\newcommand{\C}{\mathbb{C}}
\newcommand{\F}{\mathbb{F}}
\renewcommand{\H}{\mathbb{H}}
\renewcommand{\P}{\mathbb{P}}

\renewcommand{\a}{\alpha}
\renewcommand{\b}{\beta}
\newcommand{\g}{\gamma}
\renewcommand{\d}{\delta}
\newcommand{\z}{\zeta}
\renewcommand{\t}{\theta}
\renewcommand{\i}{\iota}
\renewcommand{\k}{\kappa}
\renewcommand{\l}{\lambda}
\newcommand{\s}{\sigma}
\newcommand{\w}{\omega}

\newcommand{\G}{\Gamma}
\newcommand{\D}{\Delta}
\renewcommand{\L}{\Lambda}
\newcommand{\W}{\Omega}

\newcommand{\e}{\varepsilon}
\newcommand{\vt}{\vartheta}
\newcommand{\vphi}{\varphi}
\newcommand{\emt}{\varnothing}

\newcommand{\x}{\times}
\newcommand{\ox}{\otimes}
\newcommand{\op}{\oplus}
\newcommand{\bigox}{\bigotimes}
\newcommand{\bigop}{\bigoplus}
\newcommand{\del}{\partial}
\newcommand{\<}{\langle}
\renewcommand{\>}{\rangle}
\newcommand{\lf}{\lfloor}
\newcommand{\rf}{\rfloor}
\newcommand{\wtilde}{\widetilde}
\newcommand{\what}{\widehat}
\newcommand{\conj}{\overline}
\newcommand{\cchi}{\conj{\chi}}

\DeclareMathOperator{\id}{\textrm{id}}
\DeclareMathOperator{\sgn}{\mathrm{sgn}}
\DeclareMathOperator{\im}{\mathrm{im}}
\DeclareMathOperator{\rk}{\mathrm{rk}}
\DeclareMathOperator{\tr}{\mathrm{trace}}
\DeclareMathOperator{\nm}{\mathrm{norm}}
\DeclareMathOperator{\ord}{\mathrm{ord}}
\DeclareMathOperator{\Hom}{\mathrm{Hom}}
\DeclareMathOperator{\End}{\mathrm{End}}
\DeclareMathOperator{\Aut}{\mathrm{Aut}}
\DeclareMathOperator{\Tor}{\mathrm{Tor}}
\DeclareMathOperator{\Ann}{\mathrm{Ann}}
\DeclareMathOperator{\Gal}{\mathrm{Gal}}
\DeclareMathOperator{\Trace}{\mathrm{Trace}}
\DeclareMathOperator{\Norm}{\mathrm{Norm}}
\DeclareMathOperator{\Span}{\mathrm{Span}}
\DeclareMathOperator*{\Res}{\mathrm{Res}}
\DeclareMathOperator{\Vol}{\mathrm{Vol}}
\DeclareMathOperator{\Li}{\mathrm{Li}}
\renewcommand{\Re}{\mathrm{Re}}
\renewcommand{\Im}{\mathrm{Im}}

\newcommand{\GH}{\G\backslash\H}
\newcommand{\GG}{\G_{\infty}\backslash\G}

\newenvironment{psmallmatrix}
  {\left(\begin{smallmatrix}}
  {\end{smallmatrix}\right)}

%============%
%  Comments  %
%============%
\newcommand{\todo}[1]{\textcolor{red}{\sf Todo: [#1]}}

%===================%
%  Label reminders  %
%===================%
% [label=(\roman*)]
% [label=(\alph*)]
% [label=(\arabic{enumi})]

%==================%
%  Other settings  %
%==================%
\pgfdeclarelayer{background}
\pgfsetlayers{background,main}
\tikzset{->-/.style={decoration={
  markings,
  mark=at position .5 with {\arrow{>}}},postaction={decorate}}}

%=================%
%  Title & Index  %
%=================%
\title{A quadratic double Dirichlet series II: the number field case}
\author{Henry Twiss}
\date{2024}
\makeindex

\begin{document}

\begin{abstract}
    We construct a quadratic double Dirichlet series $Z(s,w)$ built from single variable quadratic Dirichlet $L$-functions $L(s,\chi)$ over $\Q$. We prove that $Z(s,w)$ admits meromorphic continuation to the $(s,w)$-plane and satisfies a group of functional equations.
\end{abstract}

\maketitle

\section{Preliminaries}
    We present an overview of quadratic Dirichlet $L$-functions over $\Q$. We begin with the Riemann zeta-function. The zeta function $\z(s)$ is defined as the Dirichlet series or Euler product
    \[
        \z(s) = \sum_{m \ge 1}\frac{1}{m^{s}} = \prod_{\text{$p$ prime}}\left(1-\frac{1}{p^{s}}\right)^{-1},
    \]
    for $\Re(s) > 1$. The second equality is an analytic reformulation of the fundamental theorem of arithmetic. The Riemann zeta function also admits meromorphic continuation to $\C$ with a simple pole at $s = 1$ of residue $1$. The functional equation is
    \[
        \pi^{-\frac{s}{2}}\G\left(\frac{s}{2}\right)\z(s) = \pi^{-\frac{1-s}{2}}\G\left(\frac{1-s}{2}\right)\z(1-s).
    \]
    Now we recall characters on $\Z$. They are multiplicative functions $\chi:\Z \to \C$. They form a group under multiplication. The two flavors we will care about are:
    
    \begin{itemize}
        \item Dirichlet characters: multiplicative functions $\chi_{d}:\Z \to \C$ modulo $d \ge 1$ (in that they are $d$-periodic) and such that $\chi_{d}(m) = 0$ if $(m,d) > 1$.
        \item Hilbert characters: The group of characters generated by those that appear in the sign change of reciprocity statements.
    \end{itemize}
    
    The image of a Dirichlet character always lands in the roots of unity. If $\chi$ is a Dirichlet character then its conjugate $\conj{\chi}$ is also a Dirichlet character. Moreover, $\conj{\chi}$ is the multiplicative inverse to $\chi$ and the Dirichlet characters modulo $m$ form a group under multiplication. This group is always finite and its order is $\phi(d) = |(\Z/d\Z)^{\ast}|$. Dirichlet characters also satisfy orthogonality relations:

    \begin{theorem}[Orthogonality relations]
        \phantom{ }
        \begin{enumerate}[label=(\roman*)]
          \item For any two Dirichlet characters $\chi$ and $\psi$ modulo $d$,
          \[
            \frac{1}{\phi(d)}\psum_{a \tmod{d}}\chi(a)\conj{\psi}(a) = \d_{\chi,\psi}.
          \]
          \item For any $a,b \in (\Z/d\Z)^{\ast}$,
          \[
            \frac{1}{\phi(d)}\sum_{\chi \tmod{d}}\chi(a)\cchi(b) = \d_{a,b}.
          \]
        \end{enumerate}
    \end{theorem}

    The Dirichlet characters that are of interest to us are those given by the quadratic residue symbol on $\Z$. First let us recall this symbol. For any odd prime $p$ and any $m \ge 1$, we define the quadratic residue symbol $\tlegendre{m}{p}$ by
    \[
        \legendre{m}{p} \equiv m^{\frac{p-1}{2}} \tmod{p} = \begin{cases} 1 & \text{if $x^{2} \equiv m \tmod{p}$ is solvable}, \\ -1 & \text{if $x^{2} \equiv m \tmod{p}$ is not solvable}, \\ 0 & \text{if $m \equiv 0 \tmod{p}$}. \end{cases}
    \]
    This symbol only depends upon $m$ modulo $p$ and is multiplicative in $m$. We can extend the quadratic residue symbol multiplicatively in the denominator. If $d = p_{1}^{e_{1}}p_{2}^{e_{2}} \cdots p_{k}^{e_{k}}$ is the prime factorization of $d$, then we define
    \[
        \legendre{m}{d} = \prod_{1 \le i \le k}\legendre{m}{p_{i}}^{e_{i}}.
    \]
    So the quadratic residue symbol now makes sense for any odd $d \ge 1$. We can extend this symbol further and allow $d \ge 1$ to be even. To this end, we define
    \[
        \legendre{m}{2} = \begin{cases} 1 & \text{if $m \equiv 1,7 \tmod{8}$}, \\ -1 & \text{if $m \equiv 3,5 \tmod{8}$}, \\ 0 & \text{if $m \equiv 0 \tmod{2}$}, \end{cases}
    \]
    and extend $\tlegendre{m}{d}$ multiplatively in $d$ when $d$ is even. Now the quadratic residue symbol makes sense for any $m,d \ge 1$. Moreover, it is multiplicative in both $m$ and $d$ but no longer depends upon only $m$ modulo $d$ (it also depends upon $m$ modulo $8$). In particular,
    \[
        \legendre{-1}{d} = \begin{cases} 1 & \text{$d \equiv 1 \tmod{4}$}, \\ -1 & \text{$d \equiv 3 \tmod{4}$}, \\ 0 & \text{$d \equiv 0 \tmod{2}$}, \end{cases} \quad \text{and} \quad \legendre{2}{d} = \begin{cases} 1 & \text{$d \equiv 1,7 \tmod{8}$}, \\ -1 & \text{$d \equiv 3,5 \tmod{8}$}, \\ 0 & \text{$d \equiv 0 \tmod{2}$}, \end{cases}
    \]
    and if $d \not\equiv 0 \tmod{2}$, we can compactly write
    \[
        \legendre{-1}{d} = (-1)^{\frac{d-1}{2}} = \begin{cases} 1 & \text{$d \equiv 1 \tmod{4}$}, \\ -1 & \text{$d \equiv 3 \tmod{4}$}, \end{cases} \quad \text{and} \quad \legendre{2}{d} = (-1)^{\frac{d^{2}-1}{8}} = \begin{cases} 1 & \text{$d \equiv 1,7 \tmod{8}$}, \\ -1 & \text{$d \equiv 3,5 \tmod{8}$}. \end{cases}
    \]
     The quadratic residue symbol also admits the following reciprocity law:

    \begin{theorem}[Quadratic reciprocity]
        If $d,m \ge 1$, then
        \[
            \legendre{d}{m} = (-1)^{\frac{d^{(2)}-1}{2}\frac{m^{(2)}-1}{2}}\legendre{m}{d},
        \]
        where $d^{(2)}$ and $m^{(2)}$ are the parts of $d$ and $m$ relatively prime to $2$ respectively.
    \end{theorem}

    We can now define the quadratic Dirichlet characters. For any odd square-free $d \in \Z$, define the quadratic Dirichlet character $\chi_{d}$ by the following quadratic residue symbol:
    \[
        \chi_{d}(m) = \begin{cases} \legendre{d}{m} & \text{if $d \equiv 1 \tmod{4}$}, \\ \legendre{4d}{m} & \text{if $d \equiv 2,3 \tmod{4}$}. \end{cases}
    \]
    This quadratic Dirichlet character is attached to the quadratic extension $\Q(\sqrt{d})$. We extend $\chi_{d}$ multiplicatively in the denominator so that $\chi_{d}$ makes sense for any odd $d$. In particular, $\chi_{d}(m) = \pm1$ provided $d$ and $m$ are relatively prime and $\chi_{d}(m) = 0$ if $(m,d) > 1$. Quadratic reciprocity implies that $\chi_{d}$ is a Dirichlet character modulo $d$ if $d \equiv 1 \tmod{4}$ and is a Dirichlet character modulo $4d$ if $d \equiv 2,3 \tmod{4}$. Indeed, if $d \equiv 1 \tmod{4}$ then $d^{(2)} = d$ and the sign is always $1$. If $d \equiv 3 \tmod{4}$, then $d^{(2)} = d$ and the sign is $\tlegendre{-1}{m}$ which is a character modulo $4$. If $d \equiv 2 \tmod{4}$, then $d^{(2)} \equiv 1,3 \tmod{4}$ and we are reduced to one of the previous two cases. We will also require an associated character. For each $\chi_{d}$, we define $\wtilde{\chi}_{d}$ by
    \[
        \wtilde{\chi}_{d}(m) = (-1)^{\frac{d^{(2)}-1}{2}\frac{m^{(2)}-1}{2}}\chi_{d}(m).
    \]
    Equivalently, $\wtilde{\chi}_{d}(m)$ can be expressed as
    \[
        \wtilde{\chi}_{d}(m) = \begin{cases} \chi_{d}(m) & \text{if $d \equiv 1,2 \tmod{4}$}, \\ \chi_{-1}(m)\chi_{d}(m) & \text{if $d \equiv 3 \tmod{4}$}, \end{cases}
    \]
    and this implies $\wtilde{\chi}_{d}(m)$ is a quadratic Dirichlet character of the same modulus as $\chi_{d}$. We now discuss the Hilbert characters. We will only need four of them: the quadratic Dirichlet characters modulo $8$. They are given as follows:
    \begin{gather*}
        \chi_{1}(m) = \begin{cases} 1 &\text{if $m \not\equiv 0 \tmod{2}$}, \\ 0 &\text{if $m \equiv 0 \tmod{2}$}, \end{cases} \quad \chi_{-1}(m) = \begin{cases} 1 &\text{if $m \equiv 1 \tmod{4}$}, \\ -1 &\text{if $m \equiv 3 \tmod{4}$}, \\ 0 &\text{if $m \equiv 0 \tmod{2}$}, \end{cases} \\ \chi_{2}(m) = \begin{cases} 1 &\text{if $m \equiv 1,7 \tmod{8}$}, \\ -1 &\text{if $m \equiv 3,5 \tmod{8}$}, \\ 0 &\text{if $m \equiv 0 \tmod{2}$}, \end{cases} \quad \chi_{-2}(m) = \begin{cases} 1 &\text{if $m \equiv 1,3 \tmod{8}$}, \\ -1 &\text{if $m \equiv 5,7 \tmod{8}$}, \\ 0 &\text{if $m \equiv 0 \tmod{2}$}. \end{cases}
    \end{gather*}
    In general, we will denote a Hilbert character by $\chi_{a}$ with $a \in \{\pm 1,\pm 2\}$. Note that
    \[
        \chi_{-1}(m) = \legendre{-1}{m} \quad \text{and} \quad \chi_{2}(m) = \legendre{m}{2}.
    \]
    Moreover, we have the relations
    \[
        \chi_{-2}(m) = \chi_{-1}(m)\chi_{2}(m), \quad \chi_{1}(m) = \chi_{-1}(m)\chi_{-1}(m), \quad \text{and} \quad \chi_{-1}(m) = \chi_{2}(m)\chi_{-2}(m).
    \]
    Suppose $d$ is square-free. If $d \equiv 1,2,5 \tmod{8}$, then $d^{(2)} \equiv 1 \tmod{4}$ so that the sign in the statement of quadratic recipricty is $1$. If $d \equiv 3,6,7 \tmod{8}$, then $d^{(2)} \equiv 3 \tmod{4}$ and the sign is $(-1)^{\frac{m^{(2)}-1}{2}}$. This fact together with the relations for the quadratic characters modulo $8$ imply
    \[
        \chi_{d}(m) = \begin{cases} \chi_{m}(d) & \text{if $d \equiv 1 \tmod{4}$}, \\ \chi_{-1}(m)\chi_{m}(d) & \text{if $d \equiv 3 \tmod{4}$}, \\ \chi_{2}(m)\chi_{m}\left(\frac{d}{2}\right) & \text{if $d \equiv 2 \tmod{8}$}, \\ \chi_{-2}(m)\chi_{m}\left(\frac{d}{2}\right) & \text{if $d \equiv 6 \tmod{8}$}. \end{cases}
    \]
    With the Dirichlet and Hilbert characters introduced, we are ready to discuss the $L$-functions associated to quadratic Dirichlet characters. We define the $L$-function $L(s,\chi_{d})$ attached to $\chi_{d}$ by a Dirichlet series or Euler product:
    \[
        L(s,\chi_{d}) = \sum_{m \ge 1}\frac{\chi_{d}(m)}{|m|^{s}} = \prod_{p \text{ prime}}\left(1-\frac{\chi_{d}(P)}{|P|^{s}}\right)^{-1}.
    \]
    By definition of the quadratic Dirichlet character, $L(s,\chi_{d}) \ll \z(s)$ for $\Re(s) > 1$ so that $L(s,\chi_{d})$ is locally absolutely uniformly convergent in this region. $L(s,\chi_{d})$ also admits meromorphic continuation to $\C$ with a simple pole at $s = 1$ if $d$ is a perfect square. For square-free $d$, the completed $L$-function is defined as
    \[
        L^{\ast}(s,\chi_{d}) = \begin{cases} \pi^{-\frac{s}{2}}\G\left(\frac{s}{2}\right)L(s,\chi_{d}) & \text{if $\chi_{d}$ is even}, \\ \pi^{-\frac{s}{2}}\G\left(\frac{s+1}{2}\right)L(s,\chi_{d}) & \text{if $\chi_{d}$ is odd}, \end{cases}
    \]
    and satisfies the functional equation
    \[
        L^{\ast}(s,\chi_{d}) = \begin{cases} \e_{\chi}q^{\frac{1}{2}-s}L^{\ast}(1-s,\chi_{d}) & \text{if $\chi_{d}$ is even}, \\ -\e_{\chi}q^{\frac{1}{2}-s}L^{\ast}(1-s,\chi_{d}) & \text{if $\chi_{d}$ is odd}. \end{cases}
    \]
    Note that the gamma factors depend upon the partiy of $\chi_{d}$. This the root cause of an important technical issue later when deriving functional equations for the quadratic double Dirichlet series.
\section*{The Quadratic Double Dirichlet Series}
    We will now define the quadratic double Dirichlet series $Z(s,w)$. For any integer $d \ge 1$, write $d = d_{0}d_{1}^{2}$ where $d_{0}$ is square-free. Equivalently, $d_{0}$ is the square-free part of $d$ and $\frac{d}{d_{0}}$ is a perfect square. The \textbf{quadratic double Dirichlet series} $Z(s,w)$ is defined as
    \[
        Z(s,w) = \sum_{\text{$d$ odd}}\frac{L^{(2)}(s,\chi_{d_{0}})Q_{d_{0}d_{1}^{2}}(s)}{d^{w}},
    \]
    where the superscript $(2)$ indicates that the local factor at $2$ has been removed, $Q_{d_{0}d_{1}^{2}}(s)$ is the \textbf{correction polynomial} defined by
    \[
        Q_{d_{0}d_{1}^{2}}(s) = \sum_{e_{1}e_{2} \mid d_{1}}\mu(e_{1})\chi_{d_{0}}(e_{1})e_{1}^{-s}e_{2}^{1-s} = \sum_{e_{1}e_{2}e_{3} = d_{1}}\mu(e_{1})\chi_{d_{0}}(e_{1})e_{1}^{-s}e_{2}^{1-s},
    \]
    and $\mu$ is the usual M\"obius function. For $\Re(s) > 1$, there is the trivial estimate
    \[
        Q_{d_{0}d_{1}^{2}}(s) \ll \sum_{e_{1}e_{2} \mid d_{1}}1 \ll \s_{0}(d_{1})^{2} \ll_{\e} |d_{1}^{2}|^{\e} \ll_{\e} |d|^{\e},
    \]
     for any $\e > 0$. As $L(s,\chi_{d_{0}}) \ll 1$ for $\Re(s) > 1$, $Z(s,w)$ is locally absolutely uniformly convergent in the region $\L = \{(s,w) \in \C^{2}:\Re(s) > 1, \Re(w) > 1\}$. It will also be necessary to consider quadratic double Dirichlet series twisted by a pair of Hilbert characters $\chi_{a_{1}}$ and $\chi_{a_{2}}$. The \textbf{quadratic double Dirichlet series} $Z_{a_{1},a_{2}}(s,w)$ twisted by $\chi_{a_{1}}$ and $\chi_{a_{2}}$ is defined as
    \[
        Z_{a_{1},a_{2}}(s,w) = \sum_{\text{$d$ odd}}\frac{L^{(2)}(s,\chi_{a_{1}d_{0}})\chi_{a_{2}}(d)Q_{d_{0}d_{1}^{2}}(s,\chi_{a_{1}})}{|d|^{w}},
    \]
    where $Q_{d_{0}d_{1}^{2}}(s,\chi_{a_{1}})$ is the \textbf{correction polynomial} twisted by $\chi_{a_{1}}$ defined by
    \[
        Q_{d_{0}d_{1}^{2}}(s,\chi_{a_{1}}) = \sum_{e_{1}e_{2} \mid d_{1}}\mu(e_{1})\chi_{a_{1}d_{0}}(e_{1})|e_{1}|^{-s}|e_{2}|^{1-2s} = \sum_{e_{1}e_{2}e_{3} = d_{1}}\mu(e_{1})\chi_{a_{1}d_{0}}(e_{1})|e_{1}|^{-s}|e_{2}|^{1-2s},
    \]
    and $\mu$ is the usual M\"obius function. By definition of the Hilbert characters, we have the analogous bound $Q_{d_{0}d_{1}^{2}}(s,\chi_{a_{1}}) \ll |d|_{\e}$ so that $Z_{a_{1},a_{2}}(s,w)$ converges locally absolutely uniformly in the same region as $Z(s,w)$ does. In particular, $Z(s,w) = Z_{1,1}(s,w)$.
\section*{The Interchange}
    As defined, $Z_{a_{1},a_{2}}(s,w)$ is a sum of $L$-functions, and hence Euler products, in $s$. We will prove an interchange formula for $Z_{a_{1},a_{2}}(s,w)$ which will show that it can be expressed as a sum of $L$-functions in $w$. That is, we want the variables $s$ and $w$ to change places. Precisely:

    \begin{theorem}[Interchange]
        Wherever $Z_{a_{1},a_{2}}(s,w)$ converges locally absolutely uniformly,
        \[
            Z_{a_{1},a_{2}}(s,w) = \sum_{\text{$d$ odd}}\frac{L^{(2)}(s,\chi_{a_{1}d_{0}})\chi_{a_{2}}(d)Q_{d_{0}d_{1}^{2}}(s,\chi_{a_{1}})}{|d|^{w}} = \sum_{\text{$m$ odd}}\frac{L^{(2)}(w,\wtilde{\chi}_{a_{2}m_{0}})\chi_{a_{1}}(m)Q_{m_{0}m_{1}^{2}}(w,\chi_{a_{2}})}{|m|^{s}}.
        \]
    \end{theorem}
    \begin{proof}
        Only the second equality needs to be proved. To do this, first expand the $L$-function $L^{(2)}(s,\chi_{a_{1}d_{0}})$ and polynomial $Q_{d_{0}d_{1}^{2}}(s,\chi_{a_{1}})$ to get
        \begin{align*}
            Z(s,w) &= \sum_{\text{$d$ odd}}\frac{L(s,\chi_{a_{1}d_{0}})\chi_{a_{2}}(d)Q_{d_{0}d_{1}^{2}}(s,\chi_{a_{1}})}{|d|^{w}} \\
            &= \sum_{\text{$d$ odd}}\left(\sum_{\text{$m$ odd}}\chi_{a_{1}d_{0}}(m)|m|^{-s}\right)\left(\sum_{e_{1}e_{2} \mid d_{1}}\mu(e_{1})\chi_{a_{1}d_{0}}(e_{1})|e_{1}|^{-s}|e_{2}|^{1-2s}\right)\chi_{a_{2}}(d)|d|^{-w} \\
            &= \sum_{\text{$m,d$ odd}}\sum_{e_{1}e_{2} \mid d_{1}}\mu(e_{1})\chi_{a_{2}}(d)\chi_{a_{1}d_{0}}(me_{1})|e_{1}|^{-s}|e_{2}|^{1-2s}|m|^{-s}|d|^{-w}.
        \end{align*}
        Now $\chi_{a_{1}d_{0}}(me_{1}) = 0$ unless $(d_{0},me_{1}) = 1$. We make this restriction on the sum giving
        \[
            \sum_{\text{$m,d$ odd}}\sum_{\substack{e_{1}e_{2} \mid d_{1} \\ (d_{0},me_{1}) = 1}}\mu(e_{1})\chi_{a_{2}}(d)\chi_{a_{1}d_{0}}(me_{1})|e_{1}|^{-s}|e_{2}|^{1-2s}|m|^{-s}|d|^{-w}.
        \]
        Making the change of variables $me_{1} \to m$ yields
        \[
            \sum_{\text{$d$ odd}}\sum_{\substack{\text{$m$ odd} \\ e_{1} \mid m}}\sum_{\substack{e_{1}e_{2} \mid d_{1} \\ (d_{0},m) = 1}}\mu(e_{1})\chi_{a_{2}}(d)\chi_{a_{1}d_{0}}(m)|e_{2}|^{1-2s}|m|^{-s}|d|^{-w}.
        \]
        For fixed $d = d_{0}d_{1}^{2}$ and $e_{2}$, the subsum over $m$ and $e_{1}$ is
        \[
            \sum_{\substack{\text{$m$ odd} \\ e_{1} \mid m}}\sum_{\substack{e_{1} \mid \frac{d_{1}}{e_{2}} \\ (d_{0},m) = 1}}\mu(e_{1})\chi_{a_{1}d_{0}}(m)|m|^{-s} = \sum_{\substack{\text{$m$ odd} \\ (d_{0},m) = 1}}\chi_{a_{1}d_{0}}(m)|m|^{-s}\left(\sum_{e_{1} \mid \left(\frac{d_{1}}{e_{2}},m\right)}\mu(e_{1})\right).
        \]
        The inner sum over $e_{1}$ of the M\"obius function vanishes unless $\left(\frac{d_{1}}{e_{2}},m\right) = 1$ in which case it is $1$. Therefore the triple sum above becomes
        \[
            \sum_{\text{$m,d$ odd}}\sum_{\substack{e_{2} \mid d_{1} \\ \left(\frac{d_{0}d_{1}}{e_{2}},m\right) = 1}}\chi_{a_{2}}(d)\chi_{a_{1}d_{0}}(m)|e_{2}|^{1-2s}|m|^{-s}|d|^{-w}.
        \]
        Making the change of variables $d \to de_{2}^{2}$, the condition $\left(\frac{d_{0}d_{1}}{e_{2}},m\right) = 1$ becomes $(d_{0}d_{1},m) = 1$ which is equivalent to $(d,m) = 1$. Moreover, $\chi_{a_{2}}(de_{2}^{2}) = \chi_{a_{2}}(d)$. Altogether, we obtain
        \[
            \sum_{\substack{\text{$m,d$ odd} \\ (d,m) = 1}}\sum_{e_{2}}\chi_{a_{2}}(d)\chi_{a_{1}d_{0}}(m)|e_{2}|^{1-2s-2w}|m|^{-s}|d|^{-w}.
        \]
        Writing $m = m_{0}m_{1}^{2}$ analogously as for $d$, quadratic reciprocity implies $\chi_{d_{0}}(m) = \wtilde{\chi}_{m}(d_{0}) = \wtilde{\chi}_{m_{0}}(d)$ where the last equality holds because $(d,m) = 1$ and both $d_{0}$ and $m_{0}$ differ from $d$ and $m$ respectively by perfect squares. This implies $\chi_{a_{2}}(d)\chi_{a_{1}d_{0}}(m) = \chi_{a_{1}}(m)\wtilde{\chi}_{a_{2}m_{0}}(d)$ and so our expression becomes
        \[
            \sum_{\substack{\text{$m,d$ monic} \\ (d,m) = 1}}\sum_{e_{2}}\chi_{a_{1}}(m)\wtilde{\chi}_{a_{2}m_{0}}(d)|e_{2}|^{1-2s-2w}|m|^{-s}|d|^{-w}.
        \]
        But now we can reverse the argument with the roles of $d$, $m$, $\chi_{a_{1}}$, and $\wtilde{\chi}_{a_{2}}$ interchanged respectively to obtain
        \[
            Z(s,w) = \sum_{\text{$m$ monic}}\frac{L(w,\wtilde{\chi}_{a_{2}m_{0}})\chi_{a_{1}}(m)Q_{m_{0}m_{1}^{2}}(w,\chi_{a_{2}})}{|m|^{s}}.
        \]
    \end{proof}

    Note that the interchange is not completely symmetric because of the character $\wtilde{\chi}_{a_{2}m_{0}}$ in the second expression for $Z_{a_{1},a_{2}}(s,w)$. This is due to the fact that recipricty is not perfect. In even more general settings the correction polynomials in $w$ need not be equal to those in $s$.
\section*{Weighting Terms}
\section*{Functional Equations}
\section*{Meromorphic Continuation}
\section*{Poles and Residues}

\begin{thebibliography}{99}
    \bibitem{R}
    Rosen, M. (2002). Number theory in function fields (Vol. 210). Springer Science \& Business Media.

    \bibitem{H}
    Hormander, L. (1973). An introduction to complex analysis in several variables. Elsevier.

    \bibitem{CG}
    Chinta, G., \& Gunnells, P. E. (2007). Weyl group multiple Dirichlet series constructed from quadratic characters. Inventiones mathematicae, 167, 327-353.

    \bibitem{S}
    Stanley, R. (2023). Enumerative Combinatorics: Volume 2. Cambridge University Press.
\end{thebibliography}

\end{document}