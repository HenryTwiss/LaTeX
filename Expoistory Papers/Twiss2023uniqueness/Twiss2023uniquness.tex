\documentclass[12pt,reqno,oneside]{amsart}
\usepackage{import}
%===============================%
%  Packages and basic settings  %
%===============================%
\usepackage[headheight=15pt,rmargin=0.5in,lmargin=0.5in,tmargin=0.75in,bmargin=0.75in]{geometry}
\usepackage{imakeidx}
\usepackage{framed}
\usepackage{amssymb}
\usepackage{amsmath}
\usepackage{mathrsfs}
\usepackage{enumitem}
\usepackage{hyperref}
\usepackage{appendix}
\usepackage[capitalise,noabbrev]{cleveref}
\usepackage{tikz}
\usepackage{tikz-cd}
\usepackage{nomencl}\makenomenclature
\usetikzlibrary{braids,arrows,decorations.markings,calc}

%====================================%
%  Theorems, environments & cleveref  %
%====================================%
\newtheorem{theorem}{Theorem}[section]
\newtheorem{proposition}{Proposition}[section]
\newtheorem{corollary}{Corollary}[section]
\newtheorem{lemma}{Lemma}[section]
\newtheorem{conjecture}{Conjecture}[section]
\newtheorem{remark}{Remark}[section]

\newenvironment{stabular}[2][1]
  {\def\arraystretch{#1}\tabular{#2}}
  {\endtabular}

%==================================%
%  Custom commands & environments  %
%==================================%
\newcommand{\legendre}[2]{\left(\frac{#1}{#2}\right)}
\newcommand{\dlegendre}[2]{\displaystyle{\left(\frac{#1}{#2}\right)}}
\newcommand{\tlegendre}[2]{\textstyle{\left(\frac{#1}{#2}\right)}}
\newcommand{\psum}{\sideset{}{'}\sum}
\newcommand{\asum}{\sideset{}{^{\ast}}\sum}
\newcommand{\tmod}[1]{\ \left(\text{mod }#1\right)}
\newcommand{\xto}[1]{\xrightarrow{#1}}
\newcommand{\xfrom}[1]{\xleftarrow{#1}}
\newcommand{\normal}{\mathrel{\unlhd}}
\newcommand{\mf}{\mathfrak}
\newcommand{\mc}{\mathcal}
\newcommand{\ms}{\mathscr}

\newcommand{\Mat}{\mathrm{Mat}}
\newcommand{\GL}{\mathrm{GL}}
\newcommand{\SL}{\mathrm{SL}}
\newcommand{\PSL}{\mathrm{PSL}}
\renewcommand{\O}{\mathrm{O}}
\newcommand{\SO}{\mathrm{SO}}
\newcommand{\U}{\mathrm{U}}
\newcommand{\Sp}{\mathrm{Sp}}

\newcommand{\N}{\mathbb{N}}
\newcommand{\Z}{\mathbb{Z}}
\newcommand{\Q}{\mathbb{Q}}
\newcommand{\R}{\mathbb{R}}
\newcommand{\C}{\mathbb{C}}
\newcommand{\F}{\mathbb{F}}
\renewcommand{\H}{\mathbb{H}}
\renewcommand{\P}{\mathbb{P}}

\renewcommand{\a}{\alpha}
\renewcommand{\b}{\beta}
\newcommand{\g}{\gamma}
\renewcommand{\d}{\delta}
\newcommand{\z}{\zeta}
\renewcommand{\t}{\theta}
\renewcommand{\i}{\iota}
\renewcommand{\k}{\kappa}
\renewcommand{\l}{\lambda}
\newcommand{\s}{\sigma}
\newcommand{\w}{\omega}

\newcommand{\G}{\Gamma}
\newcommand{\D}{\Delta}
\renewcommand{\L}{\Lambda}
\newcommand{\W}{\Omega}

\newcommand{\e}{\varepsilon}
\newcommand{\vt}{\vartheta}
\newcommand{\vphi}{\varphi}
\newcommand{\emt}{\varnothing}

\newcommand{\x}{\times}
\newcommand{\ox}{\otimes}
\newcommand{\op}{\oplus}
\newcommand{\bigox}{\bigotimes}
\newcommand{\bigop}{\bigoplus}
\newcommand{\del}{\partial}
\newcommand{\<}{\langle}
\renewcommand{\>}{\rangle}
\newcommand{\lf}{\lfloor}
\newcommand{\rf}{\rfloor}
\newcommand{\wtilde}{\widetilde}
\newcommand{\what}{\widehat}
\newcommand{\conj}{\overline}
\newcommand{\cchi}{\conj{\chi}}

\DeclareMathOperator{\id}{\textrm{id}}
\DeclareMathOperator{\sgn}{\mathrm{sgn}}
\DeclareMathOperator{\im}{\mathrm{im}}
\DeclareMathOperator{\rk}{\mathrm{rk}}
\DeclareMathOperator{\tr}{\mathrm{trace}}
\DeclareMathOperator{\nm}{\mathrm{norm}}
\DeclareMathOperator{\ord}{\mathrm{ord}}
\DeclareMathOperator{\Hom}{\mathrm{Hom}}
\DeclareMathOperator{\End}{\mathrm{End}}
\DeclareMathOperator{\Aut}{\mathrm{Aut}}
\DeclareMathOperator{\Tor}{\mathrm{Tor}}
\DeclareMathOperator{\Ann}{\mathrm{Ann}}
\DeclareMathOperator{\Gal}{\mathrm{Gal}}
\DeclareMathOperator{\Trace}{\mathrm{Trace}}
\DeclareMathOperator{\Norm}{\mathrm{Norm}}
\DeclareMathOperator{\Span}{\mathrm{Span}}
\DeclareMathOperator*{\Res}{\mathrm{Res}}
\DeclareMathOperator{\Vol}{\mathrm{Vol}}
\DeclareMathOperator{\Li}{\mathrm{Li}}
\renewcommand{\Re}{\mathrm{Re}}
\renewcommand{\Im}{\mathrm{Im}}

\newcommand{\GH}{\G\backslash\H}
\newcommand{\GG}{\G_{\infty}\backslash\G}

\newenvironment{psmallmatrix}
  {\left(\begin{smallmatrix}}
  {\end{smallmatrix}\right)}

%============%
%  Comments  %
%============%
\newcommand{\todo}[1]{\textcolor{red}{\sf Todo: [#1]}}

%===================%
%  Label reminders  %
%===================%
% [label=(\roman*)]
% [label=(\alph*)]
% [label=(\arabic{enumi})]

%==================%
%  Other settings  %
%==================%
\pgfdeclarelayer{background}
\pgfsetlayers{background,main}
\tikzset{->-/.style={decoration={
  markings,
  mark=at position .5 with {\arrow{>}}},postaction={decorate}}}

%=================%
%  Title & Index  %
%=================%
\title{Uniquness Theorems for Rankin-Selberg \texorpdfstring{$L$}{L}-functions}
\author{Henry Twiss}
\date{\today}
\makeindex

\begin{document}

\begin{abstract}
    We sketch a result of Luo (see \cite{L}) which which asserts that two primitive Hecke eigenforms $f$ and $g$ are equal provided that the aossciated Rankin-Selberg $L$-functions $L(s,f \ox h)$ and $L(s,g \ox h)$ at the special $s = \frac{1}{2}$ are constant multiples of each other. This is reminiscent of a similar result due to Ramakrishnan and Luo (see \cite{LR-1} and \cite{LR-2}) when the twists are instead given by Dirichlet characters $\chi_{d}$ where $d$ ranges over all fundamental discriminants. Luo result follows from an application of the Petersson trace formula and we give some heurists for why Luo's result should hold in the half-integral weight setting.
\end{abstract}

\maketitle

\section{Preliminaries}
    Let $\mc{S}_{2k}(N)$ denote the space of weight $k$ cuspforms on $\G_{0}(N)\backslash\H$ with trivial nebentypus. If $f \in \mc{S}_{2k}(N,\chi)$ is a primitive Hecke eigenform, let
    \[
        f(z) = \sum_{n \ge 1}\l_{f}(n)e^{2\pi inz},
    \]
    denote the Fourier expansion of $f$. Then $T_{n}f = \l_{f}(n)f$ for every Hecke operator $T_{n}$ on $\GH$ and $\l_{f}(1) = 1$. Moreover, we can construct the \textbf{modular $L$-function} $L(s,f)$ associated to $f$ and define it as the Dirichlet series built from its Fourier coefficients:
    \[
        L(s,f) = \sum_{n \ge 1}\frac{\l_{f}(n)}{n^{s+\frac{k-1}{2}}}.
    \]
    By the Ramanujan conjecture, the Foruier coefficients admit the polynomial bound $\l_{f}(n) \ll \tau(n)n^{\frac{k-1}{2}} \ll n^{\frac{k-1}{2}+\e}$, for any $\e > 0$, so that the Dirichlet series is absolutely uniformly convergent on compacta for $\Re(s) > 1$. Moreover, it admits a degree $2$ Euler product:
    \[
        L(s,f) = \prod_{p}(1-\a_{1}(p)p^{-s})^{-1}(1-\a_{2}(p)p^{-s})^{-1},
    \]
    where $\a_{1}(p)$ and $\a_{2}(p)$ are the \textbf{local roots} at $p$ (sometimes refered to as the Satake parameters). That is, they are the roots of the polynomial $1-\l_{f}(p)p^{-\frac{k-1}{2}}p^{-s}+\chi_{N,0}p^{-2s}$ where $\chi_{N,0}$ is the principal Dirichlet character modulo $N$. $L(s,f)$ also admits analytic continuation to $\C$ and satisfies a functional equation. If $w_{N}(f) \in \{\pm 1\}$ is the eigenvalue for $f$ of the level $N$ Fricke involution $\w_{N}$, then letting
    \[
        \L(s,f) = N^{-\frac{s}{2}}(2\pi)^{-s}\G\left(s+\frac{k-1}{2}\right)L(s,f),
    \]
    we have the the functional equation
    \[
        \L(s,f) = \w_{N}(f)i^{k}\L(1-s,f).
    \]
    We can also consider twists of these modular $L$-functions by Dirichlet characters $\chi_{d}$:
    \[
        L(s,f \ox \chi_{d}) = \sum_{n \ge 1}\frac{\l_{f}(n)\chi_{d}(n)}{n^{s+\frac{k-1}{2}}}.
    \]
    These Dirichlets are absolutely uniformly convergent on compacta for $\Re(s) > 1$, admit analytic continuation to $\C$, and satisfty a functional equation.

    If $f \in \mc{S}_{2k}(N)$ and $g \in \mc{S}_{2l}(M)$, say with $N$ and $M$ square-free and relatively prime, we can form the \textbf{Rankin-Selberg convolution} $L(s,f \ox \g)$ defined by
    \[
        L(s,f \ox g) = \z^{(NM)}(2s)\sum_{n \ge 1}\frac{\l_{f}(n)a_{g}(n)}{n^{s+\frac{k+l-2}{2}}},
    \]
    where $\z^{(NM)}(2s)$ is the usual Riemann zeta function with the local factors at the primes $p \mid NM$ removed. This series is absolutely uniformly convergent on compacta for $\Re(s) > 1$ and can be shown to admit a degree $4$ Euler product, where for $p \nmid NM$, the local factor at $p$ is an amalgamation of the local factors:
    \[
        L_{p}(s,f \ox g) = \prod_{1 \le i,j \le 2}(1-\a_{i}(p)\b_{j}(p)p^{-s})^{-s},
    \]
    where $\a_{i}(p)$ and $\b_{j}(p)$ are the local roots at $p$ for $L(s,f)$ and $L(s,g)$ respectively. If $p \mid NM$, then this Euler factor needs to be modified slightly. In any case, $L(s,f \ox g)$ also admits meromorphic continuation to $\C$ and satisfies a functional equation. Setting $r = \max\{k,l\}$, $t = \min\{k,l\}$, and defining
    \[
        \L(s,f \ox g) = (NM)^{-s}(2\pi)^{-2s}\G(s+r-t)\G(s+k+l-1)L(s,f \ox g),
    \]
    we have the functional equation
    \[
        \L(s,f \ox g) = \L(1-s,f \ox g).
    \]
    Moreover, $L(s, f \ox g)$ has a pole at $s = 1$ whose residue is the Petersson inner product $\<f,g\>$ up to a constant.
\section{Uniquness Theorems}
    The first type of uniqueness theorem one ecounters for modular forms is the \textbf{multiplicity one}\index{multiplicity one} theorem:

    \begin{theorem}[Multiplicity one]
        If $f$ and $g$ are weight $k$ and level $N$ primitive Hecke eigenforms with the same eigenvalues for the Hecke operators, then $f = g$.
    \end{theorem}

    The multiplicity one theorem is a uniqueness theorem in the sense that there is exactly one Hecke eigenform per set of eigenvalues for the Hecke operators. That is, it gives an arithmetic way of checking whether or not two Hecke eigenforms normalize to the same primitive Hecke eigenform. The second uniqueness theorem usually encountered is \textbf{strong multiplicity one}\index{strong multiplicity one} theorem:

    \begin{theorem}[Strong multiplicity one]
        If $f$ and $g$ are weight $k$ and $k'$ and level $N$ and $N'$ primitive Hecke eigenforms respectively with the same eigenvalues for the Hecke operators $T_{p}$ with $(p,NN') = 1$, then $f = g$.
    \end{theorem}

    The difference between multiplicity one and strong multiplicity one is that for the latter we only need to guarentee that $f$ and $g$ have the same eigenvalues for those Hecke operators relatively prime to the level. This reduces the verification of the eigenvalues to a set of all but finitely many primes.
    
    Both of these uniqueness theorems are arithmetic in nature as the assumptions depend upon the eigenvalues of the modular forms. One might hope for ``analytic flavor'' uniqueness theorems. That is, our assumptions should depend upon properties of the associated $L$-functions $L(s,f)$ and $L(s,g)$. There is a quite elementary example of this. As we have noted, up to a constant

    \[
        \Res_{s = 1}L(s,f \ox g) = \<f,g\>.
    \]
    But the primitive Hecke eigenforms are orthogonal with respect to the Petersson inner product, so this residue is zero unless $f = g$. Therefore if the Rankin-Selberg convolution has a genuine pole at $s = 1$ we know that $f = g$. However, we might be interested in examining other $L$-functions rather than $L(s,f \ox g)$ directly. In this case, we need other uniqueness theorems. Such results due exist and are given by special values of twists of $L(s,f)$ and $L(s,g)$ at the value $s = \frac{1}{2}$. A classicial result is due to Luo and Ramakrishnan (see \cite{LR-1} and \cite{LR-2}):

    \begin{theorem}\label{thm:uniqueness_theorem_1}
        Let $f,g \in \mc{S}_{2k}(N)$ be primitive Hecke eigenforms. Suppose there is a nonzero constant $c$ such that
        \[
            L(s,f \ox \chi_{d}) = cL(s,g \ox \chi_{d}) \quad \text{or} \quad L'(s,f \ox \chi_{d}) = cL'(s,g \ox \chi_{d}),
        \]
        for all fundamental discriminants $d$ (that is $\chi_{d}$ varries over all primitive quadratic characters). Then $f = g$.
    \end{theorem}

    \cref{thm:uniqueness_theorem_1} is a uniqueness theorem where we assume that modular $L$-functions twisted by Dirichlet characters (or their derivatives) are well-behaved at the special value $s = \frac{1}{2}$. We will sketch the proof of the following result due to Luo which is of the same flavor except we assume that special values of Rankin-Selberg convolutions are well-behaved:

    \begin{theorem}\label{thm:uniqueness_theorem_2}
        Let $f \in \mc{S}_{2k}(N)$ and $g \in \mc{S}_{2k'}(N')$ be primitive Hecke eigenforms. Now suppose there is a nonzero constant $c$ and positive even integer $l$ such that
        \[
            L\left(\frac{1}{2},f \ox h\right) = cL\left(\frac{1}{2},g \ox h\right),
        \]
        for all primitive Hecke eigenforms $h \in \mc{S}_{2l}(p)$ and infinitely many primes $p$. Then $f = g$.
    \end{theorem}

    Note that in \cref{thm:uniqueness_theorem_2} we reqiure that the Rankin-Selberg convolutions are well-behaved at the special value $s = \frac{1}{2}$ for many primitive Hecke eigenforms $h$ (across infinitely many different prime levels), but we are able to relax the conditions that $f$ and $g$ are of the same weight or level.
\section{Special Values of Rankin-Selberg Convolutions}
    To prove \cref{thm:uniqueness_theorem_2}, we essentially will use complex integration to prove an asymptotic formula which will reduce the problem to that of a multiplicity one theorem. Start by defining $a_{f}(m) = \frac{\l_{f}(m)}{m^{\frac{2k-1}{2}}}$ and $a_{g}(m) = \frac{\l_{f}(m)}{m^{\frac{2l-1}{2}}}$. This normalization is so that the $L(s,f)$ and $L(s,g)$ have functional equations of shape $s \to 1-s$. Let $X$ be a positive real to be chosen later, and consider the following Perron integral:
    \[
        I = \frac{1}{2\pi i}\int_{(2)}\L\left(s+\frac{1}{2},f \ox h\right)X^{s}\,\frac{ds}{s}.
    \]
    We will compute this integral in two ways. Expanding the $L$-function $L(s,f \ox h)$ and interchanging the sum and integral we arrive at
    \begin{equation}\label{equ:I_evaulation_1}
        I = (Np)^{-\frac{1}{2}}(2\pi)^{-1}\sum_{n \ge 1}\frac{b_{f \ox h}}{n^{\frac{1}{2}}}h_{0}\left(\frac{4\pi^{2}n}{XNp}\right),
    \end{equation}
    where $b_{f \ox h}$ is given by
    \[
        b_{f \ox h} = \sum_{\substack{d^{2}m = n \\ (d,Mp) = 1}}a_{f}(m)a_{g}(m),
    \]
    and $h_{0}(\xi)$ is the integral transform
    \[
        h_{0}(\xi) = \frac{1}{2\pi i}\int_{(2)}\G\left(s+r-t+\frac{1}{2}\right)\G\left(s+k+l-\frac{1}{2}\right)\xi^{-s}\,\frac{ds}{s}.
    \]
    Now on the other hand, by shifting the line of integration to $\left(-\frac{1}{3}\right)$ we pass through a simple pole at $s = 0$ coming from the $\frac{1}{s}$ term. Therefore
    \[
        I = \L\left(\frac{1}{2},f \ox h\right)+\frac{1}{2\pi i}\int_{\left(-\frac{1}{3}\right)}\L\left(s+\frac{1}{2},f \ox h\right)X^{s}\,\frac{ds}{s}.
    \]
    Applying the functional equation, we see that
    \begin{equation}\label{equ:I_evaulation_2}
        \begin{aligned}
            I &= \L\left(\frac{1}{2},f \ox h\right)+\frac{1}{2\pi i}\int_{\left(-\frac{1}{3}\right)}\L\left(\frac{1}{2}-s,f \ox h\right)X^{s}\,\frac{ds}{s} \\
            &= \L\left(\frac{1}{2},f \ox h\right)-\frac{1}{2\pi i}\int_{\left(\frac{1}{3}\right)}\L\left(s+\frac{1}{2},f \ox h\right)X^{-s}\,\frac{ds}{s} \\
            &= \L\left(\frac{1}{2},f \ox h\right)-\frac{1}{2\pi i}\int_{(2)}\L\left(s+\frac{1}{2},f \ox h\right)X^{-s}\,\frac{ds}{s}.
        \end{aligned}
    \end{equation}
    We have computed $I$ in two ways and upon combining \cref{equ:I_evaulation_1,equ:I_evaulation_2}, we get
    \[
        \L\left(\frac{1}{2},f \ox h\right)-(Np)^{-\frac{1}{2}}(2\pi)^{-1}\sum_{n \ge 1}\frac{b_{f \ox h}}{n^{\frac{1}{2}}}h_{0}\left(\frac{4\pi^{2}nX}{Np}\right) = (Np)^{-\frac{1}{2}}(2\pi)^{-1}\sum_{n \ge 1}\frac{b_{f \ox h}}{n^{\frac{1}{2}}}h_{0}\left(\frac{4\pi^{2}n}{XNp}\right).
    \]
    Taking $X = 1$ yields
    \begin{equation}\label{equ:Rankin-Selberg_convolution_as_sum}
        \G\left(r-t+\frac{1}{2}\right)\G\left(k+l-\frac{1}{2}\right)L\left(\frac{1}{2},f \ox h\right) = 2\sum_{n \ge 1}\frac{b_{f \ox h}}{n^{\frac{1}{2}}}h_{0}\left(\frac{4\pi^{2}n}{Np}\right).
    \end{equation}
    We will state a bound for $h_{0}(\xi)$ that will be useful later. From \cite{GR}, $h_{0}(\xi)$ is
    \begin{equation}\label{equ:h_bound}
        h_{0}(\xi) = 2\int_{\xi}^{\infty}u^{r-1}K_{2t-1}(2\sqrt{u})\,du \ll \begin{cases} \int_{\xi}^{\infty}e^{-\sqrt{u}}\,du & \text{if $\xi \ge 1$}, \\ 1 & \text{if $0 < \xi < 1$}. \end{cases}
    \end{equation}
    Now define the inner products
    \[
        \<w(z),w(z)\>_{1} = \int_{\G_{0}(1)\backslash\H}|w(z)|^{2}y^{2l}\,\frac{dx\,dy}{y^{2}} \quad \text{and} \quad \<w(z),w(z)\>_{p} = \int_{\G_{0}(p)\backslash\H}|w(z)|^{2}y^{2l}\,\frac{dx\,dy}{y^{2}},
    \]
    for $w \in \mc{S}_{2l}(1)$ and $w \in \mc{S}_{2l}(p)$ respectively. The next step is to essentially sum \cref{equ:Rankin-Selberg_convolution_as_sum} over a basis of cuspforms and then apply the Petersson trace formula. For $h \in \mc{S}_{2l}^{\text{new}}(p)$ denote the Fourier coefficients by $a_{h}(\ell)$. Now for $\ell = 1,p$ where $p$ is a prime, and letting $\w_{h} = \frac{(2l-2)!}{(4\pi)^{2l-1}\<h(z),h(z)\>_{p}}$, summing over all primitive Hecke newforms $h \in \mc{S}_{2l}^{\text{new}}(p)$ gives
    \begin{equation}\label{equ:large_computation_1}
        \begin{aligned}
            &\frac{(Np)^{\frac{1}{2}}2\pi}{2}\sum_{h \in \mc{S}_{2l}^{\text{new}}(p)}\L\left(\frac{1}{2},f \ox h\right)a_{h}(\ell)\w_{h} \\
            &= \frac{1}{2}\sum_{h \in \mc{S}_{2l}^{\text{new}}(p)}\G\left(r-t+\frac{1}{2}\right)\G\left(k+l-\frac{1}{2}\right)L\left(\frac{1}{2},f \ox h\right)a_{h}(\ell)\w_{h} \\
            &= \sum_{h \in \mc{S}_{2l}^{\text{new}}(p)}\sum_{n \ge 1}\frac{b_{f \ox h}}{n^{\frac{1}{2}}}h_{0}\left(\frac{4\pi^{2}n}{Np}\right) \\
            &= \sum_{(d,Np) = 1}\frac{1}{d}\sum_{m \ge 1}\frac{a_{f}(m)}{m^{\frac{1}{2}}}h_{0}\left(\frac{4\pi^{2}d^{2}m}{Np}\right)\sum_{h \in \mc{S}_{2l}^{\text{new}}(p)}a_{h}(\ell)a_{h}(m)\w_{h}.
        \end{aligned}
    \end{equation}
    where in the second equality we have applied \cref{equ:Rankin-Selberg_convolution_as_sum}. We can almost apply the Petersson trace formula to the inner sum but first we need the inner sum to be a sum over an orthogonal basis of Hecke eigenforms in $S_{2\ell}(p)$. Actually, since $p$ is prime an orthogonal basis of eigenforms for $\mc{S}_{2l}(p)$ is
    \[
        \{h(z):h \in S_{2\ell}^{\text{new}}(p)\} \cup \{f_{h}(z) = \b_{h}h(z)+h(pz):h \in S_{2\ell}^{\text{new}}(1)\},
    \]
    where $\b_{h} = -\frac{\<h(pz),h(z)\>_{p}}{\<h(z),h(z)\>_{p}}$. Therefore
    \begin{equation}\label{equ:large_computation_2}
        \begin{aligned}
            &\sum_{h \in \mc{S}_{2l}^{\text{new}}(p)}a_{h}(\ell)a_{h}(m)\w_{h} \\
            &= \sum_{h \in \mc{S}_{2l}^{\text{new}}(p)}a_{h}(\ell)a_{h}(m)\frac{(2l-2)!}{(4\pi)^{2l-1}\<h(z),h(z)\>_{p}} \\
            &= \sum_{h \in \mc{S}_{2l}(p)}a_{h}(\ell)a_{h}(m)\frac{(2l-2)!}{(4\pi)^{2l-1}\<h(z),h(z)\>_{p}}-\sum_{\substack{f_{h}(z) \\ h \in \mc{S}_{2l}^{\text{new}}(1)}}a_{f_{h}}(\ell)a_{f_{h}}(m)\frac{(2l-2)!}{(4\pi)^{2l-1}\<f_{h}(z),f_{h}(z)\>_{p}} \\
            &= \sum_{h \in \mc{S}_{2l}(p)}a_{h}(\ell)a_{h}(m)\frac{(2l-2)!}{(4\pi)^{2l-1}\<h(z),h(z)\>_{p}}-\frac{1}{p+1}\sum_{h \in \mc{S}_{2l}(1)}a_{h}(\ell)a_{h}(m)\frac{(2l-2)!}{(4\pi)^{2l-1}\<h(z),h(z)\>_{1}} \\
            &-\sum_{h \in \mc{S}_{2l}(1)}\frac{(2l-2)!\left(\b_{h}a_{h}(\ell)+p^{\frac{1}{2}-l}a_{h}\left(\frac{\ell}{p}\right)\right)\left(\b_{h}a_{h}(m)+p^{\frac{1}{2}-l}a_{h}\left(\frac{m}{p}\right)\right)}{(4\pi)^{2l-1}\<\b_{h}h(z)+h(pz),\b_{h}h(z)+h(pz)\>_{p}},
        \end{aligned}
    \end{equation}
    where the $\frac{1}{p+1}$ term appears because $[\G_{0}(1):\G_{0}(p)] = p+1$. The firt two sums in \cref{equ:large_computation_2} are over a basis of eigenforms, we can apply the Petersson trace formula in the following form:
    \[
        \sum_{h \in \mc{S}_{2l}(M)}a_{h}(\ell)a_{h}(m)\frac{(2l-2)!}{(4\pi)^{2l-1}\<h(z),h(z)\>_{M}} = \d_{\ell,m}+(-1)^{l}2\pi\sum_{\substack{c \ge 1 \\ c \equiv 0 \tmod{M}}}\frac{S(m,n;c)}{c}J_{2l-1}\left(\frac{4\pi\sqrt{\ell m}}{c}\right),
    \]
    for $M = 1,p$. To estimate the geometric side of the Petersson trace formula we use the Weil bound and a standard estimate for the $J$-Bessel function:
    \[
        S(m,n;c) \ll_{\e} \sqrt{c(m,n,c)}c^{\e} \quad \text{and} \quad J_{2l-1}(z) \ll \min(1,z^{2l-1}),
    \]
    provided $z > 0$. Applying these bounds to the geometric side in the Petersson trace formula, the sum over $c$ can be seen to converge by summing depending on if $m \ll c$ or $m \gg c$. In the case $M = p$, we get obtain the estimate
    \begin{equation}\label{equ:Petersson_trace_estimate_1}
        \sum_{h \in \mc{S}_{2l}(p)}a_{h}(\ell)a_{h}(m)\frac{(2l-2)!}{(4\pi)^{2l-1}\<h(z),h(z)\>_{p}} = \d_{\ell,m}+O(m^{\frac{1}{2}}p^{-\frac{3}{2}}),
    \end{equation}
    and in the case $M = 1$, we have
    \begin{equation}\label{equ:Petersson_trace_estimate_2}
        \frac{1}{p+1}\sum_{h \in \mc{S}_{2l}(1)}a_{h}(\ell)a_{h}(m)\frac{(2l-2)!}{(4\pi)^{2l-1}\<h(z),h(z)\>_{1}} = O_{\e}(m^{\e}p^{-1}),
    \end{equation}
    where the diagional term has been discarded. To estimate the third sum in \cref{equ:large_computation_2}, we first esimate the inner product. To do this, it is easy to check that for primitive newforms $w \in \mc{S}_{2l}(1)$ that
    \[
        \<w(z),w(z)\>_{p} = (p+1)\<w(z),w(z)\>_{1},
    \]
    because $[\G_{0}(1):\G_{0}(p)] = p+1$. Then one can verify the two identities
    \[
        \<w(pz),w(pz)\>_{p} = p^{-2l}(p+1)\<w(z),w(z)\>_{1} \quad \text{and} \quad \<w(pz),w(z)\>_{p} = a_{w}(p)p^{\frac{1}{2}-l}\<w(z),w(z)\>_{1},
    \]
    where $a_{w}(m) = \frac{\l_{f}(m)}{m^{\frac{2l-1}{2}}}$. We can use these to compute $\b_{h}$:
    \[
        \b_{h} = -\frac{\<h(pz),h(z)\>_{p}}{\<h(z),h(z)\>_{p}} = -a_{w}(p)\frac{p^{\frac{1}{2}-l}}{p+1}.
    \]
    So
    \begin{align*}
        &\<\b_{h}h(z)+h(pz),\b_{h}h(z)+h(pz)\>_{p} \\
        &= \b_{h}^{2}\<h(z),h(z)\>_{p}+\<h(pz),h(pz)\>_{p}+\b_{h}(\<h(pz),h(z)\>_{p}+\<h(z),h(pz)\>_{p}) \\
        &= \b_{h}^{2}\<h(z),h(z)\>_{p}+\<h(pz),h(pz)\>_{p}+2\b_{h}\Re(\<h(pz),h(z)\>_{p}) \\
        &= \left(a_{w}(p)^{2}\frac{p^{1-2l}}{(p+1)^{2}}+p^{-2l}(p+1)-2a_{w}(p)^{2}\frac{p^{1-2l}}{p+1}\right)\<h(z),h(z)\>_{1} \\
        &\gg p^{1-2l}\<h(z),h(z)\>_{1},
    \end{align*}
    where the last line follows since $a_{w}(p) \ll 1$ by the Ramanujan conjecture. But then
    \begin{equation}\label{equ:inner_product_sum_estiate}
        \begin{aligned}
            &\frac{1}{p+1}\sum_{h \in \mc{S}_{2l}(1)}\frac{(2l-2)!\left(\b_{h}a_{h}(\ell)+p^{\frac{1}{2}-l}a_{h}\left(\frac{\ell}{p}\right)\right)\left(\b_{h}a_{h}(m)+p^{\frac{1}{2}-l}a_{h}\left(\frac{m}{p}\right)\right)}{(4\pi)^{2l-1}\<\b_{h}h(z)+h(pz),\b_{h}h(z)+h(pz)\>_{p}} \\
            &\gg \frac{1}{p+1}\sum_{h \in \mc{S}_{2l}(1)}\frac{(2l-2)!}{(4\pi)^{2l-1}}\frac{p^{2l-1}}{\<h(z),h(z)\>_{1}}\left(\b_{h}a_{h}(\ell)+p^{\frac{1}{2}-l}a_{h}\left(\frac{\ell}{p}\right)\right)\left(\b_{h}a_{h}(m)+p^{\frac{1}{2}-l}a_{h}\left(\frac{m}{p}\right)\right) \\
            &= \frac{1}{p+1}\sum_{h \in \mc{S}_{2l}(1)}\frac{(2l-2)!}{(4\pi)^{2l-1}}\frac{p^{2l-1}}{\<h(z),h(z)\>_{1}}\left(-a_{w}(p)\frac{p^{\frac{1}{2}-l}}{p+1}a_{h}(\ell)+p^{\frac{1}{2}-l}a_{h}\left(\frac{\ell}{p}\right)\right) \\
            &\cdot \left(-a_{w}(p)\frac{p^{\frac{1}{2}-l}}{p+1}a_{h}(m)+p^{\frac{1}{2}-l}a_{h}\left(\frac{m}{p}\right)\right) \\
            &\gg \frac{1}{p+1}\sum_{h \in \mc{S}_{2l}(1)}\frac{(2l-2)!}{(4\pi)^{2l-1}}\frac{a_{h}\left(\frac{\ell}{p}\right)a_{h}\left(\frac{m}{p}\right)}{\<h(z),h(z)\>_{1}} \\
            &\gg_{\e} m^{\e}p^{-1}.
        \end{aligned}
    \end{equation}
    Applying \cref{equ:Petersson_trace_estimate_1,equ:Petersson_trace_estimate_2,equ:inner_product_sum_estiate} to \cref{equ:large_computation_2} yields
    \begin{equation}\label{equ:newform_sum_estimate}
        \sum_{h \in \mc{S}_{2l}^{\text{new}}(p)}a_{h}(\ell)a_{h}(m)\w_{h} = \d_{\ell,m}+O(m^{\frac{1}{2}}p^{-\frac{3}{2}})+O_{\e}(m^{\e}p^{-1}).
    \end{equation}
    and then combining \cref{equ:large_computation_1,equ:newform_sum_estimate} we find
    \begin{equation}\label{equ:large_computation_3}
        \begin{aligned}
            & \frac{(Np)^{\frac{1}{2}}2\pi}{2}\sum_{h \in \mc{S}_{2l}^{\text{new}}(p)}\L\left(\frac{1}{2},f \ox h\right)a_{h}(\ell)\w_{h} \\
            &= \sum_{(d,Np) = 1}\frac{1}{d}\sum_{m \ge 1}\frac{a_{f}(m)}{m^{\frac{1}{2}}}h_{0}\left(\frac{4\pi^{2}d^{2}m}{Np}\right)\sum_{h \in \mc{S}_{2l}^{\text{new}}(p)}a_{h}(\ell)a_{h}(m)\w_{h} \\
            &= \sum_{(d,Np) = 1}\frac{1}{d}\sum_{m \ge 1}\frac{a_{f}(m)}{m^{\frac{1}{2}}}h_{0}\left(\frac{4\pi^{2}d^{2}m}{Np}\right)(\d_{\ell,m}+O(m^{\frac{1}{2}}p^{-\frac{3}{2}})+O_{\e}(m^{\e}p^{-1})) \\
            &= \frac{a_{f}(\ell)}{\ell^{\frac{1}{2}}}\sum_{(d,Np) = 1}\frac{1}{d}h_{0}\left(\frac{4\pi^{2}d^{2}\ell}{Np}\right)+O_{\e}(p^{\e-\frac{1}{2}}),
        \end{aligned}
    \end{equation}
    where the last line follows because by \cref{equ:h_bound} the $h_{0}$ term exhibits exponential decay for $d^{2}m \gg p$ and so for $d^{2}m \ll p$ we see that the contribution from the non-diagional terms is
    \[
        \ll \sum_{d^{2}m \ll p}\frac{a_{f}(m)}{dm^{\frac{1}{2}}}(O(m^{\frac{1}{2}}p^{-\frac{3}{2}})+O_{\e}(m^{\e}p^{-1})) \ll p(O(p^{-\frac{3}{2}})+O_{\e}(p^{\e-\frac{3}{2}})) = O_{\e}(p^{\e-\frac{1}{2}}).
    \]
    We would like to estimate the remaining sum over $d$. From the defintion of $h_{0}$, we have
    \begin{align*}
        \sum_{(d,Np) = 1}\frac{1}{d}h_{0}\left(\frac{4\pi^{2}d^{2}\ell}{Np}\right) &= \sum_{(d,Np) = 1}\frac{1}{2\pi id}\int_{(2)}\G\left(s+r-t+\frac{1}{2}\right)\G\left(s+k+l-\frac{1}{2}\right)\left(\frac{4\pi^{2}d^{2}\ell}{Np}\right)^{-s}\,\frac{ds}{s} \\
        &= \frac{1}{2\pi i}\int_{(2)}\G\left(s+r-t+\frac{1}{2}\right)\G\left(s+k+l-\frac{1}{2}\right)\left(\frac{4\pi^{2}\ell}{Np}\right)^{-s}\sum_{(d,Np) = 1}\frac{1}{d^{2s+1}}\,\frac{ds}{s} \\
        &= \frac{1}{2\pi i}\int_{(2)}\G\left(s+r-t+\frac{1}{2}\right)\G\left(s+k+l-\frac{1}{2}\right)\left(\frac{4\pi^{2}\ell}{Np}\right)^{-s}\z^{(Np)}(2s+1)\,\frac{ds}{s}.
    \end{align*}
    Shifting the line of integration to the left of zero, say $-\frac{1}{4}$, we pick up a double pole at $s = 0$ coming from the zeta factor. So the latter integral is equal to
    \begin{align*}
        &\Res_{s = 0}\left(\G\left(s+r-t+\frac{1}{2}\right)\G\left(s+k+l-\frac{1}{2}\right)\left(\frac{4\pi^{2}\ell}{Np}\right)^{-s}\z^{(Np)}(2s+1)\right) \\
        &+ \frac{1}{2\pi i}\int_{(-\frac{1}{4})}\G\left(s+r-t+\frac{1}{2}\right)\G\left(s+k+l-\frac{1}{2}\right)\left(\frac{4\pi^{2}\ell}{Np}\right)^{-s}\z^{(Np)}(2s+1)\,\frac{ds}{s}.
    \end{align*}
    Computing the residue term and estimating the latter integral using the fact that
    \[
        \left(\frac{4\pi^{2}\ell}{Np}\right)^{-s}\z^{(Np)}(2s+1) = \frac{A(s)}{p^{s}}+\frac{B(s)}{b^{s+1}},
    \]
    for some functions $A(s)$ and $B(s)$ independent of $p$, we obtain
    \[
        c_{f}\log(p)+d_{f}+O(p^{-1}\log(p))+O(p^{-\frac{1}{4}}),
    \]
    where $c_{f}$ is given by
    \[
        c_{f} = \frac{1}{2}\G\left(r-t+\frac{1}{2}\right)\G\left(k+l-\frac{1}{2}\right)\prod_{q \mid N}(1-q^{-1}),
    \]
    with $q$ running over prime divisors of $N$, and $d_{f}$ is an absolute constant. So altogether
    \begin{equation}\label{equ:d_sum_estimate}
        \sum_{(d,Np) = 1}\frac{1}{d}h_{0}\left(\frac{4\pi^{2}d^{2}\ell}{Np}\right) = c_{f}\log(p)+d_{f}+O(p^{-1}\log(p))+O(p^{-\frac{1}{4}}).
    \end{equation}
    Combining \cref{equ:large_computation_3,equ:d_sum_estimate} gives
    \begin{align*}
        \frac{(Np)^{\frac{1}{2}}2\pi}{2}\sum_{h \in \mc{S}_{2l}^{\text{new}}(p)}\L\left(\frac{1}{2},f \ox h\right)a_{h}(\ell)\w_{h} &= \frac{a_{f}(\ell)}{\ell^{\frac{1}{2}}}(c_{f}\log(p)+d_{f}+O(p^{-1}\log(p))+O(p^{-\frac{1}{4}}))+O_{\e}(p^{\e-\frac{1}{2}}) \\
        &= \frac{a_{f}(\ell)}{\ell^{\frac{1}{2}}}c_{f}\log(p)+O(1).
    \end{align*}
    Canceling the gamma factors and rewriting in terms of $L\left(\frac{1}{2},f \ox h\right)$ we get
    \[
        \sum_{h \in \mc{S}_{2l}^{\text{new}}(p)}L\left(\frac{1}{2},f \ox h\right)a_{h}(\ell)\w_{h} = \frac{a_{f}(\ell)}{\ell^{\frac{1}{2}}}\prod_{q \mid N}(1-q^{-1})\log(p)+O(1).
    \]
    This is the asymptotic formula we are after. We now appeal to the hypothesis. If $L\left(\frac{1}{2},f \ox h\right) = cL\left(\frac{1}{2},g \ox h\right)$, with $g \in \mc{S}_{2k'}(N')$, for all primitive Hecke eigenforms $h \in \mc{S}_{2l}(p)$ for infinitely many $p$, then taking $\ell = 1$ we have
    \[
        \prod_{q \mid N}(1-q^{-1}) = c\prod_{q \mid N'}(1-q^{-1}) \neq 0,
    \]
    by the asymptotic formula. Taking $\ell = p$ and using the product identity, the asymptotic formula implies $a_{f}(p) = a_{g}(p)$ for infinitely many primes $p$. The proof is complete by appealing to the strong multiplicity one theorem.
\section{Comments on Special Values of Rankin-Selberg Convolutions in Half-integral Weight}
    Recall that a modular form $f:\H \to \C$ of half-integral weight $\frac{2k+1}{2}$ on $\G_{0}(4N)$ is a holomorphic function on $\H$ exhibiting exponential decay near the cusps of $\G_{0}(4N)\backslash\H$ and satisfies
    \[
        f(\g z) = \legendre{c}{d}\e_{d}^{-(2k+1)}(cz+d)^{\frac{2k+1}{2}}f(z),
    \]
    for $\g = \begin{psmallmatrix} a & b \\ c & d \end{psmallmatrix} \in \G_{0}(4N)$. Here $\legendre{c}{d}$ is a modified Jacobi symbol, $\e_{d} = 1,i$ depending on if $d \equiv 1,3 \tmod{4}$, and we take the principal branch of the square root. The factor of modularity
    \[
        \legendre{c}{d}\e_{d}^{-1}(cz+d)^{\frac{1}{2}},
    \]
    is called the theta multiplier. We would like a variant of \cref{thm:uniqueness_theorem_2} for half-integral weight modular forms $f$ and $g$. A analog was achieved by Pandey and Ramakrishnan for $\G_{0}(4)$ in \cite{PR}:

    \begin{theorem}\label{thm:uniqueness_theorem_3}
        Let $f \in \mc{S}_{\frac{2k+1}{2}}^{+}(4)$ and $g \in \mc{S}_{\frac{2k'+1}{2}}^{+}(4)$ be primitive Hecke eigenforms. Now suppose that
        \[
            L\left(\frac{1}{2},f \ox h\right) = L\left(\frac{1}{2},g \ox h\right),
        \]
        for all primitive Hecke eigenforms $h \in \mc{S}_{\frac{2l+1}{2}}^{+}(4)$. Then $f = g$.
    \end{theorem}

    The proof of \cref{thm:uniqueness_theorem_3} does not follow that of Luo's and instead appeals to approximate functional equations for Rankin-Selberg convolutions. However, the underlying idea is similar. Indeed, Pandey and Ramakrishnan devlope an asymptotic for the spectral average of special values of the Rankin-Selberg convolutions and then use that asymptotic to prove the theorem using a multiplicity one result for the Kohen plus space $\mc{S}_{\frac{2l+1}{2}}^{+}(4)$.
    
    It should be possible to follow the proof presented for \cref{thm:uniqueness_theorem_2} in the half-integral weight setting. Indeed, most of the necessary tools are available:
    \begin{itemize}
        \item A half-integral weight variant of the Petersson trace formula is presented in \cite{CT}.
        \item Necessary estimates for the Sali\'e sums arising in the half-integral weight Petersson trace formula are implied by the Weil bound.
        \item The decomposition of $\mc{S}_{\frac{2l+1}{2}}^{+}(p)$ is furnished via the Shimura correspondence and the theory of the Kohen plus space.
    \end{itemize}
    The one serious obstruction seems to be that Ramanujan type bounds are not available for the Fourier coefficients of half-integral weight modular forms. It may be possible to resolve this obstruction by picking up additional decay in $p$ from other estimates.

\begin{thebibliography}{99}
    \bibitem{L}
    Luo, W. (1999). Special L-values of Rankin-Selberg convolutions. Mathematische Annalen, 314, 591-600.

    \bibitem{LR-1}
    Luo, W., \& Ramakrishnan, D. (1997). Determination of modular forms by twists of critical L-values. Inventiones mathematicae, 130(2), 371-398.

    \bibitem{LR-2}
    Luo, W., \& Ramakrishnan, D. (1997). Determination of modular elliptic curves by Heegner points. pacific journal of mathematics, 181(3), 251-258.

    \bibitem{PR}
    Pandey, M. K., \& Ramakrishnan, B. (2018, December). Determining Modular Forms of Half-Integral Weight by Central Values of Convolution L-Functions. In International conference on number theory (pp. 157-172). Singapore: Springer Singapore.

    \bibitem{GR}
    Gradshteyn, I. S., Ryzhik, I. M., \& Romer, R. H. (1988). Tables of integrals, series, and products.

    \bibitem{CT}
    Creech, S., \& Twiss, H. (2023). A converse theorem in half-integral weight. arXiv preprint arXiv:2306.02872.

\end{thebibliography}

\end{document}