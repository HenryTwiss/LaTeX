\documentclass[12pt,reqno,oneside]{amsart}
\usepackage{import}
%===============================%
%  Packages and basic settings  %
%===============================%
\usepackage[headheight=15pt,rmargin=0.5in,lmargin=0.5in,tmargin=0.75in,bmargin=0.75in]{geometry}
\usepackage{imakeidx}
\usepackage{framed}
\usepackage{amssymb}
\usepackage{amsmath}
\usepackage{mathrsfs}
\usepackage{enumitem}
\usepackage{hyperref}
\usepackage{appendix}
\usepackage[capitalise,noabbrev]{cleveref}
\usepackage{tikz}
\usepackage{tikz-cd}
\usepackage{nomencl}\makenomenclature
\usetikzlibrary{braids,arrows,decorations.markings,calc}

%====================================%
%  Theorems, environments & cleveref  %
%====================================%
\newtheorem{theorem}{Theorem}[section]
\newtheorem{proposition}{Proposition}[section]
\newtheorem{corollary}{Corollary}[section]
\newtheorem{lemma}{Lemma}[section]
\newtheorem{conjecture}{Conjecture}[section]
\newtheorem{remark}{Remark}[section]

\newenvironment{stabular}[2][1]
  {\def\arraystretch{#1}\tabular{#2}}
  {\endtabular}

%==================================%
%  Custom commands & environments  %
%==================================%
\newcommand{\legendre}[2]{\left(\frac{#1}{#2}\right)}
\newcommand{\dlegendre}[2]{\displaystyle{\left(\frac{#1}{#2}\right)}}
\newcommand{\tlegendre}[2]{\textstyle{\left(\frac{#1}{#2}\right)}}
\newcommand{\psum}{\sideset{}{'}\sum}
\newcommand{\asum}{\sideset{}{^{\ast}}\sum}
\newcommand{\tmod}[1]{\ \left(\text{mod }#1\right)}
\newcommand{\xto}[1]{\xrightarrow{#1}}
\newcommand{\xfrom}[1]{\xleftarrow{#1}}
\newcommand{\normal}{\mathrel{\unlhd}}
\newcommand{\mf}{\mathfrak}
\newcommand{\mc}{\mathcal}
\newcommand{\ms}{\mathscr}

\newcommand{\Mat}{\mathrm{Mat}}
\newcommand{\GL}{\mathrm{GL}}
\newcommand{\SL}{\mathrm{SL}}
\newcommand{\PSL}{\mathrm{PSL}}
\renewcommand{\O}{\mathrm{O}}
\newcommand{\SO}{\mathrm{SO}}
\newcommand{\U}{\mathrm{U}}
\newcommand{\Sp}{\mathrm{Sp}}

\newcommand{\N}{\mathbb{N}}
\newcommand{\Z}{\mathbb{Z}}
\newcommand{\Q}{\mathbb{Q}}
\newcommand{\R}{\mathbb{R}}
\newcommand{\C}{\mathbb{C}}
\newcommand{\F}{\mathbb{F}}
\renewcommand{\H}{\mathbb{H}}
\renewcommand{\P}{\mathbb{P}}

\renewcommand{\a}{\alpha}
\renewcommand{\b}{\beta}
\newcommand{\g}{\gamma}
\renewcommand{\d}{\delta}
\newcommand{\z}{\zeta}
\renewcommand{\t}{\theta}
\renewcommand{\i}{\iota}
\renewcommand{\k}{\kappa}
\renewcommand{\l}{\lambda}
\newcommand{\s}{\sigma}
\newcommand{\w}{\omega}

\newcommand{\G}{\Gamma}
\newcommand{\D}{\Delta}
\renewcommand{\L}{\Lambda}
\newcommand{\W}{\Omega}

\newcommand{\e}{\varepsilon}
\newcommand{\vt}{\vartheta}
\newcommand{\vphi}{\varphi}
\newcommand{\emt}{\varnothing}

\newcommand{\x}{\times}
\newcommand{\ox}{\otimes}
\newcommand{\op}{\oplus}
\newcommand{\bigox}{\bigotimes}
\newcommand{\bigop}{\bigoplus}
\newcommand{\del}{\partial}
\newcommand{\<}{\langle}
\renewcommand{\>}{\rangle}
\newcommand{\lf}{\lfloor}
\newcommand{\rf}{\rfloor}
\newcommand{\wtilde}{\widetilde}
\newcommand{\what}{\widehat}
\newcommand{\conj}{\overline}
\newcommand{\cchi}{\conj{\chi}}

\DeclareMathOperator{\id}{\textrm{id}}
\DeclareMathOperator{\sgn}{\mathrm{sgn}}
\DeclareMathOperator{\im}{\mathrm{im}}
\DeclareMathOperator{\rk}{\mathrm{rk}}
\DeclareMathOperator{\tr}{\mathrm{trace}}
\DeclareMathOperator{\nm}{\mathrm{norm}}
\DeclareMathOperator{\ord}{\mathrm{ord}}
\DeclareMathOperator{\Hom}{\mathrm{Hom}}
\DeclareMathOperator{\End}{\mathrm{End}}
\DeclareMathOperator{\Aut}{\mathrm{Aut}}
\DeclareMathOperator{\Tor}{\mathrm{Tor}}
\DeclareMathOperator{\Ann}{\mathrm{Ann}}
\DeclareMathOperator{\Gal}{\mathrm{Gal}}
\DeclareMathOperator{\Trace}{\mathrm{Trace}}
\DeclareMathOperator{\Norm}{\mathrm{Norm}}
\DeclareMathOperator{\Span}{\mathrm{Span}}
\DeclareMathOperator*{\Res}{\mathrm{Res}}
\DeclareMathOperator{\Vol}{\mathrm{Vol}}
\DeclareMathOperator{\Li}{\mathrm{Li}}
\renewcommand{\Re}{\mathrm{Re}}
\renewcommand{\Im}{\mathrm{Im}}

\newcommand{\GH}{\G\backslash\H}
\newcommand{\GG}{\G_{\infty}\backslash\G}

\newenvironment{psmallmatrix}
  {\left(\begin{smallmatrix}}
  {\end{smallmatrix}\right)}

%============%
%  Comments  %
%============%
\newcommand{\todo}[1]{\textcolor{red}{\sf Todo: [#1]}}

%===================%
%  Label reminders  %
%===================%
% [label=(\roman*)]
% [label=(\alph*)]
% [label=(\arabic{enumi})]

%==================%
%  Other settings  %
%==================%
\pgfdeclarelayer{background}
\pgfsetlayers{background,main}
\tikzset{->-/.style={decoration={
  markings,
  mark=at position .5 with {\arrow{>}}},postaction={decorate}}}

%=================%
%  Title & Index  %
%=================%
\title{A quadratic double Dirichlet series over number fields}
\author{Henry Twiss}
\date{\today}
\makeindex

\begin{document}

\begin{abstract}
    We construct a quadratic double Dirichlet series $Z(s,w)$ built from single variable quadratic Dirichlet $L$-functions $L(s,\chi)$ over $\Q$. We prove that $Z(s,w)$ admits meromorphic continuation to the $(s,w)$-plane and satisfies a group of functional equations.
\end{abstract}

\maketitle

\section{Preliminaries}
    We present an overview of quadratic Dirichlet $L$-functions over $\Q$. We begin with the Riemann zeta-function. The zeta function $\z(s)$ is defined as the Dirichlet series or Euler product
    \[
        \z(s) = \sum_{n \ge 1}\frac{1}{n^{s}} = \prod_{\text{$p$ prime}}\left(1-\frac{1}{p^{s}}\right)^{-1},
    \]
    for $\Re(s) > 1$. The second equality is an analytic reformulation of the fundamental theorem of arithmetic. The Riemann zeta function also admits meromorphic continuation to $\C$ with a simple pole at $s = 1$ of residue $1$. The functional equation is
    \[
        \pi^{-\frac{s}{2}}\G\left(\frac{s}{2}\right)\z(s) = \pi^{-\frac{1-s}{2}}\G\left(\frac{1-s}{2}\right)\z(1-s).
    \]
    Now we recall the characters on $\Z$. They are multiplicative functions $\chi:\Z \to \C$. Their image always lands in the roots of unity. The two flavors of characters of interest to use are:
    
    \begin{itemize}
        \item Dirichlet characters: multiplicative functions $\chi_{m}:\Z \to \C$ modulo $m \ge 1$ (in that they are $m$-periodic) and such that $\chi_{m}(n) = 0$ if $(m,n) > 1$.
        \item Hilbert symbols: Dirichlet characters modulo $1$.
    \end{itemize}
    
    If $\chi$ is a character then its conjugate $\conj{\chi}$ is also a character. Moreover, $\conj{\chi}$ is the multiplicative inverse to $\chi$ and the characters modulo $m$ form a group under multiplication. This group is always finite and its order is $\phi(m)$. Characters also satisfy orthogonality properties:

    \begin{theorem}[Orthogonality relations]
        Let $\chi$ and $\psi$ be any two Dirichlets character modulo $m$ and let $a$ and $b$ be any two integers modulo $m$. Then
        \begin{enumerate}[label=(\roman*)]
          \item
          \[
            \frac{1}{\phi(m)}\sum_{a \tmod{m}}\chi(a)\conj{\psi}(a) = \d_{\chi,\psi}.
          \]
          \item
          \[
            \frac{1}{\phi(m)}\sum_{\chi \tmod{m}}\chi(a)\cchi(b) = \d_{a,b}.
          \]
        \end{enumerate}
    \end{theorem}

    \todo{continue here} The Dirichlet characters that are of interest to us are those given by the quadratic residue symbol on $\F_{q}[t]$. If $f \in \F_{q}[t]$ is a monic non-constant irreducible, define the quadratic residue symbol $\chi_{f}$ by
    \[
        \chi_{f}(g) = \legendre{f}{g} = g^{\frac{|f|-1}{2}} \pmod{f},
    \]
    for any $g \in \F_{q}[t]$. Then $\chi_{f}(g) \in \{\pm 1\}$ provided $f$ and $g$ are relatively prime and $\chi_{f}(g) = 0$ if $(f,g) > 1$. If $b \in \F^{\x}$, then we define the quadratic residue symbol $\chi_{b}$ by
    \[
        \chi_{b}(g) = \legendre{b}{m} = \sgn(b)^{\deg(f)},
    \]
    where $\sgn(b) = \pm1$ depending on if $b \in (\F^{\x})^{2}$ or not. Moreover, if $d \in \F_{q}[t]$ then we set $\sgn(d) = \sgn(b_{n})$ if $d(t) = b_{n}t^{n}+b_{n-1}t^{n+1}+\cdots+b_{0}$ (with $b_{n} \neq 0$). Extending $\chi_{f}$ multiplicativity in $f$, $\chi_{f}$ is defined for any $f$ not necessarily monic. The quadratic residue symbol also has the following reciprocity property:

    \begin{theorem}[Quadratic reciprocity]
        If $f,g \in \F_{q}[t]$ are monic, square-free, and relatively prime, then
        \[
            \legendre{f}{g} = (-1)^{\frac{q-1}{2}\deg(f)\deg(g)}\legendre{g}{f}.
        \]
    \end{theorem}

    Note that if $q \equiv 1 \tmod{4}$, the sign in the statement of quadratic reciprocity is always $1$ so that the reciprocity is perfect. We now describe the Hilbert symbols on $\F_{q}[t]$. In fact, there are only two Hilbert symbols, one non-trivial, and one trivial. The non-trivial Hilbert symbol is $\chi_{\t}$ where $\t \in \F^{\x}-(\F^{\x})^{2}$:
    \[
        \chi_{\t}(f) = (-1)^{\deg(f)}.
    \]
    Note that $\conj{\chi_{\t}} = \chi_{\t}$. The other Hilbert symbol is the trivial character $\chi_{\t}^{2} = \chi_{\t\t} = \chi_{1}$. In general, we denote a Hilbert symbol by $\chi_{a}$ where $a \in \{1,\t\}$.

    We can now define the $L$-functions attached to the symbol $\chi_{f}$ for not necessarily monic $f$. We define the $L$-series $L(s,\chi_{f})$ attached to $\chi_{f}$ by a Dirichlet series or Euler product:
    \[
        L(s,\chi_{f}) = \sum_{\text{$g$ monic}}\frac{\chi_{f}(g)}{|g|^{s}} = \prod_{\text{$P$ monic irr}}\left(1-\frac{\chi_{f}(P)}{|P|^{s}}\right)^{-1}.
    \]
    By definition of the quadratic residue symbol, $L(s,\chi_{f}) \ll \z(s)$ for $\Re(s) > 1$ so that $L(s,\chi_{f})$ is absolutely uniformly convergent on compacta in this region. $L(s,\chi_{f})$ also admits meromorphic continuation to $\C$ with a simple pole at $s = 1$ if $f$ is square-free and is analytic otherwise (see \cite{R} for a proof). Moreover, $L(s,\chi_{f})$ is a polynomial in $q^{-s}$ of degree at most $\deg(f)-1$. The completed $L$-function is defined as follows:
    \[
        L^{\ast}(s,\chi_{f}) = \begin{cases} \frac{1}{1-q^{-s}}L(s,\chi_{f}) & \text{if $\deg(f)$ is even}, \\ L(s,\chi_{f}) & \text{if $\deg(f)$ is odd}, \end{cases}
    \]
    and satisfies the functional equation
    \[
        L^{\ast}(s,\chi_{f}) = \begin{cases} q^{2s-1}|f|^{\frac{1}{2}-s}L^{\ast}(1-s,\chi_{f}) & \text{if $\deg(f)$ is even}, \\ q^{2s-1}(q|f|)^{\frac{1}{2}-s}L^{\ast}(1-s,\chi_{f}) & \text{if $\deg(f)$ is odd}. \end{cases}
    \]
    Note that in the case $\deg(f)$ is even, the conductor is $|f|$ and in the case $\deg(f)$ is odd, the conductor is $q|f|$. In other words, the gamma factors depend upon the degree of $f$. This will cause a small but important technical issue later when we want to derive functional equations for the quadratic double Dirichlet series.
\section*{The Quadratic Double Dirichlet Series}
\section*{The Interchange}
\section*{Weighting Terms}
\section*{Funcational Equations}
\section*{Meromorphic Continuation}
\section*{Poles and Residues}

\begin{thebibliography}{99}
    \bibitem{R}
    Rosen, M. (2002). Number theory in function fields (Vol. 210). Springer Science \& Business Media.

    \bibitem{H}
    Hormander, L. (1973). An introduction to complex analysis in several variables. Elsevier.

    \bibitem{CG}
    Chinta, G., \& Gunnells, P. E. (2007). Weyl group multiple Dirichlet series constructed from quadratic characters. Inventiones mathematicae, 167, 327-353.

    \bibitem{S}
    Stanley, R. (2023). Enumerative Combinatorics: Volume 2. Cambridge University Press.
\end{thebibliography}

\end{document}