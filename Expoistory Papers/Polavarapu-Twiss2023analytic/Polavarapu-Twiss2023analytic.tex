\documentclass[12pt,reqno]{amsart}
\usepackage{fullpage}
\usepackage{amsmath}
\usepackage{times}
\usepackage{color}
\usepackage{textcomp}
\usepackage{esint}
\usepackage{graphicx}
\usepackage[colorlinks=true, pdfstartview=FitH, linkcolor=blue, citecolor=blue, urlcolor=blue]{hyperref}
\usepackage{mathrsfs}
%\usepackage{fouriernc,amsmath}

\newtheorem{theorem}{Theorem}[section]
\newtheorem{lemma}[theorem]{Lemma}
\newtheorem{proposition}[theorem]{Proposition}
\newtheorem{corollary}[theorem]{Corollary}
\theoremstyle{definition}
\newtheorem{definition}[theorem]{Definition}
\newtheorem{notation}[theorem]{Notation}
\newtheorem{conjecture}[theorem]{Conjecture}
\newtheorem{example}[theorem]{Example}
\newtheorem{remark}[theorem]{Remark}
\numberwithin{equation}{section}
%\numberwithin{table}{section} 
%\numberwithin{figure}{section}

% Base TeX has the \mod and \pmod commands; we never use \mod in number theory, and the default spacing of \pmod is terrible. I use this \mod command, which includes parentheses but also has better spacing then \pmod.
\renewcommand{\mod}[1]{{\ifmmode\text{\rm\ (mod~$#1$)}\else\discretionary{}{}{\hbox{ }}\rm(mod~$#1$)\fi}}
\newcommand{\ep}{\varepsilon}
\newcommand{\C}{{\mathbb C}}
\newcommand{\N}{{\mathbb N}}
\newcommand{\R}{{\mathbb R}}
\newcommand{\Z}{{\mathbb Z}}
\renewcommand{\labelenumi}{(\alph{enumi})}
\vfuzz=2pt

\begin{document}

\begin{abstract}
    We describe the analytic continuation of Dirichlet $L$-functions $L(s,\chi)$ arising from primitive characters of modulus $q > 1$ by taking the Mellin transform of a theta function. This is preluded by a recount of the analytic continuation of the Riemann zeta function in a similar manner. After proving the analytic continuation of these Dirichlet series, we give a short discussion on the underlying technique of taking the Mellin transform of a theta function and discuss the case of $L$-functions corresponding to integral weight modular forms.
\end{abstract}

\title{Analytic continuation of Dirichlet $L$-functions \& the Mellin transform}
\author{Achintya Polavarapu, Henry Twiss}
\maketitle
\thispagestyle{empty}

\section{The \texorpdfstring{$\zeta$}{z}-function \& Historical Remarks}
    The \textbf{Riemann zeta function} $\zeta(s)$ is the Dirichlet series defined as
    \[
        \zeta(s) = \sum_{n \ge 1}\frac{1}{n^{s}}.
    \]
    The zeta function is intimately connected to the distribution of primes (see \cite{H,P}) and has been one of the cornerstones of analytic number theory since its birth. It arose from Euler's study of sums of the form
    \[
        \sum_{n \ge 1}\frac{1}{n^{k}} = 1+\frac{1}{2^{k}}+\frac{1}{3^{k}}+\cdots,
    \]
    where $k \ge 2$ is an integer (see \cite{E} for more). By standard series tests, $\zeta(s)$ is absolutely uniformly convergent in the half-plane $\Re(s) > 1$ and hence defines a holomorphic function there. While it has a singularity at $s = 1$ by Landau's theorem, in 1859 Riemann analytically continued $\zeta(s)$ to all of $\C$ with a simple pole at $s = 1$ of residue $1$ (see \cite{R}). This was achieved by deriving the integral representation for $\Re(s) > 1$:
    \begin{equation}\label{equ:zeta_int_rep}
        \zeta(s) = \frac{\pi^{s/2}}{\Gamma\left(s/2\right)}\left[-\frac{1}{s(1-s)}+\int_{1}^{\infty}\theta(x)x^{(1-s)/2}\,\frac{dx}{x}+\int_{1}^{\infty}\theta(x)x^{s/2}\,\frac{dx}{x}\right],
    \end{equation}
    where $\theta(x) = \sum_{n \ge 1}e^{-\pi n^{2}x}$. This is ``essentially'' \textbf{Jacobi's theta function} $\vartheta(x)$ as
    \[
        \vartheta(x) = \sum_{n \in \mathbb{Z}}e^{-\pi n^{2}x} = 1+2\sum_{n \ge 1}e^{-\pi n^{2}x} = 1+2\theta(x).
    \]
    One derives \ref{equ:zeta_int_rep} from the preliminary integral representation
    \begin{equation}\label{equ:prelim_zeta_int_rep}
        \zeta(s) = \frac{\pi^{s/2}}{\Gamma\left(s/2\right)}\int_{0}^{\infty}\theta(x)x^{s/2}\,\frac{dx}{x}.
    \end{equation}
    This preliminary integral representation is achieved by taking the Mellin transform of the theta function $\theta(x)$. Unfortunately, while $\theta(x)$ admits exponential decay as $x \to \infty$, it does not converge as $x \to 1$ and so we cannot conclude that the integral in \ref{equ:prelim_zeta_int_rep} is analytic on $\C$. To turn \ref{equ:prelim_zeta_int_rep} into \ref{equ:zeta_int_rep}, Riemann used the following result known to Jacobi (see \cite{R}), namely
    \[
        \vartheta(s) = \frac{1}{\sqrt{s}}\vartheta\left(\frac{1}{s}\right),
    \]
    to obtain \ref{equ:zeta_int_rep}. This is necessary because in \ref{equ:zeta_int_rep}, $\theta(x)$ converges at $s = 1$ and so the integrals are analytic on $\C$. Therefore the right-hand side is naturally defined for all $s \in \mathbb{C}-\{1\}$ and at $s = 1$ the polynomial term has a simple pole of residue $1$. So taking the right-hand side as the definition of $\zeta(s)$, we see that $\zeta(s)$ is analytic on $\C$ with a simple pole at $s = 1$ of residue $1$ as previously mentioned. Moreover, by the natural symmetry of the two integral terms under $s \to 1-s$ and invariance of the polynomial term, $\zeta(s)$ also possesses the symmetric functional equation
    \[
        \frac{\Gamma\left(s/2\right)}{\pi^{s/2}}\zeta(s) = \frac{\Gamma\left((1-s)/2\right)}{\pi^{(1-s)/2}}\zeta(1-s).
    \]
    This can be viewed as the Mellin transform lifting of the transformation law for $\vartheta(s)$ to $\zeta(s)$, and it is with this functional equation that we can use the Dirichlet series representation of $\zeta(s)$ for $\Re(s) > 1$ to determine information about $\zeta(s)$ in the region $\Re(s) < 0$.

\section{The Functional Equation for Dirichlet \texorpdfstring{$L$}{L}-functions}
    The Dirichlet $L$-function attached to the character $\chi$ is the series
    \[
        L(s, \chi) = \sum_{n \ge 1}\frac{\chi(n)}{n^s}.
    \]
    Throughout let $q$ be the conductor of $\chi$. If $q = 1$ then we recover $\zeta(s)$. Since $\chi(n) \ll 1$, $L(s,\chi)$ converges absolutely uniformly for $\Re(s) > 1$. The series does not necessarily converge absolutely uniformly for $\Re(s) \le 1$, but it does admit analytic continuation to this region analogous to the case for $\zeta(s)$. Precisely, we will show the following:

    \begin{theorem}
        For a primitive Dirichlet character $\chi$ with conductor $q > 1$, $L(s,\chi)$ admits analytic continuation to $\C$.
    \end{theorem}

    We will derive the analytic continuation of $L(s,\chi)$ by expressing the $L$-function as an integral which will converge on all of $\C$. Details in the argument depend on if $\chi$ is even or odd, so to treat both cases simultaneously we define $\delta_{\chi} \in \{0,1\}$ by $\chi(-1) = (-1)^{\delta_{\chi}}$. 
    
    \begin{proof}[Proof sketch]         
        Upon substituting $s \to s+\delta_{\chi}$ into the definition of the gamma function, we obtain
        \[
            \chi(n)\Gamma\left((s+\delta_{\chi})/2\right) = \pi^{(s+\delta_{\chi})/2}n^{s}\int_{0}^{\infty}\chi(n)n^{\delta_{\chi}}e^{-\pi n^{2}x}x^{(s+\delta_{\chi})/2}\,\frac{dx}{x}.
        \]
        Proceeding exactly as for the zeta function (sum over $n \ge 1$ and apply some minor algebra), we arrive at the preliminary integral representation:
        \begin{equation}\label{equ:prelim_int_rep}
            L(s,\chi) = \frac{\pi^{(s+\delta_{\chi})/2}}{\Gamma\left((s+\delta_{\chi})/2\right)}\int_{0}^{\infty}\theta_{\chi}(x)x^{(s+\delta_{\chi})/2}\,\frac{dx}{x}.
        \end{equation}
        where
        \[
            \theta_{\chi}(x) = \sum_{n \ge 1}\chi(n)n^{\delta_{\chi}}e^{-\pi n^{2}x} = \frac{1}{2}\sum_{n \in \Z}\chi(n)n^{\delta_{\chi}}e^{-\pi n^{2}x}.
        \]
        This is essentially a twisted version of Jacobi's theta function. The key insight is that it is a sum of Schwarz functions over a lattice and so we can expect that an application of Poisson summation will give a functional equation of shape $s \to \frac{1}{s}$ just as for Jacobi's theta function in the case of $\zeta(s)$. Since our theta function has a character attached, we first sieve out the character:
        \[
            \theta_{\chi}(x) = \sum_{a \mod{q}}\chi(a)\sum_{m \in \Z}(mq+a)^{\delta_{\chi}}e^{-\pi(mq+a)^{2}x}.
        \]
        Now we apply Poisson summation to the inner sum. Set $f(y)
     = (yq+a)^{\delta_{\chi}}e^{-\pi(yq+a)^{2}x}$. Making a change of variable and completing the square of $(yq+a)^{2}x+2\pi ity$ in the exponent, the Fourier transform of $f$ becomes
     \[
        \hat{f}(t) = \int_{-\infty}^{\infty}(yq+a)^{\delta_{\chi}}e^{-\pi(yq+a)^{2}x}e^{-2\pi ity}\,dx = \frac{e^{\frac{2\pi iat}{q}}e^{-\frac{\pi t^{2}}{q^{2}s}}}{qs^{\frac{1+\delta_{\chi}}{2}}}\int_{-\infty}^{\infty}x^{\delta_{\chi}}e^{-\pi\left(x+\frac{it}{q\sqrt{s}}\right)^{2}}\,dx.
     \]
     One now complexifies the integral and shifts the line of integration to to $\Im(z) = \frac{t}{q\sqrt{s}}$, with no addition of residues since the integrand is holomorphic, obtaining
     \[
         \frac{e^{\frac{2\pi iat}{q}}}{q}\frac{e^{-\frac{\pi t^{2}}{q^{2}s}}}{\sqrt{s}}\int_{-\infty}^{\infty}\left(x-\frac{it}{qs}\right)^{\delta_{\chi}}\,dx = \frac{e^{\frac{2\pi iat}{q}}}{q}\frac{e^{-\frac{\pi t^{2}}{q^{2}s}}}{\sqrt{s}}\left(\frac{it}{qs}\right)^{\delta_{\chi}},
     \]
     where the equality follows by realising the integral as essentially a Gaussian integral. Poisson summation then yields
     \begin{align*}
        \theta_{\chi}(x) &= \sum_{a \mod{q}}\chi(a)\sum_{m \in \Z}(mq+a)^{\delta_{\chi}}e^{-\pi(mq+a)^{2}x} \\
        &= \sum_{a \mod{q}}\chi(a)\sum_{t \in \Z}\frac{e^{\frac{2\pi iat}{q}}}{q}\frac{e^{-\frac{\pi t^{2}}{q^{2}s}}}{\sqrt{s}}\left(\frac{it}{qs}\right)^{\delta_{\chi}} \\
        &= \frac{1}{i^{\delta_{\chi}}q^{1+\delta_{\chi}}s^{\frac{1}{2}+\delta_{\chi}}}\sum_{t \in \Z}t^{\delta_{\chi}}e^{-\frac{\pi t^{2}}{q^{2}s}}\tau(t,\chi) && \text{evaluation of $\tau(t,\chi)$}\\
        &= \frac{\varepsilon_{\chi}}{i^{\delta_{\chi}}q^{1+\delta_{\chi}}s^{\frac{1}{2}+\delta_{\chi}}}\theta_{\overline{\chi}}\left(\frac{1}{q^{2}x}\right).
     \end{align*}
     This is the appropriate transformation law for the twisted theta function. We can now derive the symmetric integral representation for $L(s,\chi)$. Ignoring the gamma factor in \ref{equ:prelim_int_rep} and splitting the integral at $x = 1/q$, the fixed point of the transformation law, we have
     \begin{equation}\label{equ:split_int_rep}
        \int_{0}^{\infty}\theta_{\chi}(x)x^{(s+\delta_{\chi})/2}\,\frac{dx}{x} = \int_{0}^{1/q}\theta_{\chi}(x)x^{(s+\delta_{\chi})/2}\,\frac{dx}{x}+\int_{1/q}^{\infty}\theta_{\chi}(x)x^{(s+\delta_{\chi})/2}\,\frac{dx}{x}.
     \end{equation}
     Now change variables $x \to \frac{1}{q^{2}x}$ in the first integral and apply the transformation law:
     \[
        \int_{0}^{1/q}\theta_{\chi}(x)x^{(s+\delta_{\chi})/2}\,\frac{dx}{x} = \int_{1/q}^{\infty}\theta_{\chi}\left(\frac{1}{q^{2}}\right)x^{-(s+\delta_{\chi})/2}\,\frac{dx}{x} = \frac{\varepsilon_{\chi}}{i^{\delta_{\chi}}}\int_{1/q}^{\infty}\theta_{\overline{\chi}}(x)x^{((1-s)+\delta_{\chi})/2}\,\frac{dx}{x}.
     \]
     Substituting back into \ref{equ:split_int_rep} and applying \ref{equ:prelim_int_rep} yields
     \begin{equation}\label{equ:a_sym_int_rep}
        L(s,\chi) = \frac{\pi^{(s+\delta_{\chi})/2}}{\Gamma\left((s+\delta_{\chi})/2\right)}\left[\frac{\varepsilon_{\chi}}{i^{\delta_{\chi}}}\int_{1/q}^{\infty}\theta_{\overline{\chi}}(x)x^{((1-s)+\delta_{\chi})/2}\,\frac{dx}{x}+\int_{1/q}^{\infty}\theta_{\chi}(x)x^{(s+\delta_{\chi})/2}\,\frac{dx}{x}\right]
     \end{equation}
    \end{proof}
 
    By virtue of the decay of $\theta_{\chi}(x)$ both integrals in \ref{equ:a_sym_int_rep} are holomorphic on $\C$ and thus the right-hand side gives analytic continuation to $\C$. Also, the symmetry of the right-hand side as $s \to 1-s$ immediately results in the following corollary which is better known as the functional equation for $L(s,\chi)$:

    \begin{corollary}
     \[
        q^{s/2}\frac{\Gamma\left((s+\delta_{\chi})/2\right)}{\pi^{(s+\delta_{\chi})/2}}L(s,\chi) = \frac{\varepsilon_{\chi}}{i^{\delta_{\chi}}}q^{(1-s)/2}\frac{\Gamma\left(((1-s)+\delta_{\chi})/2\right)}{\pi^{((1-s)+\delta_{\chi})/2}}L(1-s,\chi).
    \]
    \end{corollary}

\section{The Mellin Transform \& Theta Functions}
    The unifying idea underpinning functional equations of $L$-functions is to find an integral representation that is symmetric under $s \to 1-s$. The integral representation is obtained by taking the Mellin transform of a theta function, and the symmetry of the integral is lifted from a transformation law for the theta function. Let us being with the Mellin transform. If $f$ is some continuous function, then the \textbf{Mellin transform} $\{\mathcal{M}f\}(s)$ of $f$ is given by
    \[
      \{\mathcal{M}f\}(s) = \int_{0}^{\infty}f(x)x^{s}\,\frac{dx}{x}.
    \]
    If $f$ is a sufficiently nice function, say bounded near $0$ and of exponential decay near $\infty$, this integral converge in a half-plane. The classical example is when $f = e^{-x}$ so that $\{\mathcal{M}e^{-x}\}(s) = \Gamma(s)$. In our case, we want $f$ to be a theta function. A \textbf{theta function} is a absolutely convergent series that is a sum of exponentials over $\mathbb{Z}$ that is symmetric in the sign of $\mathbb{Z}$. Both the zeta function and Dirichlet $L$-functions are associated to a theta function:
    \begin{align*}
      \zeta(s) &\longleftrightarrow \theta(x) = \sum_{n \in \mathbb{Z}}e^{-\pi n^{2}x} = 1+2\sum_{n \ge 1}e^{-\pi n^{2}x}, \\
      L(s,\chi) &\longleftrightarrow \theta_{\chi}(x) = \sum_{n \in \mathbb{Z}}\chi(n)n^{\delta_{\chi}}e^{-\pi n^{2}x} = 2\sum_{n \ge 1}\chi(n)n^{\delta_{\chi}}e^{-\pi n^{2}x}.
    \end{align*}
    By ``associated'' we mean that if one takes the Mellin transform over the subsum $n \ge 1$ on the left-hand side, then the corresponding $L$-functions on the right-hand side is obtained up to gamma factors. For example, this is \ref{equ:prelim_int_rep}. More generally, given some theta function $\theta(x)$ we can obtain an $L$-functions $L(s,\theta)$ by taking the Mellin transform. In order to obtain a functional equation for the $L$-functions, the theta function must admit a transformation law:
    \[
        \theta(x) \sim \theta\left(1/cx\right),
    \]
    for some $c > 0$. In this case, we can decompose the Mellin transform as
    \[
      \{\mathcal{M}f\}(s) = \int_{0}^{1/\sqrt{c}}f(x)x^{s}\,\frac{dx}{x}+\int_{1/\sqrt{c}}^{\infty}f(x)x^{s}\,\frac{dx}{x}.
    \]
    Making the change of variables $x \to 1/cx$ to the first integral, we can apply the transformation law and symmetrize the Mellin transformation to respect $s \to 1-s$ as much as possible. Roughly,
    \[
        L(s,\theta) = \text{polar factor}+\int_{1/\sqrt{c}}^{\infty}\theta(x)x^{1-s}\,\frac{dx}{x}+\int_{1/\sqrt{c}}^{\infty}\theta(x)x^{s}\,\frac{dx}{x}
    \]
    
    The resulting integrals will be analytic by virtue of the rapid decay of $\theta(x)$, and therefore give analytic continuation of the $L$-functions to $\C$. The functional equation then follows immediately from the symmetry of the integral representation.

    Let's give an example. If $f$ is a weight $k$ cuspidal modular form on the full modular group $\mathrm{PSL}_{2}(\Z)$, then $f$ admits a Fourier expansion at the $\infty$ cusp:
    \[
        f(z) = \sum_{n \ge 1}a(n)e^{2\pi inz}.
    \]
    We can package the Fourier coefficient of $f$ into an $L$-functions $L(s,f)$ called the $L$-functions associated to $f$:
    \[
        L(s,f) = \sum_{n \ge 1}\frac{a_{f}(n)}{n^{s}}.
    \]
    where $a_{f}(n) = a(n)n^{-(k-1)/2}$. By the Ramanujan conjecture (see \cite{D}), $a_{f}(n) \ll 1$ so that $L(s,f)$ converges absolutely uniformly on compact sets for $\Re(s) > 1$. We would like to analytically continue $L(s,f)$ in the same way as for the zeta function an Dirichlet $L$-functions. What's the underlying theta function? Well, it comes naturally equip to $f$ as the Fourier series of $f$ along the positive imaginary axis:
    \[
        f(iy) = \sum_{n \ge 1}a_{\infty}(n)e^{-2\pi ny}.
    \]
    Due to the negative sign in the exponent, it exhibits the required exponential decay and by modularity
    \[
        f\left(\begin{pmatrix} 0 & 1 \\ -1 & 0 \end{pmatrix}iy\right) = (-iy)^{k}f\left(1/iy\right).
    \]
    This transformation law is more geometric in nature since the modularity of $f$ describes how $f$ changes under a M\"obius transformation.

\begin{thebibliography}{99}
    \bibitem{H}
    Hadamard, Jacques. "Sur la distribution des zéros de la fonction $\zeta (s) $ et ses conséquences arithmétiques." Bulletin de la Societé mathematique de France 24 (1896): 199-220.

    \bibitem{P}
    Poussin, Charles Jean de La Vallée. Sur la fonction [zeta](s) de Riemann et le nombre des nombres premiers inferieurs à une limite donnée. Vol. 51. Hayez, 1899.

    \bibitem{E}
    Leonhard Euler. Variae observationes circa series infinitas. Commentarii academiae scientiarum
    imperialis Petropolitanae, 9(1737):160–188, 1737.

    \bibitem{R}
    Riemann, Bernhard. "Ueber die Anzahl der Primzahlen unter einer gegebenen Grosse." Ges. Math. Werke und Wissenschaftlicher Nachlaß 2.145-155 (1859): 2.

    \bibitem{D}
    Deligne P. 1971Formes modulaires et représentations l-adiques. Séminaire Bourbaki vol. 1968/69, Lecture Notes in Math.179. New York, NY: Springer.
\end{thebibliography}

\end{document}