\documentclass[12pt,reqno,oneside]{amsart}
\usepackage{import}
%===============================%
%  Packages and basic settings  %
%===============================%
\usepackage[headheight=15pt,rmargin=0.5in,lmargin=0.5in,tmargin=0.75in,bmargin=0.75in]{geometry}
\usepackage{imakeidx}
\usepackage{framed}
\usepackage{amssymb}
\usepackage{amsmath}
\usepackage{mathrsfs}
\usepackage{enumitem}
\usepackage{hyperref}
\usepackage{appendix}
\usepackage[capitalise,noabbrev]{cleveref}
\usepackage{tikz}
\usepackage{tikz-cd}
\usepackage{nomencl}\makenomenclature
\usetikzlibrary{braids,arrows,decorations.markings,calc}

%====================================%
%  Theorems, environments & cleveref  %
%====================================%
\newtheorem{theorem}{Theorem}[section]
\newtheorem{proposition}{Proposition}[section]
\newtheorem{corollary}{Corollary}[section]
\newtheorem{lemma}{Lemma}[section]
\newtheorem{conjecture}{Conjecture}[section]
\newtheorem{remark}{Remark}[section]

\newenvironment{stabular}[2][1]
  {\def\arraystretch{#1}\tabular{#2}}
  {\endtabular}

%==================================%
%  Custom commands & environments  %
%==================================%
\newcommand{\legendre}[2]{\left(\frac{#1}{#2}\right)}
\newcommand{\dlegendre}[2]{\displaystyle{\left(\frac{#1}{#2}\right)}}
\newcommand{\tlegendre}[2]{\textstyle{\left(\frac{#1}{#2}\right)}}
\newcommand{\psum}{\sideset{}{'}\sum}
\newcommand{\asum}{\sideset{}{^{\ast}}\sum}
\newcommand{\tmod}[1]{\ \left(\text{mod }#1\right)}
\newcommand{\xto}[1]{\xrightarrow{#1}}
\newcommand{\xfrom}[1]{\xleftarrow{#1}}
\newcommand{\normal}{\mathrel{\unlhd}}
\newcommand{\mf}{\mathfrak}
\newcommand{\mc}{\mathcal}
\newcommand{\ms}{\mathscr}

\newcommand{\Mat}{\mathrm{Mat}}
\newcommand{\GL}{\mathrm{GL}}
\newcommand{\SL}{\mathrm{SL}}
\newcommand{\PSL}{\mathrm{PSL}}
\renewcommand{\O}{\mathrm{O}}
\newcommand{\SO}{\mathrm{SO}}
\newcommand{\U}{\mathrm{U}}
\newcommand{\Sp}{\mathrm{Sp}}

\newcommand{\N}{\mathbb{N}}
\newcommand{\Z}{\mathbb{Z}}
\newcommand{\Q}{\mathbb{Q}}
\newcommand{\R}{\mathbb{R}}
\newcommand{\C}{\mathbb{C}}
\newcommand{\F}{\mathbb{F}}
\renewcommand{\H}{\mathbb{H}}
\renewcommand{\P}{\mathbb{P}}

\renewcommand{\a}{\alpha}
\renewcommand{\b}{\beta}
\newcommand{\g}{\gamma}
\renewcommand{\d}{\delta}
\newcommand{\z}{\zeta}
\renewcommand{\t}{\theta}
\renewcommand{\i}{\iota}
\renewcommand{\k}{\kappa}
\renewcommand{\l}{\lambda}
\newcommand{\s}{\sigma}
\newcommand{\w}{\omega}

\newcommand{\G}{\Gamma}
\newcommand{\D}{\Delta}
\renewcommand{\L}{\Lambda}
\newcommand{\W}{\Omega}

\newcommand{\e}{\varepsilon}
\newcommand{\vt}{\vartheta}
\newcommand{\vphi}{\varphi}
\newcommand{\emt}{\varnothing}

\newcommand{\x}{\times}
\newcommand{\ox}{\otimes}
\newcommand{\op}{\oplus}
\newcommand{\bigox}{\bigotimes}
\newcommand{\bigop}{\bigoplus}
\newcommand{\del}{\partial}
\newcommand{\<}{\langle}
\renewcommand{\>}{\rangle}
\newcommand{\lf}{\lfloor}
\newcommand{\rf}{\rfloor}
\newcommand{\wtilde}{\widetilde}
\newcommand{\what}{\widehat}
\newcommand{\conj}{\overline}
\newcommand{\cchi}{\conj{\chi}}

\DeclareMathOperator{\id}{\textrm{id}}
\DeclareMathOperator{\sgn}{\mathrm{sgn}}
\DeclareMathOperator{\im}{\mathrm{im}}
\DeclareMathOperator{\rk}{\mathrm{rk}}
\DeclareMathOperator{\tr}{\mathrm{trace}}
\DeclareMathOperator{\nm}{\mathrm{norm}}
\DeclareMathOperator{\ord}{\mathrm{ord}}
\DeclareMathOperator{\Hom}{\mathrm{Hom}}
\DeclareMathOperator{\End}{\mathrm{End}}
\DeclareMathOperator{\Aut}{\mathrm{Aut}}
\DeclareMathOperator{\Tor}{\mathrm{Tor}}
\DeclareMathOperator{\Ann}{\mathrm{Ann}}
\DeclareMathOperator{\Gal}{\mathrm{Gal}}
\DeclareMathOperator{\Trace}{\mathrm{Trace}}
\DeclareMathOperator{\Norm}{\mathrm{Norm}}
\DeclareMathOperator{\Span}{\mathrm{Span}}
\DeclareMathOperator*{\Res}{\mathrm{Res}}
\DeclareMathOperator{\Vol}{\mathrm{Vol}}
\DeclareMathOperator{\Li}{\mathrm{Li}}
\renewcommand{\Re}{\mathrm{Re}}
\renewcommand{\Im}{\mathrm{Im}}

\newcommand{\GH}{\G\backslash\H}
\newcommand{\GG}{\G_{\infty}\backslash\G}

\newenvironment{psmallmatrix}
  {\left(\begin{smallmatrix}}
  {\end{smallmatrix}\right)}

%============%
%  Comments  %
%============%
\newcommand{\todo}[1]{\textcolor{red}{\sf Todo: [#1]}}

%===================%
%  Label reminders  %
%===================%
% [label=(\roman*)]
% [label=(\alph*)]
% [label=(\arabic{enumi})]

%==================%
%  Other settings  %
%==================%
\pgfdeclarelayer{background}
\pgfsetlayers{background,main}
\tikzset{->-/.style={decoration={
  markings,
  mark=at position .5 with {\arrow{>}}},postaction={decorate}}}

%=================%
%  Title & Index  %
%=================%
\title{Weyl group multiple Dirichlet series over function fields}
\author{Henry Twiss}
\date{2024}
\makeindex

\begin{document}

\begin{abstract}
    We construct a multiple Dirichlet series $Z(s_{1},\ldots,s_{r})$ in $r$ variables over a function field $\F_{q}(t)$ that is natrually associated to a simply laced rank $r$ root system $\Phi$ using the Chinta-Gunnells construction. This is the simplest example of an $r$ variable Weyl goup multiple Dirichlet series over a global field and is a derivative of \cite{chinta2007weyl} in a less general setting with slightly more detail. 
\end{abstract}

\maketitle

\section{Preliminaries}
    \subsection*{Function Fields}
        We present an overview of the zeta function and Dirichlet $L$-functions attached to $\F_{q}(t)$. For a detailed discussion see \cite{rosen2002number}. Let $q$ be a power of an odd prime and let $\F_{q}[t]$ be the polynomial ring in $t$ with coefficients in the finite field $\F_{q}$. This is a principal ideal domain. Moreover, the nonzero prime ideals in $\F_{q}[t]$ are generated by irreducible polynomials. Let $\F_{q}(t)$ denote the quotient field. Define the norm function $N(m)$ by
        \[
            N(m) = |m| = q^{\deg(m)},
        \]
        for any $m \in \F_{q}[t]$. The zeta function $\z(s)$ on $\F_{q}[t]$ is defined as the Dirichlet series or Euler product
        \[
            \z(s) = \sum_{\text{$m$ monic}}\frac{1}{|m|^{s}} = \prod_{\text{$P$ monic irr}}\left(1-\frac{1}{|P|^{s}}\right)^{-1},
        \]
        where the second equality holds since $\F_{q}[t]$ is a unique factorization domain. As for questions of convergence, there are $q^{n}$ monic polynomials of degree $n$ so, provided $\Re(s) > 1$, we can sum up the Dirichlet series according to degree and obtain an explicit expression:
        \[
            \z(s) = \sum_{n \ge 0}\frac{\text{\# of monic poly of deg $n$}}{q^{ns}} = \sum_{n \ge 1}\frac{1}{q^{n(1-s)}} = \frac{1}{1-q^{1-s}}.
        \]
        The latter expression is meromorphic on $\C$ with a simple pole at $s = 1$ of residue $\frac{1}{\log(q)}$. Therefore $\z(s)$ admits meromorphic continuation to $\C$. The zeta function also satisfies a functional equation. Define the completed zeta function (this is also the zeta function attached to $\F_{q}(t)$) by
        \[
            \z^{\ast}(s) = \frac{1}{1-q^{-s}}\z(s).
        \]
        Then
        \[
            \z^{\ast}(s) = q^{2s-1}\z^{\ast}(1-s).
        \]
        Recall that characters on $\F_{q}[t]$ are multiplicative functions to the complex numbers. The two flavors we care about are:
        
        \begin{itemize}
            \item Dirichlet characters: multiplicative functions $\chi_{d}:\F_{q}[t] \to \C$ modulo $d \in \F_{q}[t]$ (in that they are $d$-periodic) and such that $\chi_{d}(m) = 0$ if $(d,m) > 1$.
            \item Hilbert symbols: Dirichlet characters modulo $1$.
        \end{itemize}
        
        In either case, the image always lands in the roots of unity. The Dirichlet characters that are of interest to us are those given by the quadratic residue symbol on $\F_{q}[t]$. If $d \in \F_{q}[t]$ is a monic non-constant irreducible, define the quadratic character $\chi_{d}$ by the following quadratic residue symbol:
        \[
            \chi_{d}(m) = \legendre{d}{m} = m^{\frac{|d|-1}{2}} \pmod{d},
        \]
        for any $m \in \F_{q}[t]$. Then $\chi_{d}(m) \in \{\pm 1\}$ provided $d$ and $m$ are relatively prime and $\chi_{d}(m) = 0$ if $(d,m) > 1$. Recall that if $b \in \F^{\x}$, then $\sgn(b) = \pm1$ depending on if $b \in (\F^{\x})^{2}$ or not. Moreover, for $d \in \F_{q}[t]$ we have $\sgn(d) = \sgn(b_{n})$ if $d(t) = b_{n}t^{n}+b_{n-1}t^{n+1}+\cdots+b_{0}$ (with $b_{n} \neq 0$). For $b \in \F^{\x}$, we define the quadratic character $\chi_{b}$ by the following quadratic residue symbol:
        \[
            \chi_{b}(m) = \legendre{b}{m} = \sgn(b)^{\deg(m)}.
        \]
        Extending $\chi_{d}$ multiplicativity in $d$, $\chi_{d}$ is defined for any $d$ not necessarily non-constant, monic, or irreducible. The quadratic residue symbol also has the following reciprocity property:

        \begin{theorem}[Quadratic reciprocity]
            If $d,m \in \F_{q}[t]$ are monic, square-free, and relatively prime, then
            \[
                \legendre{d}{m} = (-1)^{\frac{q-1}{2}\deg(d)\deg(m)}\legendre{m}{d}.
            \]
        \end{theorem}

        Note that if $q \equiv 1 \tmod{4}$, the sign in the statement of quadratic reciprocity is always $1$ so that reciprocity is perfect. We now describe the Hilbert symbols on $\F_{q}[t]$. In fact, there is only one nontrivial character $\psi$ defined by
        \[
            \psi(d) = (-1)^{\deg(d)}.
        \]
        The other Hilbert symbol is the trivial character $\psi^{2} = 1$. To see that $\psi$ is given by a quadratic character, just notice that for $\t \in \F^{\x}-(\F^{\x})^{2}$ we have $\chi_{\t}(d) = (-1)^{\deg(d)}$. Moreover, the trivial character is then given by $\chi_{1}$.

        We are now ready to discuss the $L$-functions associated to quadratic characters. We define the $L$-function $L(s,\chi_{d})$ attached to $\chi_{d}$ by a Dirichlet series or Euler product:
        \[
            L(s,\chi_{d}) = \sum_{\text{$m$ monic}}\frac{\chi_{d}(m)}{|m|^{s}} = \prod_{\text{$P$ monic irr}}\left(1-\frac{\chi_{d}(P)}{|P|^{s}}\right)^{-1}.
        \]
        By definition of the quadratic character, $L(s,\chi_{d}) \ll \z(s)$ for $\Re(s) > 1$ so that $L(s,\chi_{d})$ is locally absolutely uniformly convergent in this region. $L(s,\chi_{d})$ also admits meromorphic continuation to $\C$ with a simple pole at $s = 1$ if $d$ is a perfect square and is analytic otherwise (see \cite{rosen2002number} for a proof). The completed $L$-function is defined as follows:
        \[
            L^{\ast}(s,\chi_{d}) = \begin{cases} \frac{1}{1-q^{-s}}L(s,\chi_{d}) & \text{if $\deg(d)$ is even}, \\ L(s,\chi_{d}) & \text{if $\deg(d)$ is odd}, \end{cases}
        \]
        and satisfies the functional equation
        \[
            L^{\ast}(s,\chi_{d}) = \begin{cases} q^{2s-1}|d|^{\frac{1}{2}-s}L^{\ast}(1-s,\chi_{d}) & \text{if $\deg(d)$ is even}, \\ q^{2s-1}(q|d|)^{\frac{1}{2}-s}L^{\ast}(1-s,\chi_{d}) & \text{if $\deg(d)$ is odd}. \end{cases}
        \]
        Note that in the case $\deg(d)$ is even, the conductor is $|d|$ and in the case $\deg(d)$ is odd, the conductor is $q|d|$. In other words, the gamma factors depend upon the degree of $d$.
    \subsection*{Root Systems}
        Throughout let $V$ be an $r$-dimensional Euclidean vector space with standard inner product $(\cdot,\cdot)$. For any nonzero $v \in V$, let
        \[
            H_{v} = \{u \in V:(v,u) = 0\},
        \]
        be the hyperplane perpendicular to $v$. Accordingly, let
        \[
            s_{v}(w) = w-2\frac{(v,w)}{(v,v)}v, 
        \]
        be the reflection through the hyperplane $H_{v}$. Lastly, we will define an operator $\<\cdot,\cdot\>:V \x V \to \R$ by
        \[
            \<w,v\> = 2\frac{(v,w)}{(v,v)},
        \]
        Note that $\<\cdot,\cdot\>$ is not an inner product as it need not be symmetric and is linear only in the first argument. However, we have the simplified formula
        \[
            s_{v}(w) = w-\<w,v\>v.
        \]
        Recall that a root system $\Phi$ in $V$, whose elements $\a \in \Phi$ are called roots, is a finite set of nonzero vectors that satisfty the following conditions:

        \begin{enumerate}[label=(\roman*)]
            \item $\Phi$ is a spanning set for $V$.
            \item For any root $\a \in \Phi$, then the only scalar multiples of $\a$ that belong to $\Phi$ are $\a$ itself and $-\a$.
            \item For every root $\a \in \Phi$, $\Phi$ is closed under the reflection $s_{\a}$ through the hyperplane $H_{\a}$ perpendicular to $\a$.
            \item If $\a,\b \in \Phi$ are roots, then the projection of $\b$ onto the line through $\a$ is an integer or half-integer multiple of $\a$.
        \end{enumerate}

        The last two conditions can be equivalently expressed in more algebraic forms:

        \begin{enumerate}[label=(\roman*)]
            \setcounter{enumi}{2}
            \item For any two roots $\a,\b \in \Phi$, $\Phi$ contains the element
            \[
                s_{\a}(\b) = \b-2\frac{(\a,\b)}{(\a,\a)}\a = \b-\<\b,\a\>\a.
            \]
            \item If $\a,\b \in \Phi$ are roots, then the number $\<\b,\a\> = 2\frac{(\a,\b)}{(\a,\a)}$ is an integer.
        \end{enumerate}
        
        We suppose that our root system $\Phi$ is irreducible and simply laced. In other words, no root $\a$ is orthogonal to all other roots other than $-\a$. Letting $\D = \{\a_{1},\ldots,\a_{r}\}$ be a set of simple roots for $\Phi$, $\Phi$ admits the decomposition
        \[
            \Phi = \Phi_{+} \cup \Phi_{-},
        \]
        into positive and negative roots where every positive root is a nonnegative linear combination of simple roots with integer coefficients and $\Phi_{-} = -\Phi_{+}$. We denote the Weyl group associated to $\Phi$ by $W$. Then $W$ is generated by the simple reflections $\s_{i} = \s_{\a_{i}}$:
        \[
            W = \<\s_{i}:1 \le i \le r\>.
        \]
        For any base $\D$, the fundamental Weyl chamber $\mc{C}(\D)$ is
        \[
            \mc{C}(\D) = \{v \in V:(v,\a) > 0 \text{ for all } \a \in \D\}.
        \]
        The fundamental Weyl chamber is a connected component of $V-\bigcup_{\a \in \Phi}H_{\a}$ and $W$ acts simply transitively on the Weyl chambers. Since $\Phi$ is simply laced, the Dynkin diagram of $\Phi$ is the graph with nodes $i$ for $1 \le i \le r$ where the nodes $i$ and $j$ are adjacent, via a single edge, if and only if $(\s_{i}\s_{j})^{3} = 1$. If $i$ and $j$ are not adjacent then $(\s_{i}\s_{j})^{2} = 1$. We write $i \sim j$ if $i$ and $j$ are adjacent in the Dynkin diagram. The action of the simple reflection $\s_{i}$ on the simple root $\a_{j}$ is given by the following:
        \[
            \s_{i}\a_{j} = \begin{cases} \a_{i}+\a_{j} & \text{if $i \sim j$}, \\ -\a_{j} & \text{if $i = j$}, \\ \a_{j} & \text{otherwise}. \end{cases}
        \]
        Letting $r(i,j)$ be the order of $\s_{i}\s_{j}$ we have
        \[
            r(i,j) = \begin{cases} 3 & \text{if $i \sim j$}, \\ 1 & \text{if $i = j$}, \\ 2 & \text{otherwise}. \end{cases}
        \]
        Equivalently, we have the relations
        \begin{align*}
            \s_{i}^{2} = 1 \qquad &\text{for all $i$}, \\
            \s_{i}\s_{j}\s_{i} = \s_{j}\s_{i}\s_{j} \qquad &\text{if $i \sim j$}, \\
            \s_{i}\s_{j} = \s_{j}\s_{i} \qquad &\text{otherwise},
        \end{align*}
        These relations give a presentation for the Weyl group:
        \[
            W = \left\< \s_{i} \text{ for } 1 \le i \le r: \substack{\s_{i}^{2} = 1 \text{ for all $i$}, \\ \s_{i}\s_{j}\s_{i} = \s_{j}\s_{i}\s_{j} \text{ if $i \sim j$}, \\ \s_{i}\s_{j} = \s_{j}\s_{i} \text{ otherwise}.} \right\>
        \]
        Since $\D$ is a basis for $\Phi$, every root $\a \in \Phi$ admits the unique expression
        \[
            \a = \sum_{1 \le i \le r}k_{i}\a_{i},
        \]
        where all of the $k_{i}$ integers and either all $k_{i} \ge 0$ or all $k_{i} \le 0$. Accordingly, we define $\Supp(\a)$ to be the subset of $1 \le i \le r$ such that $k_{i} \neq 0$. We also define the height $h(\a)$ of $\a$ by
        \[
            h(\a) = \sum_{1 \le i \le }k_{i}.
        \]
        This induces a partial ordering $<$ on the roots where $\a \le \b$ if either $\a = \b$ or $\b-\a$ is a nonnegative combination of simple roots. There is a unique highest root with respect to this ordering. More generally, let $\L_{\Phi}$ be the lattice generated by the roots. Then
        \[
            \L_{\Phi} = \Z\a_{1} \op \Z\a_{2} \op \cdots \op \Z\a_{r},
        \]
        and every $\a \in \L_{\Phi}$ admits a unique expression
        \[
            \a = \sum_{1 \le i \le r}k_{i}\a_{i},
        \]
        with the $k_{i} \in \Z$. Note that $\a$ need not be a root so the $k_{i}$ can have mixed sign. Clearly $\Phi \subset \L_{\Phi}$. Clearly, height, support, and the ordering $<$ on $\Phi$ all extend to $\L_{\Phi}$ in the obvious way. Also, recall that the Weyl vector $\rho$ is given by
        \[
            \rho = \frac{1}{2}\sum_{\a \in \Phi^{+}}\a.
        \]
        Note that the Weyl vector is not a root. Nevertheless, $\rho-w\rho$ is a root for all $w \in W$. Lastly, for $w \in W$ set
        \[
            \Phi(w) = \{\a \in \Phi^{+}:w\a \in \Phi^{-}\},
        \]
        to be the set of positive roots sent to negative roots by $w$. Recall that $\s_{i}$ permutes all of the positive roots except $\a_{i}$ which is sent to its negative. So $\Phi(\s_{i}) = \{\a_{i}\}$ for all $1 \le i \le r$. Let $\ell:W \to \Z_{\ge 0}$ denote the length function. Then $\ell(w)$ is number of simple reflections in the reduced expression for $w \in W$. We define
        \[
            \sgn(w) = (-1)^{\ell(w)}.
        \]
\section{The Chinta-Gunnells Construction}
    The Chinta-Gunnells construction is a way of building a Weyl group multiple Dirichlet series that is more multiplicative in nature. This construction is in contrast to building these objects using correction polynomials which is naturally additive. More precisely, the coefficients of a Weyl group multiple Dirichlet series are not multiplicative but only just. These coefficients satisfy a twisted version of multiplicativity instead. One might hope that the subseries corresponding to powers of a fixed prime, or $p$-th parts as they are called, contain enough structure to build the multiple Dirichlet series. This is indeed the case. The Chinta-Gunnells construction is a way of building the $p$-th parts of a multiple Dirichlet series by averaging over the elements of a Weyl group $W$ attached to some root system $\Phi$. More precisely, we define an action of the Weyl group $W$ on rational functions $f \in \C(\mathbf{x},u)$ where $\mathbf{x} = (x_{1},\ldots,x_{r})$ and $u$ is a parameter. From this action, we will construct a rational function $Z_{\Phi}(\mathbf{x};q) \in \C(\mathbf{x},u)$ that is $W$-invariant and satisfies certain limiting properties. Setting $u = q$ and expanding $Z_{\Phi}(\mathbf{x};q)$ as a power series in the $x_{i}$, the $W$-invariance will force the coefficients of this power series to satisfy certain functional equations. We will use these coefficients to build the associated global Weyl group multiple Dirichlet series $Z(\mathbf{s}) = Z(s_{1},\ldots,s_{r})$ over the rational function field $\F_{q}(t)$. In fact, under a change of variables, $Z_{\Phi}(\mathbf{x};q)$ equals the $p$-th part of a Weyl group multiple Dirichlet series over $\F_{q}(t)$. So from the $W$-invariance, we see that the $p$-th parts satisfty a group of functional equations that is naturally isomorphic to $W$ just like the global Weyl group multiple Dirichlet series. However, a much more beautiful phenomena occurs. Under a simple change of variables $(\mathbf{x},u) = (x_{1},\ldots,x_{r},u) \to (q^{1-s_{1}},\ldots,q^{1-s_{r}},q^{-1})$, the $W$-invariant rational function $Z_{\Phi}(\mathbf{x};u)$ used to contruct the $p$-th parts of $Z(\mathbf{x})$ actually equals the global Weyl group multiple Dirichlet series $Z(\mathbf{s})$. That is,
    \[
        Z_{\Phi}(q^{1-s_{1}},\ldots,q^{1-s_{r}};q^{-1}) = Z(s_{1},\ldots,s_{r}).
    \]
    Throughout we assume $\Phi$ is an irreducible and simply laced root system.
    \subsection*{The Chinta-Gunnells Action}
        Let $\a_{1},\ldots,\a_{r}$ be simple roots for $\Phi$. Let $\C(\mathbf{x},u)$ be the field of rational functions in the variables $\mathbf{x} = (x_{1},\ldots,x_{r})$ and formal parameter $u$. Moreover, let $\mathbf{x}_{i} = x_{i}$ for $1 \le i \le r$. For $\a \in \L_{\Phi}$ in the root lattice, set $\mathbf{x}^{\a} = x_{1}^{k_{1}} \cdots x_{r}^{k_{r}}$ if $\a = \sum_{1 \le i \le r}k_{i}\a_{i}$. We first define an action of simple reflection $\s_{i}$ on $r$-tuples $\mathbf{x} = (x_{1},\ldots,x_{r})$ component-wise by
        \[
            (\s_{i}\mathbf{x})_{j} = \begin{cases} \sqrt{u}x_{i}x_{j} & \text{if $i \sim j$}, \\ \frac{1}{ux_{j}} & \text{if $i = j$}, \\ x_{j} & \text{otherwise}. \end{cases}
        \]
        It is easy to verify directly that action of simple reflections extends to a $W$-action on $\C^{r}$. That is, we have the follow relations:
        \begin{equation}\label{equ:relations_for_action_of_simple_reflections_on_tuples}
            \begin{aligned}
                \s_{i}^{2}\mathbf{x} = \mathbf{x} \qquad &\text{for all $i$}, \\
                \s_{i}\s_{j}\s_{i}\mathbf{x} = \s_{j}\s_{i}\s_{j}\mathbf{x} \qquad &\text{if $i \sim j$}, \\
                \s_{i}\s_{j}\mathbf{x} = \s_{j}\s_{i}\mathbf{x} \qquad &\text{otherwise}.
            \end{aligned}
        \end{equation}
        One can also prove the useful relation
        \begin{equation}\label{equ:useful_relation}
            (w\mathbf{x})^{\a} = u^{\frac{h(w\a-\a)}{2}}\mathbf{x}^{w\a},
        \end{equation}
        which follows by induction on the length of $w$. Second, we define sign operators $\e_{i}$ on $r$-tuples $\mathbf{x}$. For $1 \le i \le r$, define $\e_{i}\mathbf{x}$ component-wise by
        \[
            (\e_{i}\mathbf{x})_{j} = \begin{cases} -x_{j} & \text{if $i \sim j$}, \\ x_{j} & \text{otherwise}. \end{cases}
        \]
        The following relations are also easily verified for all $i$ and $j$:
        \begin{equation}\label{equ:relations_for_action_of_signs_on_tuples}
            \begin{aligned}
                \e_{i}^{2}\mathbf{x} &= \mathbf{x}, \\
                \e_{i}\e_{j}\mathbf{x} &= \e_{j}\e_{i}\mathbf{x}, \\
                \s_{i}\e_{j}\mathbf{x} &= \begin{cases} \e_{i}\e_{j}\s_{i}\mathbf{x} & \text{if $i \sim j$}, \\ \e_{j}\s_{i}\mathbf{x} & \text{otherwise}. \end{cases}
            \end{aligned}
        \end{equation}
        Given $f \in \C(\mathbf{x},u)$, we will need to decompose $f$ with respect to $\e_{i}$. Accordingly, set
        \[
            f_{i}^{\pm}(\mathbf{x};u) = \frac{f(\mathbf{x};u) \pm f(\e_{i}\mathbf{x};u)}{2}.
        \]
        These are the even and odd parts of $f$ with respect to the involution $\e_{i}$. We state some properties of this operation that will be useful:
        
        \begin{proposition}\label{prop:pm_properties}
            Let $f,g \in \C(\mathbf{x},u)$ and $1 \le i \le r$. Then the following properties are true:
            \begin{enumerate}[label=(\roman*)]
                \item Taking the even or odd part with respect to $\e_{i}$ is additive. That is,
                \[
                    (f+g)_{i}^{\pm}(\mathbf{x};u) = f_{i}^{\pm}(\mathbf{x};u)+g_{i}^{\pm}(\mathbf{x};u)
                \]
                \item If $f$ is a function of $x_{i}$ and $u$ alone, then
                \[
                    (fg)_{i}^{\pm}(\mathbf{x};u) = f(x_{i};u)g_{i}^{\pm}(\mathbf{x};u).
                \]
                \item $f_{i}^{\pm}(\mathbf{x};u)$ decompose $f(\mathbf{x};u)$. That is,
                \[
                    f_{i}(\mathbf{x};u) = f_{i}^{+}(\mathbf{x};u)+f_{i}^{-}(\mathbf{x};u)
                \]
                \item 
                \[
                    f_{ii}^{\pm\pm}(\mathbf{x};u) = f_{i}^{\pm}(\mathbf{x};u) \quad \text{and} \quad f_{ii}^{\pm\mp}(\mathbf{x};u) = 0.
                \]
            \end{enumerate}
        \end{proposition}
        \begin{proof}
            Properties (i) and (iii) are clear. Property (ii) follows since $\e_{i}$ does not change the sign of $x_{i}$. As for (iv), this can be verified by direct computation.
        \end{proof}

        We can now define a $W$-action $\C(\mathbf{x},u)$. For any simple reflection $\s_{i}$ and $f \in \C(\mathbf{x},u)$, we set
        \[
            f|^{\mathrm{CG}}\s_{i}(\mathbf{x};u) = -\frac{1-ux_{i}}{ux_{i}(1-x_{i})}f_{i}^{+}(\s_{i}\mathbf{x};u)+\frac{1}{\sqrt{u}x_{i}}f_{i}^{-}(\s_{i}\mathbf{x};u).
        \]
        While we will not need it, there is an equivalent way to define this action that is sometimes used. For $f \in \C(\mathbf{x},u)$, we have
        \[
            f|^{\mathrm{CG}}\s_{i}(\mathbf{x};u) = f(\s_{i}\mathbf{x};u)J(x_{i};u,0)+f(\e_{i}\s_{i}\mathbf{x};u)J(x_{i};u,1),
        \]
        where, for $\d \in \{0,1\}$, we set
        \[
            J(x;u,\d) = \frac{1}{2}\left(-\frac{1-ux}{ux(1-x)}+\frac{(-1)^{\d}}{\sqrt{u}x}\right).
        \]
        In any case, the action $|^{\mathrm{CG}}\s_{i}$ of simple reflections $\s_{i}$ on functions $f \in \C(\mathbf{x},u)$ extends to a $W$-action on $\C(\mathbf{x},u)$:

        \begin{proposition}
        The action of simple reflections $\s_{i}$ on $\C(\mathbf{x},u)$ extends to a $W$-action.
        \end{proposition}
        \begin{proof}
            See \cite{chinta2007weyl} for a proof.
        \end{proof}
 
        We call the extended $W$-action $|^{\mathrm{CG}}w$ the \textbf{Chinta-Gunnells action}. We now state some basic properties of this action:

        \begin{proposition}\label{prop:CG_properties}
            Let $f,g \in \C(\mathbf{x},u)$ and $w \in W$. Then the following are true:
            \begin{enumerate}[label=(\roman*)]
                \item The Chinta-Gunnells action is additive. That is,
                \[
                    (f+g)|^{\mathrm{CG}}w(\mathbf{x};u) = f|^{\mathrm{CG}}w(\mathbf{x};u)+g|^{\mathrm{CG}}w(\mathbf{x};u).
                \]
                \item If $f$ is an even function in all of the $x_{i}$, then
                \[
                    fg|^{\mathrm{CG}}w(\mathbf{x};u) = f(w\mathbf{x};u) \cdot g|^{\mathrm{CG}}w(\mathbf{x};u).
                \]
            \end{enumerate}
        \end{proposition}
        \begin{proof}
            Both (i) and (ii) can be proved for a simple reflection $\s_{i}$ and then in general by induction on the length of $w$. See \cite{chinta2008parts} for a full proof.
        \end{proof}
    \subsection*{The Chinta-Gunnells Average}
        We will now construct the desired $W$-invariant function. To begin, define products
        \[
            \D_{\Phi}(\mathbf{x}) = \prod_{\a \in \Phi^{+}}(1-u^{h(\a)}\mathbf{x}^{2\a}) \quad \text{and} \quad  D_{\Phi}(\mathbf{x}) = \prod_{\a \in \Phi^{+}}(1-u^{h(\a)-1}\mathbf{x}^{2\a}).
        \]
        Also set
        \[
            j(w,\mathbf{x}) = \frac{\D_{\Phi}(\mathbf{x})}{\D_{\Phi}(w\mathbf{x})}.
        \]
        Then $j(w,\mathbf{x})$ immeditely satisfies the $1$-cocycle relation
        \[
            j(ww',\mathbf{x}) = j(w,w'\mathbf{x})j(w',\mathbf{x}),
        \]
        for any $w,w' \in W$. We will need a useful lemme telling us how to compute $j(w,\mathbf{x})$ in general:

        \begin{lemma}\label{lem:cocycle_computation}
            For any simple reflection $\s_{i}$,
            \[
                j(\s_{i},\mathbf{x}) = -ux_{i}^{2}.
            \]
            Moreover, for any $w \in W$,
            \[
                j(w,\mathbf{x}) = \sgn(w)u^{h(\rho-w^{-1}\rho)}\mathbf{x}^{2(\rho-w^{-1}\rho)}.
            \]
        \end{lemma}
        \begin{proof}
            The second statement follows from the firt and the cocycle relation for $j(w,\mathbf{x})$. As for the first statement, write
            \[
                \D_{\Phi}(\mathbf{x}) = (1-u\mathbf{x}^{2\a_{i}})\prod_{\substack{\a \in \Phi^{+} \\ \a \neq \a_{i}}}(1-u^{h(\a)}\mathbf{x}^{2\a})
            \]
            Now $\s_{i}$ permutes all of the positive roots except for $\a_{i}$ which it sends to its negative (that is $\Phi(\s_{i}) = \{\a_{i}\}$). Using \cref{equ:useful_relation} and that $\s_{i}\a_{i} = -\a_{i}$ gives
            \[
                \D_{\Phi}(\s_{i}\mathbf{x}) = (1-u(\s_{i}\mathbf{x})^{2\a_{i}})\prod_{\substack{\a \in \Phi^{+} \\ \a \neq \a_{i}}}(1-u^{h(\a)}(\s_{i}\mathbf{x})^{2\a}) = (1-u^{-1}\mathbf{x}^{-2\a_{i}})\prod_{\substack{\a \in \Phi^{+} \\ \a \neq \a_{i}}}(1-u^{h(\s_{i}\a)}\mathbf{x}^{2\s_{i}\a}).
            \]
            But since $\s_{i}$ permutes all of the positive roots except $\a_{i}$, the two products over $\a \in \Phi^{+}$ with $\a \neq \a_{i}$ are identical. Then the identity $-ux_{i}^{2}(1-u^{-1}\mathbf{x}^{-2\a_{i}}) = (1-u\mathbf{x}^{2\a_{i}})$ implies 
            \[
                \D_{\Phi}(\s_{i}\mathbf{x}) = -u^{-1}x_{i}^{-2}\D_{\Phi}(\mathbf{x}),
            \]
            which is to say that
            \[
                j(\s_{i},\mathbf{x}) = -ux_{i}^{2}.
            \]
        \end{proof}

        Notice that by \cref{lem:cocycle_computation}, $j(w,\mathbf{x})$ is an even functions of all the $x_{i}$. So is $\D_{\Phi}(\mathbf{x})$, so (ii) of \cref{prop:CG_properties} applies to both of these functions. Now define the \textbf{Chinta-Gunnells average} $Z_{\Phi}(\mathbf{x};u)$ by
        \[
            Z_{\Phi}(\mathbf{x};u) = \frac{Z_{W}(\mathbf{x};u)}{\D_{\Phi}(\mathbf{x})},
        \]
        where
        \[
            Z_{W}(\mathbf{x};u) = \sum_{w \in W}j(w,\mathbf{x})(1|^{\mathrm{CG}}w)(\mathbf{x};u).
        \]
        It turns out that $Z_{\Phi}(\mathbf{x};u)$ is $W$-invariant under the Chinta-Gunnells action and satisfies certain limiting properties (the limiting properties are the more difficut part to verify):

        \begin{theorem}\label{thm:invariant_function}
            $Z_{\Phi}(\mathbf{x};u)$ is a rational function in $\C(\mathbf{x},u)$ that is $W$-invariant with respect to the Chinta-Gunnells action and satisfies the following properties:
            \begin{enumerate}[label=(\roman*)]
                \item Let $1 \le i \le r$. If $\mathbf{x} = (x_{1},\ldots,x_{r})$ is such that $x_{j} = 0$ provided $i \sim j$, then $(1-x_{i})Z_{\Phi}(\mathbf{x};u)$ is independent of $x_{i}$.
                \item $Z_{\Phi}(\mathbf{0};u) = 1$.
            \end{enumerate}
        \end{theorem}
        \begin{proof}
            The fact that $Z_{\Phi}(\mathbf{x};u)$ is a rational function is clear from the definition of the Chinta-Gunnells action. The $W$-invariance follows from \cref{prop:CG_properties} and \cref{lem:cocycle_computation} combined with the $1$-cocycle relation for $j(w,\mathbf{x})$. For property (ii), see \cite{chinta2008parts} for a proof.
        \end{proof}

        It was also proven in \cite{chinta2008parts} that $Z_{\Phi}(\mathbf{x};u)$ can be written as
        \[
            Z_{\Phi}(\mathbf{x};u) = \frac{N_{\Phi}(\mathbf{x};u)}{D_{\Phi}(\mathbf{x};u)},
        \]
        for some polynomial $N_{\Phi}(\mathbf{x};u)$. Actually, this implies that $Z_{\Phi}(\mathbf{x};u)$ is locally absolutely uniformly convergent away from the points $1-u^{h(\a)-1}\mathbf{x}^{2\a} = 0$ for all $\a \in \Phi^{+}$. In particular, we have such convergence for $|\mathbf{x}|$ sufficiently small provided $u$ is fixed. For our purposes, it will be convienent to use this expression for $Z_{\Phi}(\mathbf{x};u)$ since $D_{\Phi}(\mathbf{x};u)$ is an explicit description for the polar structure of $Z_{\Phi}(\mathbf{x};u)$.
        
        \begin{remark}
            Since $W$ is finite, it is very easy to construct $W$-invariant rational functions in general by choosing any rational function $g$ and averaging over $g|^{\mathrm{CG}}w$ for $w \in W$. This is why property (i) in \cref{thm:invariant_function} is essential. Indeed, it is the only condition that carries information about the combinatorics of the root system $\Phi$ because it depends upon the associated Dynkin diagram. Actually, from property (i), one can reconstruct the Dynkin diagram of $\Phi$ by inspecting which variables the functions $(1-x_{i})Z_{\Phi}(\mathbf{x};u)$ are independent of. Since the Dynkin diagram is a unique graphic representation of a root system, property (i) encodes the entire root system $\Phi$ algebraically into $Z_{\Phi}(\mathbf{x};u)$.
        \end{remark}
\section{Properties of The Chinta-Gunnells Average}
    From now on we take $u$ to be positive. We will prove some basic properties of the Chinta-Gunnells average $Z_{\Phi}(\mathbf{x};u)$. Expanding $Z_{\Phi}(\mathbf{x};u)$ as a power series in $x_{1},\ldots,x_{r}$ yields
    \[
        Z_{\Phi}(\mathbf{x};u) = \sum_{k_{1},\ldots,k_{r} \ge 0}a(k_{1},\ldots,k_{r};u)x_{1}^{k_{1}} \cdots x_{r}^{k_{r}},
    \]
    for some coefficients $a(k_{1},\ldots,k_{r};u)$. We can specify some of the coefficients immeditely by using \cref{thm:invariant_function}. Indeed, since $Z_{\Phi}(\mathbf{0};u) = 1$ by property (ii) of \cref{thm:invariant_function}, this forces the constant term in the power series expansion to be $1$. So,
    \[
        a(0,\ldots,0;u) = 1.
    \]
    Actually, since every factor of $D(\mathbf{x};u)$ is of the form $(1-u^{2}\mathbf{x}^{2\a})$ for some $\a \in \Phi^{+}$ we see that the constant term of $D(\mathbf{x};u)$ is $1$ and so constant term of $N(\mathbf{x};u)$ must also be $1$. Moreover, taking $\mathbf{x}$ as in property (i) of \cref{thm:invariant_function}, we see that $(1-x_{i})$ divides $Z_{\Phi}(\mathbf{x};u)$ and is the only part of $Z_{\Phi}(\mathbf{x};u)$ depending upon $x_{i}$ (for such $\mathbf{x}$). As the power series coefficients are independent of $\mathbf{x}$, this forces
    \[
        a(k_{1},\ldots,k_{r};u) = 1,
    \]
    provided $k_{j} = 0$ for all $i \sim j$. In particular,
    \[
        a(0,\ldots,k_{i},\ldots,0;u) = 1,
    \]
    for all $k_{i} \ge 0$ and $1 \le i \le r$. We will now discuss the size of the coefficients in general. It will also be convienent to set $|k| = \sum_{1 \le i \le r}k_{i}$. Our first property is that for a fixed root system $\Phi$, these coefficients are polynomially bounded in $u$.

    \begin{proposition}\label{prop:CG_coefficients_polynomial_bound}
        For a fixed $\Phi$, there exist constants $c_{1},c_{2} > 0$ such that
        \[
            |a(k_{1},\ldots,k_{r};u)| < c_{1}u^{c_{2}|k|}.
        \]
    \end{proposition}
    \begin{proof}
        Since $N_{\Phi}(\mathbf{x};u)$ and $D_{\Phi}(\mathbf{x};u)$ are both polynomially bounded in $u$, the claim follows.
    \end{proof}

    \cref{prop:CG_coefficients_polynomial_bound} will guarentee convergence of $Z_{\Phi}(\mathbf{x};u)$ provided the real parts of the $x_{i}$ are all sufficiently small. Our second property of $Z_{\Phi}(\mathbf{x};u)$ describes functional equations for coefficients in various power series expansions of $Z_{\Phi}(\mathbf{x};u)$. Roughly speaking we make take the power series expasion of $Z_{\Phi}(\mathbf{x};u)$ in all of the $x_{i}$ save for $x_{j}$. Then the coefficients of this power series are functions of $x_{j}$. The invariance of $Z_{\Phi}(\mathbf{x};u)$ under $\s_{j}$ will force these coefficients to satisfy certain functional equations. To set up some notation, for any index $1 \le j \le r$ and $r$-tuples $k = (k_{1},\ldots,k_{r})$, set
    \[
        \what{k} = (k_{1},\ldots,k_{j-1},k_{j+1},\ldots,k_{r}),
    \]
    and let $|\what{k}| = \sum_{i \neq j}k_{i}$. For fixed index $j$ and an $(r-1)$-tuple $\what{k}$, define
    \[
        T(x_{j};\what{k},u) = \sum_{k_{j} \ge 0}a(k_{1},\ldots,k_{r};u)x_{j}^{k_{j}},
    \]
    and let
    \[
        n(\what{k}) = \sum_{\text{$i \sim j$}}k_{i}.
    \]
    Then we have the following proposition:

    \begin{proposition}\label{prop:T_functional_equations}
        Fix an index $1 \le j \le r$ and an $(r-1)$-tuple $\what{k}$. Then the following are true:
        \begin{enumerate}[label=(\roman*)]
            \item If $n(\what{k})$ is even, $T(x_{j};\what{k},u)$ satisfies the functional equation
            \[
                (1-x_{j})T(x_{j};\what{k},u) = \left(1-\frac{1}{ux_{j}}\right)(\sqrt{u}x_{j})^{n(\what{k})}T\left(\frac{1}{ux_{j}};\what{k},u\right).
            \]
            \item If $n(\what{k})$ is odd, $T(x_{j};\what{k},u)$ satisfies the functional equation
            \[
                T(x_{j};\what{k},u) = (\sqrt{u}x_{j})^{n(\what{k})-1}T\left(\frac{1}{ux_{j}};\what{k},u\right).
            \]
            \item If $|x_{j}| < u^{-c_{2}}$, then
            \[
                |T(x_{j};\what{k},u)| \ll c_{1}u^{c_{2}|\what{k}|}.
            \]
        \end{enumerate}
    \end{proposition}
    \begin{proof}
        We first prove statement (i). So suppose $n(\what{k})$ is even. Acting by $\s_{j}$ on $Z_{\Phi}(\mathbf{x};u)$, the $W$-invariance implies
        \begin{equation}\label{equ:T_functional_equations_1}
            Z_{\Phi}(\mathbf{x};u) = -\frac{1-ux_{j}}{ux_{j}(1-x_{j})}Z_{\Phi,j}^{+}(\s_{j}\mathbf{x};u)+\frac{1}{\sqrt{u}x_{j}}Z_{\Phi,j}^{-}(\s_{j}\mathbf{x};u).
        \end{equation}
        Taking the $Z_{\Phi,j}^{+}(\mathbf{x};u)$ part of both sides, using all \cref{prop:pm_properties} properties, and then multiplying by $(1-x_{j})$, yields
        \begin{equation}\label{equ:T_functional_equations_2}
            (1-x_{j})Z_{\Phi,j}^{+}(\mathbf{x};u) = \left(1-\frac{1}{ux_{j}}\right)Z_{\Phi,j}^{+}(\s_{j}\mathbf{x};u).
        \end{equation}
        On the other hand, by property (i) of \cref{prop:pm_properties}, we see that
        \[
            Z_{\Phi,j}^{+}(\mathbf{x};u) = \sum_{\text{$\what{k}$ with $n(\what{k})$ even}}T(x_{j};\what{k},u)\prod_{i \neq j}x_{i}^{k_{i}}.
        \]
        Acting by $\s_{j}$, we also get
        \[
            Z_{\Phi,j}^{+}(\s_{j}\mathbf{x};u) = \sum_{\text{$\what{k}$ with $n(\what{k})$ even}}T\left(\frac{1}{ux_{j}};\what{k},u\right)(\sqrt{u}x_{j})^{n(\what{k})}\prod_{i \neq j}x_{i}^{k_{i}}.
        \]
        Using these two expressions for $Z_{\Phi,j}^{+}(\mathbf{x};u)$ and $Z_{\Phi,j}^{+}(\s_{j}\mathbf{x};u)$ and comparing coefficients in \cref{equ:T_functional_equations_2} gives the result. Statement (ii) is proved in the same way by taking the odd parts in \cref{equ:T_functional_equations_1}. For statement (iii), \cref{prop:CG_coefficients_polynomial_bound} implies
        \[
            |T(x_{j};\what{k},u)| < \sum_{k_{j} \ge 0}|a(k_{1},\ldots,k_{r};u)x_{j}^{k_{j}}| < c_{1}u^{c_{2}|\what{k}|}\sum_{k_{j} \ge 0}u^{c_{2}}|x_{j}|^{k_{j}}.
        \]
        The latter sum is a geometric series which converges absolutely (and hence is at most a constant) provided $|u^{c_{2}}x_{j}| < 1$, or equivalently, $|x_{j}| < u^{-c_{2}}$.
    \end{proof}

    Lastly, we show a connection between the $W$-invariant function $Z_{\Phi}(\mathbf{x};u)$ and data of the root system $\Phi$. This is seen by inspecting the parameter $u$ in $Z_{W}(\mathbf{x};u)$. Expanding this function as a power series in $x_{1},\ldots,x_{r}$, yields
    \[
        Z_{W}(\mathbf{x};u) = \sum_{k_{1},\ldots,k_{r} \ge 0}b(k_{1},\ldots,k_{r};u)x_{1}^{k_{1}} \cdots x_{r}^{k_{r}}.
    \]
    The nonzero terms of this series are in bijection with the elements of $W$. Indeed, for this we may ignore questions of convergence so set $u = 1$. Then the Chinta-Gunnells action simplifies, and from the definition of $Z_{W}(\mathbf{x};u)$, we compute
    \[
        Z_{W}(\mathbf{x};1) = \sum_{w \in W}(-1)^{\ell(w)+h(\rho-w\rho)}\mathbf{x}^{\rho-w\rho}.
    \]
    But then
    \[
        b(k_{1},\ldots,k_{r};1) = \begin{cases} (-1)^{\ell(w)+h(\rho-w\rho)} & \text{if $\rho-w\rho = \sum_{1 \le i \le r}k_{i}\a_{i}$ for some $w \in W$}, \\ 0 & \text{otherwise}, \end{cases}
    \]
    provided all of the monomials $\mathbf{x}^{\rho-w\rho}$ are distinct. This is indeed true because the Weyl vector lies in the interior of the fundamental Weyl chamber and the Weyl group acts on the Weyl chambers simply transitively.
\section{The Weyl Group Multiple Dirichlet Series}
    As before, let $\Phi$ be an irreducible and simply laced root system. We will label the nodes of the Dynkin diagram of $\Phi$ such that for any node $i$ all adjacent nodes are either less than $i$ or greater than $i$. This is possible since, if not, there must be a loop in the Dynkin diagram but from the classification of finite root systems no Dynkin diagram has a loop. Now let $q$ be a power of a fixed prime $p$ and consider the rational function field $\F_{q}(t)$. Set $u = q$. We will construct the Weyl group multiple Dirichlet series associated to a root system $\Phi$ over the field $\F_{q}(t)$ from the Chinta-Gunnells average $Z_{\Phi}(\mathbf{x};q)$. The idea is to use the coefficients $a(k_{1},\ldots,k_{r};q)$ in the power series expansion
    \[
        Z_{\Phi}(\mathbf{x};q) = \sum_{k_{1},\ldots,k_{r} \ge 0}a(k_{1},\ldots,k_{r};q)x_{1}^{k_{1}} \cdots x_{r}^{k_{r}},
    \]
    to build the associated multiple Dirichlet series. For simplicity, we work over the function field $\F_{q}(t)$ but our construction can be done over other fields. Let $P$ be a monic irreducible in $\F_{q}[t]$. The series
    \[
        Z_{\Phi}(|P|^{s_{1}},\ldots,|P|^{s_{r}};|P|) = \sum_{k_{1},\ldots,k_{r} \ge 0}\frac{a(k_{1},\ldots,k_{r};|P|)}{|P|^{k_{1}s_{1}+\cdots+k_{r}s_{r}}},
    \]
    is called the \textbf{$P$-th part} of $Z(s_{1},\ldots,s_{r})$. We are now ready to start defining the Weyl group multiple Dirichlet series $Z(s_{1},\ldots,s_{r})$. We will first define coefficients $H(m_{1},\ldots,m_{r})$ via the following two properties:
    \begin{enumerate}[label=(\roman*)]
        \item For any monic irreducible $P$ and nonnegative integers $k_{1},\ldots,k_{r} \ge 0$, define
        \[
            H(P^{k_{1}},\ldots,P^{k_{r}}) = a(k_{1},\ldots,k_{r};|P|).
        \]
        \item For any $r$-tuple $(m_{1}n_{1},\ldots,m_{1}n_{1})$ of monic polynomials in $\F_{q}(t)$ such that $(m_{1} \cdots m_{r},n_{1} \cdots n_{r}) = 1$, we set
        \[
            H(m_{1}n_{1},\ldots,m_{1}n_{1}) = H(m_{1},\ldots,m_{r})H(n_{1},\ldots,n_{r})\prod_{\substack{i \sim j \\ i < j}}\legendre{m_{i}}{n_{j}}\legendre{n_{i}}{m_{j}}.
        \]
    \end{enumerate}

    Observe that property (ii) prevents $H(m_{1},\ldots,m_{r})$ from being multiplicative. We refer to property (ii) as \textbf{twisted multiplicativity} for the coefficients $H(m_{1},\ldots,m_{r})$. Moreover, these coefficients satisfty some nice properties:

    \begin{proposition}\label{prop:WMDS_coefficient_properties}
        The coefficients $H(m_{1},\ldots,m_{r})$ satisfy the following properties:
        \begin{enumerate}[label=(\roman*)]
            \item There is a constant $C > 0$ such that
            \[
                |H(m_{1},\ldots,m_{r})| \ll |m_{1} \cdots m_{r}|^{C}.
            \]
            \item If $m_{1},\ldots,m_{r}$ are pairwise relatively prime, then
            \[
                H(m_{1},\ldots,m_{r}) = \prod_{\substack{i \sim j \\ i < j}}\legendre{m_{i}}{m_{j}}.
            \]
        \end{enumerate}
    \end{proposition}
    \begin{proof}
        Property (i) follows immeditely from the definition of $H(m_{1},\ldots,m_{r})$ and \cref{prop:CG_coefficients_polynomial_bound}. Property (ii) follows from combining the definition of $H(m_{1},\ldots,m_{r})$, and the facts
        \[
            H(1,\ldots,P^{k_{i}},\ldots,1) = a(0,\ldots,k_{i},\ldots,0;|P|) = 1 \quad \text{and} \quad \legendre{1}{P} = \legendre{P}{1} = 1,
        \]
        for all monic irreducibles $P$. 
    \end{proof}

    We define the \textbf{Weyl group multiple Dirichlet series} $Z(s_{1},\ldots,s_{r})$ by
    \[
        Z(s_{1},\ldots,s_{r}) = \sum_{\text{$m_{1},\ldots,m_{r}$ monic}}\frac{H(m_{1},\ldots,m_{r})}{|m_{1}|^{s_{1}} \cdots |m_{r}|^{s_{r}}},
    \]
    where the sum is over all monics in $\F_{q}[t]$. Note that by property (i) of \cref{prop:WMDS_coefficient_properties}, $Z(s_{1},\ldots,s_{r})$ converges locally absolutely uniformly on the region $\L = \{(s_{1},\ldots,s_{r}) \in \C^{r}:\Re(s_{i}) > 1+C, 1 \le i \le r\}$ by viewing it as a Dirichlet series in one variable whose coefficients are in $r-1$ variables and then viewing the coefficients in the same way. In order to study $Z(s_{1},\ldots,s_{r})$ we will also need to consider twists of it by a set of Hilbert symbols. For each $1 \le i \le r$, let $\psi_{i}$ be a Hilbert symbol. As we are working over function fields, $\psi_{i} = \psi$ or $\psi_{i} = \chi_{1}$. Now define the \textbf{Weyl group multiple Dirichlet series} $Z_{\psi_{1},\ldots,\psi_{r}}(s_{1},\ldots,s_{r})$ twisted by $\psi_{1},\ldots,\psi_{r}$ as
    \[
        Z_{\psi_{1},\ldots,\psi_{r}}(s_{1},\ldots,s_{r}) = \sum_{\text{$m_{1},\ldots,m_{r}$ monic}}\frac{\psi_{1}(m_{1}) \cdots \psi_{r}(m_{r})H(m_{1},\ldots,m_{r})}{|m_{1}|^{s_{1}} \cdots |m_{r}|^{s_{r}}}.
    \]
    Since the Hilbert symbols are given by quadratic characters, it follows that $Z_{\psi_{1},\ldots,\psi_{r}}(s_{1},\ldots,s_{r})$ converges locally absolutely uniformly in the same region as $Z(s_{1},\ldots,s_{r})$. Note that if $\psi_{1} = \cdots = \psi_{r} = \chi_{1}$, then $Z_{\psi_{1},\ldots,\psi_{r}}(s_{1},\ldots,s_{r}) = Z(s_{1},\ldots,s_{r})$.
\section{Correction polynomials}
    We will now deduce expressions for subsums of $Z_{\psi_{1},\ldots,\psi_{r}}(s_{1},\ldots,s_{r})$. For this, we always work in the region of local absolute uniform convergence. To begin, fix an index $1 \le j \le r$. We will assume that the Dynkin diagram of $\Phi$ is labeled so that all nodes $i$ with $i \sim j$ satisfy $i < j$. Also, for any $r$-tuple $m = (m_{1},\ldots,m_{r})$, set
    \[
        \what{m} = (m_{1},\ldots,m_{j-1},m_{j+1},\ldots,m_{r}).
    \]
    To begin, summing over the index $j$ in $Z_{\psi_{1},\ldots,\psi_{r}}(s_{1},\ldots,s_{r})$ first gives
    \[
        Z_{\psi_{1},\ldots,\psi_{r}}(s_{1},\ldots,s_{r}) = \sum_{\text{$\what{m}$ monic}}\frac{\prod_{i \neq j}\psi_{i}(m_{i})}{\prod_{i \neq j}|m_{i}|^{s_{i}}}\sum_{\text{$m_{j}$ monic}}\frac{\psi_{j}(m_{j})H(m_{1},\ldots,m_{j},\ldots,m_{r})}{|m_{j}|^{s_{j}}}.
    \]
    The idea is to express the inner sum over $m_{j}$ as the product of an $L$-function and a Dirichlet polynomial up to a constant. Set $N_{j} = m_{1} \cdots m_{j-1}m_{j+1} \cdots m_{r}$. Then write $m_{j} = nn_{j}$ where $n_{j} \mid N_{j}$ and $(n,n_{j}) = 1$. Equivalently, $n$ is the part of $m_{j}$ that is relatively prime to $N_{j}$. So we also have $(n,N_{j}) = 1$. Then we may write
    \begin{align*}
        \sum_{\text{$m_{j}$ monic}}\frac{\psi_{j}(m_{j})H(m_{1},\ldots,m_{r})}{|m_{j}|^{s_{j}}} &= \sum_{\text{$nn_{j}$ monic}}\frac{\psi_{j}(nn_{j})H(m_{1},\ldots,m_{j},\ldots,m_{r})}{|nn_{j}|^{s_{j}}} \\
        &= \sum_{n_{j} \mid N_{j}^{\infty}}\sum_{(n,N_{j}) = 1}\frac{\psi_{j}(nn_{j})H(m_{1},\ldots,nn_{j},\ldots,m_{r})}{|nn_{j}|^{s_{j}}} \\
        &= \sum_{n_{j} \mid N_{j}^{\infty}}\frac{\psi_{j}(n_{j})}{|n_{j}|^{s_{j}}}\sum_{(n,N_{j}) = 1}\frac{\psi_{j}(n)H(m_{1},\ldots,nn_{j},\ldots,m_{r})}{|n|^{s_{j}}},
    \end{align*}
    where we recall that $n_{j} \mid N_{j}^{\infty}$ means that the irreducible factors of $n_{j}$ are a subset of the irreducible factors of $N_{j}$. Using twisted multiplicativity, we may pull a factor $H(m_{1},\ldots,n_{j},\ldots,m_{r})$ into the outer sum obtaining
    \[
        \sum_{n_{j} \mid N_{j}^{\infty}}\frac{\psi_{j}(n_{j})H(m_{1},\ldots,n_{j},\ldots,m_{r})}{|n_{j}|^{s_{j}}}\sum_{(n,N_{j}) = 1}\frac{\psi_{j}(n)H(1,\ldots,n,\ldots,1)}{|n|^{s_{j}}}\prod_{\substack{i \sim j \\ i < j}}\legendre{m_{i}}{n}.
    \]
    Set $M = \prod_{\substack{i \sim j \\ i < j}}m_{i}$ and let $M_{0}$ be the square-free part of $M$. Since $(n,N_{j}) = 1$, the inner product is $\chi_{M}(n) = \chi_{M_{0}}(n)$. But as $H(1,\ldots,n,\ldots,1) = 1$, these two facts imply 
    \[
        \sum_{(n,N_{j}) = 1}\frac{\psi_{j}(n)H(1,\ldots,n,\ldots,1)}{|n|^{s_{j}}}\prod_{\substack{i \sim j \\ i < j}}\legendre{m_{i}}{n} = L^{(N_{j})}(s_{j},\psi_{j}\chi_{M_{0}}),
    \]
    where we recall that $L^{(N_{j})}(s_{j},\psi_{j}\chi_{M_{0}})$ is $L(s_{j},\psi_{j}\chi_{M_{0}})$ with the local factors at the irreducibles dividing $N_{j}$ removed. This $L$-function may be factored outside of the double sum so that in total we obtain
    \begin{equation}\label{equ:WMDS_functional_equation_1}
        \sum_{\text{$m_{j}$ monic}}\frac{\psi_{j}(m_{j})H(m_{1},\ldots,m_{r})}{|m_{j}|^{s_{j}}} = L^{(N_{j})}(s_{j},\psi_{j}\chi_{M_{0}})\sum_{n_{j} \mid N_{j}^{\infty}}\frac{\psi_{j}(n_{j})H(m_{1},\ldots,n_{j},\ldots,m_{r})}{|n_{j}|^{s_{j}}}.
    \end{equation}
    We will now examine the sum in the right-hand side of \cref{equ:WMDS_functional_equation_1}. We will show that it factors as a product. To see this, let $P$ be a prime dividing $N_{j}$. Then we may write
    \begin{align*}
        \sum_{n_{j} \mid N_{j}^{\infty}}\frac{\psi_{j}(n_{j})H(m_{1},\ldots,n_{j},\ldots,m_{r})}{|n_{j}|^{s_{j}}} &= \sum_{\substack{n_{j} \mid N_{j}^{\infty} \\ (n_{j},P) = 1}}\sum_{k_{j} \ge 0}\frac{\psi_{j}(n_{j}P^{k_{j}})H(m_{1},\ldots,n_{j}P^{k_{j}},\ldots,m_{r})}{|n_{j}P^{k_{j}}|^{s_{j}}} \\
        &= \sum_{\substack{n_{j} \mid N_{j}^{\infty} \\ (n_{j},P) = 1}}\frac{\psi_{j}(n_{j})}{|n_{j}|^{s_{j}}}\sum_{k_{j} \ge 0}\frac{\psi_{j}(P^{k_{j}})H(m_{1},\ldots,n_{j}P^{k_{j}},\ldots,m_{r})}{|P^{k_{j}}|^{s_{j}}}.
    \end{align*}
    Now let $P^{\b_{i}} \mid\mid m_{i}$ and $m_{i}^{(P)} = \frac{m_{i}}{P^{\b_{i}}}$ for $1 \le i \le r$. Then by twisted multiplicativity we may pull out a factor $H(m_{1}^{(P)},\ldots,n_{j},\ldots,m_{r}^{(P)})$ obtaining
    \begin{align*}
        \sum_{\substack{n_{j} \mid N_{j}^{\infty} \\ (n_{j},P) = 1}}\frac{\psi_{j}(n_{j})H(m_{1}^{(P)},\ldots,n_{j},\ldots,m_{r}^{(P)})}{|n_{j}|^{s_{j}}}\sum_{k_{j} \ge 0}\frac{\psi_{j}(P^{k_{j}})H(P^{\b_{1}},\ldots,P^{k_{j}},\ldots,P^{\b_{r}})}{|P^{k_{j}}|^{s_{j}}}&\prod_{\substack{i \sim j \\ i < j}}\legendre{m_{i}^{(P)}}{P^{k_{j}}}\legendre{P^{\b_{i}}}{n_{j}} \\
        \cdot &\prod_{\substack{i \sim \ell \\ i < \ell \\ i,\ell \neq j}}\legendre{m_{i}^{(P)}}{P^{\b_{\ell}}}\legendre{P^{\b_{i}}}{m_{\ell}^{(P)}}.
    \end{align*}
    Since reciprocity is perfect,
    \[
        \prod_{\substack{i \sim \ell \\ i < \ell \\ i,\ell \neq j}}\legendre{m_{i}^{(P)}}{P^{\b_{\ell}}}\legendre{P^{\b_{i}}}{m_{\ell}^{(P)}} = \prod_{\substack{i \sim \ell \\ i < \ell \\ i,\ell \neq j}}\legendre{m_{i}^{(P)}}{P^{\b_{\ell}}}\legendre{m_{\ell}^{(P)}}{P^{\b_{i}}} = \prod_{\substack{i \sim \ell \\ i < \ell \\ i,\ell \neq j}}\legendre{(m_{i}m_{\ell})^{(P)}}{P^{\b_{i}+\b_{\ell}}}.
    \]
    Now set
    \[
        \e_{P}(\what{m}) = \prod_{\substack{i \sim \ell \\ i < \ell \\ i,\ell \neq j}}\prod_{\substack{i \sim \ell \\ i < \ell \\ i,\ell \neq j}}\legendre{(m_{i}m_{\ell})^{(P)}}{P^{\b_{i}+\b_{\ell}}} \quad \text{and} \quad \e_{j}(\what{m}) = \prod_{P \mid N_{j}}\e_{P}(\what{m}).
    \]
    Note that $\e_{P}(\what{m})$ is the second product in our expression above. This factor is independent of both sums and so we may pull it outside. Further factoring out the sum over $k_{j}$, we obtain
    \begin{align*}
        \e_{P}(\what{m})\sum_{k_{j} \ge 0}\frac{\psi_{j}(P^{k_{j}})H(P^{\b_{1}},\ldots,P^{k_{j}},\ldots,P^{\b_{r}})}{|P^{k_{j}}|^{s_{j}}}&\prod_{\substack{i \sim j \\ i < j}}\legendre{m_{i}^{(P)}}{P^{k_{j}}} \\
        &\cdot \sum_{\substack{n_{j} \mid N_{j}^{\infty} \\ (n_{j},P) = 1}}\frac{\psi_{j}(n_{j})H(m_{1}^{(P)},\ldots,n_{j},\ldots,m_{r}^{(P)})}{|n_{j}|^{s_{j}}}\prod_{\substack{i \sim j \\ i < j}}\legendre{P^{\b_{i}}}{n_{j}}.
    \end{align*}
    After repeating this process to
    \[
        \sum_{\substack{n_{j} \mid N_{j}^{\infty} \\ (n_{j},P) = 1}}\frac{\psi_{j}(n)H(m_{1}^{(P)},\ldots,n,\ldots,m_{r}^{(P)})}{|n|^{s_{j}}}\prod_{\substack{i \sim j \\ i < j}}\legendre{P^{\b_{i}}}{n},
    \]
    for all primes dividing $N_{j}$, the sum over $n_{j}$ factors as
    \[
        \e_{j}(\what{m})\prod_{\substack{P \mid N_{j} \\ P^{\b_{i} \mid\mid m_{i}}}}\sum_{k_{j} \ge 0}\frac{\psi_{j}(P^{k_{j}})H(P^{\b_{1}},\ldots,P^{k_{j}},\ldots,P^{\b_{r}})}{|P^{k_{j}}|^{s_{j}}}\legendre{M^{(P)}}{P^{k_{j}}}.
    \]
    Therefore, we have
    \begin{equation}\label{equ:WMDS_functional_equation_2}
        \begin{aligned}
            \sum_{\text{$m_{j}$ monic}}\frac{\psi_{j}(m_{j})H(m_{1},\ldots,m_{r})}{|m_{j}|^{s_{j}}} &= \e_{j}(\what{m})L^{(N_{j})}(s_{j},\psi_{j}\chi_{M_{0}}) \\
            &\cdot \prod_{\substack{P \mid N_{j} \\ P^{\b_{i} \mid\mid m_{i}}}}\sum_{k_{j} \ge 0}\frac{\psi_{j}(P^{k_{j}})H(P^{\b_{1}},\ldots,P^{k_{j}},\ldots,P^{\b_{r}})}{|P^{k_{j}}|^{s_{j}}}\legendre{M^{(P)}}{P^{k_{j}}}.
        \end{aligned}
    \end{equation}

    Now write $M = M_{0}M_{1}^{2}M_{2}^{2}$ with $M_{0}$ square-free, $M_{2}$ relatively prime to $M_{0}M_{1}$, and such that every irreducible divisor of $M_{1}$ divides $M_{0}$. In other words, $M_{0}$ is the square-free part of $M$, $M_{1}$ is the square part of $M$ whose irreducible factors divide $M$ to odd power, and $M_{2}$ is the square part of $M$ whose irreducible factors divide $M$ to even power. We will inspect the sum over $k_{j}$ in \cref{equ:WMDS_functional_equation_2} depending upon the order that $P$ divides $M$ to. We break this into three cases:

    \begin{enumerate}[label=(\roman*)]
        \item $P$ does not divide $M$: Suppose $(M,P) = 1$. Then $M^{(P)} = M$ so that 
        \[
            \legendre{M^{(P)}}{P^{k_{j}}} = \chi_{M^{(P)}}(P^{k_{j}}) = \chi_{M}(P^{k_{j}}) = \chi_{M_{0}}(P^{k_{j}}).
        \]
        Moreover, from the defintion of $M$ we see that $P$ is relatively prime to $m_{i}$ provided $i \sim j$. Therefore $\b_{i} = 0$ for such $i$. But then $H(P^{\b_{1}},\ldots,P^{k_{j}},\ldots,P^{\b_{r}}) = a(\b_{1},\ldots,k_{j},\ldots,\b_{r};|P|) = 1$ since $\b_{i} = 0$ if $i \sim j$. The sum over $k_{j}$ reduces to
        \[
            \sum_{k_{j} \ge 0}\frac{(\psi_{j}\chi_{M_{0}})(P^{k_{j}})}{|P^{k_{j}}|^{s_{j}}} = (1-(\psi_{j}\chi_{M_{0}})(P)|P|^{-s_{j}})^{-1},
        \]
        which is the local factor of $L(s_{j},\psi_{j}\chi_{M_{0}})$ at $P$.
        \item $P$ divides $M$ to odd order: Suppose $P^{2\a+1} \mid\mid M$ for some $\a \ge 1$. Noticing that $\tlegendre{M^{(P)}}{P^{k_{j}}} = \chi_{M^{(P)}}(P^{k_{j}})$, define 
        \[
            Q_{P^{2\a+1}}(s_{j};\psi_{j}\chi_{M^{(P)}}) = \sum_{k_{j} \ge 0}\frac{(\psi_{j}\chi_{M^{(P)}})(P^{k_{j}})H(P^{\b_{1}},\ldots,P^{k_{j}},\ldots,P^{\b_{r}})}{|P^{k_{j}}|^{s_{j}}}.
        \]
        Then $Q_{P^{2\a+1}}(s_{j};\psi_{j}\chi_{M^{(P)}})$ is the sum over $k_{j}$. Now apply statement (ii) of \cref{prop:T_functional_equations} with $x_{j} = (\psi_{j}\chi_{M^{(P)}})(P)|P|^{-s_{j}}$ and use the fact $((\psi_{j}\chi_{M^{(P)}})(P))^{2} = 1$ to produce the functional equation
        \[
            Q_{P^{2\a+1}}(s_{j};\psi_{j}\chi_{M^{(P)}}) = |P|^{\a(1-2s_{j})}Q_{P^{2\a+1}}(1-s_{j};\psi_{j}\chi_{M^{(P)}}).
        \]
        \item $P$ divides $M$ to even order: Suppose $P^{2\a} \mid\mid M$ for some $\a \ge 1$. Then $\chi_{M^{(P)}}(P^{k_{j}}) = \chi_{M_{0}}(P^{k_{j}})$ which implies $\tlegendre{M^{(P)}}{P^{k_{j}}} = \chi_{M_{0}}(P^{k_{j}})$. Now define
        \[
            Q_{P^{2\a}}(s_{j};\psi_{j}\chi_{M_{0}}) = (1-(\psi_{j}\chi_{M_{0}})(P)|P|^{-s_{j}})\sum_{k_{j} \ge 0}\frac{(\psi_{j}\chi_{M_{0}})(P^{k_{j}})H(P^{\b_{1}},\ldots,P^{k_{j}},\ldots,P^{\b_{r}})}{|P^{k_{j}}|^{s_{j}}}.
        \]
        Then $(1-(\psi_{j}\chi_{M_{0}})(P)|P|^{-s_{j}})^{-1}Q_{P^{2\a}}(s_{j};\psi_{j}\chi_{M_{0}})$ is the sum over $k_{j}$. Using statement (i) of \cref{prop:T_functional_equations} with $x_{j} = (\psi_{j}\chi_{M^{(P)}})(P)|P|^{-s_{j}}$ again and using using the fact $((\psi_{j}\chi_{M^{(P)}})(P))^{2} = 1$ yields the functional equation
        \[
            Q_{P^{2\a}}(s_{j};\psi_{j}\chi_{M_{0}}) = |P|^{\a(1-2s_{j})}Q_{P^{2\a}}(1-s_{j};\psi_{j}\chi_{M_{0}}).
        \]
    \end{enumerate}
    Upon setting
    \[
        Q_{M}(s_{j};\psi_{j}) = \prod_{P^{\a} \mid\mid M_{1}}Q_{P^{2\a+1}}(s_{j};\psi_{j}\chi_{M^{(P)}}) \cdot \prod_{P^{\a} \mid\mid M_{2}}Q_{P^{2\a}}(s_{j};\psi_{j}\chi_{M_{0}}),
    \]
    cases (ii) and (iii) combine to give the functional equation
    \[
        Q_{M}(s_{j};\psi_{j}) = |M_{1}M_{2}|^{1-2s}Q_{M}(1-s_{j};\psi_{j}).
    \]
    This function equation implies that $Q_{M}(s_{j};\psi_{j})$ is a Dirichlet polynomial (and hence $Q_{P^{2\a+1}}(s_{j})$ and $Q_{P^{2\a}}(s_{j})$ are as well). Actually, from the functional equation, the highest power of $|P|^{-s_{j}}$ appearing $Q_{M}(s_{j};\psi_{j})$ is at most $2\a$ if $P^{\a} \mid\mid M_{1}M_{2}$. $Q_{M}(s_{j};\psi_{j})$ also satisfies a polynomial bound. If $\Re(s_{j}) > c_{2}$, then statement (iii) of \cref{prop:T_functional_equations} and the definition of $Q_{M}(s_{j};\psi_{j})$ together imply
    \[
        |Q_{M}(s_{j};\psi_{j})| \ll c_{1}^{\w(N_{j})}|N_{j}|^{c_{2}},
    \]
    where $\w(N_{j})$ is the number of prime divisors of $N_{j}$. This implies the simplified estimate
    \[
        |Q_{M}(s_{j};\psi_{j})| \ll |N_{j}|^{c_{3}},
    \]
    for some $c_{3} > 0$. Combining our three cases with \cref{equ:WMDS_functional_equation_2} results in
    \begin{equation}\label{equ:WMDS_functional_equation_3}
        \sum_{\text{$m_{j}$ monic}}\frac{\psi_{j}(m_{j})H(m_{1},\ldots,m_{r})}{|m_{j}|^{s_{j}}} = \e_{j}(\what{m})L(s_{j},\psi_{j}\chi_{M_{0}})Q_{M}(s_{j};\psi_{j}),
    \end{equation}
    which is a product of an $L$-function and a Dirichlet polynomial, up to a constant, as desired. We collect this work into two theorems. First, the Dirichlet polynomial:

    \begin{theorem}\label{thm:correction_polynomial}
        Fix an index $1 \le j \le r$ and an $(r-1)$-tuple of monics $\what{m} = (m_{1},\ldots,m_{j-1},m_{j+1},\ldots,m_{r})$. Set $M = \prod_{\substack{i \sim j \\ i < j}}m_{i}$ and let $M = M_{0}M_{1}^{2}M_{2}^{2}$ be the square decomposition of $M$ stratified into even and odd powers. Also set $N_{j} = m_{1} \cdots m_{j-1}m_{j+1} \cdots m_{r}$. Then $Q_{M}(s_{j};\psi_{j})$ is a Dirichlet polynomials supported on the prime dividing $M$ to order larger than $1$. It admits an Euler product
        \[
            Q_{M}(s_{j};\psi_{j}) = \prod_{P^{\a} \mid\mid M_{1}}Q_{P^{2\a+1}}(s_{j};\psi_{j}\chi_{M^{(P)}}) \cdot \prod_{P^{\a} \mid\mid M_{2}}Q_{P^{2\a}}(s_{j};\psi_{j}\chi_{M_{0}}),
        \]
        and satisfies a functional equation
        \[
            Q_{M}(s_{j};\psi_{j}) = |M_{1}M_{2}|^{1-2s}Q_{M}(1-s_{j};\psi_{j}).
        \]
        Moreover, for $\Re(s_{j}) > c_{2}$, $Q_{M}(s_{j};\psi_{j})$ satisfies the bound
        \[
            |Q_{M}(s_{j};\psi_{j})| < |N_{j}|^{c_{3}},
        \]
        for some $c_{3} > 0$.
    \end{theorem}

    The Dirichlet polynomial $Q_{M}(s_{j};\psi_{j})$ given in \cref{thm:correction_polynomial} is called a \textbf{correction polynomial}. It is the factor that $L(s_{j},\psi_{j}\chi_{M_{0}})$ is multiplied by, when $M$ is not square-free, to allow the global Weyl group multiple Dirichlet series to admit functional equations. Our second theorem, is a decomposition of the subseries over $s_{j}$ in terms of correction polynomials:

    \begin{theorem}\label{thm:WMDS_subseries_decomposition}
        Fix an index $1 \le j \le r$ and an $(r-1)$-tuple of monics $\what{m} = (m_{1},\ldots,m_{j-1},m_{j+1},\ldots,m_{r})$. Set $M = \prod_{\substack{i \sim j \\ i < j}}m_{i}$ and let $M = M_{0}M_{1}^{2}M_{2}^{2}$ be the square decomposition of $M$ stratified into even and odd powers. Then the subseries of $Z_{\psi_{1},\ldots,\psi_{r}}(s_{1},\ldots,s_{r})$ over $s_{j}$ corresponding to $\what{m}$ admits the decomposition
        \[
            \sum_{\text{$m_{j}$ monic}}\frac{\psi_{j}(m_{j})H(m_{1},\ldots,m_{r})}{|m_{j}|^{s_{j}}} = \e_{j}(\what{m})L(s_{j},\psi_{j}\chi_{M_{0}})Q_{M}(s_{j};\psi_{j}),
        \]
        where $\e_{j}(\what{m}) = \pm1$.
    \end{theorem}

    As a consquence of \cref{thm:WMDS_subseries_decomposition}, we immeitely get the following representations of $Z_{\psi_{1},\ldots,\psi_{r}}(s_{1},\ldots,s_{r})$ for every $1 \le j \le r$:
    \begin{equation}\label{equ:WMDS_the_interchange}
        Z_{\psi_{1},\ldots,\psi_{r}}(s_{1},\ldots,s_{r}) = \sum_{\text{$\what{m}$ monic}}\frac{\prod_{i \neq j}\psi_{i}(m_{i})}{\prod_{i \neq j}|m_{i}|^{s_{i}}}\e_{j}(\what{m})L(s_{j},\psi_{j}\chi_{M_{0}})Q_{M}(s_{j};\psi_{j}).
    \end{equation}
    We refer to \cref{equ:WMDS_the_interchange} as \textbf{the interchange} for $Z_{\psi_{1},\ldots,\psi_{r}}(s_{1},\ldots,s_{r})$. 
\section{Functional Equations}
    We can now prove functional equations for $Z(s_{1},\ldots,s_{r})$. For $\mathbf{s} = (s_{1},\ldots,s_{r})$, it will be convienent to set
    \[
        Z_{\psi_{1},\ldots,\psi_{r}}(\mathbf{s}) = Z_{\psi_{1},\ldots,\psi_{r}}(s_{1},\ldots,s_{r}),
    \]
    so, in particular, $Z(\mathbf{s}) = Z(s_{1},\ldots,s_{r})$. One can actually prove functional equations for all of the twisted Weyl group multiple Dirichlet series $Z_{\psi_{1},\ldots,\psi_{r}}(s_{1},\ldots,s_{r})$ but we will not need this level of generality. So we assume $\psi_{1} = \cdots = \psi_{r} = \chi_{1}$ is the trivial character. We will also write $Q_{M}(s_{j}) = Q_{M}(s_{j};\chi_{1})$. Letting $x_{i} = q^{-s_{i}}$ for $1 \le i \le r$, the $W$-action on $\mathbf{s} = (s_{1},\ldots,s_{r})$ is easily checked to be given component-wise on simple reflections by
    \[
        (\s_{i}\mathbf{s})_{j} = \begin{cases} s_{i}+s_{j}-\frac{1}{2} & \text{if $i \sim j$}, \\ 1-s_{j} & \text{if $i = j$}, \\ s_{j} & \text{otherwise}.  \end{cases}
    \]
    We will deduce a functional equation of shape $\mathbf{s} \to \s_{j}\mathbf{s}$ for every $1 \le j \le r$. In accordance with \cref{thm:WMDS_subseries_decomposition}, define
    \[
        L(s_{j},\chi_{M};\what{m}) = \sum_{\text{$m_{j}$ monic}}\frac{H(m_{1},\ldots,m_{r})}{|m_{j}|^{s_{j}}} = \e_{j}(\what{m})L(s_{j},\chi_{M_{0}})Q_{M}(s_{j}).
    \]
    Since $Q_{M}(s_{j})$ is a Dirichlet polynomial it admits analytic continuation to $\C$. This implies $L(s_{j},\chi_{M};\what{m})$ admits meromorphic continuation to $\C$ with a simple pole at $s = 1$ if and only if $\chi_{M_{0}}$ is the trivial character. That is, if and only if $M$ is a perfect square. Now $L(s_{j},\chi_{M_{0}})$ satisfies a functional equation and by \cref{thm:correction_polynomial} we know that $Q_{M}(s_{j})$ does as well. We can combine these functional equations to deduce a functional equation for $L(s_{j},\chi_{M};\what{m})$. Define the completed $L$-function
    \[
        L(s_{j},\chi_{M};\what{m}) = \e_{j}(\what{m})L^{\ast}(s_{j},\chi_{M_{0}})Q_{M}(s_{j}).
    \]
    Then we have the functional equation
    \[
        L^{\ast}(s_{j},\chi_{M};\what{m}) = \begin{cases} q^{2s_{j}-1}|M|^{\frac{1}{2}-s_{j}}L^{\ast}(1-s_{j},\chi_{M};\what{m}) & \text{if $\deg(M)$ is even}, \\ q^{s_{j}-\frac{1}{2}}|M|^{\frac{1}{2}-s_{j}}L^{\ast}(1-s_{j},\chi_{M};\what{m}) & \text{if $\deg(M)$ is odd}. \end{cases}
    \]
    At last, we can obtain a functional equation for $Z(s_{1},\ldots,s_{r})$. By \cref{thm:WMDS_subseries_decomposition} we know
    \[
        Z(s_{1},\ldots,s_{r}) = \sum_{\text{$\what{m}$ monic}}\frac{L(s_{j},\chi_{M};\what{m})}{\prod_{i \neq j}|m_{i}|^{s_{i}}}.
    \]
    Now define
    \[
        Z_{j}^{\pm}(s_{1},\ldots,s_{r}) = \frac{Z(s_{1},\ldots,s_{r}) \pm Z_{\psi}(s_{1},\ldots,s_{r})}{2},
    \]
    where $Z_{\psi}(s_{1},\ldots,s_{r}) = Z_{\psi_{1},\ldots,\psi_{r}}(s_{1},\ldots,s_{r})$ is such that $\psi_{i} = \psi$ if $i \sim j$ and $i < j$ and $\psi_{i} = \chi_{1}$ otherwise (we are slightly abusing notation with the even and odd parts of $f \in \C(\mathbf{x},u)$ with respect to $\e_{i}$ given in the description of the Chinta-Gunnells action). By the construction of $Z_{\psi}(s_{1},\ldots,s_{r})$, we have
    \[
        Z_{j}^{+}(s_{1},\ldots,s_{r}) = \sum_{\substack{\text{$\what{m}$ monic} \\ \text{$M$ even}}}\frac{L(s_{j},\chi_{M};\what{m})}{\prod_{i \neq j}|m_{i}|^{s_{i}}} \quad \text{and} \quad Z_{j}^{-}(s_{1},\ldots,s_{r}) = \sum_{\substack{\text{$\what{m}$ monic} \\ \text{$M$ odd}}}\frac{L(s_{j},\chi_{M};\what{m})}{\prod_{i \neq j}|m_{i}|^{s_{i}}},
    \]
    are the subsums of $Z(s_{1},\ldots,s_{r})$ whose functional equations for $L(s_{j},\chi_{M};\what{m})$ have a fixed gamma factor. The subsums $Z_{j}^{+}(s_{1},\ldots,s_{r})$ and $Z_{j}^{-}(s_{1},\ldots,s_{r})$ admit functional equations, and as $Z(s_{1},\ldots,s_{r})$ is a linear combination of these two series, we obtain a functional equation for $Z(s_{1},\ldots,s_{r})$:

    \begin{theorem}\label{thm:WMDS_functional_equations}
        Fix some $1 \le j \le r$. Then $Z(\mathbf{s})$ admits the functional equation
        \[
            Z(\mathbf{s}) = \frac{1}{2}\left(\frac{q^{2s_{j}-1}(1-q^{-s_{j}})}{1-q^{s_{j}-1}}\right)Z(\s_{j}\mathbf{s})+\frac{1}{2}\left(\frac{q^{2s_{j}-1}(1-q^{-s_{j}})}{1-q^{s_{j}-1}}\right)Z_{\psi}(\s_{j}\mathbf{s}).
        \]
    \end{theorem}
    \begin{proof}
        The functional equation for $L^{\ast}(s_{j},\chi_{M};\what{m})$ implies the functional equations
        \[
            Z_{j}^{+}(\mathbf{s}) = \frac{q^{2s_{j}-1}(1-q^{-s_{j}})}{1-q^{s_{j}-1}}Z_{j}^{+}(\s_{j}\mathbf{s}) \quad \text{and} \quad Z_{j}^{-}(\mathbf{s}) = q^{2s_{j}-1}Z_{j}^{-}(\s_{j}\mathbf{s}).
        \]
        As $Z(\mathbf{s}) = Z_{j}^{+}(\mathbf{s})+Z_{j}^{-}(\mathbf{s})$, the functional equations just stated give
        \[
            Z(\mathbf{s}) = \frac{q^{2s_{j}-1}(1-q^{-s_{j}})}{1-q^{s_{j}-1}}Z_{j}^{+}(\s_{j}\mathbf{s})+q^{2s_{j}-1}Z_{j}^{-}(\s_{j}\mathbf{s}).
        \]
        The desired functional equation for $Z(\mathbf{s})$ follows by expressing $Z_{j}^{+}(\mathbf{s})$ and $Z_{j}^{-}(\mathbf{s})$ in terms of $Z(\mathbf{s})$ and $Z_{\psi}(\mathbf{s})$.
    \end{proof}

    \cref{thm:WMDS_functional_equations} says that $Z(\mathbf{s})$ admits a functional equation for each simple reflection $\s_{j}$. But since the action of these simple reflections is a $W$-action on $\C^{r}(\mathbf{s};u)$ (where $x_{i} = q^{-s_{i}}$), it follows immeditely that $Z(\mathbf{s})$ posses a group of functional equations isomorphic to $W$.
\section{Meromorphic Continuation}
    In order to meromorphically continue $\Z(\mathbf{s})$ to $\C^{r}$, we will use Bochner's theorem. To state this theorem we only require a small definition. We say that a domain $\W \subset \C^{r}$ is a \textbf{tube domain} if there is an open set $\w \subset \R^{r}$ such that
    \[
        \W = \{\mathbf{s} \in \C^{r}:\Re(\mathbf{s}) \in \w\}.
    \]
    Now we can state Bochner's theorem (see \cite{hormander2000introduction} for a proof):

    \begin{theorem}[Bochner's continuation theorem]
        If $\W$ is a connected tube domain, then any holomorphic function on $\W$ can be extended to a holomorphic functon on the convex hull $\what{\W}$.
    \end{theorem}

    By clearing polar divisors, Bochner's continuation theorem implies that any meromorphic function on a connected tube domain posessing a finite amount of hyperplane polar divisors can be extended to a meromorphic function on the convex hull. This is the situation for $Z(\mathbf{s})$, but first we need to enlarge the region $Z(\mathbf{s})$ is defined on. This is achieved by the Phragm\'en-Lindel\"of convexity principal. Choose $s_{j}$ such that $\Re(s_{j}) > -c_{2}$. The functional equation for $L^{\ast}(s_{j},\chi_{M};\what{m})$ and \cref{thm:correction_polynomial} together imply the estimate
    \[
        L(-\e,\chi_{M};\what{m}) \ll |M|^{2c_{2}+1}|N_{j}|^{c_{3}}.
    \]
    As this $L$-function has at most a simple pole at $s_{j} = 1$, the Phragm\'en-Lindel\"of convexity principal implies the weak estimate
    \[
        (s_{j}-1)L(-\e,\chi_{M};\what{m}) \ll |M|^{2c_{2}+1}|N_{j}|^{c_{3}},
    \]
    for $\Re(s_{j}) > -c_{2}$. Using the interchange it follows that
    \[
        \left(\prod_{1 \le j \le r}(s_{j}-1)\right)Z(s_{1},\ldots,s_{r}),
    \]
    is locally absolutely uniformly convergent on the region
    \[
        \L_{0} = \L \cup \bigcup_{1 \le j \le r} \{(s_{1},\ldots,s_{r}):\Re(s_{j}) > -c_{2}, \Re(s_{i}) > 2c_{2}+c_{3}+2 \text{ for } i \neq j\}.
    \]
    Technically we only need $\Re(s_{i}) > 2c_{2}+c_{3}+2$ if $i \sim j$ and $\Re(s_{i}) > 1+c_{3}$ otherwise. But this is unimportant since $\L_{0}$ is sufficiently large. Indeed, $\L_{0}$ is a connected tube domain and
    \[
        \L_{W} = \bigcup_{w \in W}w\L_{0},
    \]
    is also a connected tube doamin that is all of $\C^{r}$ except for a compact subset about the origin. By \cref{thm:WMDS_functional_equations}, the possible polar divisors of $Z(\mathbf{s})$ belong to the set
    \[
        \{(w\mathbf{s})_{j}-1 = 0:w \in W, 1 \le j \le r\},
    \]
    which is finite since $W$ is finite. Therefore by Bochner's continuation theorem, $Z(\mathbf{s})$ admits meromorphic continuation to $\C^{r}$. We collect this work as a theorem:

    \begin{theorem}
        $Z(\mathbf{s})$ admits meromorphic continuation to $\C^{r}$ with possible polar divisors belonging to the set
        \[
            \{(w\mathbf{s})_{j}-1 = 0:w \in W, 1 \le j \le r\}.
        \]
    \end{theorem}
\section{A Worked Example for \texorpdfstring{$A_{2}$}{A{2}}}
    We will fully work out the Chinta-Gunnells average for the root system of type $A_{2}$. Let $\D = \{\a_{1},\a_{2}\}$ be a base. Then
    \[
        \Phi = \{\pm\a_{1},\pm\a_{2},\pm(\a_{1}+\a_{2})\} \quad \text{and} \quad W = \<\s_{1},\s_{2}:\s_{1}^{2} = \s_{2}^{2} = 1, \s_{1}\s_{2}\s_{1} = \s_{2}\s_{1}\s_{2}\>.
    \]
    As a set is $W = \{1,\s_{1},\s_{2},\s_{1}\s_{2},\s_{2}\s_{1},\s_{0}\}$ where $\s_{0} = \s_{1}\s_{2}\s_{1} = \s_{2}\s_{1}\s_{2}$ is the longest element. The root system and its Dynkin diagram are displayed below:

    \begin{center}
    \begin{tikzpicture}[scale=1.5]
        \draw[thick,-stealth](0,0) to (60:{sqrt(3)/2});
        \draw[thick,-stealth](0,0) to (240:{sqrt(3)/2});
        \draw[thick,-stealth](0,0) to (180:{sqrt(3)/2});
        \draw[thick,-stealth](0,0) to (360:{sqrt(3)/2}) node [right] {$\a_{1}$};
        \draw[thick,-stealth] (0:0) to (120:{sqrt(3)/2}) node [above left] {$\a_{2}$};
        \draw[thick,-stealth] (0:0) to (300:{sqrt(3)/2});
        
        \draw[dotted,] (0:0) to (30:{sqrt(3)/2});
        \draw[dotted,] (0:0) to (30:{-sqrt(3)/2});
        \draw[dotted,] (0:0) to (90:{sqrt(3)/2});
        \draw[dotted,] (0:0) to (90:{-sqrt(3)/2});
        \draw[dotted,] (0:0) to (150:{sqrt(3)/2});
        \draw[dotted,] (0:0) to (150:{-sqrt(3)/2});

        \node at (0,-1.5) {$A_{2}$};
        
        \begin{scope}[shift = {(4,0)}]
            \node at (0,0) { \scalebox{1.5}{\dynkin[labels={\a_{1},\a_{2}},edge length=1cm]A2}};
        \end{scope}
    \end{tikzpicture}
    \end{center}

    The Chinta-Gunnells action on $\C(x_{1},x_{2},u)$ gives rise to the following six terms:
    \begin{align*}
        (1|^{\mathrm{CG}}1)(x_{1},x_{2};u) &= 1, \\
        (1|^{\mathrm{CG}}\s_{1})(x_{1},x_{2}) &= -\frac{1-ux_{1}}{ux_{1}(1-x_{1})}, \\
        (1|^{\mathrm{CG}}\s_{2})(x_{1},x_{2};u) &= -\frac{1-ux_{2}}{ux_{2}(1-x_{2})}, \\
        (1|^{\mathrm{CG}}\s_{1}\s_{2})(x_{1},x_{2};u) &= -\frac{u^{2}x_{1}^{2}x_{2}^{3}+ux_{1}x_{2}^{2}(1-u-ux_{1})+x_{2}(ux_{1}-x_{1}-1)+1}{u^{2}x_{1}x_{2}^{2}(1-x_{2})(1-ux_{1}^{2}x_{2}^{2})}, \\
        (1|^{\mathrm{CG}}\s_{2}\s_{1})(x_{1},x_{2};u) &= -\frac{u^{2}x_{1}^{3}x_{2}^{2}+ux_{1}^{2}x_{2}(1-u-ux_{2})+x_{1}(ux_{2}-x_{2}-1)+1}{u^{2}x_{1}^{2}x_{2}(1-x_{1})(1-ux_{1}^{2}x_{2}^{2})}, \\
        (1|^{\mathrm{CG}}\s_{0})(x_{1},x_{2};u) &= -\frac{u^{3}x_{1}^{2}x_{2}^{2}(1-x_{1}x_{2})+u^{2}x_{1}^{2}x_{2}^{2}(x_{1}+x_{2}-2)+u(2x_{1}x_{2}-x_{1}-x_{2})-x_{1}x_{2}+1}{u^{3}x_{1}^{2}x_{2}^{2}(1-x_{1})(1-x_{2})(1-ux_{1}^{2}x_{2}^{2})}.
    \end{align*}
    The associated $j(\s,x_{1},x_{2})$ factors are
    \begin{align*}
        j(1,x_{1},x_{2}) &= 1, \\
        j(\s_{1},x_{1},x_{2}) &= -ux_{1}^{2}, \\
        j(\s_{2},x_{1},x_{2}) &= -ux_{2}^{2}, \\
        j(\s_{1}\s_{2},x_{1},x_{2}) &= u^{3}x_{1}^{2}x_{2}^{4}, \\
        j(\s_{2}\s_{1},x_{1},x_{2}) &= u^{3}x_{1}^{4}x_{2}^{2}, \\
        j(\s_{0},x_{1},x_{2}) &= -u^{4}x_{1}^{4}x_{2}^{4}.
    \end{align*}
    One can then compute
    \[
        Z_{W}(x_{1},x_{2};u) = \frac{(1-x_{1}x_{2})(1-ux_{1}^{2})(1-ux_{2}^{2})(1-u^{2}x_{1}^{2}x_{2}^{2})}{(1-x_{1})(1-x_{2})(1-ux_{1}^{2}x_{2}^{2})},
    \]
    and so
    \[
        Z_{\Phi}(x_{1},x_{2};u) = \frac{1-x_{1}x_{2}}{(1-x_{1})(1-x_{2})(1-ux_{1}^{2}x_{2}^{2})}.
    \]
    Notice that
    \[
        Z_{\Phi}(q^{1-s_{1}},q^{1-s_{2}};q^{-1}) = \frac{1-q^{2-s_{1}-s_{2}}}{(1-q^{1-s_{1}})(1-q^{1-s_{2}})(1-q^{3-2s_{1}-2s_{2}})},
    \]
    is the global $A_{2}$ multiple Dirichlet series $Z(s_{1},s_{2})$.

    \bibliographystyle{plain}
    \bibliography{Twiss2024Weyl}

\end{document}