\documentclass[12pt,oneside]{article}
\usepackage{import}
%===============================%
%  Packages and basic settings  %
%===============================%
\usepackage[headheight=15pt,rmargin=0.5in,lmargin=0.5in,tmargin=0.75in,bmargin=0.75in]{geometry}
\usepackage{fancyhdr}
\usepackage{imakeidx}
\usepackage{framed}
\usepackage{amssymb}
\usepackage{amsmath}
\usepackage{mathrsfs}
\usepackage{enumitem}
\usepackage{multirow}
\usepackage{hyperref}
\usepackage[capitalise,noabbrev]{cleveref}
\usepackage{appendix}
\usepackage[hyperref,amsthm,amsmath,thref,framed,thmmarks]{ntheorem}
\usepackage{tikz}
\usepackage{tikz-cd}
\usepackage{nomencl}\makenomenclature
\usetikzlibrary{braids,arrows,decorations.markings,calc}

%=======================%
%  Book style settings  %
%=======================%
\pagestyle{fancy}
\fancyhf{}
\fancyhead[L]{\nouppercase{\leftmark}}
\fancyfoot[C]{\thepage}
\setlength\parindent{0pt}
\raggedbottom

%====================================%
%  Theorems, environments & cleveref  %
%====================================%
\theoremstyle{plain}\newtheorem{theorem}{Theorem}[section]
\theoremstyle{nonumberplain}\renewtheorem{theorem*}{Theorem}
\theoremstyle{plain}\newtheorem{proposition}[theorem]{Proposition}
\theoremstyle{nonumberplain}\renewtheorem{proposition*}{Proposition}
\theoremstyle{plain}\newtheorem{corollary}[theorem]{Corollary}
\theoremstyle{nonumberplain}\renewtheorem{corollary*}{Corollary}
\theoremstyle{plain}\newtheorem{lemma}[theorem]{Lemma}
\theoremstyle{nonumberplain}\renewtheorem{lemma*}{Lemma}
\theoremstyle{plain}\newtheorem{conjecture}[theorem]{Conjecture}
\theoremstyle{nonumberplain}\renewtheorem{conjecture*}{Conjecture}
\theoremstyle{plain}\newtheorem{remark}[theorem]{Remark}
\theoremstyle{nonumberplain}\renewtheorem{remark*}{Remark}
\theoremstyle{plain}\newtheorem{problem}[theorem]{Open Problem}
\theoremstyle{nonumberplain}\renewtheorem{problem*}{Open Problem}
\theoremstyle{plain}\newtheorem{heuristic}[theorem]{Heuristic}
\theoremstyle{nonumberplain}\renewtheorem{heuristic*}{Heuristic}
\crefname{conjecture}{Conjecture}{Conjectures}

\newenvironment{stabular}[2][1]
  {\def\arraystretch{#1}\tabular{#2}}
  {\endtabular}

%==================================%
%  Custom commands & environments  %
%==================================%
\newcommand{\legendre}[2]{\left(\frac{#1}{#2}\right)}
\newcommand{\dlegendre}[2]{\displaystyle{\left(\frac{#1}{#2}\right)}}
\newcommand{\tlegendre}[2]{\textstyle{\left(\frac{#1}{#2}\right)}}
\newcommand{\psum}{\sideset{}{'}\sum}
\newcommand{\asum}{\sideset{}{^{\ast}}\sum}
\newcommand{\tmod}[1]{\ (\mathrm{mod}\text{ }#1)}
\renewcommand{\bmod}[1]{\ \left(\mathrm{mod}\text{ }#1\right)}
\newcommand{\xto}[1]{\xrightarrow{#1}}
\newcommand{\xfrom}[1]{\xleftarrow{#1}}
\newcommand{\normal}{\mathrel{\unlhd}}
\newcommand{\mf}{\mathfrak}
\newcommand{\mc}{\mathcal}
\newcommand{\ms}{\mathscr}

\newcommand{\Mat}{\mathrm{Mat}}
\newcommand{\GL}{\mathrm{GL}}
\newcommand{\SL}{\mathrm{SL}}
\newcommand{\PSL}{\mathrm{PSL}}
\renewcommand{\O}{\mathrm{O}}
\newcommand{\SO}{\mathrm{SO}}
\newcommand{\U}{\mathrm{U}}
\newcommand{\Sp}{\mathrm{Sp}}

\newcommand{\N}{\mathbb{N}}
\newcommand{\Z}{\mathbb{Z}}
\newcommand{\Q}{\mathbb{Q}}
\newcommand{\R}{\mathbb{R}}
\newcommand{\C}{\mathbb{C}}
\newcommand{\F}{\mathbb{F}}
\renewcommand{\H}{\mathbb{H}}
\renewcommand{\P}{\mathbb{P}}

\renewcommand{\a}{\alpha}
\renewcommand{\b}{\beta}
\newcommand{\g}{\gamma}
\renewcommand{\d}{\delta}
\newcommand{\z}{\zeta}
\renewcommand{\t}{\theta}
\renewcommand{\i}{\iota}
\renewcommand{\k}{\kappa}
\renewcommand{\l}{\lambda}
\newcommand{\s}{\sigma}
\newcommand{\w}{\omega}

\newcommand{\G}{\Gamma}
\newcommand{\D}{\Delta}
\renewcommand{\L}{\Lambda}
\newcommand{\W}{\Omega}
\newcommand{\scL}{\mathscr{L}}

\newcommand{\e}{\varepsilon}
\newcommand{\vt}{\vartheta}
\newcommand{\vphi}{\varphi}
\newcommand{\emt}{\varnothing}

\newcommand{\x}{\times}
\newcommand{\ox}{\otimes}
\newcommand{\op}{\oplus}
\newcommand{\bigox}{\bigotimes}
\newcommand{\bigop}{\bigoplus}
\newcommand{\del}{\partial}
\newcommand{\<}{\langle}
\renewcommand{\>}{\rangle}
\newcommand{\lf}{\lfloor}
\newcommand{\rf}{\rfloor}
\newcommand{\wtilde}{\widetilde}
\newcommand{\what}{\widehat}
\newcommand{\conj}{\overline}
\newcommand{\cchi}{\conj{\chi}}

\DeclareMathOperator{\id}{\textrm{id}}
\DeclareMathOperator{\sgn}{\mathrm{sgn}}
\DeclareMathOperator{\im}{\mathrm{im}}
\DeclareMathOperator{\rk}{\mathrm{rk}}
\DeclareMathOperator{\adj}{\mathrm{adj}}
\DeclareMathOperator{\tr}{\mathrm{trace}}
\DeclareMathOperator{\nm}{\mathrm{norm}}
\DeclareMathOperator{\disc}{\mathrm{disc}}
\DeclareMathOperator{\ord}{\mathrm{ord}}
\DeclareMathOperator{\sym}{\mathrm{sym}}
\DeclareMathOperator{\ext}{\mathrm{ext}}
\DeclareMathOperator{\Hom}{\mathrm{Hom}}
\DeclareMathOperator{\End}{\mathrm{End}}
\DeclareMathOperator{\Aut}{\mathrm{Aut}}
\DeclareMathOperator{\Tor}{\mathrm{Tor}}
\DeclareMathOperator{\Ann}{\mathrm{Ann}}
\DeclareMathOperator{\Gal}{\mathrm{Gal}}
\DeclareMathOperator{\Trace}{\mathrm{Tr}}
\DeclareMathOperator{\Norm}{\mathrm{N}}
\DeclareMathOperator{\Cl}{\mathrm{Cl}}
\DeclareMathOperator{\Span}{\mathrm{Span}}
\DeclareMathOperator*{\Res}{\mathrm{Res}}
\DeclareMathOperator{\Vol}{\mathrm{Vol}}
\DeclareMathOperator{\Li}{\mathrm{Li}}
\DeclareMathOperator{\Supp}{\mathrm{Supp}}
\renewcommand{\Re}{\mathrm{Re}}
\renewcommand{\Im}{\mathrm{Im}}
\DeclareMathOperator{\Ph}{\mathrm{Ph}}
\DeclareMathOperator{\SC}{\mathrm{SC}}


\newcommand{\GH}{\G\backslash\H}
\newcommand{\GG}{\G_{\infty}\backslash\G}

\newenvironment{psmallmatrix}
  {\left(\begin{smallmatrix}}
  {\end{smallmatrix}\right)}

\newcommand{\smc}[1]{
    \mathchoice
    {{\scriptstyle\mathcal{#1}}}
    {{\scriptstyle\mathcal{#1}}}
    {{\scriptscriptstyle\mathcal{#1}}}
    {\scalebox{0.7}{$\scriptscriptstyle\mathcal{#1}$}}
}

%============%
%  Comments  %
%============%
\newcommand{\todo}[1]{\textcolor{red}{\sf Todo: [#1]}}

%===================%
%  Label reminders  %
%===================%
% [label=(\roman*)]
% [label=(\alph*)]
% [label=(\arabic{enumi})]

%==================%
%  Other settings  %
%==================%
\pgfdeclarelayer{background}
\pgfsetlayers{background,main}
\tikzset{->-/.style={decoration={
  markings,
  mark=at position .5 with {\arrow{>}}},postaction={decorate}}}

%=================%
%  Title & Index  %
%=================%
\title{ANT i}
\author{Henry Twiss}
\date{2025}
\makeindex

\begin{document}
  You will do a series of exercises on Mellin transforms, the Riemann zeta function, and Poisson summation. The exercises build on each other so try to do them in order. \\

  If $f(x)$ is a continuous function on $\R_{+}$ then the \textbf{Mellin transform}\index{Mellin transform} $(\mc{M}f)(s)$ of $f(x)$ is defined by
  \[
    (\mc{M}f)(s) = \int_{(0,\infty)}f(x)x^{s}\,\frac{dx}{x},
  \]
  for $s \in \C$. However, this integral is not guaranteed to converge unless specific conditions upon $f(x)$ are imposed. For example, if $f(x)$ exhibits rapid decay and remains bounded as $x \to 0$ then the integral is locally absolutely uniformly convergent for $\s > 0$. The \textbf{gamma function} $\G(s)$ is defined to be the Mellin transform of $e^{-x}$.
  \begin{enumerate}[label*=(\roman*)]
    \item Write down the definition of the gamma function. Show $\G(1) = 1$ and that $\G(s+1) = s\G(s)$. Use these facts to prove $\G(n) = (n-1)!$.
    \item Consider the function
    \[
      \w(z) = \sum_{n \ge 1}e^{\pi in^{2}z},
    \]
    which is defined for $z \in \H$. Use the Weierstrass $M$-test to show that $\w(z)$ is locally absolutely uniformly convergent for $z \in \H$.
    \item Compute the following Mellin transform:
    \[
      \int_{0}^{\infty}\w(iy)y^{\frac{s}{2}}\,\frac{dy}{y},
    \]
    using the fact that you may freely interchange sums and integrals since $\w(z)$ is locally absolutely uniformly convergent (Fubini-Tonelli theorem). Deduce an integral representation for $\z(s)$.
    \item Using the integral representation derived in part (iii) and the identity
    \begin{equation}\label{equ:1}
      \w\left(\frac{i}{y}\right) = \sqrt{y}\w(iy)+\frac{\sqrt{y}}{2}-\frac{1}{2},
    \end{equation}
    derive the following integral representation:
    \[
        \z(s) = \frac{\pi^{\frac{s}{2}}}{\G\left(\frac{s}{2}\right)}\left[-\frac{1}{s(1-s)}+\int_{1}^{\infty}\w(iy)y^{\frac{1-s}{2}}\,\frac{dy}{y}+\int_{1}^{\infty}\w(iy)y^{\frac{s}{2}}\,\frac{dy}{y}\right].
   \]
   Deduce that $\z(s)$ admits analytic continuation to $\C$.
   \item Using part (iv), derive the following functional equation:
   \[
      \frac{\G\left(\frac{s}{2}\right)}{\pi^{\frac{s}{2}}}\z(s) = \frac{\G\left(\frac{1-s}{2}\right)}{\pi^{\frac{1-s}{2}}}\z(1-s).
    \]
    Compute $\Res_{s = 1}\z(s)$ using the functional equation and the fact that
    \[
      \Res_{s = 1}\z(s) = \lim_{s \to 1}(1-s)\z(s).
    \]
  \end{enumerate}

We now introduce Fourier transforms and Fourier coefficients. Suppose $f(x)$ is absolutely integrable on $\R$. The \textbf{Fourier transform}\index{Fourier transform} $(\mc{F}f)(\xi)$ of $f(x)$ is defined by
\[
  (\mc{F}f)(\xi) = \int_{\R}f(x)e^{-2\pi i\xi x}\,dx,
\]
for $\xi \in \R$. This integral is absolutely convergent precisely because $f(x)$ is absolutely integrable on $\R$. The Fourier transform is intimately related to periodic functions. If $f(x)$ is $1$-periodic and integrable on $[0,1]$ then we define the $n$-th \textbf{Fourier coefficient}\index{Fourier coefficient} $\hat{f}(n)$ of $f(x)$ by
\[
  \hat{f}(n) = \int_{0}^{1}f(x)e^{-2\pi inx}\,dx.
\]
The \textbf{Fourier series}\index{Fourier series} of $f(x)$ is defined by the series
\[
  \sum_{n \in \Z^{n}}\hat{f}(n)e^{2\pi inx}.
\]
There is the question of whether the Fourier series of $f(x)$ converges at all and if so does it even converge to $f(x)$ itself. Under reasonable conditions this is possible:

\begin{proposition}
  If $f(x)$ is smooth and $1$-periodic then it converges uniformly to its Fourier series.
\end{proposition}

The link between the Fourier transform and Fourier series is given by the \textbf{Poisson summation formula}\index{Poisson summation formula}:

\begin{theorem*}[Poisson summation formula]
  Suppose $f(x)$ is absolutely integrable on $\R$, and the function
  \[
    F(x) = \sum_{n \in \Z}f(x+n),
  \]
  is locally absolutely uniformly convergent and smooth. Then
  \[
    \sum_{n \in \Z}f(x+n) = \sum_{t \in \Z}(\mc{F}f)(t)e^{2\pi itx},
  \]
  and
  \[
    \sum_{n \in \Z}f(n) = \sum_{t \in \Z}(\mc{F}f)(t).
  \]
\end{theorem*}

\begin{enumerate}[label*=(\roman*)]
  \item Prove the Poisson summation formula. (\textit{hint}: can you compute the Fourier coefficients of $F(x)$? What is it's Fourier series?)
  \item There are two ways of building a periodic function from an absolutely integrable function on $\R$. What does the Poisson summation formula say about these periodic functions?
\end{enumerate}

Now we develop some basic properties of Fourier transforms. In practical settings, we need a class of functions $f(x)$ for which the assumptions of the Poisson summation formula hold. We say that $f(x)$ is of \textbf{Schwarz class}\index{Schwarz class} if $f \in C^{\infty}(\R)$ and $f(x)$ along with all of its partial derivatives have \textbf{rapid decay}. This means $f(x) = o(|x|^{-n})$ for all $n \ge 0$. If $f(x)$ is of Schwarz class, the rapid decay implies that $f(x)$ and all of its derivatives are absolutely integrable over $\R$. Moreover, this also implies that $F(x) = \sum_{n \in \Z}f(x+n)$ and all of its derivatives are locally absolutely uniformly convergent by the Weierstrass $M$-test. The uniform limit theorem then implies $F(x)$ is smooth and thus the conditions of the Poisson summation formula are satisfied. We will now derive some properties of the Fourier transform including a case specific to Schwarz class functions: \\

Let $f(x)$ and $g(x)$ be absolutely integrable on $\R$. Prove the following:
\begin{enumerate}[label*=(\roman*)]
  \item For any $\a,\b \in \R$, we have
  \[
    (\mc{F}(\a f+\b g))(\xi) = \a(\mc{F}f)(\xi)+\b(\mc{F}g)(\xi).
  \]
  \item If $g(x) = f(x+\a)$ for any $\a \in \R$ then
  \[
    (\mc{F}g)(\xi) = e^{2\pi i\<\a,\xi\>}(\mc{F}f)(\xi).
  \]
  \item If $g(x) = f(\a x)$ for any $\a \in \R^{\ast}$ then
  \[
    (\mc{F}g)(\xi) = \frac{1}{|\a|}(\mc{F}f)\left(\frac{\xi}{\a}\right).
  \]
  \item If $f(x)$ is of Schwarz class and $g(x) = \frac{\del}{\del x}^{k}f(x)$ for some $k \ge 0$ then
  \[
    (\mc{F}g)(\xi) = (2\pi i\xi)^{k}(\mc{F}f).
  \]
  \item Show that
    \[
      (\mc{F}f)(\z) = \frac{e^{-\frac{\pi\z^{2}}{2\a}}}{\sqrt{2\a}}.
    \]
    In particular, what function is its own Fourier transform?
\end{enumerate}

You will now derive \cref{equ:1} using the theory of Fourier transforms and Poisson summation. Consider the function
\[
  \vt(z) = \sum_{n \in \Z}e^{2\pi in^{2}z},
\]
defined for $\z \in \H$. We will use this function to prove \cref{equ:1}:

\begin{enumerate}[label*=(\roman*)]
  \item Apply the Poisson summation formula to $\vt(z)$ to prove
  \[
    \vt(z) = \frac{1}{\sqrt{-2iz}}\vt\left(-\frac{1}{4z}\right).
  \]
  You may use the identity theorem to assume $z = iy$ for $y > 0$.
  \item Show that 
  \[
    \w(z) = \frac{\vt\left(\frac{z}{2}\right)-1}{2}.
  \]
  \item Deduce \cref{equ:1}.
\end{enumerate}

\end{document}
