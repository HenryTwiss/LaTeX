\documentclass[12pt,reqno,oneside]{amsart}
\usepackage{import}
%===============================%
%  Packages and basic settings  %
%===============================%
\usepackage[headheight=15pt,rmargin=0.5in,lmargin=0.5in,tmargin=0.75in,bmargin=0.75in]{geometry}
\usepackage{imakeidx}
\usepackage{framed}
\usepackage{amssymb}
\usepackage{amsmath}
\usepackage{mathrsfs}
\usepackage{enumitem}
\usepackage{hyperref}
\usepackage{appendix}
\usepackage[capitalise,noabbrev]{cleveref}
\usepackage{tikz}
\usepackage{tikz-cd}
\usepackage{nomencl}\makenomenclature
\usetikzlibrary{braids,arrows,decorations.markings,calc}

%====================================%
%  Theorems, environments & cleveref  %
%====================================%
\newtheorem{theorem}{Theorem}[section]
\newtheorem{proposition}{Proposition}[section]
\newtheorem{corollary}{Corollary}[section]
\newtheorem{lemma}{Lemma}[section]
\newtheorem{conjecture}{Conjecture}[section]
\newtheorem{remark}{Remark}[section]

\newenvironment{stabular}[2][1]
  {\def\arraystretch{#1}\tabular{#2}}
  {\endtabular}

%==================================%
%  Custom commands & environments  %
%==================================%
\newcommand{\legendre}[2]{\left(\frac{#1}{#2}\right)}
\newcommand{\dlegendre}[2]{\displaystyle{\left(\frac{#1}{#2}\right)}}
\newcommand{\tlegendre}[2]{\textstyle{\left(\frac{#1}{#2}\right)}}
\newcommand{\psum}{\sideset{}{'}\sum}
\newcommand{\asum}{\sideset{}{^{\ast}}\sum}
\newcommand{\tmod}[1]{\ \left(\text{mod }#1\right)}
\newcommand{\xto}[1]{\xrightarrow{#1}}
\newcommand{\xfrom}[1]{\xleftarrow{#1}}
\newcommand{\normal}{\mathrel{\unlhd}}
\newcommand{\mf}{\mathfrak}
\newcommand{\mc}{\mathcal}
\newcommand{\ms}{\mathscr}

\newcommand{\Mat}{\mathrm{Mat}}
\newcommand{\GL}{\mathrm{GL}}
\newcommand{\SL}{\mathrm{SL}}
\newcommand{\PSL}{\mathrm{PSL}}
\renewcommand{\O}{\mathrm{O}}
\newcommand{\SO}{\mathrm{SO}}
\newcommand{\U}{\mathrm{U}}
\newcommand{\Sp}{\mathrm{Sp}}

\newcommand{\N}{\mathbb{N}}
\newcommand{\Z}{\mathbb{Z}}
\newcommand{\Q}{\mathbb{Q}}
\newcommand{\R}{\mathbb{R}}
\newcommand{\C}{\mathbb{C}}
\newcommand{\F}{\mathbb{F}}
\renewcommand{\H}{\mathbb{H}}
\renewcommand{\P}{\mathbb{P}}

\renewcommand{\a}{\alpha}
\renewcommand{\b}{\beta}
\newcommand{\g}{\gamma}
\renewcommand{\d}{\delta}
\newcommand{\z}{\zeta}
\renewcommand{\t}{\theta}
\renewcommand{\i}{\iota}
\renewcommand{\k}{\kappa}
\renewcommand{\l}{\lambda}
\newcommand{\s}{\sigma}
\newcommand{\w}{\omega}

\newcommand{\G}{\Gamma}
\newcommand{\D}{\Delta}
\renewcommand{\L}{\Lambda}
\newcommand{\W}{\Omega}

\newcommand{\e}{\varepsilon}
\newcommand{\vt}{\vartheta}
\newcommand{\vphi}{\varphi}
\newcommand{\emt}{\varnothing}

\newcommand{\x}{\times}
\newcommand{\ox}{\otimes}
\newcommand{\op}{\oplus}
\newcommand{\bigox}{\bigotimes}
\newcommand{\bigop}{\bigoplus}
\newcommand{\del}{\partial}
\newcommand{\<}{\langle}
\renewcommand{\>}{\rangle}
\newcommand{\lf}{\lfloor}
\newcommand{\rf}{\rfloor}
\newcommand{\wtilde}{\widetilde}
\newcommand{\what}{\widehat}
\newcommand{\conj}{\overline}
\newcommand{\cchi}{\conj{\chi}}

\DeclareMathOperator{\id}{\textrm{id}}
\DeclareMathOperator{\sgn}{\mathrm{sgn}}
\DeclareMathOperator{\im}{\mathrm{im}}
\DeclareMathOperator{\rk}{\mathrm{rk}}
\DeclareMathOperator{\tr}{\mathrm{trace}}
\DeclareMathOperator{\nm}{\mathrm{norm}}
\DeclareMathOperator{\ord}{\mathrm{ord}}
\DeclareMathOperator{\Hom}{\mathrm{Hom}}
\DeclareMathOperator{\End}{\mathrm{End}}
\DeclareMathOperator{\Aut}{\mathrm{Aut}}
\DeclareMathOperator{\Tor}{\mathrm{Tor}}
\DeclareMathOperator{\Ann}{\mathrm{Ann}}
\DeclareMathOperator{\Gal}{\mathrm{Gal}}
\DeclareMathOperator{\Trace}{\mathrm{Trace}}
\DeclareMathOperator{\Norm}{\mathrm{Norm}}
\DeclareMathOperator{\Span}{\mathrm{Span}}
\DeclareMathOperator*{\Res}{\mathrm{Res}}
\DeclareMathOperator{\Vol}{\mathrm{Vol}}
\DeclareMathOperator{\Li}{\mathrm{Li}}
\renewcommand{\Re}{\mathrm{Re}}
\renewcommand{\Im}{\mathrm{Im}}

\newcommand{\GH}{\G\backslash\H}
\newcommand{\GG}{\G_{\infty}\backslash\G}

\newenvironment{psmallmatrix}
  {\left(\begin{smallmatrix}}
  {\end{smallmatrix}\right)}

%============%
%  Comments  %
%============%
\newcommand{\todo}[1]{\textcolor{red}{\sf Todo: [#1]}}

%===================%
%  Label reminders  %
%===================%
% [label=(\roman*)]
% [label=(\alph*)]
% [label=(\arabic{enumi})]

%==================%
%  Other settings  %
%==================%
\pgfdeclarelayer{background}
\pgfsetlayers{background,main}
\tikzset{->-/.style={decoration={
  markings,
  mark=at position .5 with {\arrow{>}}},postaction={decorate}}}

%=================%
%  Title & Index  %
%=================%
\title{Comments on coefficients of Maass forms
and the Siegel zero \& An effective zero-free region}
\author{Henry Twiss}
\date{\today}
\makeindex

\begin{document}

\begin{abstract}
    We present a condensed exposition on \cite{HL} and its appendix \cite{GHL}. All of the notation will be as in \cite{HL}.
\end{abstract}

\maketitle

\section{Main Theorems}
    Let $f$ be a Maass form that is a newform for $\G_{0}(N)$, with eigenvalue $\l$, and character $\chi$, analytically normalized so that $\<f,f\> = 1$ where $\<\cdot,\cdot\>$ is the Petersson inner product. Denote the Fourier coefficients of $f$ by $\rho(n)$ and the Hecke eigenvalues by $a(n)$. The primary aim of \cite{HL} is to establish an upper bound for $\rho(1)$ in terms of $\l$ and $N$. An upper bound for $\rho(1)$ induces an upper bound for $\rho(n)$ because
    \[
        \rho(n) = \pm\rho(-n) \quad \text{and} \quad \rho(n) = a(n)\rho(1),
    \]
    for all $n \ge 1$ since $f$ is a newform. The general stragety to estimate $\rho(1)$ is as follows: consider the Rankin-Selberg convolution
    \[
        L(s,f \x f) = \z(2s)\sum_{n \ge 1}\frac{|a(n)|^{2}}{n^{s}}.
    \]
    Then $L(s,f \x f)$ admits an Euler product and meromorphic continuation to $\C$ with a simple pole at $s = 1$. The residue is given by
    \begin{equation}\label{equ:key_equation_1}
        \Res_{s = 1}L(s,f \x f) = \frac{2\pi}{3}|\rho(1)|^{-2}.
    \end{equation}
    On the other hand, to any newform $f$, there is an associated $\GL(3)$ form $F$ called the \textbf{adjoint square lift} with Fourier coefficients $a(m,n)$ and $L$-function
    \[
        L(s,F) = \sum_{n \ge 1}\frac{a(1,n)}{n^{s}}.
    \]
    The existance of $F$ and its properties are establish in \cite{GJ}. Now $\z(s)L(s,F)$ admits an Euler product and the local $p$ factors of $\z(s)L(s,F)$ and $L(s,f \x f)$ agree provided $p \nmid N$. Therefore
    \begin{equation}\label{equ:key_equation_2}
        L(s,f \x f) = \z(s)L_{N}(s)L(s,F),
    \end{equation}
    where $L_{N}(s)$ is a Dirichlet polynomial supported on the primes dividing $N$. Moreover, $L(s,F)$ is entire and $L(1,F) \neq 0$. It follows from \cref{equ:key_equation_2} that
    \begin{equation}\label{equ:key_equation_3}
        \Res_{s = 1}L(s,f \x f) = L_{N}(1)L(1,F).
    \end{equation}
    Now one can deduce from \cite{GJ} that the growth of $L_{N}(1)$ is minor in the sense that
    \begin{equation}\label{equ:key_equation_4}
        N^{-\e} \ll_{\e} L_{N}(1) \ll_{\e} N^{\e},
    \end{equation}
    for small $\e > 0$. Combining \cref{equ:key_equation_1,equ:key_equation_3,equ:key_equation_4} yields
    \begin{equation}\label{equ:key_equation_5}
        N^{-\e}L(1,F)^{-1} \ll_{\e} |\rho(1)|^{2} \ll_{\e} N^{\e}L(1,F)^{-1}.
    \end{equation}
    We see from \cref{equ:key_equation_5} that finding effective upper bounds for $|\rho(1)|$ follows from effective lower bounds for $L(s,F)$ at the special value $s = 1$.

    This situation largely mimics the classical class number problem for quadratic number fields of discriminant $D < 0$. Indeed, let $h(D)$ be the ideal class number of $\Q(\sqrt{D})$ and let $\chi_{D}$ be the aossciated quadratic character. Then it is known that
    \[
        L(1,\chi_{D}) = \frac{2\pi}{\w_{D}\sqrt{|D|}}h(D),
    \]
    where $\w_{D}$ is the number of roots of unity in $\Q(\sqrt{D})$. It follows that estimates for $L(1,\chi_{D})$ induce corresponding estimates for $h(D)$. Via Siegel's theorem, estimates for $L(1,\chi_{D})$ are intimately related to existance of Siegel zeros for $L(1,\chi_{D})$. That is, how close real zeros of $L(s,\chi_{D})$ can be to $s = 1$. In our setting, $\rho(1)$ is playing the role of $h(D)$ and $L(s,F)$ is playing the role of $L(s,\chi_{D})$. Indeed, \cite{HL} shows that lower bounds for $L(1,F)$ are closely related to the existance or non-existance of \textbf{Siegel zeros} for $L(s,F)$. That is, real zeros of $L(s,F)$ close to $s = 1$. The main results of \cite{HL} are the following:

    \begin{theorem}\label{thm:main_theorem_1}
        Suppose there exists a constant $c$ such that $L(s,F)$ has no real zeros in the range
        \[
            1-\frac{c}{\log(\l N+1)} < s < 1.
        \]
        Then there are effective constants $c_{1}$ and $c_{2}$, depending only on $c$, such that
        \[
            L(1,F) \ge \frac{c_{1}}{\log(\l N+1)},
        \]
        and
        \[
            |\rho(1)|^{2} \le c_{2}\log(\l N+1).
        \]
    \end{theorem}

    \cref{thm:main_theorem_1} gives an upper bound for $\rho(1)$ in the case that the Siegel zero of $L(s,F)$ does not exist. The following result gives an estimate that is unconditional on the existance of a Siegel zero:

    \begin{theorem}\label{thm:main_theorem_2}
        For any $\e > 0$, there exsits an effective constant $c(\e)$ so that the inequality
        \[
            L(1,F) \ge c(\e)(\l N)^{\e},
        \]
        holds for all $F$ with at most one exception.
    \end{theorem}

    The second statement in \cref{thm:main_theorem_2} shows that unconditionally the existance of Siegel zeros are rare. In particular, \cref{thm:main_theorem_2} implies $L(1,F) \gg (\l N)^{\e}$ with an inneffective constant. Combining with \cref{equ:key_equation_5} gives the following corollary:

    \begin{corollary}
        Let $f$ be a newform for $\G_{0}(N)$ with eigenvalue $\l$ and analytically normalized so that $\<f,f\> = 1$. Then for any $\e > 0$,
        \[
            \rho(1) \ll_{\e} (\l N)^{\e}.
        \]
    \end{corollary}
\section*{The Appendix}
    About a year after \cite{HL} was circulated, an appendix (see \cite{GHL}) was written. This occured because through some discussions it became apparent how eliminate the existance of Siegel zeros of $L(s,F)$ for many $F$. This boils down to an additional factorization of $L(s,F \x F)$. In particular, \cref{thm:main_theorem_1} is true unconditionally as long as $f$ is not a lift from $\GL(1)$. Even if these forms are included, the result still holds in the $\l$-aspect but either the constant must be weakened in the $N$-aspect. Explicitely:

    \begin{theorem}\label{thm:main_theorem_3}
        Let $f$ be a Maass form that is a newform for $\G_{0}(N)$, with eigenvalue $\l$, and character $\chi$, analytically normalized so that $\<f,f\> = 1$. Let $\rho(1)$ denote the first Fourier coefficient of $f$ and let $F$ be the adjoint square lift of $f$ to $\GL(3)$. The following are true
        \begin{enumerate}[label=(\roman*)]
            \item If $f$ is not a lift from $\GL(1)$, then there exists effective constants $c_{1}$ and $c_{2}$ such that
            \[
                L(1,F) \ge \frac{c_{1}}{\log(\l N+1)},
            \]
            and
            \[
                |\rho(1)|^{2} \le c_{2}\log(\l N+1).
            \]
            \item If $f$ is a lift from $\GL(1)$, then there exists effective constants $c_{3}$ and $c_{4}$ such that
            \[
                L(1,F) \ge c_{3}\min\left(\frac{1}{\sqrt{N}},\frac{1}{\log(\l N+1)}\right),
            \]
            and
            \[
                |\rho(1)|^{2} \le c_{4}\max\left(\sqrt{N},\log(\l N+1)\right).
            \]
            Moreover, $\sqrt{N}$ can be replaced by $N^{\e}$, for any $\e > 0$, at the cost of making $c_{3}$ and $c_{4}$ inneffective depending on $\e$.
        \end{enumerate}
    \end{theorem}

    The proof of \cref{thm:main_theorem_3} naturally breaks into two cases (i) and (ii). In the first case, one can eliminate the existance of Siegel zeros for $L(s,F)$. In the second case, if $f$ is a lift from $\GL(1)$, the $L$-function $L(s,F)$ is divisible by a quadratic Dirichlet $L$-series which may exhibit a Siegel zero. Hence, $L(s,F)$ may have a Siegel zero induced from one for a quadratic Dirichlet $L$-function.

    \begin{remark}
        In many instances there are no forms $f$ which are lifts from $\GL(1)$. For example, on $\PSL_{2}(\Z)$ or $\G_{0}(N)$ for prime $N$ and trivial character.
    \end{remark}

    We being with a lemma:

    \begin{lemma}\label{lem:product_lemma}
        Let $\vphi(s)$ be a Dirichlet series with non-negative coefficients and absolutely convergent for $\Re(s) > 1$. Also suppose that $\vphi(s)$ admits an Euller product so that $\vphi(s) \neq 0$ for $\Re(s) > 1$, and the logarthmic derivative of $\vphi(s)$ is negative for real $s > 1$. Let $\vphi(s)$ have a pole of order $m$ at $s = 1$ and set
        \[
            \L(s) = s^{m}(1-s)^{m}G(s)\vphi(s),
        \]
        is entire of order $1$ satisfying the functional equation
        \[
            \L(s) = \L(1-s),
        \]
        where
        \[
            G(s) = D^{s}\prod_{1 \le i \le l}\G\left(\frac{s+c_{l}}{2}\right),
        \]
        for some constants $c_{l}$ and an integer $D > 1$. Then there exists an effective constant $c$, depending only on $l$ and $m$, such that $\vphi(s)$ has at most $m$ real zeros in the range
        \[
            1-\frac{c}{\log(M)} < s < 1,
        \]
        where $M = 1+D\max_{1 \le i \le l}\{|c_{i}|\}$.
    \end{lemma}
    \begin{proof}
        Since $\L(s)$ is entire and of order $1$, it admits the Hadamard factorization
        \[
            \L(s) = e^{A+Bs}\prod_{\rho}\left(1-\frac{s}{\rho}\right)e^{\frac{s}{\rho}},
        \]
        where the product is taken over the zeros $\rho$ of $\L(s)$. Taking the logarthmic derivative of this expression gives
        \[
            \frac{\L'}{\L}(s) = B+\sum_{\rho}\frac{1}{s-\rho}+\frac{1}{\rho},
        \]
        and using the functional equation (exactly as in the case for $\z(s)$) we see that
        \[
            B = -\sum_{\rho}\frac{1}{\rho}.
        \]
        On the other hand, the defintion of $\L(s)$ gives
        \[
            \frac{\L'}{\L}(s) = \frac{m}{s}+\frac{m}{s-1}+\frac{G'}{G}(s)+\frac{\vphi'}{\vphi}(s).
        \]
        Combing our work results in the identity
        \[
            \sum_{\rho}\frac{1}{s-\rho} = \frac{m}{s}+\frac{m}{s-1}+\frac{G'}{G}(s)+\frac{\vphi'}{\vphi}(s).
        \]
        By assumption $\frac{\vphi'}{\vphi}(s) < 0$ if $s$ is real and $s > 1$. Moreover, if we write
        \[
            \sum_{\rho}\frac{1}{s-\rho} = \asum_{\rho}\left(\frac{1}{s-\rho}+\frac{1}{s-\conj{\rho}}\right),
        \]
        where the $\ast$ indicates that we are summing over $\rho$ and not $\conj{\rho}$, then each term in the latter sum is real and positive. So upon removing $\frac{\vphi'}{\vphi}(s)$ and all of the terms $\rho$ except for those where $\rho = \b$ for some positive real root $\b \ge 1-\frac{c}{\log(M)}$, there exists an effective constant $c_{1}$ such that
        \[
            \sum_{\b}\frac{1}{s-\b} \le \frac{m}{s-1}+c_{1}\log(M).
        \]
        Setting $s = 1+\frac{\d}{\log(M)}$ with $\d < \frac{1}{c_{1}}$ and taking $c$ small enough compared to $\d$, we obtain a contradiction if there are least $m+1$ roots $\b$ in the sum.
    \end{proof}

    Now we can begin the work to prove \cref{thm:main_theorem_3}. Similar to the proof of Siegel's theroem, we will prove the claim from estimates of a Dirichlet series $\vphi(s)$ that is a multiple of two $L$-functions $L(s,F_{1})$ and $L(s,F_{2})$ corresponding to two Maass forms $f_{1}$ and $f_{2}$ and has non-negative coefficeints. As presented in \cite{HL}, this Dirichlet series is
    \[
        \vphi(s) = \z(s)L(s,F_{1})L(s,F_{2})L(s,F_{1} \x F_{2}).
    \]
    Analgous to Siegel's theorem, $L(s,F_{1})$ plays the role of a quadratic Dirichlet $L$-series with a possible Siegel zero, $L(s,F_{2})$ plays the role of quadratic Dirichlet $L$-series with another character, and $L(s,F_{1} \x F_{2})$ plays the role of the Dirichlet $L$-series formed with the product of the two characters. It was proved in \cite{HL} that if $F_{1} \neq F_{2}$, then $\vphi(s)$ has non-negative coefficients, admits meromorphic continuation to $\C$ with a simple pole at $s = 1$, posses a functional equation of shape $s \to 1-s$, and has polynomial growth in bounded vertical strips. By \cref{lem:product_lemma} it follows that $\vphi(s)$ has at most a single real zero close to $1$.
    
    In \cite{GHL}, they take $F_{1} = F_{2} = F$ so that
    \[
        \vphi(s) = \z(s)L(s,F)^{2}L(s,F \x F).
    \]
    It turns out that $\vphi(s)$ retains all of the properties in the generic case when $F_{1} \neq F_{2}$ except for the order of the pole. To compute the order of the pole, it follows from inspecting local factors that $L(s,F \x F)$ decomposes as
    \[
        L(s,F \x F) = L(s,F)L(s,F,\vee^{2}),
    \]
    where $L(s,F,\vee^{2})$ is the symmetric square $L$-function of $F$. Therefore,
    \[
        \vphi(s) = \z(s)L(s,F)^{3}L(s,F,\vee^{2}).
    \]
    The remainder of the argument breaks into cases depending on if $f$ is a lift from $\GL(1)$ or not:
    \begin{enumerate}[label=(\roman*)]
        \item Suppose $f$ is not a lift from $\GL(1)$. It was shown in \cite{BG} that $L(s,F,\vee^{2})$ has a simple pole at $s = 1$ and is analytic elsewhere. As $L(1,F) \neq 0$, it follows that $\vphi(s)$ has a double pole at $s = 1$. Moreover, any zero of $F(s,F)$ will be a zero of order at least $3$ for $\vphi(s)$. Applying \cref{lem:product_lemma} (where $m = 2$ and $M = \l N+1$) it follows that $\vphi(s)$ cannot have a triple zero within $\frac{c}{\log(M)}$ of $1$ because it only has a double pole. In short, the lemme implies the existance of an effective constant $c$ such that $L(s,F)$ has no real zeros in the interval
        \[
            1-\frac{c}{\log(\l N+1)} < s < 1.
        \]
        This eliminates the existance of Siegel zeros for such $F$ and the result then follows from \cref{thm:main_theorem_1}.
        \item Now suppose $f$ is a lift from $\GL(1)$. This means that $L(s,f) = L(s,\psi,K)$ where $K$ is a quadratic number field and $\psi$ is a Hecke character defined over $K$. Moreover, $L(s,F)$ factors as
        \[
            L(s,F) = L(s,\psi_{K})L(s,\psi^{2}(\psi^{-1} \circ N_{K/\Q}),K),
        \]
        where $L(s,\psi_{K})$ is the quadratic Dirichlet $L$-function associated to $K$. It follows that an Siegel zero for $L(s,\psi_{K})$ or $L(s,\psi^{2}(\psi^{-1} \circ N_{K/\Q}),K)$ induces a Siegel zero for $L(s,F)$. Actally, if $\psi^{2}$ is not trivial, then $L(s,\psi^{2}(\psi^{-1} \circ N_{K/\Q}),K)$ cannot have a Siegel zero. In any case, the level of $L(s,F)$ is bounded above by $N$ and then one can prove the result using the effective bound $1-\b \gg \frac{1}{\sqrt{N}}$ or the inneffective bound $1-\b \gg N^{-\e}$ coming from Siegel's theorem.
    \end{enumerate}

\begin{thebibliography}{99}
    \bibitem{HL}
    Hoffstein, J., \& Lockhart, P. (1994). Coefficients of Maass forms and the Siegel zero. Annals of Mathematics, 140(1), 161-176.

    \bibitem{GHL}
    Goldfeld, D., Hoffstein, J., \& Lieman, D. (1994). An effective zero-free region. Annals of Mathematics, 140(1), 177-181.

    \bibitem{GJ}
    Gelbart, S., \& Jacquet, H. (1978). A relation between automorphic representations of ${\rm GL}(2) $ and ${\rm GL}(3) $. In Annales scientifiques de l'École normale supérieure (Vol. 11, No. 4, pp. 471-542).

    \bibitem{BG}
    Bump, D., \& Ginzburg, D. (1992). Symmetric square L-functions on GL (r). Annals of mathematics, 136(1), 137-205.
\end{thebibliography}

\end{document}