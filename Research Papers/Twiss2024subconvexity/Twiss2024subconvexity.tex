\documentclass[12pt,reqno,oneside]{amsart}
\usepackage{import}
%===============================%
%  Packages and basic settings  %
%===============================%
\usepackage[headheight=15pt,rmargin=0.5in,lmargin=0.5in,tmargin=0.75in,bmargin=0.75in]{geometry}
\usepackage{imakeidx}
\usepackage{framed}
\usepackage{amssymb}
\usepackage{amsmath}
\usepackage{mathrsfs}
\usepackage{enumitem}
\usepackage{hyperref}
\usepackage{appendix}
\usepackage[capitalise,noabbrev]{cleveref}
\usepackage{tikz}
\usepackage{tikz-cd}
\usepackage{nomencl}\makenomenclature
\usetikzlibrary{braids,arrows,decorations.markings,calc}

%====================================%
%  Theorems, environments & cleveref  %
%====================================%
\newtheorem{theorem}{Theorem}[section]
\newtheorem{proposition}{Proposition}[section]
\newtheorem{corollary}{Corollary}[section]
\newtheorem{lemma}{Lemma}[section]
\newtheorem{conjecture}{Conjecture}[section]
\newtheorem{remark}{Remark}[section]

\newenvironment{stabular}[2][1]
  {\def\arraystretch{#1}\tabular{#2}}
  {\endtabular}

%==================================%
%  Custom commands & environments  %
%==================================%
\newcommand{\legendre}[2]{\left(\frac{#1}{#2}\right)}
\newcommand{\dlegendre}[2]{\displaystyle{\left(\frac{#1}{#2}\right)}}
\newcommand{\tlegendre}[2]{\textstyle{\left(\frac{#1}{#2}\right)}}
\newcommand{\psum}{\sideset{}{'}\sum}
\newcommand{\asum}{\sideset{}{^{\ast}}\sum}
\newcommand{\tmod}[1]{\ \left(\text{mod }#1\right)}
\newcommand{\xto}[1]{\xrightarrow{#1}}
\newcommand{\xfrom}[1]{\xleftarrow{#1}}
\newcommand{\normal}{\mathrel{\unlhd}}
\newcommand{\mf}{\mathfrak}
\newcommand{\mc}{\mathcal}
\newcommand{\ms}{\mathscr}

\newcommand{\Mat}{\mathrm{Mat}}
\newcommand{\GL}{\mathrm{GL}}
\newcommand{\SL}{\mathrm{SL}}
\newcommand{\PSL}{\mathrm{PSL}}
\renewcommand{\O}{\mathrm{O}}
\newcommand{\SO}{\mathrm{SO}}
\newcommand{\U}{\mathrm{U}}
\newcommand{\Sp}{\mathrm{Sp}}

\newcommand{\N}{\mathbb{N}}
\newcommand{\Z}{\mathbb{Z}}
\newcommand{\Q}{\mathbb{Q}}
\newcommand{\R}{\mathbb{R}}
\newcommand{\C}{\mathbb{C}}
\newcommand{\F}{\mathbb{F}}
\renewcommand{\H}{\mathbb{H}}
\renewcommand{\P}{\mathbb{P}}

\renewcommand{\a}{\alpha}
\renewcommand{\b}{\beta}
\newcommand{\g}{\gamma}
\renewcommand{\d}{\delta}
\newcommand{\z}{\zeta}
\renewcommand{\t}{\theta}
\renewcommand{\i}{\iota}
\renewcommand{\k}{\kappa}
\renewcommand{\l}{\lambda}
\newcommand{\s}{\sigma}
\newcommand{\w}{\omega}

\newcommand{\G}{\Gamma}
\newcommand{\D}{\Delta}
\renewcommand{\L}{\Lambda}
\newcommand{\W}{\Omega}

\newcommand{\e}{\varepsilon}
\newcommand{\vt}{\vartheta}
\newcommand{\vphi}{\varphi}
\newcommand{\emt}{\varnothing}

\newcommand{\x}{\times}
\newcommand{\ox}{\otimes}
\newcommand{\op}{\oplus}
\newcommand{\bigox}{\bigotimes}
\newcommand{\bigop}{\bigoplus}
\newcommand{\del}{\partial}
\newcommand{\<}{\langle}
\renewcommand{\>}{\rangle}
\newcommand{\lf}{\lfloor}
\newcommand{\rf}{\rfloor}
\newcommand{\wtilde}{\widetilde}
\newcommand{\what}{\widehat}
\newcommand{\conj}{\overline}
\newcommand{\cchi}{\conj{\chi}}

\DeclareMathOperator{\id}{\textrm{id}}
\DeclareMathOperator{\sgn}{\mathrm{sgn}}
\DeclareMathOperator{\im}{\mathrm{im}}
\DeclareMathOperator{\rk}{\mathrm{rk}}
\DeclareMathOperator{\tr}{\mathrm{trace}}
\DeclareMathOperator{\nm}{\mathrm{norm}}
\DeclareMathOperator{\ord}{\mathrm{ord}}
\DeclareMathOperator{\Hom}{\mathrm{Hom}}
\DeclareMathOperator{\End}{\mathrm{End}}
\DeclareMathOperator{\Aut}{\mathrm{Aut}}
\DeclareMathOperator{\Tor}{\mathrm{Tor}}
\DeclareMathOperator{\Ann}{\mathrm{Ann}}
\DeclareMathOperator{\Gal}{\mathrm{Gal}}
\DeclareMathOperator{\Trace}{\mathrm{Trace}}
\DeclareMathOperator{\Norm}{\mathrm{Norm}}
\DeclareMathOperator{\Span}{\mathrm{Span}}
\DeclareMathOperator*{\Res}{\mathrm{Res}}
\DeclareMathOperator{\Vol}{\mathrm{Vol}}
\DeclareMathOperator{\Li}{\mathrm{Li}}
\renewcommand{\Re}{\mathrm{Re}}
\renewcommand{\Im}{\mathrm{Im}}

\newcommand{\GH}{\G\backslash\H}
\newcommand{\GG}{\G_{\infty}\backslash\G}

\newenvironment{psmallmatrix}
  {\left(\begin{smallmatrix}}
  {\end{smallmatrix}\right)}

%============%
%  Comments  %
%============%
\newcommand{\todo}[1]{\textcolor{red}{\sf Todo: [#1]}}

%===================%
%  Label reminders  %
%===================%
% [label=(\roman*)]
% [label=(\alph*)]
% [label=(\arabic{enumi})]

%==================%
%  Other settings  %
%==================%
\pgfdeclarelayer{background}
\pgfsetlayers{background,main}
\tikzset{->-/.style={decoration={
  markings,
  mark=at position .5 with {\arrow{>}}},postaction={decorate}}}

%=================%
%  Title & Index  %
%=================%
\title{Subconvexity for $\GL_{2}$ $L$-functions via multiple Dirichlet series}
\author{Henry Twiss}
\date{\today}
\makeindex

\begin{document}

\begin{abstract}
  We use the amplification method and the analytic properties of shifted multiple Dirichlet series to obtain a subconvexity reuslt for twisted $\GL_{2}$ holomorphic cusp forms. This proves upon the subconvexity bound established in \cite{K}.
\end{abstract}

\maketitle

\section*{Subconvexity}
  Let $f$ be a weight $k$ holomorphic cusp form on $\G_{1}(q)\backslash\H$, for a prime $q$, and with trivial character. Let $\chi$ be a primitive Dirichlet character of conductor $p$. Suppose $(p,q) = 1$. Then it is a result of \cite{K} that
  \[
    L\left(\frac{1}{2},f \x \chi\right) \ll_{k,\e}(qp^{2})^{\frac{1}{4}+\e}(p^{-\frac{1}{4}}+q^{\frac{\t}{4}-\frac{1}{8}}),
  \]
  where $\t$ is the current best bound towards the Ramanujan-Petersson conjecture. We will improve upon this bound in the level $1$ case using Weyl group multiple Dirichlet series.
\section{Background Setup}
  Let $q \ge 1$ and let $\psi$ be a Dirichlet character modulo $q$. For $m \in \Z$, let
  \[
    c_{q}(m) = \sum_{a \tmod{q}}e^{\frac{2\pi ima}{q}} \quad \text{and} \quad c_{\psi}(m) = \sum_{a \tmod{q}}\psi(a)e^{\frac{2\pi ima}{q}},
  \]
  be the Ramanujan and Gauss sums respectively. In particular, for $\ell$ such that $(\ell,q) = 1$, we have
  \[
    c_{\psi}(\ell m) = \conj{\psi(\ell)}c_{\psi}(m),
  \]
  and moreover
  \[
    c_{\psi}(m) = \conj{\psi(m)}c_{\psi}(1),
  \]
  provided $\psi$ is primitive. Throughout we will let
  \[
    f(z) = \sum_{m \ge 1}a(m)e^{2\pi imz} = \sum_{m \ge 1}A(m)m^{\frac{k-1}{2}}e^{2\pi imz} \quad \text{and} \quad g(z) = \sum_{m \ge 1}b(m)e^{2\pi imz} = \sum_{m \ge 1}B(m)m^{\frac{k-1}{2}}e^{2\pi imz},
  \]
  be the Fourier expansions of two weight $k$ and level $1$ Hecke eigenforms $f$ and $g$. We define the $L$-function $L(s,f \x c_{\psi})$ by
  \[
    L(s,f \x c_{\psi}) = \sum_{m \ge 1}\frac{A(m)c_{\psi}(m)}{m^{s}}.
  \]
  When $\psi$ is primitive this is related to the $L$-function $L(s,f \x \psi)$ by
  \[
    L(s,f \x c_{\psi}) = \sqrt{q}L(s,f \x \psi).
  \]
  We will now define our primary object of interest:
  \[
    S_{f,g}(s_{1},s_{2};q) = \frac{1}{\vphi(q)}\sum_{\psi \tmod{q}}L(s_{1},f \x c_{\psi})L(s_{2},g \x \conj{c_{\psi}}).
  \]
  There are multiple Dirichlet series that are connected to these sums. Throughout we let $\{\mu_{j}\}$ represent an orthonormal basis of Maass forms with spectral parameter $t_{j}$ for $\mu_{j}$. Moreover, let $\ell_{1}$ and $\ell_{2}$ be fixed primes. Also set
  \[
    V_{f,g} = V_{f,g}(z;\ell_{1},\ell_{2}) = \conj{f(\ell_{1}z)}g(\ell_{2}z)\Im(z)^{k} \quad \text{and} \quad V_{f,v} = V_{f}(z;\ell_{1}) = \conj{f(\ell_{1}z)}E(z,s;k)\Im(z)^{\frac{k}{2}}.
  \]

  \subsection*{The Dirichlet Series \texorpdfstring{$D_{f,g}(s;h,\ell_{1},\ell_{2})$}{}}
    Let $h \ge 1$. Our first multiple Dirichlet series $D_{f,g}(s;h,\ell_{1},\ell_{2})$ is given by
    \[
      D_{f,g}(s;h,\ell_{1},\ell_{2}) = \sum_{\ell_{1}m = \ell_{2}n+h}\frac{a(m)b(n)}{n^{s+k-1}}.
    \]
    This series is absolutely convergent for $\Re(s) > 1$ and admits meromorphic continuation to $\frac{1-k}{2}-C_{1} < \Re(s)$, for any $C_{1} > 0$, and in these two regions it satisfies the bounds
    \[
      D_{f,g}(s;h,\ell_{1},\ell_{2}) \ll_{\todo{\ell_{1},\ell_{2}}} h^{\frac{k-1}{2}+\e} \quad \text{and} \quad D_{f,g}(s;h,\ell_{1},\ell_{2}) \ll_{\todo{\ell_{1},\ell_{2}}} h^{k+2C_{1}+\e},
    \]
    respectively. In the region $\frac{1-k}{2}-C_{1} < \Re(s)$, the meromorphic continuation is given by the absolutely convergent spectral expansion
    \[
      D_{f,g}(s;h,\ell_{1},\ell_{2}) = \frac{\G(1-s)}{\G(s+k-1)}\sum_{j}\conj{\rho_{t_{j}}(-h)}h^{\frac{1}{2}-s}\frac{\G\left(s-\frac{1}{2}+it_{j}\right)\G\left(s-\frac{1}{2}-it_{j}\right)}{\G\left(\frac{1}{2}+it_{j}\right)\G\left(\frac{1}{2}-it_{j}\right)}\conj{\<V_{f,g},\mu_{j}\>}.
    \]
    These two representations for $D_{f,g}(s;h,\ell_{1},\ell_{2})$ give meromorphic continuation to $\C$ but we do not have a repsentation in the strip $\frac{1-k}{2} \le \Re(s) \le 1$.
  \subsection*{The Dirichlet Series \texorpdfstring{$D_{f,v}(w;n,\ell_{1},\ell_{2})$}{}}
    Let $n \ge 1$. Our second multiple Dirichlet series $D_{f,v}(w;n,\ell_{1},\ell_{2})$ is given by
    \[
      D_{f,v}(w;n,\ell_{1},\ell_{2}) = \sum_{\ell_{1}m = \ell_{2}n+h}\frac{a(m)\s_{1-2v}(h)h^{v-\frac{1}{2}}}{h^{w+\frac{k-1}{2}}}.
    \]
    Let $c > 0$ be such that if $v$ satisfies $\z(2v) \neq 0$, then $\Re(v)\ge \frac{1}{2}-\frac{8c}{\log(2+\Im(v))}$. For such a $c$, we set
    \[
      \d(s,v,u) = \frac{c}{\log(3+|\Im(s+u)|+|\Im(v)|)} \quad \text{and} \quad \d_{v} = \d(0,0,v).
    \]
    


  The former series converges absolutely for $\Re(s) > 1$ while the latter does for $\Re(w) > \Re(v)+\frac{1}{2}$ and $\Re(v) > \frac{1}{2}$. In these regions, the series satisfy the estimates
  \[
    D_{f,g}(s;h,\ell_{1},\ell_{2}) \ll h^{\frac{k-1}{2}+\e} \quad \text{and} \quad D_{f,v}(w;n,\ell_{1},\ell_{2}) \ll n^{\frac{k-1}{2}+\e}.
  \]
  We define the associted multiple Dirichlet series
  \[
    Z_{f,g}(s,u,v;\ell_{1},\ell_{2}) = \sum_{\ell_{1}m = \ell_{2}n+h}\frac{a(m)b(n)\s_{1-2v}(h)}{n^{s+k-1}h^{u}}.
  \]
  This converges absolutely for $\Re(s) > 1$, $\Re(u) > \frac{k+1}{2}$, and $\Re(v) > \frac{1}{2}$. Moreover, in this region $Z_{f,g}$ can be expressed in terms of $D_{f,g}$ and $D_{f,v}$ as
  \[
    Z_{f,g}(s,u,v;\ell_{1},\ell_{2}) = \sum_{h \ge 1}\frac{D_{f,g}(s;h,\ell_{1},\ell_{2})\s_{1-2v}(h)}{h^{u}} = \sum_{n \ge 1}\frac{D_{f,v}\left(u+v-\frac{k}{2};n,\ell_{1},\ell_{2}\right)b(n)}{n^{s+k-1}},
  \]
  with both representations converging absolutely. The series $D_{f,g}$ and $D_{f,v}$ also admit spectral expansions. To state them, set
  \[
    V_{f,g} = V_{f,g}(z;\ell_{1},\ell_{2}) = \conj{f(\ell_{1}z)}g(\ell_{2}z)\Im(z)^{k} \quad \text{and} \quad V_{f,v} = V_{f}(z;\ell_{1}) = \conj{f(\ell_{1}z)}E(z,s;k)\Im(z)^{\frac{k}{2}}.
  \]
  From \cite{HHR}, $D_{f,g}$ admits the spectral expansion (modulo the continuous spectrum and up to constants)
  \[
    D_{f,g}(s;h,\ell_{1},\ell_{2}) = \frac{\G(1-s)}{\G(s+k-1)}\sum_{j}\conj{\rho_{t_{j}}(-h)}h^{\frac{1}{2}-s}\frac{\G\left(s-\frac{1}{2}+it_{j}\right)\G\left(s-\frac{1}{2}-it_{j}\right)}{\G\left(\frac{1}{2}+it_{j}\right)\G\left(\frac{1}{2}-it_{j}\right)}\conj{\<V_{f,g},\mu_{j}\>},
  \]
  which converges absolutely for $\Re(s) < \frac{1-k}{2}$ and at least $\e$ away from the poles. By analytic continuation, $D_{f,g}$ is meromorphic on $\C$. This induces a spectral expansion for $Z_{f,g}$ given by
  \begin{align*}
    Z_{f,g}(s,u,v;\ell_{1},\ell_{2}) = \frac{\G(1-s)}{\G(s+k-1)}&\sum_{j}\conj{\rho_{t_{j}}(-1)}h^{\frac{1}{2}-s}\frac{\G\left(s-\frac{1}{2}+it_{j}\right)\G\left(s-\frac{1}{2}-it_{j}\right)}{\G\left(\frac{1}{2}+it_{j}\right)\G\left(\frac{1}{2}-it_{j}\right)}\conj{\<V_{f,g},\mu_{j}\>} \\
    &\cdot \frac{L\left(s+u-\frac{1}{2},\mu_{j}\right)L\left(s+u+2v-\frac{3}{2},\mu_{j}\right)}{\z(2s+2u+2v-2)},
  \end{align*}
  which converges absolutely for $\Re(s) < \frac{1-k}{2}$, $\Re(u) > \frac{k+1}{2}$, and $\Re(v) > \frac{1}{2}$. Similarily, $D_{f,v}$ admits the spectral expansion (modulo the continuous spectrum and up to constants)
  \begin{align*}
    D_{f,v}\left(w;n,\ell_{1},\ell_{2}\right) = \frac{\G(1-w)\G(w)}{\G\left(w+v+\frac{k}{2}-1\right)\G\left(w-v+\frac{k}{2}\right)}&\sum_{j}\conj{\rho_{t_{j}}(-\ell_{2}n)}(\ell_{2}n)^{\frac{1}{2}-w}\frac{\G\left(w-\frac{1}{2}+it_{j}\right)\G\left(w-\frac{1}{2}-it_{j}\right)}{\G\left(\frac{1}{2}+it_{j}\right)\G\left(\frac{1}{2}-it_{j}\right)} \\
    &\cdot \conj{\<V_{f,v},\mu_{j}\>},
  \end{align*}
  which converges absolutely for $\Re(w) < \frac{1-k}{2}$. By analytic continuation, $D_{f,v}$ is meromorphic on $\C$. This induces another spectral expansion for $Z_{f,g}$ given by
  \begin{align*}
    Z_{f,g}(s,u,v;\ell_{1},\ell_{2}) &= \frac{\G\left(\frac{k}{2}+1-u-v\right)\G\left(u+v-\frac{k}{2}\right)}{\G(u+2v-1)\G(u)}\sum_{j}\conj{\rho_{t_{j}}(-1)}\frac{\G\left(u+v-\frac{k+1}{2}+it_{j}\right)\G\left(u+v-\frac{k+1}{2}-it_{j}\right)}{\G\left(\frac{1}{2}+it_{j}\right)\G\left(\frac{1}{2}-it_{j}\right)} \\
    &\cdot \conj{\<V_{f,v},\mu_{j}\>}\ell_{2}^{\frac{k+1}{2}-u-v}\frac{L^{(\ell_{2})}(s+u+v-1,g \x \mu_{j})}{\z^{(\ell_{2})}(2s+2u+2v-2)}\sum_{\a \ge 0}\frac{b(\ell_{2}^{\a})\l_{f}(\ell_{2}^{\a+1})}{(\ell_{2}^{\a})^{s+u+v-1+\frac{k-1}{2}}},
  \end{align*}
  which converges absolutely for $\Re(s) > 1$, $\Re(u) < 1-2\Re(v)$ \todo{subtly from Jeff}, and $\Re(v) > \frac{1}{2}$ \todo{update subtly from Jeff}.
\section{An Amplified Series}
  Let $\ell$ be prime, fix a primitive Dirichlet character $\chi$ modulo $Q \gg 1$, and set
  \[
    S_{f}(s;q,L) = \frac{1}{\vphi(q)}\sum_{\psi \tmod{q}}|L(s,f \x c_{\psi})|^{2}\left|\sum_{\ell \sim L}\chi(\ell)\conj{\psi(\ell)}\right|^{2},
  \]
  where $\ell \sim L$ means $\ell \in [L,2L]$. We will primarily be interested in an average of $S_{f}(s;q,L)$ over $q$ in a short interval around $Q$. Accordingly, for any $\e > 0$ define
  \[
    S_{f}(s;Q,L) = \sum_{|q-Q| \ll Q^{\e}}S_{f}(s;q,L).
  \]
  Our desired result will follow from upper and lower bounds for this sum. The presence of the sum over $\ell \sim L$ in each $S_{f}(s;q,L)$ is to amplify the term attached to the character $\chi$ (note that this only happens when $q = Q$). For the lower bound, consider $S_{f}(s;Q,L)$. When $\psi = \chi$ the prime number theorem gives
  \[
    \left|\sum_{\ell \sim L}\chi(\ell)\cchi(\ell)\right|^{2} \sim \frac{L^{2}}{\log(L)^{2}}.
  \]
  So the contribution coming from the term corresponding to $\chi$ is
  \[
    \frac{L^{2}}{\vphi(Q)\log(L)^{2}}|L(s,f \x c_{\chi})|^{2} = \frac{QL^{2}}{\vphi(Q)\log(L)^{2}}|L(s,f \x \chi)|^{2}. 
  \]
  Since every term in $S_{f}(s;Q,L)$ is nonnegative, we can discard them and obtain a lower bound of the form
  \[
    \frac{QL^{2}}{\vphi(Q)\log(L)^{2}}|L(s,f \x \chi)|^{2} \ll S_{f}(s;q,L).
  \]
  Recalling that $\vphi(Q) \sim Q$ and discarding the other $S_{f}(s;q,L)$, we arrive at the associated lower bound
  \[
    \frac{L^{2}}{\log(L)^{2}}|L(s,f \x \chi)|^{2} \ll S_{f}(s;Q,L).
  \]
  The upper bound requires much more delicate treatment. We first expand all of the sums in $S_{f}(s;q,L)$:
  \begin{align*}
    S_{f}(s;q,L) &= \frac{1}{\vphi(q)}\sum_{\psi \tmod{q}}\sum_{m,n \ge 1}\frac{A(m)c_{\psi}(m)A(n)\conj{c_{\psi}(n)}}{(mn)^{s}}\sum_{\ell_{1},\ell_{2} \sim L}\chi(\ell_{1})\conj{\psi}(\ell_{1})\cchi(\ell_{2})\psi(\ell_{2}) \\
    &= \frac{1}{\vphi(q)}\sum_{\psi \tmod{q}}\sum_{m,n \ge 1}\frac{A(m)A(n)}{(mn)^{s}}\sum_{\ell_{1},\ell_{2} \sim L}c_{\psi}(\ell_{1}m)\conj{c_{\psi}}(\ell_{2}n)\chi(\ell_{1})\cchi(\ell_{2}) \\
    &= \frac{1}{\vphi(q)}\sum_{m,n \ge 1}\sum_{\ell_{1},\ell_{2} \sim L}\frac{A(m)A(n)}{(mn)^{s}}\sum_{\psi \tmod{q}}c_{\psi}(\ell_{1}m)\conj{c_{\psi}}(\ell_{2}n)\chi(\ell_{1})\cchi(\ell_{2}).
  \end{align*}
  Now
  \[
    \frac{1}{\vphi(q)}\sum_{\psi \tmod{q}}c_{\psi}(\ell_{1}m)\conj{c_{\psi}}(\ell_{2}n) = c_{q}(\ell_{1}m-\ell_{2}n),
  \]
  so that
  \begin{align*}
    S_{f}(s;q,L) &= \sum_{\ell_{1},\ell_{2} \sim L}\sum_{m,n \ge 1}\frac{A(m)A(n)}{(mn)^{s}}c_{q}(\ell_{1}m-\ell_{2}n)\chi(\ell_{1})\cchi(\ell_{2}) \\
    &= \sum_{\ell_{1},\ell_{2} \sim L}\chi(\ell_{1})\cchi(\ell_{2})\sum_{m,n \ge 1}\frac{A(m)A(n)}{(mn)^{s}}c_{q}(\ell_{1}m-\ell_{2}n) \\
    &= \sum_{\ell_{1},\ell_{2} \sim L}\chi(\ell_{1})\cchi(\ell_{2})S_{f}(s;q,\ell_{1},\ell_{2}),
  \end{align*}
  where we have set
  \[
    S_{f}(s;q,\ell_{1},\ell_{2}) = \sum_{m,n \ge 1}\frac{A(m)A(n)}{(mn)^{s}}c_{q}(\ell_{1}m-\ell_{2}n).
  \]
  Therefore
  \[
      S_{f}(s;Q,L) = \sum_{\ell_{1},\ell_{2} \sim L}\chi(\ell_{1})\cchi(\ell_{2})\sum_{|q-Q| \ll Q^{\e}}S_{f}(s;q,\ell_{1},\ell_{2}).
  \]
  In order to carefully estimate the sum over $q$ via Perron-type formulas, we need to understand the analytic properties of the Dirichlet series with coefficients $S_{f}(s;q,\ell_{1},\ell_{2})$. Thus, we define
  \[
    S_{f}(s,v;\ell_{1},\ell_{2}) = \sum_{q \ge 1}\frac{S_{f}(s;q,\ell_{1},\ell_{2})}{q^{2v}} = \sum_{m,n \ge 1}\frac{A(m)A(n)}{(mn)^{s}}\sum_{q \ge 1}\frac{c_{q}(\ell_{1}m-\ell_{2}n)}{q^{2v}}.
  \]
  The inner sum can be expressed as
  \[
    \sum_{q \ge 1}\frac{c_{q}(\ell_{1}m-\ell_{2}n)}{q^{2v}} = \begin{cases} \frac{\z(2v-1)}{\z(2v)} & \text{if $\ell_{1}m = \ell_{2}n$}, \\ \frac{\s_{1-2v}(\ell_{1}m-\ell_{2}n)}{\z(2v)} & \text{if $\ell_{1}m \neq \ell_{2}n$}. \end{cases}
  \]
  So if we write $\ell_{1}m = \ell_{2}n+h$ with $h \ge 1$, then $S_{f}(s,v;\ell_{1},\ell_{2})$ can be expressed as the sum of a diagional and an off-diagional contribution:
  \[
    S_{f}(s,v;\ell_{1},\ell_{2}) = \frac{\z(2v-1)}{\z(2v)}\sum_{\ell_{1}m = \ell_{2}n}\frac{A(m)A(n)}{(mn)^{s}}+\frac{2}{\z(2v)}\sum_{\ell_{1}m = \ell_{2}n+h}\frac{A(m)A(n)\s_{1-2v}(h)}{(mn)^{s}}.
  \]
\section{Perron-type Estimates}
  We can now apply Perron-type formulas to upper bound
  \[
    S_{f}(s;Q,L) = \sum_{\ell_{1},\ell_{2} \sim L}\chi(\ell_{1})\cchi(\ell_{2})\sum_{|q-Q| \ll Q^{\e}}S_{f}(s;q,\ell_{1},\ell_{2}).
  \]
  Recall the inverse Mellin transform
  \[
    \frac{1}{2\pi i}\int_{(2)}\frac{e^{\frac{\pi v^{2}}{Q^{2}}}x^{2v}}{Q}\,dv = e^{-\frac{(y\log(x))^{2}}{\pi}}.
  \]
  An application of smoothed Perron's formula with this transform when $x = Q$, yields
  \begin{align*}
    \frac{1}{2\pi i}\int_{(2)}S_{f}(s,v;\ell_{1},\ell_{2})\frac{e^{\frac{\pi v^{2}}{Q^{2}}}Q^{2v}}{Q}\,dv &= \frac{1}{2\pi i}\int_{(2)}\sum_{q \ge 1}\frac{S_{f}(s;q,\ell_{1},\ell_{2})}{q^{2v}}\frac{e^{\frac{\pi v^{2}}{Q^{2}}}Q^{2v}}{Q}\,dv \\
    &= \sum_{q \ge 1}S_{f}(s;q,\ell_{1},\ell_{2})\frac{1}{2\pi i}\int_{(2)}\frac{e^{\frac{\pi v^{2}}{Q^{2}}}\left(\frac{Q}{q}\right)^{2v}}{Q}\,dv \\
    &= \sum_{|q-Q| \ll Q^{\e}}S_{f}(s;q,\ell_{1},\ell_{2})+O_{s}(Q^{-B}),
  \end{align*}
  with $B \gg 1$. We will now compute the Mellin transform in another way. We can decompose the integral into a diagional and an off-diagional term:
  \begin{align*}
    \frac{1}{2\pi i}\int_{(2)}S_{f}(s,v;\ell_{1},\ell_{2})\frac{e^{\frac{\pi v^{2}}{Q^{2}}}Q^{2v}}{Q}\,dv &= \frac{1}{2\pi i}\int_{(2)}\frac{\z(2v-1)}{\z(2v)}\sum_{\ell_{1}m = \ell_{2}n}\frac{A(m)A(n)}{(mn)^{s}}\frac{e^{\frac{\pi v^{2}}{Q^{2}}}Q^{2v}}{Q}\,dv \\
    &+ \frac{1}{2\pi i}\int_{(2)}\frac{2}{\z(2)}\sum_{\ell_{1}m = \ell_{2}n+h}\frac{A(m)A(n)\s_{1-2v}(h)}{(mn)^{s}}\frac{e^{\frac{\pi v^{2}}{Q^{2}}}Q^{2v}}{Q}\,dv.
  \end{align*}
  For the diagional term, write
  \[
    \frac{\z(2v-1)}{\z(2v)} = \sum_{q \ge 1}\frac{\vphi(q)}{q^{2v}}.
  \]
  Then we compute
  \begin{align*}
    \frac{1}{2\pi i}\int_{(2)}\frac{\z(2v-1)}{\z(2v)}\sum_{\ell_{1}m = \ell_{2}n}\frac{A(m)A(n)}{(mn)^{s}}\frac{e^{\frac{\pi v^{2}}{Q^{2}}}Q^{2v}}{Q}\,dv &= \frac{1}{2\pi i}\int_{(2)}\sum_{q \ge 1}\frac{\vphi(q)}{q^{2v}}\sum_{\ell_{1}m = \ell_{2}n}\frac{A(m)A(n)}{(mn)^{s}}\frac{e^{\frac{\pi v^{2}}{Q^{2}}}Q^{2v}}{Q}\,dv \\
    &= \sum_{q \ge 1}\vphi(q)\sum_{\ell_{1}m = \ell_{2}n}\frac{A(m)A(n)}{(mn)^{s}}\frac{1}{2\pi i}\int_{(2)}\frac{e^{\frac{\pi v^{2}}{Q^{2}}}\left(\frac{Q}{q}\right)^{2v}}{Q}\,dv \\
    &= \sum_{|q-Q| \ll Q^{\e}}\vphi(q)\sum_{\ell_{1}m = \ell_{2}n}\frac{A(m)A(n)}{(mn)^{s}}+O_{s}(Q^{-B}).
  \end{align*}
  Therefore the diagional contribution has size
  \[
    \frac{1}{2\pi i}\int_{(2)}\frac{\z(2v-1)}{\z(2v)}\sum_{\ell_{1}m = \ell_{2}n}\frac{A(m)A(n)}{(mn)^{s}}\frac{e^{\frac{\pi v^{2}}{Q^{2}}}Q^{2v}}{Q}\,dv = \sum_{|q-Q| \ll Q^{\e}}\vphi(q)\sum_{\ell_{1}m = \ell_{2}n}\frac{A(m)A(n)}{(mn)^{s}}+O_{s}(Q^{-B}).
  \]
  To estimate the sum, writing $\ell_{1}m = \ell_{2}n = \ell_{1}\ell_{2}d$ and noting that $A(m),A(n) \ll 1$ gives
  \[
    \sum_{\ell_{1}m = \ell_{2}n}\frac{A(m)A(n)}{(mn)^{s}} \ll \sum_{d \ge 1}\frac{1}{(\ell_{1}\ell_{2})^{s}d^{2s}} = \frac{1}{(\ell_{1}\ell_{2})^{s}}\z(2s).
  \]
  Specializing $s = \frac{1}{2}+\e$, we find that the diagional contribution is
  \[
    \ll_{\e} \frac{Q^{1+\e}}{L^{1+2\e}},
  \]
  where we have again used that $\vphi(q) \sim q$. For the off-diagional term, first make the following computation:
  \begin{align*}
    \sum_{\ell_{1}m = \ell_{2}n+h}\frac{A(m)A(n)\s_{1-2v}(h)}{(mn)^{s}} &= \sum_{\ell_{1}m = \ell_{2}n+h}\frac{A(m)A(n)\s_{1-2v}(h)(\ell_{1}\ell_{2})^{s+\frac{k-1}{2}}}{(mn)^{s}(\ell_{1}\ell_{2})^{s+\frac{k-1}{2}}} \\
    &= (\ell_{1}\ell_{2})^{s+\frac{k-1}{2}}\sum_{\ell_{1}m = \ell_{2}n+h}\frac{a(m)a(n)\s_{1-2v}(h)}{(\ell_{1}m)^{s+\frac{k-1}{2}}(\ell_{2}n)^{s+\frac{k-1}{2}}} \\
    &= (\ell_{1}\ell_{2})^{s+\frac{k-1}{2}}\sum_{\ell_{1}m = \ell_{2}n+h}\frac{a(m)a(n)\s_{1-2v}(h)}{(\ell_{2}n+h)^{s+\frac{k-1}{2}}(\ell_{2}n)^{s+\frac{k-1}{2}}} \\
    &= (\ell_{1}\ell_{2})^{s+\frac{k-1}{2}}\sum_{\ell_{1}m = \ell_{2}n+h}\frac{a(m)a(n)\s_{1-2v}(h)}{\left(1+\frac{h}{\ell_{2}n}\right)^{s+\frac{k-1}{2}}(\ell_{2}n)^{2s+k-1}}.
  \end{align*}
  Recall the identity
  \[
    \frac{1}{(1+t)^{\b}} = \frac{1}{2\pi i}\int_{(c)}\frac{\G(\b-u)\G(u)}{\G(\b)}t^{-u}\,du,
  \]
  for any $0 < c < \Re(\b)$. Applying this identity to our sum with $t = \frac{h}{\ell_{2}n}$ and $\b = s+\frac{k-1}{2}$ yields
  \[
    \frac{1}{2\pi i}\int_{(c)}(\ell_{1}\ell_{2})^{s+\frac{k-1}{2}}\sum_{\ell_{1}m = \ell_{2}n+h}\frac{a(m)a(n)\s_{1-2v}(h)}{(\ell_{2}n)^{2s-u+k-1}h^{u}}\frac{\G\left(s-u+\frac{k-1}{2}\right)\G(u)}{\G\left(s+\frac{k-1}{2}\right)}\,du.
  \]
  This integral can be expressed as
  \[
    \frac{1}{2\pi i}\int_{(c)}\ell_{1}^{s+\frac{k-1}{2}}\ell_{2}^{u-s-\frac{k-1}{2}}Z_{f}(2s-u,u,v;\ell_{1},\ell_{2})\frac{\G\left(s-u+\frac{k-1}{2}\right)\G(u)}{\G\left(s+\frac{k-1}{2}\right)}\,du.
  \]
  So the off-diagional contribution is
  \[
    \frac{1}{(2\pi i)^{2}}\int_{(2)}\int_{(c)}\frac{2}{\z(2v)}\ell_{1}^{s+\frac{k-1}{2}}\ell_{2}^{u-s-\frac{k-1}{2}}Z_{f}(2s-u,u,v;\ell_{1},\ell_{2})\frac{\G\left(s-u+\frac{k-1}{2}\right)\G(u)}{\G\left(s+\frac{k-1}{2}\right)}\frac{e^{\frac{\pi v^{2}}{Q^{2}}}Q^{2v}}{Q}\,du\,dv.
  \]
  Shifting the line in $v$ at $(2)$ to $\left(\frac{1}{2}+\e\right)$, we are still within the region of absolute convergence for $Z_{f}$ and do not pass over any poles of the integrand. This gives
  \[
    \frac{1}{(2\pi i)^{2}}\int_{\left(\frac{1}{2}+\e\right)}\int_{(c)}\frac{2}{\z(2v)}\ell_{1}^{s+\frac{k-1}{2}}\ell_{2}^{u-s-\frac{k-1}{2}}Z_{f}(2s-u,u,v;\ell_{1},\ell_{2})\frac{\G\left(s-u+\frac{k-1}{2}\right)\G(u)}{\G\left(s+\frac{k-1}{2}\right)}\frac{e^{\frac{\pi v^{2}}{Q^{2}}}Q^{2v}}{Q}\,du\,dv.
  \]
  Now we shift the line in $u$ at $(c)$ to $\left(\e-\frac{1}{2}\right)$. Here we pass over polar lines of $Z_{f}$. The off-diagional contribution then becomes
  \[
    \Res(s)+\frac{1}{(2\pi i)^{2}}\int_{\left(\frac{1}{2}+\e\right)}\int_{\left(\e-\frac{1}{2}\right)}\frac{2}{\z(2v)}\ell_{1}^{s+\frac{k-1}{2}}\ell_{2}^{u-s-\frac{k-1}{2}}Z_{f}(2s-u,u,v;\ell_{1},\ell_{2})\frac{\G\left(s-u+\frac{k-1}{2}\right)\G(u)}{\G\left(s+\frac{k-1}{2}\right)}\frac{e^{\frac{\pi v^{2}}{Q^{2}}}Q^{2v}}{Q}\,du\,dv,
  \]
  where
  \[
    \Res(s) = \todo{xxx}
  \]
  Ignoring the residue term for the moment, the second spectral expansion of $Z_{f}(s,u,v;\ell_{1},\ell_{2})$ along with an analgous result to one in \cite{HHR}, we have
  \[
    Z_{f}(s,u,v;\ell_{1},\ell_{2}) \ll \ell_{1}^{-\frac{k-1}{2}}\ell_{2}^{\frac{k+1}{2}-u-v}\todo{\text{polynomial in $s$}}.
  \]
  Therefore the integral is
  \[
    \ll \frac{1}{(2\pi i)^{2}}\int_{\left(\frac{1}{2}+\e\right)}\int_{\left(\e-\frac{1}{2}\right)}\frac{2}{\z(2v)}\ell_{1}^{s}\ell_{2}^{\frac{1}{2}-u-v}\todo{\text{polynomial in $s$}}\frac{\G\left(s-u+\frac{k-1}{2}\right)\G(u)}{\G\left(s+\frac{k-1}{2}\right)}\frac{e^{\frac{\pi v^{2}}{Q^{2}}}Q^{2v}}{Q}\,du\,dv,
  \]
  and from this we deduce that the off-diagional contribution at $s = \frac{1}{2}+\e$ is
  \[
    \ll_{\e} Q^{\e}L^{\frac{1}{2}+3\e}.
  \]
  Combining the diagional and off-diagional estimates, we have
  \[
    \sum_{|q-Q| \ll Q^{\e}}S_{f}\left(\frac{1}{2};q,\ell_{1},\ell_{2}\right) \ll_{\e} \frac{Q^{1+\e}}{L^{1+2\e}}+Q^{\e}L^{\frac{1}{2}+3\e}
  \]
  and summing over $\ell_{1}$ and $\ell_{2}$ yields
  \[
    S_{f}\left(\frac{1}{2},Q,L\right) \ll_{\e} \sum_{\ell_{1},\ell_{2} \sim L}\chi(\ell_{1})\cchi(\ell_{2})\left(\frac{Q^{1+\e}}{L^{1+2\e}}+Q^{\e}L^{\frac{1}{2}+3\e}\right) \ll Q^{1+\e}L^{1-2\e}+L^{\frac{5}{2}+3\e}Q^{\e}.
  \]
  Ignoring the $\e$ factors, setting $L = Q^{\frac{2}{3}}$ balances the error terms so that
  \[
    S_{f}\left(\frac{1}{2},Q,L\right) \ll_{\e} Q^{\frac{5}{3}}.
  \]
  This the desired upper bound. Combining with the lower bound and our choice of $L$ results in
  \[
      \frac{Q^{\frac{4}{3}}}{\log(Q^{\frac{2}{3}})^{2}}|L(s,f \x \chi)|^{2} \ll S_{f}\left(\frac{1}{2},Q,Q^{\frac{2}{3}}\right) \ll_{\e} Q^{\frac{5}{3}},
  \]
  which, ignoring logarithmic factors, implies
  \[
    |L(s,f \x \chi)| \ll_{\e} Q^{\frac{1}{6}}.
  \]
\section*{A Bound Analgous to Resnikov}
  We will prove the bound
  \[
    \sum_{t_{j} \sim T}|\<V_{f,v},\mu_{j}\>|^{2}e^{\frac{\pi}{2}|t_{j}|} \ll \ell_{1}^{-k}T^{2k+\e}\log(T).
  \]
  To this end, we first estimate the inner product:
  \begin{align*}
    \<V_{f,v},\mu_{j}\> &= \frac{1}{\mc{V}(\ell_{1})}\int_{\mc{F}(\ell_{1})}\conj{f(\ell_{1}z)}E(z,s;k)\conj{\mu_{j}(z)}\Im(z)^{\frac{k}{2}}\,d\mu \\
    &= \frac{1}{\mc{V}(\ell_{1})}\int_{\mc{F}(\ell_{1})}\conj{f(\ell_{1}z)}\conj{\mu_{j}(z)}\Im(z)^{\frac{k}{2}}\sum_{\g \in \G_{\infty}\backslash\G_{0}(1)}\left(\frac{j(\g,z)}{|j(\g,z)|}\right)^{-k}\Im(\g z)^{s}\,d\mu \\
    &= \frac{1}{\mc{V}(\ell_{1})}\int_{\mc{F}(\ell_{1})}\sum_{\g \in \G_{\infty}\backslash\G_{0}(1)}\conj{f(\g\ell_{1} z)}\conj{\mu_{j}(\g z)}\left(\frac{\conj{j(\g,\ell_{1}z)}j(\g,z)}{|j(\g,z)|^{2}}\right)^{-k}\Im(\g z)^{s+\frac{k}{2}}\,d\mu.
  \end{align*}
  Writing $\mc{F}(\ell_{1}) = \bigcup_{\eta \in \G_{0}(\ell_{1})\backslash\G_{0}(1)}\eta\mc{F}$, we can express the integral as
  \[
    \frac{1}{\mc{V}(\ell_{1})}\sum_{\eta \in \G_{0}(\ell_{1})\backslash\G_{0}(1)}\int_{\eta\mc{F}(\ell_{1})}\sum_{\g \in \G_{\infty}\backslash\G_{0}(1)}\conj{f(\g\ell_{1} z)}\conj{\mu_{j}(\g z)}\left(\frac{\conj{j(\g,\ell_{1}z)}j(\g,z)}{|j(\g,z)|^{2}}\right)^{-k}\Im(\g z)^{s+\frac{k}{2}}\,d\mu,
  \]
  which is equivalent to
  \[
    \frac{1}{\mc{V}(\ell_{1})}\int_{\mc{F}}\sum_{\eta \in \G_{0}(\ell_{1})\backslash\G_{0}(1)}\sum_{\g \in \G_{\infty}\backslash\G_{0}(1)}\conj{f(\g\eta\ell_{1} z)}\conj{\mu_{j}(\g\eta z)}\left(\frac{\conj{j(\g,\eta \ell_{1}z)}j(\g,\eta z)}{|j(\g,\eta z)|^{2}}\right)^{-k}\Im(\g\eta z)^{s+\frac{k}{2}}\,d\mu
  \]



\begin{thebibliography}{99}
  \bibitem{K}
  Khan, Rizwanur. "Subconvexity bounds for twisted $L$-functions, II." Transactions of the American Mathematical Society 375.10 (2022): 6769-6796.

  \bibitem{HHR}
  Hoffstein, Jeff, Thomas A. Hulse, and Andre Reznikov. "Multiple Dirichlet series and shifted convolutions." Journal of Number Theory 161 (2016): 457-533.
\end{thebibliography}

\end{document}