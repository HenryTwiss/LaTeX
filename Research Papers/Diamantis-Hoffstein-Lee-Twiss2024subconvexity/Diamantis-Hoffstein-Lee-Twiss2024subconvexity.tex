\documentclass[11pt]{amsart}
\usepackage{fullpage,verbatim,amssymb}
\usepackage{hyperref, cancel, soul}
\usepackage[normalem]{ulem}
\usepackage{amsmath}
\usepackage[active]{srcltx}
\usepackage[cmtip,arrow,matrix,curve,tips,frame]{xy}
\usepackage[usenames,dvipsnames]{color}
\usepackage{pgf,graphics}
\makeatletter
\let\@@pmod\mod
\DeclareRobustCommand{\mod}{\@ifstar\@pmods\@@pmod}
\def\@pmods#1{\mkern4mu({\operator@font mod}\mkern 6mu#1)}
\makeatother
\newcommand{\stkout}[1]{\ifmmode\text{\sout{\ensuremath{#1}}}\else\sout{#1}\fi}




%
\definecolor{blue}{rgb}{0,0,1}
\definecolor{red}{rgb}{1,0,0}
\definecolor{green}{rgb}{0,.6,.2}
\definecolor{purple}{rgb}{1,0,1}


\long\def\red#1\endred{\textcolor{red}{#1}}
\long\def\blue#1\endblue{\textcolor{blue}{#1}}
\long\def\purple#1\endpurple{\textcolor{purple}{ #1}}
\long\def\green#1\endgreen{\textcolor{green}{#1}}
\newcommand{\G}{\Gamma}  
\newcommand{\wt}{k} 
\newcommand{\vol}{}
\newcommand{\tkt}{\tfrac k2}
\newcommand{\lt}{\left}
\newcommand{\wi}{\widetilde}
\newcommand{\rt}{\right}
\newcommand{\vho}{\varrho}
\newcommand{\thf}{\tfrac 12}
\newcommand{\bsl}{\backslash}
\newcommand{\ccg}{/ \! /}
\newcommand{\hG}{\widehat{G}}
\newcommand{\A}{\mathcal{A}}
\newcommand{\Z}{\mathbb{Z}}
\newcommand{\Q}{\mathbb{Q}}
\newcommand{\R}{\mathbb{R}}
\newcommand{\C}{\mathbb{C}}
\newcommand{\F}{\mathbb{F}}
\newcommand{\N}{\mathbb{N}}
\newcommand{\HH}{\mathbb{H}}
\newcommand{\ol}{\overline}
\newcommand{\scrI}{\mathcal{I}}
\newcommand{\scrP}{\mathcal{P}}
\newcommand{\cM}{\mathcal{M}}
\newcommand{\cE}{\mathcal{E}}
\newcommand{\cS}{\mathcal{S}}
\newcommand{\cL}{\mathcal{L}}
\newcommand{\mc}{\mathcal}
\newcommand{\cuspa}{\mathfrak{a}}
\newcommand{\cuspb}{\mathfrak{b}}


\newcommand{\sh}{h}
\newcommand{\sv}{{\sigma_v}}
\newcommand{\su}{{\sigma_u}}
\newcommand{\hf}{\frac 12}
\newcommand{\sigmaN}{{\sigma^{(N)}}}
\newcommand{\cO}{\mathcal{O}}
\newcommand{\cI}{\scrI}
\newcommand{\cF}{\mathcal{F}}
\newcommand{\cR}{\mathcal{R}}
\newcommand{\kt}{\frac k2}
\newcommand{\tE}{\tilde{E}}
\newcommand{\tM}{\tilde{M}}
\newcommand{\tZ}{\tilde{Z}}
\newcommand{\tOmega}{\tilde{\Omega}}
\newcommand{\<}{\left\langle}
\renewcommand{\>}{\right\rangle}


\DeclareMathOperator{\SL}{SL}
\DeclareMathOperator{\PSL}{PSL}
\DeclareMathOperator{\GL}{GL}
\DeclareMathOperator{\PGL}{PGL}
\DeclareMathOperator{\RR}{R}


\DeclareMathOperator{\Tr}{Tr}
\DeclareMathOperator{\OD}{OD}
\DeclareMathOperator{\Sk}{S_{k}}
\DeclareMathOperator{\ord}{{ord}}
\DeclareMathOperator{\Sym}{{Sym}}


\newcommand{\m}{\mathfrak{m}}
\newcommand{\sgn}{{\rm sign}} % sign


%%%
\newcommand{\finite}{{\rm finite}}
\newcommand{\cusp}{{\rm cusp}}
\newcommand{\cont}{{\rm cont}}
%\newcommand{\Res}{{\rm res}}
\newcommand{\Res}{{\rm Res}}
\newcommand{\main}{{\rm main}}
\newcommand{\rem}{{\rm rem}}
\newcommand{\intg}{{\rm int}}




\newcommand{\spec}{{\rm spec}}
\newcommand{\fin}{{\rm finite}}


\newcommand{\fund}{\mathfrak{F}}


\newcommand{\bnu}{\bar{\nu}}


\newcommand{\tth}{\tilde{h}}




% matrix
\newcommand{\sm}{\left(\begin{smallmatrix}}
\newcommand{\esm}{\end{smallmatrix}\right)}
\newcommand{\bpm}{\begin{pmatrix}}
\newcommand{\ebpm}{\end{pmatrix}}


\newtheorem{theorem}{Theorem}
\newtheorem{lemma}[theorem]{Lemma}
\newtheorem{proposition}[theorem]{Proposition}
\newtheorem{corollary}[theorem]{Corollary}
\newtheorem{definition}[theorem]{Definition}
\newtheorem{conjecture}[theorem]{Conjecture}
\theoremstyle{remark}
\newtheorem{remarks}[theorem]{Remarks}
\newtheorem{remark}[theorem]{Remark}
\newtheorem{examples}[theorem]{Examples}
\newtheorem{exercise}[theorem]{Exercise}
\numberwithin{theorem}{section}
\numberwithin{equation}{section}


\title{Subconvexity for second moments of $L$-functions twisted by characters}
\author{Nikos Diamantis}
\author{Jeff Hoffstein}
\author{Min Lee}
\author{Henry Twiss}


\begin{document}

\begin{abstract}
  
\end{abstract}

\maketitle
\tableofcontents


\section{Introduction}
  Our aim is to prove the following subconvexity result:

  \begin{theorem}
    Let $f$ be a cusp form on $\PSL_{2}(\mathbb{Z})\backslash\mathbb{H}$ of trivial character and let $\chi$ be a Dirichlet character modulo $Q$. Then
    \[
      L\left(\frac{1}{2},f \otimes \chi\right) \ll_{\epsilon} Q^{\frac{1}{3}+\frac{\theta}{3}+\epsilon},
    \]
    where $\theta$ is the best bound towards the Ramanujan-Petersson conjecture for Maass forms.
  \end{theorem}

\section{Preliminaries}
  \subsection*{Sums}
    Let $q \ge 1$ and let $\chi$ be a Dirichlet character modulo $q$. For $n \in \mathbb{Z}$, the Gauss sum associated to $\chi$ is
    \[
      c_{\chi}(n) = \sum_{a \mod{q}}\chi(a)e^{2\pi in\frac{a}{q}}.
    \]
    Then $\tau(\chi) = c_{\chi}(1)$ is the usual Gauss sum associated to $\chi$. When $\chi$ is primitive, we have
    \[
      c_{\chi}(m) = \overline{\chi}(m)\tau(\chi),
    \]
    and $|\tau(\chi)| = \sqrt{q}$.
  \subsection*{Forms}
  \subsection*{\texorpdfstring{$L$}{L}-functions}
\section{Shifted Dirichlet Series}
  \subsection*{The Shifted Dirichlet Series \texorpdfstring{$D_{f,g}(s;h,\ell_{1},\ell_{2})$}{}}
  \subsection*{The Shifted Dirichlet Series \texorpdfstring{$D_{f,v}(w;n,\ell_{1},\ell_{2})$}{}}
  \subsection*{The Multiple Dirichlet Series \texorpdfstring{$Z_{f,g}(s,v,u,\ell_{1},\ell_{2})$}{}}
\section{An Amplified Second Moment}
\section{Triple Shifted Sums}
  \subsection*{A Triple Shifted Sum}
  \subsection*{The Diagonal Contribution}
  \subsection*{The Off-diagonal Contribution}
  \subsection*{Balancing}

\end{document}

