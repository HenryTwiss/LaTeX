\chapter{Non-vanishing Results for Degree \texorpdfstring{$1$}{1} \texorpdfstring{$L$}{L}-functions}\label{ch:Selected_topics_in_L-functions}
  We will see a first example of how $L$-functions can be used to prove interesting arithmetic results. Namely, we prove a crowning gem of analytic number theory: Dirichlet's theorem on primes in arithmetic progressions. This result is a consequence of a non-vanishing result for Dirichlet $L$-functions at $s = 1$ and it gives sufficient motivation for the study of non-vanishing results in general. After proving Dirichlet's theorem, we prove two stronger non-vanishing theorems: one for the Riemann zeta function and the other for Dirichlet $L$-functions.
    \section{Dirichlet's Theorem on Primes in Arithmetic Progressions}
      One of the more well-known arithmetic results proved using $L$-functions is \textbf{Dirichlet's theorem on primes in arithmetic progressions}\index{Dirichlet's theorem on primes in arithmetic progressions}:

      \begin{theorem}[Dirichlet's theorem on primes in arithmetic progressions]\label{thm:Dirichlet's_theorem_on_primes_in_arithmetic_progressions}
        Let $a$ and $m$ be positive integers such that $(a,m) = 1$. Then the arithmetic progression $\{a+km \mid k \in \N\}$ contain infinitely many primes.
      \end{theorem}

      We will delay the proof for the moment, for it is well-worth understanding the some of the motivation behind why this theorem is interesting and how exactly Dirichlet used the analytic techniques of $L$-functions to attack this purely arithmetic statement. We being by recalling Euclid's famous theorem on the infinitude of the primes. Euclid's proof is completely elementary and arithmetic in nature. He argues that if there were finitely many primes $p_{1},p_{2},\ldots,p_{k}$ then a short consideration of $(p_{1}p_{2} \cdots p_{k})+1$ shows that this number must either be divisible by a prime not in our list or must be prime itself. As primes are the multiplicative building blocks of arithmetic, Euclid assures us that we have an ample amount of primes to work with. Now there is a slightly stronger result due to Euler (see \cite{euler1744variae}) requiring analytic techniques:

      \begin{theorem}\label{thm:reciprocial_sum_of_primes_diverges}
        The series
        \[
          \sum_{p}\frac{1}{p},
        \]
        diverges.
      \end{theorem}
      \begin{proof}
        For $\s > 1$, $\z(s)$ is holomorphic and admits the degree $1$ Euler product
        \[
          \z(s) = \prod_{p}(1-p^{-s})^{-1}.
        \]
        Taking the logarithm, we get
        \[
          \log\z(s) = -\sum_{p}\log(1-p^{-s}).
        \]
        The Taylor series of the logarithm gives
        \[
          \log(1-p^{-s}) = \sum_{k \ge 1}(-1)^{k-1}\frac{(-p^{-s})^{k}}{k} = \sum_{k \ge 1}(-1)^{2k-1}\frac{1}{kp^{ks}},
        \]
        so that
        \begin{equation}\label{equ:log_zeta_expansion}
          \log\z(s) = \sum_{p}\sum_{k \ge 1}\frac{1}{kp^{ks}}.
        \end{equation}
        The double sum restricted to $k \ge 2$ is uniformly bounded for $\s > 1$. To see this, first observe
        \[
          \left|\sum_{k \ge 2}\frac{1}{kp^{ks}}\right| \le \sum_{k \ge 2}\left|\frac{1}{kp^{ks}}\right| \le \sum_{k \ge 2}\left|\frac{1}{p^{ks}}\right| \le \sum_{k \ge 2}\frac{1}{p^{k}} = \frac{1}{p^{2}}\sum_{k \ge 0}\frac{1}{p^{k}} = \frac{1}{p^{2}}(1-p^{-1})^{-1} \le \frac{2}{p^{2}},
        \]
        where the last inequality follows because $p \ge 2$. Then
        \begin{equation}\label{equ:reciprocial_sum_of_primes_diverges_1}
          \left|\sum_{p}\sum_{k \ge 2}\frac{1}{kp^{ks}}\right| \le 2\sum_{p}\frac{1}{p^{2}} < 2\sum_{n \ge 1}\frac{1}{n^{2}} = 2\z(2).
        \end{equation}
        So by equation \cref{equ:log_zeta_expansion,equ:reciprocial_sum_of_primes_diverges_1},
        \[
          \left|\log\z(s)-\sum_{p}\frac{1}{p^{s}}\right| = \left|\sum_{p}\sum_{k \ge 2}\frac{1}{kp^{ks}}\right|,
        \]
        remains bounded as $s \to 1$. The claim now follows since $\z(s)$ has a simple pole at $s = 1$.
      \end{proof}

      \cref{thm:reciprocial_sum_of_primes_diverges} tells us that there are infinitely many primes, but also that the primes are not too ``sparse'' in the integers for otherwise the series would converge. The idea Dirichlet used to prove his result on primes in arithmetic progressions was in a very similar spirit. He sought out to prove the divergence of the series
      \[
        \sum_{p \equiv a \tmod{m}}\frac{1}{p},
      \]
      for positive integers $a$ and $m$ with $(a,m) = 1$ as the divergence immediately implies there are infinitely many primes $p$ of the form $p \equiv a \tmod{m}$. In the case $a = 1$ and $m = 2$ we recover \cref{thm:reciprocial_sum_of_primes_diverges} exactly since every prime is odd. Dirichlet's proof proceeds in a similar way to that of \cref{thm:reciprocial_sum_of_primes_diverges} and this is where Dirichlet used what are now known as Dirichlet characters and Dirichlet $L$-functions. The proof can be broken into three steps. The first is to proceed as Euler did, but with the Dirichlet $L$-function $L(s,\chi)$ where $\chi$ has modulus $m$. That is, write $L(s,\chi)$ as a sum over primes and a bounded term as $s \to 1$. The next step is to use the orthogonality relations of the characters to sieve out the correct sum. The last step is to show the non-vanishing result $L(1,\chi) \neq 0$ for all non-principal characters $\chi$. This is the essential part of the proof as it is what assures us that the sum diverges. We will prove this non-vanishing result first and then prove Dirichlet's theorem on primes in arithmetic progressions.

      \begin{theorem}\label{thm:non-vanishing_of_Dirichlet_L-functions_at_s=1}
        For any non-principal Dirichlet character $\chi$, $L(1,\chi) \neq 0$.
      \end{theorem}
      \begin{proof}
        Choose a positive integer $m > 1$. It will be enough to prove this for all Dirichlet characters $\chi$ modulo $m$. We establish a preliminary result first. We claim there exist positive integers $f_{p}$ and $g_{p}$ with $f_{p}g_{p} = \vphi(m)$ such that
        \[
          \prod_{\chi}L(s,\chi) = \prod_{p \nmid m}(1-p^{-f_{p}s})^{-g_{p}}.
        \]
        To see this, the map $\chi \to \chi(p)$ is a homomorphism from the group of Dirichlet characters modulo $m$ into $\mu_{m}$ the group of $m$-th roots of unity. Since $\mu_{m}$ is cyclic, the image of this map is a cyclic group of order say $f_{p}$. In other words, the image is exactly $\mu_{f_{p}}$. Letting $g_{p}$ be the order of the kernel, $f_{p}g_{p} = \vphi(m)$ because $X_{m} \cong (\Z/m\Z)^{\ast}$. In other words, for every $\w \in \mu_{f_{p}}$ there are $g_{p}$ characters $\chi$ such that $\chi(p) = \w$. So for fixed $p \nmid m$, we compute
        \begin{equation}\label{equ:non-vanishing_of_Dirichlet_L-functions_at_s=1_1}
          \prod_{\chi}(1-\chi(p)p^{-s}) = \prod_{\w \in \mu_{m}}(1-\w p^{-s})^{g_{p}} = (1-p^{-f_{p}s})^{g_{p}},
        \end{equation}
        where the last equality follows since the product is over all $f_{p}$-th roots of unity. Then \cref{equ:non-vanishing_of_Dirichlet_L-functions_at_s=1_1} and the Euler product for $L(s,\chi)$ together imply
        \begin{equation}\label{equ:non-vanishing_of_Dirichlet_L-functions_at_s=1_2}
          \prod_{\chi}L(s,\chi) = \prod_{\chi}\prod_{p \nmid m}(1-\chi(p)p^{-s})^{-1} = \prod_{p \nmid m}\prod_{\chi}(1-\chi(p)p^{-s})^{-1} = \prod_{p \nmid m}(1-p^{-f_{p}s})^{-g_{p}}.
        \end{equation}
        This establishes the preliminary result. Upon expanding the last product in \cref{equ:non-vanishing_of_Dirichlet_L-functions_at_s=1_2}, we see that $\prod_{\chi}L(s,\chi)$ defines a Dirichlet series with positive coefficients and constant term $1$. Therefore it takes positive values larger than $1$ along the part of the real line in the region $\s > 1$. We will now show $L(1,\chi) \neq 0$ for non-principal $\chi$, and we will separate the cases $\chi$ is real or complex. Suppose $\chi$ is complex. If $L(1,\chi) = 0$ then the functional equation for $L(s,\wtilde{\chi})$, where $\wtilde{\chi}$ is the primitive character inducing $\chi$, implies $L(1,\cchi) = 0$. Now on the other hand, $L(s,\chi_{m,0})$ has a simple pole at $s = 1$ (coming from the $\z(s)$ factor) so altogether $\prod_{\chi}L(s,\chi)$ is zero at $s = 1$. This contradicts that it takes positive values larger than $1$ along the part of the real line in the region $\s > 1$ and so $L(1,\chi) \neq 0$. Now suppose $\chi$ is real and consider
        \[
          \frac{L(s,\chi_{m,0})L(s,\chi)}{L(2s,\chi_{m,0})} = \prod_{p \nmid m}\frac{(1-p^{-s})^{-1}(1-\chi(p)p^{-s})^{-1}}{(1-p^{-2s})^{-1}}.
        \]
        If $\chi(p) = -1$ then the corresponding factor on the right-hand side is $1$. If $\chi(p) = 1$, then
        \[
          \frac{(1-p^{-s})^{-1}(1-\chi(p)p^{-s})^{-1}}{(1-p^{-2s})^{-1}} = \frac{(1-p^{-s})^{-2}}{(1-p^{-2s})^{-1}} = \frac{(1+p^{-s})}{(1-p^{-s})} = 1+2\sum_{k \ge 1}\frac{1}{p^{ks}}.
        \]
        These facts together imply that $\frac{L(s,\chi_{m,0})L(s,\chi)}{L(2s,\chi_{m,0})}$ defines a Dirichlet series with positive coefficients and constant term $1$. Therefore it takes positive values larger than $1$ along the part of the real line in the region $\s > 1$. If $L(1,\chi) = 0$ then $\frac{L(s,\chi_{m,0})L(s,\chi)}{L(2s,\chi_{m,0})}$ is zero at $s = 1$ because the zero of $L(s,\chi)$ cancels the simple pole of $L(s,\chi_{m,0})$ at $s = 1$ and $L(2,\chi_{m,0}) \neq 0$ because $L(s,\chi_{m,0})$ is defined by a Dirichlet series with positive coefficients in the region $\s > 1$. As in the complex case, this gives a contradiction. So we have shown $L(1,\chi) \neq 0$ for all non-principal $\chi$ which completes the proof.
      \end{proof}

      We now have enough machinery to prove Dirichlet's theorem on primes in arithmetic progressions:

      \begin{proof}[Proof of Dirichlet's theorem on primes in arithmetic progressions]
          Let $\chi$ be a Dirichlet character modulo $m$. Then for $\s > 1$, $L(s,\chi)$ is holomorphic and admits the degree $1$ Euler product
          \[
            L(s,\chi) = \prod_{p}(1-\chi(p)p^{-s})^{-1}.
          \]
          Taking the logarithm gives
          \[
            \log L(s,\chi) = -\sum_{p}\log(1-\chi(p)p^{-s}).
          \]
          The Taylor series of the logarithm implies
          \[
            \log(1-\chi(p)p^{-s}) = \sum_{k \ge 1}(-1)^{k-1}\frac{(-\chi(p)p^{-s})^{k}}{k} = \sum_{k \ge 1}(-1)^{2k-1}\frac{\chi(p^{k})}{kp^{ks}},
          \]
          so that
          \begin{equation}\label{equ:log_Dirichlet_L-function_expansion}
            \log L(s,\chi) = \sum_{p}\sum_{k \ge 1}\frac{\chi(p^{k})}{kp^{ks}}.
          \end{equation}
          The double sum restricted to $k \ge 2$ is uniformly bounded for $\s > 1$. Indeed, first observe
          \[
            \left|\sum_{k \ge 2}\frac{\chi(p^{k})}{kp^{ks}}\right| \le \sum_{k \ge 2}\left|\frac{\chi(p^{k})}{kp^{ks}}\right| \le \sum_{k \ge 2}\left|\frac{1}{p^{ks}}\right| \le \sum_{k \ge 2}\frac{1}{p^{k}} = \frac{1}{p^{2}}\sum_{k \ge 0}\frac{1}{p^{k}} = \frac{1}{p^{2}}(1-p^{-1})^{-1} \le \frac{2}{p^{2}},
          \]
          where the last inequality follows because $p > 2$. Then
          \[
            \left|\sum_{p}\sum_{k \ge 2}\frac{\chi(p^{k})}{kp^{ks}}\right| \le 2\sum_{p}\frac{1}{p^{2}} < 2\sum_{n \ge 1}\frac{1}{n^{2}} = 2\z(2),
          \]
          as desired. Now using \cref{equ:log_Dirichlet_L-function_expansion}, we have
          \begin{equation}\label{equ:Dirichlet's_theorem_on_primes_in_arithmetric_progressions_1}
            \sum_{\chi \tmod{m}}\conj{\chi(a)}\log L(s,\chi) = \sum_{\chi \tmod{m}}\sum_{p}\frac{\conj{\chi(a)}\chi(p)}{p^{s}}+\sum_{\chi \tmod{m}}\conj{\chi(a)}\sum_{p}\sum_{k \ge 2}\frac{\chi(p^{k})}{kp^{ks}}.
          \end{equation}
          By the orthogonality relations (\cref{prop:Dirichlet_orthogonality_relations} (ii)), we find that
          \begin{equation}\label{equ:Dirichlet's_theorem_on_primes_in_arithmetric_progressions_2}
            \sum_{\chi \tmod{m}}\sum_{p}\frac{\conj{\chi(a)}\chi(p)}{p^{s}} = \sum_{p}\frac1{p^{s}}\sum_{\chi \tmod{m}}\conj{\chi(a)}\chi(p) = \vphi(m)\sum_{p \equiv{a} \tmod{m}}\frac{1}{p^{s}},
          \end{equation}
          and so combining \cref{equ:Dirichlet's_theorem_on_primes_in_arithmetric_progressions_1,equ:Dirichlet's_theorem_on_primes_in_arithmetric_progressions_2} gives
          \[
            \sum_{\chi \tmod{m}}\conj{\chi(a)}\log L(s,\chi)-\sum_{\chi \tmod{m}}\conj{\chi(a)}\sum_{p}\sum_{k \ge 2}\frac{\chi(p^{k})}{kp^{ks}} = \vphi(m)\sum_{p \equiv{a} \tmod{m}}\frac{1}{p^{s}}.
          \]
          The latter sum on the left-hand side is uniformly bounded for $\s > 1$ because the inner double sum is and there are finitely many Dirichlet characters modulo $m$. Therefore it suffices to show that the first sum on the left-hand side diverges as $s \to 1$. For $\chi = \chi_{m,0}$,
          \[
            L(s,\chi_{m,0}) = \z(s)\prod_{p \mid m}(1-p^{-s}).
          \]
          So the corresponding term in the sum is
          \[
            \conj{\chi_{m,0}}(a)\log L(s,\chi_{m,0}) = \log L(s,\chi_{m,0}) = \log\left(\z(s)\prod_{p \mid m}(1-p^{-s})\right) = \log\z(s)+\sum_{p \mid m}\log(1-p^{-s}),
          \]
          which diverges as $s \to 1$ because $\z(s)$ has a simple pole at $s = 1$. We will be done if $\log L(s,\chi)$ remains bounded as $s \to 1$ for all $\chi \neq \chi_{m,0}$. So assume $\chi$ is not principal. Then if $\wtilde{\chi}$ is the character inducing $\chi$, we have
          \[
            L(s,\chi) = L(s,\wtilde{\chi})\prod_{p \mid m}(1-\wtilde{\chi}(p)p^{-s}),
          \]
          where $L(s,\wtilde{\chi})$ is holomorphic. Therefore $L(s,\chi)$ is holomorphic too so it further suffices to show $L(1,\chi) \neq 0$. This follows from \cref{thm:non-vanishing_of_Dirichlet_L-functions_at_s=1} and thus the proof is complete.
      \end{proof}

      We know from \cref{thm:non-vanishing_of_Dirichlet_L-functions_at_s=1} that $L(1,\chi) \neq 0$. It is interesting to know whether or not this value is computable in general. Indeed it is. The computation is fairly straightforward and only requires some basic properties of Gauss sums that we have already developed. The idea is to rewrite the character values $\chi(n)$ so that we can collapse the infinite series into a Taylor series. Our result is the following:
      
      \begin{theorem}\label{thm:Value_of_Dirichlet_L-functions_at_s=1}
        Let $\chi$ be a primitive Dirichlet character with conductor $q > 1$. Then
        \[
          L(1,\chi) = -\frac{\chi(-1)\tau(\chi)}{q}\psum_{a \tmod{q}}\chi(a)\log\left(\sin\left(\frac{\pi a}{q}\right)\right) \quad \text{or} \quad L(1,\chi) = -\frac{\chi(-1)\tau(\chi)\pi i}{q^{2}}\psum_{a \tmod{q}}\chi(a)a,
        \]
        according to whether $\chi$ is even or odd.
      \end{theorem}
      \begin{proof}
        Make the following computation:
        \begin{align*}
          \chi(n) &= \frac{1}{\conj{\tau(\chi)}}\conj{\tau(n,\chi)} && \text{\cref{cor:gauss_sum_primitive_formula}} \\
          &= \frac{1}{\tau(\cchi)}\tau(n,\chi) && \text{\cref{prop:Gauss_sum_reduction} (i)} \\
          &= \frac{\tau(\chi)}{\tau(\chi)\tau(\cchi)}\tau(n,\chi) \\
          &= \frac{\chi(-1)\tau(\chi)}{q}\tau(n,\chi) && \text{\cref{thm:Gauss_sum_modulus,prop:epsilon_factor_relationship}} \\
          &= \frac{\chi(-1)\tau(\chi)}{q}\psum_{a \tmod{q}}\chi(a)e^{\frac{2\pi ian}{q}}.
        \end{align*}
        Substituting this result into the definition of $L(1,\chi)$, we find that
        \begin{equation}\label{equ:value_of_Dirichlet_L-functions_at_s=1_1}
          \begin{aligned}
            L(1,\chi) &= \sum_{n \ge 1}\frac{1}{n}\left(\frac{\chi(-1)\tau(\chi)}{q}\psum_{a \tmod{q}}\chi(a)e^{\frac{2\pi ian}{q}}\right) \\
            &= \frac{\chi(-1)\tau(\chi)}{q}\psum_{a \tmod{q}}\chi(a)\sum_{n \ge 1}\frac{e^{\frac{2\pi ian}{q}}}{n} \\
            &= \frac{\chi(-1)\tau(\chi)}{q}\psum_{a \tmod{q}}\chi(a)\log\left(\left(1-e^{\frac{2\pi ia}{q}}\right)^{-1}\right),
          \end{aligned}
        \end{equation}
        where in the last line we have used the Taylor series of the logarithm (notice $a \neq q$ so that $e^{\frac{2\pi ia}{q}} \neq 1$ and hence the logarithm is defined). We have now expressed $L(1,\chi)$ as a finite sum. In order to simplify the last expression in \cref{equ:value_of_Dirichlet_L-functions_at_s=1_1}, we deal with the logarithm. Since $\sin(\t) = \frac{e^{i\t}-e^{-i\t}}{2i}$, we have
        \[
          1-e^{\frac{2\pi ia}{q}} = -2ie^{\frac{\pi ia}{q}}\left(\frac{e^{\frac{\pi ia}{q}}-e^{-\frac{\pi ia}{q}}}{2i}\right) = -2ie^{\frac{\pi ia}{q}}\sin\left(\frac{\pi a}{q}\right).
        \]
        Therefore the last expression in \cref{equ:value_of_Dirichlet_L-functions_at_s=1_1} becomes
        \[
          \frac{\chi(-1)\tau(\chi)}{q}\psum_{a \tmod{q}}\chi(a)\log\left(\left(-2ie^{\frac{\pi ia}{q}}\sin\left(\frac{\pi a}{q}\right)\right)^{-1}\right).
        \]
        As $0< a < q$, we have $0 < \frac{\pi a}{q} < \pi$ so that $\sin\left(\frac{\pi a}{q}\right)$ is never negative. Therefore we can split up the logarithm term and obtain
        \[
          -\frac{\chi(-1)\tau(\chi)}{q}\left(\log(-2i)\psum_{a \tmod{q}}\chi(a)+\frac{\pi i}{q}\psum_{a \tmod{q}}\chi(a)a+\psum_{a \tmod{q}}\chi(a)\log\left(\sin\left(\frac{\pi a}{q}\right)\right)\right).
        \]
        By the orthogonality relations (\cref{cor:Dirichlet_orthogonality_relations} (i)), the first sum above vanishes. Therefore
        \begin{equation}\label{equ:value_of_Dirichlet_L-functions_at_s=1_2}
          L(1,\chi) = -\frac{\chi(-1)\tau(\chi)}{q}\left(\frac{\pi i}{q}\psum_{a \tmod{q}}\chi(a)a+\psum_{a \tmod{q}}\chi(a)\log\left(\sin\left(\frac{\pi a}{q}\right)\right)\right).
        \end{equation}
        \cref{equ:value_of_Dirichlet_L-functions_at_s=1_2} simplifies in that one of the two sums vanish depending on if $\chi$ is even or odd. For the first sum in \cref{equ:value_of_Dirichlet_L-functions_at_s=1_2}, observe that
        \[
          \frac{\pi i}{q}\psum_{a \tmod{q}}\chi(a)a = -\frac{\chi(-1)\pi i}{q}\psum_{a \tmod{q}}\chi(-a)(-a),
        \]
        which vanishes if $\chi$ is even. For the second sum in \cref{equ:value_of_Dirichlet_L-functions_at_s=1_2}, we have an analogous relation of the form
        \[
          \psum_{a \tmod{q}}\chi(a)\log\left(\sin\left(\frac{\pi a}{q}\right)\right) = \chi(-1)\psum_{a \tmod{q}}\chi(-a)\log\left(\sin\left(\frac{\pi a}{q}\right)\right),
        \]
        which vanishes if $\chi$ is odd. This finishes the proof.
      \end{proof}

      \cref{thm:Value_of_Dirichlet_L-functions_at_s=1} encodes some interesting identities. For example, if $\chi$ is the non-principal Dirichlet character modulo $4$, then $\chi$ is uniquely defined by $\chi(1) = 1$ and $\chi(3) = \chi(-1) = -1$. In particular, $\chi$ is odd and its conductor is $4$. Now
      \[
        \tau(\chi) = \psum_{a \tmod{4}}\chi(a)e^{\frac{2\pi ia}{4}} = e^{\frac{2\pi i}{4}}-e^{\frac{6\pi i}{4}} = i-(-i) = 2i,
      \]
      so by \cref{thm:Value_of_Dirichlet_L-functions_at_s=1} we get
      \[
        L(1,\chi) = -\frac{\chi(-1)\tau(\chi)\pi i}{16}(1-3) = \frac{\pi}{4}.
      \]
      Expanding out $L(1,\chi)$ gives
      \[
        1-\frac{1}{3}+\frac{1}{5}-\frac{1}{7}+\cdots = \frac{\pi}{4},
      \]
      which is the famous \textbf{Madhava–Leibniz formula}\index{Madhava–Leibniz formula} for $\pi$.
    \section{Non-vanishing on \texorpdfstring{$\s = 1$}{s = 1}}
      Here we provide proofs that the Riemann zeta function and Dirichlet $L$-functions do not vanish on the line $\s = 1$. The second of these two results can be regarded as a stronger version of \cref{thm:non-vanishing_of_Dirichlet_L-functions_at_s=1}. While both will play a role in understanding the zeros of these $L$-functions, the non-vanishing result for the Riemann zeta function is the key ingredient in the proof of the prime number theorem. We will proof the non-vanishing result for $\z(s)$ first, but we need a lemma that will be immensely useful in other investigations:

      \begin{lemma}\label{lem:zero-free_region_zeta_lemma}
        For $\s > 1$ and nonzero $t$,
        \[
          \Re\left(\frac{\eta'}{\eta}(\s)\right) = \Re\left(3\frac{\z'}{\z}(\s)+4\frac{\z'}{\z}(\s+it)+\frac{\z'}{\z}(\s+2it)\right) \le 0,
        \]
        where
        \[
          \eta(s) = \z(s)^{3}\z(s+it)^{4}\z(s+2it).
        \]
      \end{lemma}
      \begin{proof}
        In the region $\s > 1$, $\z(s)$ is holomorphic and admits the degree $1$ Euler product
        \[
          \z(s) = \prod_{p}(1-p^{-s})^{-1}.
        \]
        Taking the logarithmic derivative of $\z(s)$ gives
        \begin{equation}\label{equ:Drichlet_series_log_derivative_zeta}
          \frac{\z'}{\z}(s) = -\sum_{p}\frac{\log(p)p^{-s}}{1-p^{-s}} = -\sum_{p}\sum_{k \ge 1}\frac{\log(p)}{p^{ks}} = -\sum_{n \ge 1}\frac{\L(n)}{n^{s}}.
        \end{equation}
        Then observe that
        \begin{align*}
          \sum_{n \ge 1}\frac{\L(n)}{n^{s}} &= \sum_{n \ge 1}\frac{\L(n)}{n^{\s}n^{it}} \\
          &= \sum_{n \ge 1}\frac{\L(n)}{n^{\s}}e^{-it\log(n)} \\
          &= \sum_{n \ge 1}\frac{\L(n)}{n^{\s}}\bigg(\cos(t\log(n))-i\sin(t\log(n))\bigg).
        \end{align*}
        We conclude
        \begin{equation}\label{equ:non-vanishing_of_zeta_on_Re(s)=1_1}
          \Re\left(\frac{\z'}{\z}(s)\right) = -\sum_{n \ge 1}\frac{\L(n)}{n^{\s}}\cos(t\log(n)).
        \end{equation}
        Since $\cos(2\t) = 2\cos(\t)-1$ for any $\t$, we have
        \begin{equation}\label{equ:cosine_inequality_for_analytic_number_theory}
          3+4\cos(\t)+\cos(2\t) = 2(1+\cos(\t))^{2} \ge 0,
        \end{equation}
        provided $\t$ is real. As $\frac{\L(n)}{n^{\s}} \ge 0$ for all $n \ge 1$, \cref{equ:cosine_inequality_for_analytic_number_theory} implies
        \begin{equation}\label{equ:equ:non-vanishing_of_zeta_on_Re(s)=1_2}
          \sum_{n \ge 1}\frac{\L(n)}{n^{\s}}\bigg(3+4\cos(t\log(n))+\cos(2t\log(n))\bigg) \ge 0.
        \end{equation}
        Taking the real part of the logarithmic derivative of $\eta(\s)$, \cref{equ:non-vanishing_of_zeta_on_Re(s)=1_1,equ:equ:non-vanishing_of_zeta_on_Re(s)=1_2} together imply
        \[
           \Re\left(\frac{\eta'}{\eta}(\s)\right) = \Re\left(3\frac{\z'}{\z}(\s)+4\frac{\z'}{\z}(\s+it)+\frac{\z'}{\z}(\s+2it)\right) \le 0.
        \]
      \end{proof}

      The non-vanishing result follows very easily from \cref{lem:zero-free_region_zeta_lemma}:

      \begin{theorem}\label{thm:non-vanishing_of_zeta_on_Re(s)=1}
        $\z(s) \neq 0$ on the line $\s = 1$.
      \end{theorem}
      \begin{proof}
        We will now show $\z(s) \neq 0$ on the line $\s = 1$. We may assume $s \neq 1$ because we know $\z(s)$ has a simple pole there. So fix a nonzero $t$ and consider
        \[
          \eta(s) = \z(s)^{3}\z(s+it)^{4}\z(s+2it).
        \]
        Suppose $\z(1+it) = 0$. Then $\eta(s)$ would have a zero at $s = 1$, either simple or of order $2$, depending on if $\z(s+2it)$ was zero or not since $\z(s)^{3}(s+it)^{4}$ has a simple zero at $s = 1$. Therefore it suffices to show $\eta(s)$ is nonzero at $s = 1$. Let $d$ be the order of the zero of $\eta(s)$ at $s = 1$. Note that $d \ge 1$. Then $\eta(s) = (s-1)^{d}\eta_{1}(s)$ with $\eta_{1}(s)$ holomorphic and nonzero at $s = 1$. Upon taking the logarithmic derivative, we obtain
        \[
          \frac{\eta'}{\eta}(s) = \frac{d}{s-1}+\frac{\eta_{1}'}{\eta_{1}}(s).
        \]
        Then for $\s > 1$, it follows that
        \[
          \lim_{\s \to 1}(\s-1)\frac{\eta'}{\eta}(\s) = d.
        \]
        But from \cref{lem:zero-free_region_zeta_lemma}, $(\s-1)\frac{\eta'}{\eta}(\s)$ has nonpositive real part and so the limit cannot be $d$. This is a contradiction. Therefore $\eta(s)$ is nonzero at $s = 1$ and thus $\z(1+it) \neq 0$.
      \end{proof}

      With \cref{thm:non-vanishing_of_zeta_on_Re(s)=1}, we can make a slight improvement about the nontrivial zeros of $\z(s)$. Indeed, $\z(s)$ has no zeros for $\s > 1$ since it is an Euler product there and \cref{thm:non-vanishing_of_zeta_on_Re(s)=1} improves that region to $\s \ge 1$. We can say a little more. Rewrite the functional equation for $\z(s)$ as
      \[
        \z(s) = \pi^{s-\frac{1}{2}}\frac{\G\left(\frac{1-s}{2}\right)}{\G\left(\frac{s}{2}\right)}\z(1-s).
      \]
      Then \cref{thm:non-vanishing_of_zeta_on_Re(s)=1} implies that there are no zeros on the line $\s = 0$ either. As for other $L$-functions, there is also a completely analogous lemma and argument for Dirichlet $L$-functions:
      
      \begin{lemma}\label{lem:zero-free_region_Dirichlet_lemma}
        For any non-principal Dirichlet character $\chi$ of conductor $q > 1$, $\s > 1$, and nonzero $t$,
        \[
          \Re\left(\frac{\eta'}{\eta}(\s,\chi)\right) = \Re\left(3\frac{L'}{L}(\s,\chi_{q,0})+4\frac{L'}{L}(\s+it,\chi)+\frac{L'}{L}(\s+2it,\chi^{2})\right) \le 0,
        \]
        where
        \[
          \eta(s,\chi) = L(s,\chi_{q,0})^{3}L(s+it,\chi)^{4}L(s+2it,\chi^{2}).
        \]
      \end{lemma}
      \begin{proof}
        In the region $\s > 1$, $L(s,\chi)$ is holomorphic and admits the degree $1$ Euler product
        \[
          L(s,\chi) = \prod_{p}(1-\chi(p)p^{-s})^{-1}.
        \]
        Taking the logarithmic derivative of $L(s,\chi)$ yields
        \begin{equation}\label{equ:Drichlet_series_log_derivative_Dirichlet}
          \frac{L'}{L}(s,\chi) = -\sum_{p}\frac{\chi(p)\log(p)p^{-s}}{1-\chi(p)p^{-s}} = -\sum_{p}\sum_{k \ge 1}\frac{\chi(p^{k})\log(p)}{p^{ks}} = -\sum_{n \ge 1}\frac{\chi(n)\L(n)}{n^{s}}.
        \end{equation}
        Writing $\chi(n) = e^{i\t}$, we have
        \begin{align*}
          \sum_{n \ge 1}\frac{\chi(n)\L(n)}{n^{s}} &= \sum_{n \ge 1}\frac{\chi(n)\L(n)}{n^{\s}n^{it}} \\
          &= \sum_{n \ge 1}\frac{\L(n)}{n^{\s}}e^{i(\t-t\log(n))} \\
          &= \sum_{n \ge 1}\frac{\L(n)}{n^{\s}}\bigg(\cos(\t-t\log(n))+i\sin(\t-t\log(n))\bigg).
        \end{align*}
        It follows that
        \begin{equation}\label{equ:zero-free_region_Dirichlet_lemma_1}
          \Re\left(\frac{L'}{L}(s,\chi)\right) = -\sum_{n \ge 1}\frac{\L(n)}{n^{\s}}\cos(\t-t\log(n)).
        \end{equation}
        Because $\chi^{2}(n) = e^{2i\t}$ and $\chi_{q,0}(n) = 1$, we obtain analogous equations to \cref{equ:zero-free_region_Dirichlet_lemma_1}:
        \begin{equation}\label{equ:zero-free_region_Dirichlet_lemma_2}
          \Re\left(\frac{L'}{L}(s,\chi^{2})\right) = -\sum_{n \ge 1}\frac{\L(n)}{n^{\s}}\cos(2\t-t\log(n)),
        \end{equation}
        and
        \begin{equation}\label{equ:zero-free_region_Dirichlet_lemma_3}
          \Re\left(\frac{L'}{L}(s,\chi_{q,0})\right) = -\sum_{n \ge 1}\frac{\L(n)}{n^{\s}}\cos(t\log(n)).
        \end{equation}
        Since $\frac{\L(n)}{n^{\s}} \ge 0$ for all $n \ge 1$, \cref{equ:cosine_inequality_for_analytic_number_theory} implies
        \begin{equation}\label{equ:zero-free_region_Dirichlet_lemma_4}
          \sum_{n \ge 1}\frac{\L(n)}{n^{\s}}\bigg(3+4\cos(\t-t\log(n))+4\cos(2\t-2t\log(n))\bigg) \ge 0.
        \end{equation}
        Upon taking the real part of the logarithmic derivative of $\eta(\s,\chi)$, \cref{equ:zero-free_region_Dirichlet_lemma_1,equ:zero-free_region_Dirichlet_lemma_2,equ:zero-free_region_Dirichlet_lemma_3,equ:zero-free_region_Dirichlet_lemma_4} together imply
        \[
          \Re\left(\frac{\eta'}{\eta}(\s,\chi)\right) = \Re\left(3\frac{L'}{L}(\s,\chi_{q,0})+4\frac{L'}{L}(\s+it,\chi)+\frac{L'}{L}(\s+2it,\chi^{2})\right) \le 0.
        \]
      \end{proof}

      The non-vanishing result for $L(s,\chi)$ is proven in a manner similar to $\z(s)$ using \cref{lem:zero-free_region_Dirichlet_lemma}:

      \begin{theorem}\label{thm:non-vanishing_of_Dirichlet_on_Re(s)=1}
        For any non-principal Dirichlet character $\chi$, $L(s,\chi) \neq 0$ on the line $\s = 1$.
      \end{theorem}
      \begin{proof}
        Fix a nonzero $t$ and consider
        \[
          \eta(s,\chi) = L(s,\chi_{q,0})^{3}L(s+it,\chi)^{4}L(s+2it,\chi^{2}),
        \]
        Recall that $L(s,\chi_{q,0})$ has a simple pole at $s = 1$. Now suppose $L(1+it,\chi) = 0$. Then at $s = 1$, $\eta(s,\chi)$ would have a zero, either simple or of order $2$, depending on if $L(s+2it,\chi)$ was zero or not since $L(s,\chi_{q,0})^{3}L(s+it,\chi)^{4}$ has a simple zero at $s = 1$. Therefore we need to show $\eta(s,\chi)$ is nonzero at $s = 1$. Let $d$ be the order of the zero of $\eta(s,\chi)$ at $s = 1$. Then $d \ge 1$ and we can write $\eta(s,\chi) = (s-1)^{d}\eta_{1}(s,\chi)$ with $\eta_{1}(s,\chi)$ holomorphic and nonzero at $s = 1$. Taking the logarithmic derivative, we find
        \[
          \frac{\eta'}{\eta}(s,\chi) = \frac{d}{s-1}+\frac{\eta_{1}'}{\eta_{1}}(s,\chi).
        \]
        Then for $\s > 1$, we have
        \[
          \lim_{\s \to 1}(\s-1)\frac{\eta'}{\eta}(\s,\chi) = d.
        \]
        But from \cref{lem:zero-free_region_Dirichlet_lemma}, $(\s-1)\frac{\eta'}{\eta}(\s,\chi)$ has nonpositive real part and so the limit cannot be $d$. This gives a contradiction. Hence $\eta(s,\chi)$ is nonzero at $s = 1$ and so $L(1+it,\chi) \neq 0$.
      \end{proof}

      Just as for the Riemann zeta function, \cref{thm:non-vanishing_of_Dirichlet_on_Re(s)=1} gives the improvement that $L(s,\chi)$ has no zeros in region $\s \ge 1$ from the region $\s > 1$. Moreover, if we rewrite the functional equation for $L(s,\chi)$ as
      \[
        L(s,\chi) = \frac{\e_{\chi}}{i^{\mf{a}}}q^{\frac{1}{2}-s}\pi^{s-\frac{1}{2}}\frac{\G\left(\frac{(1-s)+\mf{a}}{2}\right)}{\G\left(\frac{s+\mf{a}}{2}\right)}L(1-s,\cchi),
      \]
      then \cref{thm:non-vanishing_of_zeta_on_Re(s)=1} implies that there are no zeros on the line $\s = 0$.