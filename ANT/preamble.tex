%===============================%
%  Packages and basic settings  %
%===============================%
\usepackage[headheight=15pt,rmargin=0.5in,lmargin=0.5in,tmargin=0.75in,bmargin=0.75in]{geometry}
\usepackage{fancyhdr}
\usepackage{imakeidx}
\usepackage{framed}
\usepackage{amssymb}
\usepackage{amsmath}
\usepackage{mathrsfs}
\usepackage{enumitem}
\usepackage{multirow}
\usepackage{hyperref}
\usepackage[capitalise,noabbrev]{cleveref}
\usepackage{appendix}
\usepackage[hyperref,amsthm,amsmath,thref,framed,thmmarks]{ntheorem}
\usepackage{tikz}
\usepackage{tikz-cd}
\usepackage{nomencl}\makenomenclature
\usetikzlibrary{braids,arrows,decorations.markings,calc}

%=======================%
%  Book style settings  %
%=======================%
\pagestyle{fancy}
\fancyhf{}
\fancyhead[L]{\nouppercase{\leftmark}}
\fancyfoot[C]{\thepage}
\setlength\parindent{0pt}
\raggedbottom

%====================================%
%  Theorems, environments & cleveref  %
%====================================%
\theoremstyle{plain}\newtheorem{theorem}{Theorem}[section]
\theoremstyle{nonumberplain}\renewtheorem{theorem*}{Theorem}
\theoremstyle{plain}\newtheorem{proposition}{Proposition}[section]
\theoremstyle{nonumberplain}\renewtheorem{proposition*}{Proposition}
\theoremstyle{plain}\newtheorem{corollary}{Corollary}[section]
\theoremstyle{nonumberplain}\renewtheorem{corollary*}{Corollary}
\theoremstyle{plain}\newtheorem{lemma}{Lemma}[section]
\theoremstyle{nonumberplain}\renewtheorem{lemma*}{Lemma}
\theoremstyle{plain}\newtheorem{conjecture}{Conjecture}[section]
\theoremstyle{nonumberplain}\renewtheorem{conjecture*}{Conjecture}
\theoremstyle{plain}\newtheorem{remark}{Remark}[section]
\theoremstyle{nonumberplain}\renewtheorem{remark*}{Remark}
\theoremstyle{plain}\newtheorem{problem}{Open Problem}[section]
\theoremstyle{nonumberplain}\renewtheorem{problem*}{Open Problem}
\crefname{conjecture}{Conjecture}{Conjectures}

\newenvironment{stabular}[2][1]
  {\def\arraystretch{#1}\tabular{#2}}
  {\endtabular}

%==================================%
%  Custom commands & environments  %
%==================================%
\newcommand{\legendre}[2]{\left(\frac{#1}{#2}\right)}
\newcommand{\dlegendre}[2]{\displaystyle{\left(\frac{#1}{#2}\right)}}
\newcommand{\tlegendre}[2]{\textstyle{\left(\frac{#1}{#2}\right)}}
\newcommand{\psum}{\sideset{}{'}\sum}
\newcommand{\asum}{\sideset{}{^{\ast}}\sum}
\newcommand{\tmod}[1]{\ (\mathrm{mod}\text{ }#1)}
\renewcommand{\bmod}[1]{\ \left(\mathrm{mod}\text{ }#1\right)}
\newcommand{\xto}[1]{\xrightarrow{#1}}
\newcommand{\xfrom}[1]{\xleftarrow{#1}}
\newcommand{\normal}{\mathrel{\unlhd}}
\newcommand{\mf}{\mathfrak}
\newcommand{\mc}{\mathcal}
\newcommand{\ms}{\mathscr}

\newcommand{\Mat}{\mathrm{Mat}}
\newcommand{\GL}{\mathrm{GL}}
\newcommand{\SL}{\mathrm{SL}}
\newcommand{\PSL}{\mathrm{PSL}}
\renewcommand{\O}{\mathrm{O}}
\newcommand{\SO}{\mathrm{SO}}
\newcommand{\U}{\mathrm{U}}
\newcommand{\Sp}{\mathrm{Sp}}

\newcommand{\N}{\mathbb{N}}
\newcommand{\Z}{\mathbb{Z}}
\newcommand{\Q}{\mathbb{Q}}
\newcommand{\R}{\mathbb{R}}
\newcommand{\C}{\mathbb{C}}
\newcommand{\F}{\mathbb{F}}
\renewcommand{\H}{\mathbb{H}}
\renewcommand{\P}{\mathbb{P}}

\renewcommand{\a}{\alpha}
\renewcommand{\b}{\beta}
\newcommand{\g}{\gamma}
\renewcommand{\d}{\delta}
\newcommand{\z}{\zeta}
\renewcommand{\t}{\theta}
\renewcommand{\i}{\iota}
\renewcommand{\k}{\kappa}
\renewcommand{\l}{\lambda}
\newcommand{\s}{\sigma}
\newcommand{\w}{\omega}

\newcommand{\G}{\Gamma}
\newcommand{\D}{\Delta}
\renewcommand{\L}{\Lambda}
\newcommand{\W}{\Omega}
\newcommand{\scL}{\mathscr{L}}

\newcommand{\e}{\varepsilon}
\newcommand{\vt}{\vartheta}
\newcommand{\vphi}{\varphi}
\newcommand{\emt}{\varnothing}

\newcommand{\x}{\times}
\newcommand{\ox}{\otimes}
\newcommand{\op}{\oplus}
\newcommand{\bigox}{\bigotimes}
\newcommand{\bigop}{\bigoplus}
\newcommand{\del}{\partial}
\newcommand{\<}{\langle}
\renewcommand{\>}{\rangle}
\newcommand{\lf}{\lfloor}
\newcommand{\rf}{\rfloor}
\newcommand{\wtilde}{\widetilde}
\newcommand{\what}{\widehat}
\newcommand{\conj}{\overline}
\newcommand{\cchi}{\conj{\chi}}

\DeclareMathOperator{\id}{\textrm{id}}
\DeclareMathOperator{\sgn}{\mathrm{sgn}}
\DeclareMathOperator{\im}{\mathrm{im}}
\DeclareMathOperator{\rk}{\mathrm{rk}}
\DeclareMathOperator{\adj}{\mathrm{adj}}
\DeclareMathOperator{\tr}{\mathrm{trace}}
\DeclareMathOperator{\nm}{\mathrm{norm}}
\DeclareMathOperator{\disc}{\mathrm{disc}}
\DeclareMathOperator{\ord}{\mathrm{ord}}
\DeclareMathOperator{\sym}{\mathrm{sym}}
\DeclareMathOperator{\ext}{\mathrm{ext}}
\DeclareMathOperator{\Hom}{\mathrm{Hom}}
\DeclareMathOperator{\End}{\mathrm{End}}
\DeclareMathOperator{\Aut}{\mathrm{Aut}}
\DeclareMathOperator{\Tor}{\mathrm{Tor}}
\DeclareMathOperator{\Ann}{\mathrm{Ann}}
\DeclareMathOperator{\Gal}{\mathrm{Gal}}
\DeclareMathOperator{\Trace}{\mathrm{Tr}}
\DeclareMathOperator{\Norm}{\mathrm{N}}
\DeclareMathOperator{\Cl}{\mathrm{Cl}}
\DeclareMathOperator{\Span}{\mathrm{Span}}
\DeclareMathOperator*{\Res}{\mathrm{Res}}
\DeclareMathOperator{\Vol}{\mathrm{Vol}}
\DeclareMathOperator{\Li}{\mathrm{Li}}
\DeclareMathOperator{\Supp}{\mathrm{Supp}}
\renewcommand{\Re}{\mathrm{Re}}
\renewcommand{\Im}{\mathrm{Im}}
\DeclareMathOperator{\Ph}{\mathrm{Ph}}
\DeclareMathOperator{\SC}{\mathrm{SC}}


\newcommand{\GH}{\G\backslash\H}
\newcommand{\GG}{\G_{\infty}\backslash\G}

\newenvironment{psmallmatrix}
  {\left(\begin{smallmatrix}}
  {\end{smallmatrix}\right)}

\newcommand{\smc}[1]{
    \mathchoice
    {{\scriptstyle\mathcal{#1}}}
    {{\scriptstyle\mathcal{#1}}}
    {{\scriptscriptstyle\mathcal{#1}}}
    {\scalebox{0.7}{$\scriptscriptstyle\mathcal{#1}$}}
}

%============%
%  Comments  %
%============%
\newcommand{\todo}[1]{\textcolor{red}{\sf Todo: [#1]}}

%===================%
%  Label reminders  %
%===================%
% [label=(\roman*)]
% [label=(\alph*)]
% [label=(\arabic{enumi})]

%==================%
%  Other settings  %
%==================%
\pgfdeclarelayer{background}
\pgfsetlayers{background,main}
\tikzset{->-/.style={decoration={
  markings,
  mark=at position .5 with {\arrow{>}}},postaction={decorate}}}