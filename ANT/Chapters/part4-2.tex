\chapter{Types of Sieves}
  \section{\todo{Combinatorial Sieves}}
    The idea behind the combinatorial sieve is to choose sieve weights such that the sieving function is a truncation of the inclusion-exclusion principle applied to the sifting function. To motivate this choice, we being by iteratively applying Buchstab's identity $r \ge 1$ times to obtain
    \[
      S(\mc{A},\mc{P},z) = \sum_{\substack{d \mid P(z) \\ \w(d) < r}}\mu(d)S(\mc{A}_{d})+(-1)^{r}\sum_{\substack{d \mid P(z) \\ \w(d) = r}}S(\mc{A}_{d},\mc{P},p_{d}),
    \]
    where $p_{d}$ is the smallest prime divisor of $d$. This identity can be thought of as an inclusion-exclusion principle for the sifting function. As $S(\mc{A}_{d},\mc{P},p_{d})$ is nonnegative, we obtain the upper bound
    \[
      S(\mc{A},\mc{P},z) \le \sum_{\substack{d \mid P(z) \\ \w(d) < r}}\mu(d)S(\mc{A}_{d}),
    \]
    if $r$ is odd, and the lower bound
    \[
      S(\mc{A},\mc{P},z) \ge \sum_{\substack{d \mid P(z) \\ \w(d) < r}}\mu(d)S(\mc{A}_{d}),
    \]
    is $r$ is even. We now wish to consider sieves whose sieving functions replace these upper and lower bounds with sums that are closer approximations to the sifting function. Precisely, we will replace the condition $\w(d) < r$ with $d \in \mc{D}$ for set of positive integers $\mc{D}$ with a small amount of small prime divisors. A \textbf{combinatorial sieve}\index{combinatorial sieve} $\L = (\l_{d})_{d \ge 1}$ is a sieve for which $\l_{d} = \mu(d)$ or $\l_{d} = 0$ with the former case occurring only for those $d$ belonging to the set
    \[
      \mc{D} = \todo{xxx}
    \]
  \section{\todo{\texorpdfstring{$\L^{2}$}{L2} Sieves}}
  \section{\todo{The Large Sieve}}