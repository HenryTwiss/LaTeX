\chapter{Types of Sieves}
  \section{\todo{The Combinatorial Sieve}}
    The idea behind the combinatorial sieve is to choose $\l_{d}$ such that it is nonzero and equal to $\mu(d)$ only when $d$ has a small number of prime divisors. We being by iteratively applying Buchstab's identity $r$ times to obtain
    \[
      S(\mc{A},\mc{P},z) = \sum_{\substack{d \mid P(z) \\ \w(d) < r}}\mu(d)S(\mc{A}_{d})+(-1)^{r}\sum_{\substack{d \mid P(z) \\ \w(d) = r}}S(\mc{A}_{d},\mc{P},p_{d}),
    \]
    where $p_{d}$ is the smallest prime divisor of $d$. \todo{xxx}
  \section{\todo{The \texorpdfstring{$\L^{2}$}{L2} Sieve}}
  \section{\todo{The Large Sieve}}