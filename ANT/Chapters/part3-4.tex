\chapter{Trace Formulas}
  There are various types of formulas that relate the Fourier coefficients of automorphic forms. One of the most important such formulas is the Petersson trace formulas.
  \section{The Petersson Trace Formula}
    From \cref{thm:newforms_characterization_holomorphic}, $\mc{S}_{k}(N,\chi)$ admits an orthonormal basis of Hecke eigenforms. In particular, $\mc{S}_{k}(N,\chi)$ admits a merely orthogonal basis. Denote this basis by $u_{1},\ldots,u_{r}$ where $r$ is the dimension of $\mc{S}_{k}(N,\chi)$. Each of these forms admits a Fourier series at the $\mf{a}$ cusp given by
    \[
      (u_{j}|\s_{\mf{a}})(z) = \sum_{n \ge 1}a_{j,\mf{a}}(n)e^{2\pi inz}.
    \]
    The Petersson trace formula is an equation relating the Fourier coefficients $a_{j,\mf{a}}(n)$ and $a_{j,\mf{b}}(n)$ for $1 \le j \le r$ at two cusps $\mf{a}$ and $\mf{b}$ of $\G_{0}(N)\backslash\H$ to a sum of $J$-Bessel functions and Sali\'e sums. To prove the Petersson trace formula we compute the inner product of two Poincar\'e series $P_{n,k,\chi,\mf{a}}(z)$ and $P_{m,k,\chi,\mf{b}}(z)$ in two different ways. One way is geometric in nature while the other is spectral. Since \cref{thm:Petersson_inner_product_with_Poincare_series} says that $\<P_{n,k,\chi,\mf{a}},P_{m,k,\chi,\mf{b}}\>$ extracts the $m$-th Fourier coefficient of $P_{n,k,\chi,\mf{a}}$ up to a constant, the Petersson trace formula amounts to computing the $m$-th Fourier coefficient of $P_{n,k,\chi,\mf{a}}$ in two different ways. We will begin with the geometric method first. This is easy as we have already computed the Fourier series of the Poincar\'e series. Applying \cref{thm:Petersson_inner_product_with_Poincare_series} to the Fourier series in \cref{prop:Fourier_series_Poincare_holomorphic} gives
    \[
      \<P_{n,k,\chi,\mf{a}},P_{m,k,\chi,\mf{b}}\> = \frac{\G(k-1)}{V_{\G_{0}(N)}(4\pi m)^{k-1}}\left(\d_{\mf{a},\mf{b}}\d_{n,m}+\left(\frac{\sqrt{m}}{\sqrt{n}}\right)^{k-1}\sum_{c \in \mc{C}_{\mf{a},\mf{b}}}\frac{2\pi i^{-k}}{c}J_{k-1}\left(\frac{4\pi\sqrt{nm}}{c}\right)S_{\chi,\mf{a},\mf{b}}(n,m,c)\right).
    \]
    This is the first half of the Petersson trace formula. To obtain the second half, we use the fact that $u_{1},\ldots,u_{r}$ is an orthogonal basis for $\mc{S}_{k}(N,\chi)$ and \cref{thm:Petersson_inner_product_with_Poincare_series} to write
    \[
      P_{n,k,\chi,\mf{a}}(z) = \sum_{1 \le j \le r}\frac{\<P_{n,k,\chi,\mf{a}},u_{j}\>}{\<u_{j},u_{j}\>}u_{j}(z) = \sum_{1 \le j \le r}\frac{\conj{\<u_{j},P_{n,k,\chi,\mf{a}}\>}}{\<u_{j},u_{j}\>}u_{j}(z) = \frac{\G(k-1)}{V_{\G_{0}(N)}(4\pi n)^{k-1}}\sum_{1 \le j \le r}\frac{\conj{a_{j,\mf{a}}(n)}}{\<u_{j},u_{j}\>}u_{j}(z).
    \]
    This last expression is the spectral decomposition of $P_{n,k,\chi,\mf{a}}(z)$ in terms of the basis $u_{1},\ldots,u_{r}$. So if we apply \cref{thm:Petersson_inner_product_with_Poincare_series} to this last expression, we obtain
    \[
      \<P_{n,k,\chi,\mf{a}},P_{m,k,\chi,\mf{b}}\> = \left(\frac{\G(k-1)}{V_{\G_{0}(N)}(4\pi \sqrt{nm})^{k-1}}\right)^{2}\sum_{1 \le j \le r}\frac{\conj{a_{j,\mf{a}}(n)}a_{j,\mf{b}}(m)}{\<u_{j},u_{j}\>},
    \]
    which is the second half of the Petersson trace formula. Equating the first and second halves and canceling the common $\frac{\G(k-1)}{V_{\G_{0}(N)}(4\pi m)^{k-1}}$ factor gives
    \[
      \frac{\G(k-1)}{V_{\G_{0}(N)}(4\pi n)^{k-1}}\sum_{1 \le j \le r}\frac{\conj{a_{j,\mf{a}}(n)}a_{j,\mf{b}}(m)}{\<u_{j},u_{j}\>} = \d_{\mf{a},\mf{b}}\d_{n,m}+\left(\frac{\sqrt{m}}{\sqrt{n}}\right)^{k-1}\sum_{c \in \mc{C}_{\mf{a},\mf{b}}}\frac{2\pi i^{-k}}{c}J_{k-1}\left(\frac{4\pi\sqrt{nm}}{c}\right)S_{\chi,\mf{a},\mf{b}}(n,m,c).
    \]
    Since $\left(\frac{\sqrt{m}}{\sqrt{n}}\right)^{k-1} = 1$ when $n = m$, we can factor this term out of the entire right-hand side and cancel it resulting in the \textbf{Petersson trace formula}\index{Petersson trace formula} relative to the $\mf{a}$ and $\mf{b}$ cusps:
    \[
      \frac{\G(k-1)}{V_{\G_{0}(N)}(4\pi \sqrt{nm})^{k-1}}\sum_{1 \le j \le r}\frac{\conj{a_{j,\mf{a}}(n)}a_{j,\mf{b}}(m)}{\<u_{j},u_{j}\>} = \d_{\mf{a},\mf{b}}\d_{n,m}+\sum_{c \in \mc{C}_{\mf{a},\mf{b}}}\frac{2\pi i^{-k}}{c}J_{k-1}\left(\frac{4\pi\sqrt{nm}}{c}\right)S_{\chi,\mf{a},\mf{b}}(n,m,c).
    \]
    We refer to the left-hand side side as the \textbf{spectral side}\index{spectral side} and the right-hand side as the \textbf{geometric side}\index{geometric side}. We collect our work as a theorem:

    \begin{theorem}[Petersson trace formula]
      Let $u_{1},\ldots,u_{r}$ be an orthogonal basis of Hecke eigenforms for $\mc{S}_{k}(N,\chi)$ with Fourier coefficients $a_{j,
      \mf{a}}(n)$ at the $\mf{a}$ cusp. Then for any positive integers $n,m \ge 1$ and any two cusps $\mf{a}$ and $\mf{b}$, we have
      \[
        \frac{\G(k-1)}{V_{\G_{0}(N)}(4\pi \sqrt{nm})^{k-1}}\sum_{1 \le j \le r}\frac{\conj{a_{j,\mf{a}}(n)}a_{j,\mf{b}}(m)}{\<u_{j},u_{j}\>} = \d_{\mf{a},\mf{b}}\d_{n,m}+\sum_{c \in \mc{C}_{\mf{a},\mf{b}}}\frac{2\pi i^{-k}}{c}J_{k-1}\left(\frac{4\pi\sqrt{nm}}{c}\right)S_{\chi,\mf{a},\mf{b}}(n,m,c).
      \]
    \end{theorem}

    Note that if we take $u_{1},\ldots,u_{r}$ to be an orthonormal basis, equivalently the basis elements are Hecke normalized, then $\<u_{j},u_{j}\> = 1$. Regardless, a particularly important case of the Petersson trace formula is when $\mf{a} = \mf{b} = \infty$. For then $\mc{C}_{\infty,\infty} = \{c \ge 1:c \equiv 0 \tmod{N}\}$, the Sali\'e sum reduces to the usual one, and the Petersson trace formula takes the form
    \[
      \frac{\G(k-1)}{V_{\G_{0}(N)}(4\pi \sqrt{nm})^{k-1}}\sum_{1 \le j \le r}\frac{\conj{a_{j}(n)}a_{j}(m)}{\<u_{j},u_{j}\>} = \d_{n,m}+\sum_{\substack{c \ge 1 \\ c \equiv 0 \tmod{N}}}\frac{2\pi i^{-k}}{c}J_{k-1}\left(\frac{4\pi\sqrt{nm}}{c}\right)S_{\chi}(n,m,c).
    \]