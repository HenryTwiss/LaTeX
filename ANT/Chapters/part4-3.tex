\chapter{Explicit Formulas \& Prime Number Theorems}
  Our main aim is to prove the prime number theorem and its variant for primes restricted to a certain residue class, the Siegel–Walfisz theorem, in the classical manner. These results will follow from explicit formulas for the Chebyshev function $\psi(x)$ and a sutiable generalization in the latter case. After deriving these results, we prove (for the second time) the prime number theorem and the Siegel–Walfisz theorem.
  \section{\todo{Explicit Formulas for Chebyshev Functions}}
    \subsection*{The Explicit Formula for \texorpdfstring{$\psi(x)$}{$\psi(x)$}}
      The explicit formula for $\psi(x)$ will be obtained by applying truncated Perron's formula. Since $\psi(x)$ is discontinuous when $x$ is a prime power, we need to work with $\psi_{0}(x)$ to apply the Mellin inversion formula. The \textbf{explicit formula}\index{explicit formula} for $\psi(x)$ is the following:

      \begin{theorem}[Explicit formula for $\psi(x)$]
        For $x \ge 2$,
        \[
          \psi_{0}(x) = x-\sum_{\rho}\frac{x^{\rho}}{\rho}-\frac{\z'}{\z}(0)-\frac{1}{2}\log(1-x^{-2}),
        \]
        where $\rho$ runs over the nontrivial zeros of $\z(s)$ counted with multiplicity and ordered with respect to the size of the ordinate.
      \end{theorem}

      A few comments are in order before we prove the explicit formula for $\psi(x)$. First, since $\rho$ is conjectured to be of the form $\rho = \frac{1}{2}+i\g$ under the Riemann hypothesis, $x$ contributes the main term in the explicit formula. This is in agreement with the fact that $\psi(x) \sim x$ which is equivalent to the prime number theorem as we already showed. The value $\frac{\z'}{\z}(0)$ can be shown to be $\log(2\pi)$ (see \cite{davenport1980multiplicative} for a proof). Also, using the Taylor series of the logarithm, the last term can be expressed as
      \[
        \frac{1}{2}\log(1-x^{-2}) = \frac{1}{2}\sum_{k \ge 1}(-1)^{k-1}\frac{(-x^{-2})^{k}}{k} = \sum_{k \ge 1}(-1)^{2k-1}\frac{x^{-2k}}{2k} = -\sum_{\w}\frac{x^{\w}}{\w},
      \]
      where $\w$ runs over the trivial zeros of $\z(s)$. We will now prove the explicit formula for $\psi(x)$:

      \begin{proof}[Proof of the explicit formula for $\psi(x)$]
        Recalling \cref{equ:Drichlet_series_log_derivative_zeta} and that $\psi(x) = \sum_{n \le x}\L(n)$, applying truncated Perron's formula to $-\frac{\z'}{\z}(s)$ gives
        \begin{equation}\label{equ:explicit_formula_proof_1}
          \psi_{0}(x)-J(x,T) \ll x^{c}\sum_{\substack{n \ge 1 \\ n \neq x}}\frac{\L(n)}{n^{c}}\min\left(1,\frac{1}{T\left|\log\left(\frac{x}{n}\right)\right|}\right)+\d_{x}\L(x)\frac{c}{T},
        \end{equation}
        where
        \[
          J(x,T) = \frac{1}{2\pi i}\int_{c-iT}^{c+iT}-\frac{\z'}{\z}(s)x^{s}\,\frac{ds}{s},
        \]
        $c > 1$, and it is understood that $\d_{x} = 0$ unless $x$ is a prime power. Take $T > 2$ not coinciding with the ordinate of a nontrivial zero and let $c = 1+\log(x^{2})^{-1}$ so that $x^{c} = \sqrt{e}x$. The first step is to estimate the right-hand side of \cref{equ:explicit_formula_proof_1}. First, we deal with the terms corresponding to $n$ such that $n$ is boudned away from $x$. So suppose $n \le \frac{3}{4}x$ or $n \ge \frac{5}{4}x$. For these $n$, $1 \ll \log\left(\frac{x}{n}\right)$ so that their contribution is
        \begin{equation}\label{equ:explicit_formula_proof_2}
          \ll \frac{x^{c}}{T}\sum_{n \ge 1}\frac{\L(n)}{n^{c}} \ll \frac{x^{c}}{T}\left(-\frac{\z'}{\z}(c)\right) \ll \frac{x\log(x)}{T},
        \end{equation}
        where the last estimate follows from \cref{equ:classical_zero-free_region_zeta_1} and our choice of $c$. Now we deal with the terms $n$ close to $x$. First consider those $n$ for which $\frac{3}{4}x < n < x$. Let $x_{1}$ be the largest prime power less than $x$. We may also suppose $\frac{3}{4}x < x_{1} < x$ since otherwise $\L(n) = 0$ and these terms do not contribute anything. For the term $n = x_{1}$, we have
        \[
          \log\left(\frac{x}{n}\right) = -\log\left(1-\frac{x-x_{1}}{x}\right) \ge \frac{x-x_{1}}{x},
        \]
        where we have obtained the inequality by using Taylor series of the logarithim truncated after the first term. The contribution of this term is then
        \begin{equation}\label{equ:explicit_formula_proof_3}
          \ll \L(x_{1})\min\left(1,\frac{x}{T(x-x_{1})}\right) \ll \log(x)\min\left(1,\frac{x}{T(x-x_{1})}\right).
        \end{equation}
        For the other such $n$, we can write $n = x_{1}-v$, where $v$ is an integer satisfying $0 < v < \frac{1}{4}x$, so that
        \[
          \log\left(\frac{x}{n}\right) \ge \log\left(\frac{x_{1}}{n}\right) = -\log\left(1-\frac{v}{x_{1}}\right) \ge \frac{v}{x_{1}},
        \]
        where we have obtained the inequality by using Taylor series of the logarithim truncated after the first term. The contribution for these $n$ is then
        \begin{equation}\label{equ:explicit_formula_proof_4}
          \ll \sum_{0 < v < \frac{1}{4}x}\L(x_{1}-v)\frac{x_{1}}{Tv} \ll \frac{x}{T}\sum_{0 < v < \frac{1}{4}x}\frac{\L(x_{1}-v)}{v} \ll \frac{x\log(x)}{T}\sum_{0 < v < \frac{1}{4}x}\frac{1}{v} \ll \frac{x\log(x)^{2}}{T}.
        \end{equation}
        The contribution for those $n$ for which $x < n < \frac{5}{4}x$ is handeled in exactly the same way with $x_{1}$ being the least prime power larger than $x$. Let $\<x\>$ be the distance between $x$ and the nearest prime power provided $x$ is itself not a prime power. Combining \cref{equ:explicit_formula_proof_3,equ:explicit_formula_proof_4} with our previous comment, the contribution for those $n$ with $\frac{3}{4}x < n < \frac{5}{4}x$ is
        \begin{equation}\label{equ:explicit_formula_proof_5}
          \ll \frac{x\log(x)^{2}}{T}+\log(x)\min\left(1,\frac{x}{T\<x\>}\right).
        \end{equation}
        Putting \cref{equ:explicit_formula_proof_2,equ:explicit_formula_proof_5} together and noticing that the error term in \cref{equ:explicit_formula_proof_2} is absorbed by the second error term in \cref{equ:explicit_formula_proof_5}, we obtain
        \begin{equation}\label{equ:explicit_formula_proof_6}
          \psi_{0}(x)-J(x,T) \ll \frac{x\log(x)^{2}}{T}+\log(x)\min\left(1,\frac{x}{T\<x\>}\right).
        \end{equation}
        This is the first part of the proof. Now we estimate $J(x,T)$ by appealing to the residue theorem. Let $U \ge 1$ be an odd integer. Let $\W$ be the region enclosed by the contours $\eta_{1},\ldots,\eta_{4}$ in \cref{fig:explict_formula_contour} and set $\eta = \sum_{1 \le i \le 4}\eta_{i}$ so that $\eta = \del \W$.

        \begin{figure}[ht]
          \centering
          \begin{tikzpicture}[scale=2]
            \def\xmin{-3.5} \def\xmax{1.5}
            \def\ymin{-2} \def\ymax{2}
            \draw[thick] (\xmin,0) -- (\xmax,0);
            \draw[thick] (0,\ymin) -- (0,\ymax);
            \draw[dashed] (0.5,\ymin) -- (0.5,\ymax);

            \draw[->-] (1,-1.5) -- (1,1.5);
            \draw[->-] (1,1.5) -- (-3,1.5);
            \draw[->-] (-3,1.5) -- (-3,-1.5);
            \draw[->-] (-3,-1.5) -- (1,-1.5);

            \node at (1,0) [below right] {\tiny{$\eta_{1}$}};
            \node at (-1,1.5) [above] {\tiny{$\eta_{2}$}};
            \node at (-3,0) [below left] {\tiny{$\eta_{3}$}};
            \node at (-1,-1.5) [below] {\tiny{$\eta_{4}$}};

            \node at (1,-1.5) [circle,fill,inner sep=1.5pt]{};
            \node at (1,1.5) [circle,fill,inner sep=1.5pt]{};
            \node at (0.5,1.5) [circle,fill,inner sep=1.5pt]{};
            \node at (-3,1.5) [circle,fill,inner sep=1.5pt]{};
            \node at (-3,-1.5) [circle,fill,inner sep=1.5pt]{};
            \node at (0.5,-1.5) [circle,fill,inner sep=1.5pt]{};

            \node at (1,-1.5) [below left] {\tiny{$c-iT$}};
            \node at (1,1.5) [above] {\tiny{$c+iT$}};
            \node at (0.5,1.5) [above left] {\tiny{$\frac{1}{2}+iT$}};
            \node at (-3,1.5) [above] {\tiny{$-U+iT$}};
            \node at (-3,-1.5) [below left] {\tiny{$-U-iT$}};
            \node at (0.5,-1.5) [below left] {\tiny{$\frac{1}{2}-iT$}};
          \end{tikzpicture}
          \caption{Contour for the explict formula for $\psi(x)$}
          \label{fig:explict_formula_contour}
        \end{figure}

        We may express $J(x,T)$ as
        \[
          J(x,T) = \frac{1}{2\pi i}\int_{\eta_{1}}-\frac{\z'}{\z}(s)x^{s}\,\frac{ds}{s}.
        \]
        The residue theorem together with the explicit formula for $\frac{\z'}{\z}(s)x^{s}$ and \cref{cor:logarithmic_derivative_of_gamma} imply
        \begin{equation}\label{equ:explicit_formula_proof_7}
          J(x,T) = x-\sum_{|\g| < T}\frac{x^{\rho}}{\rho}-\frac{\z'}{\z}(0)-\sum_{0 < 2m < U}\frac{x^{-2m}}{2m}+\frac{1}{2\pi i}\int_{\eta_{1}+\eta_{2}+\eta_{3}}-\frac{\z'}{\z}(s)x^{s}\,\frac{ds}{s},
        \end{equation}
        where $\rho = \b+i\g$ is a nontrivial zeros of $\z$. We will estimate $J(x,T)$ by estimating the integral. By \cref{cor:Riemann_von_Mangoldt_corollary} (i), the number of nontrivial zeros satisfying $|\g-T| < 1$ is $O(\log(T))$. Among the ordinates of these zeros, there must be a gap of size $\gg \log(T)^{-1}$. Upon varrying $T$ by a bounded amount (we are varrying in the interval $[T-1,T+1]$) so that it belongs to this gap, we can additionally ensure
        \[
          |\g-T| \gg \log(T)^{-1},
        \]
        for all the nontrivial zeros of $\z(s)$. To estimate part of the horizontal integrals over $\eta_{2}$ and $\eta_{4}$, \cref{equ:Riemann_von_Mangoldt_14} with $s = \s+iT$ for $-1 \le \s \le 2$ gives
        \[
          \frac{\z'}{\z}(s) = \sum_{|\g-T| < 1}\frac{1}{s-\rho}+O(\log(T)).
        \]
        By our choice of $T$, $|s-\rho| \ge |\g-T| \gg \log(T)^{-1}$ so that each term in the sum is $O(\log(T))$. There are at most $O(\log(T))$ such terms by \cref{cor:Riemann_von_Mangoldt_corollary} (i), so we find that
        \[
          \frac{\z'}{\z}(s) = O(\log(T)^{2}),
        \]
        for $-1 \le \s \le 2$. It follows that the parts of the horizontal integrals over $\eta_{2}$ and $\eta_{4}$ with  $-1 \le \s \le c$ (as $x \ge 2$ our choice of $c$ ensures $c < 2$) contribute
        \begin{equation}\label{equ:explicit_formula_proof_8}
          \ll \frac{\log(T)^{2}}{T}\int_{-1}^{c}x^{\s}\,d\s \ll \frac{\log(T)^{2}}{T}\int_{-\infty}^{c}x^{\s}\,d\s \ll \frac{x\log(T)^{2}}{T\log(x)},
        \end{equation}
        where in the first estimate we have used the fact that $s \sim T$ and in the last estiamte we have used the choice of $c$. To estimate the remainder of the horizontal integrals, we need a bound for $\frac{\z'}{\z}(s)$ when $\s < -1$ and away from the trivial zeros. To this end, write the functional equation for $\z(s)$ in the form
        \[
          \z(s) = \pi^{s-1}\frac{\G\left(\frac{1-s}{2}\right)}{\G\left(\frac{s}{2}\right)}\z(1-s),
        \]
        and take the logarithimic derivative to get
        \[
          \frac{\z'}{\z}(s) = \log(\pi)+\frac{1}{2}\frac{\G'}{\G}\left(\frac{1-s}{2}\right)-\frac{1}{2}\frac{\G'}{\G}\left(\frac{s}{2}\right)+\frac{\z'}{\z}(1-s).
        \]
        Let $s$ be such that $\s < -1$ and suppose $s$ is say distance $\frac{1}{2}$ away from the trivial zeros. We will estiamte every term on the right-hand side. The first term is contant and the four term is bounded by \cref{equ:Drichlet_series_log_derivative_zeta}. As for the digamma terms, since $s$ is away from the trivial zeros, \cref{equ:approximtion_for_digamma} implies $\frac{1}{2}\frac{\G'}{\G}\left(\frac{1-s}{2}\right) = O(\log|1-s|)$ and $\frac{1}{2}\frac{\G'}{\G}\left(\frac{s}{2}\right) = O(\log|s|)$. However, as $\s > 2$, $s$ is bounded away from zero so that $\frac{1}{2}\frac{\G'}{\G}\left(\frac{1-s}{2}\right) = O(\log|s|)$. Putting these estimates together, we see that
        \begin{equation}\label{equ:explicit_formula_proof_9}
          \frac{\z'}{\z}(s) \ll \log(|s|),
        \end{equation}
        for $\s < -1$. Using \cref{equ:explicit_formula_proof_9}, the parts of the horizontal integrals over $\eta_{2}$ and $\eta_{4}$ with $-U \le \s \le -1$ contribute
        \begin{equation}\label{equ:explicit_formula_proof_10}
          \ll \frac{\log(T)}{T}\int_{-U}^{-1}x^{\s}\,d\s \ll \frac{\log(T)}{Tx\log(x)},
        \end{equation}
        where in the first estimate we have used the fact that $s \sim T$. Combining \cref{equ:explicit_formula_proof_8,equ:explicit_formula_proof_10} gives
        \begin{equation}\label{equ:explicit_formula_proof_11}
          \frac{1}{2\pi i}\int_{\eta_{2}+\eta_{4}}-\frac{\z'}{\z}(s)x^{s}\,\frac{ds}{s} \ll \frac{x\log(T)^{2}}{T\log(x)}+\frac{\log(T)}{Tx\log(x)} \ll \frac{x\log(T)^{2}}{T\log(x)}.
        \end{equation}
        To estimate the vertical integral, we use \cref{equ:explicit_formula_proof_9} again to conclude that
        \begin{equation}\label{equ:explicit_formula_proof_12}
          \frac{1}{2\pi i}\int_{\eta_{3}}-\frac{\z'}{\z}(s)x^{s}\,\frac{ds}{s} \ll \frac{\log(U)}{U}\int_{-T}^{T}x^{-U}\,dt \ll \frac{T\log(U)}{Ux^{U}},
        \end{equation}
        where in the first estimate we have used the fact that $s \sim U$. Combining \cref{equ:explicit_formula_proof_7,equ:explicit_formula_proof_11,equ:explicit_formula_proof_12} and taking the limit as $U \to \infty$, the error term in \cref{equ:explicit_formula_proof_12} vanishes and the sum over $m$ in \cref{equ:explicit_formula_proof_7} evaulates to $\frac{1}{2}\log(1-x^{-2})$ (as we have already mentioned) giving
        \begin{equation}\label{equ:explicit_formula_proof_13}
          J(x,T) = x-\sum_{|\g| < T}\frac{x^{\rho}}{\rho}-\frac{\z'}{\z}(0)-\frac{1}{2}\log(1-x^{-2})+\frac{x\log(T)^{2}}{T\log(x)}.
        \end{equation}
        Substiuting \cref{equ:explicit_formula_proof_13} into \cref{equ:explicit_formula_proof_6}, we at last obtain
        \[
          \psi_{0}(x) = x-\sum_{|\g| < T}\frac{x^{\rho}}{\rho}-\frac{\z'}{\z}(0)-\frac{1}{2}\log(1-x^{-2})+\frac{x\log(xT)^{2}}{T}+\log(x)\min\left(1,\frac{x}{T\<x\>}\right),
        \]
        where the second to last term on the right-hand side is obtained by combining error term in \cref{equ:explicit_formula_proof_11} with the first error term in \cref{equ:explicit_formula_proof_6}. The theorem following by taking the limit as $T \to \infty$.
      \end{proof}
    \subsection*{The Explicit Formula for \texorpdfstring{$\psi(x,\chi)$}{$\psi(x,\chi)$}}
      \todo{xxx}
  \section{\todo{The Prime Number Theorem (Reprise)}}
  \section{\todo{The Siegel–Walfisz Theorem}}