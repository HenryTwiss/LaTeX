\chapter{Classical Applications}
  We will discuss some classical applications surrounding $L$-functions. Our first result is a crowning gem of analytic number theory: Dirichlet's theorem on arithmetic progressions. This result is a consequence of a non-vanishing result for Dirichlet $L$-series at $s = 1$. Next, we discuss Siegel zeros in the case of Dirichlet $L$-functions and as a result obtain a lower bound for Dirichlet $L$-functions at $s = 1$. After these two results, we prove the prime number theorem and its variant for primes restricted to a certain residue class, the Siegel–Walfisz theorem, in the classical manner.
  \section{Dirichlet's Theorem on Arithmetic Progressions}
    One of the more well-known arithmetic results proved using $L$-series is \textbf{Dirichlet's theorem on arithmetic progressions}\index{Dirichlet's theorem on arithmetic progressions}:

    \begin{theorem}[Dirichlet's theorem on arithmetic progressions]\label{thm:Dirichlet's_theorem_on_primes_in_arithmetic_progressions}
      Let $a$ and $m$ be positive integers such that $(a,m) = 1$. Then the arithmetic progression $\{a+km \mid k \in \Z_{\ge 0}\}$ contain infinitely many primes.
    \end{theorem}

    We will delay the proof for the moment, for it is well-worth understanding the some of the motivation behind why this theorem is interesting and how exactly Dirichlet used the analytic techniques of $L$-series to attack this purely arithmetic statement. We being by recalling Euclid's famous theorem on the infinitude of the primes. Euclid's proof is completely elementary and arithmetic in nature. He argues that if there were finitely many primes $p_{1},p_{2},\ldots,p_{k}$ then a short consideration of $(p_{1}p_{2} \cdots p_{k})+1$ shows that this number must either be divisible by a prime not in our list or must be prime itself. As primes are the multiplicative building blocks of arithmetic, Euclid assures us that we have an ample amount of primes to work with. Now there is a slightly stronger result due to Euler (see \cite{euler1744variae}) requiring analytic techniques:

    \begin{theorem}\label{thm:reciprocial_sum_of_primes_diverges}
      The series
      \[
        \sum_{p}\frac{1}{p},
      \]
      diverges.
    \end{theorem}
    \begin{proof}
      For $\s > 1$, taking the logarithm of the Euler product of $\z(s)$, we get
      \[
        \log\z(s) = -\sum_{p}\log(1-p^{-s}).
      \]
      The Taylor series of the logarithm gives
      \[
        \log(1-p^{-s}) = \sum_{k \ge 1}(-1)^{k-1}\frac{(-p^{-s})^{k}}{k} = \sum_{k \ge 1}(-1)^{2k-1}\frac{1}{kp^{ks}},
      \]
      so that
      \[
        \log\z(s) = \sum_{p}\sum_{k \ge 1}\frac{1}{kp^{ks}}.
      \]
      The double sum restricted to $k \ge 2$ is uniformly bounded for $\s > 1$. To see this, first observe
      \[
        \sum_{k \ge 2}\frac{1}{kp^{ks}} \ll \sum_{k \ge 2}\frac{1}{p^{k}} = \frac{1}{p^{2}}\sum_{k \ge 0}\frac{1}{p^{k}} = \frac{1}{p^{2}}(1-p^{-1})^{-1} \le \frac{2}{p^{2}},
      \]
      where the last inequality follows because $p \ge 2$. Then
      \[
       \sum_{p}\sum_{k \ge 2}\frac{1}{kp^{ks}} \ll 2\sum_{p}\frac{1}{p^{2}} < 2\sum_{n \ge 1}\frac{1}{n^{2}} = 2\z(2).
      \]
      Therefore
      \[
        \log\z(s)-\sum_{p}\frac{1}{p^{s}} = \sum_{p}\sum_{k \ge 2}\frac{1}{kp^{ks}},
      \]
      and remains bounded as $s \to 1$. The claim now follows since $\z(s)$ has a simple pole at $s = 1$.
    \end{proof}

    \cref{thm:reciprocial_sum_of_primes_diverges} tells us that there are infinitely many primes but also that the primes are not too sparse in the integers for otherwise the series would converge. The idea Dirichlet used to prove his result on primes in arithmetic progressions was in a very similar spirit. He sought out to prove the divergence of the series
    \[
      \sum_{p \equiv a \tmod{m}}\frac{1}{p},
    \]
    for positive integers $a$ and $m$ with $(a,m) = 1$ as the divergence immediately implies there are infinitely many primes $p$ of the form $p \equiv a \tmod{m}$. In the case $a = 1$ and $m = 2$ we recover \cref{thm:reciprocial_sum_of_primes_diverges} exactly since every prime is odd. Dirichlet's proof proceeds in a similar way to that of \cref{thm:reciprocial_sum_of_primes_diverges} and this is where Dirichlet used what are now known as Dirichlet characters and Dirichlet $L$-series. The proof can be broken into three steps. The first is to proceed as Euler did, but with the Dirichlet $L$-series $L(s,\chi)$ where $\chi$ has modulus $m$. That is, write $L(s,\chi)$ as a sum over primes and a bounded term as $s \to 1$. The next step is to use the orthogonality relations of the characters to sieve out the correct sum. The last step is to show the non-vanishing result $L(1,\chi) \neq 0$ for all non-principal characters $\chi$. This is the essential part of the proof as it is what assures us that the sum diverges. Luckily, we have done most of the hard work to prove this already:

    \begin{theorem}\label{thm:non-vanishing_of_Dirichlet_L-functions_at_s=1}
      Let $\chi$ be a non-principal Dirichlet character. Then $L(1,\chi)$ is finite and nonzero.
    \end{theorem}
    \begin{proof}
      This follows immediately by applying \cref{lem:non-vanshing_at_1_lemma} to $\z(s)L(s,\chi)$ and noting that $L(s,\chi)$ is holomorphic.
    \end{proof}

     We now prove Dirichlet's theorem on arithmetic progressions:

    \begin{proof}[Proof of Dirichlet's theorem on arithmetic progressions]
        Let $\chi$ be a Dirichlet character modulo $m$. Then for $\s > 1$, taking the logarithm of the Euler product of $L(s,\chi)$ gives
        \[
          \log{L(s,\chi)} = -\sum_{p}\log(1-\chi(p)p^{-s}).
        \]
        The Taylor series of the logarithm implies
        \[
          \log(1-\chi(p)p^{-s}) = \sum_{k \ge 1}(-1)^{k-1}\frac{(-\chi(p)p^{-s})^{k}}{k} = \sum_{k \ge 1}(-1)^{2k-1}\frac{\chi(p^{k})}{kp^{ks}},
        \]
        so that
        \[
          \log{L(s,\chi)} = \sum_{p}\sum_{k \ge 1}\frac{\chi(p^{k})}{kp^{ks}}.
        \]
        The double sum restricted to $k \ge 2$ is uniformly bounded for $\s > 1$. Indeed, first observe
        \[
          \left|\sum_{k \ge 2}\frac{\chi(p^{k})}{kp^{ks}}\right| \ll \sum_{k \ge 2}\frac{1}{p^{k}} = \frac{1}{p^{2}}\sum_{k \ge 0}\frac{1}{p^{k}} = \frac{1}{p^{2}}(1-p^{-1})^{-1} \le \frac{2}{p^{2}},
        \]
        where the last inequality follows because $p > 2$. Then
        \[
          \left|\sum_{p}\sum_{k \ge 2}\frac{\chi(p^{k})}{kp^{ks}}\right| \le 2\sum_{p}\frac{1}{p^{2}} < 2\sum_{n \ge 1}\frac{1}{n^{2}} = 2\z(2),
        \]
        as desired. Now write
        \[
          \sum_{\chi \tmod{m}}\cchi(a)\log{L(s,\chi)} = \sum_{\chi \tmod{m}}\sum_{p}\frac{\cchi(a)\chi(p)}{p^{s}}+\sum_{\chi \tmod{m}}\cchi(a)\sum_{p}\sum_{k \ge 2}\frac{\chi(p^{k})}{kp^{ks}}.
        \]
        By the orthogonality relations (\cref{prop:Dirichlet_orthogonality_relations} (ii)), we find that
        \[
          \sum_{\chi \tmod{m}}\sum_{p}\frac{\cchi(a)\chi(p)}{p^{s}} = \sum_{p}\frac1{p^{s}}\sum_{\chi \tmod{m}}\cchi(a)\chi(p) = \vphi(m)\sum_{p \equiv{a} \tmod{m}}\frac{1}{p^{s}},
        \]
        and so
        \[
          \sum_{\chi \tmod{m}}\cchi(a)\log{L(s,\chi)}-\sum_{\chi \tmod{m}}\cchi(a)\sum_{p}\sum_{k \ge 2}\frac{\chi(p^{k})}{kp^{ks}} = \vphi(m)\sum_{p \equiv{a} \tmod{m}}\frac{1}{p^{s}}.
        \]
        The triple sum is uniformly bounded for $\s > 1$ because the inner double sum is and there are finitely many Dirichlet characters modulo $m$. Therefore it suffices to show that the first sum on the left-hand side diverges as $s \to 1$. For $\chi = \chi_{m,0}$,
        \[
          L(s,\chi_{m,0}) = \z(s)\prod_{p \mid m}(1-p^{-s}),
        \]
        so the corresponding term in the sum is
        \[
          \conj{\chi_{m,0}}(a)\log L(s,\chi_{m,0}) = \log\z(s)+\sum_{p \mid m}\log(1-p^{-s}),
        \]
        which diverges as $s \to 1$ because $\z(s)$ has a simple pole at $s = 1$. We will be done if $\log{L(s,\chi)}$ remains bounded as $s \to 1$ for all $\chi \neq \chi_{m,0}$. So assume $\chi$ is non-principal. Then we must show $L(1,\chi)$ is finite and nonzero. This follows from \cref{thm:non-vanishing_of_Dirichlet_L-functions_at_s=1}.
    \end{proof}

    For primitive $\chi$ of conductor $q > 1$, we know from \cref{thm:non-vanishing_of_Dirichlet_L-functions_at_s=1} that $L(1,\chi)$ is finite and nonzero. It is interesting to know whether or not this value is computable in general. Indeed it is. The computation is fairly straightforward and only requires some basic properties of Ramanujan and Gauss sums that we have already developed. The idea is to rewrite the character values $\chi(n)$ so that we can collapse the infinite series into a Taylor series. Our result is the following:
    
    \begin{theorem}\label{thm:Value_of_Dirichlet_L-functions_at_s=1}
      Let $\chi$ be a primitive Dirichlet character with conductor $q > 1$. Then
      \[
        L(1,\chi) = -\frac{\tau(\chi)}{q}\sum_{a \tmod{q}}\cchi(a)\log\left(\sin\left(\frac{\pi a}{q}\right)\right) \quad \text{or} \quad L(1,\chi) = \frac{\tau(\chi)\pi i}{q^{2}}\sum_{a \tmod{q}}\cchi(a)a,
      \]
      according to whether $\chi$ is even or odd.
    \end{theorem}
    \begin{proof}
      First compute
      \begin{align*}
        \chi(n) &= \frac{1}{\tau(\cchi)}\sum_{a \tmod{q}}\cchi(a)e^{\frac{2\pi ian}{q}} && \text{\cref{cor:gauss_sum_primitive_formula}} \\
        &= \frac{\chi(-1)}{\conj{\tau(\chi)}}\sum_{a \tmod{q}}\cchi(a)e^{\frac{2\pi ian}{q}} && \text{\cref{prop:Gauss_sum_reduction} (i) and $\chi(-1)^{2} = 1$} \\
        &= \frac{\chi(-1)\tau(\chi)}{\tau(\chi)\conj{\tau(\chi)}}\sum_{a \tmod{q}}\cchi(a)e^{\frac{2\pi ian}{q}} \\
        &= \frac{\chi(-1)\tau(\chi)}{q}\sum_{a \tmod{q}}\cchi(a)e^{\frac{2\pi ian}{q}} && \text{\cref{thm:Gauss_sum_modulus}} \\
        &= \frac{\chi(-1)\tau(\chi)}{q}\sum_{a \tmod{q}}\cchi(a)e^{\frac{2\pi ian}{q}}.
      \end{align*}
      Substituting this result into the definition of $L(1,\chi)$, we find that
      \begin{equation}\label{equ:value_of_Dirichlet_L-functions_at_s=1_1}
        \begin{aligned}
          L(1,\chi) &= \sum_{n \ge 1}\frac{1}{n}\left(\frac{\chi(-1)\tau(\chi)}{q}\sum_{a \tmod{q}}\cchi(a)e^{\frac{2\pi ian}{q}}\right) \\
          &= \frac{\chi(-1)\tau(\chi)}{q}\sum_{a \tmod{q}}\cchi(a)\sum_{n \ge 1}\frac{e^{\frac{2\pi ian}{q}}}{n} \\
          &= \frac{\chi(-1)\tau(\chi)}{q}\sum_{a \tmod{q}}\cchi(a)\log\left(\left(1-e^{\frac{2\pi ia}{q}}\right)^{-1}\right),
        \end{aligned}
      \end{equation}
      where in the last line we have used the Taylor series of the logarithm. We will now simplify the last expression in \cref{equ:value_of_Dirichlet_L-functions_at_s=1_1}. Since $\sin(x) = \frac{e^{ix}-e^{-ix}}{2i}$, we have
      \[
        1-e^{\frac{2\pi ia}{q}} = -2ie^{\frac{\pi ia}{q}}\left(\frac{e^{\frac{\pi ia}{q}}-e^{-\frac{\pi ia}{q}}}{2i}\right) = -2ie^{\frac{\pi ia}{q}}\sin\left(\frac{\pi a}{q}\right).
      \]
      Therefore the last expression in \cref{equ:value_of_Dirichlet_L-functions_at_s=1_1} becomes
      \[
        \frac{\chi(-1)\tau(\chi)}{q}\sum_{a \tmod{q}}\cchi(a)\log\left(\left(-2ie^{\frac{\pi ia}{q}}\sin\left(\frac{\pi a}{q}\right)\right)^{-1}\right).
      \]
      As $a$ is taken modulo $q$, we have $0 < \frac{\pi a}{q} < \pi$ so that $\sin\left(\frac{\pi a}{q}\right)$ is never negative. Therefore we can split up the logarithm term and obtain
      \[
        -\frac{\chi(-1)\tau(\chi)}{q}\left(\log(-2i)\sum_{a \tmod{q}}\cchi(a)+\frac{\pi i}{q}\sum_{a \tmod{q}}\cchi(a)a+\sum_{a \tmod{q}}\cchi(a)\log\left(\sin\left(\frac{\pi a}{q}\right)\right)\right).
      \]
      By the orthogonality relations (\cref{cor:Dirichlet_orthogonality_relations} (i)), the first sum above vanishes. Therefore
      \begin{equation}\label{equ:value_of_Dirichlet_L-functions_at_s=1_2}
        L(1,\chi) = -\frac{\chi(-1)\tau(\chi)}{q}\left(\frac{\pi i}{q}\sum_{a \tmod{q}}\chi(a)a+\sum_{a \tmod{q}}\chi(a)\log\left(\sin\left(\frac{\pi a}{q}\right)\right)\right).
      \end{equation}
      \cref{equ:value_of_Dirichlet_L-functions_at_s=1_2} simplifies in that one of the two sums vanish depending on if $\chi$ is even or odd. For the first sum in \cref{equ:value_of_Dirichlet_L-functions_at_s=1_2}, observe that
      \[
        \frac{\pi i}{q}\sum_{a \tmod{q}}\chi(a)a = -\frac{\chi(-1)\pi i}{q}\sum_{a \tmod{q}}\chi(-a)(-a).
      \]
      As $a \mapsto -a$ is a bijection on $\Z/q\Z$, this sum vanishes if $\chi$ is even. For the second sum in \cref{equ:value_of_Dirichlet_L-functions_at_s=1_2}, we have an analogous relation of the form
      \[
        \sum_{a \tmod{q}}\chi(a)\log\left(\sin\left(\frac{\pi a}{q}\right)\right) = \chi(-1)\sum_{a \tmod{q}}\chi(-a)\log\left(\sin\left(\frac{\pi a}{q}\right)\right).
      \]
      As $a \mapsto -a$ is a bijection on $\Z/q\Z$, this sum vanishes if $\chi$ is odd. This finishes the proof.
    \end{proof}

    \cref{thm:Value_of_Dirichlet_L-functions_at_s=1} encodes some interesting identities. For example, if $\chi$ is the non-principal Dirichlet character modulo $4$, then $\chi$ is uniquely defined by $\chi(1) = 1$ and $\chi(3) = \chi(-1) = -1$. In particular, $\chi$ is odd and its conductor is $4$. Now
    \[
      \tau(\chi) = \sum_{a \tmod{4}}\chi(a)e^{\frac{2\pi ia}{4}} = e^{\frac{2\pi i}{4}}-e^{\frac{6\pi i}{4}} = i-(-i) = 2i,
    \]
    so by \cref{thm:Value_of_Dirichlet_L-functions_at_s=1} we get
    \[
      L(1,\chi) = -\frac{\chi(-1)\tau(\chi)\pi i}{16}(1-3) = \frac{\pi}{4}.
    \]
    Expanding out $L(1,\chi)$ gives
    \[
      1-\frac{1}{3}+\frac{1}{5}-\frac{1}{7}+\cdots = \frac{\pi}{4},
    \]
    which is the famous \textbf{Madhava–Leibniz formula}\index{Madhava–Leibniz formula} for $\pi$.
  \section{Siegel's Theorem}
    The discussion of Siegel zeros first arose during the study of zero-free regions for Dirichlet $L$-functions. Refining the argument used in \cref{thm:zero_free_region_generic}, we can show that Siegel zeros only exist when the character $\chi$ is quadratic. But first we improve the zero-free region for the Riemann zeta function:

    \begin{theorem}\label{thm:improved_zero-free_region_zeta}
      There exists a constant $c > 0$ such that $\z(s)$ has no zeros in the region
      \[
        \s \ge 1-\frac{c}{\log(|t|+3)}.
      \]
    \end{theorem}
    \begin{proof}
      By \cref{thm:zero_free_region_generic} applied to $\z(s)$, it suffices to show that $\z(s)$ has no real nontrivial zeros. To see this, let $\eta(s)$ be defined by
      \[
        \eta(s) = \sum_{n \ge 1}\frac{(-1)^{n-1}}{n^{s}}.
      \]
      Note that $\eta(s)$ converges for $\s > 0$ by \cref{prop:Dirichlet_series_convergence_bounded_coefficient_sum}. Now for $0 < s < 1$ and even $n$, $\frac{1}{n^{s}}-\frac{1}{(n+1)^{s}} > 0$ so that $\eta(s) > 0$. But for $\s > 0$, we have
      \[
        (1-2^{1-s})\z(s) = \sum_{n \ge 1}\frac{1}{n^{s}}-2\sum_{n \ge 1}\frac{1}{(2n)^{s}} = \sum_{n \ge 1}\frac{(-1)^{n-1}}{n^{s}} = \eta(s).
      \]
      Therefore $\z(s)$ cannot admit a zero for $0 < s < 1$ because then $\eta(s)$ would be zero too. This completes the proof.
    \end{proof}

    \cref{thm:improved_zero-free_region_zeta} shows that $\z(s)$ has no Siegel zeros. Moreover, since $1-2^{1-s} < 0$ for $0 < s < 1$, the proof shows that $\z(s) < 0$ in this interval as well. As for the height of the first zero, it occurs on the critical line (as predicted by the Riemann hypothesis) at height $t \approx 14.134$ (see \cite{davenport1980multiplicative} for a further discussion). The first $15$ zeros were computed by Gram in 1903 (see \cite{gram1903note}). Since then, billions of zeros have been computed and have all been verified to lie on the critical line. The analogous situation for Dirichlet $L$-functions is only slightly different but causes increasing complexity in further study. We first show that if a Siegel zero exists for the Dirichlet $L$-function of a primitive character, then the character is necessarily quadratic:

    \begin{theorem}\label{thm:improved_zero-free_region_Dirichlet}
      Let $\chi$ be a primitive Dirichlet character of conductor $q > 1$. Then there exists a constant $c > 0$ such that $L(s,\chi)$ has no zeros in the region
      \[
        \s \ge 1-\frac{c}{\log(q(|t|+3))},
      \]
      except for possibly one simple real zero $\b_{\chi}$ with $\b_{chi} < 1$ in the case $\chi$ is quadratic.
    \end{theorem}
    \begin{proof}
      By \cref{thm:zero_free_region_generic} applied to $\z(s)L(s,\chi)$, and shrinking $c$ if necessary, it remains to show that there not a simple real zero $\b_{\chi}$ if $\chi$ is not quadratic. For this, let $L(s,g)$ be the $L$-series defined by
      \[
        L(s,g) = L^{3}(s,\chi_{q,0})L^{4}(s,\chi)L(s,\chi^{2}).
      \]
      We have $d_{g} = 8$ and $\mathfrak{q}(g)$ satisfies
      \[
        \mathfrak{q}(g) \le \mathfrak{q}(\chi_{q,0})^{3}\mathfrak{q}(\chi)^{4}\mathfrak{q}(\chi^{2}) \le q^{8}3^{5} < (3q)^{8}.
      \]
      Moreover, $\Re(\L_{g}(n)) \ge 0$ for $(n,q) = 1$. To see this, suppose $p$ is an unramified prime. The local roots of $L(s,g)$ at $p$ are $1$ with multiplicity three, $\chi(p)$ with multiplicity four, and $\chi^{2}(p)$ with multiplicity one. So for any $k \ge 1$, the sum of $k$-th powers of these local roots is
      \[
        3+4\chi^{k}(p)+\chi^{2k}(p).
      \]
      Writing $\chi(p) = e^{ix}$, the real part of this expression is
      \[
        3+4\cos(x)+\cos(2x) = 2(1+\cos(x))^{2} \ge 0,
      \]
      where we have also made use of the identity $3+4\cos(x)+\cos(2x) = 2(1+\cos(x))^{2}$. Thus $\Re(\L_{g}(n)) \ge 0$ for $(n,q) = 1$, and the conditions of \cref{lem:non-vanshing_at_1_lemma} are satisfied for $L(s,g)$ (recall \cref{equ:non-primitive_primitive_Dirichlet_L-series_relation} for the $L$-series $L(s,\chi_{q,0})$ and $L(s,\chi^{2})$). On the one hand, if $\b$ be a real nontrivial zero of $L(s,\chi)$ then $L(s,g)$ has a real nontrivial zero at $s = \b$ of order at least $4$. On the other hand, using \cref{equ:non-primitive_primitive_Dirichlet_L-series_relation} and that $\chi^{2} \neq \chi_{q,0}$, $L(s,g)$ has a pole at $s = 1$ of order $3$. Then, upon shrinking $c$ if necessary, \cref{lem:non-vanshing_at_1_lemma} gives a contradiction since $r_{g} = 3$. This completes the proof.
    \end{proof}

    Siegel zeros present an unfortunate obstruction to zero-free region results for Dirichlet $L$-functions when the primitive character $\chi$ is quadratic. However, if we no longer require the constant $c$ in the zero-free region to be effective, we can obtain a much better result for how close the Siegel zero can be to $1$. Ultimately, this improved bound results from a lower bound for $L(1,\chi)$ (recall that this is nonzero from our discussion about Dirichlet's theorem on arithmetic progressions). \textbf{Siegel's theorem}\index{Siegel's theorem} refers to either this lower bound or to the improved zero-free region. In the lower bound version, Siegel's theorem is the following:

    \begin{theorem}[Siegel's theorem, lower bound version]
      Let $\chi$ be a primitive quadratic Dirichlet character modulo $q > 1$. Then there exists a positive constant $c_{1}(\e)$ such that
      \[
        L(1,\chi) \ge \frac{c_{1}(\e)}{q^{\e}}.
      \]
    \end{theorem}

    In the zero-free region version, Siegel's theorem takes the following form:

    \begin{theorem}[Siegel's theorem, zero-free region version]
      Let $\chi$ be a primitive quadratic Dirichlet character modulo $q > 1$. Then there exists a positive constant $c_{2}(\e)$ such that $L(s,\chi)$ has no real zeros in the segment
      \[
        \s \ge 1-\frac{c_{2}(\e)}{q^{\e}}.
      \]
    \end{theorem}

    The largest defect of Siegel's theorem, in either version, is that the implicit constants $c_{1}(\e)$ and $c_{2}(\e)$ are ineffective (and not necessarily equal). Actually, the lower bound result is slightly stronger as it implies the zero-free region result. We will first prove the zero-free region given the lower bound, and then we will prove the lower bound. Before we begin, we need two small lemmas about the size of $L'(\s,\chi)$ and $L(\s,\chi)$ for $\s$ close to $1$:

    \begin{lemma}\label{lem:Siegels_theorem_second_version_lemma}
      Let $\chi$ be a non-principal Dirichlet character modulo $m > 1$. Then $L'(\s,\chi) = O(\log^{2}(m))$ for any $\s$ such that $0 \le 1-\s \le \frac{1}{\log(m)}$.
    \end{lemma}
    \begin{proof}
      Setting $A(X) = \sum_{n \le X}\chi(n)$ we have $A(X) \ll 1$ by the orthogonality relations (\cref{cor:Dirichlet_orthogonality_relations} (i)) and that $\chi$ is periodic. Therefore $\s_{c} \le 0$ by \cref{prop:Dirichlet_series_convergence_bounded_coefficient_sum}. Hence for $\s$ in the prescribed region, $L(\s,\chi)$ is holomorphic and its derivative is given by
      \[
        L'(\s,\chi) = \sum_{n \ge 1}\frac{\chi(n)\log(n)}{n^{\s}} = \sum_{n < m}\frac{\chi(n)\log(n)}{n^{\s}}+\sum_{n \ge m}\frac{\chi(n)\log(n)}{n^{\s}}.
      \]
      We will show that the last two sums are both $O(\log^{2}(m))$. For the first sum, if $n < m$, we have
      \[
        \left|\frac{\chi(n)\log(n)}{n^{\s}}\right| \le \frac{1}{n^{\s}}\log(n) = \frac{n^{1-\s}}{n}\log(n) < \frac{m^{1-\s}}{n}\log(n) < \frac{e}{n}\log(m),
      \]
      where the last inequality follows because $1-\s \le \frac{1}{\log(m)}$. Then
      \[
        \left|\sum_{n \le m}\frac{\chi(n)\log(n)}{n^{s}}\right| < e\log(m)\sum_{n < m}\frac{1}{n} < e\log(m)\int_{1}^{m}\frac{1}{n}\,dn \ll \log^{2}(m).
      \]
      For the second sum, $A(Y) \ll 1$ so that $A(Y)\log(Y)Y^{-\s} \to 0$ as $Y \to \infty$. Then Abel's summation formula (see \cref{cor:Abels_summation_formula_limit_version}) gives
      \begin{equation}\label{equ:Siegels_theorem_second_version_lemma_1}
        \sum_{n \ge m}\frac{\chi(n)\log(n)}{n^{\s}} = -A(m)\log(m)m^{-\s}-\int_{m}^{\infty}A(u)(1-\s\log(u))u^{-(\s+1)}\,du.
      \end{equation}
      Since $0 \le 1-\s \le \frac{1}{\log(m)}$, we have $1-\s\log(u) \le \frac{\log(u)}{\log(m)}$. Also, we have the more precise estimate $|A(X)| \le m$ because $\chi$ is $m$-periodic and $|\chi(n)| \le 1$. With these estimates and \cref{equ:Siegels_theorem_second_version_lemma_1} we make the following computation:
      \begin{align*}
        \left|\sum_{n \ge m}\frac{\chi(n)\log(n)}{n^{\s}}\right| &\le |A(m)|\log(m)m^{-\s}+\int_{m}^{\infty}|A(u)|(1-\s\log(u))u^{-(\s+1)}\,du \\
        &\le |A(m)|\log(m)m^{-\s}+\log(m)\int_{m}^{\infty}|A(u)|\log(u)u^{-(\s+1)}\,du \\
        &\le m^{1-\s}\log(m)+m\int_{m}^{\infty}\log(u)u^{-(\s+1)}\,du \\
        &= m^{1-\s}\log(m)+m\left(-\log(u)\frac{u^{-\s}}{\s}\bigg|_{m}^{\infty}+\int_{m}^{\infty}\frac{u^{-(\s+1)}}{s}\,du\right) \\
        &= m^{1-\s}\log(m)+m\left(-\log(u)\frac{u^{-\s}}{\s}-\frac{u^{-\s}}{\s^{2}}\right)\bigg|_{m}^{\infty} \\
        &= m^{1-\s}\log(m)+m\left(\log(m)\frac{m^{-\s}}{\s}+\frac{m^{-\s}}{\s^{2}}\right) \\
        &\ll m^{1-\s}\log(m) \\
        &\ll e\log(m),
      \end{align*}
      where in the fourth line we have used integration by parts and the last line holds because $1-\s \le \frac{1}{\log(m)}$. But $e\log(m) = O(\log^{2}(m))$ so the second sum is also $O(\log^{2}(m))$. Therefore we have shown $L'(\s,\chi) = O(\log^{2}(m))$ finishing the proof. 
    \end{proof}

    The second lemma is even easier and is proved in exactly the same way:

    \begin{lemma}\label{lem:Siegels_theorem_first_version_lemma}
      Let $\chi$ be a non-principal Dirichlet character modulo $m > 1$. Then $L(\s,\chi) = O(\log(m))$ for any $\s$ such that $0 \le 1-\s \le \frac{1}{\log(m)}$.
    \end{lemma}
    \begin{proof}
      Note that
      \[
        L(\s,\chi) = \sum_{n \ge 1}\frac{\chi(n)}{n^{\s}} = \sum_{n < m}\frac{\chi(n)}{n^{\s}}+\sum_{n \ge m}\frac{\chi(n)}{n^{\s}}.
      \]
      It suffices to show that the last two sums are both $O(\log^{2}(m))$. For the first sum, since $n < m$, we have
      \[
        \left|\frac{\chi(n)}{n^{\s}}\right| \le \frac{1}{n^{\s}} = \frac{n^{1-\s}}{n} < \frac{m^{1-\s}}{n} < \frac{e}{n},
      \]
      where the last inequality follows because $1-\s \le \frac{1}{\log(m)}$. Therefore
      \[
        \left|\sum_{n \le m}\frac{\chi(n)}{n^{s}}\right| < e\sum_{n < m}\frac{1}{n} < e\log(m)\int_{1}^{m}\frac{1}{n}\,dn \ll \log(m).
      \]
      As for the second sum, setting $A(Y) = \sum_{n \le Y}\chi(n)$ we have $A(Y) \ll 1$ by the orthogonality relations (\cref{cor:Dirichlet_orthogonality_relations} (i)) and that $\chi$ is periodic. Thus $A(Y)Y^{-\s} \to 0$ as $Y \to \infty$. Then Abel's summation formula (see \cref{cor:Abels_summation_formula_limit_version}) gives
      \begin{equation}\label{equ:Siegels_theorem_first_version_lemma_1}
        \sum_{n \ge m}\frac{\chi(n)}{n^{\s}} = -A(m)m^{-\s}-\int_{m}^{\infty}A(u)u^{-(\s+1)}\,du.
      \end{equation}
      Using the more precise estimate $|A(X)| \le m$, because $\chi$ is $m$-periodic and $|\chi(n)| \le 1$, with \cref{equ:Siegels_theorem_first_version_lemma_1}, we make the following computation:
      \begin{align*}
        \left|\sum_{n \ge m}\frac{\chi(n)\log(n)}{n^{\s}}\right| &\le |A(m)|m^{-\s}+\int_{m}^{\infty}|A(u)|u^{-(\s+1)}\,du \\
        &\le |A(m)|m^{-\s}+\int_{m}^{\infty}|A(u)|\log(u)u^{-(\s+1)}\,du \\
        &\le m^{1-\s}+m\int_{m}^{\infty}\log(u)u^{-(\s+1)}\,du \\
        &= m^{1-\s}+m\left(-\log(u)\frac{u^{-\s}}{\s}\bigg|_{m}^{\infty}+\int_{m}^{\infty}\frac{u^{-(\s+1)}}{s}\,du\right) \\
        &= m^{1-\s}+m\left(-\log(u)\frac{u^{-\s}}{\s}-\frac{u^{-\s}}{\s^{2}}\right)\bigg|_{m}^{\infty} \\
        &= m^{1-\s}+m\left(\log(m)\frac{m^{-\s}}{\s}+\frac{m^{-\s}}{\s^{2}}\right) \\
        &\ll m^{1-\s} \\
        &\ll e,
      \end{align*}
      where in the fourth line we have used integration by parts and the last line holds because $1-\s \le \frac{1}{\log(m)}$. But $e = O(\log^{2}(m))$ so the second sum is also $O(\log^{2}(m))$. Therefore we have shown $L(\s,\chi) = O(\log(m))$ which completes the proof.
    \end{proof}

    We will now prove the zero-free region version of Siegel's theorem, assuming the lower bound version, and using \cref{lem:Siegels_theorem_second_version_lemma}:

    \begin{proof}[Proof of Siegel's theorem, zero-free region version]
      We we will prove the theorem by contradiction. Clearly the result holds for a single $q$, and notice that the result also holds provided we bound $q$ from above by taking the maximum of all the $c_{2}(\e)$. Therefore we may suppose $q$ is arbitrarily large. In this case, if there was a real zero $\b$ with $\b \ge 1-\frac{c_{2}(\e)}{q^{\e}}$, equivalently $1-\b \le \frac{c_{2}(\e)}{q^{\e}}$, then for large enough $q$ we have $0 \le 1-\b \le \frac{1}{\log(q)}$ so that $L'(\s,\chi) = O(\log^{2}(q))$ for $\b \le \s \le 1$ by \cref{lem:Siegels_theorem_second_version_lemma}. These two estimates and the mean value theorem together give
      \[
        L(1,\chi) = L(1,\chi)-L(\b,\chi) = L'(\s,\chi)(1-\b) \ll \frac{\log^{2}(q)}{q^{\e}}.
      \]
      Upon taking $\frac{\e}{2}$ in the lower bound version of Siegel's theorem, we obtain
      \[
        \frac{1}{q^{\frac{\e}{2}}} \ll L(1,\chi) \ll \frac{\log^{2}(q)}{q^{\e}},
      \]
      which is a contradiction for large $q$.
    \end{proof}

    It remains to prove the lower bound version of Siegel's theorem. The idea is to combine two Dirichlet $L$-series attached to distinct characters with distinct moduli and use this new $L$-series to derivative a lower bound for a single Dirichlet $L$-function at $s = 1$. We first need a lemma:

    \begin{lemma}\label{lem:Siegel_zero_auxiliary_L-function_lemma}
      Let $\chi_{1}$ and $\chi_{2}$ be two quadratic Dirichlet characters and let $L(s,g)$ be the $L$-series defined by
      \[
        L(s,g) = \z(s)L(s,\chi_{1})L(s,\chi_{2})L(s,\chi_{1}\chi_{2}).
      \]
      Then $\L_{g}(n) \ge 0$. In particular, $a_{g}(n) \ge 0$ and $a_{g}(0) = 1$.
    \end{lemma}
    \begin{proof}
      For any prime $p$, the local roots at $p$ are $1$ with multiplicity one, $\chi_{1}(p)$ with multiplicity one, $\chi_{2}(p)$ with multiplicity one, and $\chi_{1}\chi_{2}(p)$ with multiplicity one. So for any $k \ge 1$, the sum of $k$-th powers of these local roots is
      \[
        (1+\chi_{1}^{k}(p))(1+\chi_{2}^{k}(p)) \ge 0.
      \]
      Thus $\L_{g}(n) \ge 0$. It follows immediately from the Euler product of $L(s,g)$ that $a_{g}(n) \ge 0$ too. Also, it is clear from the Euler product of $L(s,g)$ that $a_{g}(0) = 1$.
    \end{proof}

    The key ingredient in the proof of the lower bound version of Siegel's theorem is an estimate for the $L$-series $L(s,g)$ in \cref{lem:Siegel_zero_auxiliary_L-function_lemma} relative to the modulus $q_{1}q_{2}$ in a small interval on the real axis close to $1$. We now prove the theorem:

    \begin{proof}[Proof of Siegel's theorem, lower bound version]
      Let $\chi_{1}$ and $\chi_{2}$ be two distinct primitive quadratic and non-principal characters modulo $q_{1}$ and $q_{2}$ respectively. Let $L(s,g)$ be the $L$-series defined by
      \[
        L(s,g) = \z(s)L(s,\chi_{1})L(s,\chi_{2})L(s,\chi_{1}\chi_{2}).
      \]
      Observe that $L(s,g)$ is holomorphic on $\C$ except for a simple pole at $s = 1$. Let $\l$ be the residue at this pole so that
      \[
        \l = L(1,\chi_{1})L(1,\chi_{2})L(1,\chi_{1}\chi_{2}).
      \]
     Now $L(s,g)$ is represented as an absolutely convergent series for $\s > 1$ so that it has a power series about $s = 2$ with radius $1$:
      \[
        L(s,g) = \sum_{m \ge 0}\frac{L^{(m)}(2,g)}{m!}(s-2)^{m},
      \]
      for $|s-2| < 1$. We can compute $L^{(m)}(2,g)$ using the Dirichlet series by differentiating termwise:
      \begin{equation}\label{equ:Siegels_theorem_first_version_1}
        L^{(m)}(2,g) = \frac{d^{m}}{ds^{m}}\left(\sum_{n \ge 1}\frac{a_{g}(n)}{n^{s}}\right)\Bigg|_{s = 2} = (-1)^{m}\sum_{n \ge 1}\frac{a_{g}(n)\log^{m}(n)}{n^{s}}\Bigg|_{s = 2} = (-1)^{m}\sum_{n \ge 1}\frac{a_{g}(n)\log^{m}(n)}{n^{2}}.
      \end{equation}
      Since the $a_{g}(n)$ are nonnegative by \cref{lem:Siegel_zero_auxiliary_L-function_lemma}, it follows that $L^{(m)}(2,g)$ is nonnegative and therefore we may write
      \[
        L(s,g) = \sum_{m \ge 0}b_{g}(m)(2-s)^{m},
      \]
      for $|s-2| < 1$ and with $b_{g}(m)$ nonnegative. Also, \cref{equ:Siegels_theorem_first_version_1} and the fact that the $a_{g}(n)$ are nonnegative with $a_{g}(0) = 1$ together imply that $b_{g}(0) > 1$. Then
      \begin{equation}\label{equ:Siegels_theorem_first_version_2}
        L(s,g)-\frac{\l}{s-1} = L(s,g)-\l\sum_{m \ge 0}(2-s)^{m} = \sum_{m \ge 0}(b_{g}(m)-\l)(2-s)^{m},
      \end{equation}
      and the last series must be absolutely convergent for say $|s-2| < 2$ because $L(s,g)-\frac{\l}{s-1}$ is holomorphic as we have removed the pole at $s = 1$. We now wish to estimate $L(s,g)$ and $\frac{\l}{s-1}$ on the circle $|s-2| = \frac{3}{2}$. Let $\chi$ be a non-principal Dirichlet character modulo $m$ and let $A(X) = \sum_{n \le X}\chi(n)$. Then using Abel's summation formula together with $A(X) \ll 1$ by the orthogonality relations (\cref{cor:Dirichlet_orthogonality_relations} (i)) and that $\chi$ is periodic, we have
      \[
        L(s,\chi) = s\int_{1}^{\infty}A(u)u^{-(s+1)}\,du,
      \]
      for $\s > 0$. Now suppose $\s \ge \frac{1}{2}$. As $|A(X)| \le m$, we obtain
      \[
        |L(s,\chi)| \le m|s|\int_{1}^{\infty}u^{-(\s+1)}\,du = -m|s|\frac{u^{-\s}}{\s}\bigg|_{1}^{\infty} = \frac{m|s|}{\s} \le 2m|s|.
      \]
      In particular, on the disk $|s-2| \le \frac{3}{2}$ we have the estimates
      \[
        L(s,\chi_{1}) \ll q_{1}, \quad L(s,\chi_{2}) \ll q_{2}, \quad \text{and} \quad L(s,\chi_{1}\chi_{2}) \ll q_{1}q_{2}.
      \]
      Since $\z(s)$ is bounded on the circle $|s-2| = \frac{3}{2}$ (it's a compact set) and $\l = L(1,\chi_{1})L(1,\chi_{2})L(1,\chi_{1}\chi_{2})$, we obtain the bounds
      \[
        L(s,g) \ll q_{1}^{2}q_{2}^{2} \quad \text{and} \quad \frac{\l}{s-1} \ll q_{1}^{2}q_{2}^{2},
      \]
      on this circle as well. Cauchy's inequality for the size of coefficients of a power series applied to \cref{equ:Siegels_theorem_first_version_2} on the circle $|s-2| = \frac{3}{2}$ gives
      \begin{equation}\label{equ:Siegels_theorem_first_version_3}
        b_{g}(m)-\l \ll q_{1}^{2}q_{2}^{2}\left(\frac{2}{3}\right)^{m}.
      \end{equation}
      Let $M$ be a positive integer. For real $s$ with $\frac{7}{8} < s < 1$ we have $2-s < \frac{9}{8}$ and using \cref{equ:Siegels_theorem_first_version_2,equ:Siegels_theorem_first_version_3} together we can upper bound the tail of $L(s,g)-\frac{\l}{s-1}$:
      \begin{align*}
        \left|\sum_{m \ge M}(b_{g}(m)-\l)(2-s)^{m}\right| &\le \sum_{m \ge M}|b_{g}(m)-\l|(2-s)^{m} \\
        &\ll q_{1}^{2}q_{2}^{2}\sum_{m \ge M}\left(\frac{2}{3}(2-s)\right)^{m} \\
        &\ll q_{1}^{2}q_{2}^{2}\sum_{m \ge M}\left(\frac{3}{4}\right)^{m} \\
        &\ll q_{1}^{2}q_{2}^{2}\left(\frac{3}{4}\right)^{M} \\
        &\ll q_{1}^{2}q_{2}^{2}e^{-\frac{M}{4}},
      \end{align*}
      where the last estimate follows because $(\frac{3}{4})^{M} < e^{-\frac{M}{4}}$ (which is equivalent to $\log\left(\frac{3}{4}\right) < -\frac{1}{4}$). Let $c$ be the implicit constant. Using that the $b_{g}(m)$ are nonnegative, $b(0) > 1$, and the previous estimate for the tail, we can estimate $L(s,g)-\frac{\l}{s-1}$ from below for $\frac{7}{8} < s < 1$. Indeed, discarding the $b_{g}(m)$ terms for $1 \le m \le M$, bounding the constant term below by $1$, and use the tail estimate, gives
      \begin{equation}\label{equ:Siegels_theorem_first_version_4}
        L(s,g)-\frac{\l}{s-1} \ge 1-\l\sum_{0 \le m \le M-1}(2-s)^{m}-cq_{1}^{2}q_{2}^{2}e^{-\frac{M}{4}} = 1-\l\frac{(2-s)^{M}-1}{1-s}-cq_{1}^{2}q_{2}^{2}e^{-\frac{M}{4}},
      \end{equation}
      which is valid for any positive integer $M$. Now chose $M$ such that
      \begin{equation}\label{equ:Siegels_theorem_first_version_5}
        \frac{1}{2}e^{-\frac{1}{4}} \le cq_{1}^{2}q_{2}^{2}e^{-\frac{M}{4}} < \frac{1}{2}.
      \end{equation}
      Upon isolating $L(s,g)$ in \cref{equ:Siegels_theorem_first_version_4} and using the second estimate in \cref{equ:Siegels_theorem_first_version_5}, we get
      \begin{equation}\label{equ:Siegels_theorem_first_version_6}
        L(s,g) \ge \frac{1}{2}-\l\frac{(2-s)^{M}}{1-s}.
      \end{equation}
      Taking the logarithm of the first estimate in \cref{equ:Siegels_theorem_first_version_5} and isolating $M$, we obtain
      \begin{equation}\label{equ:Siegels_theorem_first_version_7}
        M \le 8\log(q_{1}q_{2})+c,
      \end{equation}
      for some different constant $c$. It follows that
      \begin{equation}\label{equ:Siegels_theorem_first_version_8}
        (2-s)^{M} = e^{M\log(2-s)} < e^{M(1-s)} \le c(q_{1}q_{2})^{8(1-s)},
      \end{equation}
      for some different constant $c$, where in the first estimate we have used the Taylor series of the logarithm truncated at the first term and in the second estimate we have used \cref{equ:Siegels_theorem_first_version_7}. Since $1-s$ is positive for $\frac{7}{8} < s < 1$, we can combine \cref{equ:Siegels_theorem_first_version_6,equ:Siegels_theorem_first_version_8} which gives
      \begin{equation}\label{equ:Siegels_theorem_first_version_9}
        L(s,g) \ge \frac{1}{2}-\l\frac{c}{1-s}(q_{1}q_{2})^{8(1-s)}.
      \end{equation}
      This is our desired lower bound for $L(s,g)$. We will now choose the character $\chi_{1}$. If there exists a Siegel zero $\b_{1}$ with $1-\frac{\e}{16} < \b_{1} < 1$, let $\chi_{1}$ be the character corresponding to the Dirichlet $L$-function that admits this Siegel zero. Then $L(\b_{1},g) = 0$ independent of the choice of $\chi_{2}$. If there is no such Siegel zero, choose $\chi_{1}$ to be any quadratic primitive character and $\b_{1}$ to be any number such that $1-\frac{\e}{16} < \b_{1} < 1$. Then $L(\b_{1},g) < 0$ independent of the choice of $\chi_{2}$. Indeed, $\z(s)$ is negative in this segment (actually for $0 \le s < 1$) and each of the Dirichlet $L$-series defining $L(s,g)$ is positive at $s = 1$ (the Euler product implies Dirichlet $L$-series are positive for $s > 1$ and they are in fact nonzero for $s = 1$ by \cref{thm:non-vanishing_of_Dirichlet_L-functions_at_s=1}) and do not admit a zero for $\b_{1} < s \le 1$ by our choice of $\b_{1}$. In either case, $L(\b_{1},g) \le 0$ so isolating $\l$ and disregarding the constants in \cref{equ:Siegels_theorem_first_version_9} with $s = \b_{1}$ gives the weaker estimate
      \begin{equation}\label{equ:Siegels_theorem_first_version_10}
        \l \gg \frac{1-\b_{1}}{(q_{1}q_{2})^{8(1-\b_{1})}}.
      \end{equation}
      We will now choose $\chi_{2} = \chi$ and hence $q_{2} = q$ as in the statement of the theorem. Notice that, independent of any work we have done, the theorem holds for a single $q$. Moreover, the theorem holds provided we bound $q$ from above by taking the minimum of the $c_{1}(\e)$. Therefore we may assume $q$ is arbitrarily large and in particular that $q > q_{1}$. Using \cref{lem:Siegels_theorem_first_version_lemma} with $\s = 1$ applied to $L(\s,\chi_{1})$ and $L(\s,\chi_{1}\chi)$, we obtain
      \begin{equation}\label{equ:Siegels_theorem_first_version_11}
        \l \ll \log(q_{1})\log(q_{1}q)L(1,\chi).
      \end{equation}
      Combining \cref{equ:Siegels_theorem_first_version_10,equ:Siegels_theorem_first_version_11} yields
      \[
        \frac{1-\b_{1}}{(q_{1}q)^{8(1-\b_{1})}} \ll \log(q_{1})\log(q_{1}q)L(1,\chi).
      \]
      As $\b_{1}$ and $q_{1}$ are fixed and $\log(q_{1}q) = O(\log(q))$, isolating $L(1,\chi)$ gives the weaker estimate
      \begin{equation}\label{equ:Siegels_theorem_first_version_12}
        \frac{1}{q^{8(1-\b_{1})}\log(q)} \ll L(1,\chi).
      \end{equation}
      But $1-\frac{\e}{16} < \b_{1} < 1$ so that $0 < 8(1-\b_{1}) < \frac{\e}{2}$ which combined with \cref{equ:Siegels_theorem_first_version_12} yields
      \[
        \frac{1}{q^{\e}} \ll_{\e} \frac{1}{q^{\frac{\e}{2}}\log(q)} \ll L(1,\chi),
      \]
      where the first estimate follows because $\log(q) \ll_{\e} q^{\frac{\e}{2}}$ for sufficiently large $q$. This is equivalent to the statement in the theorem.
    \end{proof}

    The part of the proof of the lower bound version of Siegel's theorem which makes $c_{1}(\e)$ (and hence $c_{2}(\e)$) ineffective is the value of $\b_{1}$. The choice of $\b_{1}$ depends upon the existence of a Siegel zero near $1$ and relative to the given $\e$. Since we don't know if Siegel zeros exist, this makes estimating $\b_{1}$ relative to $\e$ ineffective. Many results in analytic number theory make use of Siegel's theorem and hence are also ineffective. Many important problems investigate methods to get around using Siegel's theorem in favor of a weaker result that is effective. So far, no Siegel zero has been shown to exist or not exist for Dirichlet $L$-functions. But some progress has been made to showing that they are rare:

    \begin{proposition}\label{prop:product_of_quadratic_Dirichlet_L-functions_has_one_zero}
      Let $\chi_{1}$ and $\chi_{2}$ be two distinct quadratic Dirichlet characters of conductors $q_{1}$ and $q_{2}$. If $L(s,\chi_{1})$ and $L(s,\chi_{2})$ have Siegel zeros $\b_{1}$ and $\b_{2}$ respectively and $\chi_{1}\chi_{2}$ is not principal, then there exists a positive constant $c$ such that
      \[
        \min(\b_{1},\b_{2}) < 1-\frac{c}{\log(q_{1}q_{2})}.
      \]
    \end{proposition}
    \begin{proof}
        We may assume $\chi_{1}$ and $\chi_{2}$ are primitive since if $\wtilde{\chi_{i}}$ is the primitive character inducing $\chi_{i}$, for $i = 1,2$, the only difference in zeros between $L(s,\chi_{i})$ and $L(s,\wtilde{\chi}_{i})$ occur on the line $\s = 0$. Now let $\wtilde{\chi}$ be the primitive character of conductor $q$ inducing $\chi_{1}\chi_{2}$. From \cref{equ:non-primitive_primitive_Dirichlet_L-series_relation} with $\chi_{1}\chi_{2}$ in place of $\chi$, we find that
        \begin{equation}\label{equ:product_of_quadratic_Dirichlet_L-functions_has_one_zero_1}
          \left|\frac{L'}{L}(s,\chi_{1}\chi_{2})-\frac{L'}{L}(s,\wtilde{\chi})\right| = \left|\sum_{p \mid q_{1}q_{2}}\frac{\wtilde{\chi}(p)\log(p)p^{-s}}{1-\wtilde{\chi}(p)p^{-s}}\right| \le \sum_{p \mid q_{1}q_{2}}\frac{\log(p)p^{-\s}}{1-p^{-\s}} \le \sum_{p \mid q_{1}q_{2}}\log(p) \le \log(q_{1}q_{2}).
        \end{equation}
        Let $s = \s$ with $1 < \s \le 2$. Using the reverse triangle inequality, we deduce from \cref{equ:product_of_quadratic_Dirichlet_L-functions_has_one_zero_1} that
        \begin{equation}\label{equ:product_of_quadratic_Dirichlet_L-functions_has_one_zero_2}
          -\frac{L'}{L}(\s,\chi_{1}\chi_{2}) < c\log(q_{1}q_{2}),
        \end{equation}
        for some positive constant $c$.
        By \cref{lem:powerful_L-function_approximation_lemma} (iv) applied to $\z(s)$ while discarding all of the terms in both sums, we have
        \begin{equation}\label{equ:product_of_quadratic_Dirichlet_L-functions_has_one_zero_3}
          -\frac{\z'}{\z}(\s) < A+\frac{1}{\s-1},
        \end{equation}
        for some positive constant $A$. By \cref{lem:powerful_L-function_approximation_lemma} (iv) applied to $L(s,\chi_{i})$ and only retaining the term corresponding to $\b_{i}$, we have
        \begin{equation}\label{equ:product_of_quadratic_Dirichlet_L-functions_has_one_zero_4}
          -\frac{L'}{L}(\s,\chi_{i}) < A\log(q_{i})+\frac{1}{\s-\b_{i}},
        \end{equation}
        for $i = 1,2$ and some possibly larger constant $A$. Now by \cref{lem:Siegel_zero_auxiliary_L-function_lemma}, $-\frac{L'}{L}(\s,g) \ge 0$. Combining \cref{equ:product_of_quadratic_Dirichlet_L-functions_has_one_zero_1,equ:product_of_quadratic_Dirichlet_L-functions_has_one_zero_2,equ:product_of_quadratic_Dirichlet_L-functions_has_one_zero_3,equ:product_of_quadratic_Dirichlet_L-functions_has_one_zero_4} with this fact implies
        \[
         0 < A+\frac{1}{\s-1}+A\log(q_{1})-\frac{1}{\s-\b_{1}}+A\log(q_{2})-\frac{1}{\s-\b_{2}}+c\log(q_{1}q_{2}).
        \]
        Taking $c$ larger, if necessary, we arrive at the simplified estimate
        \[
          0 < \frac{1}{\s-1}-\frac{1}{\s-\b_{1}}-\frac{1}{\s-\b_{2}}+c\log(q_{1}q_{2}),
        \]
        which we rewrite as
        \[
          \frac{1}{\s-\b_{1}}+\frac{1}{\s-\b_{2}} < \frac{1}{\s-1}+c\log(q_{1}q_{2}),
        \]
        Now let $\s = 1+\frac{\d}{\log(q_{1}q_{2})}$ for some $\d > 0$. Upon substituting, we have
        \[
          \frac{1}{\s-\b_{1}}+\frac{1}{\s-\b_{2}} < \left(c+\frac{1}{\d}\right)\log(q_{1}q_{2}).
        \]
        If $\min(\b_{1},\b_{2}) \ge 1-\frac{c}{\log(q_{1}q_{2})}$, then we arrive at
        \[
          2(\d+c) < c+\frac{1}{\d},
        \]
        which is a contradiction if we take $\d$ small enough so that $2\d^{2}+c\d < 1$.
      \end{proof}
    
      From \cref{prop:product_of_quadratic_Dirichlet_L-functions_has_one_zero} we immediately see that for every modulus $m > 1$ there is at most one primitive quadratic Dirichlet character that can admit a Siegel zero:

    \begin{proposition}\label{prop:at_most_one_Siegel_zero_per_modulus}
      For every integer $m > 1$, there is at most one Dirichlet character $\chi$ modulo $m$ such that $L(s,\chi)$ has a Siegel zero. If this Siegel zero exists, $\chi$ is necessarily quadratic.
    \end{proposition}
    \begin{proof}
      Let $\wtilde{\chi}$ be the primitive character inducing $\chi$. As the zeros of $L(s,\chi)$ and $L(s,\wtilde{\chi})$ differ only on the line $\s = 0$, \cref{thm:improved_zero-free_region_Dirichlet} implies that $\chi$ must be quadratic. Suppose $\chi_{1}$ and $\chi_{2}$ are two distinct character modulo $m$, of conductors $q_{1}$ and $q_{2}$, admitting Siegel zeros $\b_{1}$ and $\b_{2}$. Then $\chi_{1}\chi_{2} \neq \chi_{m,0}$. Moreover, $\b_{1} \ge 1-\frac{c_{1}}{\log(q_{1})}$ and $\b_{2} \ge 1-\frac{c_{2}}{\log(q_{2})}$ for some positive constants $c_{1}$ and $c_{2}$. Taking $c$ smaller, if necessary, we have $\min(\b_{1},\b_{2}) \ge 1-\frac{c}{\log(q_{1}q_{2})}$ which contradicts \cref{prop:product_of_quadratic_Dirichlet_L-functions_has_one_zero}.
    \end{proof}
  \section{The Prime Number Theorem}
    The function $\psi(x)$ is defined by
    \[
      \psi(x) = \sum_{n \le x}\L(n),
    \]
    for $x > 0$. We will obtain an explicit formula for $\psi(x)$ analogous to the explicit formula for the Riemann zeta function. The explicit formula for $\psi(x)$ will be obtained by applying truncated Perron's formula to the logarithmic derivative of $\z(s)$. Since $\psi(x)$ is discontinuous when $x$ is a prime power, we need to work with a slightly modified function to apply the Mellin inversion formula. Define $\psi_{0}(x)$ by
    \[
      \psi_{0}(x) = \begin{cases} \psi(x) & \text{if $x$ is not a prime power}, \\ \psi(x)-\frac{1}{2}\L(x) & \text{if $x$ is a prime power}. \end{cases}
    \]
    Equivalently, $\psi_{0}(x)$ is $\psi(x)$ except that its value is halfway between the limit values when $x$ is a prime power. Stated another way, if $x$ is a prime power the last term in the sum for $\psi_{0}(x)$ is multiplied by $\frac{1}{2}$. The \textbf{explicit formula}\index{explicit formula} for $\psi(x)$ is the following:

    \begin{theorem}[Explicit formula for $\psi(x)$]
      For $x \ge 2$,
      \[
        \psi_{0}(x) = x-\sum_{\rho}\frac{x^{\rho}}{\rho}-\frac{\z'}{\z}(0)-\frac{1}{2}\log(1-x^{-2}),
      \]
      where the sum is counted with multiplicity and ordered with respect to the size of the ordinate.
    \end{theorem}

    A few comments are in order before we prove the explicit formula for $\psi(x)$. First, since $\rho$ is conjectured to be of the form $\rho = \frac{1}{2}+i\g$ via the Riemann hypothesis for the Riemann zeta function, $x$ is conjectured to be the main term in the explicit formula. The constant $\frac{\z'}{\z}(0)$ can be shown to be $\log(2\pi)$ (see \cite{davenport1980multiplicative} for a proof). Also, using the Taylor series of the logarithm, the last term can be expressed as
    \[
      \frac{1}{2}\log(1-x^{-2}) = \frac{1}{2}\sum_{m \ge 1}(-1)^{m-1}\frac{(-x^{-2})^{m}}{m} = \sum_{m \ge 1}(-1)^{2m-1}\frac{x^{-2m}}{2m} = \sum_{m \ge 1}\frac{x^{-2m}}{-2m} = \sum_{\w}\frac{x^{\w}}{\w},
    \]
    where $\w$ runs over the trivial zeros of $\z(s)$. We will now prove the explicit formula for $\psi(x)$:

    \begin{proof}[Proof of the explicit formula for $\psi(x)$]
      Applying truncated Perron's formula to $-\frac{\z'}{\z}(s)$ gives
      \begin{equation}\label{equ:explicit_formula_zeta_proof_1}
        \psi_{0}(x)-J(x,T) \ll x^{c}\sum_{\substack{n \ge 1 \\ n \neq x}}\frac{\L(n)}{n^{c}}\min\left(1,\frac{1}{T\left|\log\left(\frac{x}{n}\right)\right|}\right)+\d_{x}\L(x)\frac{c}{T},
      \end{equation}
      where
      \[
        J(x,T) = \frac{1}{2\pi i}\int_{c-iT}^{c+iT}-\frac{\z'}{\z}(s)x^{s}\,\frac{ds}{s},
      \]
      $c > 1$, and it is understood that $\d_{x} = 0$ unless $x$ is a prime power. Take $T > 2$ not coinciding with the ordinate of a nontrivial zero and let $c = 1+\frac{1}{\log(x^{2})}$ so that $x^{c} = \sqrt{e}x$ and $1 < c < 2$. The first step is to estimate the right-hand side of \cref{equ:explicit_formula_zeta_proof_1}. We deal with the terms corresponding to $n$ such that $n$ is bounded away from $x$ before anything else. So suppose $n \le \frac{3}{4}x$ or $n \ge \frac{5}{4}x$. For these $n$, $\log\left(\frac{x}{n}\right)$ is bounded away from zero so that their contribution is
      \begin{equation}\label{equ:explicit_formula_zeta_proof_2}
        \ll \frac{x^{c}}{T}\sum_{n \ge 1}\frac{\L(n)}{n^{c}} \ll \frac{x^{c}}{T}\left(-\frac{\z'}{\z}(c)\right) \ll \frac{x\log(x)}{T},
      \end{equation}
      where the last estimate follows from \cref{lem:powerful_L-function_approximation_lemma} (iv) applied to $\z(s)$ while discarding all of the terms in both sums and our choice of $c$ (in particular $\log(c) \ll \log(x)$). Now we deal with the terms $n$ close to $x$. Consider those $n$ for which $\frac{3}{4}x < n < x$. Let $x_{1}$ be the largest prime power less than $x$. We may also suppose $\frac{3}{4}x < x_{1} < x$ since otherwise $\L(n) = 0$ and these terms do not contribute anything. Moreover, $\frac{x^{c}}{n^{c}} \ll 1$. For the term $n = x_{1}$, we have
      \[
        \log\left(\frac{x}{n}\right) = -\log\left(1-\frac{x-x_{1}}{x}\right) \ge \frac{x-x_{1}}{x},
      \]
      where we have obtained the inequality by using Taylor series of the logarithm truncated after the first term. The contribution of this term is then
      \begin{equation}\label{equ:explicit_formula_zeta_proof_3}
        \ll \L(x_{1})\min\left(1,\frac{x}{T(x-x_{1})}\right) \ll \log(x)\min\left(1,\frac{x}{T(x-x_{1})}\right).
      \end{equation}
      For the other such $n$, we can write $n = x_{1}-v$, where $v$ is an integer satisfying $0 < v < \frac{1}{4}x$, so that
      \[
        \log\left(\frac{x}{n}\right) \ge \log\left(\frac{x_{1}}{n}\right) = -\log\left(1-\frac{v}{x_{1}}\right) \ge \frac{v}{x_{1}},
      \]
      where we have obtained the latter inequality by using Taylor series of the logarithm truncated after the first term. The contribution for these $n$ is then
      \begin{equation}\label{equ:explicit_formula_zeta_proof_4}
        \ll \sum_{0 < v < \frac{1}{4}x}\L(x_{1}-v)\frac{x_{1}}{Tv} \ll \frac{x}{T}\sum_{0 < v < \frac{1}{4}x}\frac{\L(x_{1}-v)}{v} \ll \frac{x\log(x)}{T}\sum_{0 < v < \frac{1}{4}x}\frac{1}{v} \ll \frac{x\log^{2}(x)}{T}.
      \end{equation}
      The contribution for those $n$ for which $x < n < \frac{5}{4}x$ is handled in exactly the same way with $x_{1}$ being the least prime power larger than $x$. Let $\<x\>$ be the distance between $x$ and the nearest prime power other than $x$ if $x$ itself is a prime power. Combining \cref{equ:explicit_formula_zeta_proof_3,equ:explicit_formula_zeta_proof_4} with our previous comment, the contribution for those $n$ with $\frac{3}{4}x < n < \frac{5}{4}x$ is
      \begin{equation}\label{equ:explicit_formula_zeta_proof_5}
        \ll \frac{x\log^{2}(x)}{T}+\log(x)\min\left(1,\frac{x}{T\<x\>}\right).
      \end{equation}
      Putting \cref{equ:explicit_formula_zeta_proof_2,equ:explicit_formula_zeta_proof_5} together and noticing that the error term in \cref{equ:explicit_formula_zeta_proof_2} is absorbed by the second error term in \cref{equ:explicit_formula_zeta_proof_5}, we obtain
      \begin{equation}\label{equ:explicit_formula_zeta_proof_6}
        \psi_{0}(x)-J(x,T) \ll \frac{x\log^{2}(x)}{T}+\log(x)\min\left(1,\frac{x}{T\<x\>}\right).
      \end{equation}
      This is the first part of the proof. Now we estimate $J(x,T)$ by appealing to the residue theorem. Let $U \ge 1$ be an odd integer. Let $\W$ be the region enclosed by the contours $\eta_{1},\ldots,\eta_{4}$ in \cref{fig:explict_formula_zeta_contour} and set $\eta = \sum_{1 \le i \le 4}\eta_{i}$ so that $\eta = \del \W$.

      \begin{figure}[ht]
        \centering
        \begin{tikzpicture}[scale=2]
          \def\xmin{-3.5} \def\xmax{2}
          \def\ymin{-2} \def\ymax{2}
          \draw[thick] (\xmin,0) -- (\xmax,0);
          \draw[very thick] (0,\ymin) -- (0,\ymax);
          \draw[very thick] (1,\ymin) -- (1,\ymax);
          \draw[dashed] (0.5,\ymin) -- (0.5,\ymax);

          \draw[->-] (1.5,-1.5) -- (1.5,1.5);
          \draw[->-] (1.5,1.5) -- (-3,1.5);
          \draw[->-] (-3,1.5) -- (-3,-1.5);
          \draw[->-] (-3,-1.5) -- (1.5,-1.5);

          \node at (1.5,0) [below right] {\tiny{$\eta_{1}$}};
          \node at (-1,1.5) [above] {\tiny{$\eta_{2}$}};
          \node at (-3,0) [below left] {\tiny{$\eta_{3}$}};
          \node at (-1,-1.5) [below] {\tiny{$\eta_{4}$}};

          \node at (1.5,-1.5) [circle,fill,inner sep=1.5pt]{};
          \node at (1.5,1.5) [circle,fill,inner sep=1.5pt]{};
          \node at (0.5,1.5) [circle,fill,inner sep=1.5pt]{};
          \node at (-3,1.5) [circle,fill,inner sep=1.5pt]{};
          \node at (-3,-1.5) [circle,fill,inner sep=1.5pt]{};
          \node at (0.5,-1.5) [circle,fill,inner sep=1.5pt]{};

          \node at (1.5,-1.5) [below left] {\tiny{$c-iT$}};
          \node at (1.5,1.5) [above] {\tiny{$c+iT$}};
          \node at (0.5,1.5) [above left] {\tiny{$\frac{1}{2}+iT$}};
          \node at (-3,1.5) [above] {\tiny{$-U+iT$}};
          \node at (-3,-1.5) [below left] {\tiny{$-U-iT$}};
          \node at (0.5,-1.5) [below left] {\tiny{$\frac{1}{2}-iT$}};
        \end{tikzpicture}
        \caption{Contour for the explicit formula for $\psi(x)$}
        \label{fig:explict_formula_zeta_contour}
      \end{figure}

      We may express $J(x,T)$ as
      \[
        J(x,T) = \frac{1}{2\pi i}\int_{\eta_{1}}-\frac{\z'}{\z}(s)x^{s}\,\frac{ds}{s}.
      \]
      The residue theorem together with the formula for the negative logarithmic derivative in \cref{prop:explicit_formula_log_derivative} applied to $\z(s)$ and \cref{cor:logarithmic_derivative_of_gamma} imply
      \begin{equation}\label{equ:explicit_formula_zeta_proof_7}
        J(x,T) = x-\sum_{|\g| < T}\frac{x^{\rho}}{\rho}-\frac{\z'}{\z}(0)-\sum_{0 < 2m < U}\frac{x^{-2m}}{-2m}+\frac{1}{2\pi i}\int_{\eta_{2}+\eta_{3}+\eta_{4}}-\frac{\z'}{\z}(s)x^{s}\,\frac{ds}{s},
      \end{equation}
      where $\rho = \b+i\g$ is a nontrivial zero of $\z$. We will estimate $J(x,T)$ by estimating the remaining integral. By \cref{lem:powerful_L-function_approximation_lemma} (ii) applied to $\z(s)$, the number of nontrivial zeros satisfying $|\g-T| < 1$ is $O(\log(T))$. Among the ordinates of these nontrivial zeros, there must be a gap of size $\gg \frac{1}{\log(T)}$. Upon varying $T$ by a bounded amount (we are varying in the interval $[T-1,T+1]$) so that it belongs to this gap, we can additionally ensure
      \[
        \g-T \gg \frac{1}{\log(T)},
      \]
      for all the nontrivial zeros of $\z(s)$. To estimate part of the horizontal integrals over $\eta_{2}$ and $\eta_{4}$, \cref{lem:powerful_L-function_approximation_lemma} (iv) applied to $\z(s)$ gives
      \[
        \frac{\z'}{\z}(s) = \sum_{|\g-T| < 1}\frac{1}{s-\rho}+O(\log(T)),
      \]
      on the parts of these segments with $-1 \le \s \le 2$. By our choice of $T$, $|s-\rho| \ge |\g-T| \gg \frac{1}{\log(T)}$ so that each term in the sum is $O(\log(T))$. There are at most $O(\log(T))$ such terms by \cref{lem:powerful_L-function_approximation_lemma} (ii) applied to $\z(s)$, so we find that
      \[
        \frac{\z'}{\z}(s) = O(\log^{2}(T)),
      \]
      on the parts of these segments with $-1 \le \s \le 2$. It follows that the parts of the horizontal integrals over $\eta_{2}$ and $\eta_{4}$ with  $-1 \le \s \le c$ (recall $c < 2$) contribute
      \begin{equation}\label{equ:explicit_formula_zeta_proof_8}
        \ll \frac{\log^{2}(T)}{T}\int_{-1}^{c}x^{\s}\,d\s \ll \frac{\log^{2}(T)}{T}\int_{-\infty}^{c}x^{\s}\,d\s \ll \frac{x\log^{2}(T)}{T\log(x)},
      \end{equation}
      where in the last estimate we have used the choice of $c$. To estimate the remainder of the horizontal integrals, we need a bound for $\frac{\z'}{\z}(s)$ when $\s < -1$ and away from the trivial zeros. To this end, write the functional equation for $\z(s)$ in the form
      \[
        \z(s) = \pi^{s-1}\frac{\G\left(\frac{1-s}{2}\right)}{\G\left(\frac{s}{2}\right)}\z(1-s),
      \]
      and take the logarithmic derivative to get
      \[
        \frac{\z'}{\z}(s) = \log(\pi)+\frac{1}{2}\frac{\G'}{\G}\left(\frac{1-s}{2}\right)-\frac{1}{2}\frac{\G'}{\G}\left(\frac{s}{2}\right)+\frac{\z'}{\z}(1-s).
      \]
      Let $s$ be such that $\s < -1$ and suppose $s$ is distance $\frac{1}{2}$ away from the trivial zeros. We will estimate every term on the right-hand side of the previous identity. The first term is constant and the last term is bounded since it is an absolutely convergent Dirichlet series. As for the digamma terms, since $s$ is away from the trivial zeros, \cref{equ:approximtion_for_digamma} implies $\frac{1}{2}\frac{\G'}{\G}\left(\frac{1-s}{2}\right) = O(\log{|1-s|})$ and $\frac{1}{2}\frac{\G'}{\G}\left(\frac{s}{2}\right) = O(\log{|s|})$. However, as $\s < -1$ and $s$ is away from the trivial zeros, $s$ and $1-s$ are bounded away from zero so that $\frac{1}{2}\frac{\G'}{\G}\left(\frac{1-s}{2}\right) = O(\log{|s|})$. Putting these estimates together gives
      \begin{equation}\label{equ:explicit_formula_zeta_proof_9}
        \frac{\z'}{\z}(s) \ll \log(|s|),
      \end{equation}
      for $\s < -1$. Using \cref{equ:explicit_formula_zeta_proof_9}, the parts of the horizontal integrals over $\eta_{2}$ and $\eta_{4}$ with $-U \le \s \le -1$ contribute
      \begin{equation}\label{equ:explicit_formula_zeta_proof_10}
        \ll \frac{\log(T)}{T}\int_{-U}^{-1}x^{\s}\,d\s \ll \frac{\log(T)}{Tx\log(x)}.
      \end{equation}
      Combining \cref{equ:explicit_formula_zeta_proof_8,equ:explicit_formula_zeta_proof_10} gives
      \begin{equation}\label{equ:explicit_formula_zeta_proof_11}
        \frac{1}{2\pi i}\int_{\eta_{2}+\eta_{4}}-\frac{\z'}{\z}(s)x^{s}\,\frac{ds}{s} \ll \frac{x\log^{2}(T)}{T\log(x)}+\frac{\log(T)}{Tx\log(x)} \ll \frac{x\log^{2}(T)}{T\log(x)}.
      \end{equation}
      To estimate the vertical integral, we use \cref{equ:explicit_formula_zeta_proof_9} again to conclude that
      \begin{equation}\label{equ:explicit_formula_zeta_proof_12}
        \frac{1}{2\pi i}\int_{\eta_{3}}-\frac{\z'}{\z}(s)x^{s}\,\frac{ds}{s} \ll \frac{\log(U)}{U}\int_{-T}^{T}x^{-U}\,dt \ll \frac{T\log(U)}{Ux^{U}}.
      \end{equation}
      Combining \cref{equ:explicit_formula_zeta_proof_7,equ:explicit_formula_zeta_proof_11,equ:explicit_formula_zeta_proof_12} and taking the limit as $U \to \infty$, the error term in \cref{equ:explicit_formula_zeta_proof_12} vanishes and the sum over $m$ in \cref{equ:explicit_formula_zeta_proof_7} evaluates to $\frac{1}{2}\log(1-x^{-2})$ (as we have already mentioned) giving
      \begin{equation}\label{equ:explicit_formula_zeta_proof_13}
        J(x,T) = x-\sum_{|\g| < T}\frac{x^{\rho}}{\rho}-\frac{\z'}{\z}(0)-\frac{1}{2}\log(1-x^{-2})+\frac{x\log^{2}(T)}{T\log(x)}.
      \end{equation}
      Substituting \cref{equ:explicit_formula_zeta_proof_13} into \cref{equ:explicit_formula_zeta_proof_6}, we at last obtain
      \begin{equation}\label{equ:explicit_formula_zeta_proof_14}
        \psi_{0}(x) = x-\sum_{|\g| < T}\frac{x^{\rho}}{\rho}-\frac{\z'}{\z}(0)-\frac{1}{2}\log(1-x^{-2})+\frac{x\log^{2}(xT)}{T}+\log(x)\min\left(1,\frac{x}{T\<x\>}\right),
      \end{equation}
      where the second to last term on the right-hand side is obtained by combining the error term in \cref{equ:explicit_formula_zeta_proof_11} with the first error term in \cref{equ:explicit_formula_zeta_proof_6}. The theorem follows by taking the limit as $T \to \infty$.
    \end{proof}

    Note that the convergence of the right-hand side in the explicit formula for $\psi(x)$ is uniform in any interval not containing a prime power since $\psi(x)$ is continuous there. Moreover, we have an approximate formula for $\psi(x)$ as a corollary:

    \begin{corollary}\label{cor:explicit_formula_zeta_corollary}
      For $x \ge 2$ and $T > 2$,
      \[
        \psi_{0}(x) = x-\sum_{|\g| < T}\frac{x^{\rho}}{\rho}+R(x,T),
      \]
      where $\rho$ runs over the nontrivial zeros of $\z(s)$ counted with multiplicity and ordered with respect to the size of the ordinate, and
      \[
        R(x,T) \ll \frac{x\log^{2}(xT)}{T}+\log(x)\min\left(1,\frac{x}{T\<x\>}\right),
      \]
      where $\<x\>$ is the distance between $x$ and the nearest prime power other than $x$ if $x$ itself is a prime power. Moreover, if $x$ is an integer, we have the simplified estimate
      \[
        R(x,T) \ll \frac{x\log^{2}(xT)}{T}.
      \]
    \end{corollary}
    \begin{proof}
      This follows from \cref{equ:explicit_formula_zeta_proof_14} since $\frac{\z'}{\z}(0)$ is constant and $\frac{1}{2}\log(1-x^{2})$ is bounded for $x \ge 2$. If $x$ is an integer, then $\<x\> \ge 1$ so that $\log(x)\min\left(1,\frac{x}{T\<x\>}\right) \le \frac{x\log(x)}{T}$ and this term can be absorbed into $O\left(\frac{x\log^{2}(xT)}{T}\right)$.
    \end{proof}

    With our refined explicit formula in hand, we are ready to discuss and prove the prime number theorem. The \textbf{prime counting function}\index{prime counting function} $\pi(x)$ is defined by
    \[
      \pi(x) = \sum_{p \le x}1,
    \]
    for $x > 0$. Equivalently, $\pi(x)$ counts the number of primes that no larger than $x$. Euclid's infinitude of the primes is equivalent to $\pi(x) \to \infty$ as $x \to \infty$. A more interesting question is to ask how the primes are distributed among the integers. The classical \textbf{prime number theorem}\index{prime number theorem} answers this question and the precise statement is the following:

    \begin{theorem}[Prime number theorem, classical version]
      For $x \ge 2$,
      \[
        \pi(x) \sim \frac{x}{\log(x)}.
      \]
    \end{theorem}

    We will delay the proof for the moment and give some intuition and historical context to the result. Intuitively, the prime number theorem is a result about how dense the primes are in the integers. To see this, notice that the result is equivalent to the asymptotic
    \[
      \frac{\pi(x)}{x} \sim \frac{1}{\log(x)}.
    \]
    Letting $x \ge 2$, the left-hand side is the probability that a randomly chosen positive integer no larger than $x$ is prime. Thus the asymptotic result says that for large enough $x$, the probability that a randomly chosen integer no larger than $x$ is prime is approximately $\frac{1}{\log(x)}$. We can also interpret this as saying that the average gap between primes no larger than $x$ is approximately $\frac{1}{\log(x)}$. As a consequence, a positive integer with at most $2n$ digits is about half as likely to be prime than a positive integer with at most $n$ digits. Indeed, there are $10^{n}-1$ numbers with at most $n$ digits, $10^{2n}-1$ with at most $2n$ digits, and $\log(10^{2n}-1)$ is approximately $2\log(10^{n})$. Note that the prime number theorem says nothing about the exact error $\pi(x)-\frac{x}{\log(x)}$ as $x \to \infty$. The theorem only says that the relative error tends to zero:
    \[
      \lim_{x \to \infty}\frac{\pi(x)-\frac{x}{\log(x)}}{\frac{x}{\log(x)}} = 0.
    \]
    Now for some historical context. While Gauss was not the first to put forth a conjectural form of the prime number theorem, he was known for compiling extensive tables of primes and he suspected that the density of the primes up to $x$ was roughly $\frac{1}{\log(x)}$. How might one suspect this is the correct density? Well, let $d\d_{p}$ be the weighted point measure that assigns $\frac{1}{p}$ at the prime $p$ and zero everywhere else. Then
    \[
      \sum_{p \le x}\frac{1}{p} = \int_{1}^{x}\,d\d_{p}(u).
    \]
    We can interpret the integral as integrating the density $d\d_{p}$ over $[1,x]$. Let's try and find a more explicit expression for the density $d\d_{p}$. Euler (see \cite{euler1744variae}), argued
    \[
      \sum_{p \le x}\frac{1}{p} \sim \log\log(x).
    \]
    But notice that
    \[
      \log\log(x) = \int_{1}^{\log(x)}\frac{du}{u} = \int_{e}^{x}\frac{1}{u}\,\frac{du}{\log{u}},
    \]
    where in the second equality we have made the change of variables $u \mapsto \log(u)$. So altogether,
    \[
      \sum_{p \le x}\frac{1}{p} \sim \int_{e}^{x}\frac{1}{u}\,\frac{du}{\log{u}}.
    \]
    This is an asymptotic that gives a more explicit representation of the density $d\d_{p}$. Notice that both sides of this asymptotic are weighted the same, the left-hand side by $\frac{1}{p}$, and the right-hand side by $\frac{1}{u}$. If we remove these weight (this is not strictly allowed), then we might hope
    \[
      \pi(x) = \sum_{p \le x}1 \sim \int_{e}^{x}\frac{du}{\log(u)}.
    \]
    Accordingly, we define the \textbf{logarithmic integral}\index{logarithmic integral} $\Li(x)$ by
    \[
      \Li(x) = \int_{2}^{x}\frac{dt}{\log(t)},
    \]
    for $x \ge 2$. Notice that $\Li(x) \sim \frac{x}{\log{x}}$ because
    \[
      \lim_{x \to \infty}\left|\frac{\Li(x)}{\frac{x}{\log{x}}}\right| = \lim_{x \to \infty}\left|\frac{\int_{2}^{x}\frac{dt}{\log(t)}}{\frac{x}{\log{x}}}\right| = \lim_{x \to \infty}\left|\frac{\frac{1}{\log(x)}}{\frac{\log(x)-1}{\log^{2}(x)}}\right| = \lim_{x \to \infty}\left|\frac{\log(x)}{\log(x)-1}\right| = 1.
    \]
    where in the second equality we have used  L'H\^opital's rule. So an equivalent statement is the logarithmic integral version of the \textbf{prime number theorem}\index{prime number theorem}:

    \begin{theorem}[Prime number theorem, logarithmic integral version]
      For $x \ge 2$,
      \[
        \pi(x) \sim \Li(x).
      \]
    \end{theorem}

    Interpreting the logarithmic integral as an integral of density, then for large $x$ the density of primes up to $x$ is approximately $\frac{1}{\log(x)}$ which is what both versions of the prime number theorem claim. Legendre was the first to put forth a conjectural form of the prime number theorem. In 1798 (see \cite{legendre1798essai}) he claimed that $\pi(x)$ was of the form
    \[
      \frac{x}{A\log(x)+B},
    \]
    for some constants $A$ and $B$. In 1808 (see \cite{legendre1808essai}) he refined his conjecture by claiming
    \[
      \frac{x}{\log(x)+A(x)},
    \]
    where $\lim_{x \to \infty}A(x) \approx 1.08366$. Riemann's 1859 manuscript (see \cite{riemann1859ueber}) contains an outline for how to prove the prime number theorem, but it was not until 1896 that the prime number theorem was proved independently by Hadamard and de la Vall\'ee Poussin (see \cite{hadamard1896distribution} and \cite{poussin1897recherches}). Their proofs, as well as every proof thereon out until 1949, used complex analytic methods in an essential way (there are now elementary proofs due to Erd\"os and Selberg). We are now ready to prove the prime number theorem. Strictly speaking, we will prove the absolute error version of the \textbf{prime number theorem}\index{prime number theorem}, due to de la Vall\'ee Poussin, which bounds the absolute error between $\pi(x)$ and $\Li(x)$:

    \begin{theorem}[Prime number theorem, absolute error version]
      For $x \ge 2$, there exists a positive constant $c$ such that
      \[
        \pi(x) = \Li(x)+O\left(xe^{-c\sqrt{\log(x)}}\right).
      \]
    \end{theorem}
    \begin{proof}
      It suffices to assume $x$ is an integer, because $\pi(x)$ can only change value at integers and the other functions in the statement are increasing. We will first prove
      \begin{equation}\label{equ:prime_number_theorem_zeta_1}
        \psi(x) = x+O\left(xe^{-c\sqrt{\log(x)}}\right),
      \end{equation}
      for some positive constant $c$. To achieve this, we estimate the sum over the nontrivial zeros of $\z(s)$ in \cref{cor:explicit_formula_zeta_corollary}. So let $T > 2$ not coinciding with the ordinate of a nontrivial zero, and suppose $\rho = \b+i\g$ is a nontrivial zero with $|\g| < T$. By \cref{thm:improved_zero-free_region_zeta}, we know $\b < 1-\frac{c}{\log(T)}$ for some positive constant $c$. It follows that
      \begin{equation}\label{equ:prime_number_theorem_zeta_2}
        |x^{\rho}| = x^{\b} < x^{1-\frac{c}{\log(T)}} = xe^{-c\frac{\log(x)}{\log(T)}}.
      \end{equation}
      As $|\rho| > |\g|$, letting $\g_{1} > 0$ (which is bounded away from zero since the zeros of $\z(s)$ are discrete and we know that there is no real nontrivial zero) be the ordinate of the first nontrivial zero, applying integration by parts gives
      \begin{equation}\label{equ:prime_number_theorem_zeta_3}
        \sum_{|\g| < T}\frac{1}{\rho} \ll \sum_{\g_{1} \le |\g| < T}\frac{1}{\g} \ll \int_{\g_{1}}^{T}\frac{dN(t)}{t} = \frac{N(T)}{T}+\int_{\g_{1}}^{T}\frac{N(t)}{t^{2}}\,dt \ll \log^{2}(T),
      \end{equation}
      where in the last estimate we have used that $N(t) \ll t\log(t)$ which follows from \cref{cor:zero_density}. Putting \cref{equ:prime_number_theorem_zeta_2,equ:prime_number_theorem_zeta_3} together gives
      \begin{equation}\label{equ:prime_number_theorem_zeta_4}
        \sum_{|\g| < T}\frac{x^{\rho}}{\rho} \ll x\log^{2}(T)e^{-c\frac{\log(x)}{\log(T)}}.
      \end{equation}
      As $\psi(x) \sim \psi_{0}(x)$ and $x$ is an integer, \cref{equ:prime_number_theorem_zeta_4} with \cref{cor:explicit_formula_zeta_corollary} together imply
      \begin{equation}\label{equ:prime_number_theorem_zeta_5}
        \psi(x)-x \ll x\log^{2}(T)e^{-c\frac{\log(x)}{\log(T)}}+\frac{x\log^{2}(xT)}{T}.
      \end{equation}
      We will now let $T$ be determined by
      \[
        \log^{2}(T) = \log(x),
      \]
      or equivalently,
      \[
        T = e^{\sqrt{\log(x)}}.
      \]
      With this choice of $T$ (note that if $x \ge 2$ then $T > 2$), we can estimate \cref{equ:prime_number_theorem_zeta_5} as follows:
      \begin{align*}
        \psi(x)-x &\ll x\log(x)e^{-c\sqrt{\log(x)}}+x\left(\log^{2}(x)+\log(x)\right)e^{-\sqrt{\log(x)}} \\
        &\ll x\log(x)e^{-c\sqrt{\log(x)}}+x\log^{2}(x)e^{-\sqrt{\log(x)}} \\
        &\ll x\log^{2}(x)e^{-\min(1,c)\sqrt{\log(x)}}.
      \end{align*}
      As $\log(x) \ll_{\e} e^{-\e\sqrt{\log(x)}}$, we conclude that
      \[
        \psi(x)-x \ll xe^{-c\sqrt{\log(x)}},
      \]
      for some smaller $c$ with $c < 1$. This is equivalent to \cref{equ:prime_number_theorem_zeta_1}. Now let
      \[
        \pi_{1}(x) = \sum_{n \le x}\frac{\L(n)}{\log(n)}.
      \]
      We can write $\pi_{1}(x)$ in terms of $\psi(x)$ as follows:
      \begin{align*}
        \pi_{1}(x) &= \sum_{n \le x}\frac{\L(n)}{\log(n)} \\
        &= \sum_{n \le x}\L(n)\int_{n}^{x}\frac{dt}{t\log^{2}(t)}+\frac{1}{\log(x)}\sum_{n \le x}\L(n) \\
        &= \int_{2}^{x}\sum_{n \le t}\L(n)\frac{dt}{t\log^{2}(t)}+\frac{1}{\log(x)}\sum_{n \le x}\L(n) \\
        &= \int_{2}^{x}\frac{\psi(t)}{t\log^{2}(t)}\,dt+\frac{\psi(x)}{\log(x)}.
      \end{align*}
      Applying \cref{equ:prime_number_theorem_zeta_1} to the last expression yields
      \begin{equation}\label{equ:prime_number_theorem_zeta_6}
        \pi_{1}(x) = \int_{2}^{x}\frac{t}{t\log^{2}(t)}\,dt+\frac{x}{\log(x)}+O\left(\int_{2}^{x}\frac{e^{-c\sqrt{\log(t)}}}{\log^{2}(t)}\,dt+\frac{xe^{-c\sqrt{\log(x)}}}{\log(x)}\right).
      \end{equation}
      Upon applying integrating by parts to the main term in \cref{equ:prime_number_theorem_zeta_6}, we obtain
      \begin{equation}\label{equ:prime_number_theorem_zeta_7}
        \int_{2}^{x}\frac{t}{t\log^{2}(t)}\,dt+\frac{x}{\log(x)} = \int_{2}^{x}\frac{dt}{\log(t)}+\frac{2}{\log(2)} = \Li(x)+\frac{2}{\log(2)}.
      \end{equation}
      As for the error term in \cref{equ:prime_number_theorem_zeta_6}, $\log^{2}(t)$ and $\log(x)$ are both bounded away from zero so that
      \[
        \int_{2}^{x}\frac{e^{-c\sqrt{\log(t)}}}{\log^{2}(t)}\,dt+\frac{xe^{-c\sqrt{\log(x)}}}{\log(x)} \ll \int_{2}^{x}e^{-c\sqrt{\log(t)}}\,dt+xe^{-c\sqrt{\log(x)}}.
      \]
      For $t \le x^{\frac{1}{4}}$, we use the bound $e^{-c\sqrt{\log(t)}} < 1$ so that
      \[
        \int_{2}^{x^{\frac{1}{4}}}e^{-c\sqrt{\log(t)}}\,dt < \int_{2}^{x^{\frac{1}{4}}}\,dt \ll x^{\frac{1}{4}}.
      \]
      For $t > x^{\frac{1}{4}}$, $\sqrt{\log(t)} > \frac{1}{2}\sqrt{\log(x)}$ and thus
      \[
        \int_{2}^{x^{\frac{1}{4}}}e^{-c\sqrt{\log(t)}}\,dt \le e^{-c\frac{1}{2}\sqrt{\log(x)}}\int_{2}^{x^{\frac{1}{4}}}\,dt \ll x^{\frac{1}{4}}e^{-c\frac{1}{2}\sqrt{\log(x)}}. 
      \]
      All of these estimates together imply
      \begin{equation}\label{equ:prime_number_theorem_zeta_8}
        \int_{2}^{x}\frac{e^{-c\sqrt{\log(t)}}}{\log^{2}(t)}\,dt+\frac{xe^{-c\sqrt{\log(x)}}}{\log(x)} \ll xe^{-c\sqrt{\log(x)}},
      \end{equation}
      for some smaller $c$. Combining \cref{equ:prime_number_theorem_zeta_6,equ:prime_number_theorem_zeta_7,equ:prime_number_theorem_zeta_8} yields
      \begin{equation}\label{equ:prime_number_theorem_zeta_9}
        \pi_{1}(x) = \Li(x)+O\left(xe^{-c\sqrt{\log(x)}}\right),
      \end{equation}
      where the constant in \cref{equ:prime_number_theorem_zeta_7} has been absorbed into the error term. We now pass from $\pi_{1}(x)$ to $\pi(x)$. If $p$ is a prime such that $p^{m} < x$ for some $m \ge 1$, then $p < x^{\frac{1}{2}} < x^{\frac{1}{3}} < \cdots < x^{\frac{1}{m}}$. Therefore
      \begin{equation}\label{equ:prime_number_theorem_zeta_10}
        \pi_{1}(x) = \sum_{n \le x}\frac{\L(n)}{\log(n)} = \sum_{p^{m} \le x}\frac{\log(p)}{m\log(p)} = \pi(x)+\frac{1}{2}\pi(x^{\frac{1}{2}})+\cdots.
      \end{equation}
      Moreover, as $\pi(x^{\frac{1}{n}}) < x^{\frac{1}{n}}$ for any $n \ge 1$, we see that $\pi(x)-\pi_{1}(x) = O(x^{\frac{1}{2}})$. This estimate together with \cref{equ:prime_number_theorem_zeta_9,equ:prime_number_theorem_zeta_10} gives
      \[
        \pi(x) = \Li(x)+O\left(xe^{-c\sqrt{\log(x)}}\right),
      \]
      because $x^{\frac{1}{2}} \ll xe^{-c\sqrt{\log(x)}}$. This completes the proof.
    \end{proof}

    The proof of the logarithmic integral and classical versions of the prime number theorem are immediate:

    \begin{proof}[Proof of prime number theorem, logarithmic integral and classical versions]
      By the absolute error version of the prime number theorem,
      \[
        \pi(x) = \Li(x)\left(1+O\left(\frac{xe^{-c\sqrt{\log(x)}}}{\Li(x)}\right)\right).
      \]
      But we have shown $\Li(x) \sim \frac{x}{\log(x)}$ so that
      \[
        \frac{xe^{-c\sqrt{\log(x)}}}{\Li(x)} \sim \log(x)e^{-c\sqrt{\log(x)}} = o(1),
      \]
      where the equality holds since $\log(x) \ll_{\e} e^{-\e\sqrt{\log(x)}}$. The logarithm integral version follows. The classical version also holds using the asymptotic $\Li(x) \sim \frac{x}{\log(x)}$.
    \end{proof}
    
    In the proof of the logarithmic integral and classical versions of the prime number theorem, we saw that $xe^{-c\sqrt{\log(x)}} < \frac{x}{\log(x)}$ for sufficiently large $x$. Therefore the exact error $\pi(x)-\Li(x)$ grows slower than $\pi(x)-\frac{x}{\log{x}}$ for sufficiently large $x$. This means that $\Li(x)$ is a better numerical approximation to $\pi(x)$ than $\frac{x}{\log(x)}$. There is also the following result due to Hardy and Littlewood (see \cite{hardy1916contributions}) which gives us more information:

    \begin{proposition}\label{thm:Littlewood_Li_approximation_theorem}
      $\pi(x)-\Li(x)$ changes sign infinitely often as $x \to \infty$.
    \end{proposition}

    So in addition, \cref{thm:Littlewood_Li_approximation_theorem} implies that $\Li(x)$ never underestimates or overestimates $\pi(x)$ continuously. On the other hand, the exact error $\pi(x)-\frac{x}{\log(x)}$ is positive provided $x \ge 17$ (see \cite{rosser1962approximate}). It is also worthwhile to note that in 1901 Koch showed that the Riemann hypothesis improves the error term in the absolute error version of the prime number theorem (see \cite{von1901distribution}):

    \begin{proposition}
      For $x \ge 2$, we have
      \[
        \pi(x) = \Li(x)+O(\sqrt{x}\log(x)),
      \]
      provided the Riemann hypothesis for the Riemann zeta function holds.
    \end{proposition}
    \begin{proof}
      If $\rho$ is a nontrivial zero of $\z(s)$, the Riemann hypothesis implies $|x^{\rho}| = \sqrt{x}$. Therefore as in the proof of the absolute error version of the prime number theorem,
      \[
        \sum_{|\g| < T}\frac{x^{\rho}}{\rho} \ll \sqrt{x}\log^{2}(T),
      \]
      for $T > 2$ not coinciding with the ordinate of a nontrivial zero. Repeating the same argument with $T$ determined by
      \[
        T^{2} = x,
      \]
      gives
      \[
        \psi(x) = x+O(\sqrt{x}\log^{2}(x)),
      \]
      and then transferring to $\pi_{1}(x)$ and finally $\pi(x)$ gives
      \[
        \pi(x) = x+O(\sqrt{x}\log(x)).
      \]
    \end{proof}
  \section{The Siegel-Walfisz Theorem}
    Let $\chi$ be a Dirichlet character modulo $m > 1$. The function $\psi(x,\chi)$ is defined by
    \[
      \psi(x,\chi) = \sum_{n \le x}\chi(n)\L(n),
    \]
    for $x > 0$. This function plays the analogous role of $\psi(x)$ but for Dirichlet $L$-series. Accordingly, we will derive an explicit formula for $\psi(x,\chi)$ in a similar manner to that of $\psi(x)$. Because $\psi(x,\chi)$ is discontinuous when $x$ is a prime power, we also introduce a slightly modified function. Define $\psi_{0}(x,\chi)$ by
    \[
      \psi_{0}(x,\chi) = \begin{cases} \psi(x,\chi) & \text{if $x$ is not a prime power}, \\ \psi(x,\chi)-\frac{1}{2}\chi(x)\L(x) & \text{if $x$ is a prime power}. \end{cases}
    \]
    Thus $\psi_{0}(x,\chi)$ is $\psi(x,\chi)$ except that its value is halfway between the limit values when $x$ is a prime power. We will also need to define a particular constant that will come up. For a character $\chi$, define $b(\chi)$ to be $\frac{L'}{L}(0,\chi)$ if $\chi$ is odd and to be the constant term in the Laurent series of $\frac{L'}{L}(s,\chi)$ if $\chi$ is even (as in the even case $\frac{L'}{L}(s,\chi)$ has a pole at $s = 0$). The \textbf{explicit formula}\index{explicit formula} for $\psi(x,\chi)$ is the following:

    \begin{theorem}[Explicit formula for $\psi(x,\chi)$]
      Let $\chi$ be a primitive Dirichlet character of conductor $q > 1$. Then for $x \ge 2$,
      \[
        \psi_{0}(x,\chi) = -\sum_{\rho}\frac{x^{\rho}}{\rho}-b(\chi)+\tanh^{-1}(x^{-1}),
      \]
      if $\chi$ is odd, and
      \[
        \psi_{0}(x,\chi) = -\sum_{\rho}\frac{x^{\rho}}{\rho}-\log(x)-b(\chi)-\frac{1}{2}\log(1-x^{-2}),
      \]
      if $\chi$ is even, and where in both expressions, $\rho$ runs over the nontrivial zeros of $L(s,\chi)$ counted with multiplicity and ordered with respect to the size of the ordinate.
    \end{theorem}

    As for $\psi(x)$, a few comments are in order. Unlike the explicit formula for $\psi(x)$, there is no main term $x$ in the explicit formula for $\psi(x,\chi)$. This is because $L(s,\chi)$ does not have a pole at $s = 1$. The constant $b(\chi)$ can be expressed in terms of $B(\chi)$ (see \cite{davenport1980multiplicative} for a proof). Also, in the case $\chi$ is odd the Taylor series of the inverse hyperbolic tangent lets us write
    \[
      \tanh^{-1}(x^{-1}) = \sum_{m \ge 1}\frac{x^{-(2m-1)}}{2m-1} = -\sum_{m \ge 1}\frac{x^{-(2m-1)}}{-(2m-1)} = -\sum_{\w}\frac{x^{\w}}{\w},
    \]
    where $\w$ runs over the trivial zeros of $L(s,\chi)$. In the case $\chi$ is even, $\frac{1}{2}\log(1-x^{-2})$ accounts for the contribution of the trivial zeros just as for $\z(s)$. We will now prove the explicit formula for $\psi(x,\chi)$:

    \begin{proof}[Proof of the explicit formula for $\psi(x,\chi)$]
      By truncated Perron's formula applied to $-\frac{L'}{L}(s,\chi)$, we get
      \begin{equation}\label{equ:explicit_formula_Dirichlet_proof_1}
        \psi_{0}(x,\chi)-J(x,T,\chi) \ll x^{c}\sum_{\substack{n \ge 1 \\ n \neq x}}\frac{\chi(n)\L(n)}{n^{c}}\min\left(1,\frac{1}{T\left|\log\left(\frac{x}{n}\right)\right|}\right)+\d_{x}\chi(x)\L(x)\frac{c}{T},
      \end{equation}
      where
      \[
        J(x,T,\chi) = \frac{1}{2\pi i}\int_{c-iT}^{c+iT}-\frac{L'}{L}(s,\chi)x^{s}\,\frac{ds}{s},
      \]
      $c > 1$, and it is understood that $\d_{x} = 0$ unless $x$ is a prime power. Take $T > 2$ not coinciding with the ordinate of a nontrivial zero and let $c = 1+\frac{1}{\log(x^{2})}$ so that $x^{c} = \sqrt{e}x$. We will estimate the right-hand side of \cref{equ:explicit_formula_Dirichlet_proof_1}. First, we estimate the terms corresponding to $n$ such that $n$ is bounded away from $x$. So suppose $n \le \frac{3}{4}x$ or $n \ge \frac{5}{4}x$. For these $n$, $\log\left(\frac{x}{n}\right)$ is bounded away from zero so that their contribution is
      \begin{equation}\label{equ:explicit_formula_Dirichlet_proof_2}
        \ll \frac{x^{c}}{T}\sum_{n \ge 1}\frac{\chi(n)\L(n)}{n^{c}} \ll \frac{x^{c}}{T}\left(-\frac{L'}{L}(c,\chi)\right) \ll \frac{x\log(x)}{T},
      \end{equation}
      where the last estimate follows from \cref{lem:powerful_L-function_approximation_lemma} (iv) applied to $L(s,\chi)$ while discarding all of the terms in both sums and our choice of $c$ (in particular $\log(c) \ll \log(x)$). Now we estimate the terms $n$ close to $x$. So consider those $n$ for which $\frac{3}{4}x < n < x$ and let $x_{1}$ be the largest prime power less than $x$. We may also assume $\frac{3}{4}x < x_{1} < x$ since otherwise $\L(n) = 0$ and these terms do not contribute anything. Moreover, $\frac{x^{c}}{n^{c}} \ll 1$. For the term $n = x_{1}$, we have the estimate
      \[
        \log\left(\frac{x}{n}\right) = -\log\left(1-\frac{x-x_{1}}{x}\right) \ge \frac{x-x_{1}}{x},
      \]
      where we have obtained the inequality by using Taylor series of the logarithm truncated after the first term. The contribution of this term is
      \begin{equation}\label{equ:explicit_formula_Dirichlet_proof_3}
        \ll \chi(x_{1})\L(x_{1})\min\left(1,\frac{x}{T(x-x_{1})}\right) \ll \log(x)\min\left(1,\frac{x}{T(x-x_{1})}\right).
      \end{equation}
      For the other $n$, we write $n = x_{1}-v$, where $v$ is an integer satisfying $0 < v < \frac{1}{4}x$, so that
      \[
        \log\left(\frac{x}{n}\right) \ge \log\left(\frac{x_{1}}{n}\right) = -\log\left(1-\frac{v}{x_{1}}\right) \ge \frac{v}{x_{1}},
      \]
      where we have obtained the latter inequality by using Taylor series of the logarithm truncated after the first term. The contribution for these $n$ is
      \begin{equation}\label{equ:explicit_formula_Dirichlet_proof_4}
        \ll \sum_{0 < v < \frac{1}{4}x}\chi(x_{1}-v)\L(x_{1}-v)\frac{x_{1}}{Tv} \ll \frac{x}{T}\sum_{0 < v < \frac{1}{4}x}\frac{\L(x_{1}-v)}{v} \ll \frac{x\log(x)}{T}\sum_{0 < v < \frac{1}{4}x}\frac{1}{v} \ll \frac{x\log^{2}(x)}{T}.
      \end{equation}
      The contribution for those $n$ for which $x < n < \frac{5}{4}x$ is handled in exactly the same way with $x_{1}$ being the least prime power larger than $x$. Let $\<x\>$ be the distance between $x$ and the nearest prime power other than $x$ if $x$ itself is a prime power. Combining \cref{equ:explicit_formula_Dirichlet_proof_3,equ:explicit_formula_Dirichlet_proof_4} with our previous comment, the contribution for those $n$ with $\frac{3}{4}x < n < \frac{5}{4}x$ is
      \begin{equation}\label{equ:explicit_formula_Dirichlet_proof_5}
        \ll \frac{x\log^{2}(x)}{T}+\log(x)\min\left(1,\frac{x}{T\<x\>}\right).
      \end{equation}
      Putting \cref{equ:explicit_formula_Dirichlet_proof_2,equ:explicit_formula_Dirichlet_proof_5}, the error term in \cref{equ:explicit_formula_Dirichlet_proof_2} is absorbed by the second error term in \cref{equ:explicit_formula_Dirichlet_proof_5} and we obtain
      \begin{equation}\label{equ:explicit_formula_Dirichlet_proof_6}
        \psi_{0}(x,\chi)-J(x,T,\chi) \ll \frac{x\log^{2}(x)}{T}+\log(x)\min\left(1,\frac{x}{T\<x\>}\right).
      \end{equation}
      Now we estimate $J(x,T,\chi)$ by using the residue theorem. Let $U \ge 1$ be an integer with $U$ is even if $\chi$ is odd and odd if $\chi$ is even. Let $\W$ be the region enclosed by the contours $\eta_{1},\ldots,\eta_{4}$ in \cref{fig:explict_formula_Dirichlet_contour} and set $\eta = \sum_{1 \le i \le 4}\eta_{i}$ so that $\eta = \del \W$. We may write $J(x,T,\chi)$ as
      \[
        J(x,T,\chi) = \frac{1}{2\pi i}\int_{\eta_{1}}-\frac{L'}{L}(s,\chi)x^{s}\,\frac{ds}{s}.
      \]

      \begin{figure}[ht]
        \centering
        \begin{tikzpicture}[scale=2]
          \def\xmin{-3.5} \def\xmax{2}
          \def\ymin{-2} \def\ymax{2}
          \draw[thick] (\xmin,0) -- (\xmax,0);
          \draw[very thick] (0,\ymin) -- (0,\ymax);
          \draw[very thick] (1,\ymin) -- (1,\ymax);
          \draw[dashed] (0.5,\ymin) -- (0.5,\ymax);

          \draw[->-] (1.5,-1.5) -- (1.5,1.5);
          \draw[->-] (1.5,1.5) -- (-3,1.5);
          \draw[->-] (-3,1.5) -- (-3,-1.5);
          \draw[->-] (-3,-1.5) -- (1.5,-1.5);

          \node at (1.5,0) [below right] {\tiny{$\eta_{1}$}};
          \node at (-1,1.5) [above] {\tiny{$\eta_{2}$}};
          \node at (-3,0) [below left] {\tiny{$\eta_{3}$}};
          \node at (-1,-1.5) [below] {\tiny{$\eta_{4}$}};

          \node at (1.5,-1.5) [circle,fill,inner sep=1.5pt]{};
          \node at (1.5,1.5) [circle,fill,inner sep=1.5pt]{};
          \node at (0.5,1.5) [circle,fill,inner sep=1.5pt]{};
          \node at (-3,1.5) [circle,fill,inner sep=1.5pt]{};
          \node at (-3,-1.5) [circle,fill,inner sep=1.5pt]{};
          \node at (0.5,-1.5) [circle,fill,inner sep=1.5pt]{};

          \node at (1.5,-1.5) [below left] {\tiny{$c-iT$}};
          \node at (1.5,1.5) [above] {\tiny{$c+iT$}};
          \node at (0.5,1.5) [above left] {\tiny{$\frac{1}{2}+iT$}};
          \node at (-3,1.5) [above] {\tiny{$-U+iT$}};
          \node at (-3,-1.5) [below left] {\tiny{$-U-iT$}};
          \node at (0.5,-1.5) [below left] {\tiny{$\frac{1}{2}-iT$}};
        \end{tikzpicture}
        \caption{Contour for the explicit formula for $\psi(x,\chi)$}
        \label{fig:explict_formula_Dirichlet_contour}
      \end{figure}

      We now separate the cases that $\chi$ is even or odd. If $\chi$ is odd, then the residue theorem, the formula for the negative logarithmic derivative in \cref{prop:explicit_formula_log_derivative} applied to $L(s,\chi)$, and \cref{cor:logarithmic_derivative_of_gamma} together give
      \begin{equation}\label{equ:explicit_formula_Dirichlet_proof_7}
        J(x,T,\chi) = -\sum_{|\g| < T}\frac{x^{\rho}}{\rho}-b(\chi)-\sum_{0 < 2m+1 < U}\frac{x^{-(2m-1)}}{-(2m-1)}+\frac{1}{2\pi i}\int_{\eta_{2}+\eta_{3}+\eta_{4}}-\frac{L'}{L}(s,\chi)x^{s}\,\frac{ds}{s},
      \end{equation}
      where $\rho = \b+i\g$ is a nontrivial zero of $L(s,\chi)$. If $\chi$ is even, then there is a minor complication because $L(s,\chi)$ has a simple zero at $s = 0$ and so the integrand has a double pole at $s = 0$. To find the residue, the Laurent series are
      \[
        \frac{L'}{L}(s,\chi) = \frac{1}{s}+b(\chi)+\cdots \quad \text{and} \quad \frac{x^{s}}{s} = \frac{1}{s}+\log(x)+\cdots,
      \]
      and thus the residue of the integrand is $-(\log(x)+b(\chi))$. Now as before, the residue theorem, the formula for the negative logarithmic derivative in \cref{prop:explicit_formula_log_derivative} applied to $L(s,\chi)$, and \cref{cor:logarithmic_derivative_of_gamma} together give
      \begin{equation}\label{equ:explicit_formula_Dirichlet_proof_7'}
        J(x,T,\chi) = -\sum_{|\g| < T}\frac{x^{\rho}}{\rho}-\log(x)-b(\chi)-\sum_{0 < 2m < U}\frac{x^{-2m}}{-2m}+\frac{1}{2\pi i}\int_{\eta_{2}+\eta_{3}+\eta_{4}}-\frac{L'}{L}(s,\chi)x^{s}\,\frac{ds}{s},
      \end{equation}
      where $\rho = \b+i\g$ is a nontrivial zero of $L(s,\chi)$. We now estimate the remaining integrals in \cref{equ:explicit_formula_Dirichlet_proof_7,equ:explicit_formula_Dirichlet_proof_7'}. For this estimate, the parity of $\chi$ does not matter so we make no such restriction. By \cref{lem:powerful_L-function_approximation_lemma} (ii) applied to $L(s,\chi)$, the number of nontrivial zeros satisfying $|\g-T| < 1$ is $O(\log(qT))$. Among the ordinates of these nontrivial zeros, there must be a gap of size $\gg \frac{1}{\log(qT)}$. Upon varying $T$ by a bounded amount (we are varying in the interval $[T-1,T+1]$) so that it belongs to this gap, we can additionally ensure
      \[
        \g-T \gg \frac{1}{\log(qT)},
      \]
      for all the nontrivial zeros of $L(s,\chi)$. To estimate part of the horizontal integrals over $\eta_{2}$ and $\eta_{4}$, \cref{lem:powerful_L-function_approximation_lemma} (iv) applied to $L(s,\chi)$ gives
      \[
        \frac{L'}{L}(s,\chi) = \sum_{|\g-T| < 1}\frac{1}{s-\rho}+O(\log(qT)),
      \]
      on the parts of these segments with $-1 \le \s \le 2$. Our choice of $T$ implies $|s-\rho| \ge |\g-T| \gg \frac{1}{\log(qT)}$ so that each term in the sum is $O(\log(qT))$. As there are at most $O(\log(qT))$ such terms by \cref{lem:powerful_L-function_approximation_lemma} (ii) applied to $L(s,\chi)$, we have
      \[
        \frac{L'}{L}(s,\chi) = O(\log^{2}(qT)),
      \]
      on the parts of these segments with $-1 \le \s \le 2$. It follows that the parts of the horizontal integrals over $\eta_{2}$ and $\eta_{4}$ with  $-1 \le \s \le c$ (recall $c < 2$) contribute
      \begin{equation}\label{equ:explicit_formula_Dirichlet_proof_8}
        \ll \frac{\log^{2}(qT)}{T}\int_{-1}^{c}x^{\s}\,d\s \ll \frac{\log^{2}(qT)}{T}\int_{-\infty}^{c}x^{\s}\,d\s \ll \frac{x\log^{2}(qT)}{T\log(x)}.
      \end{equation}
      where in the last estimate we have used the choice of $c$. To estimate the remainder of the horizontal integrals, we require a bound for $\frac{L'}{L}(s,\chi)$ when $\s < -1$ and away from the trivial zeros. To find such a bound, write the functional equation for $L(s,\chi)$ in the form
      \[
        L(s,\chi) = \frac{\e_{\chi}}{i^{\mf{a}}}q^{\frac{1}{2}-s}\pi^{s-1}\frac{\G\left(\frac{(1-s)+\mf{a}}{2}\right)}{\G\left(\frac{s+\mf{a}}{2}\right)}L(1-s,\chi),
      \]
      and take the logarithmic derivative to get
      \[
        \frac{L'}{L}(s,\chi) = -\log(q)+\log(\pi)+\frac{1}{2}\frac{\G'}{\G}\left(\frac{(1-s)+\mf{a}}{2}\right)-\frac{1}{2}\frac{\G'}{\G}\left(\frac{s+\mf{a}}{2}\right)+\frac{L'}{L}(1-s,\chi).
      \]
      Now let $s$ be such that $\s < -1$ and suppose $s$ is distance $\frac{1}{2}$ away from the trivial zeros. We will estimate every term on the right-hand side of the identity above. The second term is constant and the last term is bounded since it is an absolutely convergent Dirichlet series. For the digamma terms, $s$ is away from the trivial zeros so \cref{equ:approximtion_for_digamma} implies $\frac{1}{2}\frac{\G'}{\G}\left(\frac{(1-s)+\mf{a}}{2}\right) = O(\log{|(1-s)+\mf{a}|})$ and $\frac{1}{2}\frac{\G'}{\G}\left(\frac{s+\mf{a}}{2}\right) = O(\log{|s+\mf{a}|})$. However, as $\s < -1$ and $s$ is away from the trivial zeros, $s+\mf{a}$ and $(1-s)+\mf{a}$ are bounded away from zero so that $\frac{1}{2}\frac{\G'}{\G}\left(\frac{(1-s)+\mf{a}}{2}\right) = O(\log{|s|})$ and $\frac{1}{2}\frac{\G'}{\G}\left(\frac{s+\mf{a}}{2}\right) = O(\log{|s|})$. Putting these estimates together with the first term yields
      \begin{equation}\label{equ:explicit_formula_Dirichlet_proof_9}
        \frac{L'}{L}(s,\chi) \ll \log(q|s|),
      \end{equation}
      for $\s < -1$. Using \cref{equ:explicit_formula_Dirichlet_proof_9}, the parts of the horizontal integrals over $\eta_{2}$ and $\eta_{4}$ with $-U \le \s \le -1$ contribute
      \begin{equation}\label{equ:explicit_formula_Dirichlet_proof_10}
        \ll \frac{\log(qT)}{T}\int_{-U}^{-1}x^{\s}\,d\s \ll \frac{\log(qT)}{Tx\log(x)}.
      \end{equation}
      Combining \cref{equ:explicit_formula_Dirichlet_proof_8,equ:explicit_formula_Dirichlet_proof_10} gives
      \begin{equation}\label{equ:explicit_formula_Dirichlet_proof_11}
        \frac{1}{2\pi i}\int_{\eta_{2}+\eta_{4}}-\frac{L'}{L}(s)x^{s}\,\frac{ds}{s} \ll \frac{x\log^{2}(qT)}{T\log(x)}+\frac{\log(qT)}{Tx\log(x)} \ll \frac{x\log^{2}(qT)}{T\log(x)}.
      \end{equation}
      To estimate the vertical integral, we use \cref{equ:explicit_formula_Dirichlet_proof_9} again to conclude that
      \begin{equation}\label{equ:explicit_formula_Dirichlet_proof_12}
        \frac{1}{2\pi i}\int_{\eta_{3}}-\frac{L'}{L}(s,\chi)x^{s}\,\frac{ds}{s} \ll \frac{\log(qU)}{U}\int_{-T}^{T}x^{-U}\,dt \ll \frac{T\log(qU)}{Ux^{U}}.
      \end{equation}
      Combining \cref{equ:explicit_formula_Dirichlet_proof_7,equ:explicit_formula_Dirichlet_proof_11,equ:explicit_formula_Dirichlet_proof_12} and taking the limit as $U \to \infty$, the error term in \cref{equ:explicit_formula_Dirichlet_proof_12} vanishes and the sum over $m$ in \cref{equ:explicit_formula_Dirichlet_proof_7,equ:explicit_formula_Dirichlet_proof_7'} evaluates to $-\tanh^{-1}(x^{-1})$ or $\frac{1}{2}\log(1-x^{-2})$ respectively (as we have already mentioned) giving
      \begin{equation}\label{equ:explicit_formula_Dirichlet_proof_13}
        J(x,T,\chi) = -\sum_{|\g| < T}\frac{x^{\rho}}{\rho}-b(\chi)+\tanh^{-1}(x^{-1})+\frac{x\log^{2}(qT)}{T\log(x)},
      \end{equation}
      if $\chi$ is odd, and
      \begin{equation}\label{equ:explicit_formula_Dirichlet_proof_13'}
        J(x,T,\chi) = -\sum_{|\g| < T}\frac{x^{\rho}}{\rho}-\log(x)-b(\chi)-\frac{1}{2}\log(1-x^{-2})+\frac{x\log^{2}(qT)}{T\log(x)},
      \end{equation}
      if $\chi$ is even.
      Substituting \cref{equ:explicit_formula_Dirichlet_proof_13,equ:explicit_formula_Dirichlet_proof_13'} into \cref{equ:explicit_formula_Dirichlet_proof_6} in the respective cases, we obtain
      \begin{equation}\label{equ:explicit_formula_Dirichlet_proof_14}
        \psi_{0}(x,\chi) = -\sum_{|\g| < T}\frac{x^{\rho}}{\rho}-b(\chi)+\tanh^{-1}(x^{-1})+\frac{x\log^{2}(xqT)}{T}+\log(x)\min\left(1,\frac{x}{T\<x\>}\right),
      \end{equation}
      if $\chi$ is odd, and
      \begin{equation}\label{equ:explicit_formula_Dirichlet_proof_14'}
        \psi_{0}(x,\chi) = -\sum_{|\g| < T}\frac{x^{\rho}}{\rho}-\log(x)-b(\chi)-\frac{1}{2}\log(1-x^{-2})+\frac{x\log^{2}(xqT)}{T}+\log(x)\min\left(1,\frac{x}{T\<x\>}\right),
      \end{equation}
      if $\chi$ is even, and where the second to last term on the right-hand side in both equations are obtained by combining the error term in \cref{equ:explicit_formula_Dirichlet_proof_11} with the first error term in \cref{equ:explicit_formula_Dirichlet_proof_6}. The theorem follows by taking the limit as $T \to \infty$.
    \end{proof}

    As was the case for $\psi(x)$, the convergence of the right-hand side in the explicit formula for $\psi(x,\chi)$ is uniform in any interval not containing a prime power since $\psi(x,\chi)$ is continuous there. Moreover, there is an approximate formula for $\psi(x,\chi)$ as a corollary which holds for all Dirichlet characters:


    \begin{corollary}\label{cor:explicit_formula_Dirichlet_corollary}
      Let $\chi$ be a Dirichlet character modulo $m > 1$. Then for $2 \le T \le x$,
      \[
        \psi_{0}(x,\chi) = -\frac{x^{\b_{\chi}}}{\b_{\chi}}-\psum_{|\g| < T}\frac{x^{\rho}}{\rho}+R(x,T,\chi),
      \]
      where $\rho$ runs over the nontrivial zeros of $L(s,\chi)$ counted with multiplicity and ordered with respect to the size of the ordinate, the $'$ in the sum indicates that we are excluding the terms corresponding to real zeros, the term corresponding to a Siegel zero $\b_{\chi}$ is present only if $L(s,\chi)$ admits a Siegel zero, and
      \[
        R(x,T,\chi) \ll \frac{x\log^{2}(xmT)}{T}+x^{1-\b_{\chi}}\log(x)+\log(x)\min\left(1,\frac{x}{T\<x\>}\right),
      \]
      where $\<x\>$ is the distance between $x$ and the nearest prime power other than $x$ if $x$ itself is a prime power and again the term corresponding to a Siegel zero $\b_{\chi}$ is present only if $L(s,\chi)$ admits a Siegel zero. Moreover, if $x$ is an integer, we have the simplified estimate
      \[
        R(x,T,\chi) \ll \frac{x\log^{2}(xmT)}{T}+x^{1-\b_{\chi}}\log(x).
      \]
    \end{corollary}
    \begin{proof}
      We first reduced to the case that $\chi$ is primitive. Let $\wtilde{\chi}$ be the primitive character inducing $\chi$ and denote its conductor by $q$. We estimate
      \begin{equation}\label{equ:difference_between_Dirichlet-Chebyshevs}
        \begin{aligned}
          |\psi_{0}(x,\chi)-\psi_{0}(x,\wtilde{\chi})| &\le \sum_{\substack{n \le x \\ (n,m) > 1}}\L(n) \\
          &= \sum_{p \mid m}\sum_{\substack{v \ge 1 \\ p^{v} \le x}}\log(p) \\
          &\ll \log(x)\sum_{p \mid m}\log(p) \\
          &\ll \log(x)\log(m) \\
          &\ll \log^{2}(xm),
        \end{aligned}
      \end{equation}
      where the third line holds because $p^{v} \le x$ implies $v \le \frac{\log(x)}{\log(p)}$ so that there are $O(\log(x))$ many terms in the inner sum and in the last line we have used the simple estimates $\log(x) \ll \log(xm)$ and $\log(m) \ll \log(xm)$. Therefore the difference between $\psi_{0}(x,\chi)$ and $\psi_{0}(x,\wtilde{\chi})$ is $O(\log^{2}(xm))$. Now for $2 \le T \le x$, we have $\log^{2}(xm) \ll \frac{x\log^{2}(xmT)}{T}$, which implies that the difference is absorbed into $O\left(\frac{x\log^{2}(xmT)}{T}\right)$ which is the first term in the error for $R(x,T,\chi)$. As $R(x,T,\wtilde{\chi}) \ll R(x,T,\chi)$ because $q \le m$, and there are finitely many nontrivial zeros of $L(s,\chi)$ that are not nontrivial zeros of $L(s,\wtilde{\chi})$ (all occurring on the line $\s = 0$), it suffices to assume that $\chi$ is primitive. The claim will follow from estimating terms in \cref{equ:explicit_formula_Dirichlet_proof_14,equ:explicit_formula_Dirichlet_proof_14'}. We will estimate the constant $b(\chi)$ first. The formula for the negative logarithmic derivative in \cref{prop:explicit_formula_log_derivative} applied to $L(s,\chi)$ at $s = 2$ implies
      \begin{equation}\label{equ:explicit_formula_Dirichlet_corollary_1}
        0 = -\frac{L'}{L}(2,\chi)-\frac{1}{2}\log(q)+\frac{1}{2}\log(\pi)-\frac{1}{2}\frac{\G'}{\G}\left(\frac{2+\mf{a}}{2}\right)+B(\chi)+\sum_{\rho}\left(\frac{1}{2-\rho}+\frac{1}{\rho}\right).
      \end{equation}
      Adding \cref{equ:explicit_formula_Dirichlet_corollary_1} to the formula for the negative logarithmic derivative in \cref{prop:explicit_formula_log_derivative} applied to $L(s,\chi)$ gives
      \[
        -\frac{L'}{L}(s,\chi) = -\frac{L'}{L}(2,\chi)-\frac{1}{2}\frac{\G'}{\G}\left(\frac{2+\mf{a}}{2}\right)+\frac{1}{2}\frac{\G'}{\G}\left(\frac{s+\mf{a}}{2}\right)-\sum_{\rho}\left(\frac{1}{s-\rho}-\frac{1}{2-\rho}\right).
      \]
      As the first two terms are constant, we obtain a weaker estimate
      \[
        -\frac{L'}{L}(s,\chi) = \frac{1}{2}\frac{\G'}{\G}\left(\frac{s+\mf{a}}{2}\right)-\sum_{\rho}\left(\frac{1}{s-\rho}-\frac{1}{2-\rho}\right)+O(1).
      \]
      If $\chi$ is odd, we set $s = 0$ since $\frac{1}{2}\frac{\G'}{\G}\left(\frac{s+\mf{a}}{2}\right)$ does not have a pole there. If $\chi$ is even, we compare constant terms in the Laurent series using the Laurent series
      \[
        \frac{L'}{L}(s,\chi) = \frac{1}{s}+b(\chi)+\cdots \quad \text{and} \quad \frac{1}{2}\frac{\G'}{\G}\left(\frac{s+\mf{a}}{2}\right) = \frac{1}{s}+b+\cdots,
      \]
      for some constant $b$. In either case, our previous estimate gives
      \begin{equation}\label{equ:explicit_formula_Dirichlet_corollary_2}
        b(\chi) = -\sum_{\rho}\left(\frac{1}{\rho}+\frac{1}{2-\rho}\right)+O(1).
      \end{equation}
      Let $\rho = \b+i\g$. For all the terms with $|\g| > 1$, we estimate
      \begin{equation}\label{equ:explicit_formula_Dirichlet_corollary_3}
        \sum_{|\g| > 1}\left(\frac{1}{\rho}+\frac{1}{2-\rho}\right) \le \sum_{|\g| > 1}\left|\frac{1}{\rho}+\frac{1}{2-\rho}\right| = \sum_{|\g| > 1}\frac{2}{|\rho(2-\rho)|} \ll \sum_{|\g| > 1}\frac{1}{|\rho|^{2}} \ll \log(q),
      \end{equation}
      where the second to last estimate holds since $2-\rho \gg \rho$ because $\b$ is bounded and the last estimate holds by the convergent sum in \cref{prop:explicit_formula_log_derivative}and \cref{lem:powerful_L-function_approximation_lemma} (ii) both applied to $L(s,\chi)$ (recall that the tail of a convergent series is bounded). For the terms corresponding to $2-\rho$ with $|\g| \le 1$, we have
      \begin{equation}\label{equ:explicit_formula_Dirichlet_corollary_4}
        \sum_{|\g| \le 1}\frac{1}{2-\rho} \le \sum_{|\g| \le 1}\frac{1}{|2-\rho|} \ll \log(q),
      \end{equation}
      where the last estimate holds by using \cref{lem:powerful_L-function_approximation_lemma} (ii) applied to $L(s,\chi)$ and because the nontrivial zeros are bounded away from $2$. Combining \cref{equ:explicit_formula_Dirichlet_corollary_2,equ:explicit_formula_Dirichlet_corollary_3,equ:explicit_formula_Dirichlet_corollary_4} yields
      \begin{equation}\label{equ:explicit_formula_Dirichlet_corollary_5}
        b(\chi) = -\sum_{|\g| \le 1}\frac{1}{\rho}+O(\log(q)).
      \end{equation}
      Inserting \cref{equ:explicit_formula_Dirichlet_corollary_5} into \cref{equ:explicit_formula_Dirichlet_proof_14,equ:explicit_formula_Dirichlet_proof_14'} and noting that $\tanh^{-1}(x^{-1})$ and $\frac{1}{2}\log(1-x^{-2})$ are both bounded for $x \ge 2$ gives
      \begin{equation}\label{equ:explicit_formula_Dirichlet_corollary_6}
        \psi_{0}(x,\chi) = -\sum_{|\g| < T}\frac{x^{\rho}}{\rho}+\sum_{|\g| \le 1}\frac{1}{\rho}+R'(x,T,\chi),
      \end{equation}
      where
      \[
        R'(x,T,\chi) \ll \frac{x\log^{2}(xqT)}{T}+\log(x)\min\left(1,\frac{x}{T\<x\>}\right),
      \]
      and we have absorbed the error in \cref{equ:explicit_formula_Dirichlet_corollary_5} into $O\left(\frac{x\log^{2}(xqT)}{T}\right)$ because $2 \le T \le x$. Extracting the terms corresponding to the possible real zeros $\b_{\chi}$ and $1-\b_{\chi}$ in \cref{equ:explicit_formula_Dirichlet_corollary_6}, we obtain
      \begin{equation}\label{equ:explicit_formula_Dirichlet_corollary_7}
        \psi_{0}(x,\chi) = -\psum_{|\g| < T}\frac{x^{\rho}}{\rho}+\psum_{|\g| \le 1}\frac{1}{\rho}-\frac{x^{\b_{\chi}}-1}{\b_{\chi}}-\frac{x^{1-\b_{\chi}}-1}{1-\b_{\chi}}+R'(x,T,\chi).
      \end{equation}
      We now estimate some of the terms in \cref{equ:explicit_formula_Dirichlet_corollary_7}. For the second sum, we have $\rho \gg \frac{1}{\log(q)}$ since $\g$ is bounded and $\b < 1-\frac{c}{\log(q|\g|)}$ for some positive constant $c$ by \cref{thm:improved_zero-free_region_Dirichlet}. Thus
      \begin{equation}\label{equ:explicit_formula_Dirichlet_corollary_8}
        \psum_{|\g| \le 1}\frac{1}{\rho} \ll \psum_{|\g| \le 1}\log(q) \ll \log^{2}(q),
      \end{equation}
      where the last estimate holds by \cref{lem:powerful_L-function_approximation_lemma} (ii) applied to $L(s,\chi)$. Similarly,
      \begin{equation}\label{equ:explicit_formula_Dirichlet_corollary_9}
        \frac{x^{1-\b_{\chi}}-1}{1-\b_{\chi}} \ll x^{1-\b_{\chi}}\log(x),
      \end{equation}
      because $\rho \gg \frac{1}{\log(q)}$ implies $1-\b_{\chi} \gg \frac{1}{\log(q)} \gg \frac{1}{\log(x)}$. Substituting \cref{equ:explicit_formula_Dirichlet_corollary_8,equ:explicit_formula_Dirichlet_corollary_9} into \cref{equ:explicit_formula_Dirichlet_corollary_7} and noting that $\b_{\chi}$ is bounded yields
      \begin{equation}\label{equ:explicit_formula_Dirichlet_corollary_10}
        \psi_{0}(x,\chi) = -\frac{x^{\b_{\chi}}}{\b_{\chi}}-\psum_{|\g| < T}\frac{x^{\rho}}{\rho}+R(x,T,\chi),
      \end{equation}
      where
      \[
        R(x,T,\chi) \ll \frac{x\log^{2}(xqT)}{T}+x^{1-\b_{\chi}}\log(x)+\log(x)\min\left(1,\frac{x}{T\<x\>}\right)
      \]
      and we have absorbed the error in \cref{equ:explicit_formula_Dirichlet_corollary_8} into $O\left(\frac{x\log^{2}(xqT)}{T}\right)$ because because $2 \le T \le x$. If $x$ is an integer, then $\<x\> \ge 1$ so that $\log(x)\min\left(1,\frac{x}{T\<x\>}\right) \le \frac{x\log(x)}{T}$ and this term can be absorbed into $O\left(\frac{x\log^{2}(xqT)}{T}\right)$.
    \end{proof}
    
    We can now discuss the Siegel–Walfisz theorem. Let $a$ and $m$ be positive integers with $m > 1$ and $(a,m) = 1$. The \textbf{prime counting function}\index{prime counting function} $\pi(x:a,m)$ is defined by
    \[
      \pi(x;a,m) = \sum_{\substack{p \le x \\ p \equiv a \tmod{m}}}1,
    \]
    for $x > 0$. Equivalently, $\pi(x;a,m)$ counts the number of primes that no larger than $x$ and are equivalent to $a$ modulo $m$. This is the analog of $\pi(x)$ that is naturally associated to Dirichlet characters modulo $m$. Accordingly, there are asymptotics for $\pi(x;a,m)$ analogous to those for $\pi(x)$. To prove them, we will require an auxiliary function. The function $\psi(x;a,m)$ is defined by
    \[
      \psi(x;a,m) = \sum_{\substack{n \le x \\ n \equiv a \tmod{m}}}\L(n),
    \]
    for $x \ge 1$. This is just $\psi(x)$ restricted to only those terms equivalent to $a$ modulo $m$. As for the asymptotic, the classical \textbf{Siegel–Walfisz theorem}\index{Siegel–Walfisz theorem} is the first of them and the precise statement is the following:

    \begin{theorem}[Siegel–Walfisz theorem, classical version]
      Let $a$ and $m$ be positive integers with $m > 1$, $(a,m) = 1$, and let $N \ge 1$. For $x \ge 2$,
      \[
        \pi(x;a,m) \sim \frac{x}{\vphi(m)\log(x)},
      \]
      provided $m \le \log^{N}(x)$.
    \end{theorem}
    
    The logarithmic integral version of the \textbf{Siegel–Walfisz theorem}\index{Siegel–Walfisz theorem} is equivalent and sometimes more useful:

    \begin{theorem}[Siegel–Walfisz theorem, logarithmic integral version]
      Let $a$ and $m$ be positive integers with $m > 1$, $(a,m) = 1$, and let $N \ge 1$. For $x \ge 2$,
      \[
        \pi(x;a,m) \sim \frac{\Li(x)}{\vphi(m)},
      \]
      provided $m \le \log^{N}(x)$.
    \end{theorem}
    
    We will prove the absolute error version of the \textbf{Siegel–Walfisz theorem}\index{Siegel–Walfisz theorem} which is slightly stronger as it bounds the absolute error between $\pi(x;a,m)$ and $\frac{\Li(x)}{\vphi(m)}$:

    \begin{theorem}[Siegel–Walfisz theorem, absolute error version]
      Let $a$ and $m$ be positive integers with $m > 1$, $(a,m) = 1$, and let $N \ge 1$. For $x \ge 2$, there exists a positive constant $c$ such that
      \[
        \pi(x;a,m) = \frac{\Li(x)}{\vphi(m)}+O\left(xe^{-c\sqrt{\log(x)}}\right),
      \]
      provided $m \le \log^{N}(x)$.
    \end{theorem}
    \begin{proof}
      It suffices to assume $x$ is an integer, because $\pi(x;a,m)$ can only change value at integers and the other functions in the statement are increasing. We being with the identity
      \begin{equation}\label{equ:Siegel-Walfisz_1}
        \psi(x;a,m) = \frac{1}{\vphi(m)}\sum_{\chi \tmod{m}}\cchi(a)\psi(x,\chi),
      \end{equation}
      which holds by the orthogonality relations (\cref{prop:Dirichlet_orthogonality_relations} (ii)). Let $\wtilde{\chi}$ be the primitive character inducing $\chi$. Then \cref{equ:difference_between_Dirichlet-Chebyshevs} implies
      \begin{equation}\label{equ:Siegel-Walfisz_2}
        \psi(x,\chi) = \psi(x,\wtilde{\chi})+O(\log^{2}(xm)).
      \end{equation}
      When $\chi = \chi_{m,0}$ we have $\psi(x,\wtilde{\chi}) = \psi(x)$, and as $\psi(x,\chi) \sim \psi_{0}(x,\chi)$, substituting \cref{equ:prime_number_theorem_zeta_1} into \cref{equ:Siegel-Walfisz_2} gives
      \begin{equation}\label{equ:Siegel-Walfisz_3}
        \psi(x,\chi_{m,0}) = \psi(x)+O\left(xe^{-c\sqrt{\log(x)}}+\log^{2}(xm)\right),
      \end{equation}
      for some positive constant $c$. Upon combining \cref{equ:Siegel-Walfisz_1,equ:Siegel-Walfisz_3}, we obtain
      \begin{equation}\label{equ:Siegel-Walfisz_4}
        \psi(x;a,m) = \frac{x}{\vphi(m)}+\frac{1}{\vphi(m)}\sum_{\substack{\chi \tmod{m} \\ \chi \neq \chi_{m,0}}}\cchi(a)\psi(x,\chi)+O\left(\frac{1}{\vphi(m)}\left(xe^{-c\sqrt{\log(x)}}+\log^{2}(xm)\right)\right).
      \end{equation}
      We now prove
      \begin{equation}\label{equ:Siegel-Walfisz_5}
        \psi(x,\chi) = -\frac{x^{\b_{\chi}}}{\b_{\chi}}+O\left(xe^{-c\sqrt{\log(x)}}\right),
      \end{equation}
      for some potentially different constant $c$, where $\chi \neq \chi_{m,0}$, and the term corresponding to $\b_{\chi}$ appears if and only if $L(s,\chi)$ admits a Siegel zero. To accomplish this, we estimate the sum over the nontrivial zeros of $L(s,\chi)$ in \cref{cor:explicit_formula_Dirichlet_corollary}. So fix a non-principal $\chi$ modulo $m$ and let $\wtilde{\chi}$ be the primitive character inducing $\chi$. Let $2 \le T \le x$ not coinciding with the ordinate of a nontrivial zero and let $\rho = \b+i\g$ be a complex nontrivial zero of $L(s,\chi)$ with $|\g| < T$. By \cref{thm:improved_zero-free_region_Dirichlet}, all of the zeros $\rho$ satisfy $\b < 1-\frac{c}{\log(mT)}$ for some possibly smaller $c$ (recall that the nontrivial zeros of $L(s,\chi)$ that are not nontrivial zeros of $L(s,\wtilde{\chi})$ lie on the line $\s = 0$). It follows that
      \begin{equation}\label{equ:Siegel-Walfisz_6}
        |x^{\rho}| = x^{\b} < x^{1-\frac{c}{\log(mT)}} = xe^{-c\frac{\log(x)}{\log(mT)}}.
      \end{equation}
      As $|\rho| > |\g|$, for those terms with $|\g| > 1$ (unlike the Riemann zeta function we do not have a positive lower bound for the first ordinate $\g_{1}$ of a nontrivial zero that is not real since Siegel zeros may exist), applying integration by parts gives
      \begin{equation}\label{equ:Siegel-Walfisz_7}
        \sum_{1 < |\g| < T}\frac{1}{\rho} \ll \sum_{1 < |\g| < T}\frac{1}{\g} \ll \int_{1}^{T}\frac{dN(t,\wtilde{\chi})}{t} = \frac{N(T,\wtilde{\chi})}{T}+\int_{1}^{T}\frac{N(t,\wtilde{\chi})}{t^{2}}\,dt \ll \log^{2}(mT) \ll \log^{2}(xm),
      \end{equation}
      where in the second to last estimate we have used that $N(t,\wtilde{\chi}) \ll t\log(qt) \ll t\log(mt)$ which follows from \cref{cor:zero_density} applied to $L(s,\wtilde{\chi})$ and that $q \le m$ and in the last estimate we have used the bound $T \le x$. For the remaining terms with $|\g| \le 1$ (that do not correspond to real nontrivial zeros), \cref{equ:explicit_formula_Dirichlet_corollary_8} along with $q \le m$ gives
      \begin{equation}\label{equ:Siegel-Walfisz_8}
        \psum_{|\g| \le 1}\frac{1}{\rho} \ll \log^{2}(m).
      \end{equation}
      Combining \cref{equ:Siegel-Walfisz_5,equ:Siegel-Walfisz_6,equ:Siegel-Walfisz_7,equ:Siegel-Walfisz_8} yields
      \begin{equation}\label{equ:Siegel-Walfisz_9}
        \psum_{|\g| < T}\frac{x^{\rho}}{\rho} \ll x\log^{2}(xm)e^{-c\frac{\log(x)}{\log(mT)}},
      \end{equation}
      where the $'$ in the sum indicates that we are excluding the terms corresponding to real zeros and the error in \cref{equ:Siegel-Walfisz_8} has been absorbed by that in \cref{equ:Siegel-Walfisz_7}. As $\psi(x,\chi) \sim \psi_{0}(x,\chi)$ and $x$ is an integer, inserting \cref{equ:Siegel-Walfisz_9} into \cref{cor:explicit_formula_Dirichlet_corollary} results in
      \begin{equation}\label{equ:Siegel-Walfisz_10}
        \psi(x,\chi)+\frac{x^{\b_{\chi}}}{\b_{\chi}} \ll x\log^{2}(xm)e^{-c\frac{\log(x)}{\log(mT)}}+\frac{x\log^{2}(xmT)}{T}+x^{1-\b_{\chi}}\log(x),
      \end{equation}
      where the terms corresponding to real zeros are present if and only if $L(s,\chi)$ admits a Siegel zero. We now let $T$ be determined by $T = x$ for $2 \le x < 3$ and
      \[
        \log^{2}(T) = \log(x),
      \]
      or equivalently,
      \[
        T = e^{\sqrt{\log(x)}},
      \]
      for $x \ge 3$. With this choice of $T$ (note that if $x \ge 2$ then $2 \le T \le x$) and that $m \ll \log^{N}(x)$, we can estimate \cref{equ:Siegel-Walfisz_10} as follows:
      \begin{align*}
        \psi(x)+\frac{x^{\b_{\chi}}}{x} &\ll x(\log^{2}(x)+\log^{2}(m))e^{-c\frac{\log(x)}{\log(m)+\sqrt{\log(x)}}}+x(\log^{2}(x)+\log^{2}(m)+\log(x))e^{-\sqrt{\log(x)}}+x^{1-\b_{\chi}}\log(x) \\
        &\ll x(\log^{2}(x)+\log^{2}(m))e^{-c\frac{\log(x)}{\log(m)+\sqrt{\log(x)}}}+x(\log^{2}(x)+\log^{2}(m)+\log(x))e^{-\sqrt{\log(x)}}+x^{1-\b_{\chi}}\log(x) \\
        &\ll x(\log^{2}(x)+\log^{2}(m))e^{-c\sqrt{\log(x)}}+x(\log^{2}(x)+\log^{2}(m))e^{-\sqrt{\log(x)}}+x^{1-\b_{\chi}}\log(x) \\
        &\ll x\log^{2}(x)e^{-c\sqrt{\log(x)}}+x\log^{2}(x)e^{-\sqrt{\log(x)}}+x^{1-\b_{\chi}}\log(x) \\
        &\ll x\log^{2}(x)e^{-\min(1,c)\frac{\sqrt{\log(x)}}{\log(m)}},
      \end{align*}
      where in the last estimate we have used that $x^{1-\b_{\chi}} \le x^{\frac{1}{2}}$ because $\b_{\chi} \ge \frac{1}{2}$. As $\log(x) \ll_{\e} e^{-\e\sqrt{\log(x)}}$, we conclude that
      \[
        \psi(x)+\frac{x^{\b_{\chi}}}{x} \ll xe^{-c\sqrt{\log(x)}},
      \]
      for some smaller $c$ with $c < 1$. This is equivalent to \cref{equ:Siegel-Walfisz_5}. Substituting \cref{equ:Siegel-Walfisz_5} into \cref{equ:Siegel-Walfisz_4} and noting that there is at most one Siegel zero for characters modulo $m$ by \cref{prop:at_most_one_Siegel_zero_per_modulus}, we arrive at
      \begin{equation}\label{equ:Siegel-Walfisz_11}
        \psi(x;a,m) = \frac{x}{\vphi(m)}-\frac{\cchi_{1}(a)x^{\b_{\chi}}}{\vphi(m)\b_{\chi}}+O\left(xe^{-c\sqrt{\log(x)}}\right),
      \end{equation}
      where $\chi_{1}$ is the single quadratic character modulo $m$ such that $L(s,\chi_{1})$ admits a Siegel zero if it exists and we have absorbed the second term in the error in \cref{equ:Siegel-Walfisz_4} into the first since $\log(x) \ll_{\e} e^{-\e\sqrt{\log(x)}}$ and $m \ll \log^{N}(x)$. Taking $\e = \frac{1}{2N}$ in the zero-free region version of Siegel's theorem, $m^\frac{1}{2N} \ll \sqrt{\log(x)}$ so that $\b_{\chi} < 1-\frac{c}{\sqrt{\log(x)}}$ for some potentially smaller constant $c$. It follows that $m^{2N}$ Therefore
      \begin{equation}\label{equ:Siegel-Walfisz_12}
        x^{\b_{\chi}} < x^{1-\frac{c}{\sqrt{\log(x)}}} = xe^{-c\log(x)}.
      \end{equation}
      Combining \cref{equ:Siegel-Walfisz_11,equ:Siegel-Walfisz_12} gives the simplified estimate
      \begin{equation}\label{equ:Siegel-Walfisz_13}
        \psi(x;a,m) = \frac{x}{\vphi(m)}+O\left(xe^{-c\sqrt{\log(x)}}\right),
      \end{equation}
      for some potentially smaller constant $c$. Now let
      \[
        \pi_{1}(x;a,m) = \sum_{\substack{n \le x \\ n \equiv a \tmod{m}}}\frac{\L(n)}{\log(n)}.
      \]
      We can write $\pi_{1}(x;a,m)$ in terms of $\psi(x;a,m)$ as follows:
      \begin{align*}
        \pi_{1}(x;a,m) &= \sum_{\substack{n \le x \\ n \equiv a \tmod{m}}}\frac{\L(n)}{\log(n)} \\
        &= \sum_{\substack{n \le x \\ n \equiv a \tmod{m}}}\L(n)\int_{n}^{x}\frac{dt}{t\log^{2}(t)}+\frac{1}{\log(x)}\sum_{\substack{n \le x \\ n \equiv a \tmod{m}}}\L(n) \\
        &= \int_{2}^{x}\sum_{\substack{n \le x \\ n \equiv a \tmod{m}}}\L(n)\frac{dt}{t\log^{2}(t)}+\frac{1}{\log(x)}\sum_{\substack{n \le x \\ n \equiv a \tmod{m}}}\L(n) \\
        &= \int_{2}^{x}\frac{\psi(t;a,m)}{t\log^{2}(t)}\,dt+\frac{\psi(x;a,m)}{\log(x)}.
      \end{align*}
      Applying \cref{equ:Siegel-Walfisz_13} to the last expression yields
      \begin{equation}\label{equ:Siegel-Walfisz_14}
        \pi_{1}(x;a,m) = \int_{2}^{x}\frac{t}{\vphi(m)t\log^{2}(t)}\,dt+\frac{x}{\vphi(m)\log(x)}+O\left(\int_{2}^{x}\frac{e^{-c\sqrt{\log(t)}}}{\log^{2}(t)}\,dt+\frac{xe^{-c\sqrt{\log(x)}}}{\log(x)}\right).
      \end{equation}
      Applying integrating by parts to the main term in \cref{equ:Siegel-Walfisz_14}, we obtain
      \begin{equation}\label{equ:Siegel-Walfisz_15}
        \int_{2}^{x}\frac{t}{\vphi(m)t\log^{2}(t)}\,dt+\frac{x}{\vphi(m)\log(x)} = \int_{2}^{x}\frac{dt}{\vphi(m)\log(t)}+\frac{2}{\vphi(m)\log(2)} = \frac{\Li(x)}{\vphi(m)}+\frac{2}{\vphi(m)\log(2)}.
      \end{equation}
      As for the error term in \cref{equ:Siegel-Walfisz_14}, we use \cref{equ:prime_number_theorem_zeta_8}. Combining \cref{equ:prime_number_theorem_zeta_8,equ:Siegel-Walfisz_14,equ:Siegel-Walfisz_15} yields
      \begin{equation}\label{equ:Siegel-Walfisz_16}
        \pi_{1}(x;a,m) = \frac{\Li(x)}{\vphi(m)}+O\left(xe^{-c\sqrt{\log(x)}}\right),
      \end{equation}
      for some smaller $c$ and where the constant in \cref{equ:Siegel-Walfisz_15} has been absorbed into the error term. As last we pass from $\pi_{1}(x;a,m)$ to $\pi(x;a,m)$. If $p$ is a prime such that $p^{m} < x$ for some $m \ge 1$, then $p < x^{\frac{1}{2}} < x^{\frac{1}{3}} < \cdots < x^{\frac{1}{m}}$. Therefore
      \begin{equation}\label{equ:Siegel-Walfisz_17}
        \pi_{1}(x;a,m) = \sum_{\substack{n \le x \\ n \equiv a \tmod{m}}}\frac{\L(n)}{\log(n)} = \sum_{\substack{p^{m} \le x \\ p^{m} \equiv a \tmod{m}}}\frac{\log(p)}{m\log(p)} = \pi(x;a,m)+\frac{1}{2}\pi(x^{\frac{1}{2}};a,m)+\cdots.
      \end{equation}
      Moreover, as $\pi(x^{\frac{1}{n}};a,m) < x^{\frac{1}{n}}$ for any $n \ge 1$, we see that $\pi(x;a,m)-\pi_{1}(x;a,m) = O(x^{\frac{1}{2}})$. This estimate together with \cref{equ:Siegel-Walfisz_16,equ:Siegel-Walfisz_17} gives
      \[
        \pi(x) = \frac{\Li(x)}{\vphi(m)}+O\left(xe^{-c\sqrt{\log(x)}}\right),
      \]
      because $x^{\frac{1}{2}} \ll xe^{-c\sqrt{\log(x)}}$. This completes the proof.
    \end{proof}

    It is interesting to note that the constant $c$ in the Siegel-Walfisz theorem is ineffective because of the use of Siegel's theorem (it also depends upon $N$). This is unlike the prime number theorem, where the constant $c$ can be made to be effective. The proof of the logarithmic integral and classical versions of the Siegel-Walfisz theorem are immediate:

    \begin{proof}[Proof of Siegel-Walfisz theorem, logarithmic integral and classical versions]
      By the absolute error version of the Siegel-Walfisz theorem,
      \[
        \pi(x;a,m) = \frac{\Li(x)}{\vphi(m)}\left(1+O\left(\frac{\vphi(m)xe^{-c\sqrt{\log(x)}}}{\Li(x)}\right)\right).
      \]
      As $\Li(x) \sim \frac{x}{\log(x)}$, we have
      \[
        \frac{\vphi(m)xe^{-c\sqrt{\log(x)}}}{\Li(x)} \sim \vphi(m)\log(x)e^{-c\sqrt{\log(x)}} = o(1),
      \]
      where the equality holds since $m \ll \log^{N}(x)$ and $\log(x) \ll_{\e} e^{-\e\sqrt{\log(x)}}$. The logarithm integral version follows. The classical version also holds using the asymptotic $\Li(x) \sim \frac{x}{\log(x)}$.
    \end{proof}
    
    Also, we have an optimal error term, in a much wider range of $m$, assuming the Riemann hypothesis for Dirichlet $L$-functions:
    
    \begin{proposition}
      Let $a$ and $m$ be positive integers with $m > 1$ and $(a,m) = 1$. For $x \ge 2$, we have
      \[
        \pi(x;a,m) = \frac{\Li(x)}{\vphi(m)}+O(\sqrt{x}\log(x)),
      \]
      provided $m \le x$ and the Riemann hypothesis for Dirichlet $L$-functions holds.
    \end{proposition}
    \begin{proof}
      Let $\chi$ be a Dirichlet character modulo $m$. If $\rho$ is a nontrivial zero of $L(s,\chi)$, the Riemann hypothesis for Dirichlet $L$-functions implies $|x^{\rho}| = \sqrt{x}$ and that Siegel zeros do not exist so we may merely assume $m \le x$. Therefore as in the proof of the absolute error version of the Siegel-Walfisz theorem,
      \[
        \sum_{|\g| < T}\frac{x^{\rho}}{\rho} \ll \sqrt{x}\log^{2}(x),
      \]
      for $2 \le T \le x$ not coinciding with the ordinate of a nontrivial zero. Repeating the same argument with $T$ determined by $T = x$ for $2 \le x < 3$ and 
      \[
        T^{2} = x,
      \]
      for $x \ge 3$ gives
      \[
        \psi(x;a,m) = \frac{x}{\vphi(m)}+O(\sqrt{x}\log^{2}(x)),
      \]
      and then transferring to $\pi_{1}(x)$ and finally $\pi(x)$ gives
      \[
        \pi(x;a,m) = \frac{x}{\vphi(m)}+O(\sqrt{x}\log(x)).
      \]
      \end{proof}
  \section{The Burgess Bound for Dirichlet \texorpdfstring{$L$}{L}-functions}
    Let $\chi$ be a Dirichlet character modulo $m$. We call any sum of the form
    \[
      \sum_{M+1 \le n \le M+N}\chi(n),
    \]
    for some $M \ge 0$ and $N \ge 1$, a \textbf{character sum}\index{character sum} of $\chi$. In the case $\chi$ is primitive of conductor $q > 1$, a infamous subconvexity result for the Dirichlet $L$-function $L(s,\chi)$ in the $q$-aspect was achieved by Burgess in the 1960's (see \cite{burgess1963character}). The key idea of his argument lies in improved bounds for character sums in certain ranges of $M$ and $N$ relative to powers of $q$ which we will describe. On the one hand, since $|\chi(n)| \le 1$ for all $n \ge 1$, we have the trivial bound
    \begin{equation}\label{equ:character_sum_trivial_bound}
      \sum_{M+1 \le n \le M+N}\chi(n) \ll N.
    \end{equation}
    On the other hand, by the orthogonality relations (\cref{cor:Dirichlet_orthogonality_relations} i) and that $\chi$ is non-principal, we have
    \begin{equation}\label{equ:character_sum_orthogonality_bound}
      \sum_{M+1 \le n \le M+N}\chi(n) \ll m.
    \end{equation}
    Note that \cref{equ:character_sum_orthogonality_bound} is only an improvement upon \cref{equ:character_sum_trivial_bound} when $m \ll N$. It turns out that a much shaper bound than \cref{equ:character_sum_orthogonality_bound} can be obtained with very little work. We will first require a small lemma:

    \begin{lemma}\label{lem:Polya-Vinogradov_lemma}
      Let $f(x)$ be a integrable convex function on $\left(a-\frac{b}{2},a+\frac{b}{2}\right)$ for some $a \in \R$ and $b > 0$. Then
      \[
        f(a) \le \frac{1}{b}\int_{a-\frac{b}{2}}^{a+\frac{b}{2}}f(x)\,dx.
      \]
    \end{lemma}
    \begin{proof}
      By the definition of convexity,
      \[
        f(a) = f\left(\frac{1}{2}x+\frac{1}{2}(2a-x)\right) \le \frac{1}{2}f(x)+\frac{1}{2}f(2a-x).
      \]
      Integrating the left-hand side over $\left(a-\frac{b}{2},a+\frac{b}{2}\right)$ yields
      \[
        \int_{a-\frac{b}{2}}^{a+\frac{b}{2}}f(a)\,dx = bf(a),
      \]
      while integrating the right-hand side over $\left[a-\frac{b}{2},a+\frac{b}{2}\right]$ gives
      \[
        \int_{a-\frac{b}{2}}^{a+\frac{b}{2}}\left(\frac{1}{2}f(x)+\frac{1}{2}f(2a-x)\right)\,dx = \frac{1}{2}\int_{a-\frac{b}{2}}^{a+\frac{b}{2}}f(x)\,dx+\int_{a-\frac{b}{2}}^{a+\frac{b}{2}}\frac{1}{2}f(2a-x)\,dx = \int_{a-\frac{b}{2}}^{a+\frac{b}{2}}f(x)\,dx,
      \]
      upon making the change of variables $x \mapsto 2a-x$ in the second integral. Hence
      \[
        bf(a) \le \int_{a-\frac{b}{2}}^{a+\frac{b}{2}}f(x)\,dx,
      \]
      which is equivalent to the claim.
    \end{proof}
    
    We can now improve upon \cref{equ:character_sum_orthogonality_bound}. This result is known as the \textbf{P\'olya-Vinogradov inequality}\index{P\'olya-Vinogradov inequality} as it was proved independently by P\'olya and Vinogradov in 1918 (see \cite{polya1918verteilung}).

    \begin{theorem}[P\'olya-Vinogradov inequality]
      Let $\chi$ be a non-principal Dirichlet character modulo $m$. Then for any $M \ge 0$ and $N \ge 1$,
      \[
        \sum_{M+1 \le n \le M+N}\chi(n) \ll \sqrt{m}\log(m).
      \]
    \end{theorem}
    \begin{proof}
      First suppose $\chi$ is primitive of conductor $q > 1$. Using \cref{cor:gauss_sum_primitive_formula}, we have
      \[
        \sum_{M+1 \le n \le M+N}\chi(n) = \frac{1}{\tau(\cchi)}\sum_{a \tmod{q}}\cchi(a)\sum_{M+1 \le n \le M+N}e^{\frac{2\pi ian}{q}}.
      \]
      The inner sum is geometric and evaluates to
      \[
        \sum_{M+1 \le n \le M+N}e^{\frac{2\pi ian}{q}} = e^{\frac{2\pi ia(M+1)}{q}}\left(\frac{1-e^{\frac{2\pi iaN}{q}}}{1-e^{\frac{2\pi ia}{q}}}\right) = e^{\frac{2\pi ia\left(M+\frac{N-1}{2}\right)}{q}}\left(\frac{e^{\frac{\pi iaN}{q}}-e^{-\frac{\pi iaN}{q}}}{e^{\frac{\pi ia}{q}}-e^{-\frac{\pi ia}{q}}}\right) = e^{\frac{2\pi ia\left(M+\frac{N-1}{2}\right)}{q}}\frac{\sin\left(\frac{\pi Na}{q}\right)}{\sin\left(\frac{\pi a}{q}\right)},
      \]
      where in the last equality we have made use of the formula $\sin(x) = \frac{e^{ix}+e^{-ix}}{2i}$. Then \cref{thm:Gauss_sum_modulus}, gives
      \[
        \sum_{M+1 \le n \le M+N}\chi(n) \ll \frac{1}{\sqrt{q}}\sum_{1 \le a \le q-1}\frac{1}{\sin\left(\frac{\pi a}{q}\right)},
      \]
      where we recall that $\cchi(0) = 0$. Now the function $\frac{1}{\sin(\pi x)}$ is convex and integrable on $(0,1)$ (its second derivative is $\pi^{2}\frac{1+\cos^{2}(\pi x)}{\sin^{3}(\pi x)} > 0$ on this interval), so from \cref{lem:Polya-Vinogradov_lemma} we get
      \[
        \frac{1}{\sqrt{q}}\sum_{1 \le a \le q-1}\frac{1}{\sin\left(\frac{\pi a}{q}\right)} \le \frac{1}{\sqrt{q}}\sum_{1 \le a \le q-1}q\int_{\frac{a}{q}-\frac{1}{2q}}^{\frac{a}{q}+\frac{1}{2q}}\frac{1}{\sin(\pi x)}\,dx = \sqrt{q}\int_{\frac{1}{2q}}^{1-\frac{1}{2q}}\frac{1}{\sin(\pi x)}\,dx = 2\sqrt{q}\int_{\frac{1}{2q}}^{\frac{1}{2}}\frac{1}{\sin(\pi x)}\,dx,
      \]
      where the last equality holds because $\sin(\pi x) = \sin(\pi-\pi x) = \sin(\pi(1-x))$ so that the integral of $\frac{1}{\sin(\pi x)}$ is symmetric over $\left(\frac{1}{2q},\frac{1}{2}\right)$ and $\left(\frac{1}{2},1-\frac{1}{2q}\right)$. Now $\sin(\pi x) \ge 2x$ on the interval $[0,\frac{1}{2}]$ (because $\sin(\pi x) = 2x$ at the boundary and both functions are increasing on the interior), so that
      \[
        2\sqrt{q}\int_{\frac{1}{2q}}^{\frac{1}{2}}\frac{1}{\sin(\pi x)}\,dx \le q\int_{\frac{1}{2q}}^{\frac{1}{2}}\frac{1}{x}\,dx \ll \sqrt{q}\log(q).
      \]
      Therefore
      \[
        \sum_{M+1 \le n \le M+N}\chi(n) \ll \sqrt{q}\log(q),
      \]
      as desired. This proves the bound in the case $\chi$ is primitive. Now suppose $\chi$ is induced by the primitive character $\wtilde{\chi}$ of conductor $q > 1$. Then $q \mid m$, and so we may write $m = kq$ for some $k \ge 1$. Rewrite the sum in terms of $\wtilde{\chi}$ as follows:
      \begin{align*}
        \sum_{M+1 \le n \le M+N}\chi(n) &= \sum_{\substack{M+1 \le n \le M+N \\ (n,k) = 1}}\wtilde{\chi}(n) \\
        &= \sum_{M+1 \le n \le M+N}\wtilde{\chi}(n)\sum_{d \mid (n,k)}\mu(d) && \text{\cref{prop:Mobius_dirac_delta}} \\
        &= \sum_{d \mid k}\mu(d)\sum_{\substack{M+1 \le n \le M+N \\ d \mid n}}\wtilde{\chi}(n) \\
        &= \sum_{d \mid k}\mu(d)\sum_{\frac{M+1}{d} \le n \le \frac{M+N}{d}}\wtilde{\chi}(dn) && \text{$n \to dn$} \\
        &= \sum_{d \mid k}\mu(d)\wtilde{\chi}(d)\sum_{\left\lfloor\frac{M+1}{d}\right\rfloor \le n \le \left\lfloor\frac{M+N}{d}\right\rfloor}\wtilde{\chi}(n).
      \end{align*}
      The inner sum is $O(\sqrt{q}\log(q))$ by the primitive case and so
      \[
        \sum_{M+1 \le n \le M+N}\chi(n) \ll \sqrt{q}\log(q)\sum_{d \mid k}|\mu(d)| \ll 2^{\w(k)}\sqrt{q}\log(q).
      \]
      But as $2^{\w(k)} \ll \s_{0}(k) \ll_{\e} k^{\e}$ (see \cref{prop:sum_of_divisors_growth_rate}) and thus $2^{\w(k)} \ll \sqrt{k}$ upon taking $\e = \frac{1}{2}$. This bound together with $\log(q) \le \log(m)$ gives
      \[
        \sum_{M+1 \le n \le M+N}\chi(n) \ll \sqrt{m}\log(m),
      \]
      as claimed.
    \end{proof}

    The P\'olya-Vinogradov inequality greatly improves upon \cref{equ:character_sum_orthogonality_bound} and is particularly useful when $N$ is much larger than $m$. A slight improvement was made in 1977 by Montgomery and Vaughn under the Riemann hypothesis for Dirichlet $L$-functions (see \cite{montgomery1977exponential}):

    \begin{theorem}\label{thm:MV_bound_character_sum}
      Let $\chi$ be a non-principal Dirichlet character modulo $m$. Then for any $M \ge 0$ and $N \ge 1$,
      \[
        \sum_{M+1 \le n \le M+N}\chi(n) \ll \sqrt{m}\log(m),
      \]
      provided the Riemann hypothesis for Dirichlet $L$-functions holds.
    \end{theorem}
    
    While \cref{thm:MV_bound_character_sum} is not much of an improvement from the P\'olya-Vinogradov inequality, it is sharp due to a result of Paley in 1932 (see \cite{paley1932theorem}). This means that, quite remarkably, the P\'olya-Vinogradov inequality is almost optimal. It is also useful to have an estimate that is sharper when $N$ is small compared to $m$. In 1963, Burgess made progress in this direction by proving the following result (see \cite{burgess1963character} for a proof which is a generalization of his 1962 papers \cite{burgess1962characterL-series} and \cite{burgess1962characterprimitive}):

    \begin{theorem}\label{thm:Burgess_inequality}
      Let $\chi$ be a primitive Dirichlet character modulo $m$ and let $N$, $M$, and $r$ be positive integers. Then for cube-free $m$ or $r = 2$,
      \[
        \sum_{M+1 \le n \le M+N}\chi(n) \ll_{\e} N^{1-\frac{1}{r}}m^{\frac{r+1}{4r^{2}}+\e}.
      \]
    \end{theorem}

    \cref{thm:Burgess_inequality} can be thought of as a blend between \cref{equ:character_sum_trivial_bound} and the P\'olya-Vinogradov inequality. Despite the proof being reasonably short, we do not give the proof as it requires some machinery beyond the scope of this text. More importantly, Burgess used \cref{thm:Burgess_inequality} in conjunction with the P\'olya-Vinogradov inequality to prove a subconvexity estimate for Dirichlet $L$-functions in the conductor aspect:

    \begin{theorem}
      Let $\chi$ be a primitive Dirichlet character of conductor $q > 1$. Then
      \[
        L\left(\frac{1}{2}+it,\chi\right) \ll_{\e} q^{\frac{3}{16}+\e}. 
      \]
    \end{theorem}
    \begin{proof}
      \todo{xxx}
    \end{proof}