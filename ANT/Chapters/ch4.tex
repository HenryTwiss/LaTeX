  \chapter{Additional Results}
    \section{Perron Formulas}
      With the Mellin inversion formula, it is not hard to prove a very useful integral expression for the sum of coefficients of a Dirichlet series. First, we setup some general notation. If $D(s)$ is a Dirichlet series with coefficients $a(n)$, then for any real $X$ we set
      \[
        A'(X) = \psum_{n \le X}a(n),
      \]
      where the ' indicates that the last term is multiplied by $\frac{1}{2}$ if $X$ is an integer. We would like to relate $A'(X)$ to an integral involving the entire Dirichlet series $D(s)$. In particular, this integral is a type of inverse Mellin transform. Any formula that relates a finite sum of coefficeints of a Dirichlet series to an integral involving the entire Dirichlet series is called a \textbf{Perron type formula}\index{Perron type formula}. We will see several of them, the first being \textbf{Perron's formula}\index{Perron's formula} which is a consquence of Abel's summation formula and the Mellin inversion formula applied to Dirichlet series:

      \begin{theorem}[Perron's formula]
        Let $D(s)$ be a Dirichlet series with coefficient $a(n)$ and finite and non-negative abscissa of absolute convergence $\s_{a}$. Then for any $c > \s_{a}$,
        \[
          A'(X) = \frac{1}{2\pi i}\int_{\Re(s) = c}D(s)X^{s}\,\frac{ds}{s}.
        \]
      \end{theorem}
      \begin{proof}
        Let $s$ be such that $\Re(s) > \s_{a}$. By Abel's summation formula (see \cref{append:Summation_Formulas}),
        \[
          \sum_{n \ge 1}\frac{a(n)}{n^{s}} = \lim_{Y \to \infty}A'(Y)Y^{-s}+s\int_{1}^{\infty}A'(u)u^{-(s+1)}\,du.
        \]
        Now $A'(Y) \le A(Y)$ and $A(Y) \ll_{\e} Y^{\s_{a}+\e}$ for any $\e > 0$ by \cref{prop:Dirichlet_series_coefficient_size_on_average}. So that $A'(Y)Y^{-s} \ll Y^{\s_{a}+\e-\Re(s)}$. Choosing $\e < \Re(s)-\s_{a}$, this latter term tends to zero as $Y \to \infty$, which implies that $A'(Y)Y^{-s}$ also tends to zero as $Y \to \infty$. Therefore we can write the equation above as
        \[
          D(s)s^{-1} = \int_{1}^{\infty}A'(u)u^{-(s+1)}\,du = \int_{0}^{\infty}A'(u)u^{-(s+1)}\,du,
        \]
        where the second equality follows because $A(u) = 0$ in the interval $[0,1)$. The Mellin inversion formula immediately gives the result.
      \end{proof}

      We would like to relate this sum to an integral involving the entire Dirichlet series $D(s)$. In particular, this integral is a type of inverse Mellin transform. Any formula that resembles  

      Perron's formula is particularly useful because it allows one examine a sum of Dirichlet coefficients, a discrete object, by means of a complex integral where analytic techniques are at our disposal. There is also a truncated formulation of Perron's formula which is often more useful for estimates rather than abstract results. To state it, we need to setup some notation and will require a lemma. For any $c > 0$, consider the discontinuous integral (see \cite{davenport2013multiplicative})
      \[
        \d(y) = \frac{1}{2\pi i}\int_{(c)}y^{s}\,\frac{ds}{s} = \begin{cases} 0 & \text{if $0 < y < 1$}, \\ \frac{1}{2} & \text{if $y = 1$}, \\ 1 & \text{if $y > 1$}. \end{cases}
      \]
      Also, for any $T > 0$, let
      \[
        I(y,T) = \frac{1}{2\pi i}\int_{c-iT}^{c+iT}y^{s}\,\frac{ds}{s},
      \]
      be $\d(y)$ truncated outside of height $T$. The lemma we require gives an approximation for how close $I(y,T)$ is to $\d(y)$ (see \cite{davenport2013multiplicative} for a proof):

      \begin{lemma}\label{lem:delta_truncation_estimate}
        For any $c > 0$, $y > 0$, and $T > 0$, we have
        \[
          I(y,T)-\d(y) = \begin{cases} O\left(y^{c}\min\left(1,\frac{1}{T|\log(y)|}\right)\right) & \text{if $y \neq 1$}, \\ O\left(\frac{c}{T}\right) & \text{if $y = 1$}. \end{cases}
        \]
      \end{lemma}

      We can now state and prove the truncated formulation of Perron's formula:

      \begin{theorem}[Perron's formula, truncated formulation]
        Let $D(s)$ be a Dirichlet series with coefficient $a(n)$ and finite and non-negative abscissa of absolute convergence $\s_{a}$. Then for any $c > \s_{a}$ and $T > 0$,
        \[
          A'(X) = \frac{1}{2\pi i}\int_{c-iT}^{c+iT}D(s)X^{s}\,\frac{ds}{s}+O\left(X^{c}\sum_{\substack{n \ge 1 \\ n \neq X}}\frac{a(n)}{n^{c}}\min\left(1,\frac{1}{T\left|\log\left(\frac{X}{n}\right)\right|}\right)+\d_{X}\frac{c}{T}\right),
        \]
        where $\d_{X} = 1,0$ according to if $X$ is an integer or not.
      \end{theorem}
      \begin{proof}
        By \cref{append:Special_Integrals}, we have
        \[
          A'(X) = \sum_{n \ge 1}a(n)\d\left(\frac{X}{n}\right).
        \]
        Now using \cref{lem:delta_truncation_estimate}, we may replace $\d\left(\frac{X}{n}\right)$ to obtain
        \[
          A'(X) = \sum_{n \ge 1}a(n)\frac{1}{2\pi i}\int_{c-iT}^{c+iT}\frac{X^{s}}{n^{s}}\,\frac{ds}{s}+\sum_{\substack{n \ge 1 \\ n \neq X}}a(n)\left[O\left(\frac{X^{c}}{n^{c}}\min\left(1,\frac{1}{T\left|\log\left(\frac{X}{n}\right)\right|}\right)+\d_{X}\frac{c}{T}\right)\right].
        \]
        Since $D(s)$ converges absolutely we may move the sum inside of the big $O$-estimate, and moreover, the dominated convergence theorem implies we may interchange the sum and the integral. The statement of the lemma follows.
      \end{proof}

      There is a slighly weaker variant of the truncated formulation of Perron's formula that follows as a corollary:

      \begin{corollary}
        Let $D(s)$ be a Dirichlet series with coefficient $a(n)$ and finite and non-negative abscissa of absolute convergence $\s_{a}$. Then for any $c > \s_{a}$ and $T > 0$,
        \[
          A'(X) = \frac{1}{2\pi i}\int_{c-iT}^{c+iT}D(s)X^{s}\,\frac{ds}{s}+O_{c}\left(\frac{X^{c}}{T}\right),
        \]
      \end{corollary}
      \begin{proof}
        For sufficiently large $X$, we have
        \[
          \min\left(1,\frac{1}{T\left|\log\left(\frac{X}{n}\right)\right|}\right) <_{c} \frac{X^{c}}{T}.
        \]
        The statement now follows from the truncated formulation of Perron's formula.
      \end{proof}

      There is also a formulation of Perron's formula where we add a smoothing function $\psi:[0,\infty) \to [0,\infty)$. These sums generally look like
      \[
        \sum_{n \ge 1}a(n)\psi\left(\frac{n}{X}\right),
      \]
      where $\psi\left(\frac{x}{N}\right)$ weights $a(n)$ in such a way that the weight is dampened as $x \to \infty$ and or as $x \to 0$. This is most useful in two cases. The first is when we choose $\psi(x)$ to be a smooth bump function. In this setting, the bump function has either weight $1$ or $0$ depending upon $n$ and we can estimate sums like
      \[
        \sum_{\frac{X}{2} \le n < X}a(n) \quad \text{or} \quad \sum_{X \le n < X+Y}a(n).
      \]
      For some $X$ and $Y$ with $Y < X$. Sums of this type are called \textbf{unweighted}\index{unweighted}. In the second case, we might want to estimate the weighted sums
      \[
        \sum_{n \ge 1}a(n)\psi\left(\frac{n}{X}\right),
      \]
      where $\psi\left(\frac{x}{N}\right)$ weights $a(n)$ in such a way that the weight is dampened as $x \to \infty$ but is not always $0$ or $1$ as in the unweighted case. Sums of this type are called \textbf{weighted}\index{weighted}. As an example of an unweighted sum, let $\psi:[0,\infty) \to [0,\infty)$ be a smooth bump function that is identically $1$ on $[0,1]$ and decays to zero in the interval $\left[1,\frac{X+1}{X}\right]$. Explicitely,
      \[
          \psi(t) = \begin{cases} 1 & \text{if $0 \le t \le 1$}, \\ e^{-\frac{1-t}{\frac{X+1}{X}-t}} & \text{if $1 < t \le \frac{X+1}{X}$}, \\ 0 & \text{if $t \ge \frac{X+1}{X}$}. \end{cases}
      \]
      Then 
      \[
        \sum_{n \ge 1}a(n)\psi\left(\frac{n}{X}\right) = \sum_{n \le X}a(n).
      \]
      In general, we require $\psi:[0,\infty) \to [0,\infty)$ to be a compactly supported smooth function. Suppose the support of $\psi(x)$ is contained in $[0,C]$. These conditions will force the Mellin transform $\Psi(s)$ of $\psi(x)$ to exist and have nice properties. To see that $\Psi(s)$ exists, let $K$ be a compact set in the region $\Re(s) > 0$ and let $\b = \inf_{s \in K}\{\Re(s)\}$. Note that $\psi(x)$ is bounded because it is compactly supported. Then for $s \in K$,
      \[
        \Psi(s) = \int_{0}^{\infty}\psi(x)x^{s}\,\frac{dx}{x} \ll \int_{0}^{C}x^{\Re(s)-1}\,dx = \frac{x^{\Re(s)}}{\Re(s)}\bigg|_{0}^{C} \le \frac{C^{\Re(s)}}{\b}.
      \]
      Therefore $\Psi(s)$ is locally absolutely bounded for $\Re(s) > 0$. In particular, the Mellin inversion formula implies that $\psi(x)$ is the Mellin inverse of $\Psi(s)$. As for nice properties, $\Psi(s)$ does not grow too fast in vertical strips:

      \begin{proposition}\label{prop:smoothing_function_Mellin_inverse_vertical_strips}
        Let $\psi:[0,\infty) \to [0,\infty)$ to be a compactly supported smooth function and let $\Psi(s)$ denote its Mellin transform. Then for any $N \ge 1$,
        \[
          \Psi(s) \ll s^{-N},
        \]
        provided $s$ is contained in the vertical strip $a < \Re(s) < b$, for any $a$ and $b$ with $0 < a < b$.
      \end{proposition}
      \begin{proof}
        Fix $a$ and $b$ with $a < b$. Also, let the support of $\psi(x)$ is contained in $[0,C]$. Now consider
        \[
          \Psi(s) = \int_{0}^{\infty}\psi(x)x^{s}\,\frac{dx}{x}.
        \]
        Since $\psi(x)$ is compactly supported, integrating by parts yields
        \[
          \Psi(s) = \frac{1}{s}\int_{0}^{\infty}\psi'(x)x^{s+1}\,\frac{dx}{x}.
        \]
        Repeatedly integrating by parts $N \ge 1$ times, we arrive at
        \[
          \Psi(s) = \frac{1}{s(s+1) \cdots (s+N-1)}\int_{0}^{\infty}\psi^{(N)}(x)x^{s+N}\,\frac{dx}{x}.
        \]
        As $s+k-1 \sim s$ for $1 \le k \le N$, we have
        \[
          \Psi(s) \ll s^{-N}\int_{0}^{\infty}\psi^{(N)}(x)x^{s+N}\,\frac{dx}{x}.
        \]
        The claim will follow if we can show that the integral is bounded by a constant. Since $\psi(x)$ is compactly supported so is $\psi^{(N)}(x)$. In particular, this implies $\psi^{(N)}(x)$ is bounded. Therefore
        \[
          \int_{0}^{\infty}\psi^{(N)}(x)x^{s+N}\,\frac{dx}{x} \ll \int_{0}^{C}x^{s+N}\,\frac{dx}{x} = \frac{x^{s+N}}{s+N}\bigg|_{0}^{C} = \frac{C^{s+N}}{s+N} \ll \frac{C^{b+N}}{N} \ll 1,
        \]
        where the second to last estimate follows because $a < \Re(s) < b$ with $0 < a < b$. So the integral is bounded by a constant and the claim follows.
      \end{proof}
      
      The following theorem is the smoothed formulation of Perron's formula:

      \begin{theorem}[Perron's formula, smoothed formulation]
        Let $D(s)$ be a Dirichlet series with coefficient $a(n)$ and finite and nonnegative abscissa of absolute convergence $\s_{a}$. Let $\psi:[0,\infty) \to [0,\infty)$ be a compactly supported smooth function. Then for any $c > \s_{a}$,
        \[
          A(X) = \frac{1}{2\pi i}\int_{\Re(s) = c}D(s)\Psi(s)X^{s}\,ds,
        \]
        where $\Psi$ is the Mellin transform of $\psi$.
      \end{theorem}
      \begin{proof}
        This is just a computation:
        \begin{align*}
          A(X) &= \sum_{n \le X}a(n) \\
          &= \sum_{n \ge 1}a(n)\psi\left(\frac{n}{X}\right) \\
          &= \sum_{n \ge 1}\frac{a(n)}{2\pi i}\int_{\Re(s) = c}\Psi(s)\left(\frac{n}{X}\right)^{-s}\,ds & \text{inverse mellin transform} \\
          &= \frac{1}{2\pi i}\int_{\Re(s) = \s}\sum_{n \ge 1}a(n)\Psi(s)\left(\frac{n}{X}\right)^{-s}\,ds & \text{DCT} \\
          &= \frac{1}{2\pi i}\int_{\Re(s) = c}D(s)\Psi(s)X^{s}\,ds. \\
      \end{align*}
      \end{proof}

      The smoothed formulation of Perron's formula is useful because it is often more versatile than working with Perron's formula directly. Indeed, from \cref{prop:smoothing_function_Mellin_inverse_vertical_strips} the convergence of the integral in the smoothed formulation improves and we can use analytic techniques to estimate this integral.
    \section{The Petersson Trace Formula}
      From \cref{thm:newforms_characterization}, $\mc{S}_{k}(N,\chi)$ admits an orthonormal basis of primitive Hecke eigenforms. Call this basis $\{u_{j}\}_{1 \le j \le r}$ where $r$ is the dimension of $\mc{S}_{k}(N,\chi)$. Each of these cusp forms admits a Fourier series
      \[
        u_{j}(z) = \sum_{n \ge 1}\l_{j}(n)e^{2\pi inz}.
      \]
      The Petersson trace formula is an equation relating the Fourier coefficients $\l_{j}(n)$ of the basis $\{u_{j}\}_{1 \le j \le r}$ to a sum of $J$-Bessel functions and Sali\'e sums. To prove the Petersson trace formula we compute the inner product of two Poincar\'e series $P_{n,k,\chi}$ and $P_{m,k,\chi}$ in two different ways. One way is geometric in nature using the unfolding method and the other uses the spectral theory of $\mc{S}_{k}(N,\chi)$. Since \cref{equ:Petersson_inner_product_with_Poincare_series} says that $\<P_{n,k,\chi},P_{m,k,\chi}\>$ essentially extacts the $m$-th Fourier coefficeint of $P_{n,k,\chi}$, the Petersson trace formula amounts to computing the $m$-th Fourier coefficient of $P_{n,k,\chi}$ in two different ways.

      We will begin with the geometric method first which utilizes \cref{equ:Petersson_inner_product_with_Poincare_series}. In order to make use of this equation we need to compute the Forier series of $P_{n,k,\chi}$ and this is achieved by the Poisson summation formula. From \cref{rem:Bruhat_bijection}, we have
      \[
        P_{n,k,\chi}(z) = e^{2\pi inz}+\sum_{\substack{c \ge 1, d \in \Z \\ c \equiv 0 \tmod{N} \\ (c,d) = 1}}\cchi(d)\frac{e^{2\pi in\left(\frac{a}{c}-\frac{1}{c^{2}z+cd}\right)}}{(cz+d)^{k}},
      \]
      where $a$ and $b$ are chosen such that $\det\left(\begin{psmallmatrix} a & b \\ c & d \end{psmallmatrix}\right) = 1$ and we have used the fact that
      \[
        \frac{a}{c}-\frac{1}{c^{2}z+cd} = \frac{az+b}{cz+d}.
      \]
      Now summing over all pairs $(c,d)$ with $c \ge 1$, $d \in \Z$, $c \equiv 0 \tmod{N}$, and $(c,d) = 1$ is the same as summing over all triples $(c,\ell,r)$ with $c \ge 1$, $\ell \in \Z$, $r \tmod{c}$, with $c \equiv 0 \tmod{N}$ and $(r,c) = 1$. Indeed, this is seen by writing $d = c\ell+r$. Moreover, since $ad-bc = 1$ we have $a(c\ell+r)-bc = 1$ which further implies that $ar \equiv 1 \tmod{c}$. So we may take $a$ to be the inverse for $r$ modulo $c$. Then
      \begin{align*}
        \sum_{\substack{c \ge 1,d \in \Z \\ (c,d) = 1 \\ c \equiv 0 \tmod{N}}}\cchi(d)\frac{e^{2\pi in\left(\frac{a}{c}-\frac{1}{c^{2}z+cd}\right)}}{(cz+d)^{k}} &= \sum_{(c,\ell,r)}\cchi(c\ell+r)\frac{e^{2\pi in\left(\frac{a}{c}-\frac{1}{c^{2}z+c^{2}\ell+cr}\right)}}{(cz+c\ell+r)^{k}} \\
        &= \sum_{(c,\ell,r)}\cchi(r)\frac{e^{2\pi in\left(\frac{a}{c}-\frac{1}{c^{2}z+c^{2}\ell+cr}\right)}}{(cz+c\ell+r)^{k}} \\
        &= \psum_{\substack{c \ge 1 \\ c \equiv 0 \tmod{N} \\ r \tmod{c}}}\sum_{\ell \in \Z}\cchi(r)\frac{e^{2\pi in\left(\frac{a}{c}-\frac{1}{c^{2}z+c^{2}\ell+cr}\right)}}{(cz+c\ell+r)^{k}} \\
        &= \psum_{\substack{c \ge 1 \\ c \equiv 0 \tmod{N} \\ r \tmod{c}}}\cchi(r)\sum_{\ell \in \Z}\frac{e^{2\pi in\left(\frac{a}{c}-\frac{1}{c^{2}z+c^{2}\ell+cr}\right)}}{(cz+c\ell+r)^{k}}, 
      \end{align*}
      where the second line holds since $\chi$ has conductor $q \mid N$ and $c \equiv 0 \tmod{N}$ and the $'$ indicates that the sum is over all $r \pmod{c}$ such that $(r,c) = 1$. We will now apply the Poisson summation formula to the innermost sum. Set
      \[
        I_{c,r}(z) = \sum_{\ell \in \Z}\frac{e^{2\pi in\left(\frac{a}{c}-\frac{1}{c^{2}z+c^{2}\ell+cr}\right)}}{(cz+c\ell+r)^{k}}.
      \]
      We will prove a transformation law for $I_{c,r}(z)$. Note that this function is holomorphic on $\H$ because $P_{n,k,\chi}(z)$ is holomorphic on $\H$. So by the identity theorem we may verify a transformation law on a set containing a limit point. Accordingly, set $z = iy$ for $y > 1$ and define
      \[
        f(x) = \frac{e^{2\pi in\left(\frac{a}{c}-\frac{1}{c^{2}x+cr+ic^{2}y}\right)}}{(cx+r+icy)^{k}}.
      \]
      To see that $f(x)$ is Schwarzm first observe
      \[
        \Im\left(\frac{a}{c}-\frac{1}{c^{2}x+cr+ic^{2}y}\right) = \Im\left(-\frac{1}{c^{2}x+cr+ic^{2}y}\right) = \Im\left(\frac{c^{2}x+cr-ic^{2}y}{|c^{2}x+cr+ic^{2}y|}\right) = \frac{c^{2}y}{|c^{2}x+cr+ic^{2}y|}.
      \]
      It follows that $\Im\left(\frac{a}{c}-\frac{1}{c^{2}x+cr+ic^{2}y}\right)$ tends to zero as $x \to \pm\infty$. Moreover, $|cx+r+icy| \ge |icy| \ge c$ so that
      \[
        f(x) \ll \frac{e^{-2\pi n\Im\left(\frac{a}{c}-\frac{1}{c^{2}x+cr+ic^{2}y}\right)}}{c},
      \]
      as $x \to \pm\infty$. Since the right-hand side of this estimate has exponential decay to zero, $f(x)$ is Schwarz. We now compute the Fourier transform:
      \[
        \hat{f}(t) = \int_{-\infty}^{\infty}f(x)e^{-2\pi itx}\,dx = \int_{-\infty}^{\infty}\frac{e^{2\pi in\left(\frac{a}{c}-\frac{1}{c^{2}x+cr+ic^{2}y}\right)}}{(cx+r+icy)^{k}}e^{-2\pi itx}\,dx.
      \]
      Complexify the integral to get
      \[
        \int_{\Im(z) = 0}\frac{e^{2\pi in\left(\frac{a}{c}-\frac{1}{c^{2}z+cr+ic^{2}y}\right)}}{(cz+r+icy)^{k}}e^{-2\pi itz}\,dz.
      \]
      Now make the change of variables $z \to z-\frac{r}{c}-iy$ to obtain
      \[
        e^{2\pi in\frac{a}{c}}e^{2\pi it\left(iy+\frac{r}{c}\right)}\int_{\Im(z) = y}\frac{e^{-\frac{2\pi in}{c^{2}z}}}{(cz)^{k}}e^{-2\pi itz}\,dz.
      \]
      The integrand is meromorphic with a pole only at $z = 0$. Therefore by shifting the line of integration we may take the limit as $\Im(z) \to \infty$ without picking up additional residues. However
      \[
        \left|e^{-\frac{2\pi in}{c^{2}z}}\right| = e^{2\pi n\left(\frac{\Im(z)}{|cz|^{2}}\right)} \quad \text{and} \quad \left|e^{-2\pi itz}\right| = e^{2\pi t\Im(z)}.
      \]
      So as $\Im(z) \to \infty$, the first expression has exponential decay to zero and the second expression does to if and only if $t < 0$. Moreover, when $t = 0$ the second expression is bounded. Altogether this means that the integral vanishes if $t \le 0$. It remains to compute the integral for $t > 0$. To do this, make the change of variables $z \to - \frac{z}{2\pi it}$ to the to obtain
      \begin{align*}
        -\frac{1}{2\pi it}\int_{\Re(x) = 2\pi ty}\frac{e^{-\frac{4\pi^{2}nt}{c^{2}z}}}{\left(-\frac{cz}{2\pi it}\right)^{k}}e^{z}\,dz &= -\frac{1}{2\pi it}\int_{\Re(z) = 2\pi ty}\left(-\frac{2\pi it}{cz}\right)^{k}e^{z-\frac{4\pi^{2}nt}{c^{2}z}}\,dz \\
        &= \frac{(-2\pi it)^{k-1}}{c^{k}}\int_{\Re(z) = 2\pi ty}z^{-k}e^{z-\frac{4\pi^{2}nt}{c^{2}z}}\,dz \\
        &= \frac{(-2\pi it)^{k-1}}{c^{k}}\int_{-\infty}^{(0^{+})}z^{-k}e^{z-\frac{4\pi^{2}nt}{c^{2}z}}\,dz \\
        &= \frac{2\pi i^{-k}}{c}\left(\frac{\sqrt{t}}{\sqrt{n}}\right)^{k-1}J_{k-1}\left(\frac{4\pi\sqrt{nt}}{c}\right),
      \end{align*}
      where in the second to last line we have homotoped the line of integration to a Hankle contour about the negative real axis and in the last line we have used the Schl\"aflin integral representation for the $J$-Bessel function (see \cref{append:Bessel_Functions}). So in total we obtain
      \[
        \left(\frac{2\pi i^{-k}}{c}\left(\frac{\sqrt{t}}{\sqrt{n}}\right)^{k-1}J_{k-1}\left(\frac{4\pi\sqrt{nt}}{c}\right)e^{2\pi in\frac{a}{c}+2\pi it\frac{r}{c}}\right)e^{2\pi t(iy)},
      \]
      when $t > 0$. By the Poisson summation formula and the identity theorem, we have
      \[
        I_{c,r}(z) = \sum_{t > 0}\left(\frac{2\pi i^{-k}}{c}\left(\frac{\sqrt{t}}{\sqrt{n}}\right)^{k-1}J_{k-1}\left(\frac{4\pi\sqrt{nt}}{c}\right)e^{2\pi in\frac{a}{c}+2\pi it\frac{r}{c}}\right)e^{2\pi itz},
      \]
      for all $z \in \H$. Plugging this back into the Poincar\'e series gives a form of the Fourier series
      \begin{align*}
        P_{n,k,\chi}(z) &= e^{2\pi inz}+\psum_{\substack{c \ge 1 \\ c \equiv 0 \tmod{N} \\ r \tmod{c}}}\cchi(r)\sum_{t > 0}\left(\frac{2\pi i^{-k}}{c}\left(\frac{\sqrt{t}}{\sqrt{n}}\right)^{k-1}J_{k-1}\left(\frac{4\pi\sqrt{nt}}{c}\right)e^{2\pi in\frac{a}{c}+2\pi it\frac{r}{c}}\right)e^{2\pi inz} \\
        &= \sum_{t > 0}\left(\d_{n,t}+\left(\frac{\sqrt{t}}{\sqrt{n}}\right)^{k-1}\psum_{\substack{c \ge 1 \\ c \equiv 0 \tmod{N} \\ r \tmod{c}}}\cchi(r)\frac{2\pi i^{-k}}{c}J_{k-1}\left(\frac{4\pi\sqrt{nt}}{c}\right)e^{2\pi in\frac{a}{c}+2\pi it\frac{r}{c}}\right)e^{2\pi itz} \\
        &= \sum_{t > 0}\left(\d_{n,t}+\left(\frac{\sqrt{t}}{\sqrt{n}}\right)^{k-1}\sum_{\substack{c \ge 1 \\ c \equiv 0 \tmod{N}}}\frac{2\pi i^{-k}}{c}J_{k-1}\left(\frac{4\pi\sqrt{nt}}{c}\right)\psum_{r \tmod{c}}\cchi(r)e^{2\pi in\frac{a}{c}+2\pi it\frac{r}{c}}\right)e^{2\pi itz}.
      \end{align*}
      We will simplify the innermostsum. Since $a$ is the inverse for $r$ modulo $c$, the innermost sum above becomes
      \[
        \psum_{r \tmod{c}}\cchi(r)e^{2\pi in\frac{a}{c}+2\pi it\frac{r}{c}} = \psum_{r \tmod{c}}\cchi(\conj{a})e^{2\pi in\frac{a}{c}+2\pi it\frac{\conj{a}}{c}} = \psum_{a \tmod{c}}\chi(a)e^{\frac{2\pi i(an+\conj{a}t)}{c}} = S_{\chi}(n,t;c).
      \]
      So at last, we obtain our desired Fourier series
      \[
        P_{n,k,\chi}(z) = \sum_{t > 0}\left(\d_{n,t}+\left(\frac{\sqrt{t}}{\sqrt{n}}\right)^{k-1}\sum_{\substack{c \ge 1 \\ c \equiv 0 \tmod{N}}}\frac{2\pi i^{-k}}{c}J_{k-1}\left(\frac{4\pi\sqrt{nt}}{c}\right)S_{\chi}(n,t;c)\right)e^{2\pi itz}.
      \]
      We can now derive the first half of the Petersson trace formula. Using \cref{equ:Petersson_inner_product_with_Poincare_series} we obtain
      \[
        \<P_{n,k,\chi},P_{m,k,\chi}\> = \frac{\G(k-1)}{V_{\G_{0}(N)}(4\pi m)^{k-1}}\left(\d_{n,m}+\left(\frac{\sqrt{m}}{\sqrt{n}}\right)^{k-1}\sum_{\substack{c \ge 1 \\ c \equiv 0 \tmod{N}}}\frac{2\pi i^{-k}}{c}J_{k-1}\left(\frac{4\pi\sqrt{nm}}{c}\right)S_{\chi}(n,m;c)\right).
      \]
      To obtain the second half of the Petersson trace formula, we use the fact that $\{u_{j}\}_{1 \le j \le r}$ is an orthonormal basis and \cref{equ:Petersson_inner_product_with_Poincare_series} to write
      \begin{align*}
        P_{n,k,\chi}(z) &= \sum_{1 \le j \le r}\<P_{n,k,\chi},u_{j}\>u_{j}(z) \\
        &= \sum_{1 \le j \le r}\conj{\<P_{n,k,\chi},u_{j}\>}u_{j}(z) \\
        &= \frac{\G(k-1)}{V_{\G_{0}(N)}(4\pi m)^{k-1}}\sum_{1 \le j \le r}\l_{j}(n)u_{j}(z) \\
        &= \frac{\G(k-1)}{V_{\G_{0}(N)}(4\pi m)^{k-1}}\sum_{1 \le j \le r}\l_{j}(n)\sum_{t \ge 1}\l_{j}(t)e^{2\pi itz} \\
        &= \sum_{t \ge 1}\left(\frac{\G(k-1)}{V_{\G_{0}(N)}(4\pi m)^{k-1}}\sum_{1 \le j \le r}\conj{\l_{j}(n)}\l_{j}(t)\right)e^{2\pi itz}.
      \end{align*}
      The last expression is an alternative representation of the Fourier series of $P_{n,k,\chi}$. Using \cref{equ:Petersson_inner_product_with_Poincare_series} again but aplied to this reprensentation, we obtain the second half of the Petersson trace formula
      \[
        \<P_{n,k,\chi},P_{m,k,\chi}\> = \left(\frac{\G(k-1)}{V_{\G_{0}(N)}(4\pi m)^{k-1}}\right)^{2}\sum_{1 \le j \le r}\conj{\l_{j}(n)}\l_{j}(m).
      \]
      Equating the first and second half and canceling the common $\frac{\G(k-1)}{V_{\G_{0}(N)}(4\pi m)^{k-1}}$ factor gives
      \[
        \frac{\G(k-1)}{V_{\G_{0}(N)}(4\pi m)^{k-1}}\sum_{1 \le j \le r}\conj{\l_{j}(n)}\l_{j}(m) = \d_{n,m}+\left(\frac{\sqrt{m}}{\sqrt{n}}\right)^{k-1}\sum_{\substack{c \ge 1 \\ c \equiv 0 \tmod{N}}}\frac{2\pi i^{-k}}{c}J_{k-1}\left(\frac{4\pi\sqrt{nm}}{c}\right)S_{\chi}(n,m;c).
      \]
      Now when $n = m$, $\left(\frac{\sqrt{m}}{\sqrt{n}}\right)^{k-1} = 1$ so we can factor this term out of the entire right-hand side and cancel it resulting in the \textbf{Petersson trace formula}\index{Petersson trace formula}:
      \[
        \frac{\G(k-1)}{V_{\G_{0}(N)}(4\pi\sqrt{nm})^{k-1}}\sum_{1 \le j \le r}\conj{\l_{j}(n)}\l_{j}(m) = \d_{n,m}+\sum_{\substack{c \ge 1 \\ c \equiv 0 \tmod{N}}}\frac{2\pi i^{-k}}{c}J_{k-1}\left(\frac{4\pi\sqrt{nm}}{c}\right)S_{\chi}(n,m;c).
      \]
      We refer to the left-hand side side as the \textbf{spectral side}\index{spectral side} and the right-hand side as the \textbf{geometric side}\index{geometric side}. We collect our work as a theorem:

      \begin{theorem}[Petersson trace formula]
        Let $\{u_{j}\}_{1 \le j \le r}$ be an orthonormal basis of primitive Hecke eigenforms for $\mc{S}_{k}(\G_{0}(N),\chi)$ with Fourier coefficeints $\l_{j}(n)$. Then for any positive integers $n,m \ge 1$, we have
        \[
          \frac{\G(k-1)}{V_{\G_{0}(N)}(4\pi\sqrt{nm})^{k-1}}\sum_{1 \le j \le r}\conj{\l_{j}(n)}\l_{j}(m) = \d_{n,m}+\sum_{\substack{c \ge 1 \\ c \equiv 0 \tmod{N}}}\frac{2\pi i^{-k}}{c}J_{k-1}\left(\frac{4\pi\sqrt{nm}}{c}\right)S_{\chi}(n,m;c).
        \]
      \end{theorem}
    \section{The Ramanujan Conjecture on Average}
      Let $f$ be a weight $k$ primitive Hecke eigenform with Fourier coefficients $\l_{f}(n)$. As an application of Rankin-Selberg convolution, it is possible to show the slightly weaker result $\l_{f}(n) \ll n^{\frac{k-1}{2}+\e}$ holds on average for any $\e > 0$ without assuming the Ramanujan conjecture. To see this, for any real $X$ we have
      \begin{equation}\label{equ:Ramanujan_conjecture_on_average_1}
        \left(\sum_{n \le X}\l_{f}(n)\right)^{2} \le X\sum_{n \le X}|\l_{f}(n)|^{2},
      \end{equation}
      by Cauchy-Schwarz. Now, without assuming the Ramanujan conjecture, the Rankin-Selberg convolution $L(s,f \ox f)$ is absolutely convergent for $\Re(s) > \frac{3}{2}$. So while the critical strip is wider, it still admits meromorphic continuation to $\C$ with a simple pole at $s = 1$. By Landau's theorem, the abscissa of convergence of $L(s,f \ox f)$ and hence $L(s,f \x f)$ is $1$ so that $\sum_{n \le X}|\l_{f}(n)|^{2} \ll_{\e} X^{k+\e}$. Substiuting this bound into \cref{equ:Ramanujan_conjecture_on_average_1}, we obtain
      \[
        \left(\sum_{n \le X}\l_{f}(n)\right)^{2} \ll_{\e} X^{k-1+\e}.
      \]
      Upon taking the square root,
      \[
        \sum_{n \le X}\l_{f}(n) \ll_{\e} X^{\frac{k-1}{2}+\e},
      \]
      for some other $\e > 0$. This bound should be compared with the implication $\l_{f}(n) \ll n^{\frac{k-1}{2}+\e}$ that follows from the Ramanujan conjecture.