\chapter{Trace Formulas}
  There are various types of formulas that relate the Fourier coefficients of holomorphic and Maass forms to various other types of functions. Two of the most important such formulas are the Petersson trace formula and Kuznetsov trace formula.
  \section{The Petersson Trace Formula}
    From \cref{thm:newforms_characterization_holomorphic}, $\mc{S}_{k}(N,\chi)$ admits an orthonormal basis of Hecke eigenforms (note that generally they are not Hecke normalized). Denote this basis by $\{u_{j}\}_{1 \le j \le r}$ where $r$ is the dimension of $\mc{S}_{k}(N,\chi)$. Each of these forms admits a Fourier series
    \[
      u_{j}(z) = \sum_{n \ge 1}a_{j}(n)e^{2\pi inz}.
    \]
    The Petersson trace formula is an equation relating the Fourier coefficients $a_{j}(n)$ of the basis $\{u_{j}\}_{1 \le j \le r}$ to a sum of $J$-Bessel functions and Sali\'e sums. To prove the Petersson trace formula we compute the inner product of two Poincar\'e series $P_{n,k,\chi}(z)$ and $P_{m,k,\chi}(z)$ in two different ways. One way is geometric in nature using the unfolding method and the other uses the spectral theory of $\mc{S}_{k}(N,\chi)$. Since \cref{equ:Petersson_inner_product_with_Poincare_series} says that $\<P_{n,k,\chi},P_{m,k,\chi}\>$ essentially extracts the $m$-th Fourier coefficient of $P_{n,k,\chi}(z)$, the Petersson trace formula amounts to computing the $m$-th Fourier coefficient of $P_{n,k,\chi}(z)$ in two different ways. We will begin with the geometric method first which utilizes \cref{equ:Petersson_inner_product_with_Poincare_series}. In order to make use of this equation we need to compute the Fourier series of $P_{n,k,\chi}(z)$ and this is achieved by the Poisson summation formula. From \cref{rem:Bruhat_bijection}, we have
    \[
      P_{n,k,\chi}(z) = e^{2\pi inz}+\sum_{\substack{c \ge 1, d \in \Z \\ c \equiv 0 \tmod{N} \\ (c,d) = 1}}\cchi(d)\frac{e^{2\pi in\left(\frac{a}{c}-\frac{1}{c^{2}z+cd}\right)}}{(cz+d)^{k}},
    \]
    where $a$ and $b$ are chosen such that $\det\left(\begin{psmallmatrix} a & b \\ c & d \end{psmallmatrix}\right) = 1$ and we have used the fact that
    \[
      \frac{a}{c}-\frac{1}{c^{2}z+cd} = \frac{az+b}{cz+d}.
    \]
    Now summing over all pairs $(c,d)$ with $c \ge 1$, $d \in \Z$, $c \equiv 0 \tmod{N}$, and $(c,d) = 1$ is the same as summing over all triples $(c,\ell,r)$ with $c \ge 1$, $\ell \in \Z$, $r \tmod{c}$, with $c \equiv 0 \tmod{N}$ and $(r,c) = 1$. Indeed, this is seen by writing $d = c\ell+r$. Moreover, since $ad-bc = 1$ we have $a(c\ell+r)-bc = 1$ which further implies that $ar \equiv 1 \tmod{c}$. So we may take $a$ to be the inverse for $r$ modulo $c$. Then
    \begin{align*}
      \sum_{\substack{c \ge 1,d \in \Z \\ (c,d) = 1 \\ c \equiv 0 \tmod{N}}}\cchi(d)\frac{e^{2\pi in\left(\frac{a}{c}-\frac{1}{c^{2}z+cd}\right)}}{(cz+d)^{k}} &= \sum_{(c,\ell,r)}\cchi(c\ell+r)\frac{e^{2\pi in\left(\frac{a}{c}-\frac{1}{c^{2}z+c^{2}\ell+cr}\right)}}{(cz+c\ell+r)^{k}} \\
      &= \sum_{(c,\ell,r)}\cchi(r)\frac{e^{2\pi in\left(\frac{a}{c}-\frac{1}{c^{2}z+c^{2}\ell+cr}\right)}}{(cz+c\ell+r)^{k}} \\
      &= \psum_{\substack{c \ge 1 \\ c \equiv 0 \tmod{N} \\ r \tmod{c}}}\sum_{\ell \in \Z}\cchi(r)\frac{e^{2\pi in\left(\frac{a}{c}-\frac{1}{c^{2}z+c^{2}\ell+cr}\right)}}{(cz+c\ell+r)^{k}} \\
      &= \psum_{\substack{c \ge 1 \\ c \equiv 0 \tmod{N} \\ r \tmod{c}}}\cchi(r)\sum_{\ell \in \Z}\frac{e^{2\pi in\left(\frac{a}{c}-\frac{1}{c^{2}z+c^{2}\ell+cr}\right)}}{(cz+c\ell+r)^{k}},
    \end{align*}
    where the second line holds since $\chi$ has conductor $q \mid N$ and $c \equiv 0 \tmod{N}$. We will now apply the Poisson summation formula to the innermost sum. Set
    \[
      I_{c,r}(z) = \sum_{\ell \in \Z}\frac{e^{2\pi in\left(\frac{a}{c}-\frac{1}{c^{2}z+c^{2}\ell+cr}\right)}}{(cz+c\ell+r)^{k}}.
    \]
    We will prove a transformation law for $I_{c,r}(z)$. Note that this function is holomorphic on $\H$ because $P_{n,k,\chi}(z)$ is holomorphic on $\H$. So by the identity theorem we may verify a transformation law on a set containing a limit point. Accordingly, set $z = iy$ for $y > 1$ and define
    \[
      f(x) = \frac{e^{2\pi in\left(\frac{a}{c}-\frac{1}{c^{2}x+cr+ic^{2}y}\right)}}{(cx+r+icy)^{k}}.
    \]
    To see that $f(x)$ is Schwarz first observe
    \[
      \Im\left(\frac{a}{c}-\frac{1}{c^{2}x+cr+ic^{2}y}\right) = \Im\left(-\frac{1}{c^{2}x+cr+ic^{2}y}\right) = \Im\left(\frac{c^{2}x+cr-ic^{2}y}{|c^{2}x+cr+ic^{2}y|}\right) = \frac{c^{2}y}{|c^{2}x+cr+ic^{2}y|}.
    \]
    It follows that $\Im\left(\frac{a}{c}-\frac{1}{c^{2}x+cr+ic^{2}y}\right)$ tends to zero as $x \to \pm\infty$. Moreover, $|cx+r+icy| \ge |icy| \ge c$ so that
    \[
      f(x) \ll \frac{e^{-2\pi n\Im\left(\frac{a}{c}-\frac{1}{c^{2}x+cr+ic^{2}y}\right)}}{c},
    \]
    as $x \to \pm\infty$. Since the right-hand side of this estimate has exponential decay, $f(x)$ is Schwarz. We now compute the Fourier transform:
    \[
      \hat{f}(t) = \int_{-\infty}^{\infty}f(x)e^{-2\pi itx}\,dx = \int_{-\infty}^{\infty}\frac{e^{2\pi in\left(\frac{a}{c}-\frac{1}{c^{2}x+cr+ic^{2}y}\right)}}{(cx+r+icy)^{k}}e^{-2\pi itx}\,dx.
    \]
    Complexify the integral to get
    \[
      \int_{\Im(z) = 0}\frac{e^{2\pi in\left(\frac{a}{c}-\frac{1}{c^{2}z+cr+ic^{2}y}\right)}}{(cz+r+icy)^{k}}e^{-2\pi itz}\,dz.
    \]
    Now make the change of variables $z \to z-\frac{r}{c}-iy$ to obtain
    \[
      e^{2\pi in\frac{a}{c}+2\pi it\left(iy+\frac{r}{c}\right)}\int_{\Im(z) = y}\frac{e^{-\frac{2\pi in}{c^{2}z}}}{(cz)^{k}}e^{-2\pi itz}\,dz.
    \]
    The integrand is meromorphic with a pole only at $z = 0$. Therefore, by shifting the line of integration we may take the limit as $y \to \infty$ without picking up additional residues. However, we have
    \[
      \left|e^{-\frac{2\pi in}{c^{2}z}}\right| = e^{2\pi n\left(\frac{y}{|cz|^{2}}\right)} \quad \text{and} \quad \left|e^{-2\pi itz}\right| = e^{2\pi ty}.
    \]
    The first expression has exponential decay while the second expression does if and only if $t < 0$. Moreover, when $t = 0$ the second expression is bounded. Altogether, this means that the integral vanishes if $t \le 0$. It remains to compute the integral for $t \ge 1$. To do this, make the change of variables $z \to - \frac{z}{2\pi it}$ to the last integral to rewrite it as
    \begin{align*}
      -\frac{1}{2\pi it}\int_{(2\pi ty)}\frac{e^{-\frac{4\pi^{2}nt}{c^{2}z}}}{\left(-\frac{cz}{2\pi it}\right)^{k}}e^{z}\,dz &= -\frac{1}{2\pi it}\int_{(2\pi ty)}\left(-\frac{2\pi it}{cz}\right)^{k}e^{z-\frac{4\pi^{2}nt}{c^{2}z}}\,dz \\
      &= \frac{(-2\pi it)^{k-1}}{c^{k}}\int_{(2\pi ty)}z^{-k}e^{z-\frac{4\pi^{2}nt}{c^{2}z}}\,dz \\
      &= \frac{(-2\pi it)^{k-1}}{c^{k}}\int_{-\infty}^{(0^{+})}z^{-k}e^{z-\frac{4\pi^{2}nt}{c^{2}z}}\,dz \\
      &= \frac{2\pi i^{-k}}{c}\left(\frac{\sqrt{t}}{\sqrt{n}}\right)^{k-1}J_{k-1}\left(\frac{4\pi\sqrt{nt}}{c}\right),
    \end{align*}
    where in the second to last line we have homotoped the line of integration to a Hankel contour about the negative real axis and in the last line we have used the Schl\"aflin integral representation for the $J$-Bessel function (see \cref{append:Bessel_Functions}). So in total we obtain
    \[
      \left(\frac{2\pi i^{-k}}{c}\left(\frac{\sqrt{t}}{\sqrt{n}}\right)^{k-1}J_{k-1}\left(\frac{4\pi\sqrt{nt}}{c}\right)e^{2\pi in\frac{a}{c}+2\pi it\frac{r}{c}}\right)e^{-2\pi ty},
    \]
    when $t \ge 1$. By the Poisson summation formula and the identity theorem, we have
    \[
      I_{c,r}(z) = \sum_{t \ge 1}\left(\frac{2\pi i^{-k}}{c}\left(\frac{\sqrt{t}}{\sqrt{n}}\right)^{k-1}J_{k-1}\left(\frac{4\pi\sqrt{nt}}{c}\right)e^{2\pi in\frac{a}{c}+2\pi it\frac{r}{c}}\right)e^{2\pi itz},
    \]
    for all $z \in \H$. Plugging this back into the Poincar\'e series gives a form of the Fourier series:
    \begin{align*}
      P_{n,k,\chi}(z) &= e^{2\pi inz}+\psum_{\substack{c \ge 1 \\ c \equiv 0 \tmod{N} \\ r \tmod{c}}}\cchi(r)\sum_{t \ge 1}\left(\frac{2\pi i^{-k}}{c}\left(\frac{\sqrt{t}}{\sqrt{n}}\right)^{k-1}J_{k-1}\left(\frac{4\pi\sqrt{nt}}{c}\right)e^{2\pi in\frac{a}{c}+2\pi it\frac{r}{c}}\right)e^{2\pi inz} \\
      &= \sum_{t \ge 1}\left(\d_{n,t}+\left(\frac{\sqrt{t}}{\sqrt{n}}\right)^{k-1}\psum_{\substack{c \ge 1 \\ c \equiv 0 \tmod{N} \\ r \tmod{c}}}\cchi(r)\frac{2\pi i^{-k}}{c}J_{k-1}\left(\frac{4\pi\sqrt{nt}}{c}\right)e^{2\pi in\frac{a}{c}+2\pi it\frac{r}{c}}\right)e^{2\pi itz} \\
      &= \sum_{t \ge 1}\left(\d_{n,t}+\left(\frac{\sqrt{t}}{\sqrt{n}}\right)^{k-1}\sum_{\substack{c \ge 1 \\ c \equiv 0 \tmod{N}}}\frac{2\pi i^{-k}}{c}J_{k-1}\left(\frac{4\pi\sqrt{nt}}{c}\right)\psum_{r \tmod{c}}\cchi(r)e^{2\pi in\frac{a}{c}+2\pi it\frac{r}{c}}\right)e^{2\pi itz}.
    \end{align*}
    We will simplify the innermost sum. Since $a$ is the inverse for $r$ modulo $c$, the innermost sum above becomes
    \[
      \psum_{r \tmod{c}}\cchi(r)e^{2\pi in\frac{a}{c}+2\pi it\frac{r}{c}} = \psum_{r \tmod{c}}\cchi(\conj{a})e^{2\pi in\frac{a}{c}+2\pi it\frac{\conj{a}}{c}} = \psum_{a \tmod{c}}\chi(a)e^{\frac{2\pi i(an+\conj{a}t)}{c}} = S_{\chi}(n,t,c).
    \]
    So at last, we obtain our desired Fourier series
    \[
      P_{n,k,\chi}(z) = \sum_{t \ge 1}\left(\d_{n,t}+\left(\frac{\sqrt{t}}{\sqrt{n}}\right)^{k-1}\sum_{\substack{c \ge 1 \\ c \equiv 0 \tmod{N}}}\frac{2\pi i^{-k}}{c}J_{k-1}\left(\frac{4\pi\sqrt{nt}}{c}\right)S_{\chi}(n,t,c)\right)e^{2\pi itz}.
    \]
    We can now derive the first half of the Petersson trace formula. Using \cref{equ:Petersson_inner_product_with_Poincare_series} we obtain
    \[
      \<P_{n,k,\chi},P_{m,k,\chi}\> = \frac{\G(k-1)}{V_{\G_{0}(N)}(4\pi m)^{k-1}}\left(\d_{n,m}+\left(\frac{\sqrt{m}}{\sqrt{n}}\right)^{k-1}\sum_{\substack{c \ge 1 \\ c \equiv 0 \tmod{N}}}\frac{2\pi i^{-k}}{c}J_{k-1}\left(\frac{4\pi\sqrt{nm}}{c}\right)S_{\chi}(n,m,c)\right).
    \]
    To obtain the second half of the Petersson trace formula, we use the fact that $\{u_{j}\}_{1 \le j \le r}$ is an orthonormal basis and \cref{equ:Petersson_inner_product_with_Poincare_series} to write
    \begin{align*}
      P_{n,k,\chi}(z) &= \sum_{1 \le j \le r}\<P_{n,k,\chi},u_{j}\>u_{j}(z) \\
      &= \sum_{1 \le j \le r}\conj{\<P_{n,k,\chi},u_{j}\>}u_{j}(z) \\
      &= \frac{\G(k-1)}{V_{\G_{0}(N)}(4\pi n)^{k-1}}\sum_{1 \le j \le r}a_{j}(n)u_{j}(z) \\
      &= \frac{\G(k-1)}{V_{\G_{0}(N)}(4\pi n)^{k-1}}\sum_{1 \le j \le r}a_{j}(n)\sum_{t \ge 1}a_{j}(t)e^{2\pi itz} \\
      &= \sum_{t \ge 1}\left(\frac{\G(k-1)}{V_{\G_{0}(N)}(4\pi n)^{k-1}}\sum_{1 \le j \le r}\conj{a_{j}(n)}a_{j}(t)\right)e^{2\pi itz}.
    \end{align*}
    The last expression is an alternative representation of the Fourier series of $P_{n,k,\chi}(z)$. Using \cref{equ:Petersson_inner_product_with_Poincare_series} again but applied to this representation, we obtain the second half of the Petersson trace formula
    \[
      \<P_{n,k,\chi},P_{m,k,\chi}\> = \left(\frac{\G(k-1)}{V_{\G_{0}(N)}(4\pi \sqrt{nm})^{k-1}}\right)^{2}\sum_{1 \le j \le r}\conj{a_{j}(n)}a_{j}(m).
    \]
    Equating the first and second half and canceling the common $\frac{\G(k-1)}{V_{\G_{0}(N)}(4\pi m)^{k-1}}$ factor gives
    \[
      \frac{\G(k-1)}{V_{\G_{0}(N)}(4\pi n)^{k-1}}\sum_{1 \le j \le r}\conj{a_{j}(n)}a_{j}(m) = \d_{n,m}+\left(\frac{\sqrt{m}}{\sqrt{n}}\right)^{k-1}\sum_{\substack{c \ge 1 \\ c \equiv 0 \tmod{N}}}\frac{2\pi i^{-k}}{c}J_{k-1}\left(\frac{4\pi\sqrt{nm}}{c}\right)S_{\chi}(n,m,c).
    \]
    Since $\left(\frac{\sqrt{m}}{\sqrt{n}}\right)^{k-1} = 1$ when $n = m$, we can factor this term out of the entire right-hand side and cancel it resulting in the \textbf{Petersson trace formula}\index{Petersson trace formula}:
    \[
      \frac{\G(k-1)}{V_{\G_{0}(N)}(4\pi\sqrt{nm})^{k-1}}\sum_{1 \le j \le r}\conj{a_{j}(n)}a_{j}(m) = \d_{n,m}+\sum_{\substack{c \ge 1 \\ c \equiv 0 \tmod{N}}}\frac{2\pi i^{-k}}{c}J_{k-1}\left(\frac{4\pi\sqrt{nm}}{c}\right)S_{\chi}(n,m,c).
    \]
    We refer to the left-hand side side as the \textbf{spectral side}\index{spectral side} and the right-hand side as the \textbf{geometric side}\index{geometric side}. We collect our work as a theorem:

    \begin{theorem}[Petersson trace formula]
      Let $\{u_{j}\}_{1 \le j \le r}$ be an orthonormal basis of Hecke eigenforms for $\mc{S}_{k}(N,\chi)$ with Fourier coefficients $a_{j}(n)$. Then for any positive integers $n,m \ge 1$, we have
      \[
        \frac{\G(k-1)}{V_{\G_{0}(N)}(4\pi\sqrt{nm})^{k-1}}\sum_{1 \le j \le r}\conj{a_{j}(n)}a_{j}(m) = \d_{n,m}+\sum_{\substack{c \ge 1 \\ c \equiv 0 \tmod{N}}}\frac{2\pi i^{-k}}{c}J_{k-1}\left(\frac{4\pi\sqrt{nm}}{c}\right)S_{\chi}(n,m,c).
      \]
    \end{theorem}
  \section{\todo{The Kuznetsov Trace Formula}}
    The Kuznetsov trace formula is an analog of the Petersson trace formula for weight zero Maass forms. From \cref{thm:the_full_spectral_resolution}, $\mc{L}_{k}(\G_{0}(N),\chi)$ admits an orthonormal basis of Maass forms for the point spectrum (these forms are generally not Hecke-Maass eigenforms because they need not be Hecke normalized or even cuspidal in the case of the discrete spectrum). However, by \cref{prop:residual_forms_weight_zero} and \cref{thm:newforms_characterization_Maass} we make take this orthonormal basis to consist of Hecke-Maass eigenforms and the constant function. Denote this basis by $\{\mu_{j}\}_{j \ge 0}$ with $\mu_{0}(z) = 1$ and let $\mu_{j}$ be of type $\nu_{j}$ for $j \ge 1$. In particular, $\{\mu_{j}\}_{j \ge 1}$ is an orthonormal basis of Hecke-Maass eigenforms and each such form admits a Fourier series
    \[
      u_{j}(z) = \sum_{n \neq 0}a_{j}(n)\sqrt{y}K_{\nu_{j}}(2\pi ny)e^{2\pi inx}.
    \]
    The Kuznetsov trace formula is an equation relating the Fourier coefficients $a_{j}(n)$ of the basis $\{u_{j}\}_{1 \le j \le r}$ to a sum of $J$-Bessel functions and Sali\'e sums. Similar to the Petersson trace formula, we will compute the inner product of two automorphic Poincar\'e series $P_{n,\chi}(z,\psi)$ and $P_{m,\chi}(z,\vphi)$ in two different ways. The first will be geometric in nature while the second will be spectral. We first compute the Fourier series of $P_{n,\chi}(z,\psi)$. From \cref{rem:Bruhat_bijection}, we have
    \[
      P_{n,\chi}(z,\psi) = \psi(\Im(z))e^{2\pi inz}+\sum_{\substack{c \ge 1, d \in \Z \\ c \equiv 0 \tmod{N} \\ (c,d) = 1}}\cchi(d)\psi\left(\frac{\Im(z)}{|cz+d|^{2}}\right)e^{2\pi in\left(\frac{a}{c}-\frac{1}{c^{2}z+cd}\right)},
    \]
    where $a$ and $b$ are chosen such that $\det\left(\begin{psmallmatrix} a & b \\ c & d \end{psmallmatrix}\right) = 1$ and we have used the fact that
    \[
      \frac{a}{c}-\frac{1}{c^{2}z+cd} = \frac{az+b}{cz+d}.
    \]
    Now summing over all pairs $(c,d)$ with $c \ge 1$, $d \in \Z$, $c \equiv 0 \tmod{N}$, and $(c,d) = 1$ is the same as summing over all triples $(c,\ell,r)$ with $c \ge 1$, $\ell \in \Z$, $r \tmod{c}$, with $c \equiv 0 \tmod{N}$ and $(r,c) = 1$. Indeed, this is seen by writing $d = c\ell+r$. Moreover, since $ad-bc = 1$ we have $a(c\ell+r)-bc = 1$ which further implies that $ar \equiv 1 \tmod{c}$. So we may take $a$ to be the inverse for $r$ modulo $c$. Then
    \begin{align*}
      \sum_{\substack{c \ge 1,d \in \Z \\ (c,d) = 1 \\ c \equiv 0 \tmod{N}}}\cchi(d)\psi\left(\frac{\Im(z)}{|cz+d|^{2}}\right)e^{2\pi in\left(\frac{a}{c}-\frac{1}{c^{2}z+cd}\right)} &= \sum_{(c,\ell,r)}\cchi(c\ell+r)\psi\left(\frac{\Im(z)}{|cz+c\ell+r|^{2}}\right)e^{2\pi in\left(\frac{a}{c}-\frac{1}{c^{2}z+c^{2}\ell+cr}\right)} \\
      &= \sum_{(c,\ell,r)}\cchi(r)\psi\left(\frac{\Im(z)}{|cz+c\ell+r|^{2}}\right)e^{2\pi in\left(\frac{a}{c}-\frac{1}{c^{2}z+c^{2}\ell+cr}\right)} \\
      &= \psum_{\substack{c \ge 1 \\ c \equiv 0 \tmod{N} \\ r \tmod{c}}}\sum_{\ell \in \Z}\cchi(r)\psi\left(\frac{\Im(z)}{|cz+c\ell+r|^{2}}\right)e^{2\pi in\left(\frac{a}{c}-\frac{1}{c^{2}z+c^{2}\ell+cr}\right)} \\
      &= \psum_{\substack{c \ge 1 \\ c \equiv 0 \tmod{N} \\ r \tmod{c}}}\cchi(r)\sum_{\ell \in \Z}\psi\left(\frac{\Im(z)}{|cz+c\ell+r|^{2}}\right)e^{2\pi in\left(\frac{a}{c}-\frac{1}{c^{2}z+c^{2}\ell+cr}\right)},
    \end{align*}
    where the second line holds since $\chi$ has conductor $q \mid N$ and $c \equiv 0 \tmod{N}$. We will now apply the Poisson summation formula to the innermost sum. Set
    \[
      I_{c,r}(z) = \sum_{\ell \in \Z}\psi\left(\frac{\Im(z)}{|cz+c\ell+r|^{2}}\right)e^{2\pi in\left(\frac{a}{c}-\frac{1}{c^{2}z+c^{2}\ell+cr}\right)}.
    \]
    We will prove a transformation law for $I_{c,r}(z)$. Note that this function is holomorphic on $\H$ because $P_{n,\chi}(z;\psi)$ is holomorphic on $\H$. So by the identity theorem we may verify a transformation law on a set containing a limit point. Accordingly, set $z = iy$ for $y > 1$ and define
    \[
      f(x) = \psi\left(\frac{y}{|cx+r+icy|^{2}}\right)e^{2\pi in\left(\frac{a}{c}-\frac{1}{c^{2}x+cr+ic^{2}y}\right)}.
    \]
    To see that $f(x)$ is Schwarz first recall that $\psi(y) \ll_{\e} y^{1+\e}$ as $y \to 0$. Therefore $\psi\left(\frac{y}{|cx+r+icy|^{2}}\right)$ remains bounded as $x \to \pm \infty$. Just as with the Petersson trace formula, we have
    \[
      \Im\left(\frac{a}{c}-\frac{1}{c^{2}x+cr+ic^{2}y}\right) = \Im\left(-\frac{1}{c^{2}x+cr+ic^{2}y}\right) = \Im\left(\frac{c^{2}x+cr-ic^{2}y}{|c^{2}x+cr+ic^{2}y|}\right) = \frac{c^{2}y}{|c^{2}x+cr+ic^{2}y|},
    \]
    and so $\Im\left(\frac{a}{c}-\frac{1}{c^{2}x+cr+ic^{2}y}\right)$ tends to zero as $x \to \pm\infty$. Thus
    \[
      f(x) \ll_{\e} \left(\frac{y}{|cx+r+icy|^{2}}\right)^{1+\e}e^{-2\pi n\Im\left(\frac{a}{c}-\frac{1}{c^{2}x+cr+ic^{2}y}\right)},
    \]
    as $x \to \pm\infty$. Since the right-hand side of this estimate has exponential decay, $f(x)$ is Schwarz. We now compute the Fourier transform:
    \[
      \hat{f}(t) = \int_{-\infty}^{\infty}f(x)e^{-2\pi itx}\,dx = \int_{-\infty}^{\infty}\psi\left(\frac{y}{|cx+r+icy|^{2}}\right)e^{2\pi in\left(\frac{a}{c}-\frac{1}{c^{2}x+cr+ic^{2}y}\right)}e^{-2\pi itx}\,dx.
    \]
    Complexify the integral to get
    \[
      \int_{\Im(z) = 0}\psi\left(\frac{y}{|cz+r+icy|^{2}}\right)e^{2\pi in\left(\frac{a}{c}-\frac{1}{c^{2}z+cr+ic^{2}y}\right)}e^{-2\pi itz}\,dz.
    \]
    Now make the change of variables $z \to z-\frac{r}{c}-iy$ to obtain
    \[
      e^{2\pi in\frac{a}{c}+2\pi it\left(iy+\frac{r}{c}\right)}\int_{\Im(z) = y}\psi\left(\frac{y}{|cz|^{2}}\right)e^{-\frac{2\pi in}{c^{2}z}}e^{-2\pi itz}\,dz.
    \]
    The integrand is meromorphic with a pole only at $z = 0$. Therefore, by shifting the line of integration we may take the limit as $y \to \infty$ without picking up additional residues. However, we have
    \[
      \psi\left(\frac{y}{|cz|^{2}}\right) \ll_{\e} \left(\frac{y}{|cz|^{2}}\right)^{1+\e}, \quad \left|e^{-\frac{2\pi in}{c^{2}z}}\right| = e^{2\pi n\left(\frac{y}{|cz|^{2}}\right)}, \quad \text{and} \quad \left|e^{-2\pi itz}\right| = e^{2\pi ty}.
    \]
    The first expression has polynomial decay, the second expression has exponential decay, and the third expression has exponential decay if and only if $t < 0$. Moreover, when $t = 0$ the third expression is bounded. Altogether, this implies that the integral vanishes if $t \le 0$. It remains to compute the integral for $t \ge 1$. To do this, make the change of variables $z \to - \frac{z}{2\pi it}$ to the last integral to rewrite it as
    \begin{align*}
      -\frac{1}{2\pi it}\int_{(2\pi ty)}\psi\left(\frac{4\pi^{2}t^{2}y}{|cz|^{2}}\right)e^{-\frac{4\pi^{2}nt}{c^{2}z}}e^{z}\,dz &= -\frac{1}{2\pi it}\int_{(2\pi ty)}\psi\left(\frac{4\pi^{2}t^{2}y}{|cz|^{2}}\right)e^{z-\frac{4\pi^{2}nt}{c^{2}z}}\,dz \\
      &= -\frac{1}{2\pi it}\int_{(2\pi ty)}\psi\left(\frac{4\pi^{2}t^{2}y}{|cz|^{2}}\right)e^{z-\frac{4\pi^{2}nt}{c^{2}z}}\,dz \\
      &= -\frac{1}{2\pi it}\int_{-\infty}^{0^{+}}\psi\left(\frac{4\pi^{2}t^{2}y}{|cz|^{2}}\right)e^{z-\frac{4\pi^{2}nt}{c^{2}z}}\,dz,
    \end{align*}
    where in the last line we have homotoped the line of integration to a Hankel contour about the negative real axis. Now set
    \[
      F_{c}(y,n,t;\psi) = -\frac{c}{2\pi it}\int_{-\infty}^{0^{+}}\psi\left(\frac{4\pi^{2}t^{2}y}{|cz|^{2}}\right)e^{z-\frac{4\pi^{2}nt}{c^{2}z}}\,dz.
    \]
    Then we obtain
    \[
      \left(\frac{1}{c}F_{c}(y,n,t;\psi)e^{2\pi in\frac{a}{c}+2\pi it\left(iy+\frac{r}{c}\right)}\right)e^{-2\pi ty},
    \]
    when $t \ge 1$. By the Poisson summation formula and the identity theorem, we have
    \[
      I_{c,r}(z) = \sum_{t \ge 1}\left(\frac{1}{c}F_{c}(y,n,t;\psi)e^{2\pi in\frac{a}{c}+2\pi it\left(iy+\frac{r}{c}\right)}\right)e^{2\pi itz},
    \]
    for all $z \in \H$. Plugging this back into the Poincar\'e series gives a form of the Fourier series:
    \begin{align*}
      P_{n,k,\chi}(z) &= \psi(y)e^{2\pi inz}+\psum_{\substack{c \ge 1 \\ c \equiv 0 \tmod{N} \\ r \tmod{c}}}\cchi(r)\sum_{t \ge 1}\left(\frac{1}{c}F_{c}(y,n,t;\psi)e^{2\pi in\frac{a}{c}+2\pi it\left(iy+\frac{r}{c}\right)}\right)e^{2\pi itz} \\
      &= \sum_{t \ge 1}\left(\d_{n,t}\psi(y)+\psum_{\substack{c \ge 1 \\ c \equiv 0 \tmod{N} \\ r \tmod{c}}}\cchi(r)\frac{1}{c}F_{c}(y,n,t;\psi)e^{2\pi in\frac{a}{c}+2\pi it\left(iy+\frac{r}{c}\right)}\right)e^{2\pi itz} \\
      &= \sum_{t \ge 1}\left(\d_{n,t}\psi(y)+\sum_{\substack{c \ge 1 \\ c \equiv 0 \tmod{N}}}\frac{1}{c}F_{c}(y,n,t;\psi)\psum_{r \tmod{c}}\cchi(r)e^{2\pi in\frac{a}{c}+2\pi it\frac{r}{c}}\right)e^{2\pi itz}.
    \end{align*}
    We will simplify the innermost sum. This is exactly the same as for the Petersson trace formula. Since $a$ is the inverse for $r$ modulo $c$, the innermost sum above becomes
    \[
      \psum_{r \tmod{c}}\cchi(r)e^{2\pi in\frac{a}{c}+2\pi it\frac{r}{c}} = \psum_{r \tmod{c}}\cchi(\conj{a})e^{2\pi in\frac{a}{c}+2\pi it\frac{\conj{a}}{c}} = \psum_{a \tmod{c}}\chi(a)e^{\frac{2\pi i(an+\conj{a}t)}{c}} = S_{\chi}(n,t,c).
    \]
    So at last, we obtain our desired Fourier series
    \[
      P_{n,k,\chi}(z) = \sum_{t \ge 1}\left(\d_{n,t}\psi(y)+\sum_{\substack{c \ge 1 \\ c \equiv 0 \tmod{N}}}\frac{1}{c}F_{c}(y,n,t;\psi)S_{\chi}(n,t,c)\right)e^{2\pi itz}.
    \]
    We can now derive the first half of the Petersson trace formula by computing the inner product between $P_{n,\chi}(z,\psi),P_{m,\chi}(z,\vphi)$: 
    \begin{align*}
      \<P_{n,\chi}(\cdot,\psi),P_{m,\chi}(\cdot,\vphi)\> &= \frac{1}{V_{\G_{0}(N)}}\int_{\mc{F}_{\G}}P_{n,\chi}(z,\psi)\conj{P_{m,\chi}(z,\vphi)}\,d\mu \\
      &= \frac{1}{V_{\G_{0}(N)}}\int_{\mc{F}_{\G}}P_{n,\chi}(z,\psi)\sum_{\g \in \G\backslash\G_{0}(N)}\chi(\g)\conj{\vphi(\Im(\g z))}e^{-2\pi im\conj{\g z}}\,d\mu \\
      &= \frac{1}{V_{\G_{0}(N)}}\int_{\mc{F}_{\G}}\sum_{\g \in \G\backslash\G_{0}(N)}\chi(\g)P_{n,\chi}(z,\psi)\conj{\vphi(\Im(\g z))}e^{-2\pi im\conj{\g z}}\,d\mu \\
      &= \frac{1}{V_{\G_{0}(N)}}\int_{\mc{F}_{\G}}\sum_{\g \in \G\backslash\G_{0}(N)}P_{n,\chi}(\g z,\psi)\conj{\vphi(\Im(\g z))}e^{-2\pi im\conj{\g z}}\,d\mu && \text{automorphy} \\
      &= \frac{1}{V_{\G_{0}(N)}}\int_{\G_{\infty}\backslash\H}P_{n,\chi}(z,\psi)\conj{\vphi(\Im(z))}e^{-2\pi im\conj{z}}\,d\mu && \text{unfolding}.
    \end{align*}
      Now substitute in the Fourier series of $P_{n,\chi}(z,\psi)$ to obtain
      \[
        \frac{1}{V_{\G_{0}(N)}}\int_{\G_{\infty}\backslash\H}\sum_{t \ge 1}\left(\d_{n,t}\psi(y)+\sum_{\substack{c \ge 1 \\ c \equiv 0 \tmod{N}}}\frac{1}{c}F_{c}(y,n,t;\psi)S_{\chi}(n,t,c)\right)\conj{\vphi(\Im(z))}e^{2\pi itz-2\pi im\conj{z}}\,d\mu,
      \]
      which is equivalent to
      \[
        \frac{1}{V_{\G_{0}(N)}}\int_{0}^{\infty}\int_{0}^{1}\sum_{t \ge 1}\left(\d_{n,t}\psi(y)+\sum_{\substack{c \ge 1 \\ c \equiv 0 \tmod{N}}}\frac{1}{c}F_{c}(y,n,t;\psi)S_{\chi}(n,t,c)\right)\conj{\vphi(y)}e^{2\pi i(t-m)x}e^{-2\pi i(t+m)y}\,\frac{dx\,dy}{y^{2}}.
      \]
      The inner integral cuts off everything except the $t = m$ term by \cref{equ:Dirac_integral_representation}. This leaves
      \[
        \frac{1}{V_{\G_{0}(N)}}\int_{0}^{\infty}\left(\d_{n,m}\psi(y)+\sum_{\substack{c \ge 1 \\ c \equiv 0 \tmod{N}}}\frac{1}{c}F_{c}(y,n,m;\psi)S_{\chi}(n,m,c)\right)\conj{\vphi(y)}e^{-4\pi imy}\,\frac{dy}{y^{2}}.
      \]
      Interchanging the integral and sum, we write this expression as
      \[
        \d_{m,n}(\psi,\vphi)+\sum_{\substack{c \ge 1 \\ c \equiv 0 \tmod{N}}}\frac{1}{c}K(n,m,c)V_{\psi,\vphi}\left(\frac{4\pi\sqrt{nm}}{c}\right),
      \]
      where we set
      \[
        (\psi,\vphi) = \frac{1}{V_{\G_{0}(N)}}\int_{0}^{\infty}\psi(y)\conj{\vphi(y)}e^{-4\pi imy}\,\frac{dy}{y^{2}},
      \]
      and
      \[
        V_{\psi,\vphi}\left(\frac{4\pi\sqrt{nm}}{c}\right) = \frac{1}{V_{\G_{0}(N)}}\int_{0}^{\infty}F_{c}(y,n,m)\psi(y)\conj{\vphi(y)}e^{-4\pi imy}\frac{dy}{y^{2}}.
      \]
      \todo{xxx}