\chapter{Trace Formulas}
  There are various types of formulas that relate the Fourier coefficients of holomorphic and Maass forms to various other types of functions. Two of the most important such formulas are the Petersson and Kuznetsov trace formulas.
  \section{The Petersson Trace Formula}
    From \cref{thm:newforms_characterization_holomorphic}, $\mc{S}_{k}(N,\chi)$ admits an orthonormal basis of Hecke eigenforms (note that generally they are not Hecke normalized). Denote this basis by $\{u_{j}\}_{1 \le j \le r}$ where $r$ is the dimension of $\mc{S}_{k}(N,\chi)$. Each of these forms admits a Fourier series at the $\mf{a}$ cusp given by
    \[
      (u_{j}|\s_{\mf{a}})(z) = \sum_{n \ge 1}a_{j,\mf{a}}(n)e^{2\pi inz}.
    \]
    Moreover, $\{u_{j}|\s_{\mf{a}}\}_{1 \le j \le r}$ is then immediately an orthonormal basis for $\mc{S}_{k}(\s_{\mf{a}}^{-1}\G_{0}(N)\s_{\mf{a}},\chi)$.
    The Petersson trace formula is an equation relating the Fourier coefficients $a_{j,\mf{a}}(n)$ and $a_{j,\mf{b}}(n)$ of the basis $\{u_{j}\}_{1 \le j \le r}$ for two cusps $\mf{a}$ and $\mf{b}$ of $\G_{0}(N)\backslash\H$ to a sum of $J$-Bessel functions and Sali\'e sums. To prove the Petersson trace formula we compute the inner product of two Poincar\'e series $(P_{n,k,\chi,\mf{a}}|\s_{\mf{b}})(z)$ and $(P_{m,k,\chi,\mf{a}}|\s_{\mf{b}})(z)$ in two different ways. One way is geometric in nature while the other is spectral. Since \cref{equ:Petersson_inner_product_with_Poincare_series} says that $\<P_{n,k,\chi,\mf{a}}|\s_{\mf{b}},P_{m,k,\chi,\mf{a}}|\s_{\mf{b}}\>$ essentially extracts the $m$-th Fourier coefficient of $P_{n,k,\chi,\mf{a}}|\s_{\mf{b}}$, the Petersson trace formula amounts to computing the $m$-th Fourier coefficient of $P_{n,k,\chi,\mf{a}}|\s_{\mf{b}}$ in two different ways. We will begin with the geometric method first. This is easy as we have already computed the Fourier series of the Poincar\'e series. Applying \cref{equ:Petersson_inner_product_with_Poincare_series} to the Poincar\'e series in \cref{prop:Fourier_series_Poincare_holomorphic} and using \cref{lem:invariance_of_volume} gives
    \begin{align*}
      \<P_{n,k,\chi,\mf{a}}|\s_{\mf{b}},P_{m,k,\chi,\mf{a}}|\s_{\mf{b}}\> = \frac{\G(k-1)}{V_{\G_{0}(N)}(4\pi m)^{k-1}}\bigg(&\d_{\mf{a},\mf{b}}\d_{n,m} \\
      &+\left(\frac{\sqrt{m}}{\sqrt{n}}\right)^{k-1}\sum_{c \in \mc{C}_{\mf{a},\mf{b}}}\frac{2\pi i^{-k}}{c}J_{k-1}\left(\frac{4\pi\sqrt{nm}}{c}\right)S_{\chi,\mf{a},\mf{b}}(n,m,c)\bigg).
    \end{align*}
    This is the first half of the Petersson trace formula. To obtain the second half, we use the fact that $\{u_{j}|\s_{\mf{b}}\}_{1 \le j \le r}$ is an orthonormal basis for $\mc{S}_{k}(\s_{\mf{b}}^{-1}\G_{0}(N)\s_{\mf{b}},\chi)$ and \cref{equ:Petersson_inner_product_with_Poincare_series} along with \cref{lem:invariance_of_volume} to write
    \begin{align*}
      (P_{n,k,\chi,\mf{a}}|\s_{\mf{b}})(z) &= \sum_{1 \le j \le r}\<P_{n,k,\chi,\mf{a}}|\s_{\mf{b}},u_{j}|\s_{\mf{b}}\>(u_{j}|\s_{\mf{b}})(z) \\
      &= \sum_{1 \le j \le r}\conj{\<u_{j}|\s_{\mf{b}},P_{n,k,\chi,\mf{a}}|\s_{\mf{b}}\>}(u_{j}|\s_{\mf{b}})(z) \\&= \sum_{1 \le j \le r}\conj{\<u_{j},P_{n,k,\chi,\mf{a}}\>}(u_{j}|\s_{\mf{b}})(z) && \text{\cref{prop:Petersson_slash_invariance_holomorphic}} \\
      &= \frac{\G(k-1)}{V_{\G_{0}(N)}(4\pi n)^{k-1}}\sum_{1 \le j \le r}\conj{a_{j,\mf{a}}(n)}(u_{j}|\s_{\mf{b}})(z) \\
      &= \frac{\G(k-1)}{V_{\G_{0}(N)}(4\pi n)^{k-1}}\sum_{1 \le j \le r}\conj{a_{j,\mf{a}}(n)}\sum_{t \ge 1}a_{j,\mf{b}}(t)e^{2\pi itz} \\
      &= \sum_{t \ge 1}\left(\frac{\G(k-1)}{V_{\G_{0}(N)}(4\pi n)^{k-1}}\sum_{1 \le j \le r}\conj{a_{j,\mf{a}}(n)}a_{j,\mf{b}}(t)\right)e^{2\pi itz}.
    \end{align*}
    This last expression is an alternative Fourier series for $P_{n,k,\chi,\mf{a}}$ at the $\mf{b}$ cusp. Applying \cref{equ:Petersson_inner_product_with_Poincare_series} to this Fourier series and using \cref{lem:invariance_of_volume} again, we obtain
    \[
      \<P_{n,k,\chi,\mf{a}}|\s_{\mf{b}},P_{m,k,\chi,\mf{a}}|\s_{\mf{b}}\> = \left(\frac{\G(k-1)}{V_{\G_{0}(N)}(4\pi \sqrt{nm})^{k-1}}\right)^{2}\sum_{1 \le j \le r}\conj{a_{j,\mf{a}}(n)}a_{j,\mf{b}}(m),
    \]
    which is the second half of the Petersson trace formula. Equating the first and second halves and canceling the common $\frac{\G(k-1)}{V_{\G_{0}(N)}(4\pi m)^{k-1}}$ factor gives
    \[
      \frac{\G(k-1)}{V_{\G_{0}(N)}(4\pi n)^{k-1}}\sum_{1 \le j \le r}\conj{a_{j,\mf{a}}(n)}a_{j,\mf{b}}(m) = \d_{\mf{a},\mf{b}}\d_{n,m}+\left(\frac{\sqrt{m}}{\sqrt{n}}\right)^{k-1}\sum_{c \in \mc{C}_{\mf{a},\mf{b}}}\frac{2\pi i^{-k}}{c}J_{k-1}\left(\frac{4\pi\sqrt{nm}}{c}\right)S_{\chi,\mf{a},\mf{b}}(n,m,c).
    \]
    Since $\left(\frac{\sqrt{m}}{\sqrt{n}}\right)^{k-1} = 1$ when $n = m$, we can factor this term out of the entire right-hand side and cancel it resulting in the \textbf{Petersson trace formula}\index{Petersson trace formula} relative to the $\mf{a}$ and $\mf{b}$ cusps:
    \[
      \frac{\G(k-1)}{V_{\G_{0}(N)}(4\pi \sqrt{nm})^{k-1}}\sum_{1 \le j \le r}\conj{a_{j,\mf{a}}(n)}a_{j,\mf{b}}(m) = \d_{\mf{a},\mf{b}}\d_{n,m}+\sum_{c \in \mc{C}_{\mf{a},\mf{b}}}\frac{2\pi i^{-k}}{c}J_{k-1}\left(\frac{4\pi\sqrt{nm}}{c}\right)S_{\chi,\mf{a},\mf{b}}(n,m,c).
    \]
    We refer to the left-hand side side as the \textbf{spectral side}\index{spectral side} and the right-hand side as the \textbf{geometric side}\index{geometric side}. We collect our work as a theorem:

    \begin{theorem}[Petersson trace formula]
      Let $\{u_{j}\}_{1 \le j \le r}$ be an orthonormal basis of Hecke eigenforms for $\mc{S}_{k}(N,\chi)$ with Fourier coefficients $a_{j,
      mf{a}}(n)$ at the $\mf{a}$ cusp. Then for any positive integers $n,m \ge 1$ and any two cusps $\mf{a}$ and $\mf{b}$, we have
      \[
        \frac{\G(k-1)}{V_{\G_{0}(N)}(4\pi \sqrt{nm})^{k-1}}\sum_{1 \le j \le r}\conj{a_{j,\mf{a}}(n)}a_{j,\mf{b}}(m) = \d_{\mf{a},\mf{b}}\d_{n,m}+\sum_{c \in \mc{C}_{\mf{a},\mf{b}}}\frac{2\pi i^{-k}}{c}J_{k-1}\left(\frac{4\pi\sqrt{nm}}{c}\right)S_{\chi,\mf{a},\mf{b}}(n,m,c).
      \]
    \end{theorem}

    A particularly important case is when $\mf{a} = \mf{b} = \infty$. For then $\mc{C}_{\infty,\infty} = \{c \ge 1:c \equiv 0 \tmod{N}\}$, the Sali\'e sum reduces to the usual one, and the Petersson trace formula takes the form
    \[
      \frac{\G(k-1)}{V_{\G_{0}(N)}(4\pi \sqrt{nm})^{k-1}}\sum_{1 \le j \le r}\conj{a_{j}(n)}a_{j}(m) = \d_{n,m}+\sum_{\substack{c \ge 1 \\ c \equiv 0 \tmod{N}}}\frac{2\pi i^{-k}}{c}J_{k-1}\left(\frac{4\pi\sqrt{nm}}{c}\right)S_{\chi}(n,m,c).
    \]
  \section{\todo{The Kuznetsov Trace Formula}}
    The Kuznetsov trace formula is an analog of the Petersson trace formula for weight zero Maass forms. From \cref{thm:the_full_spectral_resolution}, $\mc{L}(N,\chi)$ admits an orthonormal basis of Maass forms for the point spectrum (these forms are generally not Hecke-Maass eigenforms because they need not be Hecke normalized or even cuspidal in the case of the discrete spectrum). However, by \cref{prop:residual_forms_weight_zero} and \cref{thm:newforms_characterization_Maass} we make take this orthonormal basis to consist of Hecke-Maass eigenforms and the constant function. Denote this basis by $\{u_{j}\}_{j \ge 0}$ with $u_{0}(z) = 1$ and let $u_{j}$ be of type $\nu_{j}$ for $j \ge 1$. In particular, $\{u_{j}\}_{j \ge 1}$ is an orthonormal basis of Hecke-Maass eigenforms and each such form admits a Fourier series at the $\mf{a}$ cusp given by
    \[
      (u_{j}|\s_{\mf{a}})(z) = \sum_{n \neq 0}a_{j,\mf{a}}(n)\sqrt{y}K_{\nu_{j}}(2\pi ny)e^{2\pi inx}.
    \]
    Moreover, $\{u_{j}|\s_{\mf{a}}\}_{j \ge 0}$ is then immediately an orthonormal basis for $\mc{L}(\s_{\mf{a}}^{-1}\G_{0}(N)\s_{\mf{a}},\chi)$. The Kuznetsov trace formula is an equation relating the Fourier coefficients $a_{j,\mf{a}}(n)$ and $a_{j,\mf{b}}(n)$ of the basis $\{u_{j}\}_{j \ge 1}$ for two cusps $\mf{a}$ and $\mf{b}$ of $\G_{0}(N)\backslash\H$ to a sum of integral transforms involving test functions and Sali\'e sums. Similar to the Petersson trace formula, we will compute the inner product of two Poincar\'e series $(P_{n,\chi,\mf{a}}(z,\psi)|\s_{\mf{b}})(z)$ and $(P_{m,\chi,\mf{a}}(z,\vphi)|\s_{\mf{b}})(z)$ in two different ways. The first will be geometric in nature while the second will be spectral. We first need to compute the Fourier series of $(P_{m,\chi,\mf{a}}(z,\vphi)|\s_{\mf{b}})(z)$. Although we will not need it explicitly, we will work over any congruence subgroup:

    \begin{proposition}
      Let $m \ge 1$, $\chi$ be Dirichlet character with conductor dividing the level, $\mf{a}$ and $\mf{b}$ be cusps of $\GH$, and $\psi(y)$ be a smooth function such that $\psi(y) \ll_{\e} y^{1+\e}$ as $y \to 0$. The Fourier series of $P_{m,\chi,\mf{a}}(z,\psi)$ on $\GH$ at the $\mf{b}$ cusp is given by
      \[
        (P_{m,\chi,\mf{a}}|\s_{\mf{b}})(z) = \sum_{t \in \Z}\left(\d_{\mf{a},\mf{b}}\d_{m,t}\psi(\Im(z))+\sum_{\substack{c \in \mc{C}_{\mf{a},\mf{b}}}}\psi_{c}(y,m,t)S_{\chi,\mf{a},\mf{b}}(m,t,c)\right)e^{2\pi itz},
      \]
      where $\psi_{c}(y,m,t)$ is the integral transform given by
      \[
        \psi_{c}(y,m,t) = \int_{\Im(z) = y}\psi\left(\frac{y}{|cz|^{2}}\right)e^{-\frac{2\pi im}{c^{2}z}}e^{-2\pi itz}\,dz.
      \]
    \end{proposition}
    \begin{proof}
      From the cocycle condition and \cref{rem:Bruhat_modulo_infity_exact}, we have
      \[
        P_{m,\chi}(z,\psi) = \d_{\mf{a},\mf{{b}}}\psi(\Im(z))e^{2\pi imz}+\sum_{\substack{c \in \mc{C}_{\mf{a},\mf{b}}, d \in \Z \\ d \tmod{c} \in \mc{D}_{\mf{a},\mf{b}}(c)}}\cchi(d)\psi\left(\frac{\Im(z)}{|cz+d|^{2}}\right)e^{2\pi im\left(\frac{a}{c}-\frac{1}{c^{2}z+cd}\right)},
      \]
      where $a$ and $b$ are chosen such that $\det\left(\begin{psmallmatrix} a & b \\ c & d \end{psmallmatrix}\right) = 1$ and we have used the fact that
      \[
        \frac{a}{c}-\frac{1}{c^{2}z+cd} = \frac{az+b}{cz+d}.
      \]
      Summing over all pairs $(c,d)$ with $c \in \mc{C}_{\mf{a},\mf{b}}$, $d \in \Z$, and $d \in \mc{D}_{\mf{a},\mf{b}}(c)$ is the same as summing over all triples $(c,\ell,r)$ with $c \in \mc{C}_{\mf{a},\mf{b}}$, $\ell \in \Z$, and $r$ taken modulo $c$ with $r \in \mc{D}_{\mf{a},\mf{b}}(c)$. Indeed, this is seen by writing $d = c\ell+r$. Moreover, since $ad-bc = 1$ we have $a(c\ell+r)-bc = 1$ which further implies that $ar \equiv 1 \tmod{c}$. So we may take $a$ to be the inverse for $r$ modulo $c$. Then
      \begin{align*}
        \sum_{\substack{c \in \mc{C}_{\mf{a},\mf{b}}, d \in \Z \\ d \tmod{c} \in \mc{D}_{\mf{a},\mf{b}}(c)}}\cchi(d)\psi\left(\frac{\Im(z)}{|cz+d|^{2}}\right)e^{2\pi im\left(\frac{a}{c}-\frac{1}{c^{2}z+cd}\right)} &= \sum_{(c,\ell,r)}\cchi(c\ell+r)\psi\left(\frac{\Im(z)}{|cz+c\ell+r|^{2}}\right)e^{2\pi im\left(\frac{a}{c}-\frac{1}{c^{2}z+c^{2}\ell+cr}\right)} \\
        &= \sum_{(c,\ell,r)}\cchi(r)\psi\left(\frac{\Im(z)}{|cz+c\ell+r|^{2}}\right)e^{2\pi im\left(\frac{a}{c}-\frac{1}{c^{2}z+c^{2}\ell+cr}\right)} \\
        &= \sum_{\substack{c \in \mc{C}_{\mf{a},\mf{b}} \\ r \in \mc{D}_{\mf{a},\mf{b}}(c)}}\sum_{\ell \in \Z}\cchi(r)\psi\left(\frac{\Im(z)}{|cz+c\ell+r|^{2}}\right)e^{2\pi im\left(\frac{a}{c}-\frac{1}{c^{2}z+c^{2}\ell+cr}\right)} \\
        &= \sum_{\substack{c \in \mc{C}_{\mf{a},\mf{b}} \\ r \in \mc{D}_{\mf{a},\mf{b}}(c)}}\cchi(r)\sum_{\ell \in \Z}\psi\left(\frac{\Im(z)}{|cz+c\ell+r|^{2}}\right)e^{2\pi im\left(\frac{a}{c}-\frac{1}{c^{2}z+c^{2}\ell+cr}\right)},
      \end{align*}
     where on the right-hand side it is understood that we are summing over all triples $(c,\ell,r)$ with the prescribed properties and the second line holds since $\chi$ has conductor diving the level and $d \in \mc{D}_{\mf{a},\mf{b}}(c)$ is determined modulo $c$. Now let
      \[
        I_{c,r}(z,\psi) = \sum_{\ell \in \Z}\psi\left(\frac{\Im(z)}{|cz+c\ell+r|^{2}}\right)e^{2\pi im\left(\frac{a}{c}-\frac{1}{c^{2}z+c^{2}\ell+cr}\right)}.
      \]
      We apply the Poisson summation formula to $I_{c,r}(z,\psi)$. This is allowed since the summands are absolutely integrable by \cref{prop:decay_unbounded_inteval_integral}, as they exhibit polynomial decay of order $\s > 1$ because $\psi(y) \ll_{\e} y^{1+\e}$ as $y \to 0$, and $I_{c,r}(z,\psi)$ is holomorphic because $(P_{m,\chi,\mf{a}}|\s_{\mf{b}})(z,\psi)$ is. By the identity theorem it suffices to apply the Poisson summation formula for $z = iy$ with $y > 0$. So let $f(x)$ be given by
      \[
        f(x) = \psi\left(\frac{y}{|cx+r+icy|^{2}}\right)e^{2\pi im\left(\frac{a}{c}-\frac{1}{c^{2}x+cr+ic^{2}y}\right)}.
      \]
      As we have just noted, $f(x)$ is absolutely integrable on $\R$. We compute the Fourier transform:
      \[
        \hat{f}(t) = \int_{-\infty}^{\infty}f(x)e^{-2\pi itx}\,dx = \int_{-\infty}^{\infty}\psi\left(\frac{y}{|cx+r+icy|^{2}}\right)e^{2\pi im\left(\frac{a}{c}-\frac{1}{c^{2}x+cr+ic^{2}y}\right)}e^{-2\pi itx}\,dx.
      \]
      Complexify the integral to get
      \[
        \int_{\Im(z) = 0}\psi\left(\frac{y}{|cz+r+icy|^{2}}\right)e^{2\pi im\left(\frac{a}{c}-\frac{1}{c^{2}z+cr+ic^{2}y}\right)}e^{-2\pi itz}\,dz.
      \]
      Now make the change of variables $z \to z-\frac{r}{c}-iy$ to obtain
      \[
        e^{2\pi im\frac{a}{c}+2\pi it\frac{r}{c}-2\pi ty}\int_{\Im(z) = y}\psi\left(\frac{y}{|cz|^{2}}\right)e^{-\frac{2\pi im}{c^{2}z}}e^{-2\pi itz}\,dz.
      \]
      As the remaining integral is $\psi_{c}(y,m,t)$, it follows that
      \[
        \hat{f}(t) = \psi_{c}(y,m,t)e^{2\pi im\frac{a}{c}+2\pi it\frac{r}{c}-2\pi ty}.
      \]
      By the Poisson summation formula and the identity theorem, we have
      \[
        I_{c,r}(z,\psi) = \sum_{t \in \Z}(\psi_{c}(y,m,t)e^{2\pi im\frac{a}{c}+2\pi it\frac{r}{c}})e^{2\pi itz},
      \]
      for all $z \in \H$. Substituting this back into the Eisenstein series gives a form of the Fourier series:
      \begin{align*}
        (P_{m,\chi,\mf{a}}|\s_{\mf{b}})(z,\psi) &= \d_{\mf{a},\mf{{b}}}\psi(\Im(z))e^{2\pi imz}+\sum_{\substack{c \in \mc{C}_{\mf{a},\mf{b}} \\ r \in \mc{D}_{\mf{a},\mf{b}}}}\cchi(r)\sum_{t \in \Z}\psi_{c}(y,m,t)e^{2\pi im\frac{a}{c}+2\pi it\frac{r}{c}}e^{2\pi itz} \\
        &= \sum_{t \in \Z}\left(\d_{\mf{a},\mf{b}}\d_{m,t}\psi(\Im(z))+\sum_{\substack{c \in \mc{C}_{\mf{a},\mf{b}} \\ r \in \mc{D}_{\mf{a},\mf{b}}}}\cchi(r)\psi_{c}(y,m,t)e^{2\pi im\frac{a}{c}+2\pi it\frac{r}{c}}\right)e^{2\pi itz} \\
        &= \sum_{t \in \Z}\left(\d_{\mf{a},\mf{b}}\d_{m,t}\psi(\Im(z))+\sum_{\substack{c \in \mc{C}_{\mf{a},\mf{b}}}}\psi_{c}(y,m,t)\sum_{r \in \mc{D}_{\mf{a},\mf{b}}}\cchi(r)e^{2\pi im\frac{a}{c}+2\pi it\frac{r}{c}}\right)e^{2\pi itz}.
      \end{align*}
      We will simplify the innermost sum. Since $a$ is the inverse for $r$ modulo $c$, the innermost sum above becomes
      \[
        \sum_{r \in \mc{D}_{\mf{a},\mf{b}}}\cchi(r)e^{2\pi im\frac{a}{c}+2\pi it\frac{r}{c}} = \sum_{r \in \mc{D}_{\mf{a},\mf{b}}}\cchi(\conj{a})e^{2\pi im\frac{a}{c}+2\pi it\frac{\conj{a}}{c}} = \sum_{r \in \mc{D}_{\mf{a},\mf{b}}}\chi(a)e^{\frac{2\pi i(am+\conj{a}t)}{c}} = S_{\chi,\mf{a},\mf{b}}(m,t,c).
      \]
      So at last, we obtain our desired Fourier series:
      \[
        (P_{m,\chi,\mf{a}}|\s_{\mf{b}})(z) = \sum_{t \in \Z}\left(\d_{\mf{a},\mf{b}}\d_{m,t}\psi(\Im(z))+\sum_{\substack{c \in \mc{C}_{\mf{a},\mf{b}}}}\psi_{c}(y,m,t)S_{\chi,\mf{a},\mf{b}}(m,t,c)\right)e^{2\pi itz}.
      \]
    \end{proof}