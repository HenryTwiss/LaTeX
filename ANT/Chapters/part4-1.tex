\chapter{Classical Applications}
  \section{Dirichlet's Theorem on Primes in Arithmetic Progressions}
    One of the more well-known arithmetic results proved using $L$-functions is \textbf{Dirichlet's theorem on primes in arithmetic progressions}\index{Dirichlet's theorem on primes in arithmetic progressions}:

    \begin{theorem}[Dirichlet's theorem on primes in arithmetic progressions]\label{thm:Dirichlet's_theorem_on_primes_in_arithmetic_progressions}
      Let $a$ and $m$ be positive integers such that $(a,m) = 1$. Then the arithmetic progression $\{a+km \mid k \in \N\}$ contain infinitely many primes.
    \end{theorem}

    We will delay the proof for the moment, for it is well-worth understanding the some of the motivation behind why this theorem is interesting and how exactly Dirichlet used the analytic techniques of $L$-functions to attack this purely arithmetic statement. We being by recalling Euclid's famous theorem on the infinitude of the primes. Euclid's proof is completely elementary and arithmetic in nature. He argues that if there were finitely many primes $p_{1},p_{2},\ldots,p_{k}$ then a short consideration of $(p_{1}p_{2} \cdots p_{k})+1$ shows that this number must either be divisible by a prime not in our list or must be prime itself. As primes are the multiplicative building blocks of arithmetic, Euclid assures us that we have an ample amount of primes to work with. Now there is a slightly stronger result due to Euler (see \cite{euler1744variae}) requiring analytic techniques:

    \begin{theorem}\label{thm:reciprocial_sum_of_primes_diverges}
      The series
      \[
        \sum_{p}\frac{1}{p},
      \]
      diverges.
    \end{theorem}
    \begin{proof}
      For $\s > 1$, taking the logarithm of the Euler product of $\z(s)$, we get
      \[
        \log\z(s) = -\sum_{p}\log(1-p^{-s}).
      \]
      The Taylor series of the logarithm gives
      \[
        \log(1-p^{-s}) = \sum_{k \ge 1}(-1)^{k-1}\frac{(-p^{-s})^{k}}{k} = \sum_{k \ge 1}(-1)^{2k-1}\frac{1}{kp^{ks}},
      \]
      so that
      \[
        \log\z(s) = \sum_{p}\sum_{k \ge 1}\frac{1}{kp^{ks}}.
      \]
      The double sum restricted to $k \ge 2$ is uniformly bounded for $\s > 1$. To see this, first observe
      \[
        \sum_{k \ge 2}\frac{1}{kp^{ks}} \ll \sum_{k \ge 2}\frac{1}{p^{k}} = \frac{1}{p^{2}}\sum_{k \ge 0}\frac{1}{p^{k}} = \frac{1}{p^{2}}(1-p^{-1})^{-1} \le \frac{2}{p^{2}},
      \]
      where the last inequality follows because $p \ge 2$. Then
      \[
       \sum_{p}\sum_{k \ge 2}\frac{1}{kp^{ks}} \ll 2\sum_{p}\frac{1}{p^{2}} < 2\sum_{n \ge 1}\frac{1}{n^{2}} = 2\z(2).
      \]
      Therefore
      \[
        \log\z(s)-\sum_{p}\frac{1}{p^{s}} = \sum_{p}\sum_{k \ge 2}\frac{1}{kp^{ks}},
      \]
      and remains bounded as $s \to 1$. The claim now follows since $\z(s)$ has a simple pole at $s = 1$.
    \end{proof}

    \cref{thm:reciprocial_sum_of_primes_diverges} tells us that there are infinitely many primes but also that the primes are not too ``sparse'' in the integers for otherwise the series would converge. The idea Dirichlet used to prove his result on primes in arithmetic progressions was in a very similar spirit. He sought out to prove the divergence of the series
    \[
      \sum_{p \equiv a \tmod{m}}\frac{1}{p},
    \]
    for positive integers $a$ and $m$ with $(a,m) = 1$ as the divergence immediately implies there are infinitely many primes $p$ of the form $p \equiv a \tmod{m}$. In the case $a = 1$ and $m = 2$ we recover \cref{thm:reciprocial_sum_of_primes_diverges} exactly since every prime is odd. Dirichlet's proof proceeds in a similar way to that of \cref{thm:reciprocial_sum_of_primes_diverges} and this is where Dirichlet used what are now known as Dirichlet characters and Dirichlet $L$-functions. The proof can be broken into three steps. The first is to proceed as Euler did, but with the Dirichlet $L$-function $L(s,\chi)$ where $\chi$ has modulus $m$. That is, write $L(s,\chi)$ as a sum over primes and a bounded term as $s \to 1$. The next step is to use the orthogonality relations of the characters to sieve out the correct sum. The last step is to show the non-vanishing result $L(1,\chi) \neq 0$ for all non-principal characters $\chi$. This is the essential part of the proof as it is what assures us that the sum diverges. Luckily, we have done most of the hard work to prove this already:

    \begin{theorem}\label{thm:non-vanishing_of_Dirichlet_L-functions_at_s=1}
      Let $\chi$ be a non-principal Dirichlet character. Then $L(1,\chi)$ is finite and nonzero.
    \end{theorem}
    \begin{proof}
      This follows immediately by applying \cref{lem:non-vanshing_at_1_lemma} to the $L$-function $\z(s)L(s,\chi)$ and noting that $L(s,\chi)$ is holomorphic.
    \end{proof}

     We now prove Dirichlet's theorem on primes in arithmetic progressions:

    \begin{proof}[Proof of Dirichlet's theorem on primes in arithmetic progressions]
        Let $\chi$ be a Dirichlet character modulo $m$. Then for $\s > 1$, taking the logarithm of the Euler product of $L(s,\chi)$ gives
        \[
          \log L(s,\chi) = -\sum_{p}\log(1-\chi(p)p^{-s}).
        \]
        The Taylor series of the logarithm implies
        \[
          \log(1-\chi(p)p^{-s}) = \sum_{k \ge 1}(-1)^{k-1}\frac{(-\chi(p)p^{-s})^{k}}{k} = \sum_{k \ge 1}(-1)^{2k-1}\frac{\chi(p^{k})}{kp^{ks}},
        \]
        so that
        \[
          \log L(s,\chi) = \sum_{p}\sum_{k \ge 1}\frac{\chi(p^{k})}{kp^{ks}}.
        \]
        The double sum restricted to $k \ge 2$ is uniformly bounded for $\s > 1$. Indeed, first observe
        \[
          \left|\sum_{k \ge 2}\frac{\chi(p^{k})}{kp^{ks}}\right| \ll \sum_{k \ge 2}\frac{1}{p^{k}} = \frac{1}{p^{2}}\sum_{k \ge 0}\frac{1}{p^{k}} = \frac{1}{p^{2}}(1-p^{-1})^{-1} \le \frac{2}{p^{2}},
        \]
        where the last inequality follows because $p > 2$. Then
        \[
          \left|\sum_{p}\sum_{k \ge 2}\frac{\chi(p^{k})}{kp^{ks}}\right| \le 2\sum_{p}\frac{1}{p^{2}} < 2\sum_{n \ge 1}\frac{1}{n^{2}} = 2\z(2),
        \]
        as desired. Now write
        \[
          \sum_{\chi \tmod{m}}\conj{\chi(a)}\log L(s,\chi) = \sum_{\chi \tmod{m}}\sum_{p}\frac{\conj{\chi(a)}\chi(p)}{p^{s}}+\sum_{\chi \tmod{m}}\conj{\chi(a)}\sum_{p}\sum_{k \ge 2}\frac{\chi(p^{k})}{kp^{ks}}.
        \]
        By the orthogonality relations (\cref{prop:Dirichlet_orthogonality_relations} (ii)), we find that
        \[
          \sum_{\chi \tmod{m}}\sum_{p}\frac{\conj{\chi(a)}\chi(p)}{p^{s}} = \sum_{p}\frac1{p^{s}}\sum_{\chi \tmod{m}}\conj{\chi(a)}\chi(p) = \vphi(m)\sum_{p \equiv{a} \tmod{m}}\frac{1}{p^{s}},
        \]
        and so
        \[
          \sum_{\chi \tmod{m}}\conj{\chi(a)}\log L(s,\chi)-\sum_{\chi \tmod{m}}\conj{\chi(a)}\sum_{p}\sum_{k \ge 2}\frac{\chi(p^{k})}{kp^{ks}} = \vphi(m)\sum_{p \equiv{a} \tmod{m}}\frac{1}{p^{s}}.
        \]
        The triple sum is uniformly bounded for $\s > 1$ because the inner double sum is and there are finitely many Dirichlet characters modulo $m$. Therefore it suffices to show that the first sum on the left-hand side diverges as $s \to 1$. For $\chi = \chi_{m,0}$,
        \[
          L(s,\chi_{m,0}) = \z(s)\prod_{p \mid m}(1-p^{-s}),
        \]
        so the corresponding term in the sum is
        \[
          \conj{\chi_{m,0}}(a)\log L(s,\chi_{m,0}) = \log\z(s)+\sum_{p \mid m}\log(1-p^{-s}),
        \]
        which diverges as $s \to 1$ because $\z(s)$ has a simple pole at $s = 1$. We will be done if $\log L(s,\chi)$ remains bounded as $s \to 1$ for all $\chi \neq \chi_{m,0}$. So assume $\chi$ is non-principal. Then we must show $L(1,\chi)$ is finite and nonzero. This follows from \cref{thm:non-vanishing_of_Dirichlet_L-functions_at_s=1}.
    \end{proof}

    For primitive $\chi$ of conductor $q > 1$, we know from \cref{thm:non-vanishing_of_Dirichlet_L-functions_at_s=1} that $L(1,\chi)$ is finite and nonzero. It is interesting to know whether or not this value is computable in general. Indeed it is. The computation is fairly straightforward and only requires some basic properties of Gauss sums that we have already developed. The idea is to rewrite the character values $\chi(n)$ so that we can collapse the infinite series into a Taylor series. Our result is the following:
    
    \begin{theorem}\label{thm:Value_of_Dirichlet_L-functions_at_s=1}
      Let $\chi$ be a primitive Dirichlet character with conductor $q > 1$. Then
      \[
        L(1,\chi) = -\frac{\chi(-1)\tau(\chi)}{q}\psum_{a \tmod{q}}\chi(a)\log\left(\sin\left(\frac{\pi a}{q}\right)\right) \quad \text{or} \quad L(1,\chi) = -\frac{\chi(-1)\tau(\chi)\pi i}{q^{2}}\psum_{a \tmod{q}}\chi(a)a,
      \]
      according to whether $\chi$ is even or odd.
    \end{theorem}
    \begin{proof}
      Make the following computation:
      \begin{align*}
        \chi(n) &= \frac{1}{\conj{\tau(\chi)}}\conj{\tau(n,\chi)} && \text{\cref{cor:gauss_sum_primitive_formula}} \\
        &= \frac{1}{\tau(\cchi)}\tau(n,\chi) && \text{\cref{prop:Gauss_sum_reduction} (i)} \\
        &= \frac{\tau(\chi)}{\tau(\chi)\tau(\cchi)}\tau(n,\chi) \\
        &= \frac{\chi(-1)\tau(\chi)}{q}\tau(n,\chi) && \text{\cref{thm:Gauss_sum_modulus,prop:epsilon_factor_relationship}} \\
        &= \frac{\chi(-1)\tau(\chi)}{q}\psum_{a \tmod{q}}\chi(a)e^{\frac{2\pi ian}{q}}.
      \end{align*}
      Substituting this result into the definition of $L(1,\chi)$, we find that
      \begin{equation}\label{equ:value_of_Dirichlet_L-functions_at_s=1_1}
        \begin{aligned}
          L(1,\chi) &= \sum_{n \ge 1}\frac{1}{n}\left(\frac{\chi(-1)\tau(\chi)}{q}\psum_{a \tmod{q}}\chi(a)e^{\frac{2\pi ian}{q}}\right) \\
          &= \frac{\chi(-1)\tau(\chi)}{q}\psum_{a \tmod{q}}\chi(a)\sum_{n \ge 1}\frac{e^{\frac{2\pi ian}{q}}}{n} \\
          &= \frac{\chi(-1)\tau(\chi)}{q}\psum_{a \tmod{q}}\chi(a)\log\left(\left(1-e^{\frac{2\pi ia}{q}}\right)^{-1}\right),
        \end{aligned}
      \end{equation}
      where in the last line we have used the Taylor series of the logarithm (notice $a \neq q$ so that $e^{\frac{2\pi ia}{q}} \neq 1$ and hence the logarithm is defined). We have now expressed $L(1,\chi)$ as a finite sum. In order to simplify the last expression in \cref{equ:value_of_Dirichlet_L-functions_at_s=1_1}, we deal with the logarithm. Since $\sin(\t) = \frac{e^{i\t}-e^{-i\t}}{2i}$, we have
      \[
        1-e^{\frac{2\pi ia}{q}} = -2ie^{\frac{\pi ia}{q}}\left(\frac{e^{\frac{\pi ia}{q}}-e^{-\frac{\pi ia}{q}}}{2i}\right) = -2ie^{\frac{\pi ia}{q}}\sin\left(\frac{\pi a}{q}\right).
      \]
      Therefore the last expression in \cref{equ:value_of_Dirichlet_L-functions_at_s=1_1} becomes
      \[
        \frac{\chi(-1)\tau(\chi)}{q}\psum_{a \tmod{q}}\chi(a)\log\left(\left(-2ie^{\frac{\pi ia}{q}}\sin\left(\frac{\pi a}{q}\right)\right)^{-1}\right).
      \]
      As $0< a < q$, we have $0 < \frac{\pi a}{q} < \pi$ so that $\sin\left(\frac{\pi a}{q}\right)$ is never negative. Therefore we can split up the logarithm term and obtain
      \[
        -\frac{\chi(-1)\tau(\chi)}{q}\left(\log(-2i)\psum_{a \tmod{q}}\chi(a)+\frac{\pi i}{q}\psum_{a \tmod{q}}\chi(a)a+\psum_{a \tmod{q}}\chi(a)\log\left(\sin\left(\frac{\pi a}{q}\right)\right)\right).
      \]
      By the orthogonality relations (\cref{cor:Dirichlet_orthogonality_relations} (i)), the first sum above vanishes. Therefore
      \begin{equation}\label{equ:value_of_Dirichlet_L-functions_at_s=1_2}
        L(1,\chi) = -\frac{\chi(-1)\tau(\chi)}{q}\left(\frac{\pi i}{q}\psum_{a \tmod{q}}\chi(a)a+\psum_{a \tmod{q}}\chi(a)\log\left(\sin\left(\frac{\pi a}{q}\right)\right)\right).
      \end{equation}
      \cref{equ:value_of_Dirichlet_L-functions_at_s=1_2} simplifies in that one of the two sums vanish depending on if $\chi$ is even or odd. For the first sum in \cref{equ:value_of_Dirichlet_L-functions_at_s=1_2}, observe that
      \[
        \frac{\pi i}{q}\psum_{a \tmod{q}}\chi(a)a = -\frac{\chi(-1)\pi i}{q}\psum_{a \tmod{q}}\chi(-a)(-a),
      \]
      which vanishes if $\chi$ is even. For the second sum in \cref{equ:value_of_Dirichlet_L-functions_at_s=1_2}, we have an analogous relation of the form
      \[
        \psum_{a \tmod{q}}\chi(a)\log\left(\sin\left(\frac{\pi a}{q}\right)\right) = \chi(-1)\psum_{a \tmod{q}}\chi(-a)\log\left(\sin\left(\frac{\pi a}{q}\right)\right),
      \]
      which vanishes if $\chi$ is odd. This finishes the proof.
    \end{proof}

    \cref{thm:Value_of_Dirichlet_L-functions_at_s=1} encodes some interesting identities. For example, if $\chi$ is the non-principal Dirichlet character modulo $4$, then $\chi$ is uniquely defined by $\chi(1) = 1$ and $\chi(3) = \chi(-1) = -1$. In particular, $\chi$ is odd and its conductor is $4$. Now
    \[
      \tau(\chi) = \psum_{a \tmod{4}}\chi(a)e^{\frac{2\pi ia}{4}} = e^{\frac{2\pi i}{4}}-e^{\frac{6\pi i}{4}} = i-(-i) = 2i,
    \]
    so by \cref{thm:Value_of_Dirichlet_L-functions_at_s=1} we get
    \[
      L(1,\chi) = -\frac{\chi(-1)\tau(\chi)\pi i}{16}(1-3) = \frac{\pi}{4}.
    \]
    Expanding out $L(1,\chi)$ gives
    \[
      1-\frac{1}{3}+\frac{1}{5}-\frac{1}{7}+\cdots = \frac{\pi}{4},
    \]
    which is the famous \textbf{Madhava–Leibniz formula}\index{Madhava–Leibniz formula} for $\pi$.
  \section{Siegel's Theorem}
    The discussion of Siegel zeros first arose during the study of zero-free regions for Dirichlet $L$-functions. Refining the argument used in \cref{thm:zero_free_region_generic}, we can show that Siegel zeros only exist when the character $\chi$ is quadratic. But first we improve the zero-free region for the Riemann zeta function:

    \begin{theorem}\label{thm:improved_zero-free_region_zeta}
      There exists a constant $c > 0$ such that $\z(s)$ has no zeros in the region
      \[
        \s \ge 1-\frac{c}{\log(|t|+3)}.
      \]
    \end{theorem}
    \begin{proof}
      By \cref{thm:zero_free_region_generic} applied to $\z(s)$, it suffices to show that $\z(s)$ has no real nontrivial zeros. To see this, let $\eta(s)$ be defined by
      \[
        \eta(s) = \sum_{n \ge 1}\frac{(-1)^{n-1}}{n^{s}}.
      \]
      Note that $\eta(s)$ converges for $\s > 0$ by \cref{prop:Dirichlet_series_convergence_bounded_coefficient_sum}. Now for $0 < s < 1$ and even $n$, $\frac{1}{n^{s}}-\frac{1}{(n+1)^{s}} > 0$ so that $\eta(s) > 0$. But for $\s > 0$, we have
      \[
        (1-2^{1-s})\z(s) = \sum_{n \ge 1}\frac{1}{n^{s}}-2\sum_{n \ge 1}\frac{1}{(2n)^{s}} = \sum_{n \ge 1}\frac{(-1)^{n-1}}{n^{s}} = \eta(s).
      \]
      Therefore $\z(s)$ cannot admit a zero for $0 < s < 1$ because then $\eta(s)$ would be zero too. This completes the proof.
    \end{proof}

    \cref{thm:improved_zero-free_region_zeta} shows that $\z(s)$ has no Siegel zeros. Moreover, since $1-2^{1-s} < 0$ for $0 < s < 1$, the proof shows that $\z(s) < 0$ in this interval as well. As for the height of the first zero, it occurs on the critical line (as predicted by the Riemann hypothesis) at height $t \approx 14.134$ (see \cite{davenport1980multiplicative} for a further discussion). The first $15$ zeros were computed by Gram in 1903 (see \cite{gram1903note}). Since then, billions of zeros have been computed and have all been verified to lie on the critical line. The analogous situation for Dirichlet $L$-functions is only slightly different but causes increasing complexity in further study. We first show that if a Siegel zero exists for the Dirichlet $L$-function of a primitive character, then the character is necessarily quadratic:

    \begin{theorem}
      Let $\chi$ be a primitive Dirichlet character of conductor $q > 1$. Then there exists a constant $c > 0$ such that $L(s,\chi)$ has no zeros in the region
      \[
        \s \ge 1-\frac{c}{\log(q(|t|+3))},
      \]
      except for possibly one simple real zero $\b_{\chi}$ with $\b_{chi} < 1$ in the case $\chi$ is quadratic.
    \end{theorem}
    \begin{proof}
      By \cref{thm:zero_free_region_generic} applied to $\z(s)L(s,\chi)$, and shrinking $c$ if necessary, it remains to show that there not a simple real zero $\b_{\chi}$ if $\chi$ is not quadratic. For this, let $L(s,g)$ be the $L$-function defined by
      \[
        L(s,g) = L^{3}(s,\chi_{q,0})L^{4}(s,\chi)L(s,\chi^{2}).
      \]
      We have $d_{g} = 8$ and $\mathfrak{q}(g)$ satisfies
      \[
        \mathfrak{q}(g) \le \mathfrak{q}(\chi_{q,0})^{3}\mathfrak{q}(\chi)^{4}\mathfrak{q}(\chi^{2}) \le q^{8}3^{5} < (3q)^{8}.
      \]
      Moreover, $\Re(\L_{g}(n)) \ge 0$ for $(n,q) = 1$. To see this, suppose $p$ is an unramified prime. The local roots of $L(s,g)$ at $p$ are $1$ with multiplicity three, $\chi(p)$ with multiplicity four, and $\chi^{2}(p)$ with multiplicity one. So for any $k \ge 1$, the sum of $k$-th powers of these local roots is
      \[
        3+4\chi^{k}(p)+\chi^{2k}(p).
      \]
      Writing $\chi(p) = e^{i\t}$, the real part of this expression is
      \[
        3+4\cos(\t)+\cos(2\t) = 2(1+\cos(\t))^{2} \ge 0,
      \]
      where we have also made use of the identity $3+4\cos(\t)+\cos(2\t) = 2(1+\cos(\t))^{2}$. Thus $\Re(\L_{g}(n)) \ge 0$ for $(n,q) = 1$, and the conditions of \cref{lem:non-vanshing_at_1_lemma} are satisfied for $L(s,g)$ (recall \cref{equ:non-primitive_primitive_Dirichlet_L-series_relation} for the $L$-functions $L(s,\chi_{q,0})$ and $L(s,\chi^{2})$). On the one hand, if $\b$ be a real nontrivial zero of $L(s,\chi)$ then $L(s,g)$ has a real nontrivial zero at $s = \b$ of order at least $4$. On the other hand, using \cref{equ:non-primitive_primitive_Dirichlet_L-series_relation} and that $\chi^{2} \neq \chi_{q,0}$, $L(s,g)$ has a pole at $s = 1$ of order $3$. Then, upon shrinking $c$ if necessary, \cref{lem:non-vanshing_at_1_lemma} gives a contradiction since $r_{g} = 3$. This completes the proof.
    \end{proof}


    Siegel zeros present an unfortunate obstruction to zero-free region results for Dirichlet $L$-functions when the primitive character $\chi$ is quadratic. However, if we no longer require the constant $c$ in the zero-free region to be effective, we can obtain a much better result for how close the Siegel zero can be to $1$. Ultimately, this improved bound results from a lower bound for $L(1,\chi)$ (recall that this is nonzero from our discussion about Dirichlet's theorem on primes in arithmetic progressions). \textbf{Siegel's theorem}\index{Siegel's theorem} refers to either this lower bound or to the improved zero-free region. In the lower bound version, Siegel's theorem is the following:

    \begin{theorem}[Siegel's theorem, lower bound version]
      Let $\chi$ be a primitive quadratic Dirichlet character modulo $q > 1$. Then there exists a positive constant $c_{1}(\e)$ such that
      \[
        L(1,\chi) \ge \frac{c_{1}(\e)}{q^{\e}}.
      \]
    \end{theorem}

    In the zero-free region version, Siegel's theorem takes the following form:

    \begin{theorem}[Siegel's theorem, zero-free region version]
      Let $\chi$ be a primitive quadratic Dirichlet character modulo $q > 1$. Then there exists a positive constant $c_{2}(\e)$ such that $L(s,\chi)$ has no real zeros in the segment
      \[
        \s \ge 1-\frac{c_{2}(\e)}{q^{\e}}.
      \]
    \end{theorem}

    The largest defect of Siegel's theorem, in either version, is that the implicit constants $c_{1}(\e)$ and $c_{2}(\e)$ are ineffective (and not necessarily equal). Actually, the lower bound result is slightly stronger as it implies the zero-free region result. We will first prove the zero-free region given the lower bound, and then we will prove the lower bound. Before we begin, we need two small lemmas about the size of $L'(\s,\chi)$ and $L(\s,\chi)$ for $\s$ close to $1$:

    \begin{lemma}\label{lem:Siegels_theorem_second_version_lemma}
      Let $\chi$ be a non-principal Dirichlet character modulo $m > 1$. Then $L'(\s,\chi) = O(\log^{2}(m))$ for any $\s$ such that $0 \le 1-\s \le \frac{1}{\log(m)}$.
    \end{lemma}
    \begin{proof}
      Setting $A(X) = \sum_{n \le X}\chi(n)$ we have $A(X) \ll 1$ by \cref{cor:Dirichlet_orthogonality_relations} (i) and that $\chi$ is periodic. Therefore $\s_{c} \le 0$ by \cref{prop:Dirichlet_series_convergence_bounded_coefficient_sum}. Hence for $\s$ in the prescribed region, $L(\s,\chi)$ is holomorphic and its derivative is given by
      \[
        L'(\s,\chi) = \sum_{n \ge 1}\frac{\chi(n)\log(n)}{n^{\s}} = \sum_{n < m}\frac{\chi(n)\log(n)}{n^{\s}}+\sum_{n \ge m}\frac{\chi(n)\log(n)}{n^{\s}}.
      \]
      We will show that the last two sums are both $O(\log^{2}(m))$. For the first sum, if $n < m$, we have
      \[
        \left|\frac{\chi(n)\log(n)}{n^{\s}}\right| \le \frac{1}{n^{\s}}\log(n) = \frac{n^{1-\s}}{n}\log(n) < \frac{m^{1-\s}}{n}\log(n) < \frac{e}{n}\log(m),
      \]
      where the last inequality follows because $1-\s \le \frac{1}{\log(m)}$. Then
      \[
        \left|\sum_{n \le m}\frac{\chi(n)\log(n)}{n^{s}}\right| < e\log(m)\sum_{n < m}\frac{1}{n} < e\log(m)\int_{1}^{m}\frac{1}{n}\,dn \ll \log^{2}(m).
      \]
      For the second sum, $A(Y) \ll 1$ so that $A(Y)\log(Y)Y^{-\s} \to 0$ as $Y \to \infty$. Then Abel's summation formula (see \cref{append:Summation_Formulas}) gives
      \begin{equation}\label{equ:Siegels_theorem_second_version_lemma_1}
        \sum_{n \ge m}\frac{\chi(n)\log(n)}{n^{\s}} = -A(m)\log(m)m^{-\s}-\int_{m}^{\infty}A(u)(1-\s\log(u))u^{-(\s+1)}\,du.
      \end{equation}
      Since $0 \le 1-\s \le \frac{1}{\log(m)}$, we have $1-\s\log(u) \le \frac{\log(u)}{\log(m)}$. Also, we have the more precise estimate $|A(X)| \le m$ because $\chi$ is $m$-periodic and $|\chi(n)| \le 1$. With these estimates and \cref{equ:Siegels_theorem_second_version_lemma_1} we make the following computation:
      \begin{align*}
        \left|\sum_{n \ge m}\frac{\chi(n)\log(n)}{n^{\s}}\right| &\le |A(m)|\log(m)m^{-\s}+\int_{m}^{\infty}|A(u)|(1-\s\log(u))u^{-(\s+1)}\,du \\
        &\le |A(m)|\log(m)m^{-\s}+\log(m)\int_{m}^{\infty}|A(u)|\log(u)u^{-(\s+1)}\,du \\
        &\le m^{1-\s}\log(m)+m\int_{m}^{\infty}\log(u)u^{-(\s+1)}\,du \\
        &= m^{1-\s}\log(m)+m\left(-\log(u)\frac{u^{-\s}}{\s}\bigg|_{m}^{\infty}+\int_{m}^{\infty}\frac{u^{-(\s+1)}}{s}\,du\right) \\
        &= m^{1-\s}\log(m)+m\left(-\log(u)\frac{u^{-\s}}{\s}-\frac{u^{-\s}}{\s^{2}}\right)\bigg|_{m}^{\infty} \\
        &= m^{1-\s}\log(m)+m\left(\log(m)\frac{m^{-\s}}{\s}+\frac{m^{-\s}}{\s^{2}}\right) \\
        &\ll m^{1-\s}\log(m) \\
        &\ll e\log(m),
      \end{align*}
      where in the fourth line we have used integration by parts and the last line holds because $1-\s \le \frac{1}{\log(m)}$. But $e\log(m) = O(\log^{2}(m))$ so the second sum is also $O(\log^{2}(m))$. Therefore we have shown $L'(\s,\chi) = O(\log^{2}(m))$ finishing the proof. 
    \end{proof}

    The second lemma is even easier and is proved in exactly the same way:

    \begin{lemma}\label{lem:Siegels_theorem_first_version_lemma}
      Let $\chi$ be a non-principal Dirichlet character modulo $m > 1$. Then $L(\s,\chi) = O(\log(m))$ for any $\s$ such that $0 \le 1-\s \le \frac{1}{\log(m)}$.
    \end{lemma}
    \begin{proof}
      Note that
      \[
        L(\s,\chi) = \sum_{n \ge 1}\frac{\chi(n)}{n^{\s}} = \sum_{n < m}\frac{\chi(n)}{n^{\s}}+\sum_{n \ge m}\frac{\chi(n)}{n^{\s}}.
      \]
      It suffices to show that the last two sums are both $O(\log^{2}(m))$. For the first sum, since $n < m$, we have
      \[
        \left|\frac{\chi(n)}{n^{\s}}\right| \le \frac{1}{n^{\s}} = \frac{n^{1-\s}}{n} < \frac{m^{1-\s}}{n} < \frac{e}{n},
      \]
      where the last inequality follows because $1-\s \le \frac{1}{\log(m)}$. Therefore
      \[
        \left|\sum_{n \le m}\frac{\chi(n)}{n^{s}}\right| < e\sum_{n < m}\frac{1}{n} < e\log(m)\int_{1}^{m}\frac{1}{n}\,dn \ll \log(m).
      \]
      As for the second sum, setting $A(Y) = \sum_{n \le Y}\chi(n)$ we have $A(Y) \ll 1$ by \cref{cor:Dirichlet_orthogonality_relations} (i) and that $\chi$ is periodic. Thus $A(Y)Y^{-\s} \to 0$ as $Y \to \infty$. Then Abel's summation formula (see \cref{append:Summation_Formulas}) gives
      \begin{equation}\label{equ:Siegels_theorem_first_version_lemma_1}
        \sum_{n \ge m}\frac{\chi(n)}{n^{\s}} = -A(m)m^{-\s}-\int_{m}^{\infty}A(u)u^{-(\s+1)}\,du.
      \end{equation}
      Using the more precise estimate $|A(X)| \le m$, because $\chi$ is $m$-periodic and $|\chi(n)| \le 1$, with \cref{equ:Siegels_theorem_first_version_lemma_1}, we make the following computation:
      \begin{align*}
        \left|\sum_{n \ge m}\frac{\chi(n)\log(n)}{n^{\s}}\right| &\le |A(m)|m^{-\s}+\int_{m}^{\infty}|A(u)|u^{-(\s+1)}\,du \\
        &\le |A(m)|m^{-\s}+\int_{m}^{\infty}|A(u)|\log(u)u^{-(\s+1)}\,du \\
        &\le m^{1-\s}+m\int_{m}^{\infty}\log(u)u^{-(\s+1)}\,du \\
        &= m^{1-\s}+m\left(-\log(u)\frac{u^{-\s}}{\s}\bigg|_{m}^{\infty}+\int_{m}^{\infty}\frac{u^{-(\s+1)}}{s}\,du\right) \\
        &= m^{1-\s}+m\left(-\log(u)\frac{u^{-\s}}{\s}-\frac{u^{-\s}}{\s^{2}}\right)\bigg|_{m}^{\infty} \\
        &= m^{1-\s}+m\left(\log(m)\frac{m^{-\s}}{\s}+\frac{m^{-\s}}{\s^{2}}\right) \\
        &\ll m^{1-\s} \\
        &\ll e,
      \end{align*}
      where in the fourth line we have used integration by parts and the last line holds because $1-\s \le \frac{1}{\log(m)}$. But $e = O(\log^{2}(m))$ so the second sum is also $O(\log^{2}(m))$. Therefore we have shown $L(\s,\chi) = O(\log(m))$ which completes the proof.
    \end{proof}

    We will now prove the zero-free region version of Siegel's theorem, assuming the lower bound version, and using \cref{lem:Siegels_theorem_second_version_lemma}:

    \begin{proof}[Proof of Siegel's theorem, zero-free region version]
      We we will prove the theorem by contradiction. Clearly the result holds for a single $q$, and notice that the result also holds provided we bound $q$ from above by taking the maximum of all the $c_{2}(\e)$. Therefore we may suppose $q$ is arbitrarily large. In this case, if there was a real zero $\b$ with $\b \ge 1-\frac{c_{2}(\e)}{q^{\e}}$, equivalently $1-\b \le \frac{c_{2}(\e)}{q^{\e}}$, then for large enough $q$ we have $0 \le 1-\b \le \frac{1}{\log(q)}$ so that $L'(\s,\chi) = O(\log^{2}(q))$ for $\b \le \s \le 1$ by \cref{lem:Siegels_theorem_second_version_lemma}. These two estimates and the mean value theorem together give
      \[
        L(1,\chi) = L(1,\chi)-L(\b,\chi) = L'(\s,\chi)(1-\b) \ll \frac{\log^{2}(q)}{q^{\e}}.
      \]
      Upon taking $\frac{\e}{2}$ in the lower bound version of Siegel's theorem, we obtain
      \[
        \frac{1}{q^{\frac{\e}{2}}} \ll L(1,\chi) \ll \frac{\log^{2}(q)}{q^{\e}},
      \]
      which is a contradiction for large $q$.
    \end{proof}

    It remains to prove the lower bound version of Siegel's theorem. The idea is to combine two Dirichlet $L$-functions attached to distinct characters with distinct moduli and use this new $L$-function to derivative a lower bound for a single Dirichlet $L$-function at $s = 1$:

    \begin{proof}[Proof of Siegel's theorem, lower bound version]
      Let $\chi_{1}$ and $\chi_{2}$ be two distinct primitive quadratic and non-principal characters modulo $q_{1}$ and $q_{2}$ respectively. Let $L(s,g)$ be the $L$-function defined by
      \[
        L(s,g) = \z(s)L(s,\chi_{1})L(s,\chi_{2})L(s,\chi_{1}\chi_{2}).
      \]
      The key ingredient in the proof is a lower bound for $L(s,g)$ relative to the modulus $q_{1}q_{2}$ in a small interval on the real axis close to $1$. We will first deduce this estimate from which the rest of the proof follows easily. Observe that $L(s,g)$ is holomorphic on $\C$ except for a simple pole at $s = 1$. Let $\l$ be the residue at this pole so that
      \[
        \l = L(1,\chi_{1})L(1,\chi_{2})L(1,\chi_{1}\chi_{2}).
      \]
      We claim that the coefficients $a_{g}(n)$ of $L(s,g)$ are nonnegative. To see this, for any prime $p$, the local roots at $p$ are $1$ with multiplicity one, $\chi_{1}(p)$ with multiplicity one, $\chi_{2}(p)$ with multiplicity one, and $\chi_{1}\chi_{2}(p)$ with multiplicity one. So for any $k \ge 1$, the sum of $k$-th powers of these local roots is
      \[
        (1+\chi_{1}^{k}(p))(1+\chi_{2}^{k}(p)) \ge 0.
      \]
      Thus $\L_{g}(n) \ge 0$ for all $n \ge 1$ and hence $a_{g}(n) \ge 0$ for all $n \ge 1$ too. Moreover, $a(0) = 1$ which can be seen by expanding out the Dirichlet series defining $L(s,g)$. Now $L(s,g)$ is represented as an absolutely convergent series for $\s > 1$ so that it has a power series expansion about $s = 2$ with radius $1$:
      \[
        L(s,g) = \sum_{m \ge 0}\frac{L^{(m)}(2,g)}{m!}(s-2)^{m},
      \]
      for $|s-2| < 1$ (we are abusing notation with $L^{(m)}$). We can compute $L^{(m)}(2,g)$ using the Dirichlet series by differentiating termwise:
      \begin{equation}\label{equ:Siegels_theorem_first_version_1}
        L^{(m)}(2,g) = \frac{d^{m}}{ds^{m}}\left(\sum_{n \ge 1}\frac{a(n)}{n^{s}}\right)\Bigg|_{s = 2} = (-1)^{m}\sum_{n \ge 1}\frac{a(n)\log^{m}(n)}{n^{s}}\Bigg|_{s = 2} = (-1)^{m}\sum_{n \ge 1}\frac{a(n)\log^{m}(n)}{n^{2}}.
      \end{equation}
      Since the $a(n)$ are nonnegative, as just mentioned, it follows that $L^{(m)}(2,g)$ is nonnegative and therefore we may write
      \[
        L(s,g) = \sum_{m \ge 0}b(m)(2-s)^{m},
      \]
      for $|s-2| < 1$ and with $b(m)$ nonnegative. Also, \cref{equ:Siegels_theorem_first_version_1} and the fact that the $a(n)$ are nonnegative with $a(0) = 1$ together imply that $b(0) > 1$. Then
      \begin{equation}\label{equ:Siegels_theorem_first_version_2}
        L(s,g)-\frac{\l}{s-1} = L(s,g)-\l\sum_{m \ge 0}(2-s)^{m} = \sum_{m \ge 0}(b(m)-\l)(2-s)^{m},
      \end{equation}
      and the last series must be absolutely convergent for say $|s-2| < 2$ because $L(s,g)-\frac{\l}{s-1}$ is entire as we have removed the pole at $s = 1$. We now wish to estimate $L(s,g)$ and $\frac{\l}{s-1}$ on the circle $|s-2| = \frac{3}{2}$. Let $\chi$ be a non-principal Dirichlet character modulo $m$ and let $A(X) = \sum_{n \le X}\chi(n)$. Then Abel's summation formula and that $A(X) \ll 1$ (by \cref{cor:Dirichlet_orthogonality_relations} (i) and that $\chi$ is periodic) together imply
      \[
        L(s,\chi) = s\int_{1}^{\infty}A(u)u^{-(s+1)}\,du,
      \]
      for $\s > 0$. Now suppose $\s \ge \frac{1}{2}$. As $|A(X)| \le m$, we obtain
      \[
        |L(s,\chi)| \le m|s|\int_{1}^{\infty}u^{-(\s+1)}\,du = -m|s|\frac{u^{-\s}}{\s}\bigg|_{1}^{\infty} = \frac{m|s|}{\s} \le 2m|s|.
      \]
      In particular, on the disk $|s-2| \le \frac{3}{2}$ we have the estimates
      \[
        L(s,\chi_{1}) \ll q_{1}, \quad L(s,\chi_{2}) \ll q_{2}, \quad \text{and} \quad L(s,\chi_{1}\chi_{2}) \ll q_{1}q_{2}.
      \]
      Since $\z(s)$ is bounded on the circle $|s-2| = \frac{3}{2}$ (it's a compact set) and $\l = L(1,\chi_{1})L(1,\chi_{2})L(1,\chi_{1}\chi_{2})$, we obtain the bounds
      \[
        L(s,g) \ll q_{1}^{2}q_{2}^{2} \quad \text{and} \quad \frac{\l}{s-1} \ll q_{1}^{2}q_{2}^{2},
      \]
      on this circle as well. Cauchy's inequality for the size of coefficients of a power series applied to \cref{equ:Siegels_theorem_first_version_2} on the circle $|s-2| = \frac{3}{2}$ gives
      \begin{equation}\label{equ:Siegels_theorem_first_version_3}
        b(m)-\l \ll q_{1}^{2}q_{2}^{2}\left(\frac{2}{3}\right)^{m}.
      \end{equation}
      Let $M$ be a positive integer. For real $s$ with $\frac{7}{8} < s < 1$ we have $2-s < \frac{9}{8}$ and using \cref{equ:Siegels_theorem_first_version_2,equ:Siegels_theorem_first_version_3} together we can upper bound the tail of $L(s,g)-\frac{\l}{s-1}$:
      \begin{align*}
        \left|\sum_{m \ge M}(b(m)-\l)(2-s)^{m}\right| &\le \sum_{m \ge M}|b(m)-\l|(2-s)^{m} \\
        &\ll q_{1}^{2}q_{2}^{2}\sum_{m \ge M}\left(\frac{2}{3}(2-s)\right)^{m} \\
        &\ll q_{1}^{2}q_{2}^{2}\sum_{m \ge M}\left(\frac{3}{4}\right)^{m} \\
        &\ll q_{1}^{2}q_{2}^{2}\left(\frac{3}{4}\right)^{M} \\
        &\ll q_{1}^{2}q_{2}^{2}e^{-\frac{M}{4}},
      \end{align*}
      where the last estimate follows because $(\frac{3}{4})^{M} < e^{-\frac{M}{4}}$ (which is equivalent to $\log\left(\frac{3}{4}\right) < -\frac{1}{4}$). Let $c$ be the implicit constant. Using that the $b(m)$ are nonnegative, $b(0) > 1$, and the previous estimate for the tail, we can estimate $L(s,g)-\frac{\l}{s-1}$ from below for $\frac{7}{8} < s < 1$. Indeed, throwing away the $b(m)$ terms for $1 \le m \le M$, bounding the constant term below by $1$, and use the tail estimate gives
      \begin{equation}\label{equ:Siegels_theorem_first_version_4}
        L(s,g)-\frac{\l}{s-1} \ge 1-\l\sum_{0 \le m \le M-1}(2-s)^{m}-cq_{1}^{2}q_{2}^{2}e^{-\frac{M}{4}} = 1-\l\frac{(2-s)^{M}-1}{1-s}-cq_{1}^{2}q_{2}^{2}e^{-\frac{M}{4}},
      \end{equation}
      which is valid for any positive integer $M$. Now chose $M$ to be a positive integer such that
      \begin{equation}\label{equ:Siegels_theorem_first_version_5}
        \frac{1}{2}e^{-\frac{1}{4}} \le cq_{1}^{2}q_{2}^{2}e^{-\frac{M}{4}} < \frac{1}{2}.
      \end{equation}
      Upon isolating $L(s,g)$ in \cref{equ:Siegels_theorem_first_version_4} and using the second estimate in \cref{equ:Siegels_theorem_first_version_5}, we get
      \begin{equation}\label{equ:Siegels_theorem_first_version_6}
        L(s,g) \ge \frac{1}{2}-\l\frac{(2-s)^{M}}{1-s}.
      \end{equation}
      Taking the logarithm of the first estimate in \cref{equ:Siegels_theorem_first_version_5} and isolating $M$, we obtain
      \begin{equation}\label{equ:Siegels_theorem_first_version_7}
        M \le 8\log(q_{1}q_{2})+c,
      \end{equation}
      for some different constant $c$. It follows that
      \begin{equation}\label{equ:Siegels_theorem_first_version_8}
        (2-s)^{M} = e^{M\log(2-s)} < e^{M(1-s)} \le c(q_{1}q_{2})^{8(1-s)},
      \end{equation}
      for some different constant $c$, where in the first estimate we have used the Taylor series of the logarithm truncated at the first term and in the second estimate we have used \cref{equ:Siegels_theorem_first_version_7}. Since $1-s$ is positive for $\frac{7}{8} < s < 1$, we can combine \cref{equ:Siegels_theorem_first_version_6,equ:Siegels_theorem_first_version_8} which gives
      \begin{equation}\label{equ:Siegels_theorem_first_version_9}
        L(s,g) \ge \frac{1}{2}-\l\frac{c}{1-s}(q_{1}q_{2})^{8(1-s)}.
      \end{equation}
      This is our desired lower bound for $L(s,g)$. We will now choose the character $\chi_{1}$. If there exists a Siegel zero $\b_{1}$ with $1-\frac{\e}{16} < \b_{1} < 1$, let $\chi_{1}$ be the character corresponding to the Dirichlet $L$-function that admits this Siegel zero. Then $F(\b_{1}) = 0$ independent of the choice of $\chi_{2}$. If there is no such Siegel zero, choose $\chi_{1}$ to be any quadratic primitive character and $\b_{1}$ to be any number such that $1-\frac{\e}{16} < \b_{1} < 1$. Then $F(\b_{1}) < 0$ independent of the choice of $\chi_{2}$. Indeed, $\z(s)$ is negative in this segment (actually for $0 \le s < 1$) and each of the Dirichlet $L$-function defining $L(s,g)$ is positive at $s = 1$ (the Euler product implies Dirichlet $L$-function are positive for $s > 1$ and they are in fact nonzero for $s = 1$ by \cref{thm:non-vanishing_of_Dirichlet_L-functions_at_s=1}) and do not admit a zero for $\b_{1} < s \le 1$ by our choice of $\b_{1}$. In either case, $F(\b_{1}) \le 0$ so isolating $\l$ and disregarding the constants in \cref{equ:Siegels_theorem_first_version_9} with $s = \b_{1}$ gives the weaker estimate
      \begin{equation}\label{equ:Siegels_theorem_first_version_10}
        \l \gg \frac{1-\b_{1}}{(q_{1}q_{2})^{8(1-\b_{1})}}.
      \end{equation}
      We will now choose $\chi_{2} = \chi$ and hence $q_{2} = q$ as in the statement of the theorem. Notice that, independent of any work we have done, the theorem holds for a single $q$. Moreover, the theorem holds provided we bound $q$ from above by taking the minimum of the $c_{1}(\e)$. Therefore we may assume $q$ is arbitrarily large and in particular that $q > q_{1}$. Using \cref{lem:Siegels_theorem_first_version_lemma} with $\s = 1$ applied to $L(\s,\chi_{1})$ and $L(\s,\chi_{1}\chi)$, we obtain
      \begin{equation}\label{equ:Siegels_theorem_first_version_11}
        \l \ll \log(q_{1})\log(q_{1}q)L(1,\chi).
      \end{equation}
      Combining \cref{equ:Siegels_theorem_first_version_10,equ:Siegels_theorem_first_version_11} yields
      \[
        \frac{1-\b_{1}}{(q_{1}q)^{8(1-\b_{1})}} \ll \log(q_{1})\log(q_{1}q)L(1,\chi).
      \]
      As $\b_{1}$ and $q_{1}$ are fixed and $\log(q_{1}q) = O(\log(q))$, isolating $L(1,\chi)$ gives the weaker estimate
      \begin{equation}\label{equ:Siegels_theorem_first_version_12}
        L(1,\chi) \gg \frac{1}{q^{8(1-\b_{1})}\log(q)}.
      \end{equation}
      But $1-\frac{\e}{16} < \b_{1} < 1$ so that $0 < 8(1-\b_{1}) < \frac{\e}{2}$ which combined with \cref{equ:Siegels_theorem_first_version_12} yields
      \[
        L(1,\chi) \gg \frac{1}{q^{\frac{\e}{2}}(\log(q))} \gg \frac{1}{q^{\e}},
      \]
      where the last estimate follows because $\log(q) \ll q^{\frac{\e}{2}}$ for sufficiently large $q$.
    \end{proof}

    The part of the proof of the lower bound version of Siegel's theorem which makes $c_{1}(\e)$ (and hence $c_{2}(\e)$) ineffective is the value of $\b_{1}$. The choice of $\b_{1}$ depends upon the existence of a Siegel zero near $1$ and relative to the given $\e$. Since we don't know if Siegel zeros exist, this makes estimating $\b_{1}$ relative to $\e$ ineffective. Many results in analytic number theory make use of Siegel's theorem and hence are also ineffective. Many important problems investigate methods to get around using Siegel's theorem in favor of a weaker result that is effective.
  \section{\todo{Prime Number Theorems}}