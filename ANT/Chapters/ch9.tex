\chapter{Additional Results}
  \section{The Value of Dirichlet \texorpdfstring{$L$}{L}-functions at \texorpdfstring{$s = 1$}{s = 1}}
      Let $\chi$ be a primitive Dirichlet character with conductor $q > 1$. We know from \cref{thm:non-vanishing_of_Dirichlet_L-functions_at_s=1} that $L(1,\chi) \neq 0$. It is interesting to know whether or not this value is computable in general. Indeed it is. The computation is fairly straightforward and only requires some basic properties of Gauss sums that we have already devloped. The idea is to rewrite the character values $\chi(n)$ so that we can collapse the infinite series into a Taylor series. Our result is the following:
      
      \begin{theorem}\label{thm:Value_of_Dirichlet_L-functions_at_s=1}
        Let $\chi$ be a primitive Dirichlet character with conductor $q > 1$. Then
        \[
          L(1,\chi) = -\frac{\chi(-1)\tau(1,\chi)}{q}\psum_{a \tmod{q}}\chi(a)\log\left(\sin\left(\frac{\pi a}{q}\right)\right) \quad \text{or} \quad L(1,\chi) = -\frac{\chi(-1)\tau(1,\chi)\pi i}{q^{2}}\psum_{a \tmod{q}}\chi(a)a,
        \]
        according to whether $\chi$ is even or odd.
      \end{theorem}
      \begin{proof}
        Make the following computation:
        \begin{align*}
          \chi(n) &= \frac{1}{\conj{\tau(1,\chi)}}\conj{\tau(n,\chi)} && \text{\cref{prop:Gauss_sum_reduction} (ii)} \\
          &= \frac{1}{\tau(1,\cchi)}\tau(n,\chi) && \text{\cref{prop:Gauss_sum_reduction} (i)} \\
          &= \frac{\tau(1,\chi)}{\tau(1,\chi)\tau(1,\cchi)}\tau(n,\chi) \\
          &= \frac{\chi(-1)\tau(1,\chi)}{q}\tau(n,\chi) && \text{\cref{thm:Gauss_sum_modulus,prop:epsilon_factor_relationship}} \\
          &= \frac{\chi(-1)\tau(1,\chi)}{q}\psum_{a \tmod{q}}\chi(a)e^{\frac{2\pi ian}{q}}.
        \end{align*}
        Substituting this result into the definition of $L(1,\chi)$ we find that
        \begin{equation}\label{equ:value_of_Dirichlet_L-functions_at_s=1_1}
          \begin{aligned}
            L(1,\chi) &= \sum_{n \ge 1}\frac{1}{n}\left(\frac{\chi(-1)\tau(1,\chi)}{q}\psum_{a \tmod{q}}\chi(a)e^{\frac{2\pi ian}{q}}\right) \\
            &= \frac{\chi(-1)\tau(1,\chi)}{q}\psum_{a \tmod{q}}\chi(a)\sum_{n \ge 1}\frac{e^{\frac{2\pi ian}{q}}}{n} \\
            &= \frac{\chi(-1)\tau(1,\chi)}{q}\psum_{a \tmod{q}}\chi(a)\log\left(\left(1-e^{\frac{2\pi ia}{q}}\right)^{-1}\right),
          \end{aligned}
        \end{equation}
        where in the last line we have used the Taylor series of the logarithm (notice $a \neq q$ so that $e^{\frac{2\pi ia}{q}} \neq 1$ and hence the logarithm is defined). We have now expressed $L(1,\chi)$ as a finite sum. In order to simplify the last expression in \cref{equ:value_of_Dirichlet_L-functions_at_s=1_1}, we deal with the logarithm. Since $\sin(\t) = \frac{e^{i\t}-e^{-i\t}}{2i}$, we have
        \[
          1-e^{\frac{2\pi ia}{q}} = -2ie^{\frac{\pi ia}{q}}\left(\frac{e^{\frac{\pi ia}{q}}-e^{-\frac{\pi ia}{q}}}{2i}\right) = -2ie^{\frac{\pi ia}{q}}\sin\left(\frac{\pi a}{q}\right).
        \]
        Therefore the last expression in \cref{equ:value_of_Dirichlet_L-functions_at_s=1_1} becomes
        \[
          \frac{\chi(-1)\tau(1,\chi)}{q}\psum_{a \tmod{q}}\chi(a)\log\left(\left(-2ie^{\frac{\pi ia}{q}}\sin\left(\frac{\pi a}{q}\right)\right)^{-1}\right).
        \]
        As $0< a < q$, we have $0 < \frac{\pi a}{q} < \pi$ so that $\sin\left(\frac{\pi a}{q}\right)$ is never negative. Therefore we can split up the logarithm term and obtain
        \[
          -\frac{\chi(-1)\tau(1,\chi)}{q}\left(\log(-2i)\psum_{a \tmod{q}}\chi(a)+\frac{\pi i}{q}\psum_{a \tmod{q}}\chi(a)a+\psum_{a \tmod{q}}\chi(a)\log\left(\sin\left(\frac{\pi a}{q}\right)\right)\right).
        \]
        By the orthogonality relations (\cref{cor:Dirichlet_orthogonality_relations} (i)), the first sum above vanishes. Therefore
        \begin{equation}\label{equ:value_of_Dirichlet_L-functions_at_s=1_2}
          L(1,\chi) = -\frac{\chi(-1)\tau(1,\chi)}{q}\left(\frac{\pi i}{q}\psum_{a \tmod{q}}\chi(a)a+\psum_{a \tmod{q}}\chi(a)\log\left(\sin\left(\frac{\pi a}{q}\right)\right)\right).
        \end{equation}
        \cref{equ:value_of_Dirichlet_L-functions_at_s=1_2} simplifies in that one of the two sums vanish depending on if $\chi$ is even or odd. For the first sum in \cref{equ:value_of_Dirichlet_L-functions_at_s=1_2}, observe that
        \[
          \frac{\pi i}{q}\psum_{a \tmod{q}}\chi(a)a = -\frac{\chi(-1)\pi i}{q}\psum_{a \tmod{q}}\chi(-a)(-a),
        \]
        which vanishes if $\chi$ is even. For the second sum in \cref{equ:value_of_Dirichlet_L-functions_at_s=1_2}, we have an analogous relation of the form
        \[
          \psum_{a \tmod{q}}\chi(a)\log\left(\sin\left(\frac{\pi a}{q}\right)\right) = \chi(-1)\psum_{a \tmod{q}}\chi(-a)\log\left(\sin\left(\frac{\pi a}{q}\right)\right),
        \]
        which vanishes if $\chi$ is odd. This finishes the proof.
      \end{proof}

      \cref{thm:Value_of_Dirichlet_L-functions_at_s=1} encodes some interesting identities. For example, if $\chi$ is the non-principal Dirichlet character modulo $4$, then $\chi$ is uniquely defined by $\chi(1) = 1$ and $\chi(3) = \chi(-1) = -1$. In particular, $\chi$ is odd and its conductor is $4$. Now
      \[
        \tau(1,\chi) = \psum_{a \tmod{4}}\chi(a)e^{\frac{2\pi i}{4}} = e^{\frac{2\pi i}{4}}-e^{\frac{6\pi i}{4}} = i-(-i) = 2i,
      \]
      so by \cref{thm:Value_of_Dirichlet_L-functions_at_s=1} we get
      \[
        L(1,\chi) = -\frac{\chi(-1)\tau(1,\chi)\pi i}{16}(1-3) = \frac{\pi}{4}.
      \]
      Upon expanding out $L(1,\chi)$, we see that
      \[
        1-\frac{1}{3}+\frac{1}{5}-\frac{1}{7}+\cdots = \frac{\pi}{4},
      \]
      which is the famous \textbf{Madhava–Leibniz formula}\index{Madhava–Leibniz formula} for $\pi$. Alternatively, it's the Taylor series expansion for $\arctan(x)$ centered at $x = 0$ and evaulated at $x = 1$ (a boundary point of the disk of absolute convergence).