\chapter{Types of \texorpdfstring{$L$}{L}-functions}
  We discuss a variety of $L$-functions: the Riemann zeta function, $L$-functions attached to Dirichlet characters, and Hecke $L$-functions. In the case of Hecke $L$-functions, we also describe a method of Rankin and Selberg for constructing new $L$-functions from old ones.
  \section{The Riemann Zeta Function}
    \subsection*{The Definition \& Euler Product of \texorpdfstring{$\z(s)$}{\z(s)}}
      The \textbf{Riemann zeta function}\index{Riemann zeta function} or simply the \textbf{zeta function}\index{zeta function} $\z(s)$ is defined as an $L$-series:
      \[
        \z(s) = \sum_{n \ge 1}\frac{1}{n^{s}}.
      \]
      This is the prototypical example of a Dirichlet series as all the coefficients are $1$. We will see that $\z(s)$ is a Selberg class $L$-function. As the coefficients are trivially polynomially bounded, $\z(s)$ is locally absolutely uniformly convergent for $\s > 1$. Also note that $\z(s)$ is necessarily nonzero in this region. Determining the Euler product is also an easy matter. As the coefficients are obviously completely multiplicative, we have the degree $1$ Euler product:
      \[
        \z(s) = \prod_{p}(1-p^{-s})^{-1},
      \]
      in this region as well. The local factor at $p$ is $\z_{p}(s) = (1-p^{-s})^{-1}$ with local root $1$.
    \subsection*{The Integral Representation of \texorpdfstring{$\z(s)$}{\z(s)}: Part I}
      We want to find an integral representation for $\z(s)$. To do this, consider the function
      \[
        \w(z) = \sum_{n \ge 1}e^{\pi in^{2}z},
      \]
      defined for $z \in \H$. It is locally absolutely uniformly convergent in this region by the ratio test. Moreover, from the Taylor series of $\frac{1}{1-e^{y}}$ we see that
      \[
        \w(z) = O\left(\sum_{n \ge 1}e^{-\pi n^{2}y}\right) = O\left(\sum_{n \ge 1}e^{-\pi ny}\right) = O\left(\frac{1}{1-e^{-\pi y}}\right) = O(e^{-\pi y}),
      \]
      and so $\w$ exhibits rapid decay. Now consider the following Mellin transform:
      \[
        \int_{0}^{\infty}\w(iy)y^{\frac{s}{2}}\,\frac{dy}{y}.
      \]
      By the rapid decay of $\w$, this integral exists and defines an analytic function for $\s > 1$. Then we compute
      \begin{align*}
        \int_{0}^{\infty}\w(iy)y^{\frac{s}{2}}\,\frac{dy}{y} &= \int_{0}^{\infty}\sum_{n \ge 1}e^{-\pi n^{2}y}y^{\frac{s}{2}}\,\frac{dy}{y} \\
        &= \sum_{n \ge 1}\int_{0}^{\infty}e^{-\pi n^{2}y}y^{\frac{s}{2}}\,\frac{dy}{y} && \text{DCT} \\
        &= \sum_{n \ge 1}\frac{1}{\pi^{\frac{s}{2}}n^{s}}\int_{0}^{\infty}e^{-y}y^{\frac{s}{2}}\,\frac{dy}{y} && \text{$y \to \frac{y}{\pi n^{2}}$} \\
        &= \frac{\G\left(\frac{s}{2}\right)}{\pi^{\frac{s}{2}}}\sum_{n \ge 1}\frac{1}{n^{s}} \\
        &= \frac{\G\left(\frac{s}{2}\right)}{\pi^{\frac{s}{2}}}\z(s).
      \end{align*}
      Therefore we have an integral representation
      \begin{equation}\label{equ:integral_representation_zeta_1}
        \z(s) = \frac{\pi^{\frac{s}{2}}}{\G\left(\frac{s}{2}\right)}\int_{0}^{\infty}\w(iy)y^{\frac{s}{2}}\,\frac{dy}{y}.
      \end{equation}
      Unfortunately, we cannot proceed until we obtain a functional equation for $\w$. So we will make a slight detour and come back to the integral representation after.
    \subsection*{Jacobi's Theta Function \texorpdfstring{$\vt(z)$}{\vt(z)}}
      \textbf{Jacobi's theta function}\index{Jacobi's theta function} $\vt(z)$ is defined by
      \[
        \vt(z) = \sum_{n \in \Z}e^{2\pi in^{2}z},
      \]
      for $z \in \H$. Its relation to $\w$ is given by the identity
      \begin{equation}\label{equ:omega_theta_relationship_for_zeta}
        \w(z) = \frac{\vt\left(\frac{z}{2}\right)-1}{2},
      \end{equation}
      and so $\vt$ is locally absolutely uniformly convergent in this region and exhibits rapid decay. The essential we will need is the \textbf{functional equation for Jacobi's theta function}\index{functional equation for Jacobi's theta function}:

      \begin{theorem}[Functional equation for Jacobi's theta function]
        For $z \in \H$,
        \[
          \vt(z) = \frac{1}{\sqrt{-2iz}}\vt\left(-\frac{1}{4z}\right).
        \]
      \end{theorem}
      \begin{proof}
        By the identity theorem it suffices to verify this for $z = iy$ with $y > 0$. So set $f(x) = e^{-2\pi x^{2}y}$. Then $f(x)$ is of Schwarz class. We compute its Fourier transform:
        \[
          \hat{f}(t) = \int_{-\infty}^{\infty}f(x)e^{-2\pi itx}\,dx = \int_{-\infty}^{\infty}e^{-2\pi x^{2}y}e^{-2\pi itx}\,dx = \int_{-\infty}^{\infty}e^{-2\pi(x^{2}y+itx)}\,dx.
        \]
        Making the change of variables $x \to \frac{x}{\sqrt{y}}$, the last integral above becomes
        \[
          \frac{1}{\sqrt{y}}\int_{-\infty}^{\infty}e^{-2\pi\left(x^{2}+\frac{itx}{\sqrt{y}}\right)}.
        \]
        Complete the square in the exponent by noticing
        \[
          -2\pi\left(x^{2}+\frac{itx}{\sqrt{y}}\right) = -2\pi\left(\left(x+\frac{it}{2\sqrt{y}}\right)^{2}+\frac{t^{2}}{4y}\right).
        \]
        Taking exponentials, this implies that the previous integral is equal to
        \[
          \frac{e^{-\frac{\pi t^{2}}{2y}}}{\sqrt{y}}\int_{-\infty}^{\infty}e^{-2\pi\left(x+\frac{it}{2\sqrt{y}}\right)^{2}}\,dx.
        \]
        The change of variables $x \to \frac{x}{\sqrt{2}}-\frac{it}{2\sqrt{y}}$ is permitted without affecting the line of integration by viewing the integral as a complex integral, noting that the integrand is entire as a complex function, and shifting the line of integration. This gives
        \[
          \frac{e^{-\frac{\pi t^{2}}{2y}}}{\sqrt{2y}}\int_{-\infty}^{\infty}e^{-\pi x^{2}}\,dx = \frac{e^{-\frac{\pi t^{2}}{2y}}}{\sqrt{2y}},
        \]
        where the last equality follows because the last integral above is $1$ since it is the Gaussian integral (see \cref{append:Special_Integrals}). Thus
        \[
          \hat{f}(t) = \frac{e^{-\frac{\pi t^{2}}{2y}}}{\sqrt{2y}}.
        \]
        By the Poisson summation formula, we have
        \[
          \vt(iy) = \sum_{t \in \Z}\frac{e^{-\frac{\pi t^{2}}{2y}}}{\sqrt{2y}} = \frac{1}{\sqrt{2y}}\sum_{t \in \Z}e^{-\frac{\pi t^{2}}{2y}} = \frac{1}{\sqrt{2y}}\vt\left(-\frac{1}{4iy}\right),
        \]
        and the identity theorem finishes the proof.
      \end{proof}

      We will use this functional equation to analytically continue $\z(s)$.
    \subsection*{The Integral Representation of \texorpdfstring{$\z(s)$}{\z(s)}: Part II}
      Returning to the Riemann zeta function, we split the integral in \cref{equ:integral_representation_zeta_1} into two pieces
      \begin{equation}\label{equ:symmetric_integral_zeta_split}
        \int_{0}^{\infty}\w(iy)y^{\frac{s}{2}}\,\frac{dy}{y} = \int_{0}^{1}\w(iy)y^{\frac{s}{2}}\,\frac{dy}{y}+\int_{1}^{\infty}\w(iy)y^{\frac{s}{2}}\,\frac{dy}{y}.
      \end{equation}
      The idea now is to rewrite the first piece in the same form and symmetrize the result as much as possible. We being by performing a change of variables $y \to \frac{1}{y}$ to the first piece to obtain
      \[
        \int_{1}^{\infty}\w\left(\frac{i}{y}\right)y^{-\frac{s}{2}}\,\frac{dy}{y}
      \]
      Now the functional equation for Jacobi's theta function $\vt$ and \cref{equ:omega_theta_relationship_for_zeta} together imply
      \begin{align*}
        \w\left(\frac{i}{y}\right) &= \w\left(-\frac{1}{iy}\right) \\
        &= \frac{\vt\left(-\frac{1}{2iy}\right)-1}{2} \\
        &= \frac{\sqrt{y}\vt\left(\frac{iy}{2}\right)-1}{2} \\
        &= \frac{\sqrt{y}(2\w(iy)+1)-1}{2} \\
        &= \sqrt{y}\w(iy)+\frac{\sqrt{y}}{2}-\frac{1}{2}.
      \end{align*}
      This relation gives the first equality in the following chain:
      \begin{align*}
        \int_{1}^{\infty}\w\left(\frac{1}{y}\right)y^{-\frac{s}{2}}\,\frac{dy}{y} &= \int_{1}^{\infty}\left(\sqrt{y}\w(iy)+\frac{\sqrt{y}}{2}-\frac{1}{2}\right)y^{-\frac{s}{2}}\,\frac{dy}{y} \\
        &= \int_{1}^{\infty}\w(iy)y^{\frac{1-s}{2}}\,\frac{dy}{y}+\int_{1}^{\infty}\frac{x^{\frac{1-s}{2}}}{2}\,\frac{dy}{y}-\int_{1}^{\infty}\frac{y^{-\frac{s}{2}}}{2}\,\frac{dy}{y} \\
        &= \int_{1}^{\infty}\w(iy)y^{\frac{1-s}{2}}\,\frac{dy}{y}+\frac{1}{1-s}-\frac{1}{s} \\
        &= \int_{1}^{\infty}\w(iy)y^{\frac{1-s}{2}}\,\frac{dy}{y}-\frac{1}{s(1-s)}.
      \end{align*}
      Substituting this result back into \cref{equ:symmetric_integral_zeta_split} with \cref{equ:integral_representation_zeta_1} yields the integral representation
      \[
        \z(s) = \frac{\pi^{\frac{s}{2}}}{\G\left(\frac{s}{2}\right)}\left[-\frac{1}{s(1-s)}+\int_{1}^{\infty}\w(iy)y^{\frac{1-s}{2}}\,\frac{dy}{y}+\int_{1}^{\infty}\w(iy)y^{\frac{s}{2}}\,\frac{dy}{y}\right].
      \]
      This integral representation will give analytic continuation. To see this, first observe that everything outside the brackets is entire. Moreover, the two integrals are locally absolutely uniformly convergent on $\C$ by \cref{prop:decay_unbounded_inteval_integral}. The fractional term is holomorphic except for simple poles at $s = 0$ and $s = 1$. The meromorphic continuation to $\C$ follows with possible simple poles at $s = 0$ and $s = 1$. There is no pole at $s = 0$. Indeed, $\g(s,\z)$ has a simple pole coming from the gamma factor there and so its reciprocal has a simple zero. This cancels the corresponding simple pole of $\frac{1}{s(1-s)}$ so that $\z(s)$ has a removable singularity and thus is holomorphic at $s = 0$. At $s = 1$, $\g(s,\z)$ is nonzero, and so $\z(s)$ has a simple pole. Therefore $\z(s)$ has meromorphic continuation to all of $\C$ with a simple pole at $s = 1$. 
    \subsection*{The Functional Equation, Critical Strip \& Residue of \texorpdfstring{$\z(s)$}{\z(s)}}
      An immediate consequence of applying the symmetry $s \to 1-s$ to the integral representation is the following functional equation:
      \[
        \frac{\G\left(\frac{s}{2}\right)}{\pi^{\frac{s}{2}}}\z(s) = \frac{\G\left(\frac{1-s}{2}\right)}{\pi^{\frac{1-s}{2}}}\z(1-s).
      \]
      We identify the gamma factor as
      \[
        \g(s,\z) = \pi^{-\frac{s}{2}}\G\left(\frac{s}{2}\right),
      \]
      with $\k = 0$ the only local root at infinity. Clearly it satisfies the required bounds. The conductor is $q(\z) = 1$ so no primes ramify. The completed zeta function is
      \[
        \L(s,\z) = \pi^{-\frac{s}{2}}\G\left(\frac{s}{2}\right)\z(s),
      \]
      with functional equation
      \[
        \L(s,\z) = \L(1-s,\z).
      \]
      This is the functional equation of $\z(s)$ and in this case is just a reformulation of the previous functional equation. From it we find that the root number is $\e(\z) = 1$ and that $\z(s)$ is self-dual. We can now show that the order of $\z(s)$ is $1$. As there is only a simple pole at $s = 1$, multiply by $(s-1)$ to clear the polar divisor. As the integrals in the integral representation are locally absolutely uniformly convergent, computing the order amounts to estimating the gamma factor. Since the reciprocal of the gamma function is of order $1$, we have
      \[
        \frac{1}{\g(s,\z)} \ll_{\e} e^{|s|^{1+\e}}.
      \]
      Thus the reciprocal of the gamma factor is also of order $1$. It follows that
      \[
        (s-1)\z(s) \ll_{\e} e^{|s|^{1+\e}}.
      \]
      This shows $(s-1)\z(s)$ is of order $1$, and thus $\z(s)$ is as well after removing the polar divisor. We now compute the residue of $\z(s)$ at $s = 1$:

      \begin{proposition}\label{prop:zeta_residue}
        \[
          \Res_{s = 1}\z(s) = 1.
        \]
      \end{proposition}
      \begin{proof}
        The only term in the integral representation of $\z(s)$ contributing to the pole is $-\frac{\pi^{\frac{s}{2}}}{\G\left(\frac{s}{2}\right)}\frac{1}{s(1-s)}$. Observe
        \[
          \lim_{s \to 1}\frac{\pi^{\frac{s}{2}}}{\G\left(\frac{s}{2}\right)} = 1,
        \]
        because $\G\left(\frac{1}{2}\right) = \sqrt{\pi}$. Therefore
        \[
          \Res_{s = 1}\z(s) = \Res_{s = 1}-\frac{\pi^{\frac{s}{2}}}{\G\left(\frac{s}{2}\right)}\frac{1}{s(1-s)} = \Res_{s = 1}-\frac{1}{s(1-s)} = \lim_{s \to 1}-\frac{(s-1)}{s(1-s)} = 1.
        \]
      \end{proof}

      We summarize all of our work into the following theorem:

      \begin{theorem}\label{thm:zeta_Selberg}
        $\z(s)$ is a Selberg class $L$-function. For $\s > 1$, it has a degree $1$ Euler product given by
        \[
          \z(s) = \prod_{p}(1-p^{-s})^{-1}.
        \]
        Moreover, it admits meromorphic continuation to $\C$ via the integral representation
        \[
          \z(s) = \frac{\pi^{\frac{s}{2}}}{\G\left(\frac{s}{2}\right)}\left[-\frac{1}{s(1-s)}+\int_{1}^{\infty}\w(iy)y^{\frac{1-s}{2}}\,\frac{dy}{y}+\int_{1}^{\infty}\w(iy)y^{\frac{s}{2}}\,\frac{dy}{y}\right],
        \]
        with functional equation
        \[
          \pi^{-\frac{s}{2}}\G\left(\frac{s}{2}\right)\z(s) = \L(s,\z) = \L(1-s,\z),
        \]
        and there is a simple pole at $s = 1$ of residue $1$.
      \end{theorem}

      Lastly, we note that by virtue of the functional equation we can also compute $\z(0)$. Indeed, since $\Res_{s = 1}\z(s) = 1$, we have
      \[
        \lim_{s \to 1}(s-1)\L(s,\z) = \Res_{s = 1}\z(s)\lim_{s \to 1}\pi^{-\frac{s}{2}}\G\left(\frac{s}{2}\right) = 1.
      \]
      In other words, $\L(s,\z)$ has a simple pole at $s = 1$ with residue $1$ too. Since the completed zeta function is completely symmetric as $s \to 1-s$, it has a simple pole at $s = 0$ with residue $1$. Hence
      \[
        1 = \lim_{s \to 1}(s-1)\L(1-s,\z) = \Res_{s = 1}\G\left(\frac{1-s}{2}\right)\lim_{s \to 1}\pi^{-\frac{1-s}{2}}\z(1-s) = -2\z(0),
      \]
      because $\Res_{s = 0}\G(s) = 1$. Therefore $\z(0) = -\frac{1}{2}$.
  \section{Dirichlet \texorpdfstring{$L$}{L}-functions}
    \subsection*{The Definition \& Euler Product of \texorpdfstring{$L(s,\chi)$}{L(s,\chi)}}
      To every Dirichlet character $\chi$ there is an associated $L$-function. Throughout we will let $m$ denote the modulus and $q$ the conductor of $\chi$ respectively. The \textbf{Dirichlet $L$-function}\index{Dirichlet $L$-function} $L(s,\chi)$ attached to the Dirichlet character $\chi$ is defined as an $L$-series:
      \[
        L(s,\chi) = \sum_{n \ge 1}\frac{\chi(n)}{n^{s}}.
      \]
      Since $\chi(n) = 0$ if $(n,m) > 1$, the above sum can be restricted to all positive integers relatively prime to $m$. We will see that $L(s,\chi)$ is a Selberg class $L$-function if $\chi$ is primitive and of conductor $q > 1$ (in the case $q = 1$, $L(s,\chi) = \z(s)$). From now we make this assumption about $\chi$. As $|\chi(n)| \ll 1$, $L(s,\chi)$ is locally absolutely uniformly convergent for $\s > 1$. Because $\chi$ is completely multiplicative we also have the degree $1$ Euler product:
      \[
        L(s,\chi) = \prod_{p}(1-\chi(p)p^{-s})^{-1} = \prod_{p \nmid m}(1-\chi(p)p^{-s})^{-1},
      \]
      in this region as well. The last equality holds because if $p \mid m$ we have $\chi(p) = 0$. So for $p \mid m$, the local factor at $p$ is $L_{p}(s,\chi) = 1$ with local root $0$. For $p \nmid m$ the local factor at $p$ is $L_{p}(s,\chi) = (1-\chi(p)p^{-s})^{-1}$ with local root $\chi(p)$.
    \subsection*{The Integral Representation of \texorpdfstring{$L(s,\chi)$}{L(s,\chi)}: Part I}
      The integral representation for $L(s,\chi)$ is deduced in a similar way as for $\z(s)$. However, it will depend on if $\chi$ is even or odd. To handle both cases simultaneously let $\mf{a} = 0,1$ according to whether $\chi$ is even or odd. In other words,
      \[
        \mf{a} = \frac{\chi(1)-\chi(-1)}{2}.
      \]
      We also have $\chi(-1) = (-1)^{\mf{a}}$. Note that $\mf{a}$ takes the same value for both $\chi$ and $\cchi$. To find an integral representation for $L(s,\chi)$, consider the function
      \[
        \w_{\chi}(z) = \sum_{n \ge 1}\chi(n)n^{\mf{a}}e^{\pi in^{2}z},
      \]
      defined for $z \in \H$. It is locally absolutely uniformly convergent in this region by the ratio test. From the Taylor series of $\frac{1}{1-e^{y}}$ and its derivative, we have
      \[
        \w_{\chi}(z) = O\left(\sum_{n \ge 1}ne^{-\pi n^{2}y}\right) = O\left(\sum_{n \ge 1}ne^{-\pi ny}\right) = O\left(\frac{e^{-\pi y}}{(1-e^{-\pi y})^{2}}\right) = O(e^{-\pi y}),
      \]
      and thus $\w_{\chi}$ exhibits rapid decay. Now consider the following Mellin transform:
      \[
        \int_{0}^{\infty}\w_{\chi}(iy)y^{\frac{s+\mf{a}}{2}}\,\frac{dy}{y}.
      \]
      By the rapid decay of $w_{\chi}$, this integral exists and defines an analytic function for $\s > 1$. Then we compute
      \begin{align*}
        \int_{0}^{\infty}\w_{\chi}(iy)y^{\frac{s+\mf{a}}{2}}\,\frac{dy}{y} &= \int_{0}^{\infty}\sum_{n \ge 1}\chi(n)n^{\mf{a}}e^{-\pi n^{2}y}y^{\frac{s+\mf{a}}{2}}\,\frac{dy}{y} \\
        &= \sum_{n \ge 1}\int_{0}^{\infty}\chi(n)n^{\mf{a}}e^{-\pi n^{2}y}y^{\frac{s+\mf{a}}{2}}\,\frac{dy}{y} && \text{DCT} \\
        &= \sum_{n \ge 1}\frac{\chi(n)}{\pi^{\frac{s+\mf{a}}{2}}n^{s}}\int_{0}^{\infty}e^{-y}y^{\frac{s+\mf{a}}{2}}\,\frac{dy}{y} && \text{$y \to \frac{y}{\pi n^{2}}$} \\
        &= \frac{\G\left(\frac{s+\mf{a}}{2}\right)}{\pi^{\frac{s+\mf{a}}{2}}}\sum_{n \ge 1}\frac{\chi(n)}{n^{s}} \\
        &= \frac{\G\left(\frac{s+\mf{a}}{2}\right)}{\pi^{\frac{s+\mf{a}}{2}}}L(s,\chi).
      \end{align*}
      Therefore we have an integral representation
      \begin{equation}\label{equ:integral_representation_Dirichlet_L-functions_1}
        L(s,\chi) = \frac{\pi^{\frac{s+\mf{a}}{2}}}{\G\left(\frac{s+\mf{a}}{2}\right)}\int_{0}^{\infty}\w_{\chi}(iy)y^{\frac{s+\mf{a}}{2}}\,\frac{dy}{y},
      \end{equation}
      and just as for the Riemann zeta function, we need to find a functional equation for $\w_{\chi}$ before we can proceed.
    \subsection*{Dirichlet's Theta Function \texorpdfstring{$\vt_{\chi}(z)$}{\vt_{\chi}(z)}}
      \textbf{Dirichlet's theta function}\index{Dirichlet's theta function} $\vt_{\chi}(z)$ attached to the character $\chi$, is defined by
      \[
        \vt_{\chi}(z) = \sum_{n \in \Z}\chi(n)n^{\mf{a}}e^{2\pi in^{2}z},
      \]
      for $z \in \H$. The relationship to $\w_{\chi}$ is
      \begin{equation}\label{equ:twisted_omega_theta_relationship_for_Dirichlet_L-functions}
        \w_{\chi}(z) = \frac{\vt_{\chi}\left(\frac{z}{2}\right)}{2},
      \end{equation}
      and so $\vt_{\chi}$ is locally absolutely uniformly convergent in this region and exhibits rapid decay.

      \begin{remark}
        \cref{equ:twisted_omega_theta_relationship_for_Dirichlet_L-functions} is a slightly less complex relationship than \cref{equ:omega_theta_relationship_for_zeta}. This is because assuming $q > 1$ means $\chi(0) = 0$.
      \end{remark}

      The essential fact we will need is the \textbf{functional equation for Dirichlet's theta function}\index{functional equation for Dirichlet's theta function}:

      \begin{theorem}[Functional equation for Dirichlet's theta function]
        Let $\chi$ be a primitive Dirichlet character of conductor $q > 1$. For $z \in \H$,
        \[
          \vt_{\chi}(z) = \frac{\e_{\chi}}{i^{\mf{a}}(-2qiz)^{\frac{1}{2}+\mf{a}}}\vt_{\cchi}\left(-\frac{1}{4q^{2}z}\right).
        \]
      \end{theorem}
      \begin{proof}
        By the identity theorem it suffices to verify this for $z = iy$ with $y > 0$. Since $\chi$ is $q$-periodic and $\vt_{\chi}$ is absolutely convergent, we can write
        \[
          \vt_{\chi}(iy) = \sum_{a \tmod{q}}\chi(a)\sum_{m \in \Z}(mq+a)^{\mf{a}}e^{-2\pi(mq+a)^{2}y}.
        \]
        Set $f(x) = (xq+a)^{\mf{a}}e^{-2\pi(xq+a)^{2}y}$. Then $f(x)$ is of Schwarz class. We compute its Fourier transform:
        \[
          \hat{f}(t) = \int_{-\infty}^{\infty}f(x)e^{-2\pi itx}\,dx = \int_{-\infty}^{\infty}(xq+a)^{\mf{a}}e^{-2\pi(xq+a)^{2}y}e^{-2\pi itx}\,dx = \int_{-\infty}^{\infty}(xq+a)^{\mf{a}}e^{-2\pi((xq+a)^{2}y+itx)}\,dx.
        \]
        By performing the change of variables $x \to \frac{x}{q\sqrt{y}}-\frac{a}{q}$, the last integral above becomes
        \[
          \frac{e^{\frac{2\pi iat}{q}}}{qy^{\frac{1+\mf{a}}{2}}}\int_{-\infty}^{\infty}x^{\mf{a}}e^{-2\pi\left(x^{2}+\frac{itx}{q\sqrt{y}}\right)}\,dx.
        \]
        Complete the square in the exponent by observing
        \[
          -2\pi\left(x^{2}+\frac{itx}{q\sqrt{y}}\right) = -2\pi\left(\left(x+\frac{it}{2q\sqrt{y}}\right)^{2}+\frac{t^{2}}{4q^{2}y}\right).
        \]
        Taking exponentials, this implies that the previous integral is equal to
        \[
          \frac{e^{\frac{2\pi iat}{q}-\frac{\pi t^{2}}{2q^{2}y}}}{qy^{\frac{1+\mf{a}}{2}}}\int_{-\infty}^{\infty}x^{\mf{a}}e^{-2\pi\left(x+\frac{it}{2q\sqrt{y}}\right)^{2}}\,dx.
        \]
        The change of variables $x \to \frac{x}{\sqrt{2}}-\frac{it}{2q\sqrt{y}}$ is permitted without affecting the line of integration by viewing the integral as a complex integral, noting that the integrand is entire as a complex function, and shifting the line of integration. This gives
        \[
          \frac{e^{\frac{2\pi iat}{q}-\frac{\pi t^{2}}{2q^{2}y}}}{\sqrt{2}qy^{\frac{1+\mf{a}}{2}}}\int_{-\infty}^{\infty}\left(\frac{x}{\sqrt{2}}-\frac{it}{2q\sqrt{y}}\right)^{\mf{a}}e^{-\pi x^{2}}\,dx = \frac{e^{\frac{2\pi iat}{q}-\frac{\pi t^{2}}{2q^{2}y}}}{\sqrt{2}qy^{\frac{1+\mf{a}}{2}}}\int_{-\infty}^{\infty}\left(\frac{x}{\sqrt{2}}+\frac{t}{2qi\sqrt{y}}\right)^{\mf{a}}e^{-\pi x^{2}}\,dx.
        \]
        If $\mf{a} = 0$, we obtain
        \begin{equation}\label{transformation_law_for_twisted_Jacobi's_theta_function_1}
          \frac{e^{\frac{2\pi iat}{q}-\frac{\pi t^{2}}{2q^{2}y}}}{\sqrt{2}qy^{\frac{1+\mf{a}}{2}}}\int_{-\infty}^{\infty}e^{-\pi x^{2}}\,dx = \frac{e^{\frac{2\pi iat}{q}-\frac{\pi t^{2}}{2q^{2}y}}}{\sqrt{2}qy^{\frac{1+\mf{a}}{2}}},
        \end{equation}
        where the equality holds because the integral is $1$ since it is the Gaussian integral (see \cref{append:Special_Integrals}). If $\mf{a} = 1$, then by direct computation
        \[
          \int_{-\infty}^{\infty}\frac{x}{\sqrt{2}}e^{-\pi x^{2}}\,dx = -\frac{1}{\sqrt{8}\pi}e^{-\pi x^{2}}\bigg|_{-\infty}^{\infty} = 0,
        \]
        and
        \begin{equation}\label{transformation_law_for_twisted_Jacobi's_theta_function_2}
          \frac{e^{\frac{2\pi iat}{q}-\frac{\pi t^{2}}{2q^{2}y}}}{\sqrt{2}qy^{\frac{1+\mf{a}}{2}}}\int_{-\infty}^{\infty}\left(\frac{t}{2qi\sqrt{y}}\right)e^{-\pi x^{2}}\,dx = \frac{e^{\frac{2\pi iat}{q}-\frac{\pi t^{2}}{2q^{2}y}}}{\sqrt{2}qy^{\frac{1+\mf{a}}{2}}}\left(\frac{t}{2qi\sqrt{y}}\right)\int_{-\infty}^{\infty}e^{-\pi x^{2}}\,dx = \frac{e^{\frac{2\pi iat}{q}-\frac{\pi t^{2}}{2q^{2}y}}}{\sqrt{2}qy^{\frac{1+\mf{a}}{2}}}\left(\frac{t}{2qi\sqrt{y}}\right),
        \end{equation}
        where the last equality follows because the last integral is the Gaussian integral again. Since $\left(\frac{t}{2qi\sqrt{y}}\right)^{\mf{a}} = 1$ if $\mf{a} = 0$, \cref{transformation_law_for_twisted_Jacobi's_theta_function_1,transformation_law_for_twisted_Jacobi's_theta_function_2} together imply
        \[
          \hat{f}(t) = \frac{e^{\frac{2\pi iat}{q}-\frac{\pi t^{2}}{2q^{2}y}}}{\sqrt{2}qy^{\frac{1+\mf{a}}{2}}}\left(\frac{t}{2qi\sqrt{y}}\right)^{\mf{a}}.
        \]
        By the Poisson summation formula, we have
        \begin{align*}
          \vt_{\chi}(iy) &= \sum_{a \tmod{q}}\chi(a)\sum_{t \in \Z}\frac{e^{\frac{2\pi iat}{q}-\frac{\pi t^{2}}{2q^{2}y}}}{\sqrt{2}qy^{\frac{1+\mf{a}}{2}}}\left(\frac{t}{2qi\sqrt{y}}\right)^{\mf{a}} \\
          &= \frac{1}{i^{\mf{a}}q^{1+\mf{a}}(2y)^{\frac{1}{2}+\mf{a}}}\sum_{a \tmod{q}}\chi(a)\sum_{t \in \Z}t^{\mf{a}}e^{\frac{2\pi iat}{q}-\frac{\pi t^{2}}{2q^{2}y}} \\
          &= \frac{1}{i^{\mf{a}}q^{1+\mf{a}}(2y)^{\frac{1}{2}+\mf{a}}}\sum_{t \in \Z}t^{\mf{a}}e^{-\frac{\pi t^{2}}{2q^{2}y}}\sum_{a \tmod{q}}\chi(a)e^{\frac{2\pi i at}{q}} \\
          &= \frac{1}{i^{\mf{a}}q^{1+\mf{a}}(2y)^{\frac{1}{2}+\mf{a}}}\sum_{t \in \Z}t^{\mf{a}}e^{-\frac{\pi t^{2}}{2q^{2}y}}\tau(t,\chi) && \text{definition of $\tau(t,\chi)$} \\
          &= \frac{\tau(\chi)}{i^{\mf{a}}q^{1+\mf{a}}(2y)^{\frac{1}{2}+\mf{a}}}\sum_{t \in \Z}\cchi(t)t^{\mf{a}}e^{-\frac{\pi t^{2}}{2q^{2}y}} && \text{\cref{cor:gauss_sum_primitive_formula}} \\
          &= \frac{\e_{\chi}}{i^{\mf{a}}(2qy)^{\frac{1}{2}+\mf{a}}}\sum_{t \in \Z}\cchi(t)t^{\mf{a}}e^{-\frac{\pi t^{2}}{2q^{2}y}} && \text{$\e_{\chi} = \frac{\tau(\chi)}{\sqrt{q}}$} \\
          &=  \frac{\e_{\chi}}{i^{\mf{a}}(2qy)^{\frac{1}{2}+\mf{a}}}\vt_{\cchi}\left(-\frac{1}{4q^{2}iy}\right),
        \end{align*}
        and the identity theorem finishes the proof.
      \end{proof}
      Notice that the functional equation relates $\vt_{\chi}$ to $\vt_{\cchi}$. Regardless, we will use it to analytically continue $L(s,\chi)$.
    \subsection*{The Integral Representation of \texorpdfstring{$L(s,\chi)$}{L(s,\chi)}: Part II}
      Returning to $L(s,\chi)$, split the integral in \cref{equ:integral_representation_Dirichlet_L-functions_1} into two pieces
      \begin{equation}\label{equ:symmetric_integral_Dirichlet_L-functions_split}
        \int_{0}^{\infty}\w_{\chi}(iy)y^{\frac{s+\mf{a}}{2}}\,\frac{dy}{y} = \int_{0}^{\frac{1}{q}}\w_{\chi}(iy)y^{\frac{s+\mf{a}}{2}}\,\frac{dy}{y}+\int_{\frac{1}{q}}^{\infty}\w_{\chi}(iy)y^{\frac{s+\mf{a}}{2}}\,\frac{dy}{y}.
      \end{equation}
      We now rewrite the first piece in the same form and symmetrize the result as much as possible. Start by performing a change of variables $y \to \frac{1}{q^{2}y}$ to the first piece to obtain
      \[
        q^{-(s+\mf{a})}\int_{\frac{1}{q}}^{\infty}\w_{\chi}\left(\frac{i}{q^{2}y}\right)y^{-\frac{s+\mf{a}}{2}}\,\frac{dy}{y}.
      \]
      Now the functional equation for Dirichlet's theta function and \cref{equ:twisted_omega_theta_relationship_for_Dirichlet_L-functions} together imply
      \begin{align*}
        \w_{\chi}\left(\frac{i}{q^{2}y}\right) &= \w_{\chi}\left(-\frac{1}{q^{2}iy}\right) \\
        &= \frac{\vt_{\chi}\left(-\frac{1}{2q^{2}iy}\right)}{2} \\
        &= \frac{i^{\mf{a}}(qy)^{\frac{1}{2}+\mf{a}}}{\e_{\cchi}}\frac{\vt_{\cchi}\left(\frac{iy}{2}\right)}{2} \\
        &= \frac{i^{\mf{a}}(qy)^{\frac{1}{2}+\mf{a}}}{\e_{\cchi}}\frac{\vt_{\cchi}\left(\frac{iy}{2}\right)}{2} \\
        &= \e_{\chi}(-i)^{\mf{a}}(qy)^{\frac{1}{2}+\mf{a}}\frac{\vt_{\cchi}\left(\frac{iy}{2}\right)}{2} && \text{\cref{prop:epsilon_factor_relationship} and $\chi(-1) = (-1)^{\mf{a}}$} \\
        &= \frac{\e_{\chi}(qy)^{\frac{1}{2}+\mf{a}}}{i^{\mf{a}}}\frac{\vt_{\cchi}\left(\frac{iy}{2}\right)}{2} \\
        &= \frac{\e_{\chi}(qy)^{\frac{1}{2}+\mf{a}}}{i^{\mf{a}}}\w_{\cchi}(iy).
      \end{align*}
      This relation gives the first equality in the following chain:
      \begin{align*}
        q^{-(s+\mf{a})}\int_{\frac{1}{q}}^{\infty}\w_{\chi}\left(\frac{1}{q^{2}y}\right)y^{-\frac{s+\mf{a}}{2}}\,\frac{dy}{y} &= q^{-(s+\mf{a})}\int_{\frac{1}{q}}^{\infty}\left(\frac{\e_{\chi}(qy)^{\frac{1}{2}+\mf{a}}}{i^{\mf{a}}}\w_{\cchi}(iy)\right)y^{-\frac{s+\mf{a}}{2}}\,\frac{dy}{y} \\
        &= \frac{\e_{\chi}}{i^{\mf{a}}}q^{\frac{1}{2}-s}\int_{\frac{1}{q}}^{\infty}\w_{\cchi}(iy)y^{\frac{(1-s)+\mf{a}}{2}}\,\frac{dy}{y}.
      \end{align*}
      Substituting this last expression back into \cref{equ:symmetric_integral_Dirichlet_L-functions_split} with \cref{equ:integral_representation_Dirichlet_L-functions_1} gives the integral representation
      \[
        L(s,\chi) = \frac{\pi^{\frac{s+\mf{a}}{2}}}{\G\left(\frac{s+\mf{a}}{2}\right)}\left[\frac{\e_{\chi}}{i^{\mf{a}}}q^{\frac{1}{2}-s}\int_{\frac{1}{q}}^{\infty}\w_{\cchi}(iy)y^{\frac{(1-s)+\mf{a}}{2}}\,\frac{dy}{y}+\int_{\frac{1}{q}}^{\infty}\w_{\chi}(iy)y^{\frac{s+\mf{a}}{2}}\,\frac{dy}{y}\right].
      \]
      This integral representation will give analytic continuation. Indeed, we know everything outside the brackets is entire. The two integrals are locally absolutely uniformly convergent on $\C$ by \cref{prop:decay_unbounded_inteval_integral}. This gives analytic continuation to all of $\C$. In particular, $L(s,\chi)$ has no poles.
    \subsection*{The Functional Equation \& Critical Strip of \texorpdfstring{$L(s,\chi)$}{L(s,\chi)}}
      An immediate consequence of applying the symmetry $s \to 1-s$ to the integral representation is the following functional equation:
      \[
        q^{\frac{s}{2}}\frac{\G\left(\frac{s+\mf{a}}{2}\right)}{\pi^{\frac{s+\mf{a}}{2}}}L(s,\chi) = \frac{\e_{\chi}}{i^{\mf{a}}}q^{\frac{1-s}{2}}\frac{\G\left(\frac{(1-s)+\mf{a}}{2}\right)}{\pi^{\frac{(1-s)+\mf{a}}{2}}}L(1-s,\cchi).
      \]
      We identify the gamma factor as
      \[
        \g(s,\chi) = \pi^{-\frac{s}{2}}\G\left(\frac{s+\mf{a}}{2}\right),
      \]
      with $\k = \mf{a}$ the only local root at infinity. Clearly it satisfies the required bounds. The conductor is $q(\chi) = q$ and if $p$ is an unramified prime then the local root is $\chi(p) \neq 0$. The completed $L$-function is
      \[
        \L(s,\chi) = q^{\frac{s}{2}}\pi^{-\frac{s}{2}}\G\left(\frac{s+\mf{a}}{2}\right)L(s,\chi),
      \]
      with functional equation
      \[
        \L(s,\chi) = \frac{\e_{\chi}}{i^{\mf{a}}}\L(1-s,\cchi).
      \]
      From it we see that the root number is $\e(\chi) = \frac{\e_{\chi}}{i^{\mf{a}}}$ and that $L(s,\chi)$ has dual $L(s,\cchi)$. We now show that $L(s,\chi)$ is of order $1$. Since $L(s,\chi)$ has no poles, we do not need to clear any polar divisors. As the integrals in the integral representation are locally absolutely uniformly convergent, computing the order amounts to estimating the gamma factor. Since the reciprocal of the gamma function is of order $1$, we have
      \[
        \frac{1}{\g(s,\chi)} \ll_{\e} e^{|s|^{1+\e}}.
      \]
      So the reciprocal of the gamma factor is also of order $1$. It follows that
      \[
        L(s,\chi) \ll_{\e} e^{|s|^{1+\e}}.
      \]
      So $L(s,\chi)$ is of order $1$. We summarize all of our work into the following theorem:

      \begin{theorem}\label{thm:primitive_Dirichlet_Selberg}
        For any primitive Dirichlet character $\chi$ of conductor $q > 1$, $L(s,\chi)$ is a Selberg class $L$-function. For $\s > 1$, it has a degree $1$ Euler product given by
        \[
          L(s,\chi) = \prod_{p}(1-\chi(p)p^{-s})^{-1} = \prod_{p \nmid q}(1-\chi(p)p^{-s})^{-1}.
        \]
        Moreover, it admits analytic continuation to $\C$ via the integral representation
        \[
          L(s,\chi) = \frac{\pi^{\frac{s+\mf{a}}{2}}}{\G\left(\frac{s+\mf{a}}{2}\right)}\left[\frac{\e_{\chi}}{i^{\mf{a}}}q^{\frac{1}{2}-s}\int_{\frac{1}{q}}^{\infty}\w_{\cchi}(iy)y^{\frac{(1-s)+\mf{a}}{2}}\,\frac{dy}{y}+\int_{\frac{1}{q}}^{\infty}\w_{\chi}(iy)y^{\frac{s+\mf{a}}{2}}\,\frac{dy}{y}\right],
        \]
        and possesses the functional equation
        \[
          q^{\frac{s}{2}}\pi^{-\frac{s}{2}}\G\left(\frac{s+\mf{a}}{2}\right)L(s,\chi) = \L(s,\chi) = \frac{\e_{\chi}}{i^{\mf{a}}}\L(1-s,\cchi).
        \]
      \end{theorem}
    \subsection*{Beyond Primitivity of of \texorpdfstring{$L(s,\chi)$}{L(s,\chi)}}
      We can still obtain meromorphic continuation of $L(s,\chi)$ if $\chi$ is not primitive. Indeed, if $\chi$ is induced by $\wtilde{\chi}$, then $\chi(p) = \wtilde{\chi}(p)$ if $p \nmid q$ and $\chi(p) = 0$ if $p \mid m$ so that
      \begin{equation}\label{equ:non-primitive_primitive_Dirichlet_L-series_relation}
        L(s,\chi) = \prod_{p \nmid m}(1-\wtilde{\chi}(p)p^{-s})^{-1} = \prod_{p}(1-\wtilde{\chi}(p)p^{-s})^{-1}\prod_{p \mid m}(1-\wtilde{\chi}(p)p^{-s}) = L(s,\wtilde{\chi})\prod_{p \mid m}(1-\wtilde{\chi}(p)p^{-s}).
      \end{equation}
      From this relation, we can prove the following:

      \begin{theorem}\label{thm:analytic_continuation_Dirichlet}
        For any Dirichlet character $\chi$ modulo $m$ of conductor $q > 1$, $L(s,\chi)$ admits meromorphic continuation to $\C$ and if $\chi$ is principal there is a simple pole at $s = 1$ of residue $\prod_{p \mid m}(1-\wtilde{\chi}(p)p^{-1})$ where $\wtilde{\chi}$ is the primitive character inducing $\chi$.
      \end{theorem}
      \begin{proof}
        This follows from \cref{thm:zeta_Selberg,thm:primitive_Dirichlet_Selberg,equ:non-primitive_primitive_Dirichlet_L-series_relation}. 
      \end{proof}
  \section{Hecke \texorpdfstring{$L$}{L}-functions}
    \subsection*{The Definition \& Euler Product of \texorpdfstring{$L(s,f)$}{L(s,f)}}
      We will investigate the $L$-functions of holomorphic cusp forms. Let $f \in \mc{S}_{k}(N,\chi)$ and denote its Fourier series by
      \[
        f(z) = \sum_{n \ge 1}a_{f}(n)n^{\frac{k-1}{2}}e^{2\pi inz},
      \]
      with $a_{f}(1) = 1$. Thus if $f$ is a Hecke eigenform, the $a_{f}(n)$ are the Hecke eigenvalues of $f$ normalized so that they are constant on average. The \textbf{Hecke $L$-function}\index{Hecke $L$-function} $L(s,f)$ of $f$ is defined as an $L$-series:
      \[
        L(s,f) = \sum_{n \ge 1}\frac{a_{f}(n)}{n^{s}}.
      \]
      We will see that $L(s,f)$ is a Selberg class $L$-function if $f$ is a primitive Hecke eigenform. From now on, we make this assumption about $f$. As we have noted, the Hecke relations and the Ramanujan-Petersson conjecture for holomorphic forms together imply $a_{f}(n) \ll_{\e} n^{\e}$. So $L(s,f)$ is locally absolutely uniformly convergent for $\s > 1+\e$ and hence locally absolutely uniformly convergent for $\s > 1$. The $L$-function will also have an Euler product. Indeed, the Hecke relations imply that the coefficients $a_{f}(n)$ are multiplicative and satisfy
      \begin{equation}\label{equ:primitive_Hecke_eigenform_recurrence_for_coefficients_of_holomorphic_L-function}
        a_{f}(p^{n}) = \begin{cases} a_{f}(p^{n-1})a_{f}(p)-\chi(p)a_{f}(p^{n-2}) & \text{if $p \nmid N$}, \\ (a_{f}(p))^{n} & \text{if $p \mid N$}, \end{cases}
      \end{equation}
      for all primes $p$ and $n \ge 2$. Because $L(s,f)$ converges absolutely in the region $\s > 1$, multiplicativity of the Hecke eigenvalues implies
      \[
        L(s,f) = \prod_{p}\left(\sum_{n \ge 0}\frac{a_{f}(p^{n})}{p^{ns}}\right),
      \]
      in this region. We now simplify the factor inside the product using this \cref{equ:primitive_Hecke_eigenform_recurrence_for_coefficients_of_holomorphic_L-function}. On the one hand, if $p \nmid N$:
      \begin{align*}
        \sum_{n \ge 0}\frac{a_{f}(p^{n})}{p^{ns}} &= 1+\frac{a_{f}(p)}{p^{s}}+\sum_{n \ge 2}\frac{a_{f}(p^{n})}{p^{ns}} \\
        &= 1+\frac{a_{f}(p)}{p^{s}}+\sum_{n \ge 2}\frac{a_{f}(p^{n-1})a_{f}(p)-\chi(p)a_{f}(p^{n-2})}{p^{ns}} \\
        &= 1+\frac{a_{f}(p)}{p^{s}}+\frac{a_{f}(p)}{p^{s}}\sum_{n \ge 1}\frac{a_{f}(p^{n})}{p^{ns}}-\frac{\chi(p)}{p^{2s}}\sum_{n \ge 0}\frac{a_{f}(p^{n})}{p^{ns}} \\
        &= 1+\left(\frac{a_{f}(p)}{p^{s}}-\frac{\chi(p)}{p^{2s}}\right)\sum_{n \ge 0}\frac{a_{f}(p^{n})}{p^{ns}}.
      \end{align*}
      By isolating the sum we find
      \[
        \sum_{n \ge 0}\frac{a_{f}(p^{n})}{p^{ns}} = \left(1-\frac{a_{f}(p)}{p^{s}}+\frac{\chi(p)}{p^{2s}}\right)^{-1}.
      \]
      On the other hand, if $p \mid N$ we have
      \[
        \sum_{n \ge 0}\frac{a_{f}(p^{n})}{p^{ns}} = \sum_{n \ge 0}\frac{(a_{f}(p))^{n}}{p^{ns}} = \left(1-a_{f}(p)p^{-s}\right)^{-1}.
      \]
      Therefore
      \[
        L(s,f) = \prod_{p \nmid N}(1-a_{f}(p)p^{-s}+\chi(p)p^{-2s})^{-1}\prod_{p \mid N}(1-a_{f}(p)p^{-s})^{-1}.
      \]
      If $p \nmid N$, let $\a_{1}(p)$ and $\a_{2}(p)$ be the roots of $1-a_{f}(p)p^{-s}+\chi(p)p^{-2s}$. That is,
      \[
        (1-\a_{1}(p)p^{-s})(1-\a_{2}(p)p^{-s}) = (1-a_{f}(p)p^{-s}+\chi(p)p^{-2s}).
      \]
      If $p \mid N$, let $\a_{1}(p) = a_{f}(p)$ and $\a_{2}(p) = 0$. We can then express $L(s,f)$ as a degree $2$ Euler product:
      \[
        L(s,f) = \prod_{p}(1-\a_{1}(p)p^{-s})^{-1}(1-\a_{2}(p)p^{-s})^{-1}.
      \]
      The local factor at $p$ is $L_{p}(s,f) = (1-\a_{1}(p)p^{-s})^{-1}(1-\a_{2}(p)p^{-s})^{-1}$ with local roots $\a_{1}(p)$ and $\a_{2}(p)$. Upon applying partial fraction decomposition to the local factor, we find
      \[
        \frac{1}{1-\a_{1}(p)p^{-s}}\frac{1}{1-\a_{2}(p)p^{-s}} = \frac{\frac{\a_{1}(p)}{\a_{1}(p)-\a_{2}(p)}}{1-\a_{1}(p)p^{-s}}+\frac{\frac{-\a_{2}(p)}{\a_{1}(p)-\a_{2}(p)}}{1-\a_{2}(p)p^{-s}}.
      \]
      Expanding both sides as series in $p^{-s}$, and comparing coefficients gives
      \begin{equation}\label{equ:Hecke_L_function_coefficient_formula}
        a_{f}(p^{n}) = \frac{\a_{1}(p)^{n+1}-\a_{2}(p)^{n+1}}{\a_{1}(p)-\a_{2}(p)}.
      \end{equation}
    \subsection*{The Integral Representation of \texorpdfstring{$L(s,f)$}{L(s,f)}}
      We now want to find an integral representation for $L(s,f)$. Consider the following Mellin transform:
      \[
        \int_{0}^{\infty}f(iy)y^{s+\frac{k-1}{2}}\,\frac{dy}{y}.
      \]
      As $f$ has rapid decay at the cusps, this integral exists and defines an analytic function for $\s > 1$. In any case, we compute
      \begin{align*}
        \int_{0}^{\infty}f(iy)y^{s+\frac{k-1}{2}}\,\frac{dy}{y} &= \int_{0}^{\infty}\sum_{n \ge 1}a_{f}(n)n^{\frac{k-1}{2}}e^{-2\pi ny}y^{s+\frac{k-1}{2}}\,\frac{dy}{y} \\
        &= \sum_{n \ge 1}a_{f}(n)n^{\frac{k-1}{2}}\int_{0}^{\infty}e^{-2\pi ny}y^{s+\frac{k-1}{2}}\,\frac{dy}{y} &&\text{DCT} \\
        &= \sum_{n \ge 1}\frac{a_{f}(n)}{(2\pi)^{s+\frac{k-1}{2}}n^{s}}\int_{0}^{\infty}e^{-y}y^{s+\frac{k-1}{2}}\,\frac{dy}{y} &&\text{$y \to \frac{y}{2\pi n}$} \\
        &= \frac{\G\left(s+\frac{k-1}{2}\right)}{(2\pi)^{s+\frac{k-1}{2}}}\sum_{n \ge 1}\frac{a_{f}(n)}{n^{s}} \\
        &= \frac{\G\left(s+\frac{k-1}{2}\right)}{(2\pi)^{s+\frac{k-1}{2}}}L(s,f).
      \end{align*}
      Rewriting, we have an integral representation
      \begin{equation}\label{equ:integral_representation_holomorphic_1}
        L(s,f) = \frac{(2\pi)^{s+\frac{k-1}{2}}}{\G\left(s+\frac{k-1}{2}\right)}\int_{0}^{\infty}f(iy)y^{s+\frac{k-1}{2}}\,\frac{dy}{y}.
      \end{equation}
      Now split the integral on the right-hand side into two pieces
      \begin{equation}\label{equ:symmetric_integral_holomorphic_split}
        \int_{0}^{\infty}f(iy)y^{s+\frac{k-1}{2}}\,\frac{dy}{y} = \int_{0}^{\frac{1}{\sqrt{N}}}f(iy)y^{s+\frac{k-1}{2}}\,\frac{dy}{y}+\int_{\frac{1}{\sqrt{N}}}^{\infty}f(iy)y^{s+\frac{k-1}{2}}\,\frac{dy}{y}.
      \end{equation}
      Now we will rewrite the first piece in the same form and symmetrize the result as much as possible. Begin by performing the change of variables $y \to \frac{1}{Ny}$ to the first piece to obtain
      \[
        \int_{\frac{1}{\sqrt{N}}}^{\infty}f\left(\frac{i}{Ny}\right)(Ny)^{-s-\frac{k-1}{2}}\,\frac{dy}{y}.
      \]
      Rewriting in terms of the Atkin-Lehner operator and recalling that $\w_{N}f = \w_{N}(f)\conj{f}$ by \cref{prop:Atkin_Lehner_conjugation_holomorphic}, we have
      \begin{align*}
        \int_{\frac{1}{\sqrt{N}}}^{\infty}f\left(\frac{i}{Ny}\right)(Ny)^{-s-\frac{k-1}{2}}\,\frac{dy}{y} &= \int_{\frac{1}{\sqrt{N}}}^{\infty}f\left(-\frac{1}{iNy}\right)(Ny)^{-s-\frac{k-1}{2}}\,\frac{dy}{y} \\
        &= \int_{\frac{1}{\sqrt{N}}}^{\infty}\left(\sqrt{N}iy\right)^{k}(\w_{N}f)(iy)(Ny)^{-s-\frac{k-1}{2}}\,\frac{dy}{y} \\
        &= \int_{\frac{1}{\sqrt{N}}}^{\infty}\left(\sqrt{N}iy\right)^{k}\w_{N}(f)\conj{f}(iy)(Ny)^{-s-\frac{k-1}{2}}\,\frac{dy}{y} \\
        &= \w_{N}(f)i^{k}N^{\frac{1}{2}-s}\int_{\frac{1}{\sqrt{N}}}^{\infty}\conj{f}(iy)y^{(1-s)-\frac{k-1}{2}}\,\frac{dy}{y}.
      \end{align*}
      Substituting this result back into \cref{equ:symmetric_integral_holomorphic_split} with \cref{equ:integral_representation_holomorphic_1} yields the integral representation
      \[
        L(s,f) = \frac{(2\pi)^{s+\frac{k-1}{2}}}{\G\left(s+\frac{k-1}{2}\right)}\left[\w_{N}(f)i^{k}N^{\frac{1}{2}-s}\int_{\frac{1}{\sqrt{N}}}^{\infty}\conj{f}(iy)y^{(1-s)+\frac{k-1}{2}}\,\frac{dy}{y}+\int_{\frac{1}{\sqrt{N}}}^{\infty}f(iy)y^{s+\frac{k-1}{2}}\,\frac{dy}{y}\right].
      \]
      This integral representation will give analytic continuation. To see this, we know everything outside the brackets is entire. The two integrals are locally absolutely uniformly convergent on $\C$ by \cref{prop:decay_unbounded_inteval_integral}. Hence we have analytic continuation to all of $\C$. In particular, we have shown that $L(s,f)$ has no poles.
    \subsection*{The Functional Equation \& Critical Strip of \texorpdfstring{$L(s,f)$}{L(s,f)}}
      An immediate consequence of applying the symmetry $s \to 1-s$ to the integral representation is the following functional equation:
      \[
        \frac{\G\left(s+\frac{k-1}{2}\right)}{(2\pi)^{s+\frac{k-1}{2}}}L(s,f) = \w_{N}(f)i^{k}N^{-\frac{s}{2}}\frac{\G\left((1-s)+\frac{k-1}{2}\right)}{(2\pi)^{(1-s)+\frac{k-1}{2}}}L(1-s,\conj{f}).
      \]
      Using the Legendre duplication formula for the gamma function we find that
      \begin{align*}
        \frac{\G\left(s+\frac{k-1}{2}\right)}{(2\pi)^{s+\frac{k-1}{2}}} &= \frac{1}{(2\pi)^{s+\frac{k-1}{2}}2^{1-\left(s+\frac{k-1}{2}\right)}\sqrt{\pi}}\G\left(\frac{s+\frac{k-1}{2}}{2}\right)\G\left(\frac{s+\frac{k+1}{2}}{2}\right) \\
        &= \frac{1}{2\pi^{s+\frac{1}{2}}}\G\left(\frac{s+\frac{k-1}{2}}{2}\right)\G\left(\frac{s+\frac{k+1}{2}}{2}\right) \\ 
        &= \frac{1}{\sqrt{4\pi}}\pi^{-s}\G\left(\frac{s+\frac{k-1}{2}}{2}\right)\G\left(\frac{s+\frac{k+1}{2}}{2}\right).
      \end{align*}
      The constant factor in front is independent of $s$ and so can be canceled in the functional equation. Therefore we identify the gamma factor as
      \[
        \g(s,f) = \pi^{-s}\G\left(\frac{s+\frac{k-1}{2}}{2}\right)\G\left(\frac{s+\frac{k+1}{2}}{2}\right),
      \]
      with $\k_{1} = k-1$ and $\k_{2} = k+1$ the local roots at infinity. The conductor is $q(f) = N$, so the primes dividing the level ramify, and by the Ramanujan-Petersson conjecture for holomorphic forms, $\a_{1}(p) \neq 0$ and $\a_{2}(p) \neq 0$ for all primes $p \nmid N$. The completed $L$-function is
      \[
        \L(s,f) = N^{-\frac{s}{2}}\pi^{-s}\G\left(\frac{s+\frac{k-1}{2}}{2}\right)\G\left(\frac{s+\frac{k+1}{2}}{2}\right)L(s,f),
      \]
      with functional equation
      \[
        \L(s,f) = \w_{N}(f)i^{k}\L(1-s,\conj{f}).
      \]
      This is the functional equation of $L(s,f)$. From it, the root number is $\e(f) = \w_{N}(f)i^{k}$ and we see that $L(s,f)$ has dual $L(s,\conj{f})$. We will now show that $L(s,f)$ is of order $1$. Since $L(s,f)$ has no poles, we do not need to clear any polar divisors. As the integrals in the representation is locally absolutely uniformly convergent, computing the order amounts to estimating the gamma factor. Since the reciprocal of the gamma function is of order $1$, we have
      \[
        \frac{1}{\g(s,f)} \ll_{\e} e^{|s|^{1+\e}}.
      \]
      So the reciprocal of the gamma factor is also of order $1$. Then
      \[
        L(s,f) \ll_{\e} e^{|s|^{1+\e}}.
      \]
      So $L(s,f)$ is of order $1$. We summarize all of our work into the following theorem:

      \begin{theorem}\label{thm:primitive_Hecke_Selberg}
        For any primitive Hecke eigenform $f \in \mc{S}_{k}(N,\chi)$, $L(s,f)$ is a Selberg class $L$-function. For $\s > 1$, it has a degree $2$ Euler product given by 
        \[
          L(s,f) = \prod_{p \nmid N}(1-a_{f}(p)p^{-s}+\chi(p)p^{-2s})^{-1}\prod_{p \mid N}(1-a_{f}(p)p^{-s})^{-1}.
        \]
        Moreover, it admits analytic continuation to $\C$ via the integral representation
        \[
          L(s,f) = \frac{(2\pi)^{s+\frac{k-1}{2}}}{\G\left(s+\frac{k-1}{2}\right)}\left[\w_{N}(f)i^{k}N^{\frac{1}{2}-s}\int_{\frac{1}{\sqrt{N}}}^{\infty}\conj{f}(iy)y^{(1-s)+\frac{k-1}{2}}\,\frac{dy}{y}+\int_{\frac{1}{\sqrt{N}}}^{\infty}f(iy)y^{s+\frac{k-1}{2}}\,\frac{dy}{y}\right],
        \]
        and possesses the functional equation
        \[
          N^{-\frac{s}{2}}\pi^{-s}\G\left(\frac{s+\frac{k-1}{2}}{2}\right)\G\left(\frac{s+\frac{k+1}{2}}{2}\right)L(s,f) = \L(s,f) = \w_{N}(f)i^{k}\L(1-s,\conj{f}).
        \]
      \end{theorem}
    \subsection*{Beyond Primitivity of of \texorpdfstring{$L(s,f)$}{L(s,f)}}
      We can still obtain analytic continuation of $L(s,f)$ if $f$ is not a primitive Hecke eigenform. Indeed, since the primitive Hecke eigenforms form a basis for the space of newforms, we can prove the following:

      \begin{theorem}\label{thm:analytic_continuation_Hecke}
        For any $f \in \mc{S}_{k}(\G_{1}(N))$, $L(s,f)$ admits analytic continuation to $\C$.
      \end{theorem}
      \begin{proof}
        If $f$ is a newform, this follows from \cref{thm:newforms_characterization_holomorphic,thm:primitive_Hecke_Selberg}. Now suppose $f$ is an oldform. Then there is a divisor $d \mid N$ with $d > 1$ such that
        \[
          f(z) = g(z)+d^{k-1}h(dz) = g(z)+\prod_{p^{r} \mid\mid d}(V_{p}^{r}h)(z),
        \]
        for some $g,h \in \mc{S}_{k}\left(\G_{1}\left(\frac{N}{d}\right)\right)$. Note that $V_{p}h \in \mc{S}_{k}\left(\G_{1}\left(\frac{Np}{d}\right)\right)$ by \cref{lem:twisted_holomorphic_lemma}. The claim now follows by induction on the level of $f$ and that $V_{p}$ clearly preserves primitive Hecke eigenforms.
      \end{proof}
  \section{Hecke-Maass \texorpdfstring{$L$}{L}-functions}
    \subsection*{The Definition \& Euler Product of \texorpdfstring{$L(s,f)$}{L(s,f)}}
      We will investigate the $L$-functions of weight zero Maass cusp forms. Let $f \in \mc{C}_{\nu}(N,\chi)$ and denote its Fourier series by
      \[
        f(z) = \sum_{n \ge 1}a_{f}(n)\left(\sqrt{y}K_{\nu}(2\pi ny)e^{2\pi inx}+\frac{a_{f}(-n)}{a_{f}(n)}\sqrt{y}K_{\nu}(2\pi ny)e^{-2\pi inx}\right),
      \]
      with $a_{f}(1) = 1$. Thus if $f$ is a Hecke eigenform, then $f$ is even or odd, the Fourier series takes the form
      \[
        f(z) = \sum_{n \ge 1}a_{f}(n)\sqrt{y}K_{\nu}(2\pi ny)\SC(2\pi nx),
      \]
      and the $a_{f}(n)$ are the Hecke eigenvalues of $f$ normalized so that they are constant on average. The \textbf{Hecke-Maass $L$-function}\index{Hecke-Maass $L$-function} $L(s,f)$ of $f$ is defined as an $L$-series:
      \[
        L(s,f) = \sum_{n \ge 1}\frac{a_{f}(n)}{n^{s}}.
      \]
      We will see that $L(s,f)$ is a Selberg class $L$-function if $f$ is a primitive Hecke-Maass eigenform. From now on, we make this assumption about $f$. The Ramanujan-Petersson conjecture for Maass forms is not know so $L(s,f)$ has not been proven to be a Selberg class $L$-function. Although, it is conjectured to be, so throughout we will make this additional assumption. As we have noted, the Hecke relations and the Ramanujan-Petersson conjecture for Maass forms together imply $a_{f}(n) \ll_{\e} n^{\e}$. So $L(s,f)$ is locally absolutely uniformly convergent for $\s > 1+\e$ and hence locally absolutely uniformly convergent for $\s > 1$. The $L$-function will have an Euler product. Indeed, the Hecke relations imply that the coefficients $a_{f}(n)$ are multiplicative and satisfy
      \begin{equation}\label{equ:primitive_Hecke_eigenform_recurrence_for_coefficients_of_Maass_L-function}
        a_{f}(p^{n}) = \begin{cases} a_{f}(p^{n-1})a_{f}(p)-\chi(p)a_{f}(p^{n-2}) & \text{if $p \nmid N$}, \\ (a_{f}(p))^{n} & \text{if $p \mid N$}, \end{cases}
      \end{equation}
      for all primes $p$ and $n \ge 2$. Because $L(s,f)$ converges absolutely in the region $\s > 1$, multiplicativity of the Hecke eigenvalues implies
      \[
        L(s,f) = \prod_{p}\left(\sum_{n \ge 0}\frac{a_{f}(p^{n})}{p^{ns}}\right),
      \]
      in this region. We now simplify the factor inside the product using this \cref{equ:primitive_Hecke_eigenform_recurrence_for_coefficients_of_Maass_L-function}. On the one hand, if $p \nmid N$:
      \begin{align*}
        \sum_{n \ge 0}\frac{a_{f}(p^{n})}{p^{ns}} &= 1+\frac{a_{f}(p)}{p^{s}}+\sum_{n \ge 2}\frac{a_{f}(p^{n})}{p^{ns}} \\
        &= 1+\frac{a_{f}(p)}{p^{s}}+\sum_{n \ge 2}\frac{a_{f}(p^{n-1})a_{f}(p)-\chi(p)a_{f}(p^{n-2})}{p^{ns}} \\
        &= 1+\frac{a_{f}(p)}{p^{s}}+\frac{a_{f}(p)}{p^{s}}\sum_{n \ge 1}\frac{a_{f}(p^{n})}{p^{ns}}-\frac{\chi(p)}{p^{2s}}\sum_{n \ge 0}\frac{a_{f}(p^{n})}{p^{ns}} \\
        &= 1+\left(\frac{a_{f}(p)}{p^{s}}-\frac{\chi(p)}{p^{2s}}\right)\sum_{n \ge 0}\frac{a_{f}(p^{n})}{p^{ns}}.
      \end{align*}
      By isolating the sum we find
      \[
        \sum_{n \ge 0}\frac{a_{f}(p^{n})}{p^{ns}} = \left(1-\frac{a_{f}(p)}{p^{s}}+\frac{\chi(p)}{p^{2s}}\right)^{-1}.
      \]
      On the other hand, if $p \mid N$ we have
      \[
        \sum_{n \ge 0}\frac{a_{f}(p^{n})}{p^{ns}} = \sum_{n \ge 0}\frac{(a_{f}(p))^{n}}{p^{ns}} = \left(1-a_{f}(p)p^{-s}\right)^{-1}.
      \]
      Therefore
      \[
        L(s,f) = \prod_{p \nmid N}(1-a_{f}(p)p^{-s}+\chi(p)p^{-2s})^{-1}\prod_{p \mid N}(1-a_{f}(p)p^{-s})^{-1}.
      \]
      If $p \nmid N$, let $\a_{1}(p)$ and $\a_{2}(p)$ be the roots of $1-a_{f}(p)p^{-s}+\chi(p)p^{-2s}$. That is,
      \[
        (1-\a_{1}(p)p^{-s})(1-\a_{2}(p)p^{-s}) = (1-a_{f}(p)p^{-s}+\chi(p)p^{-2s}).
      \]
      If $p \mid N$, let $\a_{1}(p) = a_{f}(p)$ and $\a_{2}(p) = 0$. We can then express $L(s,f)$ as a degree $2$ Euler product:
      \[
        L(s,f) = \prod_{p}(1-\a_{1}(p)p^{-s})^{-1}(1-\a_{2}(p)p^{-s})^{-1}.
      \]
      The local factor at $p$ is $L_{p}(s,f) = (1-\a_{1}(p)p^{-s})^{-1}(1-\a_{2}(p)p^{-s})^{-1}$ with local roots $\a_{1}(p)$ and $\a_{2}(p)$. Upon applying partial fraction decomposition to the local factor, we find
      \[
        \frac{1}{1-\a_{1}(p)p^{-s}}\frac{1}{1-\a_{2}(p)p^{-s}} = \frac{\frac{\a_{1}(p)}{\a_{1}(p)-\a_{2}(p)}}{1-\a_{1}(p)p^{-s}}+\frac{\frac{-\a_{2}(p)}{\a_{1}(p)-\a_{2}(p)}}{1-\a_{2}(p)p^{-s}}.
      \]
      Expanding both sides as series in $p^{-s}$, and comparing coefficients gives
      \begin{equation}\label{equ:Hecke_Maass_L_function_coefficient_formula}
        a_{f}(p^{n}) = \frac{\a_{1}(p)^{n+1}-\a_{2}(p)^{n+1}}{\a_{1}(p)-\a_{2}(p)}.
      \end{equation}
    \subsection*{The Integral Representation of \texorpdfstring{$L(s,f)$}{L(s,f)}}
      We want to find an integral representation for $L(s,f)$. Recall that $f$ is an eigenfunction for the parity operator $T_{-1}$ with eigenvalue $\pm 1$. Equivalently, $f$ is even if the eigenvalue is $1$ and odd if the eigenvalue is $-1$. The integral representation will depend upon this parity. To handle both cases simultaneously, let $\mf{a} = 0,1$ according to whether $f$ is even or odd. In other words,
      \[
        \mf{a} = \frac{1-a_{f}(-1)}{2}.
      \]
      Now consider the following Mellin transform:
      \[
        \int_{0}^{\infty}\left(\frac{\del}{\del x}^{\mf{a}}f\right)(iy)y^{s-\frac{1}{2}+\mf{a}}\,\frac{dy}{y}.
      \]
      As $f$ has moderate decay at the cusps, this integral exists and defines an analytic function for $\s > 1$. The derivative operator is present because if $f$ is odd, $\SC(x) = i\sin(x)$. In any case, the smoothness of $f$ implies that we may differentiate its Fourier series termwise to obtain
      \[
        \left(\frac{\del}{\del x}^{\mf{a}}f\right)(z) = \sum_{n \ge 1}a_{f}(n)(2\pi in)^{\mf{a}}\sqrt{y}K_{\nu}(2\pi ny)\cos(2\pi nx).
      \]
      Therefore regardless if $f$ is even or odd, the Fourier series of $\left(\frac{\del}{\del x}^{\mf{a}}f\right)(z)$ has $\SC(x) = \cos(x)$ and the integral does not vanish identically. We compute
      \begin{align*}
        \int_{0}^{\infty}\left(\frac{\del}{\del x}^{\mf{a}}f\right)(iy)y^{s-\frac{1}{2}+\mf{a}}\,\frac{dy}{y} &= \int_{0}^{\infty}\sum_{n \ge 1}a_{f}(n)(2\pi in)^{\mf{a}}K_{\nu}(2\pi ny)y^{s+\mf{a}}\,\frac{dy}{y} \\
        &= \sum_{n \ge 1}a_{f}(n)(2\pi in)^{\mf{a}}\int_{0}^{\infty}K_{\nu}(2\pi ny)y^{s+\mf{a}}\,\frac{dy}{y} &&\text{DCT} \\
        &= \sum_{n \ge 1}\frac{a_{f}(n)}{(2\pi)^{s}n^{s}}i^{\mf{a}}\int_{0}^{\infty}K_{\nu}(y)y^{s+\mf{a}}\,\frac{dy}{y} &&\text{$y \to \frac{y}{2\pi n}$} \\
        &= \frac{\G\left(\frac{s+\mf{a}+\nu}{2}\right)\G\left(\frac{s+\mf{a}-\nu}{2}\right)}{2^{2-\mf{a}}\pi^{s}(-i)^{\mf{a}}}\sum_{n \ge 1}\frac{a_{f}(n)}{n^{s}} && \text{\cref{append:Special_Integrals}} \\
        &= \frac{\G\left(\frac{s+\mf{a}+\nu}{2}\right)\G\left(\frac{s+\mf{a}-\nu}{2}\right)}{2^{2-\mf{a}}\pi^{s}(-i)^{\mf{a}}}L(s,f).
      \end{align*}
      This last expression is analytic function for $\s > 1$ and so the integral is too. Rewriting, we have an integral representation
      \begin{equation}\label{equ:integral_representation_Maass_1}
        L(s,f) = \frac{2^{2-\mf{a}}\pi^{s}(-i)^{\mf{a}}}{\G\left(\frac{s+\mf{a}+\nu}{2}\right)\G\left(\frac{s+\mf{a}-\nu}{2}\right)}\int_{0}^{\infty}\left(\frac{\del}{\del x}^{\mf{a}}f\right)(iy)y^{s-\frac{1}{2}+\mf{a}}\,\frac{dy}{y}.
      \end{equation}
      Now split the integral on the right-hand side into two pieces
      \begin{equation}\label{equ:symmetric_integral_Maass_split}
        \int_{0}^{\infty}\left(\frac{\del}{\del x}^{\mf{a}}f\right)(iy)y^{s-\frac{1}{2}+\mf{a}}\,\frac{dy}{y} = \int_{0}^{\frac{1}{\sqrt{N}}}\left(\frac{\del}{\del x}^{\mf{a}}f\right)(iy)y^{s-\frac{1}{2}+\mf{a}}\,\frac{dy}{y}+\int_{\frac{1}{\sqrt{N}}}^{\infty}\left(\frac{\del}{\del x}^{\mf{a}}f\right)(iy)y^{s-\frac{1}{2}+\mf{a}}\,\frac{dy}{y}.
      \end{equation}
      Now we will rewrite the first piece in the same form and symmetrize the result as much as possible. Performing the change of variables $y \to \frac{1}{Ny}$ to the first piece to obtain
      \[
        \int_{\frac{1}{\sqrt{N}}}^{\infty}\left(\frac{\del}{\del x}^{\mf{a}}f\right)\left(\frac{i}{Ny}\right)(Ny)^{-s+\frac{1}{2}-\mf{a}}\,\frac{dy}{y}.
      \]
      We will rewrite this in terms of the Atkin-Lehner operator. But first we require an identity that relates $\frac{\del}{\del x}^{\mf{a}}$ with the Atkin-Lehner operator $\w_{N}$. By the identity theorem it suffices verify this for $z \in \H$ with $|z|$ fixed. Observe that $-\frac{1}{Nz} = \frac{-x}{N|z|^{2}}+\frac{iy}{N|z|^{2}}$. Now differentiate termwise to see that
      \begin{align*}
        \left(\frac{\del}{\del x}^{\mf{a}}\w_{N}f\right)(z) &= \left(\frac{\del}{\del x}^{\mf{a}}\right)f\left(-\frac{1}{Nz}\right) \\
         &= \left(\frac{\del}{\del x}^{\mf{a}}\right)\sum_{n \ge 1}a_{f}(n)\sqrt{\frac{y}{N|z|^{2}}}K_{\nu}(2\pi ny)\SC\left(-2\pi n\frac{x}{N|z|^{2}}\right) \\
        &= (-N|z|^{2})^{-\mf{a}}\sum_{n \ge 1}a_{f}(n)(2\pi in)^{\mf{a}}\sqrt{\frac{y}{N|z|^{2}}}K_{\nu}\left(2\pi n\frac{y}{N|z|^{2}}\right)\cos\left(-2\pi n\frac{x}{N|z|^{2}}\right) \\
        &= (-N|z|^{2})^{-\mf{a}}\left(\frac{\del}{\del x}^{\mf{a}}f\right)\left(-\frac{1}{Nz}\right).
      \end{align*}
      By the identity theorem, we have
      \[
        \left(\frac{\del}{\del x}^{\mf{a}}f\right)\left(-\frac{1}{Nz}\right) = (-N|z|^{2})^{\mf{a}}\left(\frac{\del}{\del x}^{\mf{a}}\w_{N}f\right)(z),
      \]
      for all $z \in \H$. Rewriting in terms of the Atkin-Lehner operator and recalling that $\w_{N}f = \w_{N}(f)\conj{f}$ by \cref{prop:Atkin_Lehner_conjugation_Maass}, we find that
      \begin{align*}
        \int_{\frac{1}{\sqrt{N}}}^{\infty}\left(\frac{\del}{\del x}^{\mf{a}}f\right)\left(\frac{i}{Ny}\right)(Ny)^{-s+\frac{1}{2}-\mf{a}}\,\frac{dy}{y} &= \int_{\frac{1}{\sqrt{N}}}^{\infty}\left(\frac{\del}{\del x}^{\mf{a}}f\right)\left(-\frac{1}{iNy}\right)(Ny)^{-s+\frac{1}{2}-\mf{a}}\,\frac{dy}{y} \\
        &= \int_{\frac{1}{\sqrt{N}}}^{\infty}(-Ny^{2})^{\mf{a}}\left(\left(\frac{\del}{\del x}^{\mf{a}}\right)\w_{N}f\right)(iy)(Ny)^{-s+\frac{1}{2}-\mf{a}}\,\frac{dy}{y} \\
        &= \int_{\frac{1}{\sqrt{N}}}^{\infty}(-Ny^{2})^{\mf{a}}\w_{N}(f)\left(\left(\frac{\del}{\del x}^{\mf{a}}\right)\conj{f}\right)(iy)(Ny)^{-s+\frac{1}{2}-\mf{a}}\,\frac{dy}{y} \\
        &= w_{N}(f)(-1)^{\mf{a}}N^{\frac{1}{2}-s}\int_{\frac{1}{\sqrt{N}}}^{\infty}\left(\left(\frac{\del}{\del x}^{\mf{a}}\right)\conj{f}\right)(iy)y^{(1-s)-\frac{1}{2}+\mf{a}}\,\frac{dy}{y}.
      \end{align*}
      Substituting this result back into \cref{equ:symmetric_integral_Maass_split} with \cref{equ:integral_representation_Maass_1} gives the integral representation
      \begin{align*}
        L(s,f) &= \frac{2^{2-\mf{a}}\pi^{s}(-i)^{\mf{a}}}{\G\left(\frac{s+\mf{a}+\nu}{2}\right)\G\left(\frac{s+\mf{a}-\nu}{2}\right)} \\
        &\cdot \left[w_{N}(f)(-1)^{\mf{a}}N^{\frac{1}{2}-s}\int_{\frac{1}{\sqrt{N}}}^{\infty}\left(\left(\frac{\del}{\del x}^{\mf{a}}\right)\conj{f}\right)(iy)y^{(1-s)-\frac{1}{2}+\mf{a}}\,\frac{dy}{y}+\int_{\frac{1}{\sqrt{N}}}^{\infty}\left(\frac{\del}{\del x}^{\mf{a}}f\right)(iy)y^{s-\frac{1}{2}+\mf{a}}\,\frac{dy}{y}\right].
      \end{align*}
      This integral representation will give analytic continuation. Indeed, everything outside the brackets is entire and the two integrals are locally absolutely uniformly convergent on $\C$ by \cref{prop:decay_unbounded_inteval_integral}. Hence we have analytic continuation to all of $\C$. In particular, $L(s,f)$ has no poles.
    \subsection*{The Functional Equation \& Critical Strip of \texorpdfstring{$L(s,f)$}{L(s,f)}}
      An immediate consequence of applying the symmetry $s \to 1-s$ to the integral representation is the following functional equation:
      \[
        \frac{\G\left(\frac{s+\mf{a}+\nu}{2}\right)\G\left(\frac{s+\mf{a}-\nu}{2}\right)}{2^{2-\mf{a}}\pi^{s}(-i)^{\mf{a}}}L(s,f) = \w_{N}(f)(-1)^{\mf{a}}N^{-\frac{s}{2}}\frac{\G\left(\frac{(1-s)+\mf{a}+\nu}{2}\right)\G\left(\frac{(1-s)+\mf{a}-\nu}{2}\right)}{2^{2-\mf{a}}\pi^{1-s}(-i)^{\mf{a}}}L(1-s,\conj{f}).
      \]
      The constant factor in the denominator is independent of $s$ and so can be canceled in the functional equation. Therefore we identify the gamma factor as
      \[
        \g(s,f) = \pi^{-s}\G\left(\frac{s+\mf{a}+\nu}{2}\right)\G\left(\frac{s+\mf{a}-\nu}{2}\right),
      \]
      with $\k_{1} = \mf{a}+\nu$ and $\k_{2} = \mf{a}-\nu$ the local roots at infinity (these are complex conjugates because $\nu$ is either purely imaginary of real). The conductor is $q(f) = N$, so the primes dividing the level ramify, and by the Ramanujan-Petersson conjecture for Maass forms, $\a_{1}(p) \neq 0$ and $\a_{2}(p) \neq 0$  for all primes $p \nmid N$. The completed $L$-function is
      \[
        \L(s,f) = N^{-\frac{s}{2}}\pi^{-s}\G\left(\frac{s+\mf{a}+\nu}{2}\right)\G\left(\frac{s+\mf{a}-\nu}{2}\right)L(s,f),
      \]
      with functional equation
      \[
        \L(s,f) = \w_{N}(f)(-1)^{\mf{a}}\L(1-s,\conj{f}).
      \]
      This is the functional equation of $L(s,f)$. From it, the root number is $\e(f) = \w_{N}(f)(-1)^{\mf{a}}$ and we see that $L(s,f)$ has dual $L(s,\conj{f})$. We will now show that $L(s,f)$ is of order $1$. Since $L(s,f)$ has no poles, we do not need to clear any polar divisors. As the integrals in the representation is locally absolutely uniformly convergent, computing the order amounts to estimating the gamma factor. Since the reciprocal of the gamma function is of order $1$, we have
      \[
        \frac{1}{\g(s,f)} \ll_{\e} e^{|s|^{1+\e}}.
      \]
      So the reciprocal of the gamma factor is also of order $1$. Then
      \[
        L(s,f) \ll_{\e} e^{|s|^{1+\e}}.
      \]
      So $L(s,f)$ is of order $1$. We summarize all of our work into the following theorem:

      \begin{theorem}\label{equ:thm:primitive_Hecke-Maass_Selberg}
        For any primitive Hecke-Maass eigenform $f \in \mc{C}_{\nu}(N,\chi)$, $L(s,f)$ is a Selberg class $L$-function provided the Ramanujan-Petersson conjecture for Maass forms holds. For $\s > 1$, it has a degree $2$ Euler product given by 
        \[
          L(s,f) = \prod_{p \nmid N}(1-a_{f}(p)p^{-s}+\chi(p)p^{-2s})^{-1}\prod_{p \mid N}(1-a_{f}(p)p^{-s})^{-1}.
        \]
        Moreover, it admits analytic continuation to $\C$ via the integral representation
        \begin{align*}
          L(s,f) &= \frac{2^{2-\mf{a}}\pi^{s}(-i)^{\mf{a}}}{\G\left(\frac{s+\mf{a}+\nu}{2}\right)\G\left(\frac{s+\mf{a}-\nu}{2}\right)} \\
          &\cdot \left[w_{N}(f)(-1)^{\mf{a}}N^{\frac{1}{2}-s}\int_{\frac{1}{\sqrt{N}}}^{\infty}\left(\left(\frac{\del}{\del x}^{\mf{a}}\right)\conj{f}\right)(iy)y^{(1-s)-\frac{1}{2}+\mf{a}}\,\frac{dy}{y}+\int_{\frac{1}{\sqrt{N}}}^{\infty}\left(\frac{\del}{\del x}^{\mf{a}}f\right)(iy)y^{s-\frac{1}{2}+\mf{a}}\,\frac{dy}{y}\right].
        \end{align*}
        and possesses the functional equation
        \[
          N^{-\frac{s}{2}}\pi^{-s}\G\left(\frac{s+\mf{a}+\nu}{2}\right)\G\left(\frac{s+\mf{a}-\nu}{2}\right)L(s,f) = \L(s,f) = \w_{N}(f)(-1)^{\mf{a}}\L(1-s,\conj{f}).
        \]
      \end{theorem}
    \subsection*{Beyond Primitivity of of \texorpdfstring{$L(s,f)$}{L(s,f)}}
      We can still obtain analytic continuation of $L(s,f)$ if $f$ is not a primitive Hecke-Maass eigenform. Similarly to the Hecke $L$-function case, this holds because the primitive Hecke-Maass eigenforms form a basis for the space of newforms:

      \begin{theorem}\label{thm:analytic_continuation_Hecke-Maass}
        For any $f \in \mc{C}_{\nu}(\G_{1}(N))$, $L(s,f)$ admits analytic continuation to $\C$.
      \end{theorem}
      \begin{proof}
        Argue as in the proof of \cref{thm:analytic_continuation_Hecke}.
      \end{proof}
  \section{The Rankin-Selberg Method}
    \subsection*{The Definition \& Euler Product of \texorpdfstring{$L(s,f \ox g)$}{L(s,f \ox g)}}
      The Rankin-Selberg method is a process by which we can construct new $L$-functions from old ones. Instead of giving the general definition outright, we first provide a full discussion of the method only in the simplest case. Many technical difficulties arise in the fully general setting. Let $f,g \in \mc{S}_{k}(1)$ be primitive Hecke eigenforms with Fourier series
      \[
        f(z) = \sum_{n \ge 1}a_{f}(n)n^{\frac{k-1}{2}}e^{2\pi inz} \quad \text{and} \quad g(z) = \sum_{n \ge 1}a_{g}(n)n^{\frac{k-1}{2}}e^{2\pi inz}.
      \]
      The $L$-function $L(s,f \x g)$ of $f$ and $g$ is given by the $L$-series
      \[
        L(s,f \x g) = \sum_{n \ge 1}\frac{a_{f \x g}(n)}{n^{s}} = \sum_{n \ge 1}\frac{a_{f}(n)\conj{a_{g}(n)}}{n^{s}} = \sum_{n \ge 1}\frac{a_{f}(n)\conj{a_{g}(n)}}{n^{s}},
      \]
      The \textbf{Rankin-Selberg convolution}\index{Rankin-Selberg convolution} $L(s,f \ox g)$ of $f$ and $g$ is defined as an $L$-series:
      \[
        L(s,f \ox g) = \sum_{n \ge 1}\frac{a_{f \ox g}(n)}{n^{s}} = \z(2s)L(s,f \x g),
      \]
      where $a_{f \ox g}(n) = \sum_{n = m\ell^{2}}a_{f}(m)\conj{a_{g}(m)}$. Since $a_{f}(n) \ll_{\e} n^{\e}$ and $a_{g}(n) \ll_{\e} n^{\e}$, $a_{f \x g}(n) \ll_{\e} n^{\e}$ as well. Hence $L(s,f \x g)$ is locally absolutely uniformly convergent for $\s > 1+\e$ and hence locally absolutely uniformly convergent for $\s > 1$. Since $\z(2s)$ is also locally absolutely uniformly convergent in this region, the same follows for $L(s,f \x g)$ too. The $L$-function $L(s,f \x g)$ will also have an Euler product. To see this, let $\a_{j}(p)$ and $\b_{\ell}(p)$ be the local roots at $p$ of $L(s,f)$ and $L(s,g)$ respectively. Since $L(s,f \ox g)$ converges absolutely in the region $\s > 1$, multiplicativity of the Hecke eigenvalues implies
      \[
        L(s,f \ox g) = \z(2s)L(s,f \x g) = \prod_{p \nmid NM}(1-p^{-2s})^{-1}\prod_{p}\left(\sum_{n \ge 0}\frac{a_{f}(p^{n})\conj{a_{g}(p^{n})}}{p^{ns}}\right),
      \]
      in this region. We now simplify the factor inside the latter product using \cref{equ:Hecke_L_function_coefficient_formula}:
      \begingroup
      \allowdisplaybreaks
          \begin{align*}
            \sum_{n \ge 0}\frac{a_{f}(p^{n})\conj{a_{g}(p^{n})}}{p^{ns}} &= \sum_{n \ge 0}\left(\frac{\a_{1}(p)^{n+1}-\a_{2}(p)^{n+1}}{\a_{1}(p)-\a_{2}(p)}\right)\left(\frac{(\conj{\b_{1}(p)})^{n+1}-(\conj{\b_{2}(p)})^{n+1}}{\conj{\b_{1}(p)}-\conj{\b_{2}(p)}}\right)p^{-ns} \\
            &= (\a_{1}(p)-\a_{2}(p))^{-1}\left(\conj{\b_{1}(p)}-\conj{\b_{2}(p)}\right)^{-1} \\
            &\cdot \bigg[\sum_{n \ge 1}\frac{\a_{1}(p)^{n}(\conj{\b_{1}(p)})^{n}}{p^{(n-1)s}}+\frac{\a_{2}(p)^{n}(\conj{\b_{2}(p)})^{n}}{p^{(n-1)s}}-\frac{\a_{1}(p)^{n}(\conj{\b_{2}(p)})^{n}}{p^{(n-1)s}}-\frac{\a_{2}(p)^{n}(\conj{\b_{1}(p)})^{n}}{p^{(n-1)s}}\bigg] \\
            &= (\a_{1}(p)-\a_{2}(p))^{-1}\left(\conj{\b_{1}(p)}-\conj{\b_{2}(p)}\right)^{-1}\bigg[\a_{1}(p)\conj{\b_{1}(p)}\left(1-\a_{1}(p)\conj{\b_{1}(p)}p^{-s}\right)^{-1} \\
            &+\a_{2}(p)\conj{\b_{2}(p)}\left(1-\a_{2}(p)\conj{\b_{2}(p)}p^{-s}\right)^{-1}-\a_{1}(p)\conj{\b_{2}(p)}\left(1-\a_{1}(p)\conj{\b_{2}(p)}p^{-s}\right)^{-1} \\
            &-\a_{2}(p)\conj{\b_{1}(p)}\left(1-\a_{2}(p)\conj{\b_{1}(p)}p^{-s}\right)^{-1}\bigg] \\
            &= (\a_{1}(p)-\a_{2}(p))^{-1}\left(\conj{\b_{1}(p)}-\conj{\b_{2}(p)}\right)^{-1}\left(1-\a_{1}(p)\conj{\b_{1}(p)}p^{-s}\right)^{-1} \\
            &\cdot\left(1-\a_{2}(p)\conj{\b_{2}(p)}p^{-s}\right)^{-1}\left(1-\a_{1}(p)\conj{\b_{2}(p)}p^{-s}\right)^{-1}\left(1-\a_{2}(p)\conj{\b_{1}(p)}p^{-s}\right)^{-1} \\
            &\cdot\bigg[\a_{1}(p)\conj{\b_{1}(p)}\left(1-\a_{2}(p)\conj{\b_{2}(p)}p^{-s}\right)\left(1-\a_{1}(p)\conj{\b_{2}(p)}p^{-s}\right)\left(1-\a_{2}(p)\conj{\b_{1}(p)}p^{-s}\right) \\
            &+\a_{2}(p)\conj{\b_{2}(p)}\left(1-\a_{1}(p)\conj{\b_{1}(p)}p^{-s}\right)\left(1-\a_{1}(p)\conj{\b_{2}(p)}p^{-s}\right)\left(1-\a_{2}(p)\conj{\b_{1}(p)}p^{-s}\right) \\
            &-\a_{1}(p)\conj{\b_{2}(p)}\left(1-\a_{1}(p)\conj{\b_{1}(p)}p^{-s}\right)\left(1-\a_{2}(p)\conj{\b_{2}(p)}p^{-s}\right)\left(1-\a_{2}(p)\conj{\b_{1}(p)}p^{-s}\right) \\
            &-\a_{2}(p)\conj{\b_{1}(p)}\left(1-\a_{1}(p)\conj{\b_{1}(p)}p^{-s}\right)\left(1-\a_{2}(p)\conj{\b_{2}(p)}p^{-s}\right)\left(1-\a_{1}(p)\conj{\b_{2}(p)}p^{-s}\right)\bigg].
          \end{align*}
      \endgroup
      The term in the brackets simplifies to
      \[
        \left(1-\a_{1}(p)\a_{2}(p)\conj{\b_{1}(p)}\conj{\b_{2}(p)}p^{-2s}\right)(\a_{1}(p)-\a_{2}(p))\left(\conj{\b_{1}(p)}-\conj{\b_{2}(p)}\right),
      \]
      because all of the other terms are killed by symmetry in $\a_{1}(p)$, $\a_{2}(p)$, $\conj{\b_{1}(p)}$, and $\conj{\b_{2}(p)}$. The Ramanujan-Petersson conjecture for holomorphic forms implies $\a_{1}(p)\a_{2}(p)\conj{\b_{1}(p)}\conj{\b_{2}(p)} = 1$. Therefore the corresponding factor above is $(1-p^{-2s})$. This factor cancels the local factor at $p$ in the Euler product of $\z(2s)$, so that
      \[
        \sum_{n \ge 0}\frac{a_{f}(p^{n})\conj{a_{g}(p^{n})}}{p^{ns}} = \prod_{1 \le j,\ell \le 2}\left(1-\a_{j}(p)\conj{\b_{\ell}(p)}p^{-s}\right)^{-1}.
      \]
      Hence
      \[
        L(s,f \ox g) = \prod_{p}\left(\sum_{n \ge 0}\frac{a_{f}(p^{n})\conj{a_{g}(p^{n})}}{p^{ns}}\right).
      \]
      In total we have a degree $4$ Euler product:
      \[
        L(s,f \ox g) = \prod_{p}\prod_{1 \le j,\ell \le 2}\left(1-\a_{j}(p)\conj{\b_{\ell}(p)}p^{-s}\right)^{-1}.
      \]
      The local factor at $p$ is $L_{p}(s,f \ox g) = \prod_{1 \le j,\ell \le 2}\left(1-\a_{j}(p)\conj{\b_{\ell}(p)}p^{-s}\right)^{-1}$.
    \subsection*{The Integral Representation of \texorpdfstring{$L(s,f \ox g)$}{L(s,f \ox g)}: Part I}
      We now look for an integral representation for $L(s,f \ox g)$. Consider the following integral:
      \[
        \int_{\G_{\infty}\backslash\H}f(z)\conj{g(z)}\Im(z)^{s+k}\,d\mu.
      \]
      This will turn out to be a Mellin transform as we will soon see. Since $f$ and $g$ have rapid decay, this integral exists and defines an analytic function for $\s > 1$. We have
      \begin{align*}
        \int_{\G_{\infty}\backslash\H}f(z)\conj{g(z)}\Im(z)^{s+k}\,d\mu &= \int_{0}^{\infty}\int_{0}^{1}f(x+iy)\conj{g(x+iy)}y^{s+k}\,\frac{dx\,dy}{y^{2}} \\
        &= \int_{0}^{\infty}\int_{0}^{1}\sum_{n,m \ge 1}a_{f}(n)\conj{a_{g}(m)}(nm)^{\frac{k-1}{2}}e^{2\pi i(n-m)x}e^{-2\pi(n+m)y}y^{s+k}\,\frac{dx\,dy}{y^{2}} \\
        &= \int_{0}^{\infty}\sum_{n,m \ge 1}\int_{0}^{1}a_{f}(n)\conj{a_{g}(m)}(nm)^{\frac{k-1}{2}}e^{2\pi i(n-m)x}e^{-2\pi(n+m)y}y^{s+k}\,\frac{dx\,dy}{y^{2}} && \text{DCT} \\
        &= \int_{0}^{\infty}\sum_{n \ge 1}a_{f}(n)\conj{a_{g}(n)}n^{k-1}e^{-4\pi ny}y^{s+k}\,\frac{dy}{y^{2}},
      \end{align*}
      where the last line follows by \cref{equ:Dirac_integral_representation}. Observe that this last integral is a Mellin transform. The rest is a computation:
      \begin{align*}
        \int_{0}^{\infty}\sum_{n \ge 1}a_{f}(n)\conj{a_{g}(n)}n^{k-1}e^{-4\pi ny}y^{s+k}\,\frac{dy}{y^{2}} &= \sum_{n \ge 1}a_{f}(n)\conj{a_{g}(n)}n^{k-1}\int_{0}^{\infty}e^{-4\pi ny}y^{s+k}\,\frac{dy}{y^{2}} &&\text{DCT} \\
        &= \sum_{n \ge 1}\frac{a_{f}(n)\conj{a_{g}(n)}}{(4\pi)^{s+k-1}n^{s}}\int_{0}^{\infty}e^{-y}y^{s+k-1}\,\frac{dy}{y} &&\text{$y \to \frac{y}{4\pi n}$} \\
        &= \frac{\G\left(s+k-1\right)}{(4\pi)^{s+k-1}}\sum_{n \ge 1}\frac{a_{f}(n)\conj{a_{g}(n)}}{n^{s}} \\
        &= \frac{\G\left(s+k-1\right)}{(4\pi)^{s+k-1}}L(s,f \x g).
      \end{align*}
      Rewriting, we have an integral representation
      \[
        L(s,f \x g) = \frac{(4\pi)^{s+k-1}}{\G(s+k-1)}\int_{\G_{\infty}\backslash\H}f(z)\conj{g(z)}\Im(z)^{s+k}\,d\mu.
      \]
      We rewrite the integral as follows:
      \begin{align*}
        \int_{\G_{\infty}\backslash\H}f(z)\conj{g(z)}\Im(z)^{s+k}\,d\mu &= \int_{\mc{F}}\sum_{\g \in \GG}f(\g z)\conj{g(\g z)}\Im(\g z)^{s+k}\,d\mu && \text{folding} \\
        &= \int_{\mc{F}}\sum_{\g \in \GG}j(\g,z)^{k}\conj{j(\g,z)^{k}}f(z)\conj{g(z)}\Im(\g z)^{s+k}\,d\mu && \text{modularity} \\
        &= \int_{\mc{F}}f(z)\conj{g(z)}\sum_{\g \in \GG}|j(\g,z)|^{2k}\Im(\g z)^{s+k}\,d\mu \\
        &= \int_{\mc{F}}f(z)\conj{g(z)}\Im(z)^{k}\sum_{\g \in \GG}\Im(\g z)^{s}\,d\mu \\
        &= \int_{\mc{F}}f(z)\conj{g(z)}\Im(z)^{k}E(z,s)\,d\mu.
      \end{align*}
      Note that $E(z,s)$ is the weight zero Eisenstein series on $\G_{1}(1)\backslash\H$ at the $\infty$ cusp. Altogether, this gives the integral representation
      \begin{equation}\label{equ:Rankin-Selberg_integral-reresentation}
        L(s,f \x g) =  \frac{(4\pi)^{s+k-1}}{\G(s+k-1)}\int_{\mc{F}}f(z)\conj{g(z)}\Im(z)^{k}E(z,s)\,d\mu.
      \end{equation}
      We cannot investigate the integral any further until we understand the Fourier series of $E(z,s)$ and have a functional equation as $s \to 1-s$. Therefore we will take a necessary detour and return to the integral after.
    \subsection*{The Fourier Series and Functional Equation of \texorpdfstring{$E(z,s)$}{E(z,s)}}
      We will compute the Fourier series of $E(z,s)$. To do this we will need the following technical lemma:

      \begin{lemma}\label{lem:Ramanujan_zeta_relation}
        For $\s > 1$ and $b \in \Z$,
        \[
          \sum_{m \ge 1}\frac{r(b,m)}{m^{2s}} = \begin{cases} \frac{\z(2s-1)}{\z(2s)} & \text{if $b = 0$}, \\ \frac{\s_{1-2s}(|b|)}{\z(2s)} & \text{if $b \neq 0$}, \end{cases}
        \]
        where $\s_{s}(b)$ is the generalized sum of divisors function.
      \end{lemma}
      \begin{proof}
        If $\s > 1$ then the desired evaluation of the sum is locally absolutely uniformly convergent because the Riemann zeta function is in that region. Hence the sum will be too provided we prove the identity. Suppose $b = 0$. Then $r(0,m) = \vphi(m)$. Since $\vphi(m)$ is multiplicative we have
        \begin{equation}\label{equ:Ramanujan_zeta_relation_1}
          \sum_{m \ge 1}\frac{\vphi(m)}{m^{2s}} = \prod_{p}\left(\sum_{k \ge 0}\frac{\vphi(p^{k})}{p^{k(2s)}}\right).
        \end{equation}
        Recalling that $\vphi(p^{k}) = p^{k}-p^{k-1}$ for $k \ge 1$, make the following computation:
        \begin{equation}\label{equ:Ramanujan_zeta_relation_2}
          \begin{aligned}
            \sum_{k \ge 0}\frac{\vphi(p^{k})}{p^{k(2s)}} &= 1+\sum_{k \ge 1}\frac{p^{k}-p^{k-1}}{p^{k(2s)}} \\
            &= \sum_{k \ge 0}\frac{1}{p^{k(2s-1)}}-\frac{1}{p}\sum_{k \ge 1}\frac{1}{p^{k(2s-1)}} \\
            &= \sum_{k \ge 0}\frac{1}{p^{k(2s-1)}}-p^{-2s}\sum_{k \ge 0}\frac{1}{p^{k(2s-1)}} \\
            &= (1-p^{-2s})\sum_{k \ge 0}\frac{1}{p^{k(2s-1)}} \\
            &= \frac{1-p^{-2s}}{1-p^{-(2s-1)}}.
          \end{aligned}
        \end{equation}
        Combining \cref{equ:Ramanujan_zeta_relation_1,equ:Ramanujan_zeta_relation_2} gives
        \[
          \sum_{m \ge 1}\frac{\vphi(m)}{m^{2s}} = \frac{\z(2s-1)}{\z(2s)}.
        \]
        Now suppose $b \neq 0$, \cref{prop:Ramanujan_sum_evaluation} gives the first equality in the following chain:
        \begin{align*}
          \sum_{m \ge 1}\frac{r(b,m)}{m^{2s}} &= \sum_{m \ge 1}m^{-2s}\sum_{\ell \mid (b,m)}\ell\mu\left(\frac{m}{\ell}\right) \\
          &= \sum_{\ell \mid b}\ell\sum_{m \ge 1}\frac{\mu(m)}{(m\ell)^{2s}} \\
          &= \left(\sum_{\ell \mid b}\ell^{1-2s}\right)\left(\sum_{m \ge 1}\frac{\mu(m)}{m^{2s}}\right) \\
          &= \s_{1-2s}(b)\sum_{m \ge 1}\frac{\mu(m)}{m^{2s}} \\
          &= \s_{1-2s}(|b|)\sum_{m \ge 1}\frac{\mu(m)}{m^{2s}} \\
          &= \frac{\s_{1-2s}(|b|)}{\z(2s)} && \text{\cref{prop:Dirichlet_Mobius_is_zeta_inverse}}.
        \end{align*}
      \end{proof}

      We can now compute the Fourier series of $E(z,s)$:

      \begin{proposition}\label{prop:Fourier_coefficients_of_real-analytic_Eisenstein_series}
        The Fourier series of $E(z,s)$ is given by
        \[
          E(z,s) = y^{s}+y^{1-s}\frac{\sqrt{\pi}\G\left(s-\frac{1}{2}\right)\z(2s-1)}{\G(s)\z(2s)}+\sum_{t \ge 1}\left(\frac{2\pi^{s}|t|^{s-\frac{1}{2}\s_{1-2s}(|t|)}}{\G(s)\z(2s)}\sqrt{y}K_{s-\frac{1}{2}}(2\pi|t|y)\right)e^{2\pi itx}.
        \]
      \end{proposition}
      \begin{proof}
        Fix $s$ with $\s > 1$. By the Bruhat decomposition for $\G_{1}(1)$ and \cref{rem:Bruhat_modulo_infity_exact}, we have
        \[
          E(z,s) = \Im(z)^{s}+\sum_{\substack{c \ge 1, d \in \Z \\ (c,d) = 1}}\frac{\Im(z)^{s}}{|cz+d|^{2s}}.
        \]
        Summing over all pairs $(c,d) \in \Z^{2}-\{\mathbf{0}\}$ with $c \ge 1$, $d \in \Z$, and $(c,d) = 1$ is the same as summing over all triples $(c,\ell,r)$ with $c \ge 1$, $\ell \in \Z$, $r$ taken modulo $c$, and $(r,c) = 1$. This is seen by writing $d = c\ell+r$. Therefore
        \[
          \sum_{\substack{c \ge 1, d \in \Z \\ (c,d) = 1}}\frac{\Im(z)^{s}}{|cx+icy+d|^{2s}} = \sum_{(c,\ell,r)}\frac{\Im(z)^{s}}{|cz+c\ell+r|^{2s}} = \psum_{\substack{c \ge 1 \\ r \tmod{c}}}\sum_{\ell \in \Z}\frac{\Im(z)^{s}}{|cz+c\ell+r|^{2s}}.
        \]
         where on the right-hand side it is understood that we are summing over all triples $(c,\ell,r)$ with the prescribed properties. Now let
        \[
          I_{c,r}(z,s) = \sum_{\ell \in \Z}\frac{\Im(z)^{s}}{|cz+c\ell+r|^{2s}}.
        \]
        We apply the Poisson summation formula to $I_{c,r}(z,s)$. This is allowed since the summands are absolutely integrable by \cref{prop:decay_unbounded_inteval_integral}, as they exhibit polynomial decay of order $\s > 1$, and $I_{c,r}(z,s)$ is holomorphic because $E(z,s)$ is. By the identity theorem it suffices to apply the Poisson summation formula for $z = iy$ with $y > 0$. So let $f(x)$ be given by
        \[
          f(x) = \frac{y^{s}}{|cx+r+icy|^{2s}}.
        \]
        Then $f(x)$ is absolutely integrable on $\R$ as we have just mentioned. We compute the Fourier transform:
        \begin{align*}
          \hat{f}(t) = \int_{-\infty}^{\infty}f(x)e^{-2\pi itx}\,dx &= \int_{-\infty}^{\infty}\frac{y^{s}}{|cx+r+icy|^{2s}}e^{-2\pi itx}\,dx \\
          &= \int_{-\infty}^{\infty}\frac{y^{s}}{((cx+r)^{2}+(cy)^{2})^{s}}e^{-2\pi itx}\,dx \\
          &= e^{2\pi it\frac{r}{c}}\int_{-\infty}^{\infty}\frac{y^{s}}{((cx)^{2}+(cy)^{2})^{s}}e^{-2\pi itx}\,dx && \text{$x \to x-\frac{r}{c}$} \\
          &= \frac{e^{2\pi it\frac{r}{c}}}{c^{2s}}\int_{-\infty}^{\infty}\frac{y^{s}}{(x^{2}+y^{2})^{s}}e^{-2\pi itx}\,dx \\
          &= \frac{e^{2\pi it\frac{r}{c}}}{c^{2s}}\int_{-\infty}^{\infty}\frac{y^{s+1}}{((xy)^{2}+y^{2})^{s}}e^{-2\pi itxy}\,dx && \text{$x \to xy$} \\
          &= \frac{e^{2\pi it\frac{r}{c}}}{c^{2s}}\int_{-\infty}^{\infty}\frac{y^{1-s}}{(x^{2}+1)^{s}}e^{-2\pi itxy}\,dx.
        \end{align*}
        Appealing to \cref{append:Special_Integrals} to compute this latter integral, we see that
        \[
          \hat{f}(t) = \begin{cases} \frac{y^{1-s}}{c^{2s}}\frac{\sqrt{\pi}\G\left(s-\frac{1}{2}\right)}{\G(s)} & \text{if $t = 0$}, \\ \frac{e^{2\pi it\frac{r}{c}}}{c^{2s}}\frac{2\pi^{s}|t|^{s-\frac{1}{2}}}{\G(s)}\sqrt{y}K_{s-\frac{1}{2}}(2\pi|t|y) & \text{if $t \neq 0$}. \end{cases}
        \]
        By the Poisson summation formula and the identity theorem, we have
        \[
          I_{c,r}(z,s) = \frac{y^{1-s}}{c^{2s}}\frac{\sqrt{\pi}\G\left(s-\frac{1}{2}\right)}{\G(s)}+\sum_{t \neq 0}\left(\frac{e^{2\pi it\frac{r}{c}}}{c^{2s}}\frac{2\pi^{s}|t|^{s-\frac{1}{2}}}{\G(s)}\sqrt{y}K_{s-\frac{1}{2}}(2\pi|t|y)\right)e^{2\pi itx},
        \]
        for all $z \in \H$. Substituting this back into the Eisenstein series gives a form of the Fourier series:
      \begin{align*}
        E_(z,s) &= y^{s}+\psum_{\substack{c \ge 1 \\ r \tmod{c}}}\left(\frac{y^{1-s}}{c^{2s}}\frac{\sqrt{\pi}\G\left(s-\frac{1}{2}\right)}{\G(s)}+\sum_{t \ge 1}\left(\frac{e^{2\pi it\frac{r}{c}}}{c^{2s}}\frac{2\pi^{s}|t|^{s-\frac{1}{2}}}{\G(s)}\sqrt{y}K_{s-\frac{1}{2}}(2\pi|t|y)\right)e^{2\pi itx}\right) \\
        &= y^{s}+y^{1-s}\psum_{\substack{c \ge 1 \\ r \tmod{c}}}\frac{1}{c^{2s}}\frac{\sqrt{\pi}\G\left(s-\frac{1}{2}\right)}{\G(s)}+\sum_{t \ge 1}\left(\psum_{\substack{c \ge 1 \\ r \tmod{c}}}\frac{e^{2\pi it\frac{r}{c}}}{c^{2s}}\frac{2\pi^{s}|t|^{s-\frac{1}{2}}}{\G(s)}\sqrt{y}K_{s-\frac{1}{2}}(2\pi|t|y)\right)e^{2\pi itx} \\
        &= y^{s}+y^{1-s}\sum_{c \ge 1}\frac{r(0,c)}{c^{2s}}\frac{\sqrt{\pi}\G\left(s-\frac{1}{2}\right)}{\G(s)}+\sum_{t \ge 1}\left(\sum_{c \ge 1}\frac{r(t,c)}{c^{2s}}\frac{2\pi^{s}|t|^{s-\frac{1}{2}}}{\G(s)}\sqrt{y}K_{s-\frac{1}{2}}(2\pi|t|y)\right)e^{2\pi itx}.
      \end{align*}
      By applying \cref{lem:Ramanujan_zeta_relation} to compute the Dirichlet series of Ramanujan sums, we obtain the desired Fourier series:
      \[
        E(z,s) = y^{s}+y^{1-s}\frac{\sqrt{\pi}\G\left(s-\frac{1}{2}\right)\z(2s-1)}{\G(s)\z(2s)}+\sum_{t \ge 1}\left(\frac{2\pi^{s}|t|^{s-\frac{1}{2}\s_{1-2s}(|t|)}}{\G(s)\z(2s)}\sqrt{y}K_{s-\frac{1}{2}}(2\pi|t|y)\right)e^{2\pi itx}.
      \]
      \end{proof}

      Having computed the Fourier series, we would like to obtain a functional equation for $E(z,s)$ as $s \to 1-s$. To this end, we define $E^{\ast}(z,s)$ by
      \[
        E^{\ast}(z,s) = \L(2s,\z)E(z,s) = \pi^{-s}\G(s)\z(2s)E(z,s).
      \]
      From \cref{prop:Fourier_coefficients_of_real-analytic_Eisenstein_series}, the Fourier coefficients $a^{\ast}(n,y,s)$ of $E^{\ast}(z,s)$ in the Fourier series
      \[
        E^{\ast}(z,s) = a^{\ast}(0,y,s)+\sum_{n \neq 0}a^{\ast}(n,y,s)e^{2\pi inx},
      \]
       are given by
      \[
        a^{\ast}(n,y,s) = \begin{cases} y^{s}\pi^{-s}\G(s)\z(2s)+y^{1-s}\pi^{-\left(s-\frac{1}{2}\right)}\G(s-\frac{1}{2})\z(2s-1) & \text{if $n = 0$}, \\ 2|n|^{s-\frac{1}{2}}\s_{1-2s}(|n|)\sqrt{y}K_{s-\frac{1}{2}}(2\pi|n|y) & \text{if $n \neq 0$}. \end{cases}
      \]
      We can now derive a functional equation for $E^{\ast}(z,s)$. Using the definition and functional equation for $\L(2s-1,\z)$, we can rewrite the second term in the constant coefficient to get
      \begin{equation}\label{equ:Fourier_coefficients_for_completed_real-analytic_Eisenstein_series}
        a^{\ast}(n,y,s) = \begin{cases} y^{s}\L(2s,\z)+y^{1-s}\L(2(1-s),\z) & \text{if $n = 0$}, \\ 2|n|^{s-\frac{1}{2}}\s_{1-2s}(|n|)\sqrt{y}K_{s-\frac{1}{2}}(2\pi|n|y) & \text{if $n \neq 0$}. \end{cases}
      \end{equation}
      Now observe that the constant coefficient is invariant under $s \to 1-s$. Each $n \neq 0$ coefficient is also invariant under $s \to 1-s$. To see this we will use two facts. First, from \cref{append:Bessel_Functions}, $K_{s}(y)$ is invariant under $s \to -s$ and so $K_{s-\frac{1}{2}}(2\pi|n|y)$ is invariant as $s \to 1-s$. Second, for $n \ge 1$ we have
      \[
        n^{s-\frac{1}{2}}\s_{1-2s}(n) = n^{\frac{1}{2}-s}n^{2s-1}\s_{1-2s}(n) = n^{\frac{1}{2}-s}n^{2s-1}\sum_{d \mid n}d^{1-2s} = n^{\frac{1}{2}-s}\sum_{d \mid n}\left(\frac{n}{d}\right)^{2s-1} = n^{\frac{1}{2}-s}\s_{2s-1}(n),
      \]
      where the second to last equality follows by writing $n^{2s-1} = \left(\frac{n}{d}\right)^{2s-1}d^{2s-1}$ for each $d \mid n$. These two facts together give the invariance of the $n \neq 0$ coefficients under $s \to 1-s$. Altogether, we have shown the following functional equation for $E^{\ast}(z,s)$:
      \[
        E^{\ast}(z,s) = E^{\ast}(z,1-s).
      \]
      We can now obtain meromorphic continuation of $E^{\ast}(z,s)$ in $s$ to all of $\C$ for any $z \in \H$. We first write $E^{\ast}(z,s)$ as a Fourier series using \cref{equ:Fourier_coefficients_for_completed_real-analytic_Eisenstein_series}:
      \[
        E^{\ast}(z,s) = y^{s}\L(2s,\z)+y^{1-s}\L(2(1-s),\z)+\sum_{n \neq 0}2|n|^{s-\frac{1}{2}}\s_{1-2s}(|n|)\sqrt{y}K_{s-\frac{1}{2}}(2\pi|n|y)e^{2\pi inx}.
      \]
      Since $\L(2s,\z)$ is meromorphic on $\C$, the constant term of $E^{\ast}(z,s)$ is as well. To finish the meromorphic continuation of $E^{\ast}(z,s)$ it now suffices to show
      \[
        \sum_{n \neq 0}2|n|^{s-\frac{1}{2}}\s_{1-2s}(|n|)\sqrt{y}K_{s-\frac{1}{2}}(2\pi|n|y)e^{2\pi inx},
      \]
      is meromorphic on $\C$. We will actually prove it is locally absolutely uniformly convergent on this region. So let $K$ be a compact subset of $\C$. Then we have to show $E^{\ast}(z,s)$ is absolutely convergent on $K$ for any $z \in \H$. To achieve this we need two bounds, one for $\s_{1-2s}(|n|)$ and one for $K_{s-\frac{1}{2}}(2\pi|n|y)$. For the first bound, we use the estimate $\s_{0}(|n|) \ll_{\e} |n|^{\e}$ (recall \cref{prop:sum_of_divisors_growth_rate}). Therefore we have the crude bound
      \[
        \s_{1-2s}(|n|) = \sum_{d \mid n}d^{1-2s} < \s_{0}(|n|)|n|^{1-2s} \ll_{\e}|n|^{1-2s+\e}.
      \]
      For the second estimate, \cref{lem:K_Bessel_function_asymptotic} implies
      \[
        K_{s-\frac{1}{2}}(2\pi|n|y) \ll e^{-2\pi|n|y}.
      \]
      Using these two estimates, we have
      \begin{equation}\label{equ:non-constant_Fourier_coefficient_bound_non-holomorphic_Eisenstein_series}
        \sum_{n \neq 0}2|n|^{s-\frac{1}{2}}\s_{1-2s}(|n|)\sqrt{y}K_{s-\frac{1}{2}}(2\pi|n|y)e^{2\pi inx} \ll_{\e} \sum_{n \ge 1}4n^{\frac{1}{2}-s+\e}\sqrt{y}e^{-2\pi ny}.
      \end{equation}
      This latter series is absolutely uniformly convergent on $K$ by the ratio test. Therefore $E^{\ast}(z,s)$ is absolutely convergent on $K$ for any $z \in \H$ and the meromorphic continuation to $\C$ follows. It remains to investigate the poles and residues. We will accomplish this from direct inspection of the Fourier coefficients:

      \begin{proposition}\label{equ:completed_real-analytic_Eisenstein_series_residues}
        $E^{\ast}(z,s)$ has simple poles at $s = 0$ and $s = 1$, and
        \[
          \Res_{s = 0}E^{\ast}(z,s) = -\frac{1}{2} \quad \text{and} \quad \Res_{s = 1}E^{\ast}(z,s) = \frac{1}{2}.
        \]
      \end{proposition}
      \begin{proof}
        Since the constant term in the Fourier series of $E^{\ast}(z,s)$ is the only non-holomorphic term, poles of $E^{\ast}(z,s)$ can only come from that term. So we are reduced to understanding the poles of
        \begin{equation}\label{equ:constant_coefficient_of_completed_non-holomorphic_Eisenstein_series}
          y^{s}\L(2s,\z)+y^{1-s}\L(2(1-s),\z).
        \end{equation}
        Notice $\L(2s,\z)$ has simple poles at $s = 0$, $s = \frac{1}{2}$ (one from the Riemann zeta function and one from the gamma factor) and no others. It follows that $E^{\ast}(z,s)$ has a simple pole at $s = 0$ coming from the $y^{s}$ term in
        \cref{equ:constant_coefficient_of_completed_non-holomorphic_Eisenstein_series}, and by the functional equation there is also a pole at $s = 1$ coming from the $y^{1-s}$ term. At $s = \frac{1}{2}$, both terms in \cref{equ:constant_coefficient_of_completed_non-holomorphic_Eisenstein_series} have simple poles and we will show that the singularity there is removable. Recall $\G\left(\frac{1}{2}\right) = \sqrt{\pi}$. Also, by \cref{prop:zeta_residue}, $\Res_{s = \frac{1}{2}}\z(2s) = \frac{1}{2}$ and $\Res_{s = \frac{1}{2}}\z(2(1-s)) = -\frac{1}{2}$. So altogether
        \[
          \Res_{s = \frac{1}{2}}E^{\ast}(z,s) = \Res_{s = \frac{1}{2}}[y^{s}\L(2s,\z)+y^{1-s}\L(2(1-s),\z)] = \frac{1}{2}y^{\frac{1}{2}}-\frac{1}{2}y^{\frac{1}{2}} = 0.
        \]
        Hence the singularity at $s = \frac{1}{2}$ is removable. As for the residues at $s = 0$ and $s = 1$, the functional equation implies that they are negatives of each other. So it suffices to compute the residue at $s = 0$. Recall $\z(0) = -\frac{1}{2}$ and $\Res_{s = 0}\G(s) = 1$. Then together we find
        \[
          \Res_{s = 0}E^{\ast}(z,s) = \Res_{s = 0}y^{s}\L(2s,\z) = -\frac{1}{2}.
        \]
      \end{proof}

      This completes our study of $E(z,s)$.
    \subsection*{The Integral Representation of \texorpdfstring{$L(s,f \ox g)$}{L(s,f \ox g)}: Part II}
      We can now continue with the Rankin-Selberg convolution $L(s,f \ox g)$. Writing \cref{equ:Rankin-Selberg_integral-reresentation} in terms of $E^{\ast}(z,s)$ and $L(s,f \ox g)$ results in the integral representation
      \[
        L(s,f \ox g) = \frac{(4\pi)^{s+k-1}\pi^{s}}{\G(s+k-1)\G(s)}\int_{\mc{F}}f(z)\conj{g(z)}\Im(z)^{k}E^{\ast}(z,s)\,d\mu.
      \]
      This integral representation will give analytic continuation. To see this, note that the gamma factors are analytic for $\s < 0$. By the functional equation for $E^{\ast}(z,s)$, the integral is invariant as $s \to 1-s$. These two facts together give analytic continuation to $\C$ outside of the critical strip. The continuation inside of the critical strip will be meromorphic because of the poles of $E^{\ast}(z,s)$. To see this, taking the integral representation and substituting the Fourier series for $E^{\ast}(z,s)$ gives
      \begin{equation}\label{equ:integral_representation_Rankin_Selberg}
        \begin{aligned}
          L(s,f \ox g) &= \frac{(4\pi)^{s+k-1}\pi^{s}}{\G(s+k-1)\G(s)}\bigg[\int_{\mc{F}}f(x+iy)\conj{g(x+iy)}y^{k}(y^{s}\L(2s,\z)+y^{1-s}\L(2(1-s),\z))\,\frac{dx\,dy}{y^{2}} \\
          &+\int_{\mc{F}}f(x+iy)\conj{g(x+iy)}y^{k}\sum_{n \neq 0}2|n|^{s-\frac{1}{2}}\s_{1-2s}(|n|)\sqrt{y}K_{s-\frac{1}{2}}(2\pi|n|y)e^{2\pi inx}\,\frac{dx\,dy}{y^{2}}\bigg],
        \end{aligned}
      \end{equation}
      and we are reduced to showing that both integrals are locally absolutely uniformly convergent in the critical strip and away from the poles of $E^{\ast}(z,s)$ . Indeed, the first integral is locally absolutely uniformly convergent in this region by \cref{prop:decay_finite_volume_integral}. As for the second integral, \cref{equ:non-constant_Fourier_coefficient_bound_non-holomorphic_Eisenstein_series} implies that it is
      \[
        O_{\e}\left(\int_{\mc{F}}f(x+iy)\conj{g(x+iy)}y^{k}\sum_{n \ge 1}4n^{\frac{1}{2}-s+\e}\sqrt{y}e^{-2\pi ny}\,\frac{dx\,dy}{y^{2}}\right).
      \]
      As the sum in the integrand is holomorphic, we can now appeal to \cref{prop:decay_finite_volume_integral}. The meromorphic continuation to the critical strip and hence to all of $\C$ follows. In particular, $L(s,f \ox g)$ has at most simple poles at $s = 0$ and $s = 1$. Actually, there is no pole at $s = 0$. Indeed, $\g(s,f \ox g)$ has a simple pole at $s = 0$ coming from the gamma factors and therefore its reciprocal has a simple zero. This cancels the simple pole at $s = 0$ coming from $E^{\ast}(z,s)$ and therefore $L(s,f \ox g)$ is has a removable singularity at $s = 0$. So there is at worst a simple pole at $s = 1$.
    \subsection*{The Functional Equation, Critical Strip \& Residues of \texorpdfstring{$L(s,f \ox g)$}{L(s,f \ox g)}}
      An immediate consequence of the symmetry of integral representation is the functional equation:
      \[
        \frac{\G(s+k-1)\G(s)}{(4\pi)^{s+k-1}\pi^{s}}L(s,f \ox g) = \frac{\G((1-s)+k-1)\G(1-s)}{(4\pi)^{(1-s)+k-1}\pi^{1-s}}L(1-s,f \ox g).
      \]
      Applying the Legendre duplication formula for the gamma function twice we see that
      \begin{equation}\label{equ:duplication_for_Rankin-Selberg_gamma_factor}
        \begin{aligned}
          \frac{\G(s+k-1)\G(s)}{(4\pi)^{s+k-1}\pi^{s}} &= \frac{2^{2s+k-3}}{(4\pi)^{s+k-1}\pi^{s+1}}\G\left(\frac{s+k-1}{2}\right)\G\left(\frac{s+k}{2}\right)\G\left(\frac{s}{2}\right)\G\left(\frac{s+1}{2}\right) \\
          &= \frac{1}{2^{k+1}\pi^{k}}\pi^{-2s}\G\left(\frac{s+k-1}{2}\right)\G\left(\frac{s+k}{2}\right)\G\left(\frac{s}{2}\right)\G\left(\frac{s+1}{2}\right).
        \end{aligned}
      \end{equation}
      The factor in front is independent of $s$ and can therefore be canceled in the functional equation. We identify the gamma factor as:
      \[
        \g(s,f \ox g) = \pi^{-2s}\G\left(\frac{s+k-1}{2}\right)\G\left(\frac{s+k}{2}\right)\G\left(\frac{s}{2}\right)\G\left(\frac{s+1}{2}\right),
      \]
      with $\mu_{1,1} = k-1$, $\mu_{2,2} = k$, $\mu_{1,2} = 0$, and $\mu_{2,1} = 1$ the local roots at infinity. The completed $L$-function is
      \[
        \L(s,f \ox g) = \pi^{-2s}\G\left(\frac{s+k-1}{2}\right)\G\left(\frac{s+k}{2}\right)\G\left(\frac{s}{2}\right)\G\left(\frac{s+1}{2}\right)L(s,f \ox g),
      \]
      so the conductor is $q(f \ox g) = 1$ and no primes ramify. Clearly $q(f \ox g) \mid q(f)^{2}q(g)^{2}$ and conductor dropping does not occur. Then
      \[
        \L(s,f \ox g) = \L(1-s,f \ox g),
      \]
      is the functional equation of $L(s,f \ox g)$. In particular, the root number $\e(f \ox g) = 1$, and $L(s,f \ox g)$ is self-dual. We can now show that $L(s,f \ox g)$ is of order $1$. Since the possible pole at $s = 1$ is simple, multiplying by $(s-1)$ clears the possible polar divisor. As the integrals in the integral representation are locally absolutely uniformly convergent, computing the order amounts to estimating the gamma factor. Since the reciprocal of the gamma function is of order $1$, we have
      \[
        \frac{1}{\g(s,f \ox g)} \ll_{\e} e^{|s|^{1+\e}}.
      \]
      So the reciprocal of the gamma factor is also of order $1$. Then we find that
      \[
        (s-1)L(s,f \ox g) \ll_{\e} e^{|s|^{1+\e}}.
      \]
      Thus $(s-1)L(s,f \ox g)$ is of order $1$, and so $L(s,f \ox g)$ is as well after removing the polar divisor. At last, we compute the residue of $L(s,f \ox g)$ at $s = 1$:

      \begin{proposition}
        Let $f,g \in \mc{S}_{k}(1)$ be primitive Hecke eigenforms. Then
        \[
          \Res_{s = 1}L(s,f \ox g) = \frac{4^{k}\pi^{k+1}V}{2\G(k)}\<f,g\>,
        \]
        where $\<f,g\>$ is the Petersson inner product.
      \end{proposition}
      \begin{proof}
        As $V = \frac{\pi}{3}$, \cref{equ:completed_real-analytic_Eisenstein_series_residues} implies
        \[
          \Res_{s = 1}L(s,f \ox g) = \frac{4^{k}\pi^{k+1}}{\G(k)}\Res_{s = 1}\int_{\mc{F}}f(z)\conj{g(z)}\Im(z)^{k}E^{\ast}(z,s)\,d\mu = \frac{4^{k}\pi^{k+1}V}{2\G(k)}\<f,g\>.
        \]
      \end{proof}

      Notice that if $f = g$, then $\<f,f\> \neq 0$ and therefore the residue at $s = 1$ is not zero and hence not a removable singularity. Actually, this is the only instance in which there is a pole since \cref{thm:newforms_characterization_holomorphic} implies that the primitive Hecke eigenforms are orthogonal so that $\<f,g\> = 0$ unless $f = g$. We summarize all of our work into the following theorem:

      \begin{theorem}
        For any two primitive Hecke eigenforms $f,g \in \mc{S}_{k}(1)$, $L(s,f \ox g)$ is a Selberg class $L$-function. For $\s > 1$, it has a degree $4$ Euler product given by
        \[
          L(s,f \ox g) = \prod_{p}\left(\sum_{n \ge 0}\frac{a_{f}(p^{n})\conj{a_{g}(p^{n})}}{p^{ns}}\right).
        \]
        Moreover, it admits meromorphic continuation to $\C$ via the integral representation
        \[
          L(s,f \ox g) = \frac{(4\pi)^{s+k-1}\pi^{s}}{\G(s+k-1)\G(s)}\int_{\mc{F}}f(z)\conj{g(z)}\Im(z)^{k}E^{\ast}(z,s)\,d\mu,
        \]
        possesses the functional equation
        \[
          \pi^{-2s}\G\left(\frac{s+k-1}{2}\right)\G\left(\frac{s+k}{2}\right)\G\left(\frac{s}{2}\right)\G\left(\frac{s+1}{2}\right)L(s,f \ox g) = \L(s,f \ox g) = \L(1-s,f \ox g),
        \]
        and if $f = g$ there is simple pole at $s = 1$ of residue $\frac{4^{k}\pi^{k+1}V}{2\G(k)}\<f,g\>$.
      \end{theorem}
    \subsection*{The Rankin-Selberg Method}
      The Rankin-Selberg method is much more complicated in general, but the argument is essentially the same. Let $f$ and $g$ both be primitive Hecke or Heck-Maass eigenforms with Fourier coefficients $a_{f}(n)$ and $a_{g}(n)$ respectively. We suppose $f$ has weight $k$/type $\nu$, level $N$, and character $\chi$, and $g$ has weight $\ell$/type $\eta$, level $M$, and character $\psi$. 
      The $L$-function $L(s,f \x g)$ of $f$ and $g$ is given by the $L$-series
      \[
        L(s,f \x g) = \sum_{n \ge 1}\frac{a_{f \x g}(n)}{n^{s}} = \sum_{n \ge 1}\frac{a_{f}(n)\conj{a_{g}(n)}}{n^{s}}.
      \]
      The \textbf{Rankin-Selberg convolution}\index{Rankin-Selberg convolution} $L(s,f \ox g)$ of $f$ and $g$ is defined as an $L$-series:
      \[
        L(s,f \ox g) = \sum_{n \ge 1}\frac{a_{f \ox g}(n)}{n^{s}} = L(2s,\chi\conj{\psi})L(s,f \x g),
      \]
      where $a_{f \ox g}(n) = \sum_{n = m\ell^{2}}\chi\conj{\psi}(\ell^{2})a_{f}(m)\conj{a_{g}(m)}$. The following argument is the \textbf{Rankin-Selberg method}\index{Rankin-Selberg method}:

      \begin{method}[Rankin-Selberg method]
        Let $f$ and $b$ both be primitive Hecke or Heck-Maass eigenforms. Also suppose the following: 
        \begin{enumerate}[label=(\roman*)]
          \item $f$ has weight $k$/type $\nu$, level $N$, and character $\chi$.
          \item $g$ has weight $\ell$/type $\eta$, level $M$, and character $\psi$.
          \item The Ramanujan-Petersson conjecture for Maass forms holds if $f$ or $g$ are Heck-Maass eigenforms.
        \end{enumerate}
        Then the Rankin-Selberg convolution $L(s,f \ox g)$ is a Selberg class $L$-function.
      \end{method}

      We make a few remarks about the Rankin-Selberg method. Local absolute uniform convergence for $\s > 1$ are proved in the exactly the same way as we have described. The argument for the Euler product is also similar. However, if either $N > 1$ or $M > 1$, the computation becomes more difficult to compute since the local $p$ factors for $p \mid NM$ change. Moreover, the situation is increasingly complicated if $(N,M) > 1$ since conductor dropping can occur. The integral representation has a similar argument, but if the weights/types are distinct the resulting Eisenstein series becomes more complicated. In particular, it is on $\G_{0}(NM)\backslash\H$ and if $NM > 1$, then there is more than just the cusp at $\infty$. Therefore the functional equation of the Eisenstein series at the $\infty$ cusp then reflects into a linear combination of Eisenstein series at the other cusps. This results in the requirement to compute the Fourier coefficients of all of these Eisenstein series. Moreover, this procedure can be generalized to remove the primitive Hecke and/or primitive Hecke-Maass eigenform conditions by taking linear combinations, but we won't attempt discussing this further. 
  \section{Applications of the Rankin-Selberg Method}
    \subsection*{The Ramanujan-Petersson Conjecture on Average}
      Let $f$ be a Hecke or Hecke-Maass eigenform. Using Rankin-Selberg convolutions, it is possible to show the weaker result that $a_{f}(n) \ll_{\e} n^{\e}$ holds on average without assuming the corresponding Ramanujan-Petersson conjecture:
      
      \begin{proposition}\label{prop:Ramanujan_Petersson_average}
        Let $f$ be a primitive Hecke or Hecke-Maass eigenform. Then for any $X > 0$, we have
        \[
        \sum_{n \le X}|a_{f}(n)| \ll_{\e} X^{1+\e},
        \]
      \end{proposition}
      \begin{proof}
        By the Cauchy-Schwarz inequality,
        \begin{equation}\label{equ:Ramanujan_conjecture_on_average_1}
          \left(\sum_{n \le X}|a_{f}(n)|\right)^{2} \le X\sum_{n \le X}|a_{f}(n)|^{2},
        \end{equation}
        The Rankin-Selberg square $L(s,f \ox f)$ is locally absolutely uniformly convergent for $\s > \frac{3}{2}$. Therefore it still admits meromorphic continuation to $\C$ with a simple pole at $s = 1$. By Landau's theorem, the abscissa of absolute convergence of $L(s,f \ox f)$, and hence $L(s,f \x f)$ too, is $1$ so by \cref{prop:Dirichlet_series_coefficient_size_on_average} we have
        \[
          \sum_{n \le X}|a_{f}(n)|^{2} \ll_{\e} X^{1+\e},
        \]
        for any $\e > 0$. Substituting this bound into \cref{equ:Ramanujan_conjecture_on_average_1}, we obtain
        \[
          \left(\sum_{n \le X}|a_{f}(n)|\right)^{2} \ll_{\e} X^{2+\e},
        \]
        and taking the square root yields
        \[
          \sum_{n \le X}|a_{f}(n)| \ll_{\e} X^{1+\e}.
        \]
      \end{proof}
      
      The bound in \cref{prop:Ramanujan_Petersson_average} should be compared with the implication $a_{f}(n) \ll_{\e} n^{\e}$ that follows from the corresponding Ramanujan-Petersson conjecture. While \cref{prop:Ramanujan_Petersson_average} is not useful in the holomorphic form case, it is in the Maass form case. Indeed, recall that if $f$ is a primitive Hecke-Maass eigenform we needed to assume the Ramanujan-Petersson conjecture for Maass forms to ensure $a_{f}(n) \ll_{\e} n^{\e}$ so that $L(s,f)$ was locally absolutely uniformly convergent for $\s > 1$. However, \cref{prop:Dirichlet_series_convergence_polynomial_bound_average,prop:Ramanujan_Petersson_average} together imply that $L(s,f)$ is locally absolutely uniformly convergent for $\s > 1+\e$ and hence for $\s > 1$.
    \subsection*{Strong Multiplicity One}
      Let $f$ be a primitive Hecke or Hecke-Maass eigenform. Then $f$ is determined by Hecke eigenvalues at primes for fixed weight/type, and level. Using Rankin-Selberg convolutions, we can prove \textbf{strong multiplicity one}\index{strong multiplicity one} for holomorphic or Maass forms which says that $f$ is determined by Hecke eigenvalues at all but finitely many primes:

      \begin{theorem}[Strong multiplicity one, holomorphic and Maass versions]
        Let $f$ and $g$ both be primitive Hecke or Hecke-Maass eigenforms. Denote the Hecke eigenvalues by $\l_{f}(n)$ and $\l_{g}(n)$ respectively. If $\l_{f}(p) = \l_{g}(p)$ for all but finitely many primes $p$, then $f = g$.
      \end{theorem}
      \begin{proof}
        Let $S$ be the set the primes for which $\l_{f}(p) \neq \l_{g}(p)$ including the primes that ramify for $L(s,f)$ and $L(s,g)$. By assumption, $S$ is finite. As the local factors of $L(s,f \ox g)$ are holomorphic and nonzero at $s = 1$, the order of the pole of $L(s,f \ox g)$ is the same as the order of the pole of
        \[
          L(s,f \ox g)\prod_{p \in S}L_{p}(s,f \ox g)^{-1} = \prod_{p \notin S}L_{p}(s,f \ox g).
        \]
        But as $\l_{f}(p) = \l_{g}(p)$ for all $p \notin S$, we have
        \[
          \prod_{p \notin S}L_{p}(s,f \ox g) = \prod_{p \notin S}L_{p}(s,f \ox f),
        \]
        and so
        \[
          L(s,f \ox g)\prod_{p \in S}L_{p}(s,f \ox g)^{-1} = L(s,f \ox f)\prod_{p \in S}L_{p}(s,f \ox f)^{-1}.
        \]
        Since $L(s,f \ox f)$ has a simple pole at $s = 1$, it follows that $L(s,f \ox g)$ does too. But then $f = g$.
      \end{proof}