\chapter{Types of \texorpdfstring{$L$}{L}-functions}
   We discuss a variety of types of Selberg class $L$-functions: the Riemann zeta function, $L$-functions attached to Dirichlet characters, and Hecke $L$-functions. In the case of Hecke $L$-functions, we also describe a method of Rankin and Selberg for constructing new $L$-functions from old ones.
    \section{The Riemann Zeta Function}
        \subsection*{The Definition \& Euler Product of \texorpdfstring{$\z$}{\z}(s)}
        The \textbf{Riemann zeta function}\index{Riemann zeta function} or simply the \textbf{zeta function}\index{zeta function} $\z(s)$ is defined as an $L$-series:
        \[
            \z(s) = \sum_{n \ge 1}\frac{1}{n^{s}}.
        \]
        This is the prototypical example of a Dirichlet series as all the coefficients are $1$. Our main goal is to show that $\z(s)$ belongs to the Selberg class. As the coefficients are trivially polynomially bounded, $\z(s)$ is locally absolutely uniformly convergent for $\Re(s) > 1$. Also note that $\z(s)$ is necessiarly nonzero in this region. Determining the Euler product is also an easy matter. As the coefficients are obviously completely multiplicative, we have the degree $1$ Euler product
        \[
            \z(s) = \prod_{p}(1-p^{-s})^{-1},
        \]
        in this region as well. The local factor at $p$ is $(1-p^{-s})^{-1}$ with local root $1$. In particular, we have shown that the zeta function satisfies property (i) of the Selberg class and we package this into a theorem:

        \begin{theorem}
            For $\Re(s) > 1$,
            \[
            \z(s) = \prod_{p}(1-p^{-s})^{-1},
            \]
            is locally absolutely uniformly convergent with degree $1$ Euler product.
        \end{theorem}
        \subsection*{The Integral Representation of \texorpdfstring{$\z(s)$}{\z(s)}: Part I}
        Riemann's ingenious insight was to analytically continue $\z(s)$. By this, he sought to find a representation of $\z(s)$ defined on a larger region than $\Re(s) > 1$. This is the approach we will take, and the argument follows the same line of reasoning as that of Riemann. We consider the gamma function $\G\left(\frac{s}{2}\right)$:
        \[
            \G\left(\frac{s}{2}\right) = \int_{0}^{\infty}e^{-x}x^{\frac{s}{2}}\,\frac{dx}{x}.
        \]

        \begin{remark}
            We have chosen to express the gamma function in terms of the measure $\frac{dx}{x}$ instead of $dx$. This is a tactical change for two reasons. The first is that $\frac{dx}{x}$ is invariant under the change of variables $x \to Cx$ for any constant $C$. The second is that under the change of variables $x \to \frac{1}{x}$ we have $\frac{dx}{x} \to -\frac{dx}{x}$ but the bounds of integration are also flipped. So we may leave the measure invariant provided we dont flip the bounds of integration. These types of change of variables are essential in the study of $L$-functions which motivates the use of this measure.
        \end{remark}

        Performing the change of variables $x \to \pi n^{2}x$ for fixed $n \ge 1$ yields
        \begin{equation}\label{equ:gamma_integral_substitution}
            \G\left(\frac{s}{2}\right) = \pi^{\frac{s}{2}} n^{s}\int_{0}^{\infty}e^{-\pi n^{2}x}x^{\frac{s}{2}}\,\frac{dx}{x}.
        \end{equation}
        Dividing by $\pi^{\frac{s}{2}}n^{s}$ and summing over $n \ge 1$, we see that for $\Re(s) > 1$,
        \begin{align*}
            \pi^{-\frac{s}{2}}\G\left(\frac{s}{2}\right)\z(s) &= \sum_{n \ge 1}\int_{0}^{\infty}e^{-\pi n^{2}x}x^{\frac{s}{2}}\,\frac{dx}{x} \\
            &= \int_{0}^{\infty}\sum_{n \ge 1}e^{-\pi n^{2}x}x^{\frac{s}{2}}\,\frac{dx}{x} && \text{DCT} \\
            &= \int_{0}^{\infty}\w(x)x^{\frac{s}{2}}\,\frac{dx}{x},
        \end{align*}
        where we set
        \[
            \w(x) = \sum_{n \ge 1}e^{-\pi n^{2}x}.
        \]
        Therefore we have an integral representation
        \begin{equation}\label{equ:integral_representation_zeta_1}
            \z(s) = \frac{\pi^{\frac{s}{2}}}{\G\left(\frac{s}{2}\right)}\int_{0}^{\infty}\w(x)x^{\frac{s}{2}}\,\frac{dx}{x}.
        \end{equation}
        This was essentially Riemann's insight: rewrite the zeta function in terms of a Mellin transform. Unfortunately, we cannot proceed until we understand $\w(x)$. So we will make a slight detour and come back to the integral representation after.
        \subsection*{Jacobi's Theta Function \texorpdfstring{$\vt(s)$}{\vt(s)}}
        \textbf{Jacobi's theta function}\index{Jacobi's theta function} $\vt(s)$ is defined for $\Re(s) > 0$ by
        \[
            \vt(s) = \sum_{n \in \Z}e^{-\pi n^{2}s} = 1+2\sum_{n \ge 1}e^{-\pi n^{2}s}.
        \]
        It is locally absolutely uniformly convergent in this region by the ratio test. It's relation to $\w(s)$ is the identity
        \begin{equation}\label{equ:omega_theta_relationship_for_zeta}
            \w(s) = \frac{\vt(s)-1}{2}.
        \end{equation}
        The essential fact about Jacobi's theta function we will need is the \textbf{transformation law for Jacobi's theta funtion}\index{transformation law for Jacobi's theta funtion} that was known to Riemann:

        \begin{theorem}[Transformation law for Jacobi's theta function]
            For $\Re(s) > 0$,
            \[
            \vt(s) = \frac{1}{\sqrt{s}}\vt\left(\frac{1}{s}\right).
            \]
        \end{theorem}
        \begin{proof}
            By the identity theorem it suffices to prove this on a set containg a limit point. We will prove this on the right-half of the real line, so take $s$ real with $s > 0$. Set $f(x) = e^{-\pi x^{2}s}$. Then $f(x)$ is a Schwarz function. We compute its Fourier transform:
            \[
            \hat{f}(t) = \int_{-\infty}^{\infty}f(x)e^{-2\pi itx}\,dx = \int_{-\infty}^{\infty}e^{-\pi x^{2}s}e^{-2\pi itx}\,dx = \int_{-\infty}^{\infty}e^{-\pi(x^{2}s+2itx)}\,dx.
            \]
            Making the change of variables $x \to \frac{x}{\sqrt{s}}$, the last integral above becomes
            \[
            \frac{1}{\sqrt{s}}\int_{-\infty}^{\infty}e^{-\pi\left(x^{2}+\frac{2itx}{\sqrt{s}}\right)}.
            \]
            Complete the square in the exponent by noticing
            \[
            -\pi\left(x^{2}+\frac{2itx}{\sqrt{s}}\right) = -\pi\left(\left(x+\frac{it}{\sqrt{s}}\right)^{2}+\frac{t^{2}}{s^{2}}\right).
            \]
            Taking exponentials, this implies that the previous integral is equal to
            \[
            \frac{e^{-\frac{\pi t^{2}}{s}}}{\sqrt{s}}\int_{-\infty}^{\infty}e^{-\pi\left(x+\frac{it}{\sqrt{s}}\right)^{2}}\,dx.
            \]
            The change of variables $x \to \frac{x}{\sqrt{s}}-\frac{it}{\sqrt{s}}$ is permitted without affecting the line of integration by viewing the integral as a complex integral, noting that the integrand is entire as a complex function, and shifting the line of integration. This gives
            \[
            \frac{e^{-\frac{\pi t^{2}}{s}}}{\sqrt{s}}\int_{-\infty}^{\infty}e^{-\pi\left(x+\frac{it}{\sqrt{s}}\right)^{2}}\,dx = \frac{e^{-\frac{\pi t^{2}}{s}}}{\sqrt{s}}\int_{-\infty}^{\infty}e^{-\pi x^{2}}\,dx = \frac{e^{-\frac{\pi t^{2}}{s}}}{\sqrt{s}},
            \]
            where the last equality follows because the last integral above is $1$ since it is the Gaussian integral (see \cref{append:Special_Integrals}). The Poisson summation formula and the identity theorem together finish the proof.
        \end{proof}

        This completes our interest in Jacobi's theta function.
        
        \subsection*{The Integral Representation of \texorpdfstring{$\z(s)$}{\z(s)}: Part II}
        Returning to the Riemann zeta function, we split the integral in \cref{equ:integral_representation_zeta_1} into two pieces
        \begin{equation}\label{equ:symmetric_integral_zeta_split}
            \int_{0}^{\infty}\w(x)x^{\frac{s}{2}}\,\frac{dx}{x} = \int_{0}^{1}\w(x)x^{\frac{s}{2}}\,\frac{dx}{x}+\int_{1}^{\infty}\w(x)x^{\frac{s}{2}}\,\frac{dx}{x}.
        \end{equation}
        Since $\w(x)$ has exponential decay to zero as $x \to \infty$, the second piece is locally absolutely uniformly bounded for $\Re(s) > 1$ by \cref{met:decay_compacta_integral}. Hence it defines an analytic function there. The idea now is to rewrite the first piece in the same form and symmetrize the result as much as possible. We being by performing a change of variables $x \to \frac{1}{x}$ to the first piece to obtain
        \[
            \int_{1}^{\infty}\w\left(\frac{1}{x}\right)x^{-\frac{s}{2}}\,\frac{dx}{x}
        \]
        Now the transformation law for $\vt(x)$ and \cref{equ:omega_theta_relationship_for_zeta} together imply
        \begin{equation}\label{equ:piece_one_zeta_1}
            \w\left(\frac{1}{x}\right) = \frac{\vt\left(\frac{1}{x}\right)-1}{2} = \frac{\sqrt{x}\vt(x)-1}{2} = \frac{\sqrt{x}(2\w(x)+1)-1}{2} = \sqrt{x}\w(x)+\frac{\sqrt{x}}{2}-\frac{1}{2}.
        \end{equation}
        \cref{equ:piece_one_zeta_1} gives the first equality in the following chain:
        \begin{align*}
            \int_{1}^{\infty}\w\left(\frac{1}{x}\right)x^{-\frac{s}{2}}\,\frac{dx}{x} &= \int_{1}^{\infty}\left(\sqrt{x}\w(x)+\frac{\sqrt{x}}{2}-\frac{1}{2}\right)x^{-\frac{s}{2}}\,\frac{dx}{x} \\
            &= \int_{1}^{\infty}\w(x)x^{\frac{1-s}{2}}\,\frac{dx}{x}+\int_{1}^{\infty}\frac{x^{\frac{1-s}{2}}}{2}\,\frac{dx}{x}-\int_{1}^{\infty}\frac{x^{-\frac{s}{2}}}{2}\,\frac{dx}{x} \\
            &= \int_{1}^{\infty}\w(x)x^{\frac{1-s}{2}}\,\frac{dx}{x}+\frac{1}{1-s}-\frac{1}{s} \\
            &= \int_{1}^{\infty}\w(x)x^{\frac{1-s}{2}}\,\frac{dx}{x}-\frac{1}{s(1-s)}.
        \end{align*}
        Substituting this result back into \cref{equ:symmetric_integral_zeta_split} with \cref{equ:integral_representation_zeta_1} yields the following result:

        \begin{theorem}
            For $\Re(s) > 1$,
            \[
            \z(s) = \frac{\pi^{\frac{s}{2}}}{\G\left(\frac{s}{2}\right)}\left[-\frac{1}{s(1-s)}+\int_{1}^{\infty}\w(x)x^{\frac{1-s}{2}}\,\frac{dx}{x}+\int_{1}^{\infty}\w(x)x^{\frac{s}{2}}\,\frac{dx}{x}\right].
            \]
        \end{theorem}

        This integral representation will give analytic continuation. To see this, first observe that we know everything outside the backets is entire. Everything inside the brackets except for the first integral is analytic for $\Re(s) > 1$. Thus the first integral must be analytic in this region too. Since the two integrals are interchanged as $s \to 1-s$ and the rational term $-\frac{1}{s(1-s)}$ is invariant, the right-hand side is analytic for $\Re(s) < 0$. This gives the analytic continuation of $\z(s)$ to the region
        \[
            \left\{s \in \C:\left|\Re(s)-\frac{1}{2}\right| > \frac{1}{2}\right\}.
        \]
        \subsection*{The Functional Equation, Critical Strip \& Residue of \texorpdfstring{$\z(s)$}{\z(s)}}
        An immediate consequence of the symmetry of the integral representation is the functional equation:
        \[
            \frac{\G\left(\frac{s}{2}\right)}{\pi^{\frac{s}{2}}}\z(s) = \frac{\G\left(\frac{1-s}{2}\right)}{\pi^{\frac{1-s}{2}}}\z(1-s).
        \]
        We identify the gamma factor as
        \[
            \g(s,\z) = \pi^{-\frac{s}{2}}\G\left(\frac{s}{2}\right),
        \]
        with $\k = \frac{s}{2}$ the only local parameter at infinity. Clearly it satisfies the required bounds. The conductor is $q(\z) = 1$ so no primes ramify. The completed zeta function is
        \[
            \L(s,\z) = \pi^{-\frac{s}{2}}\G\left(\frac{s}{2}\right)\z(s),
        \]
        with functional equation
        \[
            \L(s,\z) = \L(1-s,\z).
        \]
        This is the functional equation of $\z(s)$ and in this case is just a reformulation of the previous functional equation. From it we find that the root number is $\e(\z) = 1$ and that $\z$ is self-dual. Altogether, we have shown that $\z$ satisfies properties (ii)-(iv) of the Selberg class.

        Having obtained the funtional equation, we now use the integral representation to obtain meromorphic continuation of $\z(s)$ inside the critical strip. To get continuation inside of the critical strip, we argue that the two integrals in the integral representation are locally absolutely uniformly bounded for $|\Re(s)-\frac{1}{2}| \le \frac{1}{2}$. This follows by \cref{met:decay_compacta_integral} (we could have gotten this continuation earlier but we didn't need it until now). If we assume $s \neq 0,1$, then the analytic continuation to the inside of the critical strip follows since the fractional term $\frac{1}{s(1-s)}$ is holomorphic there. The cases $s = 0,1$ require separate inspection. When $s = 0$, $\g(s,\z)$ has a simple pole coming from the gamma factor, and therefore its reciprocal has a simple zero. This cancels the corresponding simple pole of $\frac{1}{s(1-s)}$ so that $\z(s)$ has a removable singularity and thus is holomorphic at $s = 0$. At $s = 1$, $\g(s,\z)$ is nonzero, and so $\z(s)$ has a simple pole. Therefore $\z(s)$ has meromorphic continuation to all of $\C$ with a simple pole at $s = 1$.

        We can now show that the order of $\z(s)$ is $1$ and conclude that it satisfies property (v) of the Selberg class, and is therefore an $L$-function belonging to the Selberg class. As there is only a simple pole at $s = 1$, multiply by $(s-1)$ to clear the polar divisor. Now the integral in the integral representation is absolutely bounded, so computing the order amounts to estimating the gamma factor. Since the reciprocal of the gamma function is of order $1$ and $\Re(s)$ is bounded, we have
        \begin{equation}\label{equ:zeta_function_gamma_factor_order_1}
            \frac{1}{\g(s,\z)} \ll_{\e} e^{|s|^{1+\e}},
        \end{equation}
        for any $\e > 0$. So the reciprocal of the gamma factor is of the same order. Then \cref{equ:zeta_function_gamma_factor_order_1} and the integral representation together imply
        \[
            (s-1)\z(s) \ll_{\e} e^{|s|^{1+\e}}.
        \]
        So $(s-1)\z(s)$ is of order $1$, and thus $\z(s)$ is as well after removing the polar factor. Having shown the analytic continuation of $\z(s)$, and verified that it belongs to the Selberg class, there is only one thing left to do. This is to compute the residue of $\z(s)$ at $s = 1$:

        \begin{proposition}\label{prop:zeta_residue}
            \[
            \Res_{s = 1}\z(s) = 1.
            \]
        \end{proposition}
        \begin{proof}
            The only term in the integral representation of $\z(s)$ contributing to the pole is $-\frac{1}{\g(s,\z)}\frac{1}{s(1-s)}$. Observe
            \[
            \lim_{s \to 1}\frac{1}{\g(s,\z)} = \lim_{s \to 1}\frac{\pi^{\frac{s}{2}}}{\G\left(\frac{s}{2}\right)} = 1,
            \]
            because $\G\left(\frac{1}{2}\right) = \sqrt{\pi}$. Therefore
            \[
            \Res_{s = 1}\z(s) = \Res_{s = 1}-\frac{1}{\g(s,\z)}\frac{1}{s(1-s)} = \Res_{s = 1}-\frac{1}{s(1-s)} = \lim_{s \to 1}-\frac{(s-1)}{s(1-s)} = 1.
            \]
        \end{proof}

        We summarize all of our work into the following theorem:

        \begin{theorem}
            $\z(s)$ is a Selberg class $L$-function. It admits meromorphic continuation to $\C$ via the integral representation
            \[
            \z(s) = \frac{\pi^{\frac{s}{2}}}{\G\left(\frac{s}{2}\right)}\left[-\frac{1}{s(1-s)}+\int_{1}^{\infty}\w(x)x^{\frac{1-s}{2}}\,\frac{dx}{x}+\int_{1}^{\infty}\w(x)x^{\frac{s}{2}}\,\frac{dx}{x}\right],
            \]
            with functional equation
            \[
            \pi^{-\frac{s}{2}}\G\left(\frac{s}{2}\right)\z(s) = \L(s,\z) = \L(1-s,\z),
            \]
            and there is a simple pole at $s = 1$ of residue $1$.
        \end{theorem}

        Lastly, we note that by virtue of the functional equation we can also compute $\z(0)$ as well. Indeed, since $\Res_{s = 1}\z(s) = 1$, we have
            \[
            \lim_{s \to 1}(s-1)\L(s,\z) = \Res_{s = 1}\z(s)\lim_{s \to 1}\pi^{-\frac{s}{2}}\G\left(\frac{s}{2}\right) = 1.
            \]
            In other words, $\L(s,\z)$ has a simple pole at $s = 1$ with residue $1$ too. Since the completed zeta function is completely symmetric as $s \to 1-s$, it has a simple pole at $s = 0$ with residue $1$. Hence
            \[
            1 = \lim_{s \to 1}(s-1)\L(1-s,\z) = \Res_{s = 1}\G\left(\frac{1-s}{2}\right)\lim_{s \to 1}\pi^{-\frac{1-s}{2}}\z(1-s) = -2\z(0),
            \]
            because $\Res_{s = 0}\G(s) = 1$. Therefore $\z(0) = -\frac{1}{2}$.
    \section{Dirichlet \texorpdfstring{$L$}{L}-functions}
        \subsection*{The Definition \& Euler Product of \texorpdfstring{$L(s,\chi)$}{L(s,\chi)}}
        To every Dirichlet character $\chi$ there is an associated $L$-function. Throughout we will let $m$ denote the modulus and $q$ the conductor of $\chi$ respectively. The \textbf{Dirichlet $L$-function}\index{Dirichlet $L$-function} $L(s,\chi)$ attached to the Dirichlet character $\chi$ is defined as an $L$-series:
        \[
            L(s,\chi) = \sum_{n \ge 1}\frac{\chi(n)}{n^{s}}.
        \]
        Since $\chi(n) = 0$ if $(n,m) > 1$, the above sum can be restricted to all integers relatively prime to $m$. We first obtain convergence in a half-plane. As $|\chi(n)| \ll 1$, $L(s,\chi)$ is locally absolutely uniformly convergent for $\Re(s) > 1$ just as is the case for $\z(s)$. Because $\chi$ is completely multiplicative we also have the degree $1$ Euler product,
        \[
            L(s,\chi) = \prod_{p}(1-\chi(p)p^{-s})^{-1} = \prod_{p \nmid m}(1-\chi(p)p^{-s})^{-1},
        \]
        in this region as well. The last equality holds because if $p \mid m$ we have $\chi(p) = 0$. So for $p \mid m$, the local factor at $p$ is $1$ with local root $0$. For $p \nmid m$ the local factor at $p$ is $(1-\chi(p)p^{-s})^{-1}$ with local root $\chi(p)$. Now if $\chi$ is induced by $\wtilde{\chi}$, then $\chi(p) = \wtilde{\chi}(p)$ if $p \nmid q$ and $\chi(p) = 0$ if $p \mid m$ so that
        \begin{equation}\label{equ:non-primitive_primitive_Dirichlet_L-series_relation}
            L(s,\chi) = \prod_{p \nmid m}(1-\wtilde{\chi}(p)p^{-s})^{-1} = \prod_{p}(1-\wtilde{\chi}(p)p^{-s})^{-1}\prod_{p \mid m}(1-\wtilde{\chi}(p)p^{-s}) = L(s,\wtilde{\chi})\prod_{p \mid m}(1-\wtilde{\chi}(p)p^{-s}).
        \end{equation}
        Therefore $L(s,\chi)$ belongs to the Selberg class if and only if $L(s,\wtilde{\chi})$ does. So we may assume $\chi$ is primitive. We may assume further that $q > 1$ because if not $\chi$ is principal which means $\wtilde{\chi}$ is trivial so that $L(s,\wtilde{\chi}) = \z(s)$, and this $L$-function already belongs to the Selberg class. Our main goal is now to show that $L(s,\chi)$ is a Selberg class $L$-function when $\chi$ is primitive and $q > 1$. We have already shown that $L(s,\chi)$ satisfies property (i) of the Selberg class and we package this into a theorem:

        \begin{theorem}
            Let $\chi$ be a primitive Dirichlet character with conductor $q > 1$. For $\Re(s) > 1$,
            \[
            L(s,\chi) = \prod_{p}(1-\chi(p)p^{-s})^{-1} = \prod_{p \nmid q}(1-\chi(p)p^{-s})^{-1},
            \]
            is locally absolutely uniformly convergent with degree $1$ Euler product.
        \end{theorem}
        \subsection*{The Integral Representation of \texorpdfstring{$L(s,\chi)$}{L(s,\chi)}: Part I}
        To find the integral representation for $L(s,\chi)$, we argue in a manner similar to $\z(s)$. However, the following will depend if $\chi$ is even or odd, so to handle both cases simultaneously let $\mf{a} = 0,1$ according to whether $\chi$ is even or odd. In particular, $\chi(-1) = (-1)^{\mf{a}}$. Note that $\mf{a}$ takes the same value for both $\chi$ and $\cchi$. Making the substitution $s \to s+\mf{a}$ in \cref{equ:gamma_integral_substitution} and multiplying by $\chi(n)$ yields
        \[
            \chi(n)\G\left(\frac{s+\mf{a}}{2}\right) = \pi^{\frac{s+\mf{a}}{2}} n^{s}\int_{0}^{\infty}\chi(n)n^{\mf{a}}e^{-\pi n^{2}x}x^{\frac{s+\mf{a}}{2}}\,\frac{dx}{x},
        \]
        after moving the $n^{\mf{a}}$ on the inside of the integral. Dividing by $\pi^{\frac{s+\mf{a}}{2}}n^{s}$ and summing over $n \ge 1$, we see that for $\Re(s) > 1$,
        \begin{align*}
            \pi^{-\frac{s+\mf{a}}{2}}\G\left(\frac{s+\mf{a}}{2}\right)L(s,\chi) &= \sum_{n \ge 1}\int_{0}^{\infty}\chi(n)n^{\mf{a}}e^{-\pi n^{2}x}x^{\frac{s+\mf{a}}{2}}\,\frac{dx}{x} \\
            &= \int_{0}^{\infty}\sum_{n \ge 1}\chi(n)n^{\mf{a}}e^{-\pi n^{2}x}x^{\frac{s+\mf{a}}{2}}\,\frac{dx}{x} && \text{DCT} \\
            &= \int_{0}^{\infty}\w_{\chi}(x)x^{\frac{s+\mf{a}}{2}}\,\frac{dx}{x},
        \end{align*}
        where we set
        \[
            \w_{\chi}(x) = \sum_{n \ge 1}\chi(n)n^{\mf{a}}e^{-\pi n^{2}x}.
        \]
        Therefore we have an integral representation
        \begin{equation}\label{equ:integral_representation_Dirichlet_L-functions_1}
            L(s,\chi) = \frac{\pi^{\frac{s+\mf{a}}{2}}}{\G\left(\frac{s+\mf{a}}{2}\right)}\int_{0}^{\infty}\w_{\chi}(x)x^{\frac{s+\mf{a}}{2}}\,\frac{dx}{x},
        \end{equation}
        and just like $\z(s)$ we need to find a transformation law for $\w_{\chi}(x)$ before we can proceed.
        \subsection*{The Dirichlet Theta Function \texorpdfstring{$\vt_{\chi}(s)$}{\vt_{\chi}(s)}}
        The \textbf{Dirichlet theta function}\index{Dirichlet theta function} $\vt_{\chi}(s)$, attached to the character $\chi$, is defined for $\Re(s) > 0$ by
        \[
            \vt_{\chi}(s) = \sum_{n \in \Z}\chi(n)n^{\mf{a}}e^{-\pi n^{2}s} = 2\sum_{n \ge 1}\chi(n)n^{\mf{a}}e^{-\pi n^{2}s}.
        \]
        It is locally absolutely uniformly convergent in this region by the ratio test. Notice that the term corresponding to $n = 0$ vanishes because $\chi(0) = 0$, and $\chi(n)n^{\mf{a}} = \chi(-n)(-n)^{\mf{a}}$ so that the $n$-th and $(-n)$-th terms agree. Therefore the relationship between the twisted theta function and $\w_{\chi}(s)$ is
        \begin{equation}\label{equ:twisted_omega_theta_relationship_for_Dirichlet_L-functions}
            \w_{\chi}(s) = \frac{\vt_{\chi}(s)}{2}.
        \end{equation}

        \begin{remark}
            \cref{equ:twisted_omega_theta_relationship_for_Dirichlet_L-functions} is a slightly less complex relationship that \cref{equ:omega_theta_relationship_for_zeta}. This is because assuming $q > 1$ means $\chi(0) = 0$.
        \end{remark}

        The essential fact about the twisted theta function we will need is the following transformation law similar to the transformation law for Jacobi's theta function:

        \begin{theorem}
            Let $\chi$ be a primitive Dirichlet character with conductor $q > 1$. For $\Re(s) > 0$,
            \[
            \vt_{\chi}(s) = \frac{\e_{\chi}}{i^{\mf{a}}(qs)^{\frac{1}{2}+\mf{a}}}\vt_{\cchi}\left(\frac{1}{q^{2}s}\right).
            \]
        \end{theorem}
        \begin{proof}
            By the identity theorem it suffices to prove this on a set containg a limit point. We will prove this on the right-half of the real line, so take $s$ real with $s > 0$. Since $\chi$ is $q$-periodic, we can write
            \[
            \vt_{\chi}(s) = \sum_{a \tmod{q}}\chi(a)\sum_{m \in \Z}(mq+a)^{\mf{a}}e^{-\pi(mq+a)^{2}s}.
            \]
            Set $f(x) = (xq+a)^{\mf{a}}e^{-\pi(xq+a)^{2}s}$. Then $f(x)$ is a Schwarz function. We compute its Fourier transform:
            \[
            \hat{f}(t) = \int_{-\infty}^{\infty}f(x)e^{-2\pi itx}\,dx = \int_{-\infty}^{\infty}(xq+a)^{\mf{a}}e^{-\pi(xq+a)^{2}s}e^{-2\pi itx}\,dx = \int_{-\infty}^{\infty}(xq+a)^{\mf{a}}e^{-\pi((xq+a)^{2}s+2itx)}\,dx.
            \]
            By performing the change of variables $x \to \frac{x}{q\sqrt{s}}-\frac{a}{q}$, the last integral above becomes
            \[
            \frac{e^{\frac{2\pi iat}{q}}}{qs^{\frac{1+\mf{a}}{2}}}\int_{-\infty}^{\infty}x^{\mf{a}}e^{-\pi\left(x^{2}+\frac{2itx}{q\sqrt{s}}\right)}\,dx.
            \]
            Complete the square in the exponent by observing
            \[
            -\pi\left(x^{2}+\frac{2itx}{q\sqrt{s}}\right) = -\pi\left(\left(x+\frac{it}{q\sqrt{s}}\right)^{2}+\frac{t^{2}}{q^{2}s}\right).
            \]
            Taking exponentials, this implies that the previous integral is equal to
            \[
            \frac{e^{\frac{2\pi iat}{q}}e^{-\frac{\pi t^{2}}{q^{2}s}}}{qs^{\frac{1+\mf{a}}{2}}}\int_{-\infty}^{\infty}x^{\mf{a}}e^{-\pi\left(x+\frac{it}{q\sqrt{s}}\right)^{2}}\,dx.
            \]
            The change of variables $x \to x-\frac{it}{q\sqrt{s}}$ is permitted without affecting the line of integration by viewing the integral as a complex integral, noting that the integrand is entire as a complex function, and shifting the line of integration. This gives
            \[
            \frac{e^{\frac{2\pi iat}{q}}e^{-\frac{\pi t^{2}}{q^{2}s}}}{qs^{\frac{1+\mf{a}}{2}}}\int_{-\infty}^{\infty}\left(x-\frac{it}{q\sqrt{s}}\right)^{\mf{a}}e^{-\pi x^{2}}\,dx = \frac{e^{\frac{2\pi iat}{q}}e^{-\frac{\pi t^{2}}{q^{2}s}}}{qs^{\frac{1+\mf{a}}{2}}}\int_{-\infty}^{\infty}\left(x+\frac{t}{iq\sqrt{s}}\right)^{\mf{a}}e^{-\pi x^{2}}\,dx.
            \]
            If $\mf{a} = 0$, we obtain
            \begin{equation}\label{transformation_law_for_twisted_Jacobi's_theta_function_1}
            \frac{e^{\frac{2\pi iat}{q}}e^{-\frac{\pi t^{2}}{q^{2}s}}}{qs^{\frac{1+\mf{a}}{2}}}\int_{-\infty}^{\infty}e^{-\pi x^{2}}\,dx = \frac{e^{\frac{2\pi iat}{q}}e^{-\frac{\pi t^{2}}{q^{2}s}}}{qs^{\frac{1+\mf{a}}{2}}},
            \end{equation}
            where the equality holds because the integral is $1$ since it is the Gaussian integral (see \cref{append:Special_Integrals}). If $\mf{a} = 1$, then by direct computation
            \[
            \int_{-\infty}^{\infty}xe^{-\pi x^{2}}\,dx = -\frac{1}{2\pi}e^{-\pi x^{2}}\bigg|_{-\infty}^{\infty} = 0,
            \]
            and thus
            \begin{equation}\label{transformation_law_for_twisted_Jacobi's_theta_function_2}
            \frac{e^{\frac{2\pi iat}{q}}e^{-\frac{\pi t^{2}}{q^{2}s}}}{qs^{\frac{1+\mf{a}}{2}}}\int_{-\infty}^{\infty}\left(\frac{t}{iq\sqrt{s}}\right)e^{-\pi x^{2}}\,dx = \frac{e^{\frac{2\pi iat}{q}}e^{-\frac{\pi t^{2}}{q^{2}s}}}{qs^{\frac{1+\mf{a}}{2}}}\left(\frac{t}{iq\sqrt{s}}\right)\int_{-\infty}^{\infty}e^{-\pi x^{2}}\,dx = \frac{e^{\frac{2\pi iat}{q}}e^{-\frac{\pi t^{2}}{q^{2}s}}}{qs^{\frac{1+\mf{a}}{2}}}\left(\frac{t}{iq\sqrt{s}}\right),
            \end{equation}
            where the last equality follows because the last integral is the Gaussian integral again. Since $\left(\frac{t}{iq\sqrt{s}}\right)^{\mf{a}} = 1$ if $\mf{a} = 0$, we can compactly express the right-hand sides of \cref{transformation_law_for_twisted_Jacobi's_theta_function_1,transformation_law_for_twisted_Jacobi's_theta_function_2} in the form
            \[
            \frac{e^{\frac{2\pi iat}{q}}e^{-\frac{\pi t^{2}}{q^{2}s}}}{qs^{\frac{1+\mf{a}}{2}}}\left(\frac{t}{iq\sqrt{s}}\right)^{\mf{a}}.
            \]
            The Poisson summation formula then implies
            \begin{align*}
            \vt_{\chi}(s) &= \sum_{a \tmod{q}}\chi(a)\sum_{t \in \Z}\frac{e^{\frac{2\pi iat}{q}}e^{-\frac{\pi t^{2}}{q^{2}s}}}{qs^{\frac{1+\mf{a}}{2}}}\left(\frac{t}{iq\sqrt{s}}\right)^{\mf{a}} \\
            &= \frac{1}{i^{\mf{a}}q^{1+\mf{a}}s^{\frac{1}{2}+\mf{a}}}\sum_{a \tmod{q}}\chi(a)\sum_{t \in \Z}t^{\mf{a}}e^{\frac{2\pi i at}{q}}e^{-\frac{\pi t^{2}}{q^{2}s}} \\
            &= \frac{1}{i^{\mf{a}}q^{1+\mf{a}}s^{\frac{1}{2}+\mf{a}}}\sum_{t \in \Z}t^{\mf{a}}e^{-\frac{\pi t^{2}}{q^{2}s}}\sum_{a \tmod{q}}\chi(a)e^{\frac{2\pi i at}{q}} \\
            &= \frac{1}{i^{\mf{a}}q^{1+\mf{a}}s^{\frac{1}{2}+\mf{a}}}\sum_{t \in \Z}t^{\mf{a}}e^{-\frac{\pi t^{2}}{q^{2}s}}\tau(t,\chi) && \text{definition of $\tau(t,\chi)$},
            \end{align*}
            and this holds for all $\Re(s) > 0$ by the identity theorem. Since $\chi$ is primitive, $\tau(t,\chi) = \cchi(t)\tau(\chi)$ for all $t \in \Z$ by \cref{cor:gauss_sum_primitive_formula}. Therefore we have
            \[
            \frac{1}{i^{\mf{a}}q^{1+\mf{a}}s^{\frac{1}{2}+\mf{a}}}\sum_{t \in \Z}t^{\mf{a}}e^{-\frac{\pi t^{2}}{q^{2}s}}\tau(t,\chi) = \frac{\tau(\chi)}{i^{\mf{a}}q^{1+\mf{a}}s^{\frac{1}{2}+\mf{a}}}\sum_{t \in \Z}\cchi(t)t^{\mf{a}}e^{-\frac{\pi t^{2}}{q^{2}s}} = \frac{\tau(\chi)}{i^{\mf{a}}q^{1+\mf{a}}s^{\frac{1}{2}+\mf{a}}}\vt_{\cchi}\left(\frac{1}{q^{2}s}\right).
            \]
            Recalling that $\e_{\chi} = \frac{\tau(\chi)}{\sqrt{q}}$, we conclude
            \[
            \vt_{\chi}(s) = \frac{\e_{\chi}}{i^{\mf{a}}(qs)^{\frac{1}{2}+\mf{a}}}\vt_{\cchi}\left(\frac{1}{q^{2}s}\right).
            \]
        \end{proof}
        The most striking property about this transformation law is that it relates $\vt_{\chi}(s)$ and $\vt_{\cchi}(s)$. Regardless, we can now exploit it to analytically continue $L(s,\chi)$.
        \subsection*{The Integral Representation of \texorpdfstring{$L(s,\chi)$}{L(s,\chi)}: Part II}
        Returning to $L(s,\chi)$, split the integral in \cref{equ:integral_representation_Dirichlet_L-functions_1} into two pieces
        \begin{equation}\label{equ:symmetric_integral_Dirichlet_L-functions_split}
            \int_{0}^{\infty}\w_{\chi}(x)x^{\frac{s+\mf{a}}{2}}\,\frac{dx}{x} = \int_{0}^{\frac{1}{q}}\w_{\chi}(x)x^{\frac{s+\mf{a}}{2}}\,\frac{dx}{x}+\int_{\frac{1}{q}}^{\infty}\w_{\chi}(x)x^{\frac{s+\mf{a}}{2}}\,\frac{dx}{x}.
        \end{equation}
        Since $\w_{\chi}(x)$ has exponential decay to zero as $x \to \infty$, the second piece is locally absolutely uniformly bounded for $\Re(s) > 1$ by \cref{met:decay_compacta_integral}. Hence it defines an analytic function there. We now rewrite the first piece in the same form and symmetrize the result as much as possible. Start by performing a change of variables $x \to \frac{1}{q^{2}x}$ to the first piece to obtain
        \[
            q^{-(s+\mf{a})}\int_{\frac{1}{q}}^{\infty}\w_{\chi}\left(\frac{1}{q^{2}x}\right)x^{-\frac{s+\mf{a}}{2}}\,\frac{dx}{x}.
        \]
        Now the transformation law for $\vt_{\chi}(x)$ and \cref{equ:twisted_omega_theta_relationship_for_Dirichlet_L-functions} together imply
        \begin{equation}\label{equ:piece_one_Dirichlet_1}
            \begin{aligned}
            \w_{\chi}\left(\frac{1}{q^{2}x}\right) &= \frac{\vt_{\chi}\left(\frac{1}{q^{2}x}\right)}{2} \\
            &= \frac{i^{\mf{a}}(qx)^{\frac{1}{2}+\mf{a}}}{\e_{\cchi}}\frac{\vt_{\cchi}(x)}{2} \\
            &= \e_{\chi}(-i)^{\mf{a}}(qx)^{\frac{1}{2}+\mf{a}}\frac{\vt_{\cchi}(x)}{2} && \text{\cref{prop:epsilon_factor_relationship} and $\chi(-1) = (-1)^{\mf{a}}$} \\
            &= \frac{\e_{\chi}(qx)^{\frac{1}{2}+\mf{a}}}{i^{\mf{a}}}\frac{\vt_{\cchi}(x)}{2} \\
            &= \frac{\e_{\chi}(qx)^{\frac{1}{2}+\mf{a}}}{i^{\mf{a}}}\w_{\cchi}(x).
            \end{aligned}
        \end{equation}
        \cref{equ:piece_one_Dirichlet_1} gives the first equality in the following chain:
        \begin{align*}
            q^{-(s+\mf{a})}\int_{\frac{1}{q}}^{\infty}\w_{\chi}\left(\frac{1}{q^{2}x}\right)x^{-\frac{s+\mf{a}}{2}}\,\frac{dx}{x} &= q^{-(s+\mf{a})}\int_{\frac{1}{q}}^{\infty}\left(\frac{\e_{\chi}(qx)^{\frac{1}{2}+\mf{a}}}{i^{\mf{a}}}\w_{\cchi}(x)\right)x^{-\frac{s+\mf{a}}{2}}\,\frac{dx}{x} \\
            &= \frac{\e_{\chi}}{i^{\mf{a}}}q^{\frac{1}{2}-s}\int_{\frac{1}{q}}^{\infty}\w_{\cchi}(x)x^{\frac{(1-s)+\mf{a}}{2}}\,\frac{dx}{x}.
        \end{align*}
        Substituting this last expression back into \cref{equ:symmetric_integral_Dirichlet_L-functions_split} with \cref{equ:integral_representation_Dirichlet_L-functions_1} gives the following result:

        \begin{theorem}
            Let $\chi$ be a primitive Dirichlet character with conductor $q > 1$. For $\Re(s) > 1$,
            \[
            L(s,\chi) = \frac{\pi^{\frac{s+\mf{a}}{2}}}{\G\left(\frac{s+\mf{a}}{2}\right)}\left[\frac{\e_{\chi}}{i^{\mf{a}}}q^{\frac{1}{2}-s}\int_{\frac{1}{q}}^{\infty}\w_{\cchi}(x)x^{\frac{(1-s)+\mf{a}}{2}}\,\frac{dx}{x}+\int_{\frac{1}{q}}^{\infty}\w_{\chi}(x)x^{\frac{s+\mf{a}}{2}}\,\frac{dx}{x}\right].
            \]
        \end{theorem}

        This integral representation will give analytic continuation. Indeed, we know everything outside the brackets is entire and the latter of the two integrals inside the brackets is analytic for $\Re(s) > 1$. Thus the first integral inside the brackets must be analytic in this region too. Since the two integrals are interchanged as $s \to 1-s$, save for the term $\frac{\e_{\chi}}{i^{\mf{a}}}q^{\frac{1}{2}-s}$, the right-hand side is analytic for $\Re(s) < 0$. This gives the analytic continuation of $L(s,\chi)$ to the region
        \[
            \left\{s \in \C:\left|\Re(s)-\frac{1}{2}\right| > \frac{1}{2}\right\}.
        \]
        \subsection*{The Functional Equation \& Critical Strip of \texorpdfstring{$L(s,\chi)$}{L(s,\chi)}}
        An immediate consequence of the symmetry of the integral representation is the functional equation:
        \[
            q^{\frac{s}{2}}\frac{\G\left(\frac{s+\mf{a}}{2}\right)}{\pi^{\frac{s+\mf{a}}{2}}}L(s,\chi) = \frac{\e_{\chi}}{i^{\mf{a}}}q^{\frac{1-s}{2}}\frac{\G\left(\frac{(1-s)+\mf{a}}{2}\right)}{\pi^{\frac{(1-s)+\mf{a}}{2}}}L(1-s,\cchi).
        \]
        We identify the gamma factor as
        \[
            \g(s,\chi) = \pi^{-\frac{s}{2}}\G\left(\frac{s+\mf{a}}{2}\right),
        \]
        with $\k = \frac{s+\mf{a}}{2}$ the only local parameter at infinity. Clearly it satisfies the required bounds. The conductor is $q(\chi) = q$ and if $p$ is an unramified prime then the local root is $\chi(p) \neq 0$. The completed $L$-function is
        \[
            \L(s,\chi) = q^{\frac{s}{2}}\pi^{-\frac{s}{2}}\G\left(\frac{s}{2}\right)L(s,\chi),
        \]
        with functional equation
        \[
            \L(s,\chi) = \frac{\e_{\chi}}{i^{\mf{a}}}\L(1-s,\cchi).
        \]
        From it we see that the root number is $\e(\chi) = \frac{\e_{\chi}}{i^{\mf{a}}}$ and that $L(s,\chi)$ has dual $L(s,\cchi)$. In total, $L(s,\chi)$ satisfies properties (ii)-(iv) of the Selberg class.

        We now analytically continue $L(s,\chi)$ inside the critical strip and therefore to all of $\C$. To get continuation inside the critical strip it suffices that the two integrals in the integral representation are locally absolutely uniformly bounded for $|\Re(s)-\frac{1}{2}| \le \frac{1}{2}$. This follows by \cref{met:decay_compacta_integral} (we could have deduced this continuation earlier but we didn't need it until now) and thus gives analytic continuation to the critical strip and hence to all of $\C$. In particular, we have shown that $L(s,\chi)$ has no poles.

        All we are left to show is that $L(s,\chi)$ is of order $1$ to conclude that it satisfies property (v) of the Selberg class and therefore is an $L$-function belonging to the Selberg class. Since $L(s,\chi)$ has no poles, we do not need to clear any polar divisors. As the integral in the representation is absolutely bounded, computing the order amounts to estimating the gamma factor. Since the reciprocal of the gamma function is of order $1$ and $\Re(s)$ is bounded, we have
        \begin{equation}\label{equ:Dirichlet_L-functions_factor_order_1}
            \frac{1}{\g(s,\chi)} \ll_{\e} e^{|s|^{1+\e}},
        \end{equation}
        for any $\e > 0$. So the reciprocal of the gamma factor is of the same order. Then \cref{equ:Dirichlet_L-functions_factor_order_1} and the integral representation together imply
        \[
            L(s,\chi) \ll_{\e} e^{|s|^{1+\e}}.
        \]
        So $L(s,\chi)$ is of order $1$. We summarize all of our work into the following theorem:

        \begin{theorem}
            For any primitive Dirichlet character $\chi$ with conductor $q > 1$, $L(s,\chi)$ is a Selberg class $L$-function. It admits analytic continuation to $\C$ via the integral representation
            \[
            L(s,\chi) = \frac{\pi^{\frac{s+\mf{a}}{2}}}{\G\left(\frac{s+\mf{a}}{2}\right)}\left[\frac{\e_{\chi}}{i^{\mf{a}}}q^{\frac{1}{2}-s}\int_{\frac{1}{q}}^{\infty}\w_{\cchi}(x)x^{\frac{(1-s)+\mf{a}}{2}}\,\frac{dx}{x}+\int_{\frac{1}{q}}^{\infty}\w_{\chi}(x)x^{\frac{s+\mf{a}}{2}}\,\frac{dx}{x}\right],
            \]
            and with functional equation
            \[
            q^{\frac{s}{2}}\pi^{-\frac{s}{2}}\G\left(\frac{s+\mf{a}}{2}\right)L(s,\chi) = \L(s,\chi) = \frac{\e_{\chi}}{i^{\mf{a}}}\L(1-s,\cchi).
            \]
        \end{theorem}
    \section{Hecke \texorpdfstring{$L$}{L}-functions}
        It is time to start our investigation into the $L$-functions of Hecke eigenforms. Actually, we will only be interested in the $L$-functions of holomorphic primitive Hecke eigenform on $\G_{1}(N)\backslash\H$ as these belong to the Selberg class. The theory can be extended to more general modular and Maass forms but some of the properies of Selberg class $L$-functions are lost or currently unknown.
        \subsection*{The Definition \& Euler Product of \texorpdfstring{$L(s,f)$}{L(s,f)}}
        Let $f \in \mc{S}_{k}(N,\chi)$ be a primitive Hecke eigenform and denote its Fourier series by
        \[
            f(z) = \sum_{n \ge 1}\l_{f}(n)e^{2\pi inz}.
        \]
        The \textbf{Hecke $L$-function}\index{Hecke $L$-function} $L(s,f)$ of $f$ is defined as an $L$-series:
        \[
            L(s,f) = \sum_{n \ge 1}\frac{a_{f}(n)}{n^{s}} = \sum_{n \ge 1}\frac{\l_{f}(n)}{n^{s+\frac{k-1}{2}}},
        \]
        where $a_{f}(n) = \l_{f}(n)n^{-\frac{k-1}{2}}$ (here we are abusing notation for the Fourier coefficeints of holomorphic forms). Our goal now is to show that this $L$-function belongs to the Selberg class. First, we need to ensure that the $L$-function is locally absolutely uniformly convergent. By the Ramanujan conjecture, we have $\l_{f}(n) \ll n^{\frac{k-1}{2}+\e}$ for any $\e > 0$ because the Fourier coefficients are multiplicative and $\s_{0}(n) \ll n^{\e}$ (see \cite{montgomery2006multiplicative} for a proof). In particular, $a_{f}(n) \ll n^{\e}$. So $L(s,f)$ is locally absolutely uniformly convergent for $\Re(s) > 1+\e$ and hence locally absolutely uniformly convergent for $\Re(s) > 1$. The $L$-function will also have an Euler product since $f$ is a primitive Hecke eigenform. In this case, the Hecke relations imply that the coefficients $a_{f}(n)$ are multiplicative and satisfy
        \begin{equation}\label{equ:primitive_Hecke_eigenform_recurrence_for_coefficients_of_modular_L-function}
            a_{f}(p^{n}) = \begin{cases} a_{f}(p^{n-1})a_{f}(p)-\chi(p)a_{f}(p^{n-2}) & \text{if $p \nmid N$}, \\ (a_{f}(p))^{n} & \text{if $p \mid N$}, \end{cases}
        \end{equation}
        for all primes $p$ and $n \ge 2$. Because $L(s,f)$ converges absolutely in the region $\Re(s) > 1$, multiplicativity of the Fourier coefficients implies
        \[
            L(s,f) = \prod_{p}\left(\sum_{n \ge 0}\frac{a_{f}(p^{n})}{p^{ns}}\right),
        \]
        in this region. We now simplify the factor inside the product using this \cref{equ:primitive_Hecke_eigenform_recurrence_for_coefficients_of_modular_L-function}. On the one hand, if $p \nmid N$:
        \begin{align*}
            \sum_{n \ge 0}\frac{a_{f}(p^{n})}{p^{ns}} &= 1+\frac{a_{f}(p)}{p^{s}}+\sum_{n \ge 2}\frac{a_{f}(p^{n})}{p^{ns}} \\
            &= 1+\frac{a_{f}(p)}{p^{s}}+\sum_{n \ge 2}\frac{a_{f}(p^{n-1})a_{f}(p)-\chi(p)a_{f}(p^{n-2})}{p^{ns}} \\
            &= 1+\frac{a_{f}(p)}{p^{s}}+\frac{a_{f}(p)}{p^{s}}\sum_{n \ge 1}\frac{a_{f}(p^{n})}{p^{ns}}-\frac{\chi(p)}{p^{2s}}\sum_{n \ge 0}\frac{a_{f}(p^{n})}{p^{ns}} \\
            &= 1+\left(\frac{a_{f}(p)}{p^{s}}-\frac{\chi(p)}{p^{2s}}\right)\sum_{n \ge 0}\frac{a_{f}(p^{n})}{p^{ns}}.
        \end{align*}
        By isolating the sum we find
        \begin{equation}\label{equ:Euler_factor_modular_L-function_p_not_divides_N}
            \sum_{n \ge 0}\frac{a_{f}(p^{n})}{p^{ns}} = \left(1-\frac{a_{f}(p)}{p^{s}}+\frac{\chi(p)}{p^{2s}}\right)^{-1}.
        \end{equation}
        On the other hand, if $p \mid N$:
        \begin{equation}\label{equ:Euler_factor_modular_L-function_p_divides_N}
            \sum_{n \ge 0}\frac{a_{f}(p^{n})}{p^{ns}} = \sum_{n \ge 0}\frac{(a_{f}(p))^{n}}{p^{ns}} = \left(1-a_{f}(p)p^{-s}\right)^{-1}.
        \end{equation}
        Therefore \cref{equ:Euler_factor_modular_L-function_p_not_divides_N,equ:Euler_factor_modular_L-function_p_divides_N} together give
        \[
            L(s,f) = \prod_{p \nmid N}(1-a_{f}(p)p^{-s}+\chi(p)p^{-2s})^{-1}\prod_{p \mid N}(1-a_{f}(p)p^{-s})^{-1}.
        \]
        If $p \nmid N$, let $\a_{1}(p)$ and $\a_{2}(p)$ be the roots of $1-a_{f}(p)p^{-s}+\chi(p)p^{-2s}$. That is,
        \begin{equation}\label{equ:local_parameter_definition_modfular_forms}
            (1-\a_{1}(p)p^{-s})(1-\a_{2}(p)p^{-s}) = (1-a_{f}(p)p^{-s}+\chi(p)p^{-2s}).
        \end{equation}
        If $p \mid N$, let $\a_{1}(p) = a_{f}(p)$ and $\a_{2}(p) = 0$. We can then express $L(s,f)$ as a degree $2$ Euler product:
        \[
            L(s,f) = \prod_{p}(1-\a_{1}(p)p^{-s})^{-1}(1-\a_{2}(p)p^{-s})^{-1}.
        \]
        The local factor at $p$ is $(1-\a_{1}(p)p^{-s})^{-1}(1-\a_{2}(p)p^{-s})^{-1}$ with local roots $\a_{1}(p)$ and $\a_{2}(p)$. Upon applying partial fraction decomposition to the local factor, we find
        \[
            \frac{1}{1-\a_{1}(p)p^{-s}}\frac{1}{1-\a_{2}(p)p^{-s}} = \frac{\frac{\a_{1}(p)}{\a_{1}(p)-\a_{2}(p)}}{1-\a_{1}(p)p^{-s}}+\frac{\frac{-\a_{2}(p)}{\a_{1}(p)-\a_{2}(p)}}{1-\a_{2}(p)p^{-s}}.
        \]
        Expanding both sides as series in $p^{-s}$, and comparing coefficients gives
        \begin{equation}\label{equ:Satake_coefficient_formula}
            a_{f}(p^{n}) = \frac{\a_{1}(p)^{n+1}-\a_{2}(p)^{n+1}}{\a_{1}(p)-\a_{2}(p)}.
        \end{equation}
        Altogether, $L(s,f)$ satisfies property (i) for belonging to the Selberg class. We package this into a theorem:
        \begin{theorem}
            Let $f \in \mc{S}_{k}(N,\chi)$ be a primitive Hecke eigenform. For $\Re(s) > 1$,
            \[
            L(s,f) = \prod_{p \nmid N}(1-a_{f}(p)p^{-s}+\chi(p)p^{-2s})^{-1}\prod_{p \mid N}(1-a_{f}(p)p^{-s})^{-1} = \prod_{p}(1-\a_{1}(p)p^{-s})^{-1}(1-\a_{2}(p)p^{-s})^{-1},
            \]
            is locally absolutely uniformly convergent with degree $2$ Euler product.
        \end{theorem}
        \subsection*{The Integral Representation of \texorpdfstring{$L(s,f)$}{L(s,f)}}
        We now look to represent $L(s,f)$ as a symmetric integral under $s \to 1-s$. The integral we want to consider is a Mellin tranform:
        \[
            \int_{0}^{\infty}f(iy)y^{s+\frac{k-1}{2}}\,\frac{dy}{y}.
        \]
        This time, we don't know \textit{a priori} that this integral defines an analytic function for $\Re(s) > 1$. In any case, we compute
        \begin{align*}
            \int_{0}^{\infty}f(iy)y^{s+\frac{k-1}{2}}\,\frac{dy}{y} &= \int_{0}^{\infty}\sum_{n \ge 1}\l_{f}(n)e^{-2\pi ny}y^{s+\frac{k-1}{2}}\,\frac{dy}{y} \\
            &= \sum_{n \ge 1}\l_{f}(n)\int_{0}^{\infty}e^{-2\pi ny}y^{s+\frac{k-1}{2}}\,\frac{dy}{y} &&\text{DCT} \\
            &= \sum_{n \ge 1}\frac{\l_{f}(n)}{(2\pi)^{s+\frac{k-1}{2}}n^{s+\frac{k-1}{2}}}\int_{0}^{\infty}e^{-y}y^{s+\frac{k-1}{2}}\,\frac{dy}{y} &&\text{$y \to \frac{y}{2\pi n}$} \\
            &= \frac{\G\left(s+\frac{k-1}{2}\right)}{(2\pi)^{s+\frac{k-1}{2}}}\sum_{n \ge 1}\frac{\l_{f}(n)}{n^{s+\frac{k-1}{2}}} \\
            &= \frac{\G\left(s+\frac{k-1}{2}\right)}{(2\pi)^{s+\frac{k-1}{2}}}L(s,f).
        \end{align*}
        As this last expression defines an analytic function for $\Re(s) > 1$, the integral does too. Rewriting, we have an integral representation
        \begin{equation}\label{equ:integral_representation_L-function_1}
            L(s,f) = \frac{(2\pi)^{s+\frac{k-1}{2}}}{\G\left(s+\frac{k-1}{2}\right)}\int_{0}^{\infty}f(iy)y^{s+\frac{k-1}{2}}\,\frac{dy}{y}.
        \end{equation}
        Now split the integral on the right-hand side into two pieces
        \begin{equation}\label{equ:symmetric_integral_L-function_split}
            \int_{0}^{\infty}f(iy)y^{s+\frac{k-1}{2}}\,\frac{dy}{y} = \int_{0}^{\frac{1}{\sqrt{N}}}f(iy)y^{s+\frac{k-1}{2}}\,\frac{dy}{y}+\int_{\frac{1}{\sqrt{N}}}^{\infty}f(iy)y^{s+\frac{k-1}{2}}\,\frac{dy}{y}.
        \end{equation}
        Since $f(iy)$ has exponential decay to zero as $y \to \infty$, the second piece is locally absolutely uniformly bounded for $\Re(s) > 1$ by \cref{met:decay_compacta_integral}. Hence it defines an analytic function there. Now we will rewrite the first piece in the same form and symmetrize the result as much as possible. Begin by performing the change of variables $y \to \frac{1}{Ny}$ to the first piece to obtain
        \[
            \int_{\frac{1}{\sqrt{N}}}^{\infty}f\left(\frac{i}{Ny}\right)(Ny)^{-s-\frac{k-1}{2}}\,\frac{dy}{y}.
        \]
        Rewriting in terms of the Atkin–Lehner involution $\w_{N}f$ and recalling that $\w_{N}f = \w_{N}(f)f$ by \cref{thm:eigenforms_forms_spectral_theory_with_Atkin_Lehner_holomorphic}, we have
        \begin{align*}
            \int_{\frac{1}{\sqrt{N}}}^{\infty}f\left(\frac{i}{Ny}\right)(Ny)^{-s-\frac{k-1}{2}}\,\frac{dy}{y} &= \int_{\frac{1}{\sqrt{N}}}^{\infty}f\left(-\frac{1}{iNy}\right)(Ny)^{-s-\frac{k-1}{2}}\,\frac{dy}{y} \\
            &= \int_{\frac{1}{\sqrt{N}}}^{\infty}\left(\sqrt{N}iy\right)^{k}(\w_{N}f)(iy)(Ny)^{-s-\frac{k-1}{2}}\,\frac{dy}{y} \\
            &= \int_{\frac{1}{\sqrt{N}}}^{\infty}\left(\sqrt{N}iy\right)^{k}\w_{N}(f)f(iy)(Ny)^{-s-\frac{k-1}{2}}\,\frac{dy}{y} \\
            &= \w_{N}(f)i^{k}N^{\frac{1}{2}-s}\int_{\frac{1}{\sqrt{N}}}^{\infty}f(iy)y^{(1-s)-\frac{k-1}{2}}\,\frac{dy}{y}.
        \end{align*}
        Substituting this result back into \cref{equ:symmetric_integral_L-function_split} with \cref{equ:integral_representation_L-function_1} yields the following result:

        \begin{theorem}
            Let $f \in \mc{S}_{k}(N,\chi)$ be a primitive Hecke eigenform. For $\Re(s) > 1$,
            \[
            L(s,f) = \frac{(2\pi)^{s+\frac{k-1}{2}}}{\G\left(s+\frac{k-1}{2}\right)}\left[\w_{N}(f)i^{k}N^{\frac{1}{2}-s}\int_{\frac{1}{\sqrt{N}}}^{\infty}f(iy)y^{(1-s)+\frac{k-1}{2}}\,\frac{dy}{y}+\int_{\frac{1}{\sqrt{N}}}^{\infty}f(iy)y^{s+\frac{k-1}{2}}\,\frac{dy}{y}\right].
            \]
        \end{theorem}

        This integral will give analytic continuation. To see this, we know everything outside the brackets is entire and the second of the two integrals inside the brackets is analytic for $\Re(s) > 1$. Thus the first integral inside the brackets must be analytic in this region since $L(s,f)$ is too. Now the two integrals, save for the term $\w_{N}(f)i^{k}N^{\frac{1}{2}-s}$, are interchanged as $s \to 1-s$. Hence the right-hand side is analytic for $\Re(s) < 0$ as well. Thus we have analytic continuation to the region
        \[
            \left\{s \in \C:\left|\Re(s)-\frac{1}{2}\right| > \frac{1}{2}\right\}.
        \]
        \subsection*{The Functional Equation \& Critical Strip of \texorpdfstring{$L(s,f)$}{L(s,f)}}
        An immediate consequence of the symmetry of the integral representation is the functional equation:
        \[
            \frac{\G\left(s+\frac{k-1}{2}\right)}{(2\pi)^{s+\frac{k-1}{2}}}L(s,f) = \w_{N}(f)i^{k}N^{-\frac{s}{2}}\frac{\G\left((1-s)+\frac{k-1}{2}\right)}{(2\pi)^{(1-s)+\frac{k-1}{2}}}L(1-s,f).
        \]
        Using the Legendre duplication formula for the gamma function we find that
        \begin{align*}
            \frac{\G\left(s+\frac{k-1}{2}\right)}{(2\pi)^{s+\frac{k-1}{2}}} &= \frac{1}{(2\pi)^{s+\frac{k-1}{2}}2^{1-\left(s+\frac{k-1}{2}\right)}\sqrt{\pi}}\G\left(\frac{s+\frac{k-1}{2}}{2}\right)\G\left(\frac{s+\frac{k+1}{2}}{2}\right) \\
            &= \frac{1}{2\pi^{s+\frac{1}{2}}}\G\left(\frac{s+\frac{k-1}{2}}{2}\right)\G\left(\frac{s+\frac{k+1}{2}}{2}\right) \\ 
            &= \frac{1}{\sqrt{4\pi}}\pi^{-s}\G\left(\frac{s+\frac{k-1}{2}}{2}\right)\G\left(\frac{s+\frac{k+1}{2}}{2}\right).
        \end{align*}
        The constant factor in front is independent of $s$ and so can be canceled in the functional equation. Therefore we identify the gamma factor as
        \[
            \g(s,f) = \pi^{-s}\G\left(\frac{s+\frac{k-1}{2}}{2}\right)\G\left(\frac{s+\frac{k+1}{2}}{2}\right),
        \]
        with $\k_{1} = \frac{k-1}{2}$ and $\k_{2} = \frac{k+1}{2}$ the local parameters at infinity. Clearly they satisfy the required bounds. The conductor is $q(f) = N$, so the primes dividing the level ramify, and by the Ramanujan conjecture $\a_{i}(p) \neq 0$ for $i = 1,2$ and all primes $p \nmid N$. The completed $L$-function is
        \[
            \L(s,f) = N^{-\frac{s}{2}}\pi^{-s}\G\left(\frac{s+\frac{k-1}{2}}{2}\right)\G\left(\frac{s+\frac{k+1}{2}}{2}\right)L(s,f),
        \]
        with functional equation
        \[
            \L(s,f) = \w_{N}(f)i^{k}\L(1-s,f).
        \]
        This is the functional equation of $L(s,f)$. From it, the root number is $\e(f) = \w_{N}(f)i^{k}$ and we see that $L(s,f)$ is self-dual. Altogether, this shows that $L(s,f)$ satisfies properties (ii)-(iv) of the Selberg class.

        We now analytically continue $L(s,f)$ inside the critical strip and hence to all of $\C$. To get continuation inside the critical strip it suffices that the two integrals in the integral representation are locally absolutely uniformly bounded for $|\Re(s)-\frac{1}{2}| \le \frac{1}{2}$. This follows by \cref{met:decay_compacta_integral} as we already know. Hence we have analytic continuation to the critical strip and hence to all of $\C$. In particular, we have shown that $L(s,f)$ has no poles.

        At last, all we need to show is that $L(s,f)$ is of order $1$ to conclude that it satisfies property (v) of the Selberg class and therefore is an $L$-function belonging to the Selberg class. Since $L(s,f)$ has no poles, we do not need to clear any polar divisors. As the integral in the representation is absolutely bounded, computing the order amounts to estimating the gamma factor. Since the reciprocal of the gamma function is of order $1$ and $\Re(s)$ is bounded, we have
        \begin{equation}\label{equ:modular_form_gamma_factor_order_1}
            \frac{1}{\g(s,f)} \ll_{\e} e^{|s|^{1+\e}},
        \end{equation}
        for any $\e > 0$. So the reciprocal of the gamma factor is of the same order. Then \cref{equ:modular_form_gamma_factor_order_1} and the integral representation together imply
        \[
            L(s,f) \ll_{\e} e^{|s|^{1+\e}}.
        \]
        So $L(s,f)$ is of order $1$. We summarize all of our work into the following theorem:

        \begin{theorem}
            For any primitive Hecke eigenform $f \in \mc{S}_{k}(N,\chi)$, $L(s,f)$ is a Selberg class $L$-function. It admits analytic continuation to $\C$ via the integral representation
            \[
            L(s,f) = \frac{(2\pi)^{s+\frac{k-1}{2}}}{\G\left(s+\frac{k-1}{2}\right)}\left[\w_{N}(f)i^{k}N^{\frac{1}{2}-s}\int_{\frac{1}{\sqrt{N}}}^{\infty}f(iy)y^{(1-s)+\frac{k-1}{2}}\,\frac{dy}{y}+\int_{\frac{1}{\sqrt{N}}}^{\infty}f(iy)y^{s+\frac{k-1}{2}}\,\frac{dy}{y}\right],
            \]
            and with functional equation
            \[
            N^{-\frac{s}{2}}\pi^{-s}\G\left(\frac{s+\frac{k-1}{2}}{2}\right)\G\left(\frac{s+\frac{k+1}{2}}{2}\right)L(s,f) = \L(s,f) = \w_{N}(f)i^{k}\L(1-s,f).
            \]
        \end{theorem}
    \section{The Rankin-Selberg Method}
        We are ready to describe the Rankin-Selberg method. This is a process by which we can construct new $L$-functions from old ones. We discuss the method only in the simplest case for two reasons. The first is that many technical difficulties arise in the fully general setting. The second is that in the simplest case the Rankin-Selberg convolution involves a Maass form we have studied and this establishes a direct connection between holomorphic forms and Maass forms. To somewhat combat not working in full generality, we make remarks where the theory needs to be adjusted in more general settings.
        \subsection*{The Defintion \& Euler Product of \texorpdfstring{$L(s,f \ox g)$}{L(s,f \ox g)}}
        We will start by discussing two different but related $L$-functions. Let $f,g \in \mc{S}_{k}(\PSL_{2}(\Z))$ be primitive Hecke eigenforms. Let the Fourier series for $f$ and $g$ be given by
        \[
            f(z) = \sum_{n \ge 1}\l_{f}(n)e^{2\pi inz} \quad \text{and} \quad g(z) = \sum_{n \ge 1}\l_{g}(n)e^{2\pi inz}.
        \]
        The $L$-function $L(s,f \x g)$ of $f$ and $g$ is defined as an $L$-series:
        \[
            L(s,f \x g) = \sum_{n \ge 1}\frac{a_{f \x g}(n)}{n^{s}} = \sum_{n \ge 1}\frac{a_{f}(n)\conj{b_{g}(n)}}{n^{s}} = \sum_{n \ge 1}\frac{\l_{f}(n)\conj{\l_{g}(n)}}{n^{s+k-1}},
        \]
        where $a_{f \x g}(n) = a_{f}(n)\conj{b_{g}(n)}$. The \textbf{Rankin-Selberg convolution}\index{Rankin-Selberg convolution} $L(s,f \ox g)$ of $f$ and $g$ is defined as an $L$-series:
        \[
            L(s,f \ox g) = \sum_{n \ge 1}\frac{a_{f \ox g}(n)}{n^{s}} = \z(2s)L(s,f \x g),
        \]
        where $a_{f \ox g}(n) = \sum_{m\ell^{2} = n}a_{f}(n)\conj{b_{g}(n)}$.

        \begin{remark}
            If $f \in \mc{S}_{k}(N,\chi)$ and $g \in \mc{S}_{\ell}(M,\psi)$, one needs to use the Dirichlet $L$-function $L(2s,\chi\psi)$ instead of $\z(2s)$.
        \end{remark}

        Our main goal is to show that $L(s,f \ox g)$ is actually the Rankin-Selberg convolution of $L(s,f)$ and $L(s,g)$. The first step is to prove local absolute uniform convergence for $\Re(s) > 1$. To see this, notice that $\z(2s)$ does so it suffices to show $L(s,f \x g)$ does too. In exactly the same way as we remarked for Hecke $L$-functions, $a_{f \x g}(n) \ll n^{\e}$. Hence $L(s,f \x g)$ is locally absolutely uniformly convergent for $\Re(s) > 1+\e$ and hence locally absolutely uniformly convergent for $\Re(s) > 1$.
        
        The $L$-function will have an Euler product since both $f$ and $g$ are primitive Hecke eigenforms. In this case, let $\a_{i}(p)$ and $\b_{j}(p)$ for $1 \le i,j \le 2$ be the local roots of $L(s,f)$ and $L(s,g)$ respectively. Since $L(s,f \ox g)$ converges absolutely in the region $\Re(s) > 1$, for $s$ in this region, multiplicativity of the Fourier coefficients implies
        \[
            L(s,f \ox g) = \z(2s)L(s,f \x g) = \prod_{p \nmid NM}(1-p^{-2s})^{-1}\prod_{p}\left(\sum_{n \ge 0}\frac{a_{f}(p^{n})\conj{b_{g}(p^{n})}}{p^{ns}}\right).
        \]
        We now simplify the factor inside the latter product using \cref{equ:Satake_coefficient_formula}:
        \begingroup
        \allowdisplaybreaks
            \begin{align*}
            \sum_{n \ge 0}\frac{a_{f}(p^{n})\conj{b_{g}(p^{n})}}{p^{ns}} &= \sum_{n \ge 0}\left(\frac{\a_{1}(p)^{n+1}-\a_{2}(p)^{n+1}}{\a_{1}(p)-\a_{2}(p)}\right)\left(\frac{(\conj{\b_{1}(p)})^{n+1}-(\conj{\b_{2}(p)})^{n+1}}{\conj{\b_{1}(p)}-\conj{\b_{2}(p)}}\right)p^{-ns} \\
            &= (\a_{1}(p)-\a_{2}(p))^{-1}\left(\conj{\b_{1}(p)}-\conj{\b_{2}(p)}\right)^{-1} \\
            &\cdot \bigg[\sum_{n \ge 1}\frac{\a_{1}(p)^{n}(\conj{\b_{1}(p)})^{n}}{p^{(n-1)s}}+\frac{\a_{2}(p)^{n}(\conj{\b_{2}(p)})^{n}}{p^{(n-1)s}}-\frac{\a_{1}(p)^{n}(\conj{\b_{2}(p)})^{n}}{p^{(n-1)s}}-\frac{\a_{2}(p)^{n}(\conj{\b_{1}(p)})^{n}}{p^{(n-1)s}}\bigg] \\
            &= (\a_{1}(p)-\a_{2}(p))^{-1}\left(\conj{\b_{1}(p)}-\conj{\b_{2}(p)}\right)^{-1}\bigg[\a_{1}(p)\conj{\b_{1}(p)}\left(1-\a_{1}(p)\conj{\b_{1}(p)}p^{-s}\right)^{-1} \\
            &+\a_{2}(p)\conj{\b_{2}(p)}\left(1-\a_{2}(p)\conj{\b_{2}(p)}p^{-s}\right)^{-1}-\a_{1}(p)\conj{\b_{2}(p)}\left(1-\a_{1}(p)\conj{\b_{2}(p)}p^{-s}\right)^{-1} \\
            &-\a_{2}(p)\conj{\b_{1}(p)}\left(1-\a_{2}(p)\conj{\b_{1}(p)}p^{-s}\right)^{-1}\bigg] \\
            &= (\a_{1}(p)-\a_{2}(p))^{-1}\left(\conj{\b_{1}(p)}-\conj{\b_{2}(p)}\right)^{-1}\left(1-\a_{1}(p)\conj{\b_{1}(p)}p^{-s}\right)^{-1} \\
            &\cdot\left(1-\a_{2}(p)\conj{\b_{2}(p)}p^{-s}\right)^{-1}\left(1-\a_{1}(p)\conj{\b_{2}(p)}p^{-s}\right)^{-1}\left(1-\a_{2}(p)\conj{\b_{1}(p)}p^{-s}\right)^{-1} \\
            &\cdot\bigg[\a_{1}(p)\conj{\b_{1}(p)}\left(1-\a_{2}(p)\conj{\b_{2}(p)}p^{-s}\right)\left(1-\a_{1}(p)\conj{\b_{2}(p)}p^{-s}\right)\left(1-\a_{2}(p)\conj{\b_{1}(p)}p^{-s}\right) \\
            &+\a_{2}(p)\conj{\b_{2}(p)}\left(1-\a_{1}(p)\conj{\b_{1}(p)}p^{-s}\right)\left(1-\a_{1}(p)\conj{\b_{2}(p)}p^{-s}\right)\left(1-\a_{2}(p)\conj{\b_{1}(p)}p^{-s}\right) \\
            &-\a_{1}(p)\conj{\b_{2}(p)}\left(1-\a_{1}(p)\conj{\b_{1}(p)}p^{-s}\right)\left(1-\a_{2}(p)\conj{\b_{2}(p)}p^{-s}\right)\left(1-\a_{2}(p)\conj{\b_{1}(p)}p^{-s}\right) \\
            &-\a_{2}(p)\conj{\b_{1}(p)}\left(1-\a_{1}(p)\conj{\b_{1}(p)}p^{-s}\right)\left(1-\a_{2}(p)\conj{\b_{2}(p)}p^{-s}\right)\left(1-\a_{1}(p)\conj{\b_{2}(p)}p^{-s}\right)\bigg].
            \end{align*}
        \endgroup
        The term in the brackets simplifies to
        \[
            \left(1-\a_{1}(p)\a_{2}(p)\conj{\b_{1}(p)}\conj{\b_{2}(p)}p^{-2s}\right)(\a_{1}(p)-\a_{2}(p))\left(\conj{\b_{1}(p)}-\conj{\b_{2}(p)}\right),
        \]
        because all of the other terms are killed by symmetry in $\a_{1}(p)$, $\a_{2}(p)$, $\conj{\b_{1}(p)}$, and $\conj{\b_{2}(p)}$. The Ramanujan conjecture implies $\a_{1}(p)\a_{2}(p)\conj{\b_{1}(p)}\conj{\b_{2}(p)} = 1$. Therefore the corresponding factor above is $(1-p^{-2s})$. This factor cancels the local factor at $p$ in the Euler product of $\z(2s)$, so that
        \[
            \sum_{n \ge 0}\frac{a_{f}(p^{n})\conj{b_{g}(p^{n})}}{p^{ns}} = \prod_{1 \le i,j \le 2}\left(1-\a_{i}(p)\conj{\b_{j}(p)}p^{-s}\right)^{-1}.
        \]
        So in total we have a degree $4$ Euler product,
        \[
            L(s,f \ox g) = \prod_{1 \le i,j \le 2}\left(1-\a_{i}(p)\conj{\b_{j}(p)}p^{-s}\right)^{-1}.
        \]
        Clearly $a_{f \ox g}(1) = 1$ because this is the constant term in the Euler product. In particular, we have verified property (i) and adjustment (i) for Rankin-Selberg convolutions and we package this into a theorem:

        \begin{theorem}
            Let $f,g \in \mc{S}_{k}(\PSL_{2}(\Z))$ be primitive Hecke eigenforms. For $\Re(s) > 1$,
            \[
            L(s,f \ox g) = \prod_{1 \le i,j \le 2}\left(1-\a_{i}(p)\conj{\b_{j}(p)}p^{-s}\right)^{-1},
            \]
            is locally absolutely uniformly convergent with degree $4$ Euler product.
        \end{theorem}

        \begin{remark}
            If $f \in \mc{S}_{k}(N,\chi)$ and $g \in \mc{S}_{\ell}(M,\psi)$, the Euler product becomes more difficult to compute since the local $p$ factors for $p \mid NM$ change as either $\a_{2}(p) = 0$ or $\b_{2}(p) = 0$. Moreover, the situation is increasing complicated if $(N,M) > 1$. In fact, if $(N,M) > 1$ then the conductor of $L(s,f \ox g)$ can be smaller than $NM$ which is known as \textbf{conductor dropping}\index{conductor dropping} and can cause serious obstructions to the theory.
        \end{remark}
        \subsection*{The Integral Representation of \texorpdfstring{$L(s,f \ox g)$}{L(s,f \ox g)}: Part I}
        We now look for a way to express $L(s,f \ox g)$ as an integral symmetric as $s \to 1-s$. The integral we want to consider is
        \[
            \int_{\G_{\infty}\backslash\H}f(z)\conj{g(z)}\Im(z)^{s+k}\,d\mu,
        \]
        where $\G_{\infty}$ is the associated subgroup corresponding to $\PSL_{2}(\Z)$. This will turn out to be a Mellin transform as we will soon see.

        \begin{remark}
            If $f \in \mc{S}_{k}(N,\chi)$ and $g \in \mc{S}_{\ell}(M,\psi)$ with $k \neq \ell$, then the power of $\Im(z)$ in the integrand needs to be adjusted.
        \end{remark}
        
        The region of convergence of this integral is not immediately clear because we cannot appeal to \cref{met:decay_compacta_integral} directly. Indeed,
        \[
            \G_{\infty}\backslash\H = \{z \in \H:0 \le \Re(z) \le 1 \},
        \]
        intersects infinitely many fundamental domains for $\G$. In any case, we have
        \begin{align*}
            \int_{\G_{\infty}\backslash\H}f(z)\conj{g(z)}\Im(z)^{s+k}\,d\mu &= \int_{0}^{\infty}\int_{0}^{1}f(x+iy)\conj{g(x+iy)}y^{s+k}\,\frac{dx\,dy}{y^{2}} \\
            &= \int_{0}^{\infty}\int_{0}^{1}\sum_{n,m \ge 1}a(n)\conj{b(m)}e^{2\pi i(n-m)x}e^{-2\pi(n+m)y}y^{s+k}\,\frac{dx\,dy}{y^{2}} \\
            &= \int_{0}^{\infty}\sum_{n,m \ge 1}\int_{0}^{1}a(n)\conj{b(m)}e^{2\pi i(n-m)x}e^{-2\pi(n+m)y}y^{s+k}\,\frac{dx\,dy}{y^{2}} && \text{DCT} \\
            &= \int_{0}^{\infty}\sum_{n \ge 1}a(n)\conj{b(n)}e^{-4\pi ny}y^{s+k}\,\frac{dy}{y^{2}},
        \end{align*}
        where the last line follows by \cref{equ:Dirac_integral_representation}. Notice that the last integral is a Mellin transform. The rest is a computation:
        \begin{align*}
            \int_{0}^{\infty}\sum_{n \ge 1}a(n)\conj{b(n)}e^{-4\pi ny}y^{s+k}\,\frac{dy}{y^{2}} &= \sum_{n \ge 1}a(n)\conj{b(n)}\int_{0}^{\infty}e^{-4\pi ny}y^{s+k}\,\frac{dy}{y^{2}} &&\text{DCT} \\
            &= \sum_{n \ge 1}\frac{a(n)\conj{b(n)}}{(4\pi n)^{s+k-1}}\int_{0}^{\infty}e^{-y}y^{s+k-1}\,\frac{dy}{y} &&\text{$y \to \frac{y}{4\pi n}$} \\
            &= \frac{\G\left(s+k-1\right)}{(4\pi)^{s+k-1}}\sum_{n \ge 1}\frac{a(n)\conj{b(n)}}{n^{s+k-1}} \\
            &= \frac{\G\left(s+k-1\right)}{(4\pi)^{s+k-1}}L(s,f \x g).
        \end{align*}
        This last expression is locally absolutely uniformly convergent for $\Re(s) > 1$ because the $L$-function is and the gamma factor is holomorphic in this region. Therefore the original integral is locally absolutely uniformly bounded. At this point we have an integral representation
        \[
            L(s,f \x g) = \frac{(4\pi)^{s+k-1}}{\G(s+k-1)}\int_{\G_{\infty}\backslash\H}f(z)\conj{g(z)}\Im(z)^{s+k}\,d\mu.
        \]
        We rewrite the integral as follows:
        \begin{align*}
            \int_{\G_{\infty}\backslash\H}f(z)\conj{g(z)}\Im(z)^{s+k}\,d\mu &= \int_{\mc{F}}\sum_{\g \in \GG}f(\g z)\conj{g(\g z)}\Im(\g z)^{s+k}\,d\mu && \text{folding} \\
            &= \int_{\mc{F}}\sum_{\g \in \GG}j(\g,z)^{k}\conj{j(\g,z)^{k}}f(z)\conj{g(z)}\Im(\g z)^{s+k}\,d\mu && \text{modularity} \\
            &= \int_{\mc{F}}f(z)\conj{g(z)}\sum_{\g \in \GG}|j(\g,z)|^{2k}\Im(\g z)^{s+k}\,d\mu \\
            &= \int_{\mc{F}}f(z)\conj{g(z)}\Im(z)^{k}\sum_{\g \in \GG}\Im(\g z)^{s}\,d\mu \\
            &= \int_{\mc{F}}f(z)\conj{g(z)}\Im(z)^{k}E_{\infty}(z,s)\,d\mu.
        \end{align*}
        Note that $E_{\infty}(z,s)$ is the Eisenstein series on $\GH$ at the $\infty$ cusp. Altogether this gives the integral representation
        \begin{equation}\label{equ:Rankin-Selberg_integral-reresentation}
            L(s,f \x g) =  \frac{(4\pi)^{s+k-1}}{\G(s+k-1)}\int_{\mc{F}}f(z)\conj{g(z)}\Im(z)^{k}E_{\infty}(z,s)\,d\mu.
        \end{equation}
        which is valid for $\Re(s) > 1$. We cannot investigate the integral any further until we understand the Fourier coefficients of $E_{\infty}(z,s)$. Therefore we will take a necessary detour and return to the integral after.

        \begin{remark}
            For $f \in \mc{S}_{k}(N,\chi)$ and $g \in \mc{S}_{\ell}(M,\psi)$, the integral representation involves a different Eisenstein series other than $E_{\infty}(z,s)$ because we can not easily combine the factors $\chi(\g)j(\g,z)^{k}$ and $\conj{\psi(\g)j(\g,z)^{\ell}}$ that appear after folding.
        \end{remark}
        \subsection*{The Fourier Series of \texorpdfstring{$E_{\infty}(z,s)$}{E_{\infty}(z,s)}}
        Let
        \[
            E_{\infty}(z,s) = \sum_{n \in \Z}a(n,y,s)e^{2\pi inx}.
        \]
        be the Fourier series of $E_{\infty}(z,s)$. Our task now is to compute these coefficients. To do this we will need the following technical lemma:

        \begin{lemma}\label{lem:Ramanujan_zeta_relation}
            Let $M \ge 1$ be square-free. Then for $\Re(s) > 1$ and $b \in \Z$,
            \[
            \sum_{m \ge 1}\frac{r(b;m)}{m^{2s}} = \begin{cases} \frac{\z(2s-1)}{\z(2s)} & \text{if $b = 0$}, \\ \frac{\s_{1-2s}(|b|)}{\z(2s)} & \text{if $b \neq 0$}, \end{cases}
            \]
            where $\s_{s}(b)$ is the generalized sum of divisors function.
        \end{lemma}
        \begin{proof}
            If $\Re(s) > 1$ then the desired evaulation of the sum is locally absolutely uniformly convergent because the zeta function is in that region. Hence the sum will be too provided we prove the identity. Suppose $b = 0$. Then $r(0;m) = \phi(m)$. Since $\phi(m)$ is multiplicative we have
            \begin{equation}\label{equ:Ramanujan_zeta_relation_1}
            \sum_{m \ge 1}\frac{\phi(m)}{m^{2s}} = \prod_{p}\left(\sum_{k \ge 0}\frac{\phi(p^{k})}{p^{k(2s)}}\right).
            \end{equation}
            Recalling that $\phi(p^{k}) = p^{k}-p^{k-1}$ for $k \ge 1$, make the following computation:
            \begin{equation}\label{equ:Ramanujan_zeta_relation_2}
            \begin{aligned}
                \sum_{k \ge 0}\frac{\phi(p^{k})}{p^{k(2s)}} &= 1+\sum_{k \ge 1}\frac{p^{k}-p^{k-1}}{p^{k(2s)}} \\
                &= \sum_{k \ge 0}\frac{1}{p^{k(2s-1)}}-\frac{1}{p}\sum_{k \ge 1}\frac{1}{p^{k(2s-1)}} \\
                &= \sum_{k \ge 0}\frac{1}{p^{k(2s-1)}}-p^{-2s}\sum_{k \ge 0}\frac{1}{p^{k(2s-1)}} \\
                &= (1-p^{-2s})\sum_{k \ge 0}\frac{1}{p^{k(2s-1)}} \\
                &= \frac{1-p^{-2s}}{1-p^{-(2s-1)}}.
            \end{aligned}
            \end{equation}
            Combining \cref{equ:Ramanujan_zeta_relation_1,equ:Ramanujan_zeta_relation_2} gives
            \[
            \sum_{m \ge 1}\frac{\phi(m)}{m^{2s}} = \frac{\z(2s-1)}{\z(2s)}.
            \]
            Now suppose $b \neq 0$, \cref{prop:Ramanujan_sum_evaluation} gives the first equality in the following chain:
            \begin{align*}
            \sum_{m \ge 1}\frac{r(b;m)}{m^{2s}} &= \sum_{m \ge 1}m^{-2s}\sum_{\ell \mid (b,m)}\ell\mu\left(\frac{m}{\ell}\right) \\
            &= \sum_{\ell \mid b}\ell\sum_{m \ge 1}\frac{\mu(m)}{(m\ell)^{2s}} \\
            &= \left(\sum_{\ell \mid b}\ell^{1-2s}\right)\left(\sum_{m \ge 1}\frac{\mu(m)}{m^{2s}}\right) \\
            &= \s_{1-2s}(b)\sum_{m \ge 1}\frac{\mu(m)}{m^{2s}} \\
            &= \s_{1-2s}(|b|)\sum_{m \ge 1}\frac{\mu(m)}{m^{2s}} \\
            &= \frac{\s_{1-2s}(|b|)}{\z(2s)} && \text{\cref{prop:Dirichlet_Mobius_is_zeta_inverse}}.
            \end{align*}
        \end{proof}

        We can now compute the Fourier coefficients $a(n,y,s)$:

        \begin{proposition}\label{prop:Fourier_coefficients_of_real-analytic_Eisenstein_series}
            The Fourier coefficients $a(n,y,s)$ are given by
            \[
            a(n,y,s) = \begin{cases} y^{s}+y^{1-s}\frac{\sqrt{\pi}\G\left(s-\frac{1}{2}\right)\z(2s-1)}{\G(s)\z(2s)} & \text{if $n = 0$}, \\ 2\pi^{s}\frac{|n|^{s-\frac{1}{2}}\s_{1-2s}(|n|)}{\G(s)\z(2s)}\sqrt{y}K_{s-\frac{1}{2}}(2\pi|n|y) & \text{if $n \neq 0$}, \end{cases}
            \]
            where $K_{s}(y)$ is the $K$-Bessel function.
        \end{proposition}
        \begin{proof}
            By \cref{rem:Bruhat_bijection}, we have
            \[
            E_{\infty}(z,s) = y^{s}+\sum_{\substack{c \ge 1, d \in \Z \\ (c,d) = 1}}\frac{y^{s}}{|cx+icy+d|^{2s}}.
            \]
            So the definition of the Fourier coefficients gives
            \begin{equation}\label{equ:real-analytic_Eisenstein_fourier_coefficient_definition}
            a(n,y,s) = \int_{0}^{1}E(x+iy,s)e^{-2\pi inx}\,dx = y^{s}\d_{n,0}+\int_{0}^{1}\sum_{\substack{c \ge 1, d \in \Z \\ (c,d) = 1}}\frac{y^{s}}{|cx+icy+d|^{2s}}e^{-2\pi inx}\,dx,
            \end{equation}
            where we have used \cref{equ:Dirac_integral_representation}. We are now reduced to computing
            \[
            I(z,s) = \int_{0}^{1}\sum_{\substack{c \ge 1, d \in \Z \\ (c,d) = 1}}\frac{y^{s}}{|cx+icy+d|^{2s}}e^{-2\pi inx}\,dx.
            \]
            Make the following observation: summing over all pairs $(c,d)$ with $c \ge 1$, $d \in \Z$, $c \equiv 0 \tmod{NM}$, and $(c,d) = 1$ is the same as summing over all triples $(c,\ell,r)$ with $c \ge 1$, $\ell \in \Z$, $r \tmod{c}$, with $c \equiv 0 \tmod{NM}$ and $(r,c) = 1$. This is seen by writing $d = c\ell+r$. Therefore
            \begin{align*}
            I(z,s) &= \int_{0}^{1}\sum_{\substack{c \ge 1, d \in \Z \\ (c,d) = 1}}\frac{y^{s}}{|cx+icy+d|^{2s}}e^{-2\pi inx}\,dx \\
            &= \int_{0}^{1}\sum_{(c,\ell,r)}\frac{y^{s}}{|c(x+\ell)+r+icy|^{2s}}e^{-2\pi inx}\,dx \\
            &= \sum_{(c,\ell,r)}\int_{0}^{1}\frac{y^{s}}{|c(x+\ell)+r+icy|^{2s}}e^{-2\pi inx}\,dx &&\text{DCT} \\
            &= \sum_{(c,\ell,r)}\frac{1}{c^{2s}}\int_{0}^{1}\frac{y^{s}}{\left|x+\ell+\frac{r}{c}+iy\right|^{2s}}e^{-2\pi inx}\,dx \\
            &= \sum_{(c,\ell,r)}\frac{1}{c^{2s}}\int_{\ell}^{\ell+1}\frac{y^{s}}{\left|x+\frac{r}{c}+iy\right|^{2s}}e^{-2\pi inx}\,dx &&\text{$x \to x-\ell$} \\
            &= \psum_{\substack{c \ge 1 \\ r \tmod{c}}}\frac{1}{c^{2s}}\int_{-\infty}^{\infty}\frac{y^{s}}{\left|x+\frac{r}{c}+iy\right|^{2s}}e^{-2\pi inx}\,dx &&\text{DCT} \\
            &= \sum_{c \ge 1}\frac{1}{c^{2s}}\psum_{r \tmod{c}}e^{\frac{2\pi inr}{c}}\int_{-\infty}^{\infty}\frac{y^{s}}{(x^{2}+y^{2})^{s}}e^{-2\pi inx}\,dx &&\text{$x \to x-\frac{r}{c}$} \\
            &= \sum_{c \ge 1}\frac{r(n;c)}{c^{2s}}\int_{-\infty}^{\infty}\frac{y^{s}}{(x^{2}+y^{2})^{s}}e^{-2\pi inx}\,dx \\
            &= \sum_{c \ge 1}\frac{r(n;c)}{c^{2s}}y^{1-s}\int_{-\infty}^{\infty}\frac{1}{(x^{2}+1)^{s}}e^{-2\pi inxy}\,dx && \text{$x \to xy$},
            \end{align*}
            where on the right-hand side it is understood we are summing over all triples $(c,\ell,r)$ with the prescribed properties. Using \cref{lem:Ramanujan_zeta_relation} yields
            \[
            I(z,s) = \begin{cases} \frac{\z(2s-1)}{\z(2s)}y^{1-s}\int_{-\infty}^{\infty}\frac{1}{(x^{2}+1)^{s}}\,dx & \text{if $n = 0$}, \\ \frac{\s_{1-2s}(|n|)}{\z(2s)}y^{1-s}\int_{-\infty}^{\infty}\frac{1}{(x^{2}+1)^{s}}e^{-2\pi inxy}\,dx & \text{if $n \neq 0$}. \end{cases}
            \]
            Appealing to \cref{append:Special_Integrals} for this last integral and substiuting the result back into \cref{equ:real-analytic_Eisenstein_fourier_coefficient_definition} finishes the proof.
        \end{proof}
        \subsection*{The Completed Real-analytic Eisenstein Series \texorpdfstring{$E^{\ast}(z,s)$}{E^{\ast}(z,s)}}
        We would like to analytically continue $E_{\infty}(z,s)$ in $s$ past the region $\Re(s) > 1$ (this follows from \cref{thm:meromorphic_continuation_of_Eisenstein_series} but we will demonstrate a full proof for the single Eisenstein series on $\PSL_{2}(\Z)\backslash\H$). From the computation of the Fourier coefficients we will have possible poles coming from the denominator of the Fourier coefficients. To remove this difficultly, we will multiply by a factor to clear the poles. In turn, this will give us a functional equation as $s \to 1-s$. The factor will be the completed zeta function $\L(2s,\z) = \pi^{-s}\G(s)\z(2s)$ (scaled by $2$). We define the the \textbf{completed real-analytic Eisenstein series}\index{completed real-analytic Eisenstein series} $E^{\ast}(z,s)$ by
        \[
            E^{\ast}(z,s) = \L(2s,\z)E_{\infty}(z,s) = \pi^{-s}\G(s)\z(2s)E_{\infty}(z,s).
        \]
        From \cref{prop:Fourier_coefficients_of_real-analytic_Eisenstein_series}, the Fourier coefficients $a^{\ast}(n,y,s)$ of $E^{\ast}(z,s)$ are given by
        \[
            a^{\ast}(n,y,s) = \begin{cases} y^{s}\pi^{-s}\G(s)\z(2s)+y^{1-s}\pi^{-\left(s-\frac{1}{2}\right)}\G(s-\frac{1}{2})\z(2s-1) & \text{if $n = 0$}, \\ 2|n|^{s-\frac{1}{2}}\s_{1-2s}(|n|)\sqrt{y}K_{s-\frac{1}{2}}(2\pi|n|y) & \text{if $n \neq 0$}. \end{cases}
        \]
        Our goal is to now derive a functional equation for $E^{\ast}(z,s)$. Using the definition and functional equation for $\L(2s-1,\z)$, we can rewrite the second term in the $n = 0$ coefficient to get
        \begin{equation}\label{equ:Fourier_coefficients_for_completed_real-analytic_Eisenstein_series}
            a^{\ast}(n,y,s) = \begin{cases} y^{s}\L(2s,\z)+y^{1-s}\L(2(1-s),\z) & \text{if $n = 0$}, \\ 2|n|^{s-\frac{1}{2}}\s_{1-2s}(|n|)\sqrt{y}K_{s-\frac{1}{2}}(2\pi|n|y) & \text{if $n \neq 0$}. \end{cases}
        \end{equation}
        Now observe that the $n = 0$ coefficient is invariant under $s \to 1-s$. Each $n \neq 0$ coefficeint is also invariant under $s \to 1-s$. To see this we will use two facts. First, from \cref{append:Bessel_Functions}, $K_{s}(y)$ is invariant under $s \to -s$ and $s-\frac{1}{2} \mapsto \frac{1}{2}-s$ under $s \to 1-s$. Therefore $K_{s-\frac{1}{2}}(2\pi|n|y)$ is invariant as $s \to 1-s$. Second, for $n \ge 1$ we have
        \[
            n^{s-\frac{1}{2}}\s_{1-2s}(n) = n^{\frac{1}{2}-s}n^{2s-1}\s_{1-2s}(n) = n^{\frac{1}{2}-s}n^{2s-1}\sum_{d \mid n}d^{1-2s} = n^{\frac{1}{2}-s}\sum_{d \mid n}\left(\frac{n}{d}\right)^{2s-1} = n^{\frac{1}{2}-s}\s_{2s-1}(n),
        \]
        where the second to last equality follows by writing $n^{2s-1} = \left(\frac{n}{d}\right)^{2s-1}d^{2s-1}$ for each $d \mid n$. These two facts together give the invariance of the $n \neq 0$ coefficients under $s \to 1-s$. Altogether, we have shown the following functional equation for $E^{\ast}(z,s)$:
        \[
            E^{\ast}(z,s) = E^{\ast}(z,1-s).
        \]
        It follows that for fixed $z \in \H$, $E^{\ast}(z,s)$ is holomorphic on the region
        \[
            \left\{s \in \C:\left|\Re(s)-\frac{1}{2}\right| > \frac{1}{2}\right\}.
        \]

        \begin{remark}
            For $f \in \mc{S}_{k}(N,\chi)$ and $g \in \mc{S}_{\ell}(M,\psi)$, we use a modified version of the Eisenstein series $E_{\infty}(z,s)$ on $\G_{0}(NM)\backslash\H$. In particular if $NM > 1$, then on this congruence subgroup there are more cusps than just the cusp at $\infty$. In particular, there is also a cusp at $0$. The functional equation of the modified Eisenstein series at the $\infty$ cusp becomes much more complicated as it reflects into a modified version of the Eisenstein series at the $0$ cusp as $s \to 1-s$. Thus, we need to understand the Fourier coefficients of both of these Eisenstein series.
        \end{remark}

        We now obtain meromorphic continuation of $E^{\ast}(z,s)$ inside the critical strip and therefore meromorphic continuation in $s$ to all of $\C$. We first write $E^{\ast}(z,s)$ as a Fourier series using \cref{equ:Fourier_coefficients_for_completed_real-analytic_Eisenstein_series}:
        \[
            E^{\ast}(z,s) = y^{s}\L(2s,\z)+y^{1-s}\L(2(1-s),\z)+\sum_{n \neq 0}2|n|^{s-\frac{1}{2}}\s_{1-2s}(|n|)\sqrt{y}K_{s-\frac{1}{2}}(2\pi|n|y)e^{2\pi inx}.
        \]
        This is valid for any $y > 0$ and $|\Re(s)-\frac{1}{2}| > \frac{1}{2}$. Since $\L(2s,\z)$ is meromorphic in the critical strip $|\Re(s)-\frac{1}{2}| \le \frac{1}{2}$, the constant term of $E^{\ast}(z,s)$ is meromorphic in this region. To prove the meromorphic continuation of $E^{\ast}(z,s)$ it now suffices to show
        \[
            \sum_{n \neq 0}2|n|^{s-\frac{1}{2}}\s_{1-2s}(|n|)\sqrt{y}K_{s-\frac{1}{2}}(2\pi|n|y)e^{2\pi inx},
        \]
        is meromorphic as well. We will actually prove it is locally absolutely uniformly convergent. To achieve this we need two bounds, one for $\s_{1-2s}(|n|)$ and one for $K_{s-\frac{1}{2}}(2\pi|n|y)$. For the first bound, we use the estimate $\s_{0}(n) \ll n^{\e}$ (see \cite{montgomery2006multiplicative} for a proof). Therefore we have the crude bound
        \begin{equation}\label{completed_real-analytic_Eisenstein_series_bound_1}
            \s_{1-2s}(|n|) = \sum_{d \mid n}d^{1-2s} < \s_{0}(|n|)|n|^{1-2s} \ll_{\e}|n|^{1-2s+\e}.
        \end{equation}
        For the second estimate, \cref{lem:exponential_decay_K-Bessel_function} and that $\Re(s)$ is bounded give
        \begin{equation}\label{completed_real-analytic_Eisenstein_series_bound_2}
            K_{s-\frac{1}{2}}(2\pi|n|y) \ll e^{-2\pi|n|y}.
        \end{equation}
        Then \cref{completed_real-analytic_Eisenstein_series_bound_1,completed_real-analytic_Eisenstein_series_bound_2} together imply
        \begin{equation}\label{equ:non-constant_Fourier_coefficient_bound_non-holomorphic_Eisenstein_series}
            \sum_{n \neq 0}2|n|^{s-\frac{1}{2}}\s_{1-2s}(|n|)\sqrt{y}K_{s-\frac{1}{2}}(2\pi|n|y)e^{2\pi inx} \ll_{\e} \sum_{n \ge 1}4n^{\frac{1}{2}-s+\e}\sqrt{y}e^{-2\pi ny},
        \end{equation}
        and this latter series is locally absolutely uniformly convergent by the ratio test. The meromorphic continuation of $E^{\ast}(z,s)$ to $\C$ in $s$ follows. It remains to investigate the poles and residues which we now do. We will acomplish this from direct inspection of the Fourier coefficients:

        \begin{proposition}\label{equ:completed_real-analytic_Eisenstein_series_residues}
            $E^{\ast}(z,s)$ has simple poles at $s = 0$ and $s = 1$, and
            \[
            \Res_{s = 0}E^{\ast}(z,s) = -\frac{1}{2} \quad \text{and} \quad \Res_{s = 1}E^{\ast}(z,s) = \frac{1}{2}.
            \]
        \end{proposition}
        \begin{proof}
            Since the constant term in the Fourier series of $E^{\ast}(z,s)$ is the only non-holomorphic term, poles of $E^{\ast}(z,s)$ can only come from that term. So we are reduced to understanding the poles of
            \begin{equation}\label{equ:constant_coefficient_of_completed_non-holomorphic_Eisenstein_series}
                y^{s}\L(2s,\z)+y^{1-s}\L(2(1-s),\z).
            \end{equation}
            Notice $\L(2s,\z)$ has simple poles at $s = 0$, $s = \frac{1}{2}$ (one from the zeta function and one from the gamma factor) and no others. It follows that $E^{\ast}(z,s)$ has a simple pole at $s = 0$ coming from the $y^{s}$ term in
            \cref{equ:constant_coefficient_of_completed_non-holomorphic_Eisenstein_series}, and by the functional equation there is also a pole at $s = 1$ coming from the $y^{1-s}$ term. At $s = \frac{1}{2}$, both terms in \cref{equ:constant_coefficient_of_completed_non-holomorphic_Eisenstein_series} have simple poles and we will show that the singularity there is removable. Recall $\G\left(\frac{1}{2}\right) = \sqrt{\pi}$. Also, by \cref{prop:zeta_residue}, $\Res_{s = \frac{1}{2}}\z(2s) = \frac{1}{2}$ and $\Res_{s = \frac{1}{2}}\z(2(1-s)) = -\frac{1}{2}$. So altogether
            \[
            \Res_{s = \frac{1}{2}}E^{\ast}(z,s) = \Res_{s = \frac{1}{2}}[y^{s}\L(2s,\z)+y^{1-s}\L(2(1-s),\z)] = \frac{1}{2}y^{\frac{1}{2}}-\frac{1}{2}y^{\frac{1}{2}} = 0.
            \]
            Hence the singularity at $s = \frac{1}{2}$ is removable. As for the residues at $s = 0$ and $s = 1$, the functional equation implies that they are negatives of each other. So it suffices to compute the residue at $s = 0$. Recall $\z(0) = -\frac{1}{2}$ and $\Res_{s = 0}\G(s) = 1$. Then together we find
            \[
            \Res_{s = 0}E^{\ast}(z,s) = \Res_{s = 0}y^{s}\L(2s,\z) = -\frac{1}{2}.
            \]
        \end{proof}

        This completes our study of the real-analytic Eisenstein series $E_{\infty}(z,s)$.
        \subsection*{The Integral Representation of \texorpdfstring{$L(s,f \ox g)$}{L(s,f \ox g)}: Part II}
        We now have enough information to further understand the Rankin-Selberg convolution $L(s,f \ox g)$. Writing \cref{equ:Rankin-Selberg_integral-reresentation} in terms of $E^{\ast}(z,s)$ and $L(s,f \ox g)$ gives the following result:

        \begin{theorem}
            Let $f,g \in \mc{S}_{k}(\PSL_{2}(\Z))$ be primitive Hecke eigenforms. For $\Re(s) > 1$,
            \[
            L(s,f \ox g) = \frac{(4\pi)^{s+k-1}\pi^{s}}{\G(s+k-1)\G(s)}\int_{\mc{F}}f(z)\conj{g(z)}\Im(z)^{k}E^{\ast}(z,s)\,d\mu.
            \]
        \end{theorem}

        This integral will give analytic continuation. To see this, note that the gamme factors are analytic for $\Re(s) < 0$. By the functional equation for $E^{\ast}(z,s)$, the integral is invariant as $s \to 1-s$. These two facts together give analytic continuation to the region
        \[
            \left\{s \in \C:\left|\Re(s)-\frac{1}{2}\right| > \frac{1}{2}\right\}.
        \]
        \subsection*{The Functional Equation, Critical Strip \& Residues of \texorpdfstring{$L(s,f \ox g)$}{L(s,f \ox g)}}
        An immediate consequence of the symmetry of integral representation is the functional equation:
        \[
            \frac{\G(s+k-1)\G(s)}{(4\pi)^{s+k-1}\pi^{s}}L(s,f \ox g) = \frac{\G((1-s)+k-1)\G(1-s)}{(4\pi)^{(1-s)+k-1}\pi^{1-s}}L(1-s,f \ox g).
        \]
        Applying the Legendre duplication formula for the gamma function twice we see that
        \begin{equation}\label{equ:duplication_for_Rankin-Selberg_gamma_factor}
            \begin{aligned}
            \frac{\G(s+k-1)\G(s)}{(4\pi)^{s+k-1}\pi^{s}} &= \frac{2^{2s+k-3}}{(4\pi)^{s+k-1}\pi^{s+1}}\G\left(\frac{s+k-1}{2}\right)\G\left(\frac{s+k}{2}\right)\G\left(\frac{s}{2}\right)\G\left(\frac{s+1}{2}\right) \\
            &= \frac{1}{2^{k+1}\pi^{k}}\pi^{-2s}\G\left(\frac{s+k-1}{2}\right)\G\left(\frac{s+k}{2}\right)\G\left(\frac{s}{2}\right)\G\left(\frac{s+1}{2}\right).
            \end{aligned}
        \end{equation}
        The factor in front is independent of $s$ and can therefore be canceled in the functional equation. We identify the gamma factor as:
        \[
            \g(s,f \ox g) = \pi^{-2s}\G\left(\frac{s+k-1}{2}\right)\G\left(\frac{s+k}{2}\right)\G\left(\frac{s}{2}\right)\G\left(\frac{s+1}{2}\right),
        \]
        with $\mu_{1,1} = k-1$, $\mu_{2,2} = k$, $\mu_{1,2} = 0$, and $\mu_{2,1} = 1$ the local parameters at infinity. Clearly they satisfy the required bounds. The completed $L$-function is
        \[
            \L(s,f \ox g) = \pi^{-2s}\G\left(\frac{s+k-1}{2}\right)\G\left(\frac{s+k}{2}\right)\G\left(\frac{s}{2}\right)\G\left(\frac{s+1}{2}\right)L(s,f \ox g),
        \]
        so the conductor is $q(f \ox g) = 1$ and no primes ramify. Clearly, $q(f \ox g) \mid q(f)^{2}q(g)^{2}$. Then
        \[
            \L(s,f \ox g) = \L(1-s,f \ox g),
        \]
        is the functional equation of $L(s,f \ox g)$. In particular, the root number $\e(f \ox g) = 1$, and $L(s,f \ox g)$ is self-dual. In total, we have verified properties (ii)-(iv) and adjustments (ii) and (iii) for Rankin-Selberg convolutions.

        We now want to get continuation inside of the critical strip. However, we will only be able to obtain meromorphic continuation inside the critical strip because of the presence of poles for $E^{\ast}(z,s)$. Taking the integral representation and substituting the Fourier series for $E^{\ast}(z,s)$ gives
        \begin{equation}\label{equ:integral_representation_Rankin_Selberg}
            \begin{aligned}
            L(s,f \ox g) &= \frac{(4\pi)^{s+k-1}\pi^{s}}{\G(s+k-1)\G(s)}\bigg[\int_{\mc{F}}f(x+iy)\conj{g(x+iy)}y^{k}(y^{s}\L(2s,\z)+y^{1-s}\L(2(1-s),\z))\,\frac{dx\,dy}{y^{2}} \\
            &+\int_{\mc{F}}f(x+iy)\conj{g(x+iy)}y^{k}\sum_{n \neq 0}2|n|^{s-\frac{1}{2}}\s_{1-2s}(|n|)\sqrt{y}K_{s-\frac{1}{2}}(2\pi|n|y)e^{2\pi inx}\,\frac{dx\,dy}{y^{2}}\bigg],
            \end{aligned}
        \end{equation}
        and we are reduced to showing that both integrals are locally absolutely uniformly bounded in this region away from poles. To this end, let $s$ be such that $|\Re(s)-\frac{1}{2}| \le \frac{1}{2}$ and is away the points $0$ and $1$ so that the completed zeta functions are holomorphic. Then the first integral is locally absolutely uniformly bounded in this region by \cref{met:decay_compacta_integral}. As for the second integral, \cref{equ:non-constant_Fourier_coefficient_bound_non-holomorphic_Eisenstein_series} implies that it is
        \[
            O_{\e}\left(\int_{\mc{F}}f(x+iy)\conj{g(x+iy)}y^{k}\sum_{n \ge 1}4n^{\frac{1}{2}-s+\e}\sqrt{y}e^{-2\pi ny}\,\frac{dx\,dy}{y^{2}}\right).
        \]
        As the sum in the integrand is holomorphic, we can now appeal to \cref{met:decay_compacta_integral}. The meromorphic continuation to the critical strip and hence to all of $\C$ follows.

        Since the poles of $E^{\ast}(z,s)$ are simple poles at $s = 0$ and $s = 1$, $L(s,f \ox g)$ has at most simple poles here too. Actually, $\g(s,f \ox g)$ has a simple pole at $s = 0$ coming from the gamma factors and therefore its reciprocal has a simple zero. This cancels the possible simple pole at $s = 0$ coming from $E^{\ast}(z,s)$ and therefore $L(s,f \ox g)$ is actually holomorphic at $s = 0$. So there is at most a simple pole at $s = 1$.
        
        We can now show that $L(s,f \ox g)$ is of order $1$ and conclude that it satisfies property (v) of Rankin-Selberg convolutions. Since the pole at $s = 1$ is at worst simple, multiplying by $(s-1)$ clears the polar divisor. As the integral in the integral representation is absolutely bounded, computing the order amounts to estimating the gamma factor. Since the reciprocal of the gamma function is of order $1$ and $\Re(s)$ is bounded, we have
        \begin{equation}\label{equ:Rankin-Selberg_gamma_factor_order_1}
            \frac{1}{\g(s,f \ox g)} \ll_{\e} e^{|s|^{1+\e}},
        \end{equation}
        for any $\e > 0$. So the reciprocal of the gamma factor is of the same order. Then \cref{equ:Rankin-Selberg_gamma_factor_order_1,equ:integral_representation_Rankin_Selberg} together imply
        \[
            (s-1)L(s,f \ox g) \ll_{\e} e^{|s|^{1+\e}}.
        \]
        Thus $(s-1)L(s,f \ox g)$ is of order $1$, and so $L(s,f \ox g)$ is as well after removing the polar factor. At last, we compute the residue of $L(s,f \ox g)$ at $s = 1$:
        \begin{proposition}
            Let $f,g \in \mc{S}_{k}(\PSL_{2}(\Z))$ be primitive Hecke eigenforms. Then
            \[
            \Res_{s = 1}L(s,f \ox g) = \frac{4^{k}\pi^{k+1}}{2\G(k)}\<f,g\>,
            \]
            where $\<f,g\>$ is the Petersson inner product.
        \end{proposition}
        \begin{proof}
            From \cref{equ:duplication_for_Rankin-Selberg_gamma_factor}, we see that
            \[
            \lim_{s \to 1}2^{k+1}\pi^{k}\frac{1}{\g(s,f \ox g)} = \frac{4^{k}\pi^{k+1}}{\G(k)}.
            \]
            Then \cref{equ:completed_real-analytic_Eisenstein_series_residues} implies
            \[
            \Res_{s = 1}L(s,f \ox g) = \frac{4^{k}\pi^{k+1}}{\G(k)}\Res_{s = 1}\int_{\mc{F}}f(z)\conj{g(z)}\Im(z)^{k}E^{\ast}(z,s)\,d\mu = \frac{4^{k}\pi^{k+1}}{2\G(k)}\<f,g\>.
            \]
        \end{proof}

        Notice that if $g = f$, then $\<f,f\> \neq 0$ and therefore the residue at $s = 1$ is not zero and hence there is a genuine pole. Actually, by \cref{thm:newforms_characterization_holomorphic} the primitive Hecke eigenforms are orthogonal so that $\<f,g\> = 0$ unless $f = g$. This is the only instance in which there is a pole. This verifies adjustment (iv) for Rankin-Selberg convolutions, and therefore we have shown altogether that $L(s,f \ox g)$ is the Rankin-Selberg convolution of $L(s,f)$ and $L(s,g)$. We summarize all of our work into the following theorem:

        \begin{theorem}
            For any two primitive Hecke eigenforms $f,g \in \mc{S}_{k}(\PSL_{2}(\Z))$, $L(s,f \ox g)$ is a Selberg class $L$-function. It admits meromorphic continuation to $\C$ via the integral representation
            \[
            L(s,f \ox g) = \frac{(4\pi)^{s+k-1}\pi^{s}}{\G(s+k-1)\G(s)}\int_{\mc{F}}f(z)\conj{g(z)}\Im(z)^{k}E^{\ast}(z,s)\,d\mu,
            \]
            with functional equation
            \[
            \pi^{-2s}\G\left(\frac{s+k-1}{2}\right)\G\left(\frac{s+k}{2}\right)\G\left(\frac{s}{2}\right)\G\left(\frac{s+1}{2}\right)L(s,f \ox g) = \L(s,f \ox g) = \L(1-s,f \ox g),
            \]
            and if $f = g$ there is simple pole at $s = 1$ of residue $\frac{4^{k}\pi^{k+1}}{2\G(k)}\<f,g\>$.
        \end{theorem}
    \section{Theta Functions}
        At this point we have shown that the Riemann zeta function, Dirichlet $L$-functions, and Hecke $L$-functions, all admit meromorphic continuation to $\C$ and satisfy a functional equation of shape $s \to 1-s$. In all of these cases, the idea was to find an integral representation that is meromorphic on $\C$ and symmetric under $s \to 1-s$. There is a unifying idea which encompasses all of these cases and more. That idea is lifting a transformation law of a theta function by taking its Mellin transform. To connect the Mellin transform to $L$-functions, we require theta functions. For our purposes, a \textbf{theta function}\index{theta function} is an infinite series indexed over a lattice whose terms are exponentials. We also require the theta function to be holomorphic on $\C$ and admit exponential decay to zero near $\infty$. Each of the $L$-functions we have studied, excluding the Rankin-Selberg convolution, is associated to a theta function:
        \begin{align*}
        \z(s) &\longleftrightarrow \vt(s) = \sum_{n \in \Z}e^{-\pi n^{2}s}, \\
        L(s,\chi) &\longleftrightarrow \vt_{\chi}(s) = \sum_{n \in \Z}\chi(n)n^{\mf{a}}e^{-\pi n^{2}s}, \\
        L(s,f) &\longleftrightarrow f(iy) = \sum_{n \in \Z}a(n)e^{-\pi ny},
        \end{align*}
        where in the last case we note that $a(n) = 0$ for $n < 0$ because $f$ is holomorphic and $a(0) = 0$ because $f$ is cuspidal. Now on the one hand, all of these theta functions can be written as sums over $n \ge 1$: the first two cases by symmetry of the $n$ and $-n$ terms and the last case by the above comment. Isolating the subsum over $n \ge 1$ and specializing at a nonnegative real variable we get $\w(x)$, $\w_{\chi}(x)$, and $f(iy)$ respectively. Taking the Mellin transform of these latter functions reproduced the associated $L$-functions up to gamma factors. We then decomposed the Mellin transform into two pieces and symmetrized the result by using transformation laws for the theta functions. The most difficult part of all of these arguments was cooking up the theta function that corresponds to the $L$-function solely from its representation as a Dirichlet series in the region of absolute convergence. For the Riemann zeta function this was essentially ``Riemann's insight'': start with the Gamma function, apply a change of variables, and sum over all $n \ge 1$ to obtain the Mellin transform of the corresponding theta function. For $L(s,\chi)$, the argument is adapted from that of Riemann although it is more complicated since the associated theta function depends on if the character $\chi$ is even or odd (odd being the more difficult case). On the other hand, $L(s,f)$ has the advantage that the theta function is easy to guess outright. It's the Fourier series of $f$ along the upper-half imaginary axis. Moreover, it comes equip with the necessary transformation law via the modularity of $f$. In more general settings, we need to know something algebraic or geometric about a given theta function in order to deduce a transformation law that can be used to prove meromorphic continuation and a functional equation for its associated $L$-function. In general the method of meromorphic continuation is as follows:

        \begin{method}
        Suppose we are given an $L$-series
        \[
            L(s,f) = \sum_{n \ge 1}\frac{a(n)}{n^{s}},
        \]
        that is locally absolutely uniformly convergent for $\Re(s) > 1$, and there is a theta function $\w_{f}(s)$ such that $L(s,f)$ is approximately the Mellin transform of $\w_{f}(s)$. That is, 
        \[
            L(s,f) \approx \int_{0}^{\infty}\w_{f}(s)x^{s}\,\frac{dx}{x}.
        \]
        Also suppose $\w_{f}(s)$ satisfies a tranformation law approximately of the form
        \[
            \w_{f}(x) \approx \w_{f}\left(\frac{1}{q(f)^{2}x}\right),
        \]
        for some constant $q(f)$ (that will be the conductor of the $L$-function). Then $L(s,f)$ admits meromorphic continuation to $\C$. To acomplish this, first decompose the Mellin transform into two pieces:
        \[
            \int_{0}^{\infty}\w_{f}(x)x^{s}\,\frac{dx}{x} = \int_{0}^{\frac{1}{q(f)}}\w_{f}(x)x^{s}\,\frac{dx}{x}+\int_{\frac{1}{q(f)}}^{\infty}\w_{f}(x)x^{s}\,\frac{dx}{x}.
        \]
        Then apply the transformation law for $\w_{f}(s)$ to the first piece and symmetrize the result to obtain an integral representation of the following form:
        \[
            L(s,f) \approx \text{polar factor}+\int_{\frac{1}{q(f)}}^{\infty}\w_{f}(x)x^{1-s}\,\frac{dx}{x}+\int_{\frac{1}{q(f)}}^{\infty}\w_{f}(x)x^{s}\,\frac{dx}{x},
        \]
        where the polar factor will appear if there is a constant term in $\w_{f}(x)$. Both of the integrals will be locally absolutely uniformly bounded by the exponential decay of $\w_{f}(x)$. Since the integral representation is symmetric under $s \to 1-s$, this gives the meromorphic continuation to $\C$.
        \end{method}