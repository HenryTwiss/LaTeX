\chapter{\texorpdfstring{$L$}{L}-functions of Arithmetic Functions}
  We discuss the $L$-functions attached to arithmetic functions. Namely, we develop the theory of the Riemann zeta function and Dirichlet $L$-functions.
  \section{The Riemann Zeta Function}
    \subsection*{The Definition and Euler Product}
      The \textbf{Riemann zeta function}\index{Riemann zeta function} $\z(s)$ is defined by the following Dirichlet series:
      \[
        \z(s) = \sum_{n \ge 1}\frac{1}{n^{s}}.
      \]
      This is the prototypical example of a Dirichlet series as all the coefficients are $1$. We will see that $\z(s)$ is a Selberg class $L$-function. As the coefficients are uniformly bounded and completely multiplicative, $\z(s)$ is locally absolutely uniformly convergent for $\s > 1$ and admits the following degree $1$ Euler product by \cref{prop:Dirichlet_series_Euler_product}:
      \[
        \z(s) = \prod_{p}(1-p^{-s})^{-1}.
      \]
      The local factor at $p$ is
      \[
        \z_{p}(s) = (1-p^{-s})^{-1},
      \]
      with local root $1$.
    \subsection*{The Integral Representation: Part I}
      We want to find an integral representation for $\z(s)$. To do this, consider the function
      \[
        \w(z) = \sum_{n \ge 1}e^{\pi in^{2}z},
      \]
      defined for $z \in \H$. It is locally absolutely uniformly convergent in this region by Weierstrass $M$-test. Moreover, we have
      \[
        \w(z) = O\left(\sum_{n \ge 1}e^{-\pi n^{2}y}\right) = O(e^{-\pi y}),
      \]
      where the second equality holds because each term is of smaller order than the next so that the series is bounded by a constant times the first term. It follows that $\w(z)$ exhibits exponential decay. Now consider the following Mellin transform:
      \[
        \int_{0}^{\infty}\w(iy)y^{\frac{s}{2}}\,\frac{dy}{y}.
      \]
      By the exponential decay of $\w(z)$, this integral is locally absolutely uniformly convergent for $\s > 1$ and hence defines an analytic function there. Then we compute
      \begin{align*}
        \int_{0}^{\infty}\w(iy)y^{\frac{s}{2}}\,\frac{dy}{y} &= \int_{0}^{\infty}\sum_{n \ge 1}e^{-\pi n^{2}y}y^{\frac{s}{2}}\,\frac{dy}{y} \\
        &= \sum_{n \ge 1}\int_{0}^{\infty}e^{-\pi n^{2}y}y^{\frac{s}{2}}\,\frac{dy}{y} && \text{FTT} \\
        &= \sum_{n \ge 1}\frac{1}{\pi^{\frac{s}{2}}n^{s}}\int_{0}^{\infty}e^{-y}y^{\frac{s}{2}}\,\frac{dy}{y} && \text{$y \mapsto \frac{y}{\pi n^{2}}$} \\
        &= \frac{\G\left(\frac{s}{2}\right)}{\pi^{\frac{s}{2}}}\sum_{n \ge 1}\frac{1}{n^{s}} \\
        &= \frac{\G\left(\frac{s}{2}\right)}{\pi^{\frac{s}{2}}}\z(s).
      \end{align*}
      Therefore we have an integral representation
      \begin{equation}\label{equ:integral_representation_zeta_1}
        \z(s) = \frac{\pi^{\frac{s}{2}}}{\G\left(\frac{s}{2}\right)}\int_{0}^{\infty}\w(iy)y^{\frac{s}{2}}\,\frac{dy}{y}.
      \end{equation}
      Unfortunately, we cannot proceed until we obtain a functional equation for $\w(z)$. So we will make a detour and come back to the integral representation after.
    \subsection*{The Jacobi Theta Function}
      The \textbf{Jacobi theta function}\index{Jacobi theta function} $\vt(z)$ is defined by
      \[
        \vt(z) = \sum_{n \in \Z}e^{2\pi in^{2}z},
      \]
      for $z \in \H$. It is locally absolutely uniformly convergent in this region by the Weierstrass $M$-test. Moreover,
      \[
        \vt(z)-1 = O\left(\sum_{n \in \Z-\{0\}}e^{-2\pi n^{2}y}\right) = O(e^{-2\pi y}),
      \]
      where the second equality holds because each term is of smaller order than the $n = \pm 1$ terms so that the series is bounded by a constant times the order of these terms. In particular, $\vt(z)-1$ exhibits exponential decay. The relationship to $\w(z)$ is given by
      \[
        \w(z) = \frac{\vt\left(\frac{z}{2}\right)-1}{2}.
      \]
      The essential fact we will need is a functional equation for the Jacobi theta function:

      \begin{theorem}\label{thm:functional_equation_Jacobi_theta}
        For $z \in \H$,
        \[
          \vt(z) = \frac{1}{\sqrt{-2iz}}\vt\left(-\frac{1}{4z}\right).
        \]
      \end{theorem}
      \begin{proof}
        We will apply the Poisson summation formula to
        \[
          \vt(z) = \sum_{n \in \Z}e^{2\pi in^{2}z}.
        \]
        To do this, we compute the Fourier transform of the summand and by the identity theorem it suffices to verify this for $z = iy$ with $y > 0$. So set
        \[
          f(x) = e^{-2\pi x^{2}y}.
        \]
        Then $f(x)$ is of Schwarz class. By \cref{prop:Fourier_transform_of_exponential_single_variable}, we have
        \[
          (\mc{F}f)(t) = \frac{e^{-\frac{\pi t^{2}}{2y}}}{\sqrt{2y}}.
        \]
        By the Poisson summation formula and the identity theorem, we have
        \[
          \vt(z) = \sum_{n \in \Z}e^{2\pi in^{2}z} = \sum_{t \in \Z}\frac{e^{-\frac{\pi t^{2}}{-2iz}}}{\sqrt{-2iz}} = \frac{1}{\sqrt{-2iz}}\sum_{t \in \Z}e^{-\frac{\pi t^{2}}{-2iz}} = \frac{1}{\sqrt{-2iz}}\sum_{t \in \Z}e^{2\pi it^{2}\left(-\frac{1}{4z}\right)} = \frac{1}{\sqrt{-2iz}}\vt\left(-\frac{1}{4z}\right).
        \]
      \end{proof}

      We will use \cref{thm:functional_equation_Jacobi_theta} to analytically continue $\z(s)$.
    \subsection*{The Integral Representation: Part II}
      Returning to the Riemann zeta function, we split the integral in \cref{equ:integral_representation_zeta_1} into two pieces by writing
      \begin{equation}\label{equ:symmetric_integral_zeta_split}
        \int_{0}^{\infty}\w(iy)y^{\frac{s}{2}}\,\frac{dy}{y} = \int_{0}^{1}\w(iy)y^{\frac{s}{2}}\,\frac{dy}{y}+\int_{1}^{\infty}\w(iy)y^{\frac{s}{2}}\,\frac{dy}{y}.
      \end{equation}
      The idea now is to rewrite the first piece in the same form and symmetrize the result as much as possible. We being by performing a change of variables $y \mapsto \frac{1}{y}$ to the first piece to obtain
      \[
        \int_{1}^{\infty}\w\left(\frac{i}{y}\right)y^{-\frac{s}{2}}\,\frac{dy}{y}
      \]
      Now we compute
      \begin{align*}
        \w\left(\frac{i}{y}\right) &= \w\left(-\frac{1}{iy}\right) \\
        &= \frac{\vt\left(-\frac{1}{2iy}\right)-1}{2} \\
        &= \frac{\sqrt{y}\vt\left(\frac{iy}{2}\right)-1}{2} && \text{\cref{thm:functional_equation_Jacobi_theta}} \\
        &= \sqrt{y}\w(iy)+\frac{\sqrt{y}}{2}-\frac{1}{2}.
      \end{align*}
      This chain implies that our first piece can be expressed as
      \[
        \int_{1}^{\infty}\left(\sqrt{y}\w(iy)+\frac{\sqrt{y}}{2}-\frac{1}{2}\right)y^{-\frac{s}{2}}\,\frac{dy}{y},
      \]
      which is further equivalent to
      \[
        \int_{1}^{\infty}\w(iy)y^{\frac{1-s}{2}}\,\frac{dy}{y}-\frac{1}{s(1-s)},
      \]
      because the integral over the last two pieces is $\frac{1}{1-s}-\frac{1}{s} = -\frac{1}{s(1-s)}$. Substituting this result back into \cref{equ:symmetric_integral_zeta_split} and combining with \cref{equ:integral_representation_zeta_1} yields the integral representation
      \begin{equation}\label{equ:integral_representation_zeta_final}
        \z(s) = \frac{\pi^{\frac{s}{2}}}{\G\left(\frac{s}{2}\right)}\left[-\frac{1}{s(1-s)}+\int_{1}^{\infty}\w(iy)y^{\frac{1-s}{2}}\,\frac{dy}{y}+\int_{1}^{\infty}\w(iy)y^{\frac{s}{2}}\,\frac{dy}{y}\right].
      \end{equation}
      This integral representation will give analytic continuation. To see this, first observe that everything outside the brackets is entire. Moreover, the integrands exhibit exponential decay and therefore the integrals are locally absolutely uniformly convergent on $\C$. The fractional term is holomorphic except for simple poles at $s = 0$ and $s = 1$. The meromorphic continuation to $\C$ follows with possible simple poles at $s = 0$ and $s = 1$. There is no pole at $s = 0$. Indeed, $\G\left(\frac{s}{2}\right)$ has a simple pole at $s = 0$ and so its reciprocal has a simple zero. This cancels the corresponding simple pole of $-\frac{1}{s(1-s)}$ so that $\z(s)$ has a removable singularity and thus is holomorphic at $s = 0$. At $s = 1$, $\G\left(\frac{s}{2}\right)$ is nonzero and so $\z(s)$ has a simple pole. Therefore $\z(s)$ has meromorphic continuation to all of $\C$ with a simple pole at $s = 1$.
    \subsection*{The Functional Equation}
      An immediate consequence of applying the symmetry $s \mapsto 1-s$ to \cref{equ:integral_representation_zeta_final} is the following functional equation:
      \[
        \frac{\G\left(\frac{s}{2}\right)}{\pi^{\frac{s}{2}}}\z(s) = \frac{\G\left(\frac{1-s}{2}\right)}{\pi^{\frac{1-s}{2}}}\z(1-s).
      \]
      We identify the gamma factor as
      \[
        \g(s,\z) = \pi^{-\frac{s}{2}}\G\left(\frac{s}{2}\right),
      \]
      with $\k = 0$ the only local root at infinity. Clearly it satisfies the required bounds. The conductor is $q(\z) = 1$ so no primes ramify. The completed Riemann zeta function is
      \[
        \L(s,\z) = \pi^{-\frac{s}{2}}\G\left(\frac{s}{2}\right)\z(s),
      \]
      with functional equation
      \[
        \L(s,\z) = \L(1-s,\z).
      \]
      This is the functional equation of $\z(s)$ and in this case is just a reformulation of the previous functional equation. From it we find that the root number is $\e(\z) = 1$ and that $\z(s)$ is self-dual. We can now show that the order of $\z(s)$ is $1$. Since there is only a simple pole at $s = 1$, multiply by $(s-1)$ to clear the polar divisor. As the integrals in \cref{equ:integral_representation_zeta_final} are locally absolutely uniformly convergent, computing the order amounts to estimating the gamma factor. Since the reciprocal of the gamma function is of order $1$, we have
      \[
        \frac{1}{\g(s,\z)} \ll_{\e} e^{|s|^{1+\e}}.
      \]
      Thus the reciprocal of the gamma factor is also of order $1$. It follows that
      \[
        (s-1)\z(s) \ll_{\e} e^{|s|^{1+\e}}.
      \]
      This shows $(s-1)\z(s)$ is of order $1$ and thus $\z(s)$ is as well after removing the polar divisor. We now compute the residue of $\z(s)$ at $s = 1$. First observe that the only term in \cref{equ:integral_representation_zeta_final} contributing to the pole is $-\frac{\pi^{\frac{s}{2}}}{\G\left(\frac{s}{2}\right)}\frac{1}{s(1-s)}$. Then
      \[
        \Res_{s = 1}\z(s) = \Res_{s = 1}\left(-\frac{\pi^{\frac{s}{2}}}{\G\left(\frac{s}{2}\right)}\frac{1}{s(1-s)}\right) = \lim_{s \to 1}\left(\frac{\pi^{\frac{s}{2}}}{\G\left(\frac{s}{2}\right)}\frac{1}{s}\right) = 1,
      \]
      where the second equality follows because $\G\left(\frac{1}{2}\right) = \sqrt{\pi}$. We summarize all of our work into the following theorem:

      \begin{theorem}\label{thm:zeta_Selberg}
        $\z(s)$ is a Selberg class $L$-function with degree $1$ Euler product given by
        \[
          \z(s) = \prod_{p}(1-p^{-s})^{-1}.
        \]
        Moreover, it admits meromorphic continuation to $\C$, possesses the functional equation
        \[
          \pi^{-\frac{s}{2}}\G\left(\frac{s}{2}\right)\z(s) = \L(s,\z) = \L(1-s,\z),
        \]
        and has a simple pole at $s = 1$ of residue $1$.
      \end{theorem}

      Lastly, we note that by virtue of the functional equation we can also compute $\z(0)$. Indeed, since $\Res_{s = 1}\z(s) = 1$, we have
      \[
        \lim_{s \to 1}(s-1)\L(s,\z) = \left(\Res_{s = 1}\z(s)\right)\left(\lim_{s \to 1}\pi^{-\frac{s}{2}}\G\left(\frac{s}{2}\right)\right) = 1.
      \]
      In other words, $\L(s,\z)$ has a simple pole at $s = 1$ of residue $1$ too. It follows that $\L(s,\z)$ also has a simple pole at $s = 0$ of residue $1$. Hence
      \[
        1 = \lim_{s \to 1}(s-1)\L(1-s,\z) = \left(\Res_{s = 1}\G\left(\frac{1-s}{2}\right)\right)\left(\lim_{s \to 1}\pi^{-\frac{1-s}{2}}\z(1-s)\right) = -2\z(0),
      \]
      because $\Res_{s = 0}\G(s) = 1$. Therefore $\z(0) = -\frac{1}{2}$.
  \section{Dirichlet \texorpdfstring{$L$}{L}-functions}
    \subsection*{The Definition and Euler Product}
      To every Dirichlet character $\chi$ there is an associated $L$-function. Throughout we will let $m$ denote the modulus and $q$ the conductor of $\chi$ respectively. The \textbf{Dirichlet $L$-series}\index{Dirichlet $L$-series} (respectively \textbf{Dirichlet $L$-function}\index{Dirichlet $L$-function} if it is an $L$-function) $L(s,\chi)$ attached to the Dirichlet character $\chi$ is defined by the following Dirichlet series:
      \[
        L(s,\chi) = \sum_{n \ge 1}\frac{\chi(n)}{n^{s}}.
      \]
      Since $\chi(n) = 0$ if $(n,m) > 1$, the above sum can be restricted to all positive integers relatively prime to $m$. We will see that $L(s,\chi)$ is a Selberg class $L$-function if $\chi$ is primitive and of conductor $q > 1$ (in the case $q = 1$, $L(s,\chi) = \z(s)$). From now we make this assumption about $\chi$. As the coefficients are uniformly bounded and completely multiplicative, $L(s,\chi)$ is locally absolutely uniformly convergent for $\s > 1$ and admits the following degree $1$ Euler product by \cref{prop:Dirichlet_series_Euler_product}:
      \[
        L(s,\chi) = \prod_{p}(1-\chi(p)p^{-s})^{-1} = \prod_{p \nmid q}(1-\chi(p)p^{-s})^{-1},
      \]
      where the second equality holds because if $p \mid q$ we have $\chi(p) = 0$. The local factor at $p$ is
      \[
        L_{p}(s,\chi) = 1 \quad \text{or} \quad L_{p}(s,\chi) = (1-\chi(p)p^{-s})^{-1},
      \]
      with local root $0$ or $\chi(p)$ respectively and according to if $p \mid q$ or not.
    \subsection*{The Integral Representation: Part I}
      The integral representation for $L(s,\chi)$ is deduced in a similar way as for $\z(s)$. However, it will depend on if $\chi$ is even or odd. To handle both cases simultaneously let $\d_{\chi} = 0,1$ according to whether $\chi$ is even or odd. In other words,
      \[
        \d_{\chi} = \frac{\chi(1)-\chi(-1)}{2}.
      \]
      We also have $\chi(-1) = (-1)^{\d_{\chi}}$. Note that $\d_{\chi}$ takes the same value for both $\chi$ and $\cchi$. To find an integral representation for $L(s,\chi)$, consider the function
      \[
        \w_{\chi}(z) = \sum_{n \ge 1}\chi(n)n^{\d_{\chi}}e^{\pi in^{2}z},
      \]
      defined for $z \in \H$. It is locally absolutely uniformly convergent in this region by the Weierstrass $M$-test. Moreover, we have
      \[
        \w_{\chi}(z) = O\left(\sum_{n \ge 1}ne^{-\pi n^{2}y}\right) = O(e^{-\pi y}),
      \]
      where the second equality holds because each term is of smaller order than the next so that the series is bounded by a constant times the first term. Hence $\w_{\chi}(z)$ exhibits exponential decay. Now consider the following Mellin transform:
      \[
        \int_{0}^{\infty}\w_{\chi}(iy)y^{\frac{s+\d_{\chi}}{2}}\,\frac{dy}{y}.
      \]
      By the exponential decay of $w_{\chi}$, this integral is locally absolutely uniformly convergent for $\s > 1$ and hence defines an analytic function there. Then we compute
      \begin{align*}
        \int_{0}^{\infty}\w_{\chi}(iy)y^{\frac{s+\d_{\chi}}{2}}\,\frac{dy}{y} &= \int_{0}^{\infty}\sum_{n \ge 1}\chi(n)n^{\d_{\chi}}e^{-\pi n^{2}y}y^{\frac{s+\d_{\chi}}{2}}\,\frac{dy}{y} \\
        &= \sum_{n \ge 1}\int_{0}^{\infty}\chi(n)n^{\d_{\chi}}e^{-\pi n^{2}y}y^{\frac{s+\d_{\chi}}{2}}\,\frac{dy}{y} && \text{FTT} \\
        &= \sum_{n \ge 1}\frac{\chi(n)}{\pi^{\frac{s+\d_{\chi}}{2}}n^{s}}\int_{0}^{\infty}e^{-y}y^{\frac{s+\d_{\chi}}{2}}\,\frac{dy}{y} && \text{$y \mapsto \frac{y}{\pi n^{2}}$} \\
        &= \frac{\G\left(\frac{s+\d_{\chi}}{2}\right)}{\pi^{\frac{s+\d_{\chi}}{2}}}\sum_{n \ge 1}\frac{\chi(n)}{n^{s}} \\
        &= \frac{\G\left(\frac{s+\d_{\chi}}{2}\right)}{\pi^{\frac{s+\d_{\chi}}{2}}}L(s,\chi).
      \end{align*}
      Therefore we have an integral representation
      \begin{equation}\label{equ:integral_representation_Dirichlet_L-functions_1}
        L(s,\chi) = \frac{\pi^{\frac{s+\d_{\chi}}{2}}}{\G\left(\frac{s+\d_{\chi}}{2}\right)}\int_{0}^{\infty}\w_{\chi}(iy)y^{\frac{s+\d_{\chi}}{2}}\,\frac{dy}{y},
      \end{equation}
      and just as for the Riemann zeta function, we need to find a functional equation for $\w_{\chi}(z)$ before we can proceed.
    \subsection*{The Dirichlet Theta Function}
      The \textbf{Dirichlet theta function}\index{Dirichlet theta function} $\vt_{\chi}(z)$ attached to the character $\chi$, is defined by
      \[
        \vt_{\chi}(z) = \sum_{n \in \Z}\chi(n)n^{\d_{\chi}}e^{2\pi in^{2}z},
      \]
      for $z \in \H$. It is locally absolutely uniformly convergent in this region by the Weierstrass $M$-test. Moreover,
      \[
        \vt_{\chi}(z) = O\left(\sum_{n \in \Z}ne^{-2\pi n^{2}y}\right) = O(e^{-2\pi y}),
      \]
      where the second equality holds because each term is of smaller order than the $n = \pm 1$ terms so that the series is bounded by a constant times the order of these terms. In particular, $\vt_{\chi}(z)$ exhibits exponential decay. The relationship to $\w_{\chi}(z)$ is given by
      \[
        \w_{\chi}(z) = \frac{\vt_{\chi}\left(\frac{z}{2}\right)}{2}.
      \]
      This is a slightly less complex relationship than the analog for the Jacobi theta function because assuming $q > 1$ means $\chi(0) = 0$. The essential fact we will need is a functional equation for the Dirichlet theta function:

      \begin{theorem}\label{thm:functional_equation_Dirichlet_theta}
        Let $\chi$ be a primitive Dirichlet character of conductor $q > 1$. For $z \in \H$,
        \[
          \vt_{\chi}(z) = \frac{\e_{\chi}}{i^{\d_{\chi}}(-2qiz)^{\frac{1}{2}+\d_{\chi}}}\vt_{\cchi}\left(-\frac{1}{4q^{2}z}\right).
        \]
      \end{theorem}
      \begin{proof}
        Since $\chi$ is $q$-periodic and $\vt_{\chi}(z)$ is absolutely convergent, we can write
        \[
          \vt_{\chi}(z) = \sum_{a \tmod{q}}\chi(a)\sum_{m \in \Z}(mq+a)^{\d_{\chi}}e^{2\pi i(mq+a)^{2}z}.
        \]
        Set
        \[
          I_{a}(z) = \sum_{m \in \Z}(mq+a)^{\d_{\chi}}e^{2\pi i(mq+a)^{2}z}.
        \]
        We will apply the Poisson summation formula to $I_{a}(z)$. To do this, we compute the Fourier transform of the summand and by the identity theorem it suffices to verify this for $z = iy$ with $y > 0$. So set
        \[
          f(x) = (xq+a)^{\d_{\chi}}e^{-2\pi(xq+a)^{2}y}.
        \]
        Then $f(x)$ is of Schwarz class. By \cref{prop:Fourier_transform_properties} (i)-(iv) and \cref{prop:Fourier_transform_of_exponential_single_variable}, we have
        \[
          (\mc{F}f)(t) = \left(\frac{-it}{2qy}\right)^{\d_{\chi}}\frac{e^{\frac{2\pi iat}{q}-\frac{\pi t^{2}}{2q^{2}y}}}{\sqrt{2q^{2}y}}.
        \]
        By the Poisson summation formula and the identity theorem, we have
        \[
          I_{a}(z) = \sum_{m \in \Z}(mq+a)^{\d_{\chi}}e^{2\pi i(mq+a)^{2}z} = \sum_{t \in \Z}\left(\frac{-it}{2q(-iz)}\right)^{\d_{\chi}}\frac{e^{\frac{2\pi iat}{q}-\frac{\pi t^{2}}{2q^{2}(-iz)}}}{\sqrt{2q^{2}(-iz)}}.
        \]
        Plugging this back into the definition of $\vt_{\chi}(z)$ yields
        \begin{align*}
          \vt_{\chi}(z) &= \sum_{a \tmod{q}}\chi(a)\sum_{t \in \Z}\left(\frac{-it}{2q(-iz)}\right)^{\d_{\chi}}\frac{e^{\frac{2\pi iat}{q}-\frac{\pi t^{2}}{2q^{2}(-iz)}}}{\sqrt{2q^{2}(-iz)}} \\
          &= \sum_{a \tmod{q}}\chi(a)\sum_{t \in \Z}\left(\frac{t}{i(-2qiz)}\right)^{\d_{\chi}}\frac{e^{\frac{2\pi iat}{q}-\frac{\pi it^{2}}{2q^{2}z}}}{\sqrt{q(-2qiz)}} \\
          &= \frac{1}{i^{\d_{\chi}}\sqrt{q}(-2qiz)^{\frac{1}{2}+\d_{\chi}}}\sum_{t \in \Z}t^{\d_{\chi}}e^{-\frac{\pi it^{2}}{2q^{2}z}}\sum_{a \tmod{q}}\chi(a)e^{\frac{2\pi i at}{q}} \\
          &= \frac{1}{i^{\d_{\chi}}\sqrt{q}(-2qiz)^{\frac{1}{2}+\d_{\chi}}}\sum_{t \in \Z}t^{\d_{\chi}}e^{-\frac{\pi it^{2}}{2q^{2}z}}\tau(t,\chi) && \text{definition of $\tau(t,\chi)$} \\
          &= \frac{\tau(\chi)}{i^{\d_{\chi}}\sqrt{q}(-2qiz)^{\frac{1}{2}+\d_{\chi}}}\sum_{t \in \Z}\cchi(t)t^{\d_{\chi}}e^{-\frac{\pi it^{2}}{2q^{2}z}} && \text{\cref{cor:gauss_sum_primitive_formula}} \\
          &= \frac{\e_{\chi}}{i^{\d_{\chi}}(-2qiz)^{\frac{1}{2}+\d_{\chi}}}\sum_{t \in \Z}\cchi(t)t^{\d_{\chi}}e^{-\frac{\pi it^{2}}{2q^{2}z}} && \text{$\e_{\chi} = \frac{\tau(\chi)}{\sqrt{q}}$} \\
          &= \frac{\e_{\chi}}{i^{\d_{\chi}}(-2qiz)^{\frac{1}{2}+\d_{\chi}}}\sum_{t \in \Z}\cchi(t)t^{\d_{\chi}}e^{2\pi it^{2}\left(-\frac{1}{4q^{2}z}\right)} \\
          &=  \frac{\e_{\chi}}{i^{\d_{\chi}}(-2qiz)^{\frac{1}{2}+\d_{\chi}}}\vt_{\cchi}\left(-\frac{1}{4q^{2}z}\right).
        \end{align*}
      \end{proof}

      Notice that the functional equation relates $\vt_{\chi}(z)$ to $\vt_{\cchi}(z)$. Regardless, we will use \cref{thm:functional_equation_Dirichlet_theta} to analytically continue $L(s,\chi)$.
    \subsection*{The Integral Representation: Part II}
      Returning to $L(s,\chi)$, split the integral in \cref{equ:integral_representation_Dirichlet_L-functions_1} into two pieces by writing
      \begin{equation}\label{equ:symmetric_integral_Dirichlet_L-functions_split}
        \int_{0}^{\infty}\w_{\chi}(iy)y^{\frac{s+\d_{\chi}}{2}}\,\frac{dy}{y} = \int_{0}^{\frac{1}{q}}\w_{\chi}(iy)y^{\frac{s+\d_{\chi}}{2}}\,\frac{dy}{y}+\int_{\frac{1}{q}}^{\infty}\w_{\chi}(iy)y^{\frac{s+\d_{\chi}}{2}}\,\frac{dy}{y}.
      \end{equation}
      We now rewrite the first piece in the same form and symmetrize the result as much as possible. Start by performing a change of variables $y \mapsto \frac{1}{q^{2}y}$ to the first piece to obtain
      \[
        \int_{\frac{1}{q}}^{\infty}\w_{\chi}\left(\frac{i}{q^{2}y}\right)(q^{2}y)^{-\frac{s+\d_{\chi}}{2}}\,\frac{dy}{y}.
      \]
      Now we compute
      \begin{align*}
        \w_{\chi}\left(\frac{i}{q^{2}y}\right) &= \w_{\chi}\left(-\frac{1}{q^{2}iy}\right) \\
        &= \frac{\vt_{\chi}\left(-\frac{1}{2q^{2}iy}\right)}{2} \\
        &= \frac{i^{\d_{\chi}}(qy)^{\frac{1}{2}+\d_{\chi}}}{\e_{\cchi}}\frac{\vt_{\cchi}\left(\frac{iy}{2}\right)}{2} && \text{\cref{thm:functional_equation_Dirichlet_theta}} \\
        &= \e_{\chi}(-i)^{\d_{\chi}}(qy)^{\frac{1}{2}+\d_{\chi}}\frac{\vt_{\cchi}\left(\frac{iy}{2}\right)}{2} && \text{\cref{prop:epsilon_factor_relationship} and that $\chi(-1) = (-1)^{\d_{\chi}}$} \\
        &= \frac{\e_{\chi}(qy)^{\frac{1}{2}+\d_{\chi}}}{i^{\d_{\chi}}}\frac{\vt_{\cchi}\left(\frac{iy}{2}\right)}{2} \\
        &= \frac{\e_{\chi}(qy)^{\frac{1}{2}+\d_{\chi}}}{i^{\d_{\chi}}}\w_{\cchi}(iy).
      \end{align*}
      This chain implies that our first piece can be expressed as
      \[
        \frac{\e_{\chi}}{i^{\d_{\chi}}}q^{\frac{1}{2}-s}\int_{\frac{1}{q}}^{\infty}\w_{\cchi}(iy)y^{\frac{(1-s)+\d_{\chi}}{2}}\,\frac{dy}{y}.
      \]
      Substituting this expression back into \cref{equ:symmetric_integral_Dirichlet_L-functions_split} and combining with \cref{equ:integral_representation_Dirichlet_L-functions_1} gives the integral representation
      \begin{equation}\label{equ:integral_representation_Dirichlet_final}
        L(s,\chi) = \frac{\pi^{\frac{s+\d_{\chi}}{2}}}{\G\left(\frac{s+\d_{\chi}}{2}\right)}\left[\frac{\e_{\chi}}{i^{\d_{\chi}}}q^{\frac{1}{2}-s}\int_{\frac{1}{q}}^{\infty}\w_{\cchi}(iy)y^{\frac{(1-s)+\d_{\chi}}{2}}\,\frac{dy}{y}+\int_{\frac{1}{q}}^{\infty}\w_{\chi}(iy)y^{\frac{s+\d_{\chi}}{2}}\,\frac{dy}{y}\right].
      \end{equation}
      This integral representation will give analytic continuation. Indeed, we know everything outside the brackets is entire. The integrands exhibit exponential decay and therefore the integrals are locally absolutely uniformly convergent on $\C$. This gives analytic continuation to all of $\C$. In particular, $L(s,\chi)$ has no poles.
    \subsection*{The Functional Equation}
      An immediate consequence of applying the symmetry $s \mapsto 1-s$ to \cref{equ:integral_representation_Dirichlet_final} is the following functional equation:
      \[
        q^{\frac{s}{2}}\frac{\G\left(\frac{s+\d_{\chi}}{2}\right)}{\pi^{\frac{s+\d_{\chi}}{2}}}L(s,\chi) = \frac{\e_{\chi}}{i^{\d_{\chi}}}q^{\frac{1-s}{2}}\frac{\G\left(\frac{(1-s)+\d_{\chi}}{2}\right)}{\pi^{\frac{(1-s)+\d_{\chi}}{2}}}L(1-s,\cchi).
      \]
      We identify the gamma factor as
      \[
        \g(s,\chi) = \pi^{-\frac{s}{2}}\G\left(\frac{s+\d_{\chi}}{2}\right),
      \]
      with $\k = \d_{\chi}$ the only local root at infinity. Clearly it satisfies the required bounds. The conductor is $q(\chi) = q$ and if $p$ is unramified then the local root is $\chi(p) \neq 0$. The completed $L$-function is
      \[
        \L(s,\chi) = q^{\frac{s}{2}}\pi^{-\frac{s}{2}}\G\left(\frac{s+\d_{\chi}}{2}\right)L(s,\chi),
      \]
      with functional equation
      \[
        \L(s,\chi) = \frac{\e_{\chi}}{i^{\d_{\chi}}}\L(1-s,\cchi).
      \]
      From it we see that the root number is $\e(\chi) = \frac{\e_{\chi}}{i^{\d_{\chi}}}$ and that $L(s,\chi)$ has dual $L(s,\cchi)$. We now show that $L(s,\chi)$ is of order $1$. Since $L(s,\chi)$ has no poles, we do not need to clear any polar divisors. As the integrals in \cref{equ:integral_representation_Dirichlet_final} are locally absolutely uniformly convergent, computing the order amounts to estimating the gamma factor. Since the reciprocal of the gamma function is of order $1$, we have
      \[
        \frac{1}{\g(s,\chi)} \ll_{\e} e^{|s|^{1+\e}}.
      \]
      So the reciprocal of the gamma factor is also of order $1$. It follows that
      \[
        L(s,\chi) \ll_{\e} e^{|s|^{1+\e}}.
      \]
      So $L(s,\chi)$ is of order $1$. We summarize all of our work into the following theorem:

      \begin{theorem}\label{thm:primitive_Dirichlet_Selberg}
        For any primitive Dirichlet character $\chi$ of conductor $q > 1$, $L(s,\chi)$ is a Selberg class $L$-function with degree $1$ Euler product given by
        \[
          L(s,\chi) = \prod_{p}(1-\chi(p)p^{-s})^{-1}.
        \]
        Moreover, it admits analytic continuation to $\C$ and possesses the functional equation
        \[
          q^{\frac{s}{2}}\pi^{-\frac{s}{2}}\G\left(\frac{s+\d_{\chi}}{2}\right)L(s,\chi) = \L(s,\chi) = \frac{\e_{\chi}}{i^{\d_{\chi}}}\L(1-s,\cchi).
        \]
      \end{theorem}
    \subsection*{Beyond Primitivity}
      We can still obtain meromorphic continuation of the $L$-series $L(s,\chi)$ if $\chi$ is imprimitive. Indeed, if $\chi$ is induced by $\wtilde{\chi}$ then $\chi(p) = \wtilde{\chi}(p)$ if $p \nmid q$ and $\chi(p) = 0$ if $p \mid m$. Then just as for primitive characters, \cref{prop:Dirichlet_series_Euler_product} implies that $L(s,\chi)$ is locally absolutely uniformly convergent for $\s > 1$ and admits the following degree $1$ Euler product:
      \begin{equation}\label{equ:non-primitive_primitive_Dirichlet_L-series_relation}
        L(s,\chi) = \prod_{p \nmid m}(1-\wtilde{\chi}(p)p^{-s})^{-1} = \prod_{p}(1-\wtilde{\chi}(p)p^{-s})^{-1}\prod_{p \mid m}(1-\wtilde{\chi}(p)p^{-s}) = L(s,\wtilde{\chi})\prod_{p \mid m}(1-\wtilde{\chi}(p)p^{-s}).
      \end{equation}
      From this relation, we can prove the following:

      \begin{theorem}\label{thm:analytic_continuation_Dirichlet}
        For any Dirichlet character $\chi$ modulo $m$, let $\wtilde{\chi}$ be the primitive character inducing $\chi$. Then $L(s,\chi)$ is locally absolutely uniformly convergent for $\s > 1$ with degree $1$ Euler product given by
        \[
          L(s,\chi) = \prod_{p \nmid m}(1-\wtilde{\chi}(p)p^{-s})^{-1}.
        \]
        Moreover, it admits meromorphic continuation to $\C$ and if $\chi$ is principal there is a simple pole at $s = 1$ of residue $\prod_{p \mid m}(1-\wtilde{\chi}(p)p^{-1})$.
      \end{theorem}
      \begin{proof}
        This follows from \cref{thm:zeta_Selberg,thm:primitive_Dirichlet_Selberg,equ:non-primitive_primitive_Dirichlet_L-series_relation}. 
      \end{proof}

      It is worth noting that for any non-principal Dirichlet character $\chi$ modulo $m$, the $L$-series $L(s,\chi)$ converges for $\s > 0$. Indeed, setting $A(X) = \sum_{n \le X}\chi(n)$ we have $A(X) \le m$ by  Dirichlet orthogonality relations (namely (i)) and that $\chi$ is $m$-periodic, and then the claim follows from \cref{prop:Dirichlet_series_convergence_bounded_coefficient_sum}. While this fact is not too important from an analytic continuation standpoint, it is useful because it allows for the manipulation (without rearrangement since we do not have absolute convergence) of the $L$-series $L(s,\chi)$ in the vertical strip $0 < \s \le 1$.