\chapter{Types of Number Fields}
  We provide a more detailed discussion of some number fields with particularly simple structure. Namely, we discuss quadratic and cyclotomic number fields because their rings of integers are monogenic.
  \section{Quadratic Number Fields}
    We will now classify and discuss the structure of quadratic number fields. We first show that quadratic number fields are exactly those where we adjoint the square-root of a square-free integer other than $0$ or $1$:

    \begin{proposition}\label{prop:classification_of_quadratic_number_fields}
      Every quadratic number field $K$ is of the form $K = \Q(\sqrt{d})$ for some square-free integer $d$ other than $0$ or $1$.
    \end{proposition}
    \begin{proof}
      Suppose $K$ is a quadratic number field. In particular, $K/\Q$ is separable so by the primitive element theorem there exists $\t \in K$ such that $K = \Q(\t)$. The minimal polynomial $m_{\t}(x)$ of $\t$ is of the form
      \[
        m_{\t}(x) = x^{2}+ax+b,
      \]
      for $a,b \in \Q$. Then the quadratic formula gives
      \[
        \t = -\frac{a}{2}\pm\frac{\sqrt{q}}{2},
      \]
      where $q = a^{2}-4b \in \Q$. Clearly $q \neq 0$ and $q \neq 1$ for otherwise $\t \in \Q$. It follows that $K = \Q(\sqrt{q})$. Write $q = \frac{n}{m}$ for relatively prime $n,m \in \Z$ and set $d = m^{2}q = nm \in \Z$. Then $d$ is square-free, $d \neq 0$, and $d \neq 1$. Moreover, $\sqrt{d} = m\sqrt{q}$ so that $K = \Q(\sqrt{d})$. 
    \end{proof}

    From \cref{prop:classification_of_quadratic_number_fields}, we see that the $d$ for a quadratic number field $\Q(\sqrt{d})$ satisfies $d \equiv 1,2,3 \tmod{4}$ (otherwise $d$ is not square-free). Moreover, any element of a quadratic number field is of the form $a+b\sqrt{d}$ with $a,b \in \Q$ and for some square-free $d$ other than $0$ or $1$. We say that a quadratic number field $\Q(\sqrt{d})$ is \textbf{real}\index{real} if $d > 0$ and \textbf{imaginary}\index{imaginary} if $d < 0$. Now $\Q(\sqrt{d})$ is real or imaginary according to if $\sqrt{d}$ is real or purely imaginary so that the two $\Q$-embeddings $\s_{1}$ and $\s_{2}$ of $\Q(\sqrt{d})$ into $\C$ are
    \[
      \s_{1}(a+b\sqrt{d}) = a+b\sqrt{d} \quad \text{and} \quad \s_{2}(a+b\sqrt{d}) = a-b\sqrt{d},
    \]
    because the roots of the minimal polynomial for $\sqrt{d}$ are $\pm\sqrt{d}$. In particular, the signature is $(2,0)$ or $(0,1)$ according to if $\Q(\sqrt{d})$ is real or imaginary. In either case, \cref{prop:formulas_for_trace_and_norm} shows that the trace and norm of $\k = a+b\sqrt{d} \in \Q(\sqrt{d})$ are given by
    \[
      \Trace(\k) = 2a \quad \text{and} \quad \Norm(\k) = a^{2}-b^{2}d.
    \]
    We will now begin describing the ring of integers, discriminant, and the factorization of primes in $\Q(\sqrt{d})$. For simplicity, we will write $\mc{O}_{d} = \mc{O}_{\Q(\sqrt{d})}$ and $\D_{d} = \D_{\Q(\sqrt{d})}$. The ring of integers has a particularly simple description since it is monogenic as the following proposition shows:
    
    \begin{proposition}\label{prop:ring_of_integers_quadratic}
      Let $\Q(\sqrt{d})$ be a quadratic number field. Then $\Q(\sqrt{d})$ is monogenic and
      \[
        \mc{O}_{d} = \begin{cases} \Z\left[\frac{1+\sqrt{d}}{2}\right] & \text{if $d \equiv 1 \tmod{4}$}, \\ \Z[\sqrt{d}] & \text{if $d \equiv 2,3 \tmod{4}$}. \end{cases}
      \]
    \end{proposition}
    \begin{proof}
      Let $\a = a+b\sqrt{d} \in \Q(\sqrt{d})$ be an algebraic integer. If $b = 0$ then $\a \in \Q$ and since the only elements of $\Q$ that are algebraic integers are the integers themselves we must have that $\a$ is an integer. Now suppose $b \neq 0$. Then the minimal polynomial of $\a$ is
      \[
        m_{\a}(x) = x^{2}+2ax+(a^{2}-b^{2}d) = (x-(a+b\sqrt{d}))(x-(a-b\sqrt{d})).
      \]
      As $\a$ is an algebraic integer, $2a \in \Z$ and $a^{2}-b^{2}d \in \Z$ (note that these are the trace and norm of $\a$ respectively). In particular, $(2a)^{2}+(2b)^{2}d \in \Z$ and hence $(2b)^{2} \in \Z$ is as well. But as $b \in \Q$, it must be the case that $2b \in \Z$. If $2a = n+1$ is odd then $n$ is even. We compute
      \[
        a^{2}-b^{2}d = \left(\frac{n+1}{2}\right)^{2}-b^{2}d = \frac{n^{2}+2n+1+4b^{2}d}{4},
      \]
      and since the right-hand side must be an integer $b \notin \Z$. For if $b \in \Z$, the numerator of the right-hand side is equivalent to $1$ modulo $4$ because $n$ is even. As $2b \in \Z$ it follows that $2b$ must be odd so set $2b = m+1$ with $m$ even. Again, we compute
      \[
        a^{2}-b^{2}d = \left(\frac{n+1}{2}\right)^{2}-\left(\frac{m+1}{2}\right)^{2}d = \frac{n^{2}+2n+1-d(m^{2}+2m+1)}{4},
      \]
      and since the right-hand side must be an integer the numerator must be divisible by $4$. As $n$ and $m$ are even, this is equivalent to $d \equiv 1 \tmod{4}$. So we have shown $2a$ or $2b$ is odd if and only if $d \equiv 1 \tmod{4}$. Thus if $d \equiv 1 \tmod{4}$, we have $a = \frac{a'}{2}$ and $b = \frac{b'}{2}$ for some $a',b' \in \Z$ and hence $\a \in \Z\left[\frac{1+\sqrt{d}}{2}\right]$. Otherwise, $d \equiv 2,3 \tmod{4}$ (because $d$ is square-free) so that $2a$ and $2b$ are both even, $a,b \in \Z$, and therefore $\a \in \Z[\sqrt{d}]$. We have now shown that $\mc{O}_{d} \subseteq \Z\left[\frac{1+\sqrt{d}}{2}\right]$ and $\mc{O}_{d} \subseteq \Z[\sqrt{d}]$ according to if $d \equiv 1 \tmod{4}$ or $d \equiv 2,3 \tmod{4}$ respectively. For the reverse containment, it suffices to show that $\frac{1+\sqrt{d}}{2}$ and $\sqrt{d}$ are algebraic integers according to if $d \equiv 1 \tmod{4}$ or $d \equiv 2,3 \tmod{4}$ respectively since $\mc{O}_{K}$ is a ring. Indeed they are since their minimal polynomials are
      \[
        m_{\frac{1+\sqrt{d}}{2}}(x) = x^{2}-x+\frac{1-d}{4} \quad \text{and} \quad m_{\sqrt{d}}(x) = x^{2}-d,
      \]
      where $\frac{1-d}{4} \in \Z$ because $d \equiv 1 \tmod{4}$.
    \end{proof}

    It follows from \cref{prop:ring_of_integers_quadratic} that
    \[
      1,\frac{1+\sqrt{d}}{2} \quad \text{and} \quad 1,\sqrt{d},
    \]
    are integral bases for $\mc{O}_{d}$ according to if $d \equiv 1 \tmod{4}$ or $d \equiv 2,3 \tmod{4}$ respectively. Let us now show that the discriminants quadratic number fields are exactly the fundamental discriminants other than $1$:

    \begin{proposition}\label{prop:discriminant_quadratic}
      Let $\Q(\sqrt{d})$ be a quadratic number field. Then
      \[
        \D_{d} = \begin{cases} d & \text{if $d \equiv 1 \tmod{4}$}, \\ 4d & \text{if $d \equiv 2,3 \tmod{4}$}. \end{cases}
      \]
      In particular, the discriminants quadratic number fields are exactly the fundamental discriminants other than $1$.
    \end{proposition}
    \begin{proof}
      Let $\s_{1}$ and $\s_{2}$ be the two $\Q$-embeddings of $\Q(\sqrt{d})$ into $\C$ where $\s_{1}$ is the identity and $\s_{2}$ is given by sending $\sqrt{d}$ to its conjugate. If $d \equiv 1 \tmod{4}$, an integral basis for $\mc{O}_{d}$ is $1,\frac{1+\sqrt{d}}{2}$. In this case, the embedding matrix is
      \[
        M\left(1,\frac{1+\sqrt{d}}{2}\right) = \begin{pmatrix} 1 & \frac{1+\sqrt{d}}{2} \\ 1 & \frac{1-\sqrt{d}}{2} \end{pmatrix},
      \]
      and thus $\D_{d} = d$. If $d \equiv 2,3 \tmod{4}$, an integral basis for $\mc{O}_{d}$ is $1,\sqrt{d}$. In this case, the embedding matrix is
      \[
        M(1,\sqrt{d}) = \begin{pmatrix} 1 & \sqrt{d} \\ 1 & -\sqrt{d} \end{pmatrix},
      \]
      and hence $\D_{d} = 4d$. This proves the first statement and the second statement is clear since $d$ is square-free and not $0$ or $1$.
    \end{proof}

    From now on, we will write fundamental discriminants other than $1$ as $\D_{d}$ instead of $D$ to clarify the connection to quadratic number fields. We will now discuss the factorization of a prime $p$ in a quadratic number field $\Q(\sqrt{d})$. Since $\Q(\sqrt{d})$ is a degree $2$ extension, \cref{prop:inertia_ramification_relation} implies that $p$ is either inert, totally ramified, or totally split. In other words, there are three possible cases for how $p\mc{O}_{d}$ factors:
    \[
      p\mc{O}_{d} = \mf{p}, \quad p\mc{O}_{d} = \mf{p}^{2}, \quad \text{and} \quad p\mc{O}_{d} = \mf{p}\mf{q},
    \]
    according to if $p$ is inert, totally ramified, or totally split. Since $\Q(\sqrt{d})$ is monogenic by \cref{prop:ring_of_integers_quadratic}, we can describe the factorization using the Dedekind-Kummer theorem and connect it to the quadratic character $\chi_{\D_{d}}$ associated to the fundamental discriminant $\D_{d}$:

    \begin{proposition}\label{prop:factorization_of_primes_quadratic}
      Let $\Q(\sqrt{d})$ be a quadratic number field and let $\chi_{\D_{d}}$ be the quadratic character given by the fundamental discriminant $\D_{d}$. Then for any prime $p$, we have
      \[
        \chi_{\D_{d}}(p) = \begin{cases} 1 & \text{if $p$ is totally split}, \\ -1 & \text{if $p$ is inert}, \\ 0 & \text{if $p$ is totally ramified}. \end{cases}
      \]
    \end{proposition}
    \begin{proof}
      Recall that $p$ is ramified if and only if it divides $|\D_{d}|$ and note that this is exactly when $\chi_{\D_{d}}(p) = 0$. Therefore it suffices to prove the cases when $p$ is totally split or inert. First suppose $d \equiv 1 \tmod{4}$ so that $\mc{O}_{d} = \Z\left[\frac{1+\sqrt{d}}{2}\right]$ and $\D_{d} = d$ by \cref{prop:ring_of_integers_quadratic,prop:discriminant_quadratic}. The minimal polynomial $m_{\frac{1+\sqrt{d}}{2}}(x)$ for $\frac{1+\sqrt{d}}{2}$ is
      \[
        m_{\frac{1+\sqrt{d}}{2}}(x) = x^{2}-x+\frac{1-d}{4},
      \]
      where $\frac{1-d}{4} \in \Z$ because $d \equiv 1 \tmod{4}$. The reduction of $m_{\frac{1+\sqrt{d}}{2}}(x)$ modulo $p$ is either irreducible, factors into two distinct linear factors, or is a square, and Dedekind-Kummer theorem implies that this is equivalent to $p$ being inert, totally split, or totally ramified accordingly because the prime factorization is unique. First suppose $p \neq 2$. Then from the quadratic formula, $m_{\frac{1+\sqrt{d}}{2}}(x)$ reduces modulo $p$ as
      \[
        m_{\frac{1+\sqrt{d}}{2}}(x) \equiv \left(x-\frac{1+\sqrt{d}}{2}\right)\left(x-\frac{1-\sqrt{d}}{2}\right) \pmod{p},
      \]
      if and only if the roots $\frac{1\pm\sqrt{d}}{2}$ are elements of $\F_{p}$ and is otherwise irreducible. As $p \neq 2$, these factors are distinct. Moreover, $\frac{1\pm\sqrt{d}}{2}$ is an element of $\F_{p}$ if and only if $d$ is a square modulo $p$ and hence $p$ is totally split or inert according to if $\chi_{d}(p) = \pm1$. Now suppose $p = 2$. Since $m_{\frac{1+\sqrt{d}}{2}}(x)$ has a nonzero linear term with an odd coefficient, it reduces modulo $2$ as
      \[
        m_{\frac{1+\sqrt{d}}{2}}(x) \equiv x(x-1) \tmod{2},
      \]
      if and only if $\frac{1-d}{4} \equiv 0 \tmod{2}$ and is otherwise irreducible. Clearly these factors are distinct. Now observe $\frac{1-d}{4} \equiv 0 \tmod{2}$ is equivalent to $d \equiv 1 \tmod{8}$ provided $d > 0$ and $d \equiv 7 \tmod{8}$ provided $d < 0$ and thus $p$ is totally split or inert according to if $\chi_{d}(2) = \pm1$. This completes the argument in the case $d \equiv 1 \tmod{4}$. Now suppose $d \equiv 2,3 \tmod{4}$ so that $\mc{O}_{d} = \Z[\sqrt{d}]$ and $\D_{d} = 4d$ by \cref{prop:ring_of_integers_quadratic,prop:discriminant_quadratic}. The minimal polynomial $m_{\sqrt{d}}(x)$ for $\sqrt{d}$ is
      \[
        m_{\sqrt{d}}(x) = x^{2}-d.
      \]
      As $\D_{d} = 4d$, we see that $2$ is ramified and therefore we may assume $p \neq 2$. Similarly, the reduction of $m_{\sqrt{d}}(x)$ modulo $p$ is either irreducible, factors into two distinct linear factors, or is a square, and Dedekind-Kummer theorem implies that this is equivalent to $p$ being inert, totally split, or totally ramified accordingly because the prime factorization is unique. As $p \neq 2$, the quadratic formula implies that $m_{\sqrt{d}}(x)$ reduces modulo $p$ as
      \[
        m_{\sqrt{d}}(x) \equiv (x-\sqrt{d})(x+\sqrt{d}) \pmod{p},
      \]
      if and only if the roots $\pm\sqrt{d}$ are elements of $\F_{p}$. As $p \neq 2$, these factors are distinct. Moreover, $\sqrt{d}$ is an element of $\F_{p}$ if and only if $d$ and hence $4d$ are squares modulo $p$ so that $p$ is totally split or inert according to if $\chi_{4d}(p) = \pm1$. This completes the verification in the case $d \equiv 2,3 \tmod{4}$.
    \end{proof}

    From \cref{prop:factorization_of_primes_quadratic}, the factorization of primes in $\Q(\sqrt{d})$ is controlled by the quadratic character $\chi_{\D_{d}}$ associated to the fundamental discriminant $\D_{d}$. In other words, the factorization of $p$ depends completely upon if $\D_{d}$ is a square modulo $p$. While splitting of primes can be explicitly described for quadratic number fields, the class number is a significantly more difficult problem. We will write $h_{d} = h_{\Q(\sqrt{d})}$. The \textbf{class number problem}\index{class number problem} was originally introduced by Gauss and aims to classify all quadratic number fields of a given class number:

    \begin{problem*}[Class number problem]
      For a fixed $n \ge 1$, classify all quadratic number fields $\Q(\sqrt{d})$ of class number $n$.
    \end{problem*}

    Some progress has been made toward the class number problem. In 1801, Gauss found nine imaginary quadratic numbers fields of class number $1$ (see \cite{gauss1801disquisitiones}). They are listed according to $d$ as follows:
    \[
      d  \in \{-1,-2,-3,-7,-11,-19,-43,-67,-163\}.
    \]
    Gauss also conjectured that these are the only imaginary quadratic numbers fields of class number $1$. An argument was presented by Heegner in 1952 (see \cite{heegner1952diophantische}) which was correct up to some minor flaws. Baker and Stark both independently gave independent proofs in the mid 1960's (see \cite{baker1967linear,stark1967complete}) resulting in the following theorem which solves the class number problem for imaginary quadratic number fields in the case $n = 1$:

    \begin{theorem}
      If $\Q(\sqrt{d})$ is an imaginary quadratic number field of class number $1$ then
      \[
        d \in \{-1,-2,-3,-7,-11,-19,-43,-67,-163\}.
      \]
      Equivalently, an imaginary quadratic number field $\Q(\sqrt{d})$ has class number $1$ if and only if
      \[
        \D_{d} \in \{-3,-4,-7,-8,-11,-19,-43,-67,-163\}.
      \]
    \end{theorem}

    As for real quadratic fields, we know much less. In the same 1801 paper of Gauss (see \cite{gauss1801disquisitiones}), he conjectured that there should be infinitely many real quadratic fields and that the class number should remain unbounded:

    \begin{conjecture}
      There are infinitely many real quadratic fields $\Q(\sqrt{d})$ that have class number $1$. Moreover,
      \[
        \lim_{d \to \infty}h_{d} = \infty.
      \]
    \end{conjecture}

    While the class number problem remains quite out of reach, the structure of the unit group is much easier to classify. Write $\mu(d) = \mu(\Q(\sqrt{d}))$ and $w_{d} = w_{\Q(\sqrt{d})}$. In fact, by Dirichlet's unit theorem we only need to understand the roots of unity $\mu_{d}$ of $\Q(\sqrt{d})$. In all but two cases, $\mu(d) = \<-1\>$:

    \begin{proposition}\label{prop:unit_group_quadratic}
      Let $\Q(\sqrt{d})$ be a quadratic number field. Then
      \[
        \mu(d) = \begin{cases} \<i\> & \text{if $d = -1$}, \\ \<\w_{6}\> & \text{if $d = -3$}, \\ \<-1\> & \text{otherwise}, \end{cases} \quad \text{and} \quad w_{d} = \begin{cases} 4 & \text{if $d = -1$}, \\ 6 & \text{if $d = -3$}, \\ 2 & \text{otherwise}, \end{cases}
      \]
      where $\w_{6}$ is primitive $6$-th root of unity. In particular,
      \[
        \mc{O}_{d}^{\ast} = \begin{cases} \mu(d) \x \<\e_{d}\> & \text{if $d > 0$}, \\ \mu(d) & \text{if $d < 0$}, \end{cases}
      \]
      where $\e_{d}$ is a fundamental unit.
    \end{proposition}
    \begin{proof}
        First suppose $d > 0$. Then $\Q(\sqrt{d}) \subset \R$ and thus $\mu(d) = \<-1\>$ since these are the only roots of unity in $\R$ and clearly they are in $\Q(\sqrt{d})$. Now suppose $d < 0$. Then $\Q(\sqrt{d})$ is imaginary and its signature is $(0,1)$. Recall that $\a \in \mc{O}_{d}$ is a unit if and only if the norm of $\a$ is $\pm 1$. Actually, since $d < 0$ the definition of the norm shows that the norm is always nonnegative. Hence $\a$ is a unit if and only if its norm is $1$. First suppose $d \equiv 2,3 \tmod{4}$. Then \cref{prop:ring_of_integers_quadratic} implies
      \[
        \a = a+b\sqrt{d},
      \]
      for some $a,b \in \Z$, and $\a$ is a unit if and only if
      \[
        \Norm(\a) = a^{2}-b^{2}d = a^{2}+b^{2}|d| = 1.
      \]
      Since $|d| \equiv d \equiv 2,3 \tmod{4}$, this happens if and only if $b = 0$ unless $d = -1$. In the former case, $d < 0$ and $\a = a$ is a unit if and only if $a^{2} = 1$ which is to say that $\a = \pm 1$. In the latter case, $d = -1$ and $\a = a+bi$ with $b \neq 0$ (for otherwise we are in the former case) is a unit if and only if $a^{2}+b^{2} = 1$ which means $a = \pm1$ and $b = 0$ or $a = 0$ and $b = \pm 1$ so that $\a$ runs over the $4$-th roots of unity. Altogether, we have shown that $\mu(d) = \<-1\>$ provided $d \equiv 2,3 \tmod{4}$ unless $d = -1$ in which case $\mu(-1) = \<i\>$. Now suppose $d \equiv 1 \tmod{4}$. Then \cref{prop:ring_of_integers_quadratic} implies
      \[
        \a = a+b\frac{1+\sqrt{d}}{2} = \frac{2a+b}{2}+\frac{b}{2}\sqrt{d},
      \]
      for some $a,b \in \Z$, and $\a$ is a unit if and only if
      \[
        \Norm(\a) = \frac{4a^{2}+4ab+b^{2}}{4}-\frac{b^{2}}{4}d = a^{2}+ab+(1+|d|)\frac{b^{2}}{4} = 1.
      \]
      Since $|d| \equiv d \equiv 1 \tmod{4}$, this happens if and only if $b = 0$ or $d = -3$ (if $b \neq 0$ then $a^{2}+ab+(1+|d|)\frac{b^{2}}{4} > 1$ for such $d$ unless $d = -3$). In the former case, $\a = a$ is a unit if and only if $a^{2} = 1$ which means $\a = \pm1$. In the latter case, $\a = a+b\frac{1+\sqrt{-3}}{2}$ is a unit if and only if $a^{2}+ab+b^{2} = 1$ which happens if $a = \pm 1$ and $b = 0$, $a = 0$ and $b = \pm 1$, $a = 1$ and $b = -1$, or $a = -1$ and $b = 1$ so that $\a$ runs over the $6$-th roots of unity. This shows $\mu(d) = \<-1\>$ provided $d \equiv 1 \tmod{4}$ unless $d = -3$ in which case $\mu(-3) = \<\w_{6}\>$. This proves the claim about $\mu(d)$ in all cases and the statement about $w_{d}$ follows immediately. To prove the last statement, the signature of $\Q(\sqrt{d})$ is $(2,0)$ or $(0,1)$ according to if $d > 0$ or $d < 0$. Applying Dirichlet's unit theorem completes the proof.
    \end{proof}

    Lastly, we discuss the regulator. Set $R_{d} = R_{\Q(\sqrt{d})}$. Then we have the following proposition:

    \begin{proposition}\label{prop:regulator_quadratic}
      Let $\Q(\sqrt{d})$ be a quadratic number field. Then
      \[
        R_{d} = \begin{cases} \log|\e_{d}| & \text{if $d > 0$}, \\ 1 & \text{if $d < 0$}, \end{cases}
      \]
      where $\e_{d}$ is a fundamental unit.
    \end{proposition}
    \begin{proof}
      If $d > 0$ then \cref{prop:unit_group_quadratic} implies that a system of fundamental units for $\Q(\sqrt{d})$ is given by a single fundamental unit $\e_{d}$. Since the signature of $\Q(\sqrt{d})$ is $(2,0)$, we have $\l(\e_{d}) = (\log|\e_{d}|,\log|\e_{d}|)$ and therefore $R_{K} = \log|\e_{d}|$. If $d < 0$ then \cref{prop:unit_group_quadratic} implies that there are no fundamental units and thus $R_{K} = 1$.
    \end{proof}
  \section{\todo{Cyclotomic Number Fields}}