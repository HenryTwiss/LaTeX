\chapter{Quadratic and Cyclotomic Number Fields}
  We provide a more detailed discussion of number fields with particularly simple structure. Namely, we discuss quadratic and cyclotomic number fields because they are monogenic.
  \section{Quadratic Number Fields}
    We will classify and discuss the structure of quadratic number fields. Note that since any degree $2$ extension is normal, quadratic number fields are automatically Galois. We first show that quadratic number fields are exactly those where we adjoin the square-root of a square-free integer other than $0$ or $1$:

    \begin{proposition}\label{prop:classification_of_quadratic_number_fields}
      Every quadratic number field $K$ is of the form $K = \Q(\sqrt{d})$ for some square-free integer $d$ other than $0$ or $1$.
    \end{proposition}
    \begin{proof}
      Suppose $K$ is a quadratic number field. In particular, $K/\Q$ is separable so by the primitive element theorem there exists $\t \in K$ such that $K = \Q(\t)$. The minimal polynomial $m_{\t}(x)$ of $\t$ over $\Q$ is of the form
      \[
        m_{\t}(x) = x^{2}+ax+b,
      \]
      for $a,b \in \Q$. Then the quadratic formula gives
      \[
        \t = -\frac{a}{2}\pm\frac{\sqrt{q}}{2},
      \]
      where $q = a^{2}-4b \in \Q$. Clearly $q \neq 0$ and $q \neq 1$ for otherwise $\t \in \Q$. It follows that $K = \Q(\sqrt{q})$. Write $q = \frac{n}{m}$ for relatively prime $n,m \in \Z$ and set $d = m^{2}q = nm \in \Z$. Then $d$ is square-free, $d \neq 0$, and $d \neq 1$. Moreover, $\sqrt{d} = m\sqrt{q}$ so that $K = \Q(\sqrt{d})$. 
    \end{proof}

    From \cref{prop:classification_of_quadratic_number_fields}, we see that the $d$ for a quadratic number field $\Q(\sqrt{d})$ satisfies $d \equiv 1,2,3 \tmod{4}$ (otherwise $d$ is not square-free). Moreover, any element of a quadratic number field is of the form $a+b\sqrt{d}$ with $a,b \in \Q$ and for some square-free $d$ other than $0$ or $1$. We say that a quadratic number field $\Q(\sqrt{d})$ is \textbf{real}\index{real} if $d > 0$ and \textbf{imaginary}\index{imaginary} if $d < 0$. Now $\Q(\sqrt{d})$ is real or imaginary according to if $\sqrt{d}$ is real or purely imaginary so that the two elements $\s_{1}$ and $\s_{2}$ of $\Hom_{\Q}(\Q(\sqrt{d}),\conj{\Q})$ are
    \[
      \s_{1}(a+b\sqrt{d}) = a+b\sqrt{d} \quad \text{and} \quad \s_{2}(a+b\sqrt{d}) = a-b\sqrt{d},
    \]
    because the roots of the minimal polynomial for $\sqrt{d}$ over $\Q$ are $\pm\sqrt{d}$. In particular, the signature is $(2,0)$ or $(0,1)$ according to if $\Q(\sqrt{d})$ is real or imaginary. We write $\Trace_{d} = \Trace_{\Q(\sqrt{d})}$ and $\Norm_{d} = \Norm_{\Q(\sqrt{d})}$ for the field trace and field norm of $\Q(\sqrt{d})$ respectively. Then \cref{prop:formulas_for_trace_and_norm} shows that the trace and norm of $\k = a+b\sqrt{d} \in \Q(\sqrt{d})$ are given by
    \[
      \Trace_{d}(\k) = 2a \quad \text{and} \quad \Norm_{d}(\k) = a^{2}-b^{2}d.
    \]
    We will now begin describing the ring of integers, discriminant, and the factorization of primes in $\Q(\sqrt{d})$. For simplicity, we will write $\mc{O}_{d} = \mc{O}_{\Q(\sqrt{d})}$ and $\D_{d} = \D_{\Q(\sqrt{d})}$. The ring of integers has a particularly simple description since quadratic number fields are monogenic as the following proposition shows:
    
    \begin{proposition}\label{prop:ring_of_integers_quadratic}
      Let $\Q(\sqrt{d})$ be a quadratic number field. Then $\Q(\sqrt{d})$ is monogenic where
      \[
        \mc{O}_{d} = \begin{cases} \Z\left[\frac{1+\sqrt{d}}{2}\right] & \text{if $d \equiv 1 \tmod{4}$}, \\ \Z[\sqrt{d}] & \text{if $d \equiv 2,3 \tmod{4}$}. \end{cases}
      \]
    \end{proposition}
    \begin{proof}
      Let $\a = a+b\sqrt{d} \in \Q(\sqrt{d})$ be an algebraic integer. If $b = 0$ then $\a \in \Q$ and since the only elements of $\Q$ that are algebraic integers are the integers themselves we must have that $\a$ is an integer. Now suppose $b \neq 0$. Then the minimal polynomial of $\a$ over $\Q$ is
      \[
        m_{\a}(x) = x^{2}+2ax+(a^{2}-b^{2}d) = (x-(a+b\sqrt{d}))(x-(a-b\sqrt{d})).
      \]
      As $\a$ is an algebraic integer, $2a \in \Z$ and $a^{2}-b^{2}d \in \Z$ (note that these are the trace and norm of $\a$ respectively). In particular, $(2a)^{2}+(2b)^{2}d \in \Z$ and hence $(2b)^{2} \in \Z$ is as well. But as $b \in \Q$, it must be the case that $2b \in \Z$. If $2a = n+1$ is odd then $n$ is even. We compute
      \[
        a^{2}-b^{2}d = \left(\frac{n+1}{2}\right)^{2}-b^{2}d = \frac{n^{2}+2n+1+4b^{2}d}{4},
      \]
      and since the right-hand side must be an integer $b \notin \Z$. For if $b \in \Z$, the numerator of the right-hand side is equivalent to $1$ modulo $4$ because $n$ is even. As $2b \in \Z$ it follows that $2b$ must be odd so set $2b = m+1$ with $m$ even. Again, we compute
      \[
        a^{2}-b^{2}d = \left(\frac{n+1}{2}\right)^{2}-\left(\frac{m+1}{2}\right)^{2}d = \frac{n^{2}+2n+1-d(m^{2}+2m+1)}{4},
      \]
      and since the right-hand side must be an integer the numerator must be divisible by $4$. As $n$ and $m$ are even, this is equivalent to $d \equiv 1 \tmod{4}$. So we have shown $2a$ or $2b$ is odd if and only if $d \equiv 1 \tmod{4}$. Thus if $d \equiv 1 \tmod{4}$, we have $a = \frac{a'}{2}$ and $b = \frac{b'}{2}$ for some $a',b' \in \Z$ and hence $\a \in \Z\left[\frac{1+\sqrt{d}}{2}\right]$. Otherwise, $d \equiv 2,3 \tmod{4}$ (because $d$ is square-free) so that $2a$ and $2b$ are both even, $a,b \in \Z$, and therefore $\a \in \Z[\sqrt{d}]$. We have now shown that $\mc{O}_{d} \subseteq \Z\left[\frac{1+\sqrt{d}}{2}\right]$ and $\mc{O}_{d} \subseteq \Z[\sqrt{d}]$ according to if $d \equiv 1 \tmod{4}$ or $d \equiv 2,3 \tmod{4}$ respectively. For the reverse containment, it suffices to show that $\frac{1+\sqrt{d}}{2}$ and $\sqrt{d}$ are algebraic integers according to if $d \equiv 1 \tmod{4}$ or $d \equiv 2,3 \tmod{4}$ respectively since $\mc{O}_{K}$ is a ring. Indeed they are since their minimal polynomials over $\Q$ are
      \[
        m_{\frac{1+\sqrt{d}}{2}}(x) = x^{2}-x+\frac{1-d}{4} \quad \text{and} \quad m_{\sqrt{d}}(x) = x^{2}-d,
      \]
      where $\frac{1-d}{4} \in \Z$ because $d \equiv 1 \tmod{4}$.
    \end{proof}

    It follows from \cref{prop:ring_of_integers_quadratic} that
    \[
      1,\frac{1+\sqrt{d}}{2} \quad \text{and} \quad 1,\sqrt{d},
    \]
    are integral bases for $\mc{O}_{d}$ according to if $d \equiv 1 \tmod{4}$ or $d \equiv 2,3 \tmod{4}$ respectively. Let us now show that the discriminants quadratic number fields are exactly the fundamental discriminants other than $1$:

    \begin{proposition}\label{prop:discriminant_quadratic}
      Let $\Q(\sqrt{d})$ be a quadratic number field. Then
      \[
        \D_{d} = \begin{cases} d & \text{if $d \equiv 1 \tmod{4}$}, \\ 4d & \text{if $d \equiv 2,3 \tmod{4}$}. \end{cases}
      \]
      In particular, the discriminants quadratic number fields are exactly the fundamental discriminants other than $1$.
    \end{proposition}
    \begin{proof}
      Let $\s_{1}$ and $\s_{2}$ be the two elements of $\Hom_{\Q}(\Q(\sqrt{d}),\conj{\Q})$ where $\s_{1}$ is the identity and $\s_{2}$ is given by sending $\sqrt{d}$ to its conjugate. If $d \equiv 1 \tmod{4}$, an integral basis for $\mc{O}_{d}$ is $1,\frac{1+\sqrt{d}}{2}$. In this case, the embedding matrix is
      \[
        M\left(1,\frac{1+\sqrt{d}}{2}\right) = \begin{pmatrix} 1 & \frac{1+\sqrt{d}}{2} \\ 1 & \frac{1-\sqrt{d}}{2} \end{pmatrix},
      \]
      and thus $\D_{d} = d$. If $d \equiv 2,3 \tmod{4}$, an integral basis for $\mc{O}_{d}$ is $1,\sqrt{d}$. In this case, the embedding matrix is
      \[
        M(1,\sqrt{d}) = \begin{pmatrix} 1 & \sqrt{d} \\ 1 & -\sqrt{d} \end{pmatrix},
      \]
      and hence $\D_{d} = 4d$. This proves the first statement and the second statement is clear since $d$ is square-free and not $0$ or $1$.
    \end{proof}

    From now on, we will write fundamental discriminants other than $1$ as $\D_{d}$ instead of $D$ to clarify the connection to quadratic number fields. We will now discuss the factorization of $p\mc{O}_{d}$ in the quadratic number field $\Q(\sqrt{d})$ for a prime $p$. Since $\Q(\sqrt{d})$ is a degree $2$ extension, the fundamental equality (or \cref{prop:Galois_action_on_primes_is_transitive} using that $\Q(\sqrt{d})/\Q$ is Galois) implies that $p$ is either inert, totally ramified, or totally split. In other words, there are three possible cases for how $p\mc{O}_{d}$ factors:
    \[
      p\mc{O}_{d} = \mf{p}, \quad p\mc{O}_{d} = \mf{p}^{2}, \quad \text{and} \quad p\mc{O}_{d} = \mf{p}\mf{q},
    \]
    for some primes $\mf{p}$ and $\mf{q}$, according to if $p$ is inert, totally ramified, or totally split. Since $\Q(\sqrt{d})$ is monogenic by \cref{prop:ring_of_integers_quadratic}, the conductor is $\mc{O}_{K}$ and so we can describe the factorization using the Dedekind-Kummer theorem for every prime $p$. In particular, we will connect the prime factorization to the quadratic character $\chi_{\D_{d}}$ associated to the fundamental discriminant $\D_{d}$:

    \begin{proposition}\label{prop:factorization_of_primes_quadratic}
      Let $\Q(\sqrt{d})$ be a quadratic number field and let $\chi_{\D_{d}}$ be the quadratic character given by the fundamental discriminant $\D_{d}$. Then for any prime $p$, we have
      \[
        \chi_{\D_{d}}(p) = \begin{cases} 1 & \text{if $p$ is totally split}, \\ -1 & \text{if $p$ is inert}, \\ 0 & \text{if $p$ is totally ramified}. \end{cases}
      \]
    \end{proposition}
    \begin{proof}
      Recall that $p$ is ramified if and only if it divides $|\D_{d}|$ and note that this is exactly when $\chi_{\D_{d}}(p) = 0$. Therefore it suffices to prove the cases when $p$ is totally split or inert. First suppose $d \equiv 1 \tmod{4}$ so that $\mc{O}_{d} = \Z\left[\frac{1+\sqrt{d}}{2}\right]$ and $\D_{d} = d$ by \cref{prop:ring_of_integers_quadratic,prop:discriminant_quadratic}. The minimal polynomial $m_{\frac{1+\sqrt{d}}{2}}(x)$ for $\frac{1+\sqrt{d}}{2}$ over $\Q$ is
      \[
        m_{\frac{1+\sqrt{d}}{2}}(x) = x^{2}-x+\frac{1-d}{4},
      \]
      where $\frac{1-d}{4} \in \Z$ because $d \equiv 1 \tmod{4}$. The reduction of $m_{\frac{1+\sqrt{d}}{2}}(x)$ modulo $p$ is either irreducible, factors into two distinct linear factors, or is a square, and Dedekind-Kummer theorem implies that this is equivalent to $p$ being inert, totally split, or totally ramified accordingly because the prime factorization of fractional ideals is unique. First suppose $p \neq 2$. Then from the quadratic formula, $m_{\frac{1+\sqrt{d}}{2}}(x)$ reduces modulo $p$ as
      \[
        m_{\frac{1+\sqrt{d}}{2}}(x) \equiv \left(x-\frac{1+\sqrt{d}}{2}\right)\left(x-\frac{1-\sqrt{d}}{2}\right) \pmod{p},
      \]
      if and only if the roots $\frac{1\pm\sqrt{d}}{2}$ are elements of $\F_{p}$ and is otherwise irreducible. As $p \neq 2$, these factors are distinct. Moreover, $\frac{1\pm\sqrt{d}}{2}$ is an element of $\F_{p}$ if and only if $d$ is a square modulo $p$ and hence $p$ is totally split or inert according to if $\chi_{d}(p) = \pm1$. Now suppose $p = 2$. Since $m_{\frac{1+\sqrt{d}}{2}}(x)$ has a nonzero linear term with an odd coefficient, it reduces modulo $2$ as
      \[
        m_{\frac{1+\sqrt{d}}{2}}(x) \equiv x(x-1) \tmod{2},
      \]
      if and only if $\frac{1-d}{4} \equiv 0 \tmod{2}$ and is otherwise irreducible. Clearly these factors are distinct. Now observe $\frac{1-d}{4} \equiv 0 \tmod{2}$ is equivalent to $d \equiv 1 \tmod{8}$ provided $d > 0$ and $d \equiv 7 \tmod{8}$ provided $d < 0$ and thus $p$ is totally split or inert according to if $\chi_{d}(2) = \pm1$. This completes the argument in the case $d \equiv 1 \tmod{4}$. Now suppose $d \equiv 2,3 \tmod{4}$ so that $\mc{O}_{d} = \Z[\sqrt{d}]$ and $\D_{d} = 4d$ by \cref{prop:ring_of_integers_quadratic,prop:discriminant_quadratic}. The minimal polynomial $m_{\sqrt{d}}(x)$ for $\sqrt{d}$ over $\Q$ is
      \[
        m_{\sqrt{d}}(x) = x^{2}-d.
      \]
      As $\D_{d} = 4d$, we see that $2$ is ramified and therefore we may assume $p \neq 2$. Similarly, the reduction of $m_{\sqrt{d}}(x)$ modulo $p$ is either irreducible, factors into two distinct linear factors, or is a square, and Dedekind-Kummer theorem implies that this is equivalent to $p$ being inert, totally split, or totally ramified accordingly because the prime factorization of fractional ideals is unique. As $p \neq 2$, the quadratic formula implies that $m_{\sqrt{d}}(x)$ reduces modulo $p$ as
      \[
        m_{\sqrt{d}}(x) \equiv (x-\sqrt{d})(x+\sqrt{d}) \pmod{p},
      \]
      if and only if the roots $\pm\sqrt{d}$ are elements of $\F_{p}$. As $p \neq 2$, these factors are distinct. Moreover, $\sqrt{d}$ is an element of $\F_{p}$ if and only if $d$ and hence $4d$ are squares modulo $p$ so that $p$ is totally split or inert according to if $\chi_{4d}(p) = \pm1$. This completes the verification in the case $d \equiv 2,3 \tmod{4}$.
    \end{proof}

    From \cref{prop:factorization_of_primes_quadratic}, the factorization of primes in $\Q(\sqrt{d})$ is controlled by the quadratic character $\chi_{\D_{d}}$ associated to the fundamental discriminant $\D_{d}$. In other words, the factorization of $p$ depends completely upon if $\D_{d}$ is a square modulo $p$. While splitting of primes can be explicitly described for quadratic number fields, the class number is a significantly more difficult problem. We will write $h_{d} = h_{\Q(\sqrt{d})}$. The \textbf{class number problem}\index{class number problem} was originally introduced by Gauss and aims to classify all quadratic number fields of a given class number:

    \begin{problem*}[Class number problem]
      For a fixed $n \ge 1$, classify all quadratic number fields $\Q(\sqrt{d})$ of class number $n$.
    \end{problem*}

    Some progress has been made toward the class number problem. In 1801, Gauss found nine imaginary quadratic numbers fields of class number $1$ (see \cite{gauss1801disquisitiones}). They are listed according to $d$ as follows:
    \[
      d  \in \{-1,-2,-3,-7,-11,-19,-43,-67,-163\}.
    \]
    Gauss also conjectured that these are the only imaginary quadratic numbers fields of class number $1$. An argument was presented by Heegner in 1952 (see \cite{heegner1952diophantische}) which was correct up to some minor flaws. Baker and Stark both independently gave independent proofs in the mid 1960's (see \cite{baker1967linear,stark1967complete}) resulting in the following theorem which solves the class number problem for imaginary quadratic number fields in the case $n = 1$:

    \begin{theorem}
      If $\Q(\sqrt{d})$ is an imaginary quadratic number field of class number $1$ then
      \[
        d \in \{-1,-2,-3,-7,-11,-19,-43,-67,-163\}.
      \]
      Equivalently, an imaginary quadratic number field $\Q(\sqrt{d})$ has class number $1$ if and only if
      \[
        \D_{d} \in \{-3,-4,-7,-8,-11,-19,-43,-67,-163\}.
      \]
    \end{theorem}

    As for real quadratic fields, we know much less. In the same 1801 paper of Gauss (see \cite{gauss1801disquisitiones}), he conjectured that there should be infinitely many real quadratic fields and that the class number should remain unbounded:

    \begin{conjecture}
      There are infinitely many real quadratic fields $\Q(\sqrt{d})$ that have class number $1$. Moreover,
      \[
        \lim_{d \to \infty}h_{d} = \infty.
      \]
    \end{conjecture}

    While the class number problem remains quite out of reach, the structure of the unit group is much easier to classify. Write $\mu(d) = \mu(\Q(\sqrt{d}))$ and $w_{d} = w_{\Q(\sqrt{d})}$. In fact, by Dirichlet's unit theorem we only need to understand the roots of unity $\mu_{d}$ of $\Q(\sqrt{d})$. In all but two cases, $\mu(d) = \<-1\>$:

    \begin{proposition}\label{prop:unit_group_quadratic}
      Let $\Q(\sqrt{d})$ be a quadratic number field. Then
      \[
        \mu(d) = \begin{cases} \<i\> & \text{if $d = -1$}, \\ \<\w_{6}\> & \text{if $d = -3$}, \\ \<-1\> & \text{otherwise}, \end{cases} \quad \text{and} \quad w_{d} = \begin{cases} 4 & \text{if $d = -1$}, \\ 6 & \text{if $d = -3$}, \\ 2 & \text{otherwise}, \end{cases}
      \]
      where $\w_{6}$ is primitive $6$-th root of unity. In particular,
      \[
        \mc{O}_{d}^{\ast} = \begin{cases} \mu(d) \x \<\e_{d}\> & \text{if $d > 0$}, \\ \mu(d) & \text{if $d < 0$}, \end{cases}
      \]
      where $\e_{d}$ is a fundamental unit.
    \end{proposition}
    \begin{proof}
        First suppose $d > 0$. Then $\Q(\sqrt{d}) \subset \R$ and thus $\mu(d) = \<-1\>$ since these are the only roots of unity in $\R$ and clearly they are in $\Q(\sqrt{d})$. Now suppose $d < 0$. Then $\Q(\sqrt{d})$ is imaginary and its signature is $(0,1)$. Recall that $\a \in \mc{O}_{d}$ is a unit if and only if the norm of $\a$ is $\pm 1$. Actually, since $d < 0$ the definition of the norm shows that the norm is always nonnegative. Hence $\a$ is a unit if and only if its norm is $1$. First suppose $d \equiv 2,3 \tmod{4}$. Then \cref{prop:ring_of_integers_quadratic} implies
      \[
        \a = a+b\sqrt{d},
      \]
      for some $a,b \in \Z$, and $\a$ is a unit if and only if
      \[
        \Norm_{d}(\a) = a^{2}-b^{2}d = a^{2}+b^{2}|d| = 1.
      \]
      Since $|d| \equiv d \equiv 2,3 \tmod{4}$, this happens if and only if $b = 0$ unless $d = -1$. In the former case, $d < 0$ and $\a = a$ is a unit if and only if $a^{2} = 1$ which is to say that $\a = \pm 1$. In the latter case, $d = -1$ and $\a = a+bi$ with $b \neq 0$ (for otherwise we are in the former case) is a unit if and only if $a^{2}+b^{2} = 1$ which means $a = \pm1$ and $b = 0$ or $a = 0$ and $b = \pm 1$ so that $\a$ runs over the $4$-th roots of unity. Altogether, we have shown that $\mu(d) = \<-1\>$ provided $d \equiv 2,3 \tmod{4}$ unless $d = -1$ in which case $\mu(-1) = \<i\>$. Now suppose $d \equiv 1 \tmod{4}$. Then \cref{prop:ring_of_integers_quadratic} implies
      \[
        \a = a+b\frac{1+\sqrt{d}}{2} = \frac{2a+b}{2}+\frac{b}{2}\sqrt{d},
      \]
      for some $a,b \in \Z$, and $\a$ is a unit if and only if
      \[
        \Norm_{d}(\a) = \frac{4a^{2}+4ab+b^{2}}{4}-\frac{b^{2}}{4}d = a^{2}+ab+(1+|d|)\frac{b^{2}}{4} = 1.
      \]
      Since $|d| \equiv d \equiv 1 \tmod{4}$, this happens if and only if $b = 0$ or $d = -3$ (if $b \neq 0$ then $a^{2}+ab+(1+|d|)\frac{b^{2}}{4} > 1$ for such $d$ unless $d = -3$). In the former case, $\a = a$ is a unit if and only if $a^{2} = 1$ which means $\a = \pm1$. In the latter case, $\a = a+b\frac{1+\sqrt{-3}}{2}$ is a unit if and only if $a^{2}+ab+b^{2} = 1$ which happens if $a = \pm 1$ and $b = 0$, $a = 0$ and $b = \pm 1$, $a = 1$ and $b = -1$, or $a = -1$ and $b = 1$ so that $\a$ runs over the $6$-th roots of unity. This shows $\mu(d) = \<-1\>$ provided $d \equiv 1 \tmod{4}$ unless $d = -3$ in which case $\mu(-3) = \<\w_{6}\>$. This proves the claim about $\mu(d)$ in all cases and the statement about $w_{d}$ follows immediately. To prove the last statement, the signature of $\Q(\sqrt{d})$ is $(2,0)$ or $(0,1)$ according to if $d > 0$ or $d < 0$. Applying Dirichlet's unit theorem completes the proof.
    \end{proof}

    Lastly, we discuss the regulator. Set $R_{d} = R_{\Q(\sqrt{d})}$. Then we have the following proposition:

    \begin{proposition}\label{prop:regulator_quadratic}
      Let $\Q(\sqrt{d})$ be a quadratic number field. Then
      \[
        R_{d} = \begin{cases} \log|\e_{d}| & \text{if $d > 0$}, \\ 1 & \text{if $d < 0$}, \end{cases}
      \]
      where $\e_{d}$ is a fundamental unit.
    \end{proposition}
    \begin{proof}
      If $d > 0$ then \cref{prop:unit_group_quadratic} implies that a system of fundamental units for $\Q(\sqrt{d})$ is given by a single fundamental unit $\e_{d}$. Since the signature of $\Q(\sqrt{d})$ is $(2,0)$, we have $\l(\e_{d}) = (\log|\e_{d}|,\log|\e_{d}|)$ and therefore $R_{K} = \log|\e_{d}|$. If $d < 0$ then \cref{prop:unit_group_quadratic} implies that there are no fundamental units and thus $R_{K} = 1$.
    \end{proof}
  \section{Cyclotomic Number Fields}
    Let $\w$ be a primitive $n$-th root of unity. We call $\Q(\w)$ the $n$-th \textbf{cyclotomic field}\index{cyclotomic field}. Note that $\Q(\w)$ is independent of the choice of primitive root $\w$ since $\Q(\w)$ contains all $n$-th roots of unity. As $\w$ is a root of $x^{n}-1$, we see that $\Q(\w)/\Q$ is a finite extension of degree at most $n$. In particular, $\Q(\w)$ is a number field. More generally, we say that a number field $K$ is \textbf{cyclotomic}\index{cyclotomic} if $K$ is the $n$-th cyclotomic field for some $n \ge 1$. That is, $K = \Q(\w)$ for some primitive $n$-th root of unity $\w$. In any case, our aim is to study the structure of cyclotomic number fields $\Q(\w)$. Our first step is to compute the degree of $\Q(\w)$ which is the degree of the minimal polynomial of $\w$ over $\Q$. Accordingly, we define the $n$-th \textbf{cyclotomic polynomial}\index{cyclotomic polynomial} $\Phi_{n}(x)$ by
    \[
      \Phi_{n}(x) = \prod_{k \in (\Z/n\Z)^{\ast}}(x-\w^{k}).
    \]
    That is, $\Phi_{n}(x)$ is the polynomial whose roots are the primitive $n$-th roots of unity. It is clearly monic, of degree $\vphi(n)$, and divides $x^{n}-1$. As every $n$-th root of unity is a primitive $d$-th root of unity for some $d \mid n$, we also find that
    \begin{equation}\label{equ:cyclotomic_product_identity}
      x^{n}-1 = \prod_{d \mid n}\Phi_{d}(x).
    \end{equation}
    Clearly $\Phi_{1}(x) = x-1$ and $\Phi_{2}(x) = x+1$. When $n = p$ for a prime $p$, \cref{equ:cyclotomic_product_identity} implies
    \[
      \Phi_{p}(x) = \frac{x^{p}-1}{x-1} = x^{p-1}+x^{p-2}+\cdots+1.
    \]
    More generally, writing $n = p^{k}$ and inducting on $k$ using \cref{equ:cyclotomic_product_identity} gives
    \begin{equation}\label{equ:cyclotomic_for_prime_power}
      \Phi_{p^{k}}(x) = \frac{x^{p^{k}}-1}{x^{p^{k-1}}-1} = \frac{x^{p^{k}}-1}{x^{p^{k-1}}-1} = x^{p^{k-1}(p-1)}+x^{p^{k-1}(p-2)}+\cdots+1.
    \end{equation}
    Observe from \cref{equ:cyclotomic_for_prime_power} that $\Phi_{p^{k}}(x)$ has coefficients in $\Z$. This is true for a general cyclotomic polynomial $\Phi_{n}(x)$ in addition to irreducibility over $\Z$ as the following proposition shows:

    \begin{proposition}\label{prop:cyclotomic_is_irreducible}
      $\Phi_{n}(x)$ has coefficients in and is irreducible over $\Z$.
    \end{proposition}
    \begin{proof}
      We first show $\Phi_{n}(x)$ has coefficients in $\Z$ and we will argue by induction. The claim is true for $n = 1$ since $\Phi_{1}(x) = x-1$. So assume by induction that it is true for all $1 \le d < n$. In view of \cref{equ:cyclotomic_product_identity}, we have
      \[
        x^{n}-1 = \Phi_{n}(x)\prod_{\substack{d \mid n \\ d < n}}\Phi_{d}(x),
      \]
      and $\prod_{\substack{d \mid n \\ d < n}}\Phi_{d}(x)$ has coefficients in $\Z$. Therefore $\prod_{\substack{d \mid n \\ d < n}}\Phi_{d}(x)$ divides $x^{n}-1$ in $\Q[x]$ and hence in $\Z[x]$ as well by Gauss's lemma. Thus $\Phi_{n}(x)$ has coefficients in $\Z$ as desired. We now show $\Phi_{n}(x)$ is irreducible over $\Z$. So suppose
      \[
        \Phi_{n}(x) = f(x)g(x),
      \]
      for monic polynomials $f(x),g(x) \in \Z[x]$ (recall $\Phi_{n}(x)$ is monic) with $f(x)$ irreducible. Then it suffices to show $f(x) = \Phi_{n}(x)$. Now let $\w$ be a root of $f(x)$. Then $\w$ is also a root of $\Phi_{n}(x)$ and necessarily a primitive $n$-th roots of unity. Since $f(x)$ is monic and irreducible it is necessarily the minimal polynomial of $\w$ over $\Q$. Now let $p$ be any prime not dividing $n$. Then $\w^{p}$ is also a primitive $n$-th root of unity and hence a root of either $f(x)$ or $g(x)$. Suppose $\w^{p}$ is a root of $g(x)$. Then $\w$ is a root of $g(x^{p})$, and since $f(x)$ is the minimal polynomial of $\w$ over $\Q$, $f(x)$ divides $g(x^{p})$ in $\Q[x]$. By Gauss's lemma, it follows that $f(x)$ divides $g(x^{p})$ in $\Z[x]$ too. Therefore
      \[
        g(x^{p}) = f(x)h(x),
      \]
      for a monic polynomial $h(x) \in \Z[x]$. Reducing this factorization modulo $p$, we obtain
      \[
        \conj{g}(x^{p}) \equiv \conj{g}(x)^{p} \equiv \conj{f}(x)\conj{h}(x) \pmod{p},
      \]
      where the first congruence holds since $\conj{g}(x^{p}) = \conj{g}(x)^{p}$ in $\F_{p}[x]$ (recall Fermat's little theorem and that the characteristic of $\F_{p}$ is $p$). As $p \ge 2$, this equivalence shows that $\conj{f}(x)$ and $\conj{h}(x)$ must have a common factor. In other words, $\conj{g}(x^{p})$ has a multiple root in $\F_{p}$ and therefore $\conj{g}(x)$ does as well. Reducing the factorization for $\Phi_{n}(x)$ modulo $p$ gives
      \[
        \conj{\Phi_{n}}(x) \equiv \conj{f}(x)\conj{g}(x) \pmod{p}.
      \]
      Then $\conj{\Phi_{n}}(x)$ has a multiple root in $\F_{p}$ since $\conj{g}(x)$ does. As $\conj{\Phi_{n}}(x)$ divides $x^{n}-1$ (because $\Phi_{n}(x)$ does and $x^{n}-1$ is itself reduced modulo $p$), it follows that $x^{n}-1$ has a multiple root in $\F_{p}$. This is impossible since $x^{n}-1$ has $n$ distinct roots in $\conj{\F_{p}}$ as $p$ does not divide $n$ (recall that the derivative of $x^{n}-1$ is $nx^{n-1}$ which is relatively prime to $p$). It follows that $\w^{p}$ cannot be a root of $g(x)$ and is therefore a root of $f(x)$. Now let $k \in (\Z/n\Z)^{\ast}$ and write $k = p_{1}p_{2} \cdots p_{k}$ as a product of primes not dividing $n$. Then $\w^{k} = \w^{p_{1}p_{2} \cdots p_{k}}$ is a root of $f(x)$ and hence every primitive $n$-th root of unity is a root of $f(x)$. Thus $f(x) = \Phi_{n}(x)$ which proves $\Phi_{n}(x)$ is irreducible over $\Z$.
    \end{proof}

    Since $\Phi_{n}(x)$ is monic, \cref{prop:cyclotomic_is_irreducible} implies that $\Phi_{n}(x)$ is the minimal polynomial of $\w$ over $\Q$ and hence of every primitive $n$-th root of unity over $\Q$. It follows that the degree of $\Q(\w)$ is $\vphi(n)$ because this is the degree of $\Phi_{n}(x)$. This implies $\Q(\w)$ is the splitting field of $\Phi_{n}(x)$ over $\Q$ because if one primitive $n$-th root of unity belongs to a field then they all do (as they are powers of each other). In particular, $\Q(\w)/\Q$ is normal and hence Galois. Moreover, every primitive $n$-root of unity is an algebraic integer since $\Phi_{n}(x)$ also has coefficients in $\Z$ by \cref{prop:cyclotomic_is_irreducible}. We now turn to the question of the ring of integers of $\Q(\w)$. For convenience write $\mc{O}_{\w} = \mc{O}_{\Q(\w)}$ and set
    \[
      \mf{p}_{\w} = (1-\w)\mc{O}_{\w}.
    \]
    We will first prove a useful lemma which shows that $\mf{p}_{\w}$ is a prime of $\Q(\w)$ and more in the case $n$ is a prime power:

    \begin{lemma}\label{lem:prime_power_cyclotomic_lemma}
      Let $\Q(\w)$ be the cyclotomic number field generated by a primitive $p^{e}$-th root of unity $\w$ with for some prime $p$ and $e \ge 1$. Then
      \[
        p\mc{O}_{\w} = \mf{p}_{\w}^{\vphi(p^{e})}.
      \]
      In particular, $\mf{p}_{\w}$ is a prime above $p$ with $f_{p}(\mf{p}_{\w}) = 1$. Moreover, $1,\w,\ldots,\w^{\vphi(p^{e})-1}$ is a basis for $\Q(\w)/\Q$ with
      \[
        d_{\Q(\w)/\Q}(1,\w,\ldots,\w^{\vphi(p^{e})-1}) = \pm p^{\vphi(p^{e})e-p^{e-1}}.
      \]
    \end{lemma}
    \begin{proof}
      In view of the definition of $\Phi_{p^{e}}(x)$ and \cref{equ:cyclotomic_for_prime_power}, we have
      \[
        x^{p^{e-1}(p-1)}+x^{p^{e-1}(p-2)}+\cdots+1 = \prod_{k \in (\Z/p^{e}\Z)^{\ast}}(x-\w^{k}).
      \]
      Setting $x = 1$ gives
      \[
        p = \prod_{k \in (\Z/p^{e}\Z)^{\ast}}(1-\w^{k}).
      \]
      In the case $e = 1$, $\w$ is a primitive $p$-th root of unity. Then $\Norm_{\Q(\w)/\Q}(1-\w) = p$ by \cref{prop:formulas_for_trace_and_norm} since $\Q(\w)/\Q$ is Galois. In any case, the factors $1-\w^{k}$ are clearly algebraic integers because $\w$ is (as a consequence of \cref{prop:cyclotomic_is_irreducible}). Then
      \[
        \e_{k} = \frac{1-\w^{k}}{1-\w} = \w^{k-1}+\w^{k-2}+\cdots+1,
      \]
      is also an algebraic integer and satisfies $1-\w^{k} = \e_{k}(1-\w)$. Moreover,
      \[
        \e_{k}^{-1} = \frac{1-\w}{1-\w^{k}} = \frac{1-\w^{k\conj{k}}}{1-\w^{k}} = \w^{k(\conj{k}-1)}+\w^{k(\conj{k}-2)}+\cdots+1,
      \]
      is also an algebraic integer. This means $\e_{k}$ is a unit in $\mc{O}_{\w}$. So upon setting $\e = \prod_{k \in (\Z/p^{e}\Z)^{\ast}}\e_{k}$, we conclude that
      \[
        p = \e(1-\w)^{\vphi(p^{e})},
      \]
      and therefore
      \[
        p\mc{O}_{\w} = \mf{p}_{\w}^{\vphi(p^{e})}.
      \]
      Since the degree of $\Q(\w)$ is $\vphi(p^{e})$, the fundamental equality implies that $\mf{p}_{\w}$ is prime (otherwise any prime factor has ramification index at least $\vphi(p^{e})$) and that $f_{p}(\mf{p}_{\w}) = 1$. This proves the first two statements. For the last two statements, $1,\w,\ldots,\w^{\vphi(p^{e})-1}$ is a basis for $\Q(\w)/\Q$ since $\w$ is a primitive element for $\Q(\w)/\Q$. Now let $\w_{1},\ldots,\w_{\vphi(p^{e})}$ be the conjugates of $\w$ with $\w_{1} = \w$. Then
      \[
        \Phi_{p^{e}}(x) = \prod_{1 \le i \le \vphi(p^{e})}(x-\w_{i}).
      \]
      Now \cref{equ:Vandermonde_determinant_for_discriminant} and \cref{prop:formulas_for_trace_and_norm} (since $\Q(\w)/\Q$ is Galois) give the first and last equalities in the following chain respectively:
      \[
        d(1,\l,\ldots,\l^{\vphi(p^{e})}) = \pm\prod_{\substack{1 \le i,j \le \vphi(p^{e}) \\ i \neq j}}(\w_{i}-\w_{j})^{2} = \pm\prod_{1 \le i \le \vphi(p^{e})}\Phi_{p^{e}}'(\w_{i}) = \pm\Norm_{\Q(\w)/\Q}(\Phi_{p^{e}}'(\w)).
      \]
      It remains to show $\Norm_{\Q(\w)/\Q}(\Phi_{p^{e}}'(\w)) = \pm p^{p^{(e-1)(ep-e-1)}}$. To this end, \cref{equ:cyclotomic_for_prime_power} implies
      \[
        (x^{p^{e-1}}-1)\Phi_{p^{e}}(x) = x^{p^{e}}-1,
      \]
      and differentiating gives
      \[
        \left(p^{e-1}-1\right)x^{p^{e-1}-1}\Phi_{p^{e}}(x)+\left(x^{p^{e-1}}-1\right)\Phi_{p^{e}}'(x) = p^{e}x^{p^{e}-1}.
      \]
      Now set $x = \w$ and let $\xi = \w^{p^{e-1}}$ to obtain
      \[
        \left(\xi-1\right)\Phi_{p^{e}}'(\w) = p^{e}\w^{-1},
      \]
      where $\xi$ is a primitive $p$-th root of unity. As $\Norm_{\Q(\xi)/\Q}(1-\xi) = p$ from our previous work, we compute
      \begin{align*}
        \Norm_{\Q(\w)/\Q}(1-\xi) &= \prod_{k \in (\Z/p^{e}\Z)^{\ast}}(1-\xi^{k}) \\
        &= \w^{p+2p+\cdots+(p^{e-1}-1)p}\left(\prod_{k \in (\Z/p\Z)^{\ast}}(1-\xi^{k})\right)^{p^{e-1}} \\
        &= \w^{\frac{p^{n}(p^{n-1}-1)}{2}}\left(\prod_{k \in (\Z/p\Z)^{\ast}}(1-\xi^{k})\right)^{p^{e-1}} \\
        &= \left(\prod_{k \in (\Z/p\Z)^{\ast}}(1-\xi^{k})\right)^{p^{e-1}} \\
        &= \Norm_{\Q(\xi)/\Q}(1-\xi)^{p^{e-1}} \\
        &= p^{p^{e-1}},
      \end{align*}
      where the first and second to last equalities follow by \cref{prop:formulas_for_trace_and_norm} since $\Q(\w)/\Q$ and $\Q(\xi)/\Q$ are Galois. Thus $\Norm_{\Q(\w)/\Q}(\xi-1) = \pm p^{p^{e-1}}$. Our previous identity is equivalent to
      \[
        \Phi_{p^{e}}'(\w) = \frac{p^{e}\w^{-1}}{\left(\xi-1\right)},
      \]
      and multiplicativity of the norm together with \cref{prop:unit_if_and_only_if_AKBL} give
      \[
        \Norm_{\Q(\w)/\Q}(\Phi_{p^{e}}'(\w)) = \frac{p^{\vphi(p^{e})e}\Norm_{\Q(\w)/\Q}(\w^{-1})}{\Norm_{\Q(\w)/\Q}(\xi-1)} = \pm p^{\vphi(p^{e})e-p^{e-1}}.
      \]
      This completes the proof.
    \end{proof}

    Note that \cref{lem:prime_power_cyclotomic_lemma} says $\mf{p}_{\w}$ is totally ramified. With \cref{lem:prime_power_cyclotomic_lemma} we can prove $\mc{O}_{\w}$ is monogenic in full generality:

    \begin{proposition}\label{prop:cyclotomic_is_monogenic}
      Let $\Q(\w)$ be the cyclotomic number field generated by a primitive $n$-th root of unity $\w$. Then $\Q(\w)$ is monogenic where
      \[
        \mc{O}_{\w} = \Z[\w].
      \]
    \end{proposition}
    \begin{proof}
      The claim is trivial when $n = 1$ so assume $n \ge 2$. We will now prove the claim when $n = p^{e}$ for a prime $p$ and $e \ge 1$. By \cref{lem:prime_power_cyclotomic_lemma}, $1,\w,\ldots,\w^{\vphi(p^{e})-1}$ is a basis for $\Q(\w)/\Q$ and
      \[
        d_{\Q(\w)/\Q}(1,\w,\ldots,\w^{\vphi(p^{e})-1}) = \pm p^{\vphi(p^{e})e-p^{e-1}}.
      \]
      Then \cref{lem:lemma_for_integral_basis_AKBL} implies
      \[
        p^{\vphi(p^{e})e-p^{e-1}}\mc{O}_{\w} \subseteq \Z[\w] \subseteq \mc{O}_{\w}.
      \]
      Moreover, $\F_{\mf{p}_{\w}} \cong \F_{p}$ since $\mf{p}_{\w}$ is a prime above $p$ with $f_{p}(\mf{p}_{\w}) = 1$ by \cref{lem:prime_power_cyclotomic_lemma}. Therefore $\mc{O}_{\w} = \Z+\mf{p}_{\w}$ which implies
      \[
        \mc{O}_{\w} = \Z[\w]+\mf{p}_{\w}.
      \]
      Multiplying by $1-\w$ gives $\mf{p}_{\w} = (1-\w)\Z[\w]+\mf{p}_{\w}^{2}$. Combining with the previous identity results in
      \[
        \mc{O}_{\w} = \Z[\w]+\mf{p}_{\w}^{2},
      \]
      because $(1-\w)\Z[\w] \subseteq \Z[\w]$. Iterating this procedure gives
      \[
        \mc{O}_{\w} = \Z[\w]+\mf{p}_{\w}^{t},
      \]
      for any $t \ge 1$. Taking $t = \vphi(p^{e})(\vphi(p^{e})e-p^{e-1})$ shows that
      \[
        \mc{O}_{\w} = \Z[\w]+p^{\vphi(p^{e})e-p^{e-1}}\mc{O}_{\w} = \Z[\w],
      \]
      because $p\mc{O}_{\w} = \mf{p}_{\w}^{\vphi(p^{e})}$ by \cref{lem:prime_power_cyclotomic_lemma} and $p^{\vphi(p^{e})e-p^{e-1}}\mc{O}_{\w} \subseteq \Z[\w]$. This proves the claim in the case $n$ is a prime power. For the general case, let $n = p_{1}^{e_{1}} \cdots p_{r}^{e_{r}}$ be the prime factorization of $n$. Then $\w_{i} = \w^{\frac{n}{ p_{i}^{e_{i}}}}$ is a primitive $p_{i}^{e_{i}}$-th root of unity for $1 \le i \le r$ and $\w = \w_{1} \cdots \w_{r}$. This factorization of $\w$ implies
      \[
        \Q(\w) = \Q(\w_{1}) \cdots \Q(\w_{r}),
      \]
      and since $p_{1}^{e_{1}},\ldots,p_{r}^{e_{r}}$ are pairwise relatively prime, we have
      \[
        \Q(\w_{1}) \cdots \Q(\w_{i-1}) \cap \Q(\w_{i}) = \Q,
      \]
      for all $i$ (since the degree of $(\Q(\w_{1}) \cdots \Q(\w_{i-1}) \cap \Q(\w_{i}))/\Q$ must divide both $ p_{1}^{e_{1}} \cdots p_{i-1}^{e_{i-1}}$ and $p_{i}^{e_{i}}$ and thus is $1$). This also implies $\Q(\w_{1}) \cdots \Q(\w_{i-1})$ and $\Q(\w_{i})$ are linearly disjoint over $\Q$ in $\conj{\Q}$ for all $i$ by \cref{prop:Galois_linearly_disjoint} because $\Q(\w_{1}) \cdots \Q(\w_{i-1})/\Q$ and $\Q(\w_{i})/\Q$ are both Galois. As the discriminants of $d_{\Q(\w_{1})/\Q}(1,\w_{1},\ldots,\w_{1}^{\vphi(p_{1}^{e_{1}})-1}),\ldots,d_{\Q(\w_{r})/\Q}(1,\w_{r},\ldots,\w_{r}^{\vphi(p_{r}^{e_{r}})-1})$ are pairwise relatively prime, successive applications of \cref{prop:linearly_disjoint_integral_basis} shows that $1,\w,\ldots,\w^{\vphi(n)-1}$ is an integral basis for $\Q(\w)$. This means
      \[
        \mc{O}_{\w} = \Z[\w],
      \]
      as desired.
    \end{proof}

    We can now leverage the Dedekind-Kummer theorem to prove how $p\mc{O}_{\w}$ decomposes in $\mc{O}_{\w}$ for any prime $p$. We setup some notation to do this. For every prime $p$, let $e_{p} \ge 0$ satisfy $p^{e_{p}} \mid\mid n$ and let $f_{p} \ge 1$ be the smallest positive integer such that
      \[
        p^{f_{p}} \equiv 1 \bmod{\frac{n}{p^{e_{p}}}}.
      \]
    Then we have the following result:

    \begin{proposition}\label{prop:factorization_of_primes_cyclotomic}
      Let $\Q(\w)$ be the cyclotomic number field generated by a primitive $n$-th root of unity $\w$. If $\mf{p}_{1},\ldots,\mf{p}_{r}$ are the prime factors of $p\mc{O}_{\w}$ then
      \[
        p\mc{O}_{\w} = (\mf{p}_{1} \cdots \mf{p}_{r})^{\vphi(p^{e_{p}})},
      \]
      and
      \[
        f_{p}(\mf{p}_{1}) = \cdots = f_{p}(\mf{p}_{r}) = f_{p}.
      \]
    \end{proposition}
    \begin{proof}
      Since $\Q(\w)$ is monogenic by \cref{prop:cyclotomic_is_monogenic}, the conductor of $\Q(\w)$ relative to $\w$ is $\mc{O}_{\w}$. Therefore we may apply the Dedekind-Kummer theorem to every prime $p$ and as $\Phi_{n}(x)$ is the minimal polynomial for $\w$ over $\Q$, we simply have to show the prime factorization
      \[
        \conj{\Phi_{n}}(x) = (\conj{m_{1}}(x) \cdots \conj{m_{r}}(x))^{\vphi(p^{e_{p}})},
      \]
      in $\F_{p}[x]$ for distinct irreducibles $\conj{m_{1}}(x),\ldots,\conj{m_{r}}(x)$ of degree $f_{p}$. To this end, let $n = p^{e}m$ and let $\xi$ and $\eta$ be primitive $p^{e}$-th and $m$-th roots of unity respectively. Then $\Phi_{n}(x)$ can be expressed as
      \[
        \Phi_{n}(x) = \prod_{\substack{k \in (\Z/p^{e}\Z)^{\ast} \\ \ell \in (\Z/m\Z)^{\ast}}}(x-\xi^{k}\eta^{\ell}).
      \]
      Recall that
      \[
        (x-1)^{p^{e_{p}}} \equiv x^{p^{e_{p}}}-1 \tmod{p}.
      \]
      Taking $x = \xi$ gives $(\xi-1)^{p^{e_{p}}} \equiv 0 \tmod{p}$ and thus $\xi \equiv 1 \tmod{p}$. Then from our expression of $\Phi_{n}(x)$, we find that
      \[
        \Phi_{n}(x) \equiv \prod_{1 \le \ell \le (\Z/m\Z)^{\ast}}(1-\eta^{\ell})^{\vphi(p^{e})} \equiv \Phi_{m}(x)^{\vphi(p^{e_{p}})} \tmod{p}.
      \]
      This is to say
      \[
        \conj{\Phi_{n}}(x) = \conj{\Phi_{m}}(x)^{\vphi(p^{e_{p}})}.
      \]
      Therefore it suffices to show
      \[
        \conj{\Phi_{m}}(x) = \conj{m_{1}}(x) \cdots \conj{m_{r}}(x),
      \]
      and that $\conj{m_{1}}(x),\ldots,\conj{m_{r}}(x)$ are all of degree $f_{p}$. Now let $\mf{p}$ be a prime above $p$. Then $\F_{\mf{p}}$ is an $\F_{p}$-vector space and therefore has characteristic $p$. It follows that $x^{m}-1$ has $m$ distinct roots in in $\conj{\F_{p}}$ since $p$ does not divide $m$ (recall that the derivative of $x^{m}-1$ is $mx^{m-1}$ which is relatively prime to $p$). Therefore $x^{m}-1$ does not have a multiple root in $\F_{\mf{p}}$. As $\mc{O}_{\w}$ contains all the $m$-th roots of unity by \cref{prop:cyclotomic_is_monogenic} and that $m$ divides $n$, it follows that $\F_{\mf{p}}$ does as well so the surjective homomorphism
      \[
        \pi:\mc{O}_{\w} \to \mc{O}_{\w}/\mf{p} \qquad \a \mapsto \a \tmod{\mf{p}},
      \]
      maps the primitive $m$-th roots of unity onto themselves bijectively. This implies that the roots of $\conj{\Phi_{m}}(x)$ are exactly the primitive $m$-th roots of unity. Now the smallest extension of $\F_{p}$ containing all the primitive $m$-th roots of unity is $\F_{p^{f_{p}}}$ because its multiplicative group $\F_{p^{f_{p}}}^{\ast}$ is cyclic (as the multiplicative group of any finite field is cyclic), of order $p^{f_{p}}-1$ dividing $m$ by assumption, and with $f_{p}$ minimal. As $\conj{\Phi_{m}}(x)$ divides $x^{n}-1$ in $\F_{p}[x]$ (because $\Phi_{m}(x)$ divides $x^{m}-1$ in $\Z[x]$), $\conj{\Phi_{m}}(x)$ has no multiple roots in $\F_{p}$. Therefore it factors as
      \[
        \conj{\Phi_{m}}(x) = \conj{m_{1}}(x) \cdots \conj{m_{r}}(x),
      \]
      in $\F_{p}[x]$ for distinct irreducibles $\conj{m_{1}}(x),\ldots,\conj{m_{r}}(x)$. These are also necessarily monic because $\Phi_{m}(x)$, and hence $\conj{\Phi_{m}}(x)$, is. Moreover, as the roots of $\conj{\Phi_{m}}(x)$ are the primitive $m$-th roots of unity, each factor $\conj{m_{i}}(x)$ is the minimal polynomial of a primitive $m$-th root of unity in $\F_{p^{f_{p}}}$ for $1 \le i \le r$. The degree of this minimal polynomial is necessarily the degree of $\F_{p^{f_{p}}}/\F_{p}$ which is $f_{p}$. Therefore $\conj{m_{1}}(x),\ldots,\conj{m_{r}}(x)$ are all of degree $f_{p}$ completing the proof.
    \end{proof}