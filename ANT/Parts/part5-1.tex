\chapter{Applications of Dirichlet \texorpdfstring{$L$}{L}-functions}
  We discuss some classical applications of Dirichlet $L$-functions. Our first result is a gem of analytic number theory: Dirichlet's theorem on arithmetic progressions. This is a consequence of a non-vanishing result for Dirichlet $L$-series at $s = 1$. Next we discuss Siegel's theorem and the existence of Siegel zeros in the case of Dirichlet $L$-functions. After these two results, we deduce estimates for the class number of quadratic number fields. Then we prove a subconvexity estimate of Burgess about Dirichlet $L$-functions.
  \section{Dirichlet's Theorem on Arithmetic Progressions}
    One of the more well-known arithmetic results proved using $L$-series is \textbf{Dirichlet's theorem on arithmetic progressions}\index{Dirichlet's theorem on arithmetic progressions}:

    \begin{theorem*}[Dirichlet's theorem on arithmetic progressions]\label{thm:Dirichlet's_theorem_on_primes_in_arithmetic_progressions}
      Let $a$ and $m$ be integers such that $m \ge 1$ and $(a,m) = 1$. Then the series
      \[
        \sum_{p \equiv a \tmod{m}}\frac{1}{p},
      \]
      diverges. In particular, the arithmetic progression $\{a+km \mid k \in \Z_{\ge 0}\}$ contain infinitely many primes.
    \end{theorem*}

    The proof can be broken into three steps. The first is to estimate the logarithm of a Dirichlet $L$-series $L(s,\chi)$ and show that almost all of the terms are uniformly bounded as $s \to 1$. The next step is to use the Dirichlet orthogonality relations of the characters to sieve out the correct sum. The last step is to show the non-vanishing result $L(1,\chi) \neq 0$ for all non-principal characters $\chi$. This is the essential part of the proof as it is what assures us that the sum diverges. Fortunately, we have done all of the difficult work to prove this already:

    \begin{theorem}\label{thm:non-vanishing_of_Dirichlet_L-functions_at_s=1}
      Let $\chi$ be a non-principal Dirichlet character. Then $L(1,\chi)$ is finite and nonzero.
    \end{theorem}
    \begin{proof}
      This follows immediately by applying \cref{lem:non-vanshing_at_1_lemma} to $\z(s)L(s,\chi)$ and noting that $L(s,\chi)$ is holomorphic.
    \end{proof}

     We now prove Dirichlet's theorem on arithmetic progressions:

    \begin{proof}[Proof of Dirichlet's theorem on arithmetic progressions]
        Let $\chi$ be a Dirichlet character modulo $m$. Then for $\s > 1$, taking the logarithm of the Euler product of $L(s,\chi)$ gives
        \[
          \log{L(s,\chi)} = -\sum_{p}\log(1-\chi(p)p^{-s}).
        \]
        The Taylor series of the logarithm implies
        \[
          \log(1-\chi(p)p^{-s}) = \sum_{k \ge 1}(-1)^{k-1}\frac{(-\chi(p)p^{-s})^{k}}{k} = \sum_{k \ge 1}(-1)^{2k-1}\frac{\chi(p^{k})}{kp^{ks}},
        \]
        so that
        \begin{equation}\label{equ:Dirichlets_theorem_1}
          \log{L(s,\chi)} = \sum_{p}\sum_{k \ge 1}\frac{\chi(p^{k})}{kp^{ks}}.
        \end{equation}
        The double sum restricted to $k \ge 2$ is uniformly bounded for $\s > 1$. Indeed, first observe
        \[
          \left|\sum_{k \ge 2}\frac{\chi(p^{k})}{kp^{ks}}\right| \ll \sum_{k \ge 2}\frac{1}{p^{k}} = \frac{1}{p^{2}}\sum_{k \ge 0}\frac{1}{p^{k}} = \frac{1}{p^{2}}(1-p^{-1})^{-1} \le \frac{2}{p^{2}},
        \]
        where the last inequality follows because $p > 2$. Then
        \[
          \left|\sum_{p}\sum_{k \ge 2}\frac{\chi(p^{k})}{kp^{ks}}\right| \le 2\sum_{p}\frac{1}{p^{2}} < 2\sum_{n \ge 1}\frac{1}{n^{2}} = 2\z(2),
        \]
        as desired. Now let $a$ and $m$ be integers such that $m \ge 1$ and $(a,m) = 1$. Using \cref{equ:Dirichlets_theorem_1}, we may write
        \[
          \sum_{\chi \tmod{m}}\cchi(a)\log{L(s,\chi)} = \sum_{\chi \tmod{m}}\sum_{p}\frac{\cchi(a)\chi(p)}{p^{s}}+\sum_{\chi \tmod{m}}\cchi(a)\sum_{p}\sum_{k \ge 2}\frac{\chi(p^{k})}{kp^{ks}},
        \]
        and by the Dirichlet orthogonality relations (namely (ii)), we have
        \[
          \sum_{\chi \tmod{m}}\sum_{p}\frac{\cchi(a)\chi(p)}{p^{s}} = \sum_{p}\frac1{p^{s}}\sum_{\chi \tmod{m}}\cchi(a)\chi(p) = \vphi(m)\sum_{p \equiv{a} \tmod{m}}\frac{1}{p^{s}}.
        \]
        Combining these two identities together gives
        \[
          \sum_{\chi \tmod{m}}\cchi(a)\log{L(s,\chi)}-\sum_{\chi \tmod{m}}\cchi(a)\sum_{p}\sum_{k \ge 2}\frac{\chi(p^{k})}{kp^{ks}} = \vphi(m)\sum_{p \equiv{a} \tmod{m}}\frac{1}{p^{s}}.
        \]
        The triple sum in this identity is uniformly bounded for $\s > 1$ because the inner double sum is and there are finitely many Dirichlet characters modulo $m$. Therefore it suffices to show that the first sum on the left-hand side diverges as $s \to 1$. We have
        \[
          L(s,\chi_{m,0}) = \z(s)\prod_{p \mid m}(1-p^{-s}),
        \]
        for $\chi = \chi_{m,0}$, and so the corresponding term in the sum is
        \[
          \conj{\chi_{m,0}}(a)\log L(s,\chi_{m,0}) = \log\z(s)+\sum_{p \mid m}\log(1-p^{-s}),
        \]
        which diverges as $s \to 1$ because $\z(s)$ has a simple pole at $s = 1$. We will be done if $\log{L(s,\chi)}$ remains bounded as $s \to 1$ for all $\chi \neq \chi_{m,0}$. So assume $\chi$ is non-principal. Then we must show $L(1,\chi)$ is finite and nonzero. This follows from \cref{thm:non-vanishing_of_Dirichlet_L-functions_at_s=1}.
    \end{proof}
  \section{Siegel's Theorem and Siegel Zeros}
    The discussion of Siegel zeros first arose during the study of zero-free regions for Dirichlet $L$-functions. Refining the argument used in \cref{thm:zero_free_region_generic}, we can show that Siegel zeros only exist when the character $\chi$ is quadratic. But first we improve the zero-free region for the Riemann zeta function:

    \begin{theorem}\label{thm:improved_zero-free_region_zeta}
      There exists a constant $c > 0$ such that $\z(s)$ has no zeros in the region
      \[
        \s \ge 1-\frac{c}{\log(|t|+3)}.
      \]
    \end{theorem}
    \begin{proof}
      By \cref{thm:zero_free_region_generic} applied to $\z(s)$, it suffices to show that $\z(s)$ has no real nontrivial zeros. To see this, let $\eta(s)$ be defined by
      \[
        \eta(s) = \sum_{n \ge 1}\frac{(-1)^{n-1}}{n^{s}}.
      \]
      Note that $\eta(s)$ converges for $\s > 0$ by \cref{prop:Dirichlet_series_convergence_bounded_coefficient_sum}. Now for $0 < s < 1$ and even $n$, $\frac{1}{n^{s}}-\frac{1}{(n+1)^{s}} > 0$ so that $\eta(s) > 0$. But for $\s > 0$, we have
      \[
        (1-2^{1-s})\z(s) = \sum_{n \ge 1}\frac{1}{n^{s}}-2\sum_{n \ge 1}\frac{1}{(2n)^{s}} = \sum_{n \ge 1}\frac{(-1)^{n-1}}{n^{s}} = \eta(s).
      \]
      Therefore $\z(s)$ cannot admit a zero for $0 < s < 1$ because then $\eta(s)$ would be zero too. This completes the proof.
    \end{proof}

    \cref{thm:improved_zero-free_region_zeta} shows that $\z(s)$ has no Siegel zeros. Moreover, the proof shows that $\z(s)$ is negative for $0 < s < 1$. As for the height of the first zero, it occurs on the critical line (as predicted by the Riemann hypothesis) at height $t \approx 14.134$ (see \cite{davenport1980multiplicative} for a further discussion). The first $15$ zeros were computed by Gram in 1903 (see \cite{gram1903note}). Since then, billions of zeros have been computed and have all been verified to lie on the critical line. Our improved zero-free region for Dirichlet $L$-functions is only slightly different but causes increasing complexity in further study. The improvement is that if a Siegel zero exists for a Dirichlet $L$-function then the character is necessarily quadratic:

    \begin{theorem}\label{thm:improved_zero-free_region_Dirichlet}
      Let $\chi$ be a primitive Dirichlet character of conductor $q > 1$. Then there exists a constant $c > 0$ such that $L(s,\chi)$ has no zeros in the region
      \[
        \s \ge 1-\frac{c}{\log(q(|t|+3))},
      \]
      except for possibly one simple real zero $\b_{\chi}$ with $\b_{\chi} < 1$ in the case $\chi$ is quadratic.
    \end{theorem}
    \begin{proof}
      By \cref{thm:zero_free_region_generic} applied to $\z(s)L(s,\chi)$, and shrinking $c$ if necessary, it remains to show that there not a simple real zero $\b_{\chi}$ if $\chi$ is not quadratic. For this, let $L(s,f)$ be the $L$-series defined by
      \[
        L(s,f) = L^{3}(s,\chi_{q,0})L^{4}(s,\chi)L(s,\chi^{2}).
      \]
      Then $d_{f} = 8$ and $\mathfrak{q}(f)$ satisfies
      \[
        \mathfrak{q}(f) \le \mathfrak{q}(\chi_{q,0})^{3}\mathfrak{q}(\chi)^{4}\mathfrak{q}(\chi^{2}) \le 3^{8}q^{5} < (3q)^{8}.
      \]
      Moreover, $\Re(\L_{g}(n)) \ge 0$ for $(n,q) = 1$. To see this, suppose $p$ is an unramified prime. The local roots of $L(s,f)$ at $p$ are $1$ with multiplicity three, $\chi(p)$ with multiplicity four, and $\chi^{2}(p)$ with multiplicity one. So for any $k \ge 1$, the sum of $k$-th powers of these local roots is
      \[
        3+4\chi^{k}(p)+\chi^{2k}(p).
      \]
      Writing $\chi(p) = e^{ix}$, the real part of this expression is
      \[
        3+4\cos(x)+\cos(2x) = 2(1+\cos(x))^{2} \ge 0,
      \]
      where we have also made use of the identity $3+4\cos(x)+\cos(2x) = 2(1+\cos(x))^{2}$. Thus $\Re(\L_{f}(n)) \ge 0$ for $(n,q) = 1$, and the conditions of \cref{lem:non-vanshing_at_1_lemma} are satisfied for $L(s,f)$ (recall \cref{equ:non-primitive_primitive_Dirichlet_L-series_relation} for the $L$-series $L(s,\chi_{q,0})$ and $L(s,\chi^{2})$). On the one hand, if $\b$ be a real nontrivial zero of $L(s,\chi)$ then $L(s,f)$ has a real nontrivial zero at $s = \b$ of order at least $4$. On the other hand, using \cref{equ:non-primitive_primitive_Dirichlet_L-series_relation} and that $\chi^{2} \neq \chi_{q,0}$, $L(s,f)$ has a pole at $s = 1$ of order $3$. Then, upon shrinking $c$ if necessary, \cref{lem:non-vanshing_at_1_lemma} gives a contradiction since $r_{g} = 3$. This completes the proof.
    \end{proof}

    Siegel zeros present an unfortunate obstruction to zero-free region results for Dirichlet $L$-functions when the primitive character $\chi$ is quadratic. However, if we no longer require the constant $c$ in the zero-free region to be effective, we can obtain a much better result for how close the Siegel zero can be to $1$. This result will follow from a lower bound for $L(1,\chi)$ which significantly improves upon \cref{equ:lower_bound_at_1_Dirichlet_real_character}. The result is known as \textbf{Siegel's theorem}\index{Siegel's theorem}:

    \begin{theorem*}[Siegel's theorem]
      Let $\chi$ be a primitive quadratic Dirichlet character $\chi$ of conductor $q > 1$. Then there exists a positive constant $c_{1}(\e)$ such that
      \[
        L(1,\chi) \ge \frac{c_{1}(\e)}{q^{\e}}.
      \]
      In particular, there exists a positive constant $c_{2}(\e)$ such that $L(s,\chi)$ has no real zeros in the interval
      \[
        \s \ge 1-\frac{c_{2}(\e)}{q^{\e}}.
      \]
    \end{theorem*}

    The largest defect of Siegel's theorem is that the implicit constants $c_{1}(\e)$ and $c_{2}(\e)$ are ineffective (and not necessarily equal). To prove Siegel's theorem, we we will require two lemmas. The first is about the size of $L^{(k)}(\s,\chi)$ for $\s$ close to $1$:

    \begin{lemma}\label{lem:log_growth_of_Dirichlet_L-series_near_1}
      Let $\chi$ be a non-principal Dirichlet character modulo $m > 1$. Then for any for any $k \ge 0$, we have $L^{(k)}(\s,\chi) = O(\log^{k+1}(m))$ provided $\s$ is such that $0 \le 1-\s \le \frac{1}{\log(m)}$.
    \end{lemma}
    \begin{proof}
      Recall that the Dirichlet $L$-series $L(s,\chi)$ converges for $\s > 0$ and thus for $0 \le 1-\s \le \frac{1}{\log(m)}$. As $L(s,\chi)$ admits analytic continuation to $\C$, it is holomorphic for $0 \le 1-\s \le \frac{1}{\log(m)}$ and so its $k$-th derivative is given by
      \[
        L^{(k)}(\s,\chi) = \sum_{n \ge 1}\frac{\chi(n)\log^{k}(n)}{n^{\s}} = \sum_{n < m}\frac{\chi(n)\log^{k}(n)}{n^{\s}}+\sum_{n \ge m}\frac{\chi(n)\log^{k}(n)}{n^{\s}}.
      \]
      We will show that the last two sums are both $O(\log^{k+1}(m))$. For the first sum, if $n < m$, we have
      \[
        \left|\frac{\chi(n)\log^{k}(n)}{n^{\s}}\right| \le \frac{1}{n^{\s}}\log^{k}(n) = \frac{n^{1-\s}}{n}\log^{k}(n) < \frac{m^{1-\s}}{n}\log^{k}(n) < \frac{e}{n}\log^{k}(m),
      \]
      where the last inequality follows because $1-\s \le \frac{1}{\log(m)}$. Then
      \[
        \left|\sum_{n \le m}\frac{\chi(n)\log^{k}(n)}{n^{\s}}\right| < e\log^{k}(m)\sum_{n < m}\frac{1}{n} < e\log^{k}(m)\int_{1}^{m}\frac{1}{n}\,dn \ll \log^{k+1}(m).
      \]
      For the second sum, let $A(Y) = \sum_{n \le Y}\chi(n)$. Then $|A(Y)| \le m$ by the Dirichlet orthogonality relations (namely \cref{cor:Dirichlet_orthogonality_relations} (i)) and that $\chi$ is $m$-periodic. Hence $A(Y)\log(Y)Y^{-\s} \to 0$ as $Y \to \infty$ and Abel's summation formula (see \cref{cor:Abels_summation_formula_limit_version}) gives
      \begin{equation}\label{equ:Siegels_theorem_second_version_lemma_1}
        \sum_{n \ge m}\frac{\chi(n)\log^{k}(n)}{n^{\s}} = -A(m)\log^{k}(m)m^{-\s}-\int_{m}^{\infty}A(u)(k-\s\log(u))\log^{k-1}(u)u^{-(\s+1)}\,du.
      \end{equation}
      Since $0 \le 1-\s \le \frac{1}{\log(m)}$, we have $k-\s\log(u) \ll \log(u)$. With this estimate, that $|A(u)| \le m$, and \cref{equ:Siegels_theorem_second_version_lemma_1}, we make the following computation:
      \begin{align*}
        \left|\sum_{n \ge m}\frac{\chi(n)\log^{k}(n)}{n^{\s}}\right| &\le |A(m)|\log^{k}(m)m^{-\s}+\int_{m}^{\infty}|A(u)(k-\s\log(u))|\log^{k-1}(u)u^{-(\s+1)}\,du \\
        &\ll m^{1-\s}\log^{k}(m)+m\int_{m}^{\infty}\log^{k}(u)u^{-(\s+1)}\,du \\
        &= m^{1-\s}\log^{k}(m)+\frac{m}{\s}\left(\log^{k}(m)m^{-\s}+\int_{m}^{\infty}\log^{k-1}(u)u^{-(\s+1)}\,du\right) \\
        &\ll m^{1-\s}\log^{k}(m)+\frac{m^{1-\s}}{\s}\log^{k}(m) \\
        &\ll \log^{k+1}(m),
      \end{align*}
      where in the third line we have used integration by parts, in the fourth we have used that the remaining integral is bounded by the former, and in the last line we have used that $1-\s \le \frac{1}{\log(m)}$ (which implies that $\frac{1}{\s} \ll \log(m)$). Therefore the second sum is $O(\log^{k+1}(m))$ and hence $L^{(k)}(\s,\chi) = O(\log^{k+1}(m))$.
    \end{proof}

    Our second lemma is concerned with the positivity of the coefficients of a Dirichlet series formed by combining two Dirichlet $L$-series attached to distinct quadratic characters with distinct moduli:

    \begin{lemma}\label{lem:Siegel_zero_auxiliary_L-function_lemma}
      Let $\chi_{1}$ and $\chi_{2}$ be quadratic Dirichlet characters and let $L(s,g)$ be the $L$-series defined by
      \[
        L(s,g) = \z(s)L(s,\chi_{1})L(s,\chi_{2})L(s,\chi_{1}\chi_{2}).
      \]
      Then $\L_{g}(n) \ge 0$. In particular, $a_{g}(n) \ge 0$ and $a_{g}(0) = 1$.
    \end{lemma}
    \begin{proof}
      For any prime $p$, the local roots at $p$ are $1$ with multiplicity one, $\chi_{1}(p)$ with multiplicity one, $\chi_{2}(p)$ with multiplicity one, and $\chi_{1}\chi_{2}(p)$ with multiplicity one. So for any $k \ge 1$, the sum of $k$-th powers of these local roots is
      \[
        (1+\chi_{1}^{k}(p))(1+\chi_{2}^{k}(p)) \ge 0.
      \]
      Thus $\L_{g}(n) \ge 0$. It follows immediately from the Euler product of $L(s,g)$ that $a_{g}(n) \ge 0$ too. Also, it is clear from the Euler product of $L(s,g)$ that $a_{g}(0) = 1$.
    \end{proof}

    We are now ready to prove Siegel's theorem and we follow a particularly simple proof due to Goldfeld (see \cite{goldfeld1974simple}):

    \begin{proof}[Proof of Siegel's theorem]
      Observe that the first statement holds holds for a single $q$ and hence for bounded $q$ by taking the minimum of all the constants. Therefore we may assume $q$ arbitrarily large throughout. We may also assume $\e \le \frac{1}{4}$ because $\frac{1}{q^{\e}}$ is a decreasing function of $\e$. We now prove that the first statement implies the second by contradiction. For if there was a real zero $\b$ in the desired interval then for large enough $q$ we have $0 \le 1-\b \le \frac{1}{\log(q)}$ so that $L'(\s,\chi) = O(\log^{2}(q))$ for $\b \le \s \le 1$ by \cref{lem:log_growth_of_Dirichlet_L-series_near_1}. These two estimates and the mean value theorem together give
      \[
        L(1,\chi) = L(1,\chi)-L(\b,\chi) = L'(\s,\chi)(1-\b) \ll \frac{\log^{2}(q)}{q^{\e}}.
      \]
      Upon taking $\frac{\e}{2}$ in the lower bound, we obtain
      \[
        \frac{1}{q^{\frac{\e}{2}}} \ll L(1,\chi) \ll \frac{\log^{2}(q)}{q^{\e}},
      \]
      which is a contradiction. Therefore the second statement holds provided the first does. We will now prove the first statement which will complete the proof. Let $\chi_{1}$ and $\chi_{2}$ be distinct primitive quadratic characters of conductors $q_{1} > 1$ and $q_{2} > 1$ respectively. Consider the $L$-series $L(s,g)$ defined by
      \[
        L(s,g) = \z(s)L(s,\chi_{1})L(s,\chi_{2})L(s,\chi_{1}\chi_{2}).
      \]
      Observe that $L(s,g)$ is holomorphic except for a simple pole at $s = 1$. Let $\l$ be the residue at this pole so that
      \[
        \l = L(1,\chi_{1})L(1,\chi_{2})L(1,\chi_{1}\chi_{2}).
      \]
      If there exists a Siegel zero $\b$ with $1-\e \le \b < 1$, let $\chi_{1}$ be the character corresponding to the Dirichlet $L$-function that admits this Siegel zero. Then $L(\b,g) = 0$ independent of the choice of $\chi_{2}$. If there is no such Siegel zero, choose $\chi_{1}$ to be any quadratic primitive character and $\b$ to be any number such that $1-\e \le \b < 1$. Then $L(\b,g) < 0$ independent of the choice of $\chi_{2}$. Indeed, $\z(s)$ is negative in this interval (actually $0 < s < 1$) while each of the Dirichlet $L$-series defining $L(s,g)$ is positive in this interval (the Dirichlet $L$-series converge for $\s > 0$, are positive for $\s > 1$ by their Euler product, are nonzero at $s = 1$ by \cref{thm:non-vanishing_of_Dirichlet_L-functions_at_s=1}, and do not admit a real zero in this interval by our choice of $\b$). So with our choice of $\chi_{1}$ (depending on the existence of a Siegel zero or not) we see that $L(\b,g) < 0$ for any choice of $\chi_{2}$. We now take $\chi_{2} = \chi$. Let $\psi(y)$ be a bump function that is identically $1$ on $[0,1]$ and compactly supported. Let $\Psi(s)$ be the Mellin transform of $\psi(y)$. On the one hand, our choice of $\psi(y)$ and \cref{lem:Siegel_zero_auxiliary_L-function_lemma} together imply
      \begin{equation}\label{equ:Siegels_theorem_1}
        \sum_{n \ge 1}\frac{a_{g}(n)}{n^{\b}}\psi\left(\frac{n}{X}\right) \ge 1,
      \end{equation}
      for any $X \ge 1$. On the other hand, smoothed Perron's formula gives
      \[
        \sum_{n \ge 1}\frac{a_{g}(n)}{n^{\b}}\psi\left(\frac{n}{X}\right) = \frac{1}{2\pi i}\int_{(c)}L(s+\b,g)\Psi(s)X^{s}\,ds,
      \]
      for any $c > 1-\b$. Shifting the line of integration to $\left(\frac{1}{2}-\b\right)$, we pass by a simple pole at $s = 1-\b$ from $L(s+\b,g)$ and obtain
      \begin{equation}\label{equ:Siegels_theorem_2}
        \sum_{n \ge 1}\frac{a_{g}(n)}{n^{\b}}\psi\left(\frac{n}{X}\right) = \frac{1}{2\pi i}\int_{\left(\frac{1}{2}-\b\right)}L(s+\b,g)\Psi(s)X^{s}\,ds+\l\Psi(1-\b)X^{1-\b}.
      \end{equation}
      By the convexity bound and \cref{prop:smoothing_function_Mellin_inverse_vertical_strips}, the remaining integral satisfies the weak bound $O(qX^{\frac{1}{2}-\b})$ and we have the estimate $\Psi(1-\b) \ll (1-\b)^{-1}$. These bounds together with \cref{equ:Siegels_theorem_1,equ:Siegels_theorem_2} give
      \[
        1 \ll qX^{\frac{1}{2}-\b}+X^{1-\b}\frac{\l}{1-\b}.
      \]
      As $1-\e \le \b < 1$ and $\e \le \frac{1}{4}$, we can take $X \asymp q^{4}$ to ensure that $qX^{\frac{1}{2}-\b} \ll 1$. With this restriction on $X$, our estimate becomes
      \[
        1 \ll q^{4(1-\b)}\frac{\l}{1-\b}.
      \]
      By \cref{lem:log_growth_of_Dirichlet_L-series_near_1}, $\l \ll \log^{2}(q)L(1,\chi)$ and we have
      \[
        1 \ll q^{4(1-\b)}\frac{\log^{2}(q)L(1,\chi)}{1-\b}.
      \]
      Isolating $L(1,\chi)$ results in
      \[
        \frac{1-\b}{q^{4(1-\b)}\log^{2}(q)} \ll L(1,\chi).
      \]
      But $1-\e \le \b < 1$ implies $0< 1-\b \le \e$ and so
      \[
        \frac{1}{q^{\e}} \ll_{\e} L(1,\chi),
      \]
      where we have used that $\log(q) \ll_{\e} q^{\e}$. This is equivalent to the desired lower bound.
    \end{proof}

    The part of the proof in Siegel's theorem which makes $c_{1}(\e)$ and $c_{2}(\e)$ ineffective is the value of $\b$. The choice of $\b$ depends upon the existence of a Siegel zero near $1$ relative to the given $\e$. Since we don't know if Siegel zeros exist, this makes estimating $\b$ relative to $\e$ ineffective and hence the constants $c_{1}(\e)$ and $c_{2}(\e)$ as well. Many results in analytic number theory make use of Siegel's theorem and hence also rely on ineffective constants. Moreover, some important problems investigate methods to get around using Siegel's theorem in favor of a weaker result that is effective. So far, no Siegel zero has been shown to exist or not exist for Dirichlet $L$-functions. But some progress has been made to showing that they are rare:

    \begin{proposition}\label{prop:product_of_quadratic_Dirichlet_L-functions_has_one_zero}
      Let $\chi_{1}$ and $\chi_{2}$ be distinct quadratic Dirichlet characters of conductors $q_{1}$ and $q_{2}$. If $L(s,\chi_{1})$ and $L(s,\chi_{2})$ have Siegel zeros $\b_{1}$ and $\b_{2}$ respectively and $\chi_{1}\chi_{2}$ is not principal then there exists a positive constant $c$ such that
      \[
        \min(\b_{1},\b_{2}) < 1-\frac{c}{\log(q_{1}q_{2})}.
      \]
    \end{proposition}
    \begin{proof}
        We may assume $\chi_{1}$ and $\chi_{2}$ are primitive since if $\wtilde{\chi_{i}}$ is the primitive character inducing $\chi_{i}$, for $i = 1,2$, the only difference in zeros between $L(s,\chi_{i})$ and $L(s,\wtilde{\chi}_{i})$ occur on the line $\s = 0$. Now let $\wtilde{\chi}$ be the primitive character of conductor $q$ inducing $\chi_{1}\chi_{2}$. From \cref{equ:non-primitive_primitive_Dirichlet_L-series_relation} with $\chi_{1}\chi_{2}$ in place of $\chi$, we find that
        \begin{equation}\label{equ:product_of_quadratic_Dirichlet_L-functions_has_one_zero_1}
          \left|\frac{L'}{L}(s,\chi_{1}\chi_{2})-\frac{L'}{L}(s,\wtilde{\chi})\right| = \left|\sum_{p \mid q_{1}q_{2}}\frac{\wtilde{\chi}(p)\log(p)p^{-s}}{1-\wtilde{\chi}(p)p^{-s}}\right| \le \sum_{p \mid q_{1}q_{2}}\frac{\log(p)p^{-\s}}{1-p^{-\s}} \le \sum_{p \mid q_{1}q_{2}}\log(p) \le \log(q_{1}q_{2}).
        \end{equation}
        Let $s = \s$ with $1 < \s \le 2$. Using the reverse triangle inequality, we deduce from \cref{equ:product_of_quadratic_Dirichlet_L-functions_has_one_zero_1} that
        \begin{equation}\label{equ:product_of_quadratic_Dirichlet_L-functions_has_one_zero_2}
          -\frac{L'}{L}(\s,\chi_{1}\chi_{2}) < c\log(q_{1}q_{2}),
        \end{equation}
        for some positive constant $c$.
        By \cref{lem:powerful_L-function_approximation_lemma} (iv) applied to $\z(s)$ while discarding all of the terms in both sums, we have
        \begin{equation}\label{equ:product_of_quadratic_Dirichlet_L-functions_has_one_zero_3}
          -\frac{\z'}{\z}(\s) < A+\frac{1}{\s-1},
        \end{equation}
        for some positive constant $A$. By \cref{lem:powerful_L-function_approximation_lemma} (iv) applied to $L(s,\chi_{i})$ and only retaining the term corresponding to $\b_{i}$, we have
        \begin{equation}\label{equ:product_of_quadratic_Dirichlet_L-functions_has_one_zero_4}
          -\frac{L'}{L}(\s,\chi_{i}) < A\log(q_{i})+\frac{1}{\s-\b_{i}},
        \end{equation}
        for $i = 1,2$ and some possibly larger constant $A$. Now let $L(s,g)$ be the $L$-series defined by
        \[
          L(s,g) = \z(s)L(s,\chi_{1})L(s,\chi_{2})L(s,\chi_{1}\chi_{2}),
        \]
        so that $-\frac{L'}{L}(\s,g) \ge 0$ by \cref{lem:Siegel_zero_auxiliary_L-function_lemma}. Combining \cref{equ:product_of_quadratic_Dirichlet_L-functions_has_one_zero_1,equ:product_of_quadratic_Dirichlet_L-functions_has_one_zero_2,equ:product_of_quadratic_Dirichlet_L-functions_has_one_zero_3,equ:product_of_quadratic_Dirichlet_L-functions_has_one_zero_4} with this fact implies
        \[
         0 < A+\frac{1}{\s-1}+A\log(q_{1})-\frac{1}{\s-\b_{1}}+A\log(q_{2})-\frac{1}{\s-\b_{2}}+c\log(q_{1}q_{2}).
        \]
        Taking $c$ larger, if necessary, we arrive at the simplified estimate
        \[
          0 < \frac{1}{\s-1}-\frac{1}{\s-\b_{1}}-\frac{1}{\s-\b_{2}}+c\log(q_{1}q_{2}),
        \]
        which we rewrite as
        \[
          \frac{1}{\s-\b_{1}}+\frac{1}{\s-\b_{2}} < \frac{1}{\s-1}+c\log(q_{1}q_{2}),
        \]
        Now let $\s = 1+\frac{\d}{\log(q_{1}q_{2})}$ for some $\d > 0$. Upon substituting, we have
        \[
          \frac{1}{\s-\b_{1}}+\frac{1}{\s-\b_{2}} < \left(c+\frac{1}{\d}\right)\log(q_{1}q_{2}).
        \]
        If $\min(\b_{1},\b_{2}) \ge 1-\frac{c}{\log(q_{1}q_{2})}$, it follows that
        \[
          2(\d+c) < c+\frac{1}{\d},
        \]
        which is a contradiction if we take $\d$ small enough to ensure $2\d^{2}+c\d < 1$.
      \end{proof}
    
      From \cref{prop:product_of_quadratic_Dirichlet_L-functions_has_one_zero} we immediately see that for every modulus $m > 1$ there is at most one primitive quadratic Dirichlet character that can admit a Siegel zero:

    \begin{proposition}\label{prop:at_most_one_Siegel_zero_per_modulus}
      For every integer $m > 1$, there is at most one Dirichlet character $\chi$ modulo $m$ such that $L(s,\chi)$ has a Siegel zero. If this Siegel zero exists, $\chi$ is necessarily quadratic.
    \end{proposition}
    \begin{proof}
      Let $\wtilde{\chi}$ be the primitive character inducing $\chi$. As the zeros of $L(s,\chi)$ and $L(s,\wtilde{\chi})$ differ only on the line $\s = 0$, \cref{thm:improved_zero-free_region_Dirichlet} implies that $\wtilde{\chi}$, and hence $\chi$, must be quadratic. So suppose $\chi_{1}$ and $\chi_{2}$ are two distinct characters modulo $m$, of conductors $q_{1}$ and $q_{2}$, admitting Siegel zeros $\b_{1}$ and $\b_{2}$. Then $\chi_{1}\chi_{2} \neq \chi_{m,0}$. Moreover, $\b_{1} \ge 1-\frac{c_{1}}{\log(q_{1})}$ and $\b_{2} \ge 1-\frac{c_{2}}{\log(q_{2})}$ for some positive constants $c_{1}$ and $c_{2}$. Taking $c$ smaller, if necessary, we have $\min(\b_{1},\b_{2}) \ge 1-\frac{c}{\log(q_{1}q_{2})}$ which contradicts \cref{prop:product_of_quadratic_Dirichlet_L-functions_has_one_zero}.
    \end{proof}
  \section{The Class Number of Quadratic Number Fields}
    Recall that the Dirichlet class number formula gives a relationship between the class number $h_{d}$ of the quadratic number field $\Q(\sqrt{d})$ and the value of the Dirichlet $L$-function $L(s,\chi_{\D_{d}})$ at $s = 1$ where $\chi_{\D_{d}}$ is the quadratic character given by the Kronecker symbol. As $\chi_{\D_{d}}$ is a primitive quadratic character of conductor $|\D_{d}| > 1$ (recall \cref{thm:fundamental_discriminant_character_primitive}), we know from \cref{thm:non-vanishing_of_Dirichlet_L-functions_at_s=1} that $L(1,\chi_{\D_{d}})$ is finite and nonzero. It is interesting to know whether or not this value is computable in general so that we may obtain another formula for the class number. We will actually obtain a formula for $L(1,\chi)$ where $\chi$ is any primitive character $\chi$ of conductor $q > 1$. The computation is fairly straightforward and only requires some basic properties of Ramanujan and Gauss sums that we have already developed. The idea is to rewrite the character values $\chi(n)$ so that we can collapse the infinite series into a Taylor series. Our result is the following:
    
    \begin{theorem}\label{thm:Value_of_Dirichlet_L-functions_at_s=1}
      Let $\chi$ be a primitive Dirichlet character with conductor $q > 1$. Then
      \[
        L(1,\chi) = -\frac{\tau(\chi)}{q}\sum_{a \tmod{q}}\cchi(a)\log\sin\left(\frac{\pi a}{q}\right) \quad \text{or} \quad L(1,\chi) = \frac{\pi i\tau(\chi)}{q^{2}}\sum_{a \tmod{q}}\cchi(a)a,
      \]
      according to whether $\chi$ is even or odd.
    \end{theorem}
    \begin{proof}
      Recall that the Dirichlet $L$-series $L(s,\chi)$ converges for $\s > 0$ and thus at $s = 1$. First compute
      \begin{align*}
        \chi(n) &= \frac{1}{\tau(\cchi)}\sum_{a \tmod{q}}\cchi(a)e^{\frac{2\pi ian}{q}} && \text{\cref{cor:gauss_sum_primitive_formula}} \\
        &= \frac{\chi(-1)}{\conj{\tau(\chi)}}\sum_{a \tmod{q}}\cchi(a)e^{\frac{2\pi ian}{q}} && \text{\cref{prop:Gauss_sum_reduction} (i) and $\chi(-1)^{2} = 1$} \\
        &= \frac{\chi(-1)\tau(\chi)}{\tau(\chi)\conj{\tau(\chi)}}\sum_{a \tmod{q}}\cchi(a)e^{\frac{2\pi ian}{q}} \\
        &= \frac{\chi(-1)\tau(\chi)}{q}\sum_{a \tmod{q}}\cchi(a)e^{\frac{2\pi ian}{q}} && \text{\cref{thm:Gauss_sum_modulus}}.
      \end{align*}
      Substituting the above result into the definition of $L(1,\chi)$, we find that
      \begin{equation}\label{equ:value_of_Dirichlet_L-functions_at_s=1_1}
        \begin{aligned}
          L(1,\chi) &= \sum_{n \ge 1}\frac{1}{n}\left(\frac{\chi(-1)\tau(\chi)}{q}\sum_{a \tmod{q}}\cchi(a)e^{\frac{2\pi ian}{q}}\right) \\
          &= \frac{\chi(-1)\tau(\chi)}{q}\sum_{a \tmod{q}}\cchi(a)\sum_{n \ge 1}\frac{e^{\frac{2\pi ian}{q}}}{n} \\
          &= \frac{\chi(-1)\tau(\chi)}{q}\sum_{a \tmod{q}}\cchi(a)\log\left(\left(1-e^{\frac{2\pi ia}{q}}\right)^{-1}\right),
        \end{aligned}
      \end{equation}
      where in the last line we have used the Taylor series of the logarithm. We will now simplify the last expression in \cref{equ:value_of_Dirichlet_L-functions_at_s=1_1}. Since $\sin(x) = \frac{e^{ix}-e^{-ix}}{2i}$, we have
      \[
        1-e^{\frac{2\pi ia}{q}} = -2ie^{\frac{\pi ia}{q}}\left(\frac{e^{\frac{\pi ia}{q}}-e^{-\frac{\pi ia}{q}}}{2i}\right) = -2ie^{\frac{\pi ia}{q}}\sin\left(\frac{\pi a}{q}\right).
      \]
      Therefore the last expression in \cref{equ:value_of_Dirichlet_L-functions_at_s=1_1} becomes
      \[
        \frac{\chi(-1)\tau(\chi)}{q}\sum_{a \tmod{q}}\cchi(a)\log\left(\left(-2ie^{\frac{\pi ia}{q}}\sin\left(\frac{\pi a}{q}\right)\right)^{-1}\right).
      \]
      As $a$ is taken modulo $q$, we have $0 < \frac{\pi a}{q} < \pi$ so that $\sin\left(\frac{\pi a}{q}\right)$ is never negative. Therefore we can split up the logarithm term and obtain
      \[
        -\frac{\chi(-1)\tau(\chi)}{q}\left(\log(-2i)\sum_{a \tmod{q}}\cchi(a)+\frac{\pi i}{q}\sum_{a \tmod{q}}\cchi(a)a+\sum_{a \tmod{q}}\cchi(a)\log\sin\left(\frac{\pi a}{q}\right)\right).
      \]
      By the Dirichlet orthogonality relations (namely \cref{cor:Dirichlet_orthogonality_relations} (i)), the first sum above vanishes. Therefore
      \begin{equation}\label{equ:value_of_Dirichlet_L-functions_at_s=1_2}
        L(1,\chi) = -\frac{\chi(-1)\tau(\chi)}{q}\left(\frac{\pi i}{q}\sum_{a \tmod{q}}\chi(a)a+\sum_{a \tmod{q}}\chi(a)\log\sin\left(\frac{\pi a}{q}\right)\right).
      \end{equation}
      \cref{equ:value_of_Dirichlet_L-functions_at_s=1_2} simplifies in that one of the two sums vanish depending on if $\chi$ is even or odd. For the first sum in \cref{equ:value_of_Dirichlet_L-functions_at_s=1_2}, the change of variables $a \to -a$ shows that
      \[
        \frac{\pi i}{q}\sum_{a \tmod{q}}\chi(a)a = -\frac{\chi(-1)\pi i}{q}\sum_{a \tmod{q}}\chi(a)a.
      \]
      Hence this sum vanishes if $\chi$ is even which proves the even case. For the second sum in \cref{equ:value_of_Dirichlet_L-functions_at_s=1_2}, the change of variables $a \to q-a$ shows that
      \[
        \sum_{a \tmod{q}}\chi(a)\log\sin\left(\frac{\pi a}{q}\right) = \chi(-1)\sum_{a \tmod{q}}\chi(a)\log\sin\left(\frac{\pi a}{q}\right).
      \]
      Therefore this sum vanishes if $\chi$ is odd proving in the odd case and completing the proof.
    \end{proof}
    
    \begin{remark}
      \cref{thm:Value_of_Dirichlet_L-functions_at_s=1} encodes some interesting identities. For example, if $\chi$ is the non-principal Dirichlet character modulo $4$ then $\chi$ is uniquely defined by $\chi(1) = 1$ and $\chi(3) = \chi(-1) = -1$. In particular, $\chi$ is odd and its conductor is $4$. Now
      \[
        \tau(\chi) = \sum_{a \tmod{4}}\chi(a)e^{\frac{2\pi ia}{4}} = e^{\frac{2\pi i}{4}}-e^{\frac{6\pi i}{4}} = i-(-i) = 2i,
      \]
      and so by \cref{thm:Value_of_Dirichlet_L-functions_at_s=1}, we get
      \[
        L(1,\chi) = \frac{\pi i\tau(\chi)}{16}(1-3) = \frac{\pi}{4}.
      \]
      Expanding $L(1,\chi)$ gives
      \[
        1-\frac{1}{3}+\frac{1}{5}-\frac{1}{7}+\cdots = \frac{\pi}{4},
      \]
      which is the famous \textbf{Madhava–Leibniz formula}\index{Madhava–Leibniz formula} for $\pi$.
    \end{remark}

    From the definition of $\chi_{\D_{d}}$, we see that $\chi_{\D_{d}}$ is even or odd according to if $d > 0$ or $d < 0$ (recall \cref{prop:discriminant_quadratic}). This gives an explicit formula for the class number of quadratic number fields:

    \begin{corollary}\label{cor:formula_for_class_number_quadratic_number_field}
      Let $\Q(\sqrt{d})$ be a quadratic number field. Then,
      \[
        h_{d} = -\frac{\tau(\chi)}{2\log|\e_{d}|\sqrt{|\D_{d}|}}\sum_{a \tmod{|\D_{d}|}}\cchi(a)\log\sin\left(\frac{\pi a}{|\D_{d}|}\right) \quad \text{or} \quad h_{d} = \frac{w_{d}i\tau(\chi)}{2|\D_{d}|^{\frac{3}{2}}}\sum_{a \tmod{|\D_{d}|}}\cchi(a)a,
      \]
      where $\e_{d}$ is a fundamental unit, according to if $d > 0$ or $d < 0$.
    \end{corollary}
    \begin{proof}
      This follows from Dirichlet's unit theorem, \cref{thm:Value_of_Dirichlet_L-functions_at_s=1}, and that $\chi_{\D_{d}}$ is even or odd according to if $d > 0$ or $d < 0$.
    \end{proof}

    Having given an explicit formula for the class number of quadratic number fields, we now turn to more useful estimates. By what we have seen, it suffices to obtain bounds for Dirichlet $L$-functions at $s = 1$ and the tighter these bounds are the tighter our estimates for the class number will be. Upper bounds are not too difficult to obtain:

    \begin{proposition}\label{prop:class_number_upper_bound}
      Let $\Q(\sqrt{d})$ be a quadratic number field. Then
      \[
        h_{d}\log|\e_{d}| \ll \sqrt{d}\log(d) \quad \text{or} \quad  h_{d} \ll \sqrt{d}\log(d),
      \]
      where $\e_{d}$ is a fundamental unit, according to if $d > 0$ or $d < 0$.
    \end{proposition}
    \begin{proof}
      This follows from the Dirichlet class number formula and \cref{lem:log_growth_of_Dirichlet_L-series_near_1} since $\D_{d} \asymp d$ (recall \cref{prop:discriminant_quadratic}).
    \end{proof}

    Effective lower bounds are difficult to obtain. However, if we allow the implicit constant to be ineffective we may use Siegel's theorem:

    \begin{proposition}\label{prop:class_number_lower_bound}
      Let $\Q(\sqrt{d})$ be a quadratic number field. Then there exists a positive constant $c(\e)$ such that
      \[
        h_{d}\log|\e_{d}| \ge c(\e)d^{\frac{1}{2}-\e} \quad \text{or} \quad  h_{d} \ge c(\e)d^{\frac{1}{2}-\e},
      \]
      where $\e_{d}$ is a fundamental unit, according to if $d > 0$ or $d < 0$.
    \end{proposition}
    \begin{proof}
      This follows from the Dirichlet class number formula and Siegel's theorem since $\D_{d} \asymp d$ (recall \cref{prop:discriminant_quadratic}).
    \end{proof}

    As a corollary we find that there are only finitely many imaginary quadratic number fields of a fixed class number:

    \begin{corollary}\label{cor:finitely_many_imaginary_quadratic_of_fixed_class_number}
      Let $n \ge 1$. Then there are finitely many imaginary quadratic number fields $\Q(\sqrt{d})$ of class number $n$.
    \end{corollary}
    \begin{proof}
      By \cref{prop:class_number_lower_bound}, we see that $h_{d} \to \infty$ as $d \to -\infty$. The claim follows at once.
    \end{proof}

    It is in \cref{cor:finitely_many_imaginary_quadratic_of_fixed_class_number} that we can see where the ineffectiveness of Siegel's theorem becomes apparent. Indeed, although there are finitely many imaginary quadratic number fields of class number $n \ge 1$, we cannot use the bound
    \[
      h_{d} \ge c(\e)d^{\frac{1}{2}-\e},
    \]
    from \cref{prop:class_number_lower_bound} to reduce this to a finite computation because the constant $c(\e)$ is ineffective. In other words, we cannot obtain lower bound on $d$ as $d \to -\infty$ that tells us when all class numbers are larger than $n$ even though we know such a bound exists.
  \section{The Burgess Bound for Dirichlet \texorpdfstring{$L$}{L}-functions}
    Let $\chi$ be a Dirichlet character modulo $m$. We call the sum
    \[
      S_{\chi}(M,N) = \sum_{M+1 \le n \le M+N}\chi(n),
    \]
    for $M \ge 0$ and $N \ge 1$, the \textbf{character sum}\index{character sum} of $\chi$. If $M = 0$, we write $S_{\chi}(N) = S_{\chi}(0,N)$ for simplicity. In the case $\chi$ is primitive of conductor $q > 1$, an infamous subconvexity estimate for the Dirichlet $L$-function $L(s,\chi)$ in the $q$-aspect was achieved by Burgess in the 1960's (see \cite{burgess1963character}). The key idea of his argument lies in improved bounds for character sums in certain ranges of $M$ and $N$ relative to powers of $q$ which we will describe. On the one hand, since $|\chi(n)| \le 1$ for all $n \ge 1$, we have the trivial bound
    \begin{equation}\label{equ:character_sum_trivial_bound}
      S_{\chi}(M,N) \ll N.
    \end{equation}
    On the other hand, by the Dirichlet orthogonality relations (namely \cref{cor:Dirichlet_orthogonality_relations} i) and that $\chi$ is non-principal, we have
    \begin{equation}\label{equ:character_sum_orthogonality_bound}
      S_{\chi}(M,N) \ll m.
    \end{equation}
    Note that \cref{equ:character_sum_orthogonality_bound} is only an improvement upon \cref{equ:character_sum_trivial_bound} when $m \ll N$. It turns out that a much shaper bound than \cref{equ:character_sum_orthogonality_bound} can be obtained with very little work. We will first require a small lemma:

    \begin{lemma}\label{lem:Polya-Vinogradov_lemma}
      Let $f(x)$ be a integrable convex function on $\left(a-\frac{b}{2},a+\frac{b}{2}\right)$ for some $a \in \R$ and $b > 0$. Then
      \[
        f(a) \le \frac{1}{b}\int_{a-\frac{b}{2}}^{a+\frac{b}{2}}f(x)\,dx.
      \]
    \end{lemma}
    \begin{proof}
      By the definition of convexity,
      \[
        f(a) = f\left(\frac{1}{2}x+\frac{1}{2}(2a-x)\right) \le \frac{1}{2}f(x)+\frac{1}{2}f(2a-x).
      \]
      Integrating the left-hand side over $\left(a-\frac{b}{2},a+\frac{b}{2}\right)$ yields
      \[
        \int_{a-\frac{b}{2}}^{a+\frac{b}{2}}f(a)\,dx = bf(a),
      \]
      while integrating the right-hand side over $\left[a-\frac{b}{2},a+\frac{b}{2}\right]$ gives
      \[
        \int_{a-\frac{b}{2}}^{a+\frac{b}{2}}\left(\frac{1}{2}f(x)+\frac{1}{2}f(2a-x)\right)\,dx = \frac{1}{2}\int_{a-\frac{b}{2}}^{a+\frac{b}{2}}f(x)\,dx+\int_{a-\frac{b}{2}}^{a+\frac{b}{2}}\frac{1}{2}f(2a-x)\,dx = \int_{a-\frac{b}{2}}^{a+\frac{b}{2}}f(x)\,dx,
      \]
      upon making the change of variables $x \mapsto 2a-x$ in the second integral. Hence
      \[
        bf(a) \le \int_{a-\frac{b}{2}}^{a+\frac{b}{2}}f(x)\,dx,
      \]
      which is equivalent to the claim.
    \end{proof}
    
    We can now improve upon \cref{equ:character_sum_orthogonality_bound}. This result is known as the \textbf{P\'olya-Vinogradov inequality}\index{P\'olya-Vinogradov inequality} as it was proved independently by P\'olya and Vinogradov in 1918 (see \cite{polya1918verteilung} and \cite{vinogradov1918distribution}).

    \begin{theorem*}[P\'olya-Vinogradov inequality]
      Let $\chi$ be a non-principal Dirichlet character modulo $m$. Then for any $M \ge 0$ and $N \ge 1$,
      \[
        S_{\chi}(M,N) \ll \sqrt{m}\log(m).
      \]
    \end{theorem*}
    \begin{proof}
      First suppose $\chi$ is primitive of conductor $q > 1$. Using \cref{cor:gauss_sum_primitive_formula}, we have
      \[
        S_{\chi}(M,N) = \frac{1}{\tau(\cchi)}\sum_{a \tmod{q}}\cchi(a)\sum_{M+1 \le n \le M+N}e^{\frac{2\pi ian}{q}}.
      \]
      The inner sum is geometric and evaluates to
      \[
        \sum_{M+1 \le n \le M+N}e^{\frac{2\pi ian}{q}} = e^{\frac{2\pi ia(M+1)}{q}}\left(\frac{1-e^{\frac{2\pi iaN}{q}}}{1-e^{\frac{2\pi ia}{q}}}\right) = e^{\frac{2\pi ia\left(M+\frac{N-1}{2}\right)}{q}}\left(\frac{e^{\frac{\pi iaN}{q}}-e^{-\frac{\pi iaN}{q}}}{e^{\frac{\pi ia}{q}}-e^{-\frac{\pi ia}{q}}}\right) = e^{\frac{2\pi ia\left(M+\frac{N-1}{2}\right)}{q}}\frac{\sin\left(\frac{\pi Na}{q}\right)}{\sin\left(\frac{\pi a}{q}\right)},
      \]
      where in the last equality we have made use of the formula $\sin(x) = \frac{e^{ix}+e^{-ix}}{2i}$. Then \cref{thm:Gauss_sum_modulus}, gives
      \[
        S_{\chi}(M,N) \ll \frac{1}{\sqrt{q}}\sum_{1 \le a \le q-1}\frac{1}{\sin\left(\frac{\pi a}{q}\right)},
      \]
      where we recall that $\cchi(0) = 0$. Now the function $\frac{1}{\sin(\pi x)}$ is convex and integrable on $(0,1)$ (its second derivative is $\pi^{2}\frac{1+\cos^{2}(\pi x)}{\sin^{3}(\pi x)} > 0$ on this interval), so from \cref{lem:Polya-Vinogradov_lemma} we get
      \[
        \frac{1}{\sqrt{q}}\sum_{1 \le a \le q-1}\frac{1}{\sin\left(\frac{\pi a}{q}\right)} \le \frac{1}{\sqrt{q}}\sum_{1 \le a \le q-1}q\int_{\frac{a}{q}-\frac{1}{2q}}^{\frac{a}{q}+\frac{1}{2q}}\frac{1}{\sin(\pi x)}\,dx = \sqrt{q}\int_{\frac{1}{2q}}^{1-\frac{1}{2q}}\frac{1}{\sin(\pi x)}\,dx = 2\sqrt{q}\int_{\frac{1}{2q}}^{\frac{1}{2}}\frac{1}{\sin(\pi x)}\,dx,
      \]
      where the last equality holds because $\sin(\pi x) = \sin(\pi-\pi x) = \sin(\pi(1-x))$ so that the integral of $\frac{1}{\sin(\pi x)}$ is symmetric over $\left(\frac{1}{2q},\frac{1}{2}\right)$ and $\left(\frac{1}{2},1-\frac{1}{2q}\right)$. Now $\sin(\pi x) \ge 2x$ on the interval $[0,\frac{1}{2}]$ (because $\sin(\pi x) = 2x$ at the boundary and both functions are increasing on the interior), so that
      \[
        2\sqrt{q}\int_{\frac{1}{2q}}^{\frac{1}{2}}\frac{1}{\sin(\pi x)}\,dx \le q\int_{\frac{1}{2q}}^{\frac{1}{2}}\frac{1}{x}\,dx \ll \sqrt{q}\log(q).
      \]
      Therefore
      \[
        S_{\chi}(M,N) \ll \sqrt{q}\log(q),
      \]
      as desired. This proves the bound in the case $\chi$ is primitive. Now suppose $\chi$ is induced by the primitive character $\wtilde{\chi}$ of conductor $q > 1$. Then $q \mid m$, and so we may write $m = kq$ for some $k \ge 1$. Rewrite the sum in terms of $\wtilde{\chi}$ as follows:
      \begin{align*}
        S_{\chi}(M,N) &= \sum_{\substack{M+1 \le n \le M+N \\ (n,k) = 1}}\wtilde{\chi}(n) \\
        &= \sum_{M+1 \le n \le M+N}\wtilde{\chi}(n)\sum_{d \mid (n,k)}\mu(d) && \text{\cref{prop:Mobius_dirac_delta}} \\
        &= \sum_{d \mid k}\mu(d)\sum_{\substack{M+1 \le n \le M+N \\ d \mid n}}\wtilde{\chi}(n) \\
        &= \sum_{d \mid k}\mu(d)\sum_{\frac{M+1}{d} \le n \le \frac{M+N}{d}}\wtilde{\chi}(dn) && \text{$n \to dn$} \\
        &= \sum_{d \mid k}\mu(d)\wtilde{\chi}(d)\sum_{\left\lfloor\frac{M+1}{d}\right\rfloor \le n \le \left\lfloor\frac{M+N}{d}\right\rfloor}\wtilde{\chi}(n).
      \end{align*}
      The inner sum is $O(\sqrt{q}\log(q))$ by the primitive case and so
      \[
        S_{\chi}(M,N) \ll \sqrt{q}\log(q)\sum_{d \mid k}|\mu(d)| \ll 2^{\w(k)}\sqrt{q}\log(q).
      \]
      Since $2^{\w(k)} \ll \s_{0}(k) \ll_{\e} k^{\e}$ (see \cref{prop:sum_of_divisors_growth_rate}) it follows that $2^{\w(k)} \ll \sqrt{k}$ upon taking $\e = \frac{1}{2}$. This bound together with $\log(q) \le \log(m)$ gives
      \[
        S_{\chi}(M,N) \ll \sqrt{m}\log(m),
      \]
      as claimed.
    \end{proof}

    The P\'olya-Vinogradov inequality greatly improves upon \cref{equ:character_sum_orthogonality_bound} and is particularly useful when $N$ is much larger than $m$. A slight improvement was made in 1977 by Montgomery and Vaughn under the Riemann hypothesis for Dirichlet $L$-functions (see \cite{montgomery1977exponential}):

    \begin{theorem}\label{thm:MV_bound_character_sum}
      Let $\chi$ be a non-principal Dirichlet character modulo $m$. Then for any $M \ge 0$ and $N \ge 1$,
      \[
        S_{\chi}(M,N) \ll \sqrt{m}\log(m),
      \]
      provided the Riemann hypothesis for Dirichlet $L$-functions holds.
    \end{theorem}
    
    While \cref{thm:MV_bound_character_sum} is not much of an improvement from the P\'olya-Vinogradov inequality, it is sharp due to a result of Paley in 1932 (see \cite{paley1932theorem}). This means that, quite remarkably, the P\'olya-Vinogradov inequality is almost optimal. It is also useful to have an estimate that is sharper when $N$ is small compared to $m$. In 1963, Burgess made progress in this direction by proving the following result (see \cite{burgess1963character} for a proof which is a generalization of his 1962 papers \cite{burgess1962characterL-series} and \cite{burgess1962characterprimitive}):

    \begin{theorem}\label{thm:Burgess_inequality}
      Let $\chi$ be a non-principal Dirichlet character modulo $m > 1$ and $M \ge 0$, $N \ge 1$, and $r \ge 1$ be integers. Then
      \[
        S_{\chi}(M,N) \ll_{\e} N^{1-\frac{1}{r}}m^{\frac{r+1}{4r^{2}}+\e},
      \]
      provided $m$ is cube-free or $r = 2$.
    \end{theorem}

    \cref{thm:Burgess_inequality} can be thought of as a blend between \cref{equ:character_sum_trivial_bound} and the P\'olya-Vinogradov inequality. Despite the proof being reasonably short, we do not provide the argument as it requires some machinery beyond the scope of this text. More importantly, Burgess used \cref{thm:Burgess_inequality} in conjunction with the P\'olya-Vinogradov inequality to prove a subconvexity estimate for Dirichlet $L$-functions in the conductor-aspect:

    \begin{theorem}\label{thm:Burgess_bound_conductor_aspect_Dirichlet}
      Let $\chi$ be a primitive Dirichlet character of conductor $q > 1$. Then for $0 < \s < 1$, we have
      \[
        L(s,\chi) \ll_{t,\e} \begin{cases} q^{\frac{4-5\s}{8}+\e} & \text{if $0 < \s \le \frac{1}{2}$}, \\ q^{\frac{3-3\s}{8}+\e} & \text{if $\frac{1}{2} \le \s < 1$}. \end{cases}
      \]
      In particular,
      \[
        L\left(\frac{1}{2}+it,\chi\right) \ll_{\e} q^{\frac{3}{16}+\e}. 
      \]
    \end{theorem}
    \begin{proof}
      Let $0 < \s < 1$. Summation by parts (see \cref{cor:summation_by_parts_corollary}) gives
      \[
        \sum_{n \le N}\frac{\chi(n)}{n^{s}} = S_{\chi}(N)N^{-s}+\sum_{n \le N-1}S_{\chi}(n)(n^{-s}-(n+1)^{-s}),
      \]
      for any $N \ge 1$. By applying the mean value theorem to $f(x) = \frac{1}{x^{s}}$ on the interval $[n,n+1]$ we have $(n^{-s}-(n+1)^{-s}) \ll_{t} \frac{1}{n^{\s+1}}$, and thus
      \[
        \sum_{n \le N}\frac{\chi(n)}{n^{s}} \ll_{t} |S_{\chi}(N)|N^{-\s}+\sum_{n \le N-1}\frac{|S_{\chi}(n)|}{n^{\s+1}}.
      \]
      As $|S_{\chi}(N)||N^{-\s}| \to 0$ as $N \to \infty$ (recall \cref{equ:character_sum_orthogonality_bound}), taking the limit as $N \to \infty$ results in
      \[
        L(s,\chi) \ll_{t} \sum_{n \ge 1}\frac{|S_{\chi}(n)|}{n^{\s+1}},
      \]
      where the estimate follows  Letting $M > 2$, $N \ge 1$, and decompose the right-hand side, we obtain
      \[
        L(s,\chi) \ll_{t} \sum_{1 \le M-1}\frac{|S_{\chi}(n)|}{n^{\s+1}}+\sum_{M \le n \le N}\frac{|S_{\chi}(n)|}{n^{\s+1}}+\sum_{n \ge N+1}\frac{|S_{\chi}(n)|}{n^{\s+1}}.
      \]
      Applying \cref{equ:character_sum_trivial_bound} to the first character sum, \cref{thm:Burgess_inequality} (in the case $r = 2$) to the second character sum, and the P\'olya-Vinogradov inequality to the third character sum, gives
      \[
        L(s,\chi) \ll_{t,\e} \sum_{1 \le M-1}\frac{1}{n^{\s}}+q^{\frac{3}{16}+\e}\sum_{M \le n \le N}\frac{1}{n^{\s+\frac{1}{2}}}+q^{\frac{1}{2}+\e}\sum_{n \ge N+1}\frac{1}{n^{\s+1}},
      \]
      where we have used that $\log(q) \ll_{\e} q^{\e}$. Upper bounding the first and third sums with an integral and integrating, we get
      \[
        L(s,\chi) \ll_{t,\e} M^{1-\s}+q^{\frac{3}{16}+\e}\sum_{M \le n \le N}\frac{1}{n^{\s+\frac{1}{2}}}+q^{\frac{1}{2}+\e}N^{-\s},
      \]
      since $0 < \s < 1$. Performing the same argument with the remaining sum, we see that
      \[
        \sum_{M \le n \le N}\frac{1}{n^{\s+\frac{1}{2}}} \ll \begin{cases} N^{\frac{1}{2}-\s} & \text{if $0 < \s < \frac{1}{2}$}, \\ \log(N) & \text{if $\s = \frac{1}{2}$}, \\ M^{\frac{1}{2}-\s} & \text{if $\frac{1}{2} < \s < 1$}. \end{cases}
      \]
       Upon setting $M = \lfloor q^{\frac{3}{8}} \rfloor$ and $N = \lceil q^{\frac{5}{8}} \rceil$, we obtain
      \[
        L(s,\chi) \ll_{t,\e} \begin{cases} q^{\frac{4-5\s}{8}+\e} & \text{if $0 < \s \le \frac{1}{2}$}, \\ q^{\frac{3-3\s}{8}+\e} & \text{if $\frac{1}{2} \le \s < \frac{1}{2}$}, \end{cases}
      \]
      because the third term dominates for $0 < \s < \frac{1}{2}$, the first term dominates for $\frac{1}{2} < \s < 1$, and at $s = \frac{1}{2}$ the first and third terms are balanced and dominate the second term. This proves the first statement. The second follows by taking $\s = \frac{1}{2}$.
    \end{proof}

    In particular, \cref{thm:Burgess_bound_conductor_aspect_Dirichlet} shows that the Burgess bound in the conductor-aspect holds for Dirichlet $L$-functions. This is how the Burgess bound got its name as it was the first subconvexity estimate for Dirichlet $L$-functions in the conductor-aspect.