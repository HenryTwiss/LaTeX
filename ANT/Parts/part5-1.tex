\chapter{Subconvexity Results}
  We discuss a classical subconvexity result for Dirichlet $L$-functions due to Burgess which relies upon subtle estimates for sums of Dirichlet characters. Necessary to this discussion is the infamous P\'olya-Vinogradov inequality which we also prove.
  \section{The Burgess Bound for Dirichlet \texorpdfstring{$L$}{L}-functions}
    Let $\chi$ be a Dirichlet character modulo $m$. We call the sum
    \[
      S_{\chi}(M,N) = \sum_{M+1 \le n \le M+N}\chi(n),
    \]
    for $M \ge 0$ and $N \ge 1$, the \textbf{character sum}\index{character sum} of $\chi$. If $M = 0$, we write $S_{\chi}(N) = S_{\chi}(0,N)$ for simplicity. In the case $\chi$ is primitive of conductor $q > 1$, an infamous subconvexity estimate for the Dirichlet $L$-function $L(s,\chi)$ in the $q$-aspect was achieved by Burgess in the 1960's (see \cite{burgess1963character}). The key idea of his argument lies in improved bounds for character sums in certain ranges of $M$ and $N$ relative to powers of $q$ which we will describe. On the one hand, since $|\chi(n)| \le 1$ for all $n \ge 1$, we have the trivial bound
    \begin{equation}\label{equ:character_sum_trivial_bound}
      S_{\chi}(M,N) \ll N.
    \end{equation}
    On the other hand, by the Dirichlet orthogonality relations (namely \cref{cor:Dirichlet_orthogonality_relations} i) and that $\chi$ is non-principal, we have
    \begin{equation}\label{equ:character_sum_orthogonality_bound}
      S_{\chi}(M,N) \ll m.
    \end{equation}
    Note that \cref{equ:character_sum_orthogonality_bound} is only an improvement upon \cref{equ:character_sum_trivial_bound} when $m \ll N$. It turns out that a much shaper bound than \cref{equ:character_sum_orthogonality_bound} can be obtained with very little work. We will first require a small lemma:

    \begin{lemma}\label{lem:Polya-Vinogradov_lemma}
      Let $f(x)$ be a integrable convex function on $\left(a-\frac{b}{2},a+\frac{b}{2}\right)$ for some $a \in \R$ and $b > 0$. Then
      \[
        f(a) \le \frac{1}{b}\int_{a-\frac{b}{2}}^{a+\frac{b}{2}}f(x)\,dx.
      \]
    \end{lemma}
    \begin{proof}
      By the definition of convexity,
      \[
        f(a) = f\left(\frac{1}{2}x+\frac{1}{2}(2a-x)\right) \le \frac{1}{2}f(x)+\frac{1}{2}f(2a-x).
      \]
      Integrating the left-hand side over $\left(a-\frac{b}{2},a+\frac{b}{2}\right)$ yields
      \[
        \int_{a-\frac{b}{2}}^{a+\frac{b}{2}}f(a)\,dx = bf(a),
      \]
      while integrating the right-hand side over $\left[a-\frac{b}{2},a+\frac{b}{2}\right]$ gives
      \[
        \int_{a-\frac{b}{2}}^{a+\frac{b}{2}}\left(\frac{1}{2}f(x)+\frac{1}{2}f(2a-x)\right)\,dx = \frac{1}{2}\int_{a-\frac{b}{2}}^{a+\frac{b}{2}}f(x)\,dx+\int_{a-\frac{b}{2}}^{a+\frac{b}{2}}\frac{1}{2}f(2a-x)\,dx = \int_{a-\frac{b}{2}}^{a+\frac{b}{2}}f(x)\,dx,
      \]
      upon making the change of variables $x \mapsto 2a-x$ in the second integral. Hence
      \[
        bf(a) \le \int_{a-\frac{b}{2}}^{a+\frac{b}{2}}f(x)\,dx,
      \]
      which is equivalent to the claim.
    \end{proof}
    
    We can now improve upon \cref{equ:character_sum_orthogonality_bound}. This result is known as the \textbf{P\'olya-Vinogradov inequality}\index{P\'olya-Vinogradov inequality} as it was proved independently by P\'olya and Vinogradov in 1918 (see \cite{polya1918verteilung} and \cite{vinogradov1918distribution}).

    \begin{theorem*}[P\'olya-Vinogradov inequality]
      Let $\chi$ be a non-principal Dirichlet character modulo $m$. Then for any $M \ge 0$ and $N \ge 1$,
      \[
        S_{\chi}(M,N) \ll \sqrt{m}\log(m).
      \]
    \end{theorem*}
    \begin{proof}
      First suppose $\chi$ is primitive of conductor $q > 1$. Using \cref{cor:gauss_sum_primitive_formula}, we have
      \[
        S_{\chi}(M,N) = \frac{1}{\tau(\cchi)}\sum_{a \tmod{q}}\cchi(a)\sum_{M+1 \le n \le M+N}e^{\frac{2\pi ian}{q}}.
      \]
      The inner sum is geometric and evaluates to
      \[
        \sum_{M+1 \le n \le M+N}e^{\frac{2\pi ian}{q}} = e^{\frac{2\pi ia(M+1)}{q}}\left(\frac{1-e^{\frac{2\pi iaN}{q}}}{1-e^{\frac{2\pi ia}{q}}}\right) = e^{\frac{2\pi ia\left(M+\frac{N-1}{2}\right)}{q}}\left(\frac{e^{\frac{\pi iaN}{q}}-e^{-\frac{\pi iaN}{q}}}{e^{\frac{\pi ia}{q}}-e^{-\frac{\pi ia}{q}}}\right) = e^{\frac{2\pi ia\left(M+\frac{N-1}{2}\right)}{q}}\frac{\sin\left(\frac{\pi Na}{q}\right)}{\sin\left(\frac{\pi a}{q}\right)},
      \]
      where in the last equality we have made use of the formula $\sin(x) = \frac{e^{ix}+e^{-ix}}{2i}$. Then \cref{thm:Gauss_sum_modulus}, gives
      \[
        S_{\chi}(M,N) \ll \frac{1}{\sqrt{q}}\sum_{1 \le a \le q-1}\frac{1}{\sin\left(\frac{\pi a}{q}\right)},
      \]
      where we recall that $\cchi(0) = 0$. Now the function $\frac{1}{\sin(\pi x)}$ is convex and integrable on $(0,1)$ (its second derivative is $\pi^{2}\frac{1+\cos^{2}(\pi x)}{\sin^{3}(\pi x)} > 0$ on this interval), so from \cref{lem:Polya-Vinogradov_lemma} we get
      \[
        \frac{1}{\sqrt{q}}\sum_{1 \le a \le q-1}\frac{1}{\sin\left(\frac{\pi a}{q}\right)} \le \frac{1}{\sqrt{q}}\sum_{1 \le a \le q-1}q\int_{\frac{a}{q}-\frac{1}{2q}}^{\frac{a}{q}+\frac{1}{2q}}\frac{1}{\sin(\pi x)}\,dx = \sqrt{q}\int_{\frac{1}{2q}}^{1-\frac{1}{2q}}\frac{1}{\sin(\pi x)}\,dx = 2\sqrt{q}\int_{\frac{1}{2q}}^{\frac{1}{2}}\frac{1}{\sin(\pi x)}\,dx,
      \]
      where the last equality holds because $\sin(\pi x) = \sin(\pi-\pi x) = \sin(\pi(1-x))$ so that the integral of $\frac{1}{\sin(\pi x)}$ is symmetric over $\left(\frac{1}{2q},\frac{1}{2}\right)$ and $\left(\frac{1}{2},1-\frac{1}{2q}\right)$. Now $\sin(\pi x) \ge 2x$ on the interval $[0,\frac{1}{2}]$ (because $\sin(\pi x) = 2x$ at the boundary and both functions are increasing on the interior), so that
      \[
        2\sqrt{q}\int_{\frac{1}{2q}}^{\frac{1}{2}}\frac{1}{\sin(\pi x)}\,dx \le q\int_{\frac{1}{2q}}^{\frac{1}{2}}\frac{1}{x}\,dx \ll \sqrt{q}\log(q).
      \]
      Therefore
      \[
        S_{\chi}(M,N) \ll \sqrt{q}\log(q),
      \]
      as desired. This proves the bound in the case $\chi$ is primitive. Now suppose $\chi$ is induced by the primitive character $\wtilde{\chi}$ of conductor $q > 1$. Then $q \mid m$, and so we may write $m = kq$ for some $k \ge 1$. Rewrite the sum in terms of $\wtilde{\chi}$ as follows:
      \begin{align*}
        S_{\chi}(M,N) &= \sum_{\substack{M+1 \le n \le M+N \\ (n,k) = 1}}\wtilde{\chi}(n) \\
        &= \sum_{M+1 \le n \le M+N}\wtilde{\chi}(n)\sum_{d \mid (n,k)}\mu(d) && \text{\cref{prop:Mobius_dirac_delta}} \\
        &= \sum_{d \mid k}\mu(d)\sum_{\substack{M+1 \le n \le M+N \\ d \mid n}}\wtilde{\chi}(n) \\
        &= \sum_{d \mid k}\mu(d)\sum_{\frac{M+1}{d} \le n \le \frac{M+N}{d}}\wtilde{\chi}(dn) && \text{$n \to dn$} \\
        &= \sum_{d \mid k}\mu(d)\wtilde{\chi}(d)\sum_{\left\lfloor\frac{M+1}{d}\right\rfloor \le n \le \left\lfloor\frac{M+N}{d}\right\rfloor}\wtilde{\chi}(n).
      \end{align*}
      The inner sum is $O(\sqrt{q}\log(q))$ by the primitive case and so
      \[
        S_{\chi}(M,N) \ll \sqrt{q}\log(q)\sum_{d \mid k}|\mu(d)| \ll 2^{\w(k)}\sqrt{q}\log(q).
      \]
      Since $2^{\w(k)} \ll \s_{0}(k) \ll_{\e} k^{\e}$ (see \cref{prop:sum_of_divisors_growth_rate}) it follows that $2^{\w(k)} \ll \sqrt{k}$ upon taking $\e = \frac{1}{2}$. This bound together with $\log(q) \le \log(m)$ gives
      \[
        S_{\chi}(M,N) \ll \sqrt{m}\log(m),
      \]
      as claimed.
    \end{proof}

    The P\'olya-Vinogradov inequality greatly improves upon \cref{equ:character_sum_orthogonality_bound} and is particularly useful when $N$ is much larger than $m$. A slight improvement was made in 1977 by Montgomery and Vaughn under the Riemann hypothesis for Dirichlet $L$-functions (see \cite{montgomery1977exponential}):

    \begin{theorem}\label{thm:MV_bound_character_sum}
      Let $\chi$ be a non-principal Dirichlet character modulo $m$. Then for any $M \ge 0$ and $N \ge 1$,
      \[
        S_{\chi}(M,N) \ll \sqrt{m}\log{\log(m)},
      \]
      provided the Riemann hypothesis for Dirichlet $L$-functions holds.
    \end{theorem}
    
    While \cref{thm:MV_bound_character_sum} is not much of an improvement from the P\'olya-Vinogradov inequality, it is sharp due to a result of Paley in 1932 (see \cite{paley1932theorem}). This means that, quite remarkably, the P\'olya-Vinogradov inequality is almost optimal. It is also useful to have an estimate that is sharper when $N$ is small compared to $m$. In 1963, Burgess made progress in this direction by proving the following result (see \cite{burgess1963character} for a proof which is a generalization of his 1962 papers \cite{burgess1962characterL-series} and \cite{burgess1962characterprimitive}):

    \begin{theorem}\label{thm:Burgess_inequality}
      Let $\chi$ be a non-principal Dirichlet character modulo $m > 1$ and $M \ge 0$, $N \ge 1$, and $r \ge 1$ be integers. Then
      \[
        S_{\chi}(M,N) \ll_{\e} N^{1-\frac{1}{r}}m^{\frac{r+1}{4r^{2}}+\e},
      \]
      provided $m$ is cube-free or $r = 2$.
    \end{theorem}

    \cref{thm:Burgess_inequality} can be thought of as a blend between \cref{equ:character_sum_trivial_bound} and the P\'olya-Vinogradov inequality. Despite the proof being reasonably short, we do not provide the argument as it requires some machinery beyond the scope of this text. More importantly, Burgess used \cref{thm:Burgess_inequality} in conjunction with the P\'olya-Vinogradov inequality to prove a subconvexity estimate for Dirichlet $L$-functions in the conductor-aspect:

    \begin{theorem}\label{thm:Burgess_bound_conductor_aspect_Dirichlet}
      Let $\chi$ be a primitive Dirichlet character of conductor $q > 1$. Then for $0 < \s < 1$, we have
      \[
        L(s,\chi) \ll_{t,\e} \begin{cases} q^{\frac{4-5\s}{8}+\e} & \text{if $0 < \s \le \frac{1}{2}$}, \\ q^{\frac{3-3\s}{8}+\e} & \text{if $\frac{1}{2} \le \s < 1$}. \end{cases}
      \]
      In particular,
      \[
        L\left(\frac{1}{2}+it,\chi\right) \ll_{\e} q^{\frac{3}{16}+\e}. 
      \]
    \end{theorem}
    \begin{proof}
      Let $0 < \s < 1$. Summation by parts (see \cref{cor:summation_by_parts_corollary}) gives
      \[
        \sum_{n \le N}\frac{\chi(n)}{n^{s}} = S_{\chi}(N)N^{-s}+\sum_{n \le N-1}S_{\chi}(n)(n^{-s}-(n+1)^{-s}),
      \]
      for any $N \ge 1$. By applying the mean value theorem to $f(x) = \frac{1}{x^{s}}$ on the interval $[n,n+1]$ we have $(n^{-s}-(n+1)^{-s}) \ll_{t} \frac{1}{n^{\s+1}}$, and thus
      \[
        \sum_{n \le N}\frac{\chi(n)}{n^{s}} \ll_{t} |S_{\chi}(N)|N^{-\s}+\sum_{n \le N-1}\frac{|S_{\chi}(n)|}{n^{\s+1}}.
      \]
      As $|S_{\chi}(N)||N^{-\s}| \to 0$ as $N \to \infty$ (recall \cref{equ:character_sum_orthogonality_bound}), taking the limit as $N \to \infty$ results in
      \[
        L(s,\chi) \ll_{t} \sum_{n \ge 1}\frac{|S_{\chi}(n)|}{n^{\s+1}},
      \]
      where the estimate follows  Letting $M > 2$, $N \ge 1$, and decompose the right-hand side, we obtain
      \[
        L(s,\chi) \ll_{t} \sum_{1 \le M-1}\frac{|S_{\chi}(n)|}{n^{\s+1}}+\sum_{M \le n \le N}\frac{|S_{\chi}(n)|}{n^{\s+1}}+\sum_{n \ge N+1}\frac{|S_{\chi}(n)|}{n^{\s+1}}.
      \]
      Applying \cref{equ:character_sum_trivial_bound} to the first character sum, \cref{thm:Burgess_inequality} (in the case $r = 2$) to the second character sum, and the P\'olya-Vinogradov inequality to the third character sum, gives
      \[
        L(s,\chi) \ll_{t,\e} \sum_{1 \le M-1}\frac{1}{n^{\s}}+q^{\frac{3}{16}+\e}\sum_{M \le n \le N}\frac{1}{n^{\s+\frac{1}{2}}}+q^{\frac{1}{2}+\e}\sum_{n \ge N+1}\frac{1}{n^{\s+1}},
      \]
      where we have used that $\log(q) \ll_{\e} q^{\e}$. Upper bounding the first and third sums with an integral and integrating, we get
      \[
        L(s,\chi) \ll_{t,\e} M^{1-\s}+q^{\frac{3}{16}+\e}\sum_{M \le n \le N}\frac{1}{n^{\s+\frac{1}{2}}}+q^{\frac{1}{2}+\e}N^{-\s},
      \]
      since $0 < \s < 1$. Performing the same argument with the remaining sum, we see that
      \[
        \sum_{M \le n \le N}\frac{1}{n^{\s+\frac{1}{2}}} \ll \begin{cases} N^{\frac{1}{2}-\s} & \text{if $0 < \s < \frac{1}{2}$}, \\ \log(N) & \text{if $\s = \frac{1}{2}$}, \\ M^{\frac{1}{2}-\s} & \text{if $\frac{1}{2} < \s < 1$}. \end{cases}
      \]
       Upon setting $M = \lfloor q^{\frac{3}{8}} \rfloor$ and $N = \lceil q^{\frac{5}{8}} \rceil$, we obtain
      \[
        L(s,\chi) \ll_{t,\e} \begin{cases} q^{\frac{4-5\s}{8}+\e} & \text{if $0 < \s \le \frac{1}{2}$}, \\ q^{\frac{3-3\s}{8}+\e} & \text{if $\frac{1}{2} \le \s < \frac{1}{2}$}, \end{cases}
      \]
      because the third term dominates for $0 < \s < \frac{1}{2}$, the first term dominates for $\frac{1}{2} < \s < 1$, and at $s = \frac{1}{2}$ the first and third terms are balanced and dominate the second term. This proves the first statement. The second follows by taking $\s = \frac{1}{2}$.
    \end{proof}

    In particular, \cref{thm:Burgess_bound_conductor_aspect_Dirichlet} shows that the Burgess bound in the conductor-aspect holds for Dirichlet $L$-functions. This is how the Burgess bound got its name as it was the first subconvexity estimate for Dirichlet $L$-functions in the conductor-aspect.