\chapter{Moment Results}
  Obtaining any type of moment result is, generally speaking, a very difficult task requiring delicate analytic techniques. Since moments of $L$-functions are still a modern topic of research, there is not yet a universal theory to obtain such results. Even worse, currently methods almost only work when $k \le 4$. So, rather than trying to develop a general theory, we highlight some infamous moment results and their techniques.
  \section{The Second Moment of the Riemann Zeta Function}
    We will discuss the second moment of the Riemann zeta function. Of primary importance is to prove the infamous result of Hardy and Littlewood (see \cite{hardy1916contributions}). Of secondary importance is to illustrate where the primary obstructions appear in the proof and how inputting more refined analytic data will lead to improved asymptotics. Throughout, it will be useful to recall that
    \begin{equation}\label{equ:approximation_for_harmonic_sums}
      \sum_{n \ll N}\frac{1}{n} = \log(N)+O(1),
    \end{equation}
    for any $N \ge 1$, which follows from \cref{equ:Euler-Mascheroni_constant_definition}. The aforementioned result of Hardy and Littlewood (see \cite{hardy1916contributions}) is an asymptotic equivalence for $M_{2}(T,\z)$. Actually, we prove something slightly stronger:

    \begin{theorem}\label{thm:second_moment_of_Riemann_zeta_asymptotic_equivalence}
      For $T > 2$,
      \[
        M_{2}(T,\z) = T\log(T)+O\left(T\log^{\frac{3}{4}}(T)\right).
      \]
      In particular,
      \[
        M_{2}(T,\z) \sim T\log(T).
      \]
    \end{theorem}
    \begin{proof}
      Taking $s = \frac{1}{2}+it$, $X = \sqrt{\frac{t}{\log(t)}}$ for $t \ge 2$, and $\Phi(u) = \cos^{-4M}\left(\frac{\pi u}{4M}\right)$ with $M \ge 1$ in the approximate functional equation gives
      \begin{align*}
        \z\left(\frac{1}{2}+it\right) &= \sum_{n \ge 1}\frac{1}{n^{\frac{1}{2}+it}}V_{\frac{1}{2}+it}\left(n\sqrt{\frac{\log(t)}{t}}\right) \\
        &+ \e\left(\frac{1}{2}+it,\z\right)\sum_{n \ge 1}\frac{1}{n^{\frac{1}{2}-it}}V_{\frac{1}{2}-it}\left(n\sqrt{\frac{t}{\log(t)}}\right)+\frac{R\left(\frac{1}{2}+it,\sqrt{\frac{t}{\log(t)}},\z\right)}{\g\left(\frac{1}{2}+it,\z\right)}.
      \end{align*}
      By \cref{prop:V_function_decay}, we find that $V_{\frac{1}{2}+it}\left(n\sqrt{\frac{\log(t)}{t}}\right)$ and $V_{\frac{1}{2}-it}\left(n\sqrt{\frac{t}{\log(t)}}\right)$ are bounded for $n \ll \frac{t}{\sqrt{\log(t)}}$ and $n \ll \sqrt{\log(t)}$ respectively and then exhibit polynomial decay of arbitrarily large order thereafter. Moreover, \cref{equ:modified_gamma_estimates,equ:choice_for_V_decay_estimate} together imply that the third term exhibits exponential decay and is therefore absolutely bounded. Lastly, $\e\left(\frac{1}{2}+it,\z\right)$ is absolutely bounded by \cref{equ:gamma_factor_analytic_conductor_estimate}. Altogether, this means
      \[
        \z\left(\frac{1}{2}+it\right) = \sum_{n \ll \frac{t}{\sqrt{\log(t)}}}\frac{1}{n^{\frac{1}{2}+it}}+O\left(\sum_{n \ll \sqrt{\log(t)}}\frac{1}{\sqrt{n}}\right)+O(1),
      \]
      and using the estimate
      \[
        \sum_{n \ll \sqrt{\log(t)}}\frac{1}{\sqrt{n}} = O\left(\int_{1}^{\sqrt{\log(t)}}\frac{1}{\sqrt{x}}\,dx\right) = O\left(\log^{\frac{1}{4}}(t)\right),
      \]
      we arrive at
      \begin{equation}\label{equ:second_moment_zeta_asymptotic_equivalence_1}
        \z\left(\frac{1}{2}+it\right) = \sum_{n \ll \frac{t}{\sqrt{\log(t)}}}\frac{1}{n^{\frac{1}{2}+it}}+O\left(\log^{\frac{1}{4}}(t)\right),
      \end{equation}
      since $1 = O\left(\log^{\frac{1}{4}}(t)\right)$ for $t \ge 2$. Applying \cref{equ:second_moment_zeta_asymptotic_equivalence_1} to the definition of $M_{2}(T,\z)$ and recalling that $\z\left(\frac{1}{2}+it\right)$ is absolutely bounded for $0 \le t \le 2$, yields
      \begin{equation}\label{equ:second_moment_zeta_asymptotic_equivalence_2}
        \begin{aligned}
          M_{2}(T,\z) &= \int_{2}^{T}\left|\sum_{n \ll \frac{t}{\sqrt{\log(t)}}}\frac{1}{n^{\frac{1}{2}+it}}\right|^{2}\,dt \\
          &+ O\left(\int_{2}^{T}\left|\sum_{n \ll \frac{t}{\sqrt{\log(t)}}}\frac{1}{n^{\frac{1}{2}+it}}\right|\log^{\frac{1}{4}}(t)\,dt\right)+O\left(\int_{2}^{T}\sqrt{\log(t)}\,dt\right).
        \end{aligned}
      \end{equation}
      We will now simplify \cref{equ:second_moment_zeta_asymptotic_equivalence_2}. For the second term, the Cauchy-Schwarz inequality gives
      \begin{align*}
        \int_{2}^{T}\left|\sum_{n \ll \frac{t}{\sqrt{\log(t)}}}\frac{1}{n^{\frac{1}{2}+it}}\right|\log^{\frac{1}{4}}(t)\,dt &= O\left(\left(\int_{2}^{T}\left|\sum_{n \ll \frac{t}{\sqrt{\log(t)}}}\frac{1}{n^{\frac{1}{2}+it}}\right|^{2}\int_{2}^{T}\sqrt{\log(t)}\,dt\right)^{\frac{1}{2}}\right) \\
        &= O\left(\left(\int_{2}^{T}\left|\sum_{n \ll \frac{t}{\sqrt{\log(t)}}}\frac{1}{n^{\frac{1}{2}+it}}\right|^{2}\,dt\right)^{\frac{1}{2}}\sqrt{T}\log^{\frac{1}{4}}(T)\right),
      \end{align*}
      where in the second line we have made use of the estimate
      \[
        \int_{2}^{T}\sqrt{\log(t)}\,dt = T\sqrt{\log(T)}+O\left(\int_{2}^{T}\frac{1}{\sqrt{\log(t)}}\,dt\right) = T\sqrt{\log(T)}+O(T) = O\left(T\sqrt{\log(T)}\right),
      \]
      which follows from an application of integration by parts. Applying this estimate to third term as well, \cref{equ:second_moment_zeta_asymptotic_equivalence_2} becomes
      \begin{equation}\label{equ:second_moment_zeta_asymptotic_equivalence_3}
        \begin{aligned}
          M_{2}(T,\z) &= \int_{2}^{T}\left|\sum_{n \ll \frac{t}{\sqrt{\log(t)}}}\frac{1}{n^{\frac{1}{2}+it}}\right|^{2}\,dt \\
          &+ O\left(\left(\int_{2}^{T}\left|\sum_{n \ll \frac{t}{\sqrt{\log(t)}}}\frac{1}{n^{\frac{1}{2}+it}}\right|^{2}\,dt\right)^{\frac{1}{2}}\sqrt{T}\log^{\frac{1}{4}}(T)\right)+O\left(T\sqrt{\log(T)}\right).
        \end{aligned}
      \end{equation}
      We now show that the remaining integral is asymptotically equivalent to $T\log(T)$. We first expand the sum and interchange it with the integral by the Fubini-Tonelli theorem to get
      \begin{equation}\label{equ:second_moment_zeta_asymptotic_equivalence_4}
        \begin{aligned}
           \int_{2}^{T}\left|\sum_{n \ll \frac{t}{\sqrt{\log(t)}}}\frac{1}{n^{\frac{1}{2}+it}}\right|^{2} &= \int_{2}^{T}\sum_{n,m \ll \frac{t}{\sqrt{\log(t)}}}\frac{1}{n^{\frac{1}{2}+it}m^{\frac{1}{2}-it}}\,dt \\
          &= \sum_{n,m \ll \frac{T}{\sqrt{\log(T)}}}\int_{\max(m,n,2)}^{T}\frac{1}{n^{\frac{1}{2}+it}m^{\frac{1}{2}-it}}\,dt \\
          &= \sum_{n \ll \frac{T}{\sqrt{\log(T)}}}\frac{T-n}{n}+ \sum_{\substack{n,m \ll \frac{T}{\sqrt{\log(T)}} \\ n \neq m}}\frac{1}{\sqrt{nm}}\int_{\max(n,m,2)}^{T}\left(\frac{m}{n}\right)^{it}\,dt \\
          &= T\sum_{n \ll \frac{T}{\sqrt{\log(T)}}}\frac{1}{n}+O\left(\sum_{n \ll \frac{T}{\sqrt{\log(T)}}}1\right)+\left(\sum_{\substack{n,m \ll \frac{T}{\sqrt{\log(T)}} \\ n \neq m}}\frac{1}{\sqrt{nm}\log\left(\frac{m}{n}\right)}\right),
        \end{aligned}
      \end{equation}
      where in the third line we have separated the terms for which $n = m$ or not. We now estimate all of the remaining terms in \cref{equ:second_moment_zeta_asymptotic_equivalence_4}. For the first term, by \cref{equ:approximation_for_harmonic_sums} we have
      \begin{equation}\label{equ:second_moment_zeta_asymptotic_equivalence_5}
        T\sum_{n \ll \frac{T}{\sqrt{\log(T)}}}\frac{1}{n} = T\log\left(\frac{T}{\sqrt{\log(T)}}\right)+O(T) = T\log(T)+O(T\log\log(T)).
      \end{equation}
      The second term is easier since
      \begin{equation}\label{equ:second_moment_zeta_asymptotic_equivalence_6}
        \sum_{n \ll \frac{T}{\sqrt{\log(T)}}}1 = O\left(\frac{T}{\sqrt{\log(T)}}\right) = O(T).
      \end{equation}
      For the last term, separate the sum into the terms for which $n < \frac{m}{2}$ or not, so that
      \begin{equation}\label{equ:second_moment_zeta_asymptotic_equivalence_7}
        \sum_{\substack{n,m \ll \frac{T}{\sqrt{\log(T)}} \\ n \neq m}}\frac{1}{\sqrt{nm}\log\left(\frac{m}{n}\right)} = \sum_{\substack{n,m \ll \frac{T}{\sqrt{\log(T)}} \\ n < \frac{m}{2}}}\frac{1}{\sqrt{nm}\log\left(\frac{m}{n}\right)}+\sum_{\substack{n,m \ll \frac{T}{\sqrt{\log(T)}} \\ n \ge \frac{m}{2} \\ n \neq m}}\frac{1}{\sqrt{nm}\log\left(\frac{m}{n}\right)}.
      \end{equation}
      In the first sum on the right-hand side of \cref{equ:second_moment_zeta_asymptotic_equivalence_7} we have $\log\left(\frac{m}{n}\right) \ge \log\left(2\right)$ so that $\log\left(\frac{m}{n}\right)$ is bounded from below. Hence
      \[
        \sum_{\substack{n,m \ll \frac{T}{\sqrt{\log(T)}} \\ n < \frac{m}{2}}}\frac{1}{\sqrt{nm}\log\left(\frac{m}{n}\right)} = O\left(\sum_{\substack{n,m \ll \frac{T}{\sqrt{\log(T)}} \\ n < \frac{m}{2}}}\frac{1}{\sqrt{nm}}\right) = O\left(\left(\sum_{n \ll \frac{T}{\sqrt{\log(T)}}}\frac{1}{\sqrt{n}}\right)^{2}\right).
      \]
      But as
      \[
        \sum_{n \ll \frac{T}{\sqrt{\log(T)}}}\frac{1}{\sqrt{n}} = O\left(\int_{1}^{\frac{T}{\sqrt{\log(T)}}}\frac{1}{\sqrt{x}}\,dx\right) = O\left(\frac{\sqrt{T}}{\log^{\frac{1}{4}}(T)}\right),
      \]
      upon combining the previous two estimates, we find that
      \begin{equation}\label{equ:second_moment_zeta_asymptotic_equivalence_8}
        \sum_{\substack{n,m \ll \frac{T}{\sqrt{\log(T)}} \\ n < \frac{m}{2}}}\frac{1}{\sqrt{nm}\log\left(\frac{m}{n}\right)} = O\left(\frac{T}{\sqrt{\log(T)}}\right).
      \end{equation}
      For the second sum, write $n = m-r$ where $1 \le r \le \frac{m}{2}$ so that $\log\left(\frac{m}{n}\right) = \log\left(\frac{m}{m-r}\right) = -\log(1-\frac{r}{m}) \ge \frac{r}{m}$ where the inequality follows from the Taylor series of the logarithm. Whence
      \[
        \sum_{\substack{n,m \ll \frac{T}{\sqrt{\log(T)}} \\ n \ge \frac{m}{2} \\ n \neq m}}\frac{1}{\sqrt{nm}\log\left(\frac{m}{n}\right)} = O\left(\sum_{m \ll \frac{T}{\sqrt{\log(T)}}}\sum_{r \le \frac{m}{2}}\frac{m}{r\sqrt{m(m-r)}}\right) = O\left(\sum_{m \ll \frac{T}{\sqrt{\log(T)}}}\sum_{r \le \frac{m}{2}}\frac{1}{r}\right).
      \]
      To estimate the double sum, using \cref{equ:approximation_for_harmonic_sums} again, we have
      \[
        \sum_{r \le \frac{m}{2}}\frac{1}{r} = \log(m)+O(1).
      \]
      This estimate together with
      \[
        \sum_{m \ll \frac{T}{\sqrt{\log(T)}}}\log(m)+O(1) = \frac{T}{\sqrt{\log(T)}}\log\left(\frac{T}{\sqrt{\log(T)}}\right)+O\left(\frac{T}{\sqrt{\log(T)}}\right) = O\left(T\sqrt{\log(T)}\right),
      \]
      gives
      \begin{equation}\label{equ:second_moment_zeta_asymptotic_equivalence_9}
        \sum_{\substack{n,m \ll \frac{T}{\sqrt{\log(T)}} \\ n \ge \frac{m}{2} \\ n \neq m}}\frac{1}{\sqrt{nm}\log\left(\frac{m}{n}\right)} = O\left(T\sqrt{\log(T)}\right).
      \end{equation}
      Substituting \cref{equ:second_moment_zeta_asymptotic_equivalence_8,equ:second_moment_zeta_asymptotic_equivalence_9} into \cref{equ:second_moment_zeta_asymptotic_equivalence_7} yields
      \begin{equation}\label{equ:second_moment_zeta_asymptotic_equivalence_10}
        \sum_{\substack{n,m \ll \frac{T}{\sqrt{\log(T)}} \\ n \neq m}}\frac{1}{\sqrt{nm}\log\left(\frac{m}{n}\right)} = O\left(T\sqrt{\log(T)}\right),
      \end{equation}
      because $O\left(T\sqrt{\log(T)}\right)$ is the largest of the present $O$-estimates. Then \cref{equ:second_moment_zeta_asymptotic_equivalence_5,equ:second_moment_zeta_asymptotic_equivalence_6,equ:second_moment_zeta_asymptotic_equivalence_10} together with \cref{equ:second_moment_zeta_asymptotic_equivalence_4} yields
      \begin{equation}\label{equ:second_moment_zeta_asymptotic_equivalence_11}
        \int_{2}^{T}\left|\sum_{n \ll \frac{t}{\sqrt{\log(t)}}}\frac{1}{n^{\frac{1}{2}+it}}\right|^{2} = T\log(T)+O\left(T\sqrt{\log(T)}\right),
      \end{equation}
      again because $O\left(T\sqrt{\log(T)}\right)$ is the largest of all the present $O$-estimates. Applying \cref{equ:second_moment_zeta_asymptotic_equivalence_11} to \cref{equ:second_moment_zeta_asymptotic_equivalence_3}, we at last obtain
      \[
        M_{2}(T,\z) = T\log(T)+O\left(T\log^{\frac{3}{4}}(T)\right)+O\left(T\sqrt{\log(T)}\right) = T\log(T)+O\left(T\log^{\frac{3}{4}}(T)\right).
      \]
      which is the desired asymptotic formula. The asymptotic equivalence follows from the asymptotic formula since $T\log^{\frac{3}{4}}(T) = o(T\log(T))$.
    \end{proof}
    
    A few comments about the proof of \cref{thm:second_moment_of_Riemann_zeta_asymptotic_equivalence} are in order. The error term in the asymptotic formula only saves $\log^{\frac{1}{4}}(T)$ from the main term so not much is saved at all. In fact, this is much weaker than the $k = 1$ case in \cref{conj:CFKRS_conjectures_zeta}. As this conjecture has been verified for $k = 1$ (actually $k = 2$ as well as we have already noted), it is possible to do much better. See \cite{titchmarsh1986theory} for a detailed discussion of these estimates and the increasingly refined analytic techniques that are necessary to obtain them. Nevertheless, as $\log(T) \ll_{\e} T^{\e}$, the asymptotic equivalence implies that the necessary bound for the Riemann zeta function is satisfied for \cref{prop:equivalence_Lindelof_hypothesis_and_moments} in the case $k = 1$.

    \begin{remark}
      As far as the truth of the Lindel\"of hypothesis is concerned, \cref{prop:equivalence_Lindelof_hypothesis_and_moments} implies that we do not even need to obtain asymptotic formulas for the $2k$-th moments of $L$-functions. 
    \end{remark}
    
    The main analytic data being fed into \cref{thm:second_moment_of_Riemann_zeta_asymptotic_equivalence} is the approximate functional equation for the Riemann zeta function. The main contribution from the resulting asymptotic, namely \cref{equ:second_moment_zeta_asymptotic_equivalence_1}, in $M_{2}(T,\z)$ is $T\log(T)$ which comes from the diagonal contribution in
    \[
      \int_{2}^{T}\left|\sum_{n \ll \frac{t}{\sqrt{\log(t)}}}\frac{1}{n^{\frac{1}{2}+it}}\right|^{2} = \int_{2}^{T}\sum_{n,m \ll \frac{t}{\sqrt{\log(t)}}}\frac{1}{n^{\frac{1}{2}+it}m^{\frac{1}{2}-it}}\,dt,
    \]
    occurring when $n = m$, as computed in \cref{equ:second_moment_zeta_asymptotic_equivalence_4}. This sum is
    \[
      T\sum_{n \ll \frac{T}{\sqrt{\log(T)}}}\frac{1}{n},
    \]
    and it is estimated in \cref{equ:second_moment_zeta_asymptotic_equivalence_5}. This is the longest sum coming from \cref{equ:second_moment_zeta_asymptotic_equivalence_2} of highest order (because of the factor of $T$). It should also be observed that the contribution for those terms when $n \neq m$ and $n \ge \frac{m}{2}$ comes very close to the main term. This sum is
    \[
      \sum_{\substack{n,m \ll \frac{T}{\sqrt{\log(T)}} \\ n \ge \frac{m}{2} \\ n \neq m}}\frac{1}{\sqrt{nm}\log\left(\frac{m}{n}\right)},
    \]
    and it is estimated in \cref{equ:second_moment_zeta_asymptotic_equivalence_9} to be $O\left(T\sqrt{\log(T)}\right)$. If the length of our sum in \cref{equ:second_moment_zeta_asymptotic_equivalence_1} was increased by a factor of $\sqrt{\log(t)}$ from $\frac{t}{\sqrt{\log(t)}}$ to $t$, then this term would contribute $O\left(T\log(T)\right)$ which would ruin both the asymptotic formula and asymptotic equivalence. In essence, this means that the choice of the length of our sum in the proof of \cref{thm:second_moment_of_Riemann_zeta_asymptotic_equivalence} is essentially optimal.
    