\chapter{Moment Results}
  Obtaining any type of moment result is, generally speaking, a very difficult task requiring delicate analytic techniques. Since moments of $L$-functions are still a modern topic of research, there is not yet a universal theory to obtain such results. Even worse, currently methods almost only work when $k \le 4$. So, rather than trying to develop a general theory, we highlight some infamous moment results and their techniques.
  \section{The Second Moment of the Riemann Zeta Function}
    We will prove the infamous result of Hardy and Littlewood (see \cite{hardy1916contributions}) about the second moment of the Riemann zeta function. This is one of the simplest moment results to obtain since the only real tool we will need is the approximate functional equation. It will also be useful to recall that
    \begin{equation}\label{equ:approximation_for_harmonic_sums}
      \sum_{n \ll N}\frac{1}{n} = \log(N)+O(1),
    \end{equation}
    for any $N \ge 1$, which follows from the identity
    \[
      \sum_{n \le N}\frac{1}{n} = \log(N)+\g_{N},
    \]
    where $\g_{N}$ approaches the Euler-Mascheroni constant $\g$ as $N \to \infty$. Precisely, we prove the following:

    \begin{theorem}\label{thm:second_moment_of_Riemann_zeta}
      For $T > 2$,
      \[
        M_{2}(T,\z) \sim T\log(T).
      \]
    \end{theorem}
    \begin{proof}
      Taking $\s = \frac{1}{2}+it$, $X = \sqrt{\frac{t}{\log(t)}}$ for $t \ge 2$, and $\Phi(u) = \cos^{-4M}\left(\frac{\pi u}{4M}\right)$ with $M \ge 1$ in the approximate functional equation gives
      \begin{align*}
        \z\left(\frac{1}{2}+it\right) &= \sum_{n \ge 1}\frac{1}{n^{\frac{1}{2}+it}}V_{\frac{1}{2}+it}\left(n\sqrt{\frac{\log(t)}{t}}\right) \\
        &+ \e\left(\frac{1}{2}+it,\z\right)\sum_{n \ge 1}\frac{1}{n^{\frac{1}{2}-it}}V_{\frac{1}{2}-it}\left(n\sqrt{\frac{t}{\log(t)}}\right)+\frac{R\left(\frac{1}{2}+it,\sqrt{\frac{t}{\log(t)}},\z\right)}{\g\left(\frac{1}{2}+it,\z\right)}.
      \end{align*}
      By \cref{prop:V_function_decay}, we find that $V_{\frac{1}{2}+it}\left(n\sqrt{\frac{\log(t)}{t}}\right)$ and $V_{\frac{1}{2}-it}\left(n\sqrt{\frac{t}{\log(t)}}\right)$ are bounded for $n \ll \frac{t}{\sqrt{\log(t)}}$ and $n \ll \sqrt{\log(t)}$ respectively and then exhibit polynomial decay of arbitrarily large order thereafter. Moreover, \cref{equ:modified_gamma_estimates,equ:choice_for_V_decay_estimate} together imply that the third term exhibits exponential decay and is therefore absolutely bounded. Lastly, $\e\left(\frac{1}{2}+it,\z\right)$ is absolutely bounded by \cref{equ:gamma_factor_analytic_conductor_estimate}. Altogether, this means
      \[
        \z\left(\frac{1}{2}+it\right) = \sum_{n \ll \frac{t}{\sqrt{\log(t)}}}\frac{1}{n^{\frac{1}{2}+it}}+O\left(\sum_{n \ll \sqrt{\log(t)}}\frac{1}{\sqrt{n}}\right)+O(1),
      \]
      and using the estimate
      \[
        \sum_{n \ll \sqrt{\log(t)}}\frac{1}{\sqrt{n}} = O\left(\int_{1}^{\sqrt{\log(t)}}\frac{1}{\sqrt{x}}\,dx\right) = O\left(\log^{\frac{1}{4}}(t)\right),
      \]
      we arrive at
      \begin{equation}\label{equ:second_moment_zeta_1}
        \z\left(\frac{1}{2}+it\right) = \sum_{n \ll \frac{t}{\sqrt{\log(t)}}}\frac{1}{n^{\frac{1}{2}+it}}+O\left(\log^{\frac{1}{4}}(t)\right),
      \end{equation}
      since $1 = O(\log^{\frac{1}{4}}(t))$ for $t \ge 2$. Applying \cref{equ:second_moment_zeta_1} to the definition of $M_{2}(T,\z)$ and recalling that $\z\left(\frac{1}{2}+it\right)$ is absolutely bounded for $0 \le t \le 2$, yields
      \begin{equation}\label{equ:second_moment_zeta_2}
        \begin{aligned}
          M_{2}(T,\z) &= \int_{2}^{T}\left|\sum_{n \ll \frac{t}{\sqrt{\log(t)}}}\frac{1}{n^{\frac{1}{2}+it}}\right|^{2}\,dt \\
          &+ O\left(\int_{2}^{T}\left|\sum_{n \ll \frac{t}{\sqrt{\log(t)}}}\frac{1}{n^{\frac{1}{2}+it}}\right|\log^{\frac{1}{4}}(t)\,dt\right)+O\left(\int_{2}^{T}\log^{\frac{1}{2}}(t)\,dt\right).
        \end{aligned}
      \end{equation}
      We will now simplify \cref{equ:second_moment_zeta_2}. For the second term, the Cauchy-Schwarz inequality gives
      \begin{align*}
        \int_{2}^{T}\left|\sum_{n \ll \frac{t}{\sqrt{\log(t)}}}\frac{1}{n^{\frac{1}{2}+it}}\right|\log^{\frac{1}{4}}(t)\,dt &= O\left(\left(\int_{2}^{T}\left|\sum_{n \ll \frac{t}{\sqrt{\log(t)}}}\frac{1}{n^{\frac{1}{2}+it}}\right|^{2}\int_{2}^{T}\log^{\frac{1}{2}}(t)\,dt\right)^{\frac{1}{2}}\right) \\
        &= O\left(\left(\int_{2}^{T}\left|\sum_{n \ll \frac{t}{\sqrt{\log(t)}}}\frac{1}{n^{\frac{1}{2}+it}}\right|^{2}\,dt\right)^{\frac{1}{2}}\sqrt{T}\log^{\frac{1}{4}}(T)\right),
      \end{align*}
      where in the second line we have made use of the estimate
      \[
        \int_{2}^{T}\log^{\frac{1}{2}}(t)\,dt = T\log^{\frac{1}{2}}(T)+O\left(\int_{2}^{T}\frac{1}{\log^{\frac{1}{2}}(t)}\,dt\right) = T\log^{\frac{1}{2}}(T)+O(T) = O(T\log^{\frac{1}{2}}(T)),
      \]
      which follows from an application of integration by parts. In fact, this estimate shows that the integral is also $o(T\log(T))$. Applying this latter estimate to third term as well, \cref{equ:second_moment_zeta_2} becomes
      \begin{equation}\label{equ:second_moment_zeta_3}
        \begin{aligned}
          M_{2}(T,\z) &= \int_{2}^{T}\left|\sum_{n \ll \frac{t}{\sqrt{\log(t)}}}\frac{1}{n^{\frac{1}{2}+it}}\right|^{2}\,dt \\
          &+ O\left(\left(\int_{2}^{T}\left|\sum_{n \ll \frac{t}{\sqrt{\log(t)}}}\frac{1}{n^{\frac{1}{2}+it}}\right|^{2}\,dt\right)^{\frac{1}{2}}\sqrt{T}\log^{\frac{1}{4}}(T)\right)+o(T\log(T)).
        \end{aligned}
      \end{equation}
      Therefore, it suffices to show that the remaining integral is asymptotically equivalent to $T\log(T)$. We compute
      \begin{equation}\label{equ:second_moment_zeta_4}
        \begin{aligned}
          \left|\sum_{n \ll \frac{t}{\sqrt{\log(t)}}}\frac{1}{n^{\frac{1}{2}+it}}\right|^{2} &= \int_{2}^{T}\sum_{n,m \ll \frac{t}{\sqrt{\log(t)}}}\frac{1}{n^{\frac{1}{2}+it}m^{\frac{1}{2}-it}}\,dt \\
          &= \sum_{n,m \ll \frac{T}{\sqrt{\log(T)}}}\int_{\max(m,n,2)}^{T}\frac{1}{n^{\frac{1}{2}+it}m^{\frac{1}{2}-it}}\,dt && \text{FTT} \\
          &= \sum_{n \ll \frac{T}{\sqrt{\log(T)}}}\frac{T-n}{n}+ \sum_{\substack{n,m \ll \frac{T}{\sqrt{\log(T)}} \\ n \neq m}}\frac{1}{\sqrt{nm}}\int_{\max(n,m,2)}^{T}\left(\frac{m}{n}\right)^{it}\,dt \\
          &= T\sum_{n \ll \frac{T}{\sqrt{\log(T)}}}\frac{1}{n}+O\left(\sum_{n \ll \frac{T}{\sqrt{\log(T)}}}1\right)+\left(\sum_{\substack{n,m \ll \frac{T}{\sqrt{\log(T)}} \\ n \neq m}}\frac{1}{\sqrt{nm}\log\left(\frac{m}{n}\right)}\right),
        \end{aligned}
      \end{equation}
      where in the second to last line we have separated the terms for which $n = m$ or not. We now estimate all of the remaining terms in \cref{equ:second_moment_zeta_4}. For the first term, by \cref{equ:approximation_for_harmonic_sums} we have
      \begin{equation}\label{equ:second_moment_zeta_5}
        T\sum_{n \ll \frac{T}{\sqrt{\log(T)}}}\frac{1}{n} = T\log\left(\frac{T}{\sqrt{\log(T)}}\right)+O(T) = T\log(T)+o(T\log(T)).
      \end{equation}
      The second term is easier since
      \begin{equation}\label{equ:second_moment_zeta_6}
        \sum_{n \ll \frac{T}{\sqrt{\log(T)}}}1 = O\left(\frac{T}{\sqrt{\log(T)}}\right) = O(T) = o(T\log(T)).
      \end{equation}
      For the last term, separate the sum into the terms for which $n < \frac{m}{2}$ or not, so that
      \begin{equation}\label{equ:second_moment_zeta_7}
        \sum_{\substack{n,m \ll \frac{T}{\sqrt{\log(T)}} \\ n \neq m}}\frac{1}{\sqrt{nm}\log\left(\frac{m}{n}\right)} = \sum_{\substack{n,m \ll \frac{T}{\sqrt{\log(T)}} \\ n < \frac{m}{2}}}\frac{1}{\sqrt{nm}\log\left(\frac{m}{n}\right)}+\sum_{\substack{n,m \ll \frac{T}{\sqrt{\log(T)}} \\ n \ge \frac{m}{2} \\ n \neq m}}\frac{1}{\sqrt{nm}\log\left(\frac{m}{n}\right)}.
      \end{equation}
      In the first sum on the right-hand side of \cref{equ:second_moment_zeta_7} we have $\log\left(\frac{m}{n}\right) \ge \log\left(2\right)$ so that $\log\left(\frac{m}{n}\right)$ is bounded from below. Hence
      \[
        \sum_{\substack{n,m \ll \frac{T}{\sqrt{\log(T)}} \\ n < \frac{m}{2}}}\frac{1}{\sqrt{nm}\log\left(\frac{m}{n}\right)} = O\left(\sum_{\substack{n,m \ll \frac{T}{\sqrt{\log(T)}} \\ n < \frac{m}{2}}}\frac{1}{\sqrt{nm}}\right) = O\left(\left(\sum_{n \ll \frac{T}{\sqrt{\log(T)}}}\frac{1}{\sqrt{n}}\right)^{2}\right).
      \]
      But as
      \[
        \sum_{n \ll \frac{T}{\sqrt{\log(T)}}}\frac{1}{\sqrt{n}} = O\left(\int_{1}^{\frac{T}{\sqrt{\log(T)}}}\frac{1}{\sqrt{x}}\,dx\right) = O\left(\frac{\sqrt{T}}{\log^{\frac{1}{4}}(T)}\right),
      \]
      we find that
      \begin{equation}\label{equ:second_moment_zeta_8}
        \sum_{\substack{n,m \ll \frac{T}{\sqrt{\log(T)}} \\ n < \frac{m}{2}}}\frac{1}{\sqrt{nm}\log\left(\frac{m}{n}\right)} = O\left(\frac{T}{\log^{\frac{1}{2}}(T)}\right) = o(T\log(T)).
      \end{equation}
      For the second sum, write $n = m-r$ where $1 \le r \le \frac{m}{2}$ so that $\log\left(\frac{m}{n}\right) = \log\left(\frac{m}{m-r}\right) = -\log(1-\frac{r}{m}) \ge \frac{r}{m}$ where the inequality follows from the Taylor series of the logarithm. Whence
      \[
        \sum_{\substack{n,m \ll \frac{T}{\sqrt{\log(T)}} \\ n \ge \frac{m}{2} \\ n \neq m}}\frac{1}{\sqrt{nm}\log\left(\frac{m}{n}\right)} = O\left(\sum_{m \ll \frac{T}{\sqrt{\log(T)}}}\sum_{r \le \frac{m}{2}}\frac{m}{r\sqrt{m(m-r)}}\right) = O\left(\sum_{m \ll \frac{T}{\sqrt{\log(T)}}}\sum_{r \le \frac{m}{2}}\frac{1}{r}\right).
      \]
      To estimate the double sum, using \cref{equ:approximation_for_harmonic_sums} again, we have
      \[
        \sum_{r \le \frac{m}{2}}\frac{1}{r} = \log(m)+O(1).
      \]
      This estimate together with
      \[
        \sum_{m \ll \frac{T}{\sqrt{\log(T)}}}\log(m)+O(1) = \frac{T}{\sqrt{\log(T)}}\log\left(\frac{T}{\sqrt{\log(T)}}\right)+O\left(\frac{T}{\sqrt{\log(T)}}\right) = o(T\log(T)),
      \]
      gives
      \begin{equation}\label{equ:second_moment_zeta_9}
        \sum_{\substack{n,m \ll \frac{T}{\sqrt{\log(T)}} \\ n \ge \frac{m}{2} \\ n \neq m}}\frac{1}{\sqrt{nm}\log\left(\frac{m}{n}\right)} = T\log(T)+o(T\log(T)).
      \end{equation}
      Substituting \cref{equ:second_moment_zeta_8,equ:second_moment_zeta_9} into \cref{equ:second_moment_zeta_7} yields
      \begin{equation}\label{equ:second_moment_zeta_10}
        \sum_{\substack{n,m \ll \frac{T}{\sqrt{\log(T)}} \\ n \neq m}}\frac{1}{\sqrt{nm}\log\left(\frac{m}{n}\right)} = T\log(T)+o(T\log(T)).
      \end{equation}
      Then \cref{equ:second_moment_zeta_5,equ:second_moment_zeta_6,equ:second_moment_zeta_10} together with \cref{equ:second_moment_zeta_4} yields
      \begin{equation}\label{equ:second_moment_zeta_11}
        \left|\sum_{n \ll \frac{t}{\sqrt{\log(t)}}}\frac{1}{n^{\frac{1}{2}+it}}\right|^{2} = T\log(T)+o(T\log(T)).
      \end{equation}
      Applying \cref{equ:second_moment_zeta_11} to \cref{equ:second_moment_zeta_3}, we at last obtain
      \[
        M_{2}(T,\z) = T\log(T)+O(T\log^{\frac{3}{4}}(T))+o(T\log(T)) = T\log(T)+o(T\log(T)),
      \]
      which implies the desired asymptotic.
    \end{proof}
    
    A few comments about the proof of \cref{thm:second_moment_of_Riemann_zeta} are in order. The main analytic data being fed into \cref{thm:second_moment_of_Riemann_zeta} is the approximate functional equation for the Riemann zeta function (apart from some general asymptotics such as \cref{equ:approximation_for_harmonic_sums} and an application of the Cauchy-Schwarz inequality). The main term from the resulting asymptotic, namely \cref{equ:second_moment_zeta_1}, in $M_{2}(T,\z)$ is the $T\log(T)$ which comes from the diagonal contribution in
    \[
      \left|\sum_{n \ll \frac{t}{\sqrt{\log(t)}}}\frac{1}{n^{\frac{1}{2}+it}}\right|^{2} = \int_{2}^{T}\sum_{n,m \ll \frac{t}{\sqrt{\log(t)}}}\frac{1}{n^{\frac{1}{2}+it}m^{\frac{1}{2}-it}}\,dt,
    \]
    occurring when $n = m$, as computed in \cref{equ:second_moment_zeta_4}. This sum is
    \[
      T\sum_{n \ll \frac{T}{\sqrt{\log(T)}}}\frac{1}{n},
    \]
    and it is estimated in \cref{equ:second_moment_zeta_5}. This is the key obstruction in the proof of \cref{thm:second_moment_of_Riemann_zeta} as this is the longest sum coming from \cref{equ:second_moment_zeta_2} of the highest order (because of the factor of $T$).