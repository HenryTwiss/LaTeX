\chapter{Moment Results}
  Obtaining any type of moment result is, generally speaking, a very difficult task requiring delicate analytic techniques. Since moments of $L$-functions are still a modern topic of research, there is not yet a universal theory to obtain such results. Even worse, currently methods almost only work when $k \le 4$. So, rather than trying to develop a general theory, we highlight some infamous moment results and their techniques.
  \section{The Second Moment of the Riemann Zeta Function}
    We will discuss two results of increasing strength concerning the second moment of the Riemann zeta function. The primary aim here is to illustrate how inputting more refined analytic data results in improved asymptotics. In particular, will prove the infamous result of Hardy and Littlewood (see \cite{hardy1916contributions}). Throughout, it will be useful to recall that
    \begin{equation}\label{equ:approximation_for_harmonic_sums}
      \sum_{n \ll N}\frac{1}{n} = \log(N)+O(1),
    \end{equation}
    for any $N \ge 1$, which follows from recalling that the Euler-Mascheroni constant $\g$ is defined as the limit
    \[
      \g = \lim_{N \to \infty}\left(\sum_{n \le N}\frac{1}{n}-\log(N)\right).
    \]
    Our first result will only require the Euler-Maclaurin summation formula. However, we first need the following lemma:

    \begin{lemma}\label{lem:weak_approximate_functional_equation_Riemann_zeta}
      For any $X > 0$ and fixed $\d > 0$, we have
      \[
        \z(s) = \sum_{n \le X}\frac{1}{n^{s}}-\frac{X^{1-s}}{1-s}+O\left(\frac{1}{X^{\s}}\right),
      \]
      provided $\d < \s < 1$ and $t \ll X$.
    \end{lemma}
    \begin{proof}
      Let $N$ be an integer such that $N > X$, take $\s > 1$, and consider
      \[
        \z(s)-\sum_{n \le N}\frac{1}{n^{s}} = \sum_{n > N}\frac{1}{n^{s}}.
      \]
      As $A(M) \le M$ and $\s > 1$, we have $A(M)M^{-s} \to 0$ as $M \to \infty$. Then \cref{cor:Euler_Maclaurin_summation_formula_integer_limit_version_specialization} implies that we can rewrite the sum on the right-hand side as
      \[
        \sum_{n > N}\frac{1}{n^{s}} = \frac{N^{-s}}{2}+\int_{N}^{\infty}u^{-s}+\left(u-\lfloor u \rfloor-\frac{1}{2}\right)-su^{-(s+1)}\,du.
      \]
      Since
      \[
        \int_{N}^{\infty}u^{-s}\,du = -\frac{N^{1-s}}{1-s},
      \]
      we further have
      \[
        \sum_{n > N}\frac{1}{n^{s}} = \frac{N^{-s}}{2}-\frac{N^{1-s}}{1-s}-s\int_{N}^{\infty}\left(u-\lfloor u \rfloor-\frac{1}{2}\right)u^{-(s+1)}\,du.
      \]
      Therefore we arrive at
      \begin{equation}\label{equ:weak_approximate_functional_equation_Riemann_zeta_1}
        \z(s) = \sum_{n \le N}\frac{1}{n^{s}}+\frac{N^{-s}}{2}-\frac{N^{1-s}}{1-s}-s\int_{N}^{\infty}\left(u-\lfloor u \rfloor-\frac{1}{2}\right)u^{-(s+1)}\,du,
      \end{equation}
      for $\s > 1$. In fact, the right-hand side of \cref{equ:weak_approximate_functional_equation_Riemann_zeta_1} is meromorphic for $\s > 0$. Indeed, as the first three terms are with the third term having a simple pole at $s = 1$, so it suffices to show that the integral is as well. To do so we will show that the integral is locally absolutely uniformly convergent in this region. Indeed, let $K$ is a compact subset in this region and set $\a = \min_{s \in K}(\s)$ and $\b = \max_{s \in K}(|s|)$. Then we have to show that the integral is absolutely uniformly convergent on $K$. As $\lfloor u \rfloor-u-\frac{1}{2} \ll 1$, we have
      \[
        s\int_{N}^{\infty}\left(u-\lfloor u \rfloor-\frac{1}{2}\right)u^{-(s+1)}\,du \ll \b\int_{x}^{\infty}\frac{1}{u^{\a+1}}\,du = \frac{\b}{\a x^{\a}} \ll_{\a,\b} 1,
      \]
      as desired. Therefore \cref{equ:weak_approximate_functional_equation_Riemann_zeta_1} for $\s > 0$. Now let $\d < \s < 1$ and $t \ll X$. Then $|s| \ll X$ and $\s$ is bounded away from zero. From the analogous estimate
      \begin{equation}\label{equ:weak_approximate_functional_equation_Riemann_zeta_2}
        s\int_{N}^{\infty}\left(u-\lfloor u \rfloor-\frac{1}{2}\right)u^{-(s+1)}\,du \ll |s|\int_{N}^{\infty}\frac{1}{u^{\s+1}}\,du \ll \frac{|s|}{N^{\s}},
      \end{equation}
      and \cref{equ:weak_approximate_functional_equation_Riemann_zeta_1}, we arrive at
      \begin{equation}\label{equ:weak_approximate_functional_equation_Riemann_zeta_3}
        \z(s) = \sum_{n \le N}\frac{1}{n^{s}}-\frac{N^{1-s}}{1-s}+O\left(\frac{|s|+1}{N^{\s}}\right).
      \end{equation}
      We will refine \cref{equ:weak_approximate_functional_equation_Riemann_zeta_3} by truncating the remaining sum. Write
      \[
        \sum_{n \le N}\frac{1}{n^{s}} = \sum_{n \le X}\frac{1}{n^{s}}+\sum_{X < n \le N}\frac{1}{n^{s}}.
      \]
      Applying the Euler-Maclaurin summation formula to the last sum gives
      \begin{equation}\label{equ:weak_approximate_functional_equation_Riemann_zeta_4}
        \begin{aligned}
          \sum_{X < n \le N}\frac{1}{n^{s}} &= \left(X-\lfloor X \rfloor-\frac{1}{2}\right)X^{-s}+\frac{1}{2}N^{-s} \\
          &+ \int_{X}^{N}u^{-s}\,du-s\int_{X}^{N}\left(u-\lfloor u \rfloor-\frac{1}{2}\right)u^{-(s+1)}\,du.
        \end{aligned}
      \end{equation}
      Now observe
      \begin{equation}\label{equ:weak_approximate_functional_equation_Riemann_zeta_5}
        \int_{X}^{N}u^{-s}\,du = \frac{N^{1-s}}{1-s}-\frac{X^{1-s}}{1-s}.
      \end{equation}
      Inserting the estimates in \cref{equ:weak_approximate_functional_equation_Riemann_zeta_2,equ:weak_approximate_functional_equation_Riemann_zeta_5} into \cref{equ:weak_approximate_functional_equation_Riemann_zeta_4} yields
      \begin{equation}\label{equ:weak_approximate_functional_equation_Riemann_zeta_6}
        \sum_{X < n \le N}\frac{1}{n^{s}} = \frac{N^{1-s}}{1-s}-\frac{X^{1-s}}{1-s}+O\left(\frac{|s|}{N^{\s}}+\frac{1}{X^{\s}}\right),
      \end{equation}
      where we have also used the fact that $\left(X-\lfloor X \rfloor-\frac{1}{2}\right)X^{-s} \ll \frac{1}{X^{\s}}$. Upon combining \cref{equ:weak_approximate_functional_equation_Riemann_zeta_3,equ:weak_approximate_functional_equation_Riemann_zeta_6}, we have
      \[
        \z(s) = \sum_{n \le X}\frac{1}{n^{s}}-\frac{X^{1-s}}{1-s}+O\left(\frac{|s|}{N^{\s}}+\frac{1}{X^{\s}}\right).
      \]
      Taking the limit as $N \to \infty$ completes the proof.
    \end{proof}
    
    \cref{lem:weak_approximate_functional_equation_Riemann_zeta} can be thought of as a weak approximate functional equation for the Riemann zeta function. Our first result bounds the growth rate of $M_{2}(T,\z)$ only using \cref{lem:weak_approximate_functional_equation_Riemann_zeta}:

    \begin{theorem}\label{thm:second_moment_of_Riemann_zeta_big_O}
      For $T > 2$,
      \[
        M_{2}(T,\z) = O(T\log(T)).
      \]
    \end{theorem}
    \begin{proof}
      Taking $s = \frac{1}{2}+it$ and $X = t$ for $t \ge 2$ in \cref{lem:weak_approximate_functional_equation_Riemann_zeta} gives
      \[
        \z\left(\frac{1}{2}+it\right) = \sum_{n \le t}\frac{1}{n^{\frac{1}{2}+it}}-\frac{t^{\frac{1}{2}-it}}{\frac{1}{2}-it}+O\left(\frac{1}{\sqrt{t}}\right).
      \]
      This implies the weaker estimate
      \begin{equation}\label{equ:second_moment_of_Riemann_zeta_big_O_1}
        \z\left(\frac{1}{2}+it\right) = \sum_{n \le t}\frac{1}{n^{\frac{1}{2}+it}}+O\left(\frac{1}{\sqrt{t}}\right).
      \end{equation}
      Applying \cref{equ:second_moment_of_Riemann_zeta_big_O_1} to the definition of $M_{2}(T,\z)$ and recalling that $\z\left(\frac{1}{2}+it\right)$ is absolutely bounded for $0 \le t \le 2$, yields
      \begin{equation}\label{equ:second_moment_of_Riemann_zeta_big_O_2}
        \begin{aligned}
          M_{2}(T,\z) &= \int_{2}^{T}\left|\sum_{n \le t}\frac{1}{n^{\frac{1}{2}+it}}\right|^{2}\,dt \\
          &+ O\left(\int_{2}^{T}\left|\sum_{n \le t}\frac{1}{n^{\frac{1}{2}+it}}\right|\frac{1}{\sqrt{t}}\,dt\right)+O\left(\int_{2}^{T}\frac{1}{t}\,dt\right).
        \end{aligned}
      \end{equation}
      We will now simplify \cref{equ:second_moment_of_Riemann_zeta_big_O_2}. For the second term, the Cauchy-Schwarz inequality gives
      \begin{align*}
        \int_{2}^{T}\left|\sum_{n \le t}\frac{1}{n^{\frac{1}{2}+it}}\right|\frac{1}{\sqrt{t}}\,dt &= O\left(\left(\int_{2}^{T}\left|\sum_{n \le t}\frac{1}{n^{\frac{1}{2}+it}}\right|^{2}\int_{2}^{T}\frac{1}{t}\,dt\right)^{\frac{1}{2}}\right) \\
        &= O\left(\left(\int_{2}^{T}\left|\sum_{n \le t}\frac{1}{n^{\frac{1}{2}+it}}\right|^{2}\,dt\right)^{\frac{1}{2}}\sqrt{\log(T)}\right),
      \end{align*}
      where in the second line we have made use of the estimate
      \[
        \int_{2}^{T}\frac{1}{t}\,dt = O(\log(T)).
      \]
      Applying this latter estimate to third term as well, \cref{equ:second_moment_of_Riemann_zeta_big_O_2} becomes
      \begin{equation}\label{equ:second_moment_of_Riemann_zeta_big_O_3}
        \begin{aligned}
          M_{2}(T,\z) &= \int_{2}^{T}\left|\sum_{n \le t}\frac{1}{n^{\frac{1}{2}+it}}\right|^{2}\,dt \\
          &+ O\left(\left(\int_{2}^{T}\left|\sum_{n \le t}\frac{1}{n^{\frac{1}{2}+it}}\right|^{2}\,dt\right)^{\frac{1}{2}}\sqrt{\log(T)}\right)+O(\log(T)).
        \end{aligned}
      \end{equation}
      Therefore, it suffices to show that the remaining integral is $O\left(T\log(T)\right)$. We compute
      \begin{equation}\label{equ:second_moment_of_Riemann_zeta_big_O_4}
        \begin{aligned}
           \int_{2}^{T}\left|\sum_{n \le t}\frac{1}{n^{\frac{1}{2}+it}}\right|^{2} &= \int_{2}^{T}\sum_{n,m \le t}\frac{1}{n^{\frac{1}{2}+it}m^{\frac{1}{2}-it}}\,dt \\
          &= \sum_{n,m \le T}\int_{\max(m,n,2)}^{T}\frac{1}{n^{\frac{1}{2}+it}m^{\frac{1}{2}-it}}\,dt && \text{FTT} \\
          &= \sum_{n \le T}\frac{T-n}{n}+ \sum_{\substack{n,m \le T \\ n \neq m}}\frac{1}{\sqrt{nm}}\int_{\max(n,m,2)}^{T}\left(\frac{m}{n}\right)^{it}\,dt \\
          &= T\sum_{n \le T}\frac{1}{n}+O\left(\sum_{n \le T}1\right)+\left(\sum_{\substack{n,m \le T \\ n \neq m}}\frac{1}{\sqrt{nm}\log\left(\frac{m}{n}\right)}\right),
        \end{aligned}
      \end{equation}
      where in the second to last line we have separated the terms for which $n = m$ or not. We now estimate all of the remaining terms in \cref{equ:second_moment_of_Riemann_zeta_big_O_4}. For the first term, by \cref{equ:approximation_for_harmonic_sums} we have
      \begin{equation}\label{equ:second_moment_of_Riemann_zeta_big_O_5}
        T\sum_{n \le T}\frac{1}{n} = T\log(T)+O(T) = O(T\log(T)).
      \end{equation}
      The second term is easier since
      \begin{equation}\label{equ:second_moment_of_Riemann_zeta_big_O_6}
        \sum_{n \le T}1 = O(T) = O(T\log(T)).
      \end{equation}
      For the last term, separate the sum into the terms for which $n < \frac{m}{2}$ or not, so that
      \begin{equation}\label{equ:second_moment_of_Riemann_zeta_big_O_7}
        \sum_{\substack{n,m \le T \\ n \neq m}}\frac{1}{\sqrt{nm}\log\left(\frac{m}{n}\right)} = \sum_{\substack{n,m \le T \\ n < \frac{m}{2}}}\frac{1}{\sqrt{nm}\log\left(\frac{m}{n}\right)}+\sum_{\substack{n,m \le T \\ n \ge \frac{m}{2} \\ n \neq m}}\frac{1}{\sqrt{nm}\log\left(\frac{m}{n}\right)}.
      \end{equation}
      In the first sum on the right-hand side of \cref{equ:second_moment_of_Riemann_zeta_big_O_7} we have $\log\left(\frac{m}{n}\right) \ge \log\left(2\right)$ so that $\log\left(\frac{m}{n}\right)$ is bounded from below. Hence
      \[
        \sum_{\substack{n,m \le T \\ n < \frac{m}{2}}}\frac{1}{\sqrt{nm}\log\left(\frac{m}{n}\right)} = O\left(\sum_{\substack{n,m \le T \\ n < \frac{m}{2}}}\frac{1}{\sqrt{nm}}\right) = O\left(\left(\sum_{n \le T}\frac{1}{\sqrt{n}}\right)^{2}\right).
      \]
      But as
      \[
        \sum_{n \le T}\frac{1}{\sqrt{n}} = O\left(\int_{1}^{T}\frac{1}{\sqrt{x}}\,dx\right) = O(\sqrt{T}),
      \]
      we find that
      \begin{equation}\label{equ:second_moment_of_Riemann_zeta_big_O_8}
        \sum_{\substack{n,m \le T \\ n < \frac{m}{2}}}\frac{1}{\sqrt{nm}\log\left(\frac{m}{n}\right)} = O(T) = O(T\log(T)).
      \end{equation}
      For the second sum, write $n = m-r$ where $1 \le r \le \frac{m}{2}$ so that $\log\left(\frac{m}{n}\right) = \log\left(\frac{m}{m-r}\right) = -\log(1-\frac{r}{m}) \ge \frac{r}{m}$ where the inequality follows from the Taylor series of the logarithm. Whence
      \[
        \sum_{\substack{n,m \le T \\ n \ge \frac{m}{2} \\ n \neq m}}\frac{1}{\sqrt{nm}\log\left(\frac{m}{n}\right)} = O\left(\sum_{m \le T}\sum_{r \le \frac{m}{2}}\frac{m}{r\sqrt{m(m-r)}}\right) = O\left(\sum_{m \le T}\sum_{r \le \frac{m}{2}}\frac{1}{r}\right).
      \]
      To estimate the double sum, using \cref{equ:approximation_for_harmonic_sums} again, we have
      \[
        \sum_{r \le \frac{m}{2}}\frac{1}{r} = \log(m)+O(1).
      \]
      This estimate together with
      \[
        \sum_{m \le T}\log(m)+O(1) = T\log(T)+O(T) = O(T\log(T)),
      \]
      gives
      \begin{equation}\label{equ:second_moment_of_Riemann_zeta_big_O_9}
        \sum_{\substack{n,m \le T \\ n \ge \frac{m}{2} \\ n \neq m}}\frac{1}{\sqrt{nm}\log\left(\frac{m}{n}\right)} = O(T\log(T)).
      \end{equation}
      Substituting \cref{equ:second_moment_of_Riemann_zeta_big_O_8,equ:second_moment_of_Riemann_zeta_big_O_9} into \cref{equ:second_moment_of_Riemann_zeta_big_O_7} yields
      \begin{equation}\label{equ:second_moment_of_Riemann_zeta_big_O_10}
        \sum_{\substack{n,m \le T \\ n \neq m}}\frac{1}{\sqrt{nm}\log\left(\frac{m}{n}\right)} = O(T\log(T)).
      \end{equation}
      Then \cref{equ:second_moment_of_Riemann_zeta_big_O_5,equ:second_moment_of_Riemann_zeta_big_O_6,equ:second_moment_of_Riemann_zeta_big_O_10} together with \cref{equ:second_moment_of_Riemann_zeta_big_O_4} yields
      \begin{equation}\label{equ:second_moment_of_Riemann_zeta_big_O_11}
        \int_{2}^{T}\left|\sum_{n \le t}\frac{1}{n^{\frac{1}{2}+it}}\right|^{2} = O(T\log(T)).
      \end{equation}
      Applying \cref{equ:second_moment_of_Riemann_zeta_big_O_11} to \cref{equ:second_moment_of_Riemann_zeta_big_O_3}, we at last obtain
      \[
        M_{2}(T,\z) = O(T\log(T))+O(\sqrt{T}\log(T))+O(T) = O(T\log(T)),
      \]
      as desired.
    \end{proof}

    As $\log(T) \ll_{\e} T^{\e}$, \cref{thm:second_moment_of_Riemann_zeta_big_O} already implies the required estimate needed for the second moment in \cref{prop:equivalence_Lindelof_hypothesis_and_moments} in the case of the Riemann zeta function. We can also improve \cref{thm:second_moment_of_Riemann_zeta_big_O} to asymptotic equivalence, but this requires the approximate functional equation for the Riemann zeta function. The essential reason is that the main contributions from the resulting the asymptotic, namely \cref{equ:second_moment_of_Riemann_zeta_big_O_1}, in $M_{2}(T,\z)$ come from the terms in
    \[
      \int_{2}^{T}\left|\sum_{n \le t}\frac{1}{n^{\frac{1}{2}+it}}\right|^{2} = \int_{2}^{T}\sum_{n,m \le t}\frac{1}{n^{\frac{1}{2}+it}m^{\frac{1}{2}-it}}\,dt,
    \]
    when $n = m$ and when $n \ge \frac{m}{2}$ but $n \neq m$. The corresponding sums are
    \[
      T\sum_{n \le T}\frac{1}{n} \quad \text{and} \quad \sum_{\substack{n,m \le T \\ n \ge \frac{m}{2} \\ n \neq m}}\frac{1}{\sqrt{nm}\log\left(\frac{m}{n}\right)},
    \]
    and are estimated in \cref{equ:second_moment_of_Riemann_zeta_big_O_5,equ:second_moment_of_Riemann_zeta_big_O_9} respectively. In order to obtain asymptotic equivalence we need to shorten these sums so that one of them is $o(T\log(T))$. In effect, the sum in \cref{lem:weak_approximate_functional_equation_Riemann_zeta} must be shortened. We will demonstrate this by improving \cref{thm:second_moment_of_Riemann_zeta_big_O} to the infamous result of Hardy and Littlewood (see \cite{hardy1916contributions}):

    \begin{theorem}\label{thm:second_moment_of_Riemann_zeta_asymptotic_equivalence}
      For $T > 2$,
      \[
        M_{2}(T,\z) \sim T\log(T).
      \]
    \end{theorem}
    \begin{proof}
      Taking $s = \frac{1}{2}+it$, $X = \sqrt{\frac{t}{\log(t)}}$ for $t \ge 2$, and $\Phi(u) = \cos^{-4M}\left(\frac{\pi u}{4M}\right)$ with $M \ge 1$ in the approximate functional equation gives
      \begin{align*}
        \z\left(\frac{1}{2}+it\right) &= \sum_{n \ge 1}\frac{1}{n^{\frac{1}{2}+it}}V_{\frac{1}{2}+it}\left(n\sqrt{\frac{\log(t)}{t}}\right) \\
        &+ \e\left(\frac{1}{2}+it,\z\right)\sum_{n \ge 1}\frac{1}{n^{\frac{1}{2}-it}}V_{\frac{1}{2}-it}\left(n\sqrt{\frac{t}{\log(t)}}\right)+\frac{R\left(\frac{1}{2}+it,\sqrt{\frac{t}{\log(t)}},\z\right)}{\g\left(\frac{1}{2}+it,\z\right)}.
      \end{align*}
      By \cref{prop:V_function_decay}, we find that $V_{\frac{1}{2}+it}\left(n\sqrt{\frac{\log(t)}{t}}\right)$ and $V_{\frac{1}{2}-it}\left(n\sqrt{\frac{t}{\log(t)}}\right)$ are bounded for $n \ll \frac{t}{\sqrt{\log(t)}}$ and $n \ll \sqrt{\log(t)}$ respectively and then exhibit polynomial decay of arbitrarily large order thereafter. Moreover, \cref{equ:modified_gamma_estimates,equ:choice_for_V_decay_estimate} together imply that the third term exhibits exponential decay and is therefore absolutely bounded. Lastly, $\e\left(\frac{1}{2}+it,\z\right)$ is absolutely bounded by \cref{equ:gamma_factor_analytic_conductor_estimate}. Altogether, this means
      \[
        \z\left(\frac{1}{2}+it\right) = \sum_{n \ll \frac{t}{\sqrt{\log(t)}}}\frac{1}{n^{\frac{1}{2}+it}}+O\left(\sum_{n \ll \sqrt{\log(t)}}\frac{1}{\sqrt{n}}\right)+O(1),
      \]
      and using the estimate
      \[
        \sum_{n \ll \sqrt{\log(t)}}\frac{1}{\sqrt{n}} = O\left(\int_{1}^{\sqrt{\log(t)}}\frac{1}{\sqrt{x}}\,dx\right) = O\left(\log^{\frac{1}{4}}(t)\right),
      \]
      we arrive at
      \begin{equation}\label{equ:second_moment_zeta_asymptotic_equivalence_1}
        \z\left(\frac{1}{2}+it\right) = \sum_{n \ll \frac{t}{\sqrt{\log(t)}}}\frac{1}{n^{\frac{1}{2}+it}}+O\left(\log^{\frac{1}{4}}(t)\right),
      \end{equation}
      since $1 = O\left(\log^{\frac{1}{4}}(t)\right)$ for $t \ge 2$. Applying \cref{equ:second_moment_zeta_asymptotic_equivalence_1} to the definition of $M_{2}(T,\z)$ and recalling that $\z\left(\frac{1}{2}+it\right)$ is absolutely bounded for $0 \le t \le 2$, yields
      \begin{equation}\label{equ:second_moment_zeta_asymptotic_equivalence_2}
        \begin{aligned}
          M_{2}(T,\z) &= \int_{2}^{T}\left|\sum_{n \ll \frac{t}{\sqrt{\log(t)}}}\frac{1}{n^{\frac{1}{2}+it}}\right|^{2}\,dt \\
          &+ O\left(\int_{2}^{T}\left|\sum_{n \ll \frac{t}{\sqrt{\log(t)}}}\frac{1}{n^{\frac{1}{2}+it}}\right|\log^{\frac{1}{4}}(t)\,dt\right)+O\left(\int_{2}^{T}\sqrt{\log(t)}\,dt\right).
        \end{aligned}
      \end{equation}
      We will now simplify \cref{equ:second_moment_zeta_asymptotic_equivalence_2}. For the second term, the Cauchy-Schwarz inequality gives
      \begin{align*}
        \int_{2}^{T}\left|\sum_{n \ll \frac{t}{\sqrt{\log(t)}}}\frac{1}{n^{\frac{1}{2}+it}}\right|\log^{\frac{1}{4}}(t)\,dt &= O\left(\left(\int_{2}^{T}\left|\sum_{n \ll \frac{t}{\sqrt{\log(t)}}}\frac{1}{n^{\frac{1}{2}+it}}\right|^{2}\int_{2}^{T}\sqrt{\log(t)}\,dt\right)^{\frac{1}{2}}\right) \\
        &= O\left(\left(\int_{2}^{T}\left|\sum_{n \ll \frac{t}{\sqrt{\log(t)}}}\frac{1}{n^{\frac{1}{2}+it}}\right|^{2}\,dt\right)^{\frac{1}{2}}\sqrt{T}\log^{\frac{1}{4}}(T)\right),
      \end{align*}
      where in the second line we have made use of the estimate
      \[
        \int_{2}^{T}\sqrt{\log(t)}\,dt = T\sqrt{\log(T)}+O\left(\int_{2}^{T}\frac{1}{\sqrt{\log(t)}}\,dt\right) = T\sqrt{\log(T)}+O(T) = O\left(T\sqrt{\log(T)}\right),
      \]
      which follows from an application of integration by parts. In fact, this estimate shows that the integral is also $o(T\log(T))$. Applying this latter estimate to third term as well, \cref{equ:second_moment_zeta_asymptotic_equivalence_2} becomes
      \begin{equation}\label{equ:second_moment_zeta_asymptotic_equivalence_3}
        \begin{aligned}
          M_{2}(T,\z) &= \int_{2}^{T}\left|\sum_{n \ll \frac{t}{\sqrt{\log(t)}}}\frac{1}{n^{\frac{1}{2}+it}}\right|^{2}\,dt \\
          &+ O\left(\left(\int_{2}^{T}\left|\sum_{n \ll \frac{t}{\sqrt{\log(t)}}}\frac{1}{n^{\frac{1}{2}+it}}\right|^{2}\,dt\right)^{\frac{1}{2}}\sqrt{T}\log^{\frac{1}{4}}(T)\right)+o(T\log(T)).
        \end{aligned}
      \end{equation}
      Therefore, it suffices to show that the remaining integral is asymptotically equivalent to $T\log(T)$. We first expand the sum and interchange it with the integral by the Fubini-Tonelli theorem:
      \begin{equation}\label{equ:second_moment_zeta_asymptotic_equivalence_4}
        \begin{aligned}
           \int_{2}^{T}\left|\sum_{n \ll \frac{t}{\sqrt{\log(t)}}}\frac{1}{n^{\frac{1}{2}+it}}\right|^{2} &= \int_{2}^{T}\sum_{n,m \ll \frac{t}{\sqrt{\log(t)}}}\frac{1}{n^{\frac{1}{2}+it}m^{\frac{1}{2}-it}}\,dt \\
          &= \sum_{n,m \ll \frac{T}{\sqrt{\log(T)}}}\int_{\max(m,n,2)}^{T}\frac{1}{n^{\frac{1}{2}+it}m^{\frac{1}{2}-it}}\,dt \\
          &= \sum_{n \ll \frac{T}{\sqrt{\log(T)}}}\frac{T-n}{n}+ \sum_{\substack{n,m \ll \frac{T}{\sqrt{\log(T)}} \\ n \neq m}}\frac{1}{\sqrt{nm}}\int_{\max(n,m,2)}^{T}\left(\frac{m}{n}\right)^{it}\,dt \\
          &= T\sum_{n \ll \frac{T}{\sqrt{\log(T)}}}\frac{1}{n}+O\left(\sum_{n \ll \frac{T}{\sqrt{\log(T)}}}1\right)+\left(\sum_{\substack{n,m \ll \frac{T}{\sqrt{\log(T)}} \\ n \neq m}}\frac{1}{\sqrt{nm}\log\left(\frac{m}{n}\right)}\right),
        \end{aligned}
      \end{equation}
      where in the third line we have separated the terms for which $n = m$ or not. We now estimate all of the remaining terms in \cref{equ:second_moment_zeta_asymptotic_equivalence_4}. For the first term, by \cref{equ:approximation_for_harmonic_sums} we have
      \begin{equation}\label{equ:second_moment_zeta_asymptotic_equivalence_5}
        T\sum_{n \ll \frac{T}{\sqrt{\log(T)}}}\frac{1}{n} = T\log\left(\frac{T}{\sqrt{\log(T)}}\right)+O(T) = T\log(T)+o(T\log(T)).
      \end{equation}
      The second term is easier since
      \begin{equation}\label{equ:second_moment_zeta_asymptotic_equivalence_6}
        \sum_{n \ll \frac{T}{\sqrt{\log(T)}}}1 = O\left(\frac{T}{\sqrt{\log(T)}}\right) = O(T) = o(T\log(T)).
      \end{equation}
      For the last term, separate the sum into the terms for which $n < \frac{m}{2}$ or not, so that
      \begin{equation}\label{equ:second_moment_zeta_asymptotic_equivalence_7}
        \sum_{\substack{n,m \ll \frac{T}{\sqrt{\log(T)}} \\ n \neq m}}\frac{1}{\sqrt{nm}\log\left(\frac{m}{n}\right)} = \sum_{\substack{n,m \ll \frac{T}{\sqrt{\log(T)}} \\ n < \frac{m}{2}}}\frac{1}{\sqrt{nm}\log\left(\frac{m}{n}\right)}+\sum_{\substack{n,m \ll \frac{T}{\sqrt{\log(T)}} \\ n \ge \frac{m}{2} \\ n \neq m}}\frac{1}{\sqrt{nm}\log\left(\frac{m}{n}\right)}.
      \end{equation}
      In the first sum on the right-hand side of \cref{equ:second_moment_zeta_asymptotic_equivalence_7} we have $\log\left(\frac{m}{n}\right) \ge \log\left(2\right)$ so that $\log\left(\frac{m}{n}\right)$ is bounded from below. Hence
      \[
        \sum_{\substack{n,m \ll \frac{T}{\sqrt{\log(T)}} \\ n < \frac{m}{2}}}\frac{1}{\sqrt{nm}\log\left(\frac{m}{n}\right)} = O\left(\sum_{\substack{n,m \ll \frac{T}{\sqrt{\log(T)}} \\ n < \frac{m}{2}}}\frac{1}{\sqrt{nm}}\right) = O\left(\left(\sum_{n \ll \frac{T}{\sqrt{\log(T)}}}\frac{1}{\sqrt{n}}\right)^{2}\right).
      \]
      But as
      \[
        \sum_{n \ll \frac{T}{\sqrt{\log(T)}}}\frac{1}{\sqrt{n}} = O\left(\int_{1}^{\frac{T}{\sqrt{\log(T)}}}\frac{1}{\sqrt{x}}\,dx\right) = O\left(\frac{\sqrt{T}}{\log^{\frac{1}{4}}(T)}\right),
      \]
      we find that
      \begin{equation}\label{equ:second_moment_zeta_asymptotic_equivalence_8}
        \sum_{\substack{n,m \ll \frac{T}{\sqrt{\log(T)}} \\ n < \frac{m}{2}}}\frac{1}{\sqrt{nm}\log\left(\frac{m}{n}\right)} = O\left(\frac{T}{\sqrt{\log(T)}}\right) = o(T\log(T)).
      \end{equation}
      For the second sum, write $n = m-r$ where $1 \le r \le \frac{m}{2}$ so that $\log\left(\frac{m}{n}\right) = \log\left(\frac{m}{m-r}\right) = -\log(1-\frac{r}{m}) \ge \frac{r}{m}$ where the inequality follows from the Taylor series of the logarithm. Whence
      \[
        \sum_{\substack{n,m \ll \frac{T}{\sqrt{\log(T)}} \\ n \ge \frac{m}{2} \\ n \neq m}}\frac{1}{\sqrt{nm}\log\left(\frac{m}{n}\right)} = O\left(\sum_{m \ll \frac{T}{\sqrt{\log(T)}}}\sum_{r \le \frac{m}{2}}\frac{m}{r\sqrt{m(m-r)}}\right) = O\left(\sum_{m \ll \frac{T}{\sqrt{\log(T)}}}\sum_{r \le \frac{m}{2}}\frac{1}{r}\right).
      \]
      To estimate the double sum, using \cref{equ:approximation_for_harmonic_sums} again, we have
      \[
        \sum_{r \le \frac{m}{2}}\frac{1}{r} = \log(m)+O(1).
      \]
      This estimate together with
      \[
        \sum_{m \ll \frac{T}{\sqrt{\log(T)}}}\log(m)+O(1) = \frac{T}{\sqrt{\log(T)}}\log\left(\frac{T}{\sqrt{\log(T)}}\right)+O\left(\frac{T}{\sqrt{\log(T)}}\right) = o(T\log(T)),
      \]
      gives
      \begin{equation}\label{equ:second_moment_zeta_asymptotic_equivalence_9}
        \sum_{\substack{n,m \ll \frac{T}{\sqrt{\log(T)}} \\ n \ge \frac{m}{2} \\ n \neq m}}\frac{1}{\sqrt{nm}\log\left(\frac{m}{n}\right)} = o(T\log(T)).
      \end{equation}
      Substituting \cref{equ:second_moment_zeta_asymptotic_equivalence_8,equ:second_moment_zeta_asymptotic_equivalence_9} into \cref{equ:second_moment_zeta_asymptotic_equivalence_7} yields
      \begin{equation}\label{equ:second_moment_zeta_asymptotic_equivalence_10}
        \sum_{\substack{n,m \ll \frac{T}{\sqrt{\log(T)}} \\ n \neq m}}\frac{1}{\sqrt{nm}\log\left(\frac{m}{n}\right)} = o(T\log(T)).
      \end{equation}
      Then \cref{equ:second_moment_zeta_asymptotic_equivalence_5,equ:second_moment_zeta_asymptotic_equivalence_6,equ:second_moment_zeta_asymptotic_equivalence_10} together with \cref{equ:second_moment_zeta_asymptotic_equivalence_4} yields
      \begin{equation}\label{equ:second_moment_zeta_asymptotic_equivalence_11}
        \int_{2}^{T}\left|\sum_{n \ll \frac{t}{\sqrt{\log(t)}}}\frac{1}{n^{\frac{1}{2}+it}}\right|^{2} = T\log(T)+o(T\log(T)).
      \end{equation}
      Applying \cref{equ:second_moment_zeta_asymptotic_equivalence_11} to \cref{equ:second_moment_zeta_asymptotic_equivalence_3}, we at last obtain
      \[
        M_{2}(T,\z) = T\log(T)+O(T\log^{\frac{3}{4}}(T))+o(T\log(T)) = T\log(T)+o(T\log(T)),
      \]
      which implies the desired asymptotic.
    \end{proof}
    
    A few comments about the proof of \cref{thm:second_moment_of_Riemann_zeta_asymptotic_equivalence} are in order. The main analytic data being fed into \cref{thm:second_moment_of_Riemann_zeta_asymptotic_equivalence} is the approximate functional equation for the Riemann zeta function. The main contribution from the resulting asymptotic, namely \cref{equ:second_moment_zeta_asymptotic_equivalence_1}, in $M_{2}(T,\z)$ is the $T\log(T)$ which comes from the diagonal contribution in
    \[
      \int_{2}^{T}\left|\sum_{n \ll \frac{t}{\sqrt{\log(t)}}}\frac{1}{n^{\frac{1}{2}+it}}\right|^{2} = \int_{2}^{T}\sum_{n,m \ll \frac{t}{\sqrt{\log(t)}}}\frac{1}{n^{\frac{1}{2}+it}m^{\frac{1}{2}-it}}\,dt,
    \]
    occurring when $n = m$, as computed in \cref{equ:second_moment_zeta_asymptotic_equivalence_4}. This sum is
    \[
      T\sum_{n \ll \frac{T}{\sqrt{\log(T)}}}\frac{1}{n},
    \]
    and it is estimated in \cref{equ:second_moment_zeta_asymptotic_equivalence_5}. This is the longest sum coming from \cref{equ:second_moment_zeta_asymptotic_equivalence_2} of highest order (because of the factor of $T$).