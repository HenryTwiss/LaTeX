\chapter{Moment Results}
  Obtaining any type of moment result is, generally speaking, a very difficult task requiring delicate analytic techniques. Since moments of $L$-functions are still a modern topic of research, there is not yet a universal theory to obtain such results. Even worse, currently methods almost only work when $k \le 4$. So, rather than trying to develop a general theory, we highlight some infamous moment results and their techniques.
  \section{\todo{The Second Moment of the Riemann Zeta Function}}
    We will prove the infamous result of Hardy and Littlewood (see \cite{hardy1916contributions}) about the second moment of the Riemann zeta function. This is one of the simplest moment results to obtain since the only real tool we will need is the approximate functional equation. Precisely, we prove the following:

    \begin{theorem}
      For $T > 2$,
      \[
        M_{2}(T,\z) \sim T\log(T).
      \]
    \end{theorem}
    \begin{proof}
      Taking $\s = \frac{1}{2}$, $X = \sqrt{t}$, and $\Phi(u) = \cos^{-4M}\left(\frac{\pi u}{4M}\right)$ with $M \ge 1$ in the approximate functional equation gives
      \[
        \z\left(\frac{1}{2}+it\right) = \sum_{n \ge 1}\frac{1}{n^{\frac{1}{2}+it}}V_{\frac{1}{2}+it}\left(\frac{n}{\sqrt{t}}\right)+\e\left(\frac{1}{2}+it,\z\right)\sum_{n \ge 1}\frac{1}{n^{\frac{1}{2}-it}}V_{\frac{1}{2}-it}\left(n\sqrt{t}\right)+\frac{R\left(\frac{1}{2}+it,\sqrt{t},\z\right)}{\g\left(\frac{1}{2}+it,\z\right)}.
      \]
      \todo{xxx}
    \end{proof}