\chapter{Moment Results}
  Obtaining any type of moment result is, generally speaking, a very difficult task requiring delicate analytic techniques. Since moments of $L$-functions are still a modern topic of research, there is not yet a universal theory to obtain such results. Even worse, currently methods almost only work when $k \le 4$. So, rather than trying to develop a general theory, we highlight some infamous moment results and their techniques.
  \section{\todo{The Second Moment of the Riemann Zeta Function}}
    We will prove the infamous result of Hardy and Littlewood (see \cite{hardy1916contributions}) about the second moment of the Riemann zeta function. This is one of the simplest moment results to obtain since the only real tool we will need is the approximate functional equation. Precisely, we prove the following:

    \begin{theorem}
      For $T > 2$,
      \[
        M_{2}(T,\z) \sim T\log(T).
      \]
    \end{theorem}
    \begin{proof}
      Taking $\s = \frac{1}{2}+it$, $X = \sqrt{\frac{t}{\log(t)}}$ for $t > 2$, and $\Phi(u) = \cos^{-4M}\left(\frac{\pi u}{4M}\right)$ with $M \ge 1$ in the approximate functional equation gives
      \begin{align*}
        \z\left(\frac{1}{2}+it\right) &= \sum_{n \ge 1}\frac{1}{n^{\frac{1}{2}+it}}V_{\frac{1}{2}+it}\left(n\sqrt{\frac{\log(t)}{t}}\right) \\
        &+ \e\left(\frac{1}{2}+it,\z\right)\sum_{n \ge 1}\frac{1}{n^{\frac{1}{2}-it}}V_{\frac{1}{2}-it}\left(n\sqrt{\frac{t}{\log(t)}}\right)+\frac{R\left(\frac{1}{2}+it,\sqrt{\frac{t}{\log(t)}},\z\right)}{\g\left(\frac{1}{2}+it,\z\right)}.
      \end{align*}
      By \cref{prop:V_function_decay}, we find that $V_{\frac{1}{2}+it}\left(n\sqrt{\frac{\log(t)}{t}}\right)$ and $V_{\frac{1}{2}-it}\left(n\sqrt{\frac{t}{\log(t)}}\right)$ are bounded for $n \ll \frac{t}{\sqrt{\log(t)}}$ and $n \ll \sqrt{\log(t)}$ respectively and then exhibit polynomial decay of arbitrarily large order thereafter. Moreover, \cref{equ:modified_gamma_estimates,equ:choice_for_V_decay_estimate} together imply that the third term exhibits exponential decay and is therefore absolutely bounded. Lastly, $\e\left(\frac{1}{2}+it,\z\right)$ is absolutely bounded by \cref{equ:gamma_factor_analytic_conductor_estimate}. Altogether, this means
      \[
        \z\left(\frac{1}{2}+it\right) = \sum_{n \ll \frac{t}{\sqrt{\log(t)}}}\frac{1}{n^{\frac{1}{2}+it}}+O\left(\sum_{n \ll \sqrt{\log(t)}}\frac{1}{\sqrt{n}}\right)+O(1).
      \]
      and using the estimate
      \[
        \sum_{n \ll \sqrt{\log(t)}}\frac{1}{\sqrt{n}} = O\left(\int_{1}^{\sqrt{\log(t)}}\frac{1}{\sqrt{x}}\,dx\right) = O\left(\log^{\frac{1}{4}}(t)\right),
      \]
      we arrive at
      \begin{equation}\label{equ:second_moment_zeta_1}
        \z\left(\frac{1}{2}+it\right) = \sum_{n \ll \frac{t}{\sqrt{\log(t)}}}\frac{1}{n^{\frac{1}{2}+it}}+O\left(\log^{\frac{1}{4}}(t)\right).
      \end{equation}
      Applying \cref{equ:second_moment_zeta_1} to the definition of $M_{2}(T,\z)$ and recalling that $\z\left(\frac{1}{2}+it\right)$ is absolutely bounded for $0 \le t \le 2$, yields
      \begin{equation}\label{equ:second_moment_zeta_2}
        \begin{aligned}
          M_{2}(T,\z) &= \int_{2}^{T}\left|\sum_{n \ll \frac{t}{\sqrt{\log(t)}}}\frac{1}{n^{\frac{1}{2}+it}}\right|^{2}\,dt \\
          &+ O\left(\int_{2}^{T}\left|\sum_{n \ll \frac{t}{\sqrt{\log(t)}}}\frac{1}{n^{\frac{1}{2}+it}}\right|\log^{\frac{1}{4}}(t)\,dt\right)+O\left(\int_{2}^{T}\left(\log^{\frac{1}{4}}(t)\right)^{2}\,dt\right).
        \end{aligned}
      \end{equation}
      We will now simplify \cref{equ:second_moment_zeta_2}. For the second term, the Cauchy-Schwarz inequality gives
      \begin{align*}
        \int_{2}^{T}\left|\sum_{n \ll \frac{t}{\sqrt{\log(t)}}}\frac{1}{n^{\frac{1}{2}+it}}\right|\log^{\frac{1}{4}}(t)\,dt &= O\left(\left(\int_{2}^{T}\left|\sum_{n \ll \frac{t}{\sqrt{\log(t)}}}\frac{1}{n^{\frac{1}{2}+it}}\right|^{2}\int_{2}^{T}\log^{\frac{1}{2}}(t)\,dt\right)^{\frac{1}{2}}\right) \\
        &= O\left(\left(\int_{2}^{T}\left|\sum_{n \ll \frac{t}{\sqrt{\log(t)}}}\frac{1}{n^{\frac{1}{2}+it}}\right|^{2}\,dt\right)^{\frac{1}{2}}\sqrt{T}\log^{\frac{1}{2}}(T)\right).
      \end{align*}
      where in the second line we have made use of the estimate
      \[
        \int_{2}^{T}\log^{\frac{1}{2}}(t)\,dt = O\left(\int_{2}^{T}\log(t)\,dt\right) = O(T\log(T)).
      \]
      Applying this estimate to the third term as well, we arrive at
      \begin{equation}\label{equ:second_moment_zeta_3}
        \begin{aligned}
          M_{2}(T,\z) &= O\left(\int_{2}^{T}\left|\sum_{n \ll \frac{t}{\sqrt{\log(t)}}}\frac{1}{n^{\frac{1}{2}+it}}\right|^{2}\,dt\right) \\
          &+ O\left(\left(\int_{2}^{T}\left|\sum_{n \ll \frac{t}{\sqrt{\log(t)}}}\frac{1}{n^{\frac{1}{2}+it}}\right|^{2}\,dt\right)^{\frac{1}{2}}\sqrt{T}\log^{\frac{1}{2}}(T)\right)+O(T\log(T)).
        \end{aligned}
      \end{equation}
      Therefore, it suffices to show that the remaining integral is $O(T\log(T))$. We compute
      \begin{align*}
        \left|\sum_{n \ll \frac{t}{\sqrt{\log(t)}}}\frac{1}{n^{\frac{1}{2}+it}}\right|^{2} &= \int_{2}^{T}\sum_{n,m \ll \frac{t}{\sqrt{\log(t)}}}\frac{1}{n^{\frac{1}{2}+it}m^{\frac{1}{2}-it}}\,dt \\
        &= \sum_{n,m \ll \frac{T}{\sqrt{\log(T)}}}\int_{\max(m,n,2)}^{T}\frac{1}{n^{\frac{1}{2}+it}m^{\frac{1}{2}-it}}\,dt && \text{FTT} \\
        &= \sum_{n \ll \frac{T}{\sqrt{\log(T)}}}\frac{T-n}{n}+ \sum_{\substack{n,m \ll \frac{T}{\sqrt{\log(T)}} \\ n \neq m}}\frac{1}{\sqrt{nm}}\int_{\max(n,m,2)}^{T}\left(\frac{m}{n}\right)^{it}\,dt \\
        &= T\sum_{n \ll \frac{T}{\sqrt{\log(T)}}}\frac{1}{n}+O\left(\sum_{n \ll \frac{T}{\sqrt{\log(T)}}}1\right)+\left(\sum_{\substack{n,m \ll \frac{T}{\sqrt{\log(T)}} \\ n \neq m}}\frac{1}{\sqrt{nm}\log\left(\frac{m}{n}\right)}\right),
      \end{align*}
      where in the second to last line we have separated the terms for which $m = n$ or not. We now estimate all of the remaining terms individually. For the first term, we have
      \[
        T\sum_{n \ll \frac{T}{\sqrt{\log(T)}}}\frac{1}{n} = O\left(T\int_{1}^{\frac{T}{\sqrt{\log(T)}}}\frac{1}{x}\,dx\right) = O\left(T\log\left(\frac{T}{\sqrt{\log(T)}}\right)\right) = O(T\log(T)).
      \]
      The second term is easier since
      \[
        \sum_{n \ll \frac{T}{\sqrt{\log(T)}}}1 = O\left(\frac{T}{\sqrt{\log(T)}}\right) = O(T).
      \]
      For the last term \todo{xxx}
    \end{proof}