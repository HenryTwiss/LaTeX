\chapter{Applications of Hecke and Hecke-Maass \texorpdfstring{$L$}{L}-functions}
  We will discuss two applications of Rankin-Selberg convolution $L$-functions of Hecke and Hecke-Maass eigenforms: strong multiplicity one and the Ramanujan-Petersson conjecture on average. This first result is a strengthening of multiplicity one for holomorphic and Maass forms. The second result is weaker than the Ramanujan-Petersson conjecture for holomorphic or Maass forms but is often a sufficient replacement.
  \section{Strong Multiplicity One}
    Recall that multiplicity one determines and eigenform, up to a constant, by its Hecke eigenvalues at primes. Using Rankin-Selberg convolution $L$-functions, we can prove \textbf{strong multiplicity one}\index{strong multiplicity one} for holomorphic or Maass forms which says that eigenforms are determined by their Hecke eigenvalues at all but finitely many primes:

    \begin{theorem*}[Strong multiplicity one, holomorphic and Maass]
      Let $f$ and $g$ both be Hecke or Hecke-Maass eigenforms. If they have the same Hecke eigenvalues for all but finitely many primes $p$ then $f = g$.
    \end{theorem*}
    \begin{proof}
      By \cref{thm:newforms_characterization_holomorphic,thm:newforms_characterization_Maass} we may assume that $f$ and $g$ are primitive Hecke eigenforms. Denote the Hecke eigenvalues by $\l_{f}(n)$ and $\l_{g}(n)$ respectively. Let $S$ be the set the primes for which $\l_{f}(p) \neq \l_{g}(p)$ including the primes that ramify for $L(s,f)$ and $L(s,g)$. Then $S$ is finite by assumption. As the local factors of $L(s,f \ox g)$ are holomorphic and nonzero at $s = 1$, the order of the pole of $L(s,f \ox g)$ is the same as the order of the pole of
      \[
        L(s,f \ox g)\prod_{p \in S}L_{p}(s,f \ox g)^{-1} = \prod_{p \notin S}L_{p}(s,f \ox g).
      \]
      But as $\l_{f}(p) = \l_{g}(p)$ for all $p \notin S$, we have
      \[
        \prod_{p \notin S}L_{p}(s,f \ox g) = \prod_{p \notin S}L_{p}(s,f \ox f),
      \]
      and so
      \[
        L(s,f \ox g)\prod_{p \in S}L_{p}(s,f \ox g)^{-1} = L(s,f \ox f)\prod_{p \in S}L_{p}(s,f \ox f)^{-1}.
      \]
      Since $L(s,f \ox f)$ has a simple pole at $s = 1$, it follows that $L(s,f \ox g)$ does too. But then $f = g$.
    \end{proof}
  \section{The Ramanujan-Petersson Conjecture on Average}
    Rankin-Selberg convolution $L$-functions can also be used to obtain the Ramanujan-Petersson conjecture on average:

    \begin{proposition}\label{prop:Ramanujan_Petersson_average}
      Let $f$ be a primitive Hecke or Hecke-Maass eigenform. Then for any $X > 0$, we have
      \[
      \sum_{n \le X}|a_{f}(n)| \ll_{\e} X^{1+\e},
      \]
    \end{proposition}
    \begin{proof}
      By the Cauchy-Schwarz inequality,
      \begin{equation}\label{equ:Ramanujan_conjecture_on_average_1}
        \left(\sum_{n \le X}|a_{f}(n)|\right)^{2} \le X\sum_{n \le X}|a_{f}(n)|^{2},
      \end{equation}
      The Rankin-Selberg square $L(s,f \ox f)$ is locally absolutely uniformly convergent for $\s > \frac{3}{2}$. Therefore it still admits meromorphic continuation to $\C$ with a simple pole at $s = 1$. By Landau's theorem, the abscissa of absolute convergence of $L(s,f \ox f)$, and hence $L(s,f \x f)$ too, is $1$. By \cref{prop:Dirichlet_series_coefficient_size_on_average}, we have
      \[
        \sum_{n \le X}|a_{f}(n)|^{2} \ll_{\e} X^{1+\e}.
      \]
      Substituting this bound into \cref{equ:Ramanujan_conjecture_on_average_1}, we obtain
      \[
        \left(\sum_{n \le X}|a_{f}(n)|\right)^{2} \ll_{\e} X^{2+\e},
      \]
      and taking the square root yields
      \[
        \sum_{n \le X}|a_{f}(n)| \ll_{\e} X^{1+\e}.
      \]
    \end{proof}

    The bound in \cref{prop:Ramanujan_Petersson_average} should be compared with the implication $a_{f}(n) \ll_{\e} n^{\e}$ that follows from the corresponding Ramanujan-Petersson conjecture. While \cref{prop:Ramanujan_Petersson_average} is not useful in the holomorphic form case, it is in the Maass form case. Indeed, recall that if $f$ is a primitive Hecke-Maass eigenform we needed to assume the Ramanujan-Petersson conjecture for Maass forms to ensure $a_{f}(n) \ll_{\e} n^{\e}$ so that $L(s,f)$ was locally absolutely uniformly convergent for $\s > 1$. However, \cref{prop:Dirichlet_series_convergence_polynomial_bound_average,prop:Ramanujan_Petersson_average} now together imply $L(s,f)$ is locally absolutely uniformly convergent for $\s > 1$ without this assumption. Often \cref{prop:Ramanujan_Petersson_average} is all that is needed for additional applications instead of outright assuming the Ramanujan-Petersson conjecture for Maass forms.