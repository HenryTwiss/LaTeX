\chapter{Abstract Algebraic Number Theory}
  Algebraic number theory, in full generality, requires the development of some tools in abstract algebra. The most notable concepts are that of integrality, Dedekind domains, localization, and orders. We develop these theories in the following.
  \section{Integrality}
    Let $A$ and $B$ be rings with $A \subseteq B$. We say that $b \in B$ is \textbf{integral}\index{integral} over $A$ if $\b$ is the root of a monic polynomial $f(x) \in A[x]$. In other words, $\b$ satisfies
    \[
      \b^{n}+\a_{n-1}\b^{n-1}+\cdots+\a_{0} = 0,
    \]
    for some $n \ge 1$ and $\a_{i} \in A$ for $0 \le i \le n-1$. We say that $B$ is \textbf{integral}\index{integral} over $A$ if every element of $B$ is integral over $A$. The following proposition shows that integral elements form a ring:

    \begin{proposition}\label{prop:integral_if_finitely_generated}
      Let $A$ and $B$ be rings with $A \subseteq B$. Then the finitely many elements $\b_{1},\ldots,\b_{n} \in B$ are all integral over $A$ if and only if $A[\b_{1},\ldots,\b_{n}]$ is a finitely generated $A$-module. In particular, the elements of $B$ that are integral over $A$ form a ring.
    \end{proposition}
    \begin{proof}
      First suppose $\b \in B$ is integral over $A$. Then there exists a monic polynomial $f(x) \in A[x]$, of say degree $n \ge 1$, such that $f(\b) = 0$. Now for any $g(x) \in A[x]$, Euclidean division implies
      \[
        g(x) = q(x)f(x)+r(x),
      \]
      with $q(x),r(x) \in A[x]$ and $\deg(r(x)) < n$. Letting
      \[
        r(x) = \a_{n-1}x^{n-1}+\cdots+\a_{1}x+a_{0},
      \]
      with $\a_{i} \in \Z$ for $0 \le i \le n-1$, it follows that
      \[
        g(\b) = r(\b) = \a_{n-1}\b^{n-1}+\cdots+\a_{1}\b+\a_{0}.
      \]
      As $g(x)$ was arbitrary, we see that $1,\b,\ldots,\b^{n-1}$ generates $A[\b]$ as a $A$-module. Now suppose $\b_{1},\ldots,\b_{n} \in B$ are all integral over $A$. We will prove that $A[\b_{1},\ldots,\b_{n}]$ is finitely generated as an $A$-module by induction. Our previous work implies the base case. So assume by induction that $R = A[\b_{1},\ldots,\b_{n-1}]$ is a finitely generated $A$-module. Then $R[\b_{n}] = A[\b_{1},\ldots,\b_{n}]$ is a finitely generated $R$-module and hence a finitely generated $\Z$-module as well by our induction hypothesis. This proves the forward implication of the first statement. For the reverse implication, suppose $A[\b_{1},\ldots,\b_{n}]$ is a finitely generated $A$-module. Let $\w_{1},\ldots,\w_{r}$ be a basis for $A[\b_{1},\ldots,\b_{n}]$. Then for any $\b \in A[\b_{1},\ldots,\b_{n}]$, we have
      \[
        \b\w_{i} = \sum_{1 \le j \le r}\a_{i,j}\w_{j},
      \]
      with $\a_{i,j} \in A$ for $1 \le i,j \le r$. We can rewrite this as
      \[
        (\b-\a_{i,i})\w_{i}-\sum_{\substack{1 \le j \le r \\ j \neq i}}\a_{i,j}\w_{j} = 0,
      \]
      for all $i$. These $r$ equations are equivalent to the identity
      \[
        \begin{pmatrix} \b-\a_{1,1} & \a_{1,2} & \cdots & -\a_{1,r} \\ -\a_{2,1} & \b-\a_{2,2} & & \\ \vdots & & \ddots & \\ -\a_{r,1} & & & \b-\a_{r,r} \end{pmatrix}\begin{pmatrix} \w_{1} \\ \w_{2} \\ \vdots \\ \w_{r} \end{pmatrix} = \mathbf{0}.
      \]
      Thus the determinant of the matrix on the left-hand side must be zero. This shows that $\b$ is the root of the characteristic polynomial $\det(xI-(\a_{i,j})_{i,j})$ which is a monic polynomial with coefficients in $A$. Hence $\b$ is integral over $A$. As $\b$ was arbitrary, this shows that the elements $\b_{1},\ldots,\b_{n}$ are all integral over $A$ and that the sum and product of elements that are integral over $A$ are also integral over $A$. This proves the reverse implication and the second statement completing the proof.
    \end{proof}

    Integrality is also transitive via the following statement:

    \begin{proposition}\label{prop:integrality_is_transitive}
      Suppose $A$, $B$, and $C$ are rings with $A \subseteq B \subseteq C$. If $C$ is integral over $B$ and $B$ is integral over $A$, then $C$ is integral over $A$.
    \end{proposition}
    \begin{proof}
      Let $\g \in C$. Since $C$ is integral over $B$, we have
      \[
        \g^{n}+\b_{n-1}\g^{n-1}+\cdots+\b_{0} = 0,
      \]
      for some $n \ge 1$ and $\b_{i} \in B$ for $0 \le i \le n-1$. Set $R = A[\b_{0},\ldots,\b_{n-1}]$. Then $R[\g]$ is a finitely generated $R$-module and since $B$ is integral over $A$, \cref{prop:integral_if_finitely_generated} implies that $R[\g]$ is also a finitely generated $A$-module. Thus $\g$ is integral over $A$ by \cref{prop:integral_if_finitely_generated} again. As $\g$ was arbitrary, $C$ is integral over $A$.
    \end{proof}

    In light of \cref{prop:integrality_is_transitive}, we define the \textbf{integral closure}\index{integral closure} $\overline{A}$ of $A$ in $B$ by
    \[
      \overline{A} = \{\b \in B:\textbf{$\b$ is integral over $A$}\}.
    \]
    Clearly $A \subseteq \overline{A}$. Moreover, we say that $A$ is \textbf{integrally closed}\index{integrally closed} in $B$ if $A = \overline{A}$. As $A \subseteq \overline{A} \subseteq \overline{\overline{A}}$, \cref{prop:integrality_is_transitive} implies that $\overline{A}$ is automatically integrally closed in $B$. Now suppose $A$ is an integral domain with field of fractions $K$. Then we call the integral closure $\overline{A}$ of $A$ in $K$ the \textbf{normalization}\index{normalization} of $A$ and simply say that $A$ is \textbf{integrally closed}\index{integrally closed} if $A$ is equal to its normalization. With this in mind, we will now define a setup that will be often employed in the following. Let $A$ be an integrally closed integral domain with field of fractions $K$. Let $L/K$ be a finite separable extension and let $B$ be the integral closure of $A$ in $L$. We call the quadruple $(A,K,B,L)$ the \textbf{$AKBL$ setup}\index{$AKBL$} setup. For the $AKBL$ setup, the field of fractions of $B$ has a simple description:

    \begin{proposition}\label{prop:field_of_fractions_AKBL}
      Let $(A,K,B,L)$ be the the $AKBL$ setup. Then every $\l \in L$ is of the form
      \[
        \l = \frac{\b}{\a},
      \]
      for some $\b \in B$ and nonzero $\a \in A$. In particular, $L$ is the field of fractions of $B$. Moreover, $\l \in L$ is integral over $A$ if and only if the minimal polynomial $m_{\l}(x)$ of $\l$ has coefficients in $A$.
    \end{proposition}
    \begin{proof}
      As $L/K$ is finite, it is necessarily algebraic so that any $\l \in L$ satisfies
      \[
        \a\l^{n}+\a_{n-1}\l^{n-1}+\cdots+\a_{0} = 0,
      \]
      with $\a_{i} \in A$ for $0 \le i \le n-1$ and nonzero $\a \in A$. We claim that $\a\l$ is integral over $A$. Indeed, multiplying the previous identity by $\a^{n-1}$ yields
      \[
        (\a\l)^{n}+\a'_{n-1}(\a\l)^{n-1}+\cdots+\a'_{0} = 0,
      \]
      where $\a'_{i} = \a_{i}\a^{n-1-i}$ for $0 \le i \le n-1$, and so $\a\l$ is the root of a monic polynomial with coefficients in $A$. Then $\a\l \in B$ and so $\a\l = \b$ for some $\b \in B$ which is equivalent to $\l = \frac{\b}{\a}$. As $A \subseteq B$, this also implies that $L$ is the field of fractions of $B$. For the last statement, suppose $\l \in L$. If the minimal polynomial $m_{\l}(x)$ of $\l$ has coefficients in $A$ then $\l$ is automatically integral over $A$ (since the minimal polynomial is monic). So suppose $\l$ is an integral over $A$ so that $\l$ is a root of a monic polynomial $f(x) \in A[x]$. Then $m_{\l}(x)$ divides $f(x)$ and thus all of the roots of $m_{\l}(x)$ are integral over $A$ too. By Vieta's formulas, the coefficients of $m_{\k}(x)$ integral over $A$ as well. But then $m_{\k}(x) \in A[x]$. This completes the proof.
    \end{proof}
  \section{Traces \& Norms}
    We will now introduce norms and traces of algebras. Let $K$ be a field and let $R$ be an $n$-dimensional $K$-algebra. Then the \textbf{trace}\index{trace} and \textbf{norm}\index{norm} of $R$, denoted $\Trace_{R/K}$ and $\Norm_{R/K}$ respectively, are defined by
    \[
      \Trace_{R/K}(\rho) = \tr(T_{\rho}) \quad \text{and} \quad \Norm_{R/K}(\rho) = \det(T_{\rho}),
    \]
    for any $\rho \in R$, where $T_{\rho}:R \to R$ is the linear operator defined by
    \[
      T_{\rho}(x) = \rho x,
    \]
    for all $x \in R$. That is, $T_{\rho}$ is the multiplication by $\rho$ map. Letting $f_{\rho}(x)$ denote the characteristic polynomial of $T_{\rho}$, we have
    \[
      f_{\rho}(x) = \det(xI-T_{\rho}) = x^{n}-\k_{n-1}x^{n-1}+\cdots+(-1)^{n}\k_{0},
    \]
    with $\k_{i} \in K$ for $0 \le i \le n-1$. Then the trace and the norm are given by
    \begin{equation}\label{equ:trace_and_norm_characteristic_polynomial}
      \Trace_{R/K}(\rho) = \k_{n-1} \quad \text{and} \quad \Norm_{R/K}(\rho) = \k_{0},
    \end{equation}
    and therefore take values in $K$. Moreover, we have
    \[
      \Trace_{R/K}(\k\rho) = \k\Trace_{R/K}(\rho) \quad \text{and} \quad \Norm_{R/K}(\k\rho) = \k^{m}\Norm_{R/K}(\rho),
    \]
    for all $\k \in K$ because $T_{\k\l} = \k T_{\l}$. Also note that $T_{\l+\nu} = T_{\l}+T_{\nu}$ and $T_{\l \nu} = T_{\l}T_{\nu}$ for all $\l,\nu \in R$. In the case of a degree $n$ extension $L/K$, we call $\Trace_{L/K}$ and $\Norm_{L/K}$ the \textbf{trace}\index{trace} and \textbf{norm}\index{norm} of $L/K$. Moreover, $\Norm(\l) = 0$ if and only if $\l = 0$ because otherwise $T_{\l}$ has inverse $T_{\l^{-1}}$ and hence a nonzero determinant. Therefore we obtain homomorphisms
    \[
      \Trace_{L/K}:L \to K \quad \text{and} \quad \Norm_{L/K}:L^{\ast} \to K^{\ast}.
    \]
    When $L/K$ is separable, we can derive alternative descriptions of the trace and norm of $L/K$. This additional assumption is weak because we are mostly interested in finite extensions of $\Q$ and $\F_{p}$ which are always separable (because both $\Q$ and $\F_{p}$ are perfect). In any case, to do this we need to work in the algebraic closure $\conj{K}$ of $K$. As $L/K$ is a degree $n$ separable extension, there are exactly $n$ distinct $K$-embeddings $\s_{1},\ldots,\s_{n}$ of $L$ into $\conj{K}$ (each given by letting $\t$ be a primitive element for $L$ so that $L = K[\t]$ and sending $\t$ to one of its conjugate roots in the minimal polynomial $m_{\t}(x)$ of $\t$). Moreover, we prove the following proposition:

    \begin{proposition}\label{prop:formulas_for_trace_and_norm}
      Let $L/K$ be a degree $n$ separable extension and let $\s$ run over all $K$-embeddings $\s$ of $L$ into $\conj{K}$. For any $\l \in L$, the characteristic polynomial $f_{\l}(x)$ of $T_{\l}$ is a power of the minimal polynomial $m_{\l}(x)$ of $\l$ and satisfies
      \[
        f_{\l}(x) = \prod_{\s}(x-\s(\l)).
      \]
      In particular,
      \[
        \Trace_{L/K}(\l) = \sum_{\s}\s(\l) \quad \text{and} \quad \Norm_{L/K}(\l) = \prod_{\s}\s(\l).
      \]
    \end{proposition}
    \begin{proof}
      Let
      \[
        m_{\l}(x) = x^{m}+\k_{m-1}x^{m-1}+\cdots+\k_{0},
      \]
      with $\k_{i} \in K$ for $0 \le i \le n-1$, be the minimal polynomial of $\l$ (necessarily $m$ is the degree of $K(\l)/K$). Let $d$ be the degree of $L/K(\l)$. We first show that $f_{\l}(x)$ is a power of $m_{\l}(x)$. Precisely, we claim that
      \[
        f_{\l}(x) = m_{\l}(x)^{d}.
      \]
      To see this, recall that $1,\l,\ldots,\l^{n-1}$ is a basis for $K(\l)/K$. If $\a_{1},\ldots,\a_{d}$ is a basis for $L/K(\l)$, then
      \[
        \a_{1},\a_{1}\l,\ldots,\a_{1}\l^{m-1},\ldots,\a_{d},\a_{d}\l,\ldots,\a_{d}\l^{m-1},
      \]
      is a basis for $L/K$. Because the minimal polynomial $m_{\l}(x)$ gives the linear relation
      \[
        \l^{m} = -\k_{0}-k_{1}\l-\cdots-\k_{m-1}\l^{m-1},
      \]
      the matrix of $T_{\l}$ is block diagonal with $d$ blocks each of the form
      \[
        \begin{pmatrix} & 1 & & \\ & & \ddots & \\ & & & 1 \\ -\k_{0} & -\k_{1} & \cdots & -\k_{m-1} \\ \end{pmatrix}.
      \]
      This is the companion matrix to $m_{\l}(x)$ and hence the characteristic polynomial is $m_{\l}(x)$ as well. Our claim follows since the matrix of $T_{\l}$ is block diagonal. Since $\l$ is algebraic over $K$ of degree $m$, $K(\l)$ is the splitting field of $m_{\l}(x)$ and there are $m$ distinct $K$-embeddings of $K(\l)$ into $\conj{K}$. Let $\tau$ be such an $K$-embedding. Then the $K$-embeddings $\s$ are partitioned into $m$ many equivalence classes each of size $d$ (because $L/K(\l)$ is degree $d$) where $\s$ and $\s'$ are in the same class if and only if $\s(\l) = \s'(\l)$. In particular, a complete set of representatives is given by the $\tau$. But then
      \[
        f_{\l}(x) = m_{\l}(x)^{d} = \left(\prod_{\tau}(x-\tau(\l))\right)^{d} = \prod_{\s}(x-\s(\l)),
      \]
      which proves the first statement. The formulas for the trace and norm follow from Vieta's formulas and \cref{equ:trace_and_norm_characteristic_polynomial}.
    \end{proof}

    As an application of \cref{prop:formulas_for_trace_and_norm}, we can show how the field trace and field norm act in the $AKBL$ setup:

    \begin{proposition}\label{prop:norm_and_trace_AKBL}
      Let $(A,K,B,L)$ be the the $AKBL$ setup If $\l \in L$ is integral over $A$, then $\Trace_{L/K}(\l)$ and $\Norm_{L/K}(\l)$ are in $A$.
    \end{proposition}
    \begin{proof}
      By \cref{prop:field_of_fractions_AKBL}, if $\l \in L$ is integral over $A$ then its minimal polynomial $m_{\l}(x)$ has coefficients in $A$. By \cref{prop:formulas_for_trace_and_norm}, the characteristic polynomial $f_{\l}(x)$ is a power of $m_{\l}(x)$ and hence $f_{\l}(x)$ has coefficients in $A$ too. From \cref{equ:trace_and_norm_characteristic_polynomial} we conclude that $\Trace_{L/K}(\l)$ and $\Norm_{L/K}(\l)$ are in $A$.
    \end{proof}

    We can also classify the units of $B$ in terms of the units of $A$:

    \begin{proposition}\label{prop:unit_if_and_only_if_AKBL}
      Let $(A,K,B,L)$ be the the $AKBL$ setup. Then $\b \in B$ is a unit if and only if $\Norm_{L/K}(\l) \in A$ is a unit.
    \end{proposition}
    \begin{proof}
      First suppose $\b \in B$ is a unit. Then $\frac{1}{\b} \in B$ and so
      \[
        \Norm_{L/K}(\b)\Norm_{L/K}\left(\frac{1}{\b}\right) = \Norm(1) = 1.
      \]
      By \cref{prop:norm_and_trace_AKBL}, $\Norm_{L/K}(\b),\Norm_{L/K}\left(\frac{1}{\b}\right) \in A$ and hence $\Norm_{L/K}(\b)$ is a unit. Now suppose $\Norm_{L/K}(\b) \in A$ is a unit. By \cref{prop:field_of_fractions_AKBL}, the minimal polynomial $m_{\b}(x)$ of $\b$ has coefficients in $A$. Moreover, \cref{equ:trace_and_norm_characteristic_polynomial,prop:formulas_for_trace_and_norm} together imply that the constant term is a unit because $\Norm_{L/K}(\b)$ is. Letting the degree of $m_{\b}(x)$ be $m$, we have shown that
      \[
        m_{\b}(x) = x^{m}+\a_{m-1}x^{m-1}+\cdots+\a,
      \]
      with $\a_{i} \in A$ for $1 \le i \le m-1$ and $\a \in A$ a unit. Dividing $m_{\b}(\b)$ by $\b^{m}$, we find that $\frac{1}{\b}$ is a root of the polynomial
      \[
        f(x) = \a x^{m}+\a_{1}x^{m-1}+\cdots+1.
      \]
      Multiplying by $\frac{1}{\a}$, it follows that $\frac{1}{\b}$ is a root of a monic polynomial with coefficients in $A$. Hence $\frac{1}{\b} \in B$ and thus $\b$ is a unit.
    \end{proof}
  \section{Discriminants}
    We will now discuss discriminants of algebras. Let $K$ be a field and $R$ be an $n$-dimensional $K$-algebra. If $\rho_{1},\ldots,\rho_{n}$ is a basis for $R$, we set
    \[
      \disc_{R/K}(\rho_{1},\ldots,\rho_{n}) = \det((\Trace_{R/K}(\rho_{i}\rho_{j}))_{i,j}).
    \]
    In particular, $\disc_{R/K}(\rho_{1},\ldots,\rho_{n})$ is an element of $K$. It is also independent of the choice of basis up to elements of $(K^{\ast})^{2}$. Indeed, if $\rho'_{1},\ldots,\rho'_{n}$ is another basis then we have
    \[
      \rho'_{i} = \sum_{1 \le j \le m}\k_{i,j}\rho_{j},
    \]
    with $\k_{i,j} \in K$ for $1 \le i,j \le n$. Then $(\k_{i,j})_{i,j}$ is the base change matrix from $\rho_{1},\ldots,\rho_{n}$ to $\rho'_{1},\ldots,\rho'_{n}$ and so has nonzero determinant. Thus $\det((\k_{i,j})_{i,j}) \in K^{\ast}$. Moreover, we have
    \[
      (\Trace_{R/K}(\rho'_{i}\rho'_{j}))_{i,j} = (\k_{i,j})_{i,j}(\Trace_{R/K}(\rho_{i}\rho_{j}))_{i,j}(\k_{i,j})_{i,j}^{t},
    \]
    which, upon taking the determinant, shows that
    \begin{equation}\label{equ:discriminant_base_change}
      \disc_{R/K}(\rho'_{1},\ldots,\rho'_{n}) = \det((\k_{i,j})_{i,j})^{2}\disc_{R/K}(\rho_{1},\ldots,\rho_{n}),
    \end{equation}
    as claimed. We define the \textbf{discriminant}\index{discriminant} $\disc_{K}(R)$ of $R$ by
    \[
      \disc_{K}(R) = \disc_{R/K}(\rho_{1},\ldots,\rho_{n}) \pmod{(K^{\ast})^{2}}.
    \]
    for any basis $\rho_{1},\ldots,\rho_{n}$ of $R$. By what we have shown, $\disc_{K}(R)$ is well-defined. The discriminant is also multiplicative with respect to direct sums:

    \begin{proposition}\label{prop:discriminant_and_direct_sums}
      Let $K$ be a field and $R$ be an $n$-dimensional $K$-algebra. Suppose we have a direct sum decomposition
      \[
        R = R_{1} \op R_{2},
      \]
      for $K$-algebras $R_{1}$ and $R_{2}$ of dimensions $n_{1}$ and $n_{2}$ respectively. Also let $\eta_{1},\ldots,\eta_{n_{1}}$ and $\g_{1},\ldots,\g_{n_{2}}$ be bases of $R_{1}$ and $R_{2}$ respectively. Then
      \[
        \disc_{R/K}(\eta_{1},\ldots,\eta_{n_{1}},\g_{1},\ldots,\g_{n_{2}}) = \disc_{R/K}(\eta_{1},\ldots,\eta_{n_{1}})\disc_{R}(\g_{1},\ldots,\g_{n_{2}}).
      \]
      In particular,
      \[
        \disc_{K}(R) = \disc_{K}(R_{1})\disc_{K}(R_{2}).
      \]
    \end{proposition}
    \begin{proof}
      The second statement follows immediately from the first. To prove the first statement, write
      \[
        \disc_{R/K}(\eta_{1},\ldots,\eta_{n_{1}}) = \det((\Trace_{R/K}(\eta_{i}\eta_{j}))_{i,j}) \quad \text{and} \quad \disc_{R}(\g_{1},\ldots,\g_{n_{2}}) = \det((\Trace_{R/K}(\g_{k}\g_{\ell}))_{k,\ell}).
      \]
      As $R$ is the direct sum of $R_{1}$ and $R_{2}$ as $K$-modules, we have $\eta_{i}\g_{k} = 0$ for all $1 \le i \le n_{1}$ and $1 \le k \le n_{2}$. It follows that $\disc_{R/K}(\eta_{1},\ldots,\eta_{n_{1}},\g_{1},\ldots,\g_{n_{2}})$ is the determinant of the block diagonal matrix
      \[
        \begin{pmatrix} (\Trace_{R/K}(\eta_{i}\eta_{j}))_{i,j} & \\ & (\Trace_{R/K}(\g_{k}\g_{\ell}))_{k,\ell} \end{pmatrix}.
      \]
      Moreover, we have
      \[
        \Trace_{R/K}(\rho_{1}) = \Trace_{R_{1}/K}(\rho_{1}) \quad \text{and} \quad \Trace_{R/K}(\rho_{2}) = \Trace_{R_{2}/K}(\rho_{2})
      \]
      for any $\rho_{1} \in R_{1}$ and $\rho_{2} \in R_{2}$. Indeed, multiplication by $\rho_{1}$ and $\rho_{2}$ annihilate $R_{2}$ and $R_{1}$ respectively. But then
      \[
        \begin{pmatrix} (\Trace_{R/K}(\eta_{i}\eta_{j}))_{i,j} & \\ & (\Trace_{R/K}(\g_{k}\g_{\ell}))_{k,\ell} \end{pmatrix} = \begin{pmatrix} (\Trace_{R_{1}/K}(\eta_{i}\eta_{j}))_{i,j} & \\ & (\Trace_{R_{2}/K}(\g_{k}\g_{\ell}))_{k,\ell} \end{pmatrix}.
      \]
      The determinant of the matrix on right-hand side is $\disc_{R/K}(\eta_{1},\ldots,\eta_{n_{1}})\disc_{R}(\g_{1},\ldots,\g_{n_{2}})$. This completes the proof.
    \end{proof}
    
    We now specialize to the setting of a degree $n$ separable extension $L/K$. It turns out that the discriminant is nonzero. To see this, we require a lemma:

    \begin{lemma}\label{lem:trace_is_nondegenerate}
      Let $L/K$ be a finite separable extension. Then the map
      \[
        \Trace_{L/K}:L \x L \to K \qquad (\l,\eta) \mapsto \Trace_{L/K}(\l\eta),
      \]
      is a nondegenerate symmetric bilinear form.
    \end{lemma}
    \begin{proof}
      From the definition of the trace, it is clear that the map is a symmetric bilinear form. To see that is is nondegenerate, suppose $L/K$ is degree $n$. Then for any nonzero $\l \in L$, \cref{prop:formulas_for_trace_and_norm} implies that
      \[
        \Trace_{L/K}(\l\l^{-1}) = \Trace_{L/K}(1) = n.
      \]
      Hence the symmetric bilinear form is nondegenerate.
    \end{proof}

    We can now show that the discriminant is never zero:

    \begin{proposition}\label{prop:discriminant_not_zero}
      Let $L/K$ be a degree $n$ separable extension and let $\l_{1},\ldots,\l_{n}$ be a basis for $L$. Then we have that $\disc_{L/K}(\l_{1},\ldots,\l_{n}) \neq 0$. In particular, $\disc_{K}(L) \neq 0$.
    \end{proposition}
    \begin{proof}
      The second statement follows immediately from the first. To prove the first statement, suppose to the contrary that $\disc_{L/K}(\l_{1},\ldots,\l_{n}) = 0$. Then the matrix $(\Trace_{L/K}(\l_{i}\l_{j}))_{i,j}$ is not invertible. Hence there exists a nonzero column vector $(\k_{1},\ldots,\k_{n})^{t}$ with $\k_{i} \in K$ for $1 \le i \le n$ such that
      \[
        (\Trace_{L/K}(\l_{i}\l_{j}))_{i,j}(\k_{1},\ldots,\k_{n})^{t} = \mathbf{0}.
      \]
      This is equivalent to the $n$ equations
      \[
        \sum_{1 \le j \le n}\k_{j}\Trace_{L/K}(\l_{i}\l_{j}) = 0,
      \]
      for all $i$. Setting
      \[
        \l = \sum_{1 \le j \le n}\k_{j}\l_{j},
      \]
      linearity of the trace implies that these $n$ equations are equivalent to the fact that $\Trace_{L/K}(\l\l_{i}) = 0$ for all $i$. As $\l_{1},\ldots,\l_{n}$ is a basis for $L$, it follows that $\l \in L$ is a nonzero element for which $\Trace_{L/K}(\l\eta) = 0$ for all $\eta \in L$. This is impossible by \cref{lem:trace_is_nondegenerate}. Hence $\disc_{L/K}(\l_{1},\ldots,\l_{n}) \neq 0$. The second statement follows immediately.
    \end{proof}

    In addition to $\disc_{L/K}(\l_{1},\ldots,\l_{n})$ never vanishing, we can also write it in an alternative form. To do this, for any basis $\l_{1},\ldots,\l_{n}$ of $L$ we define the associated \textbf{embedding matrix}\index{embedding matrix} $M(\l_{1},\ldots,\l_{n})$ by
    \[
      M(\l_{1},\ldots,\l_{n}) = \begin{pmatrix} \s_{1}(\l_{1}) & \cdots & \s_{1}(\l_{n}) \\ \vdots & & \vdots \\ \s_{n}(\l_{1}) & \cdots & \s_{n}(\l_{n}) \end{pmatrix},
    \]
    where $\s_{1},\ldots,\s_{n}$ are the $n$ distinct $K$-embeddings of $L$ into $\conj{K}$. Then we have the following result:

    \begin{proposition}\label{disc_as_square_of_embedding_matrix}
      Let $L/K$ be a degree $n$ separable extension. Then for any basis $\l_{1},\ldots,\l_{n}$ of $L$, we have
      \[
        \disc_{L/K}(\l_{1},\ldots,\l_{n}) = \det(M(\l_{1},\ldots,\l_{n}))^{2}.
      \]
    \end{proposition}
    \begin{proof}
      Recalling that the $(i,j)$-entry of $M(\l_{1},\ldots,\l_{n})^{t}M(\l_{1},\ldots,\l_{n})$ is the dot product of the $i$-th and $j$-th columns of $M(\l_{1},\ldots,\l_{n})$, we have
      \begin{align*}
        \det(M(\l_{1},\ldots,\l_{n}))^{2} &= \det(M(\l_{1},\ldots,\l_{n})^{t}M(\l_{1},\ldots,\l_{n})) \\
        &= \det\left(\left(\sum_{\s}\s(\l_{i})\s(\l_{j})\right)_{i,j}\right) \\
        &= \det\left(\left(\sum_{\s}\s(\l_{i}\l_{j})\right)_{i,j}\right) \\
        &= \det((\Trace_{L/K}(\l_{i}\l_{j}))_{i,j}) \\
        &= \disc_{L/K}(\l_{1},\ldots,\l_{n}),
      \end{align*}
      where the sums run over all $K$-embeddings $\s$ of $L$ into $\conj{K}$ and the second to last equality follows by \cref{prop:formulas_for_trace_and_norm}, as desired.
    \end{proof}

    Discriminant are useful because of the following lemma for the $AKBL$ setup:

    \begin{lemma}\label{lem:lemma_for_integral_basis_AKBL}
      Let $(A,K,B,L)$ be the $AKBL$ setup where $L/K$ is of degree $n$. If $\l_{1},\ldots,\l_{n}$ is a basis of $L$ that is contained in $B$, then
      \[
        \disc_{L/K}(\l_{1},\ldots,\l_{n})B \subseteq A\l_{1}+\cdots+A\l_{n}.
      \]
    \end{lemma}
    \begin{proof}
      Since $\l_{1},\ldots,\l_{n}$ is a basis, we may write $\b = \k_{1}\l_{1}+\cdots+\k_{n}\l_{n}$ for any $\b \in B$. Linearity of the trace implies
      \[
        \sum_{1 \le j \le n}\k_{j}\Trace_{L/K}(\l_{i}\l_{j}) = \Trace_{L/K}(\l_{i}\b),
      \]
      for $1 \le i \le n$. These $n$ equations are equivalent to
      \[
        (\Trace_{L/K}(\l_{i}\l_{j}))_{i,j}(\k_{1},\ldots,\k_{n})^{t} = (\Trace_{L/K}(\b\l_{1}),\ldots,\Trace_{L/K}(\b\l_{n})).
      \]
      Multiplying on the left by the adjugate matrix of $(\Trace_{L/K}(\l_{i}\l_{j}))_{i,j}$ and recalling that a matrix times its adjugate is its determinate times the identity, we see that
      \[
        \k_{i} = \frac{\Trace_{L/K}(\b\l_{i})}{\disc_{L/K}(\l_{1},\ldots,\l_{n})},
      \]
      for all $i$. By \cref{prop:norm_and_trace_AKBL}, $\Trace_{L/K}(\b\l_{i}) \in A$ for all $i$ and therefore $\disc_{L/K}(\l_{1},\ldots,\l_{n})\k_{i} \in A$ for all $i$. But then
      \[
        \disc_{L/K}(\l_{1},\ldots,\l_{n})B \subseteq A\l_{1}+\cdots+A\l_{n},
      \]
      as desired.
    \end{proof}

    Again suppose $L/K$ is of degree $n$. We say that $\b_{1},\ldots,\b_{n}$ is an \textbf{integral basis}\index{integral basis} for $B$ over $A$ if $\b_{1},\ldots,\b_{n}$ is such that
    \[
      B = A\b_{1}+\cdots+A\b_{n}.
    \]
    Equivalently, $B$ is a free $A$-module of rank $n$. An integral basis is necessarily a basis for $L$ as a vector space over $K$ \cref{prop:field_of_fractions_AKBL}. For a general $AKBL$ setup, an integral basis need not exist. However, if $L/K$ is separable and $A$ is an integral domain, one is guaranteed to exist:
    
    \begin{theorem}\label{thm:integral_basis_AKBL}
      Let $(A,K,B,L)$ be the $AKBL$ setup where $L/K$ is a degree $n$ separable extension and $A$ is a principal ideal domain. Then $B$ admits an integral basis over $A$. Moreover, every finitely generated nonzero $B$-submodule of $L$ is a free $A$-module of rank $n$.
    \end{theorem}
    \begin{proof}
      Let $\l_{1},\ldots,\l_{n}$ be a basis for $L$ as a vector space over $K$. By \cref{prop:field_of_fractions_AKBL} we may multiply by a nonzero element of $A$, if necessary, to ensure that this basis belongs to $B$. Then \cref{lem:lemma_for_integral_basis_AKBL} implies
      \[
        \disc_{L/K}(\l_{1},\ldots,\l_{n})B \subseteq A\l_{1}+\cdots+A\l_{n}.
      \]
      Since $A\l_{1}+\cdots+A\l_{n}$ is a free $A$-module of rank $n$ and $A$ is a principal ideal domain, it follows that $B$ is also a free $A$-module of rank at most $n$. But any basis of $B$ as an $A$-module must also be a basis for $L$ as a vector space over $K$ by \cref{prop:field_of_fractions_AKBL}. Hence the rank is exactly $n$ and $B$ admits an integral basis over $A$ which proves the first statement. Now suppose $M$ is a nonzero $B$-submodule of $L$ and let $\w_{1},\ldots,\w_{r}$ be generators. By \cref{prop:field_of_fractions_AKBL} again we may multiply by a nonzero element of $A$, if necessary, to ensure that these generators belong to $B$. But then
      \[
        \disc_{L/K}(\l_{1},\ldots,\l_{n})M \subseteq \disc_{L/K}(\l_{1},\ldots,\l_{n})B,
      \]
      and by the above it follows that $M$ is a free $A$-module of rank at most $n$. To see that the rank is $n$, let $\a \in M$ be nonzero and, as before, let $\l_{1},\ldots,\l_{n}$ be a basis for $L$ that is contained in $B$. Then $\a\l_{1},\ldots,\a\l_{n} \in M$ are linearity independent since $\l_{1},\ldots,\l_{n}$ is a basis. Thus the rank of $M$ is at least $n$, and in particular it must be $n$. This completes the proof.
    \end{proof}
  \section{Dedekind Domains}
    We say that a ring $\mc{O}$ is a \textbf{Dedekind domain}\index{Dedekind domain} if $\mc{O}$ satisfies the following properties:
    \begin{enumerate}[label=(\roman*)]
      \item $\mc{O}$ is noetherian.
      \item $\mc{O}$ is an integrally closed integral domain.
      \item Every nonzero prime ideal of $\mc{O}$ is maximal.
    \end{enumerate}
    If $\mc{O}$ is an arbitrary Dedekind domain, we will denote its field of fractions by $K$. Moreover, we call any nonzero ideal of $\mc{O}$ \textbf{integral ideal}\index{integral ideal}. So property (i) can be rephrased as every integral ideal is finitely generated while (iii) is equivalent to the fact that every prime integral ideal is maximal. One should think of Dedekind domains as generalizations of $\Z$. Note that we have not assumed $\mc{O}$ is a principal ideal domain. In fact, the most interesting Dedekind domains are not principal ideal domains. Our primary concern regarding Dedekind domains will be to show that, while there need not be unique factorization of elements, unique factorization of ideals holds. We first show containment in one direction:

    \begin{lemma}\label{lem:integral_ideal_prime_containment}
      Let $\mc{O}$ be a Dedekind domain. For every integral ideal $\mf{a}$, there exist prime integral ideals $\mf{p}_{1},\ldots,\mf{p}_{k}$ such that
      \[
        \mf{p}_{1}\cdots\mf{p}_{k} \subseteq \mf{a}.
      \]
    \end{lemma}
    \begin{proof}
      Let $\mc{S}$ be the set of integral ideals which do not contain a product of prime integral. Then it suffices to show $\mc{S}$ is empty. Assuming otherwise and ordering $\mc{S}$ by inclusion, the fact that $\mc{O}$ is noetherian implies that there exists a maximal integral ideal $\mf{a} \in \mc{S}$. Moreover $\mf{a}$ cannot be prime for otherwise $\mf{a}$ contains a product of prime integral ideals (namely itself). Since $\mf{a}$ is not prime, there exist $\a_{1},\a_{2} \in \mc{O}$ with $\a_{1}\a_{2} \in \mf{a}$ and such that $\a_{1},\a_{2} \notin \mf{a}$. Now define integral ideals
      \[
        \mf{b}_{1} = \mf{a}+\a_{1}\mc{O} \quad \text{and} \quad \mf{b}_{2} = \mf{a}+\a_{2}\mc{O}.
      \]
      Note that $\mf{b}_{1}$ and $\mf{b}_{2}$ strictly contain $\mf{a}$ because $\a_{1},\a_{2} \notin \mf{a}$. Moreover, $\mf{b}_{1}\mf{b}_{2} \subseteq \mf{a}$ because
      \[
        \mf{b}_{1}\mf{b}_{2} = (\mf{a}+\a_{1}\mc{O})(\mf{a}+\a_{2}\mc{O}) = \mf{a}^{2}+\a_{1}\mc{O}+\a_{2}\mc{O}+\a_{1}\a_{2}\mc{O},
      \]
      and $\a_{1}\a_{2} \in \mf{a}$. Maximality of $\mf{a}$ implies that there exist prime integral ideals $\mf{p}_{1},\ldots,\mf{p}_{k}$ and $\mf{q}_{1},\ldots,\mf{q}_{\ell}$ such that
      \[
        \mf{p}_{1}\cdots\mf{p}_{k} \subseteq \mf{b}_{1} \quad \text{and} \quad \mf{q}_{1}\cdots\mf{q}_{\ell} \subseteq \mf{b}_{2}.
      \]
      But then
      \[
        \mf{p}_{1}\cdots\mf{p}_{k}\mf{q}_{1}\cdots\mf{q}_{\ell} \subseteq \b_{1}\b_{2} \subseteq \mf{a},
      \]
      which contradicts the fact that $\mf{a} \in \mc{S}$. Hence $\mc{S}$ is empty as desired.
    \end{proof}

    In order to obtain the reverse containment in \cref{lem:integral_ideal_prime_containment}, we need to do more work. Precisely, we want to show that every integral ideal factors into a product of prime integral ideals. To accomplish this, we will need to work in a slightly more general setting. First observe that an integral ideal $\mf{a}$ is just an $\mc{O}$-submodule of $\mc{O}$. Moreover, it is finitely generated $\mc{O}$-submodule since $\mc{O}$ is noetherian. Hence it is a finitely generated $\mc{O}$-submodule of $K$. Accordingly, we say $\mf{f}$ is a \textbf{fractional ideal}\index{fractional ideal} of $K$ if $\mf{f}$ a nonzero finitely generated $\mc{O}$-submodule of $K$. In particular, all integral ideals are fractional ideals. Now let $\k_{1},\ldots,\k_{r} \in K$ be generators for the fractional ideal $\mf{f}$. Since $K$ is the field of fractions of $\mc{O}$, $\k_{i} = \frac{\a_{i}}{\d_{i}}$ with $\a_{i},\d_{i} \in \mc{O}$ and where $\d_{i}$ is nonzero for $1 \le i \le r$. Setting $\d = \d_{1} \cdots \d_{r}$, we have that $\d\k_{i} \in \mc{O}$ for all $i$ and hence $\d\mf{f}$ is an integral ideal. Conversely, if there exists some nonzero $\d \in \mc{O}$ such that $\d\mf{f} = \mf{a}$ is an integral ideal then $\mf{f}$ is a fractional ideal because $\mf{a}$ is a finitely generated $\mc{O}$-submodule of $K$ and hence $\mf{f}$ is too. Thus for any fractional ideal $\mf{f}$, there exists a nonzero $\d \in \mc{O}$ and an integral ideal $\mf{a}$ such that
    \[
      \mf{f} = \frac{1}{\d}\mf{a}.
    \]
    Every fractional ideal is of this form, and integral ideals are precisely those for which $\d = 1$, and fractional ideals need not be a subgroup of $\mc{O}$. Now let $\mf{p}$ be a prime integral ideal. We define $\mf{p}^{-1}$ by
    \[
      \mf{p}^{-1} = \{\k \in K:\k\mf{p} \subseteq \mc{O}\}.
    \]
    It turns out that $\mf{p}^{-1}$ is a fractional ideal. Indeed, since $\mf{p}$ is an integral ideal there exists a nonzero $\a \in \mf{p}$. By definition of $\mf{p}^{-1}$, we have that $\a\mf{p}^{-1} \subseteq \mc{O}$. Hence $\a\mf{p}^{-1}$ is an integral ideal and therefore $\mf{p}^{-1}$ is a fractional ideal. Unlike integral ideals, $1 \in \mf{p}^{-1}$ so that $\mf{p}^{-1}$ contains units. The following proposition proves a stronger version of this and more:

    \begin{lemma}\label{lem:inverse_for_prime_ideals}
      Let $\mc{O}$ be a Dedekind domain and $\mf{p}$ be a prime integral ideal. Then the following hold:
      \begin{enumerate}[label=(\roman*)]
        \item
        \[
          \mc{O} \subset \mf{p}^{-1}.
        \]
        \item
        \[
          \mf{p}^{-1}\mf{p} = \mc{O}.
        \]
      \end{enumerate}
    \end{lemma}
    \begin{proof}
      We will prove the latter two statement separately:
      \begin{enumerate}[label=(\roman*)]
        \item Clearly $\mc{O} \subseteq \mf{p}^{-1}$ so it suffices to show that $\mf{p}^{-1}-\mc{O}$ is nonempty. To this end, let $\a \in \mf{p}$ be nonzero. By \cref{lem:integral_ideal_prime_containment} let $k \ge 1$ be the minimal integer such that there exist prime integral ideals $\mf{p}_{1},\ldots,\mf{p}_{k}$ with
        \[
          \mf{p}_{1} \cdots \mf{p}_{k} \subseteq \a\mc{O}.
        \]
        As $\a \in \mf{p}$, we have $\a\mc{O} \subseteq \mf{p}$. Since $\mf{p}$ is prime, there must be some $i$ with $1 \le i \le k$ such that $\mf{p}_{i} \subseteq \mf{p}$. Without loss of generality, we may assume $\mf{p}_{1} \subseteq \mf{p}$. As prime integral ideals are maximal since $\mc{O}$ is noetherian, we conclude $\mf{p}_{1} = \mf{p}$. Moreover, since $k$ is minimal we must have
        \[
          \mf{p}_{2} \cdots \mf{p}_{k} \not\subseteq \a\mc{O}.
        \]
        Hence there exists $\b \in \mf{p}_{2} \cdots \mf{p}_{k}$ with $\b \notin \a\mc{O}$. We will now show that $\b\a^{-1}$ is an element in $\mf{p}^{-1}-\mc{O}$. Since $\mf{p}_{1} = \mf{p}$, what we have previously shown implies $\b\mf{p} \subseteq \a\mc{O}$ and hence $\b\a^{-1}\mf{p} \in \mc{O}$ which means $\b\a^{-1} \in \mf{p}^{-1}$. But as $\b \notin \a\mc{O}$, we also have $\b\a^{-1} \notin \mc{O}$. Hence $\b\a^{-1} \in \mf{p}^{-1}-\mc{O}$ which proves (i).
        \item By (i) and the definition of $\mf{p}^{-1}$, we have $\mf{p} \subseteq \mf{p}^{-1}\mf{p} \subseteq \mc{O}$. Since $\mf{p}$ is maximal because $\mc{O}$ is noetherian, it follows that $\mf{p}^{-1}\mf{p}$ is either $\mf{p}$ or $\mc{O}$. So it suffices to show that the first case cannot hold. Assume by contradiction that $\mf{p}^{-1}\mf{p} = \mf{p}$. Let $\w_{1},\ldots,\w_{r}$ generate $\mf{p}$ and let $\a \in \mf{p}^{-1}-\mc{O}$ which exists by (i). Then $\a\w_{i} \in \mf{p}^{-1}\mf{p}$ for $1 \le i \le r$ and hence $\a\mf{p} \subseteq \mf{p}^{-1}\mf{p}$. By our assumption, this further implies that $\a\mf{p} \subseteq \mf{p}$. But then
        \[
          \a\w_{i} = \sum_{1 \le j \le r}\a_{i,j}\w_{j},
        \]
        with $\a_{i,j} \in \mc{O}$ for $1 \le i,j \le r$. We can rewrite this as,
        \[
          (\a-\a_{i,i})\w_{i}-\sum_{\substack{1 \le j \le r \\ j \neq i}}\a_{i,j}\w_{j} = 0,
        \]
        for all $i$. These $r$ equations are equivalent to the identity
        \[
          \begin{pmatrix} \a-\a_{1,1} & \a_{1,2} & \cdots & -\a_{1,r} \\ -\a_{2,1} & \a-\a_{2,2} & & \\ \vdots & & \ddots & \\ -\a_{r,1} & & & \a-\a_{r,r} \end{pmatrix}\begin{pmatrix} \w_{1} \\ \w_{2} \\ \vdots \\ \w_{r} \end{pmatrix} = \mathbf{0}.
        \]
        Thus the determinant of the matrix on the left-hand side must be zero. But this means $\a$ is a root of the characteristic polynomial $\det(xI-(\a_{i,j}))$ which is a monic polynomial with coefficients $\mc{O}$. As $\mc{O}$ is integrally closed, $\a \in \mc{O}$ which is a contraction. Thus $\mf{p}^{-1}\mf{p} = \mc{O}$ proving (ii).
      \end{enumerate}
    \end{proof}

    We can now show that every integral ideal factors uniquely into a product of prime integral ideals (up to reordering of the factors):

    \begin{theorem}\label{thm:unique_product_prime_ideals}
      Let $\mc{O}$ be a Dedekind domain. Then for every integral ideal $\mf{a}$ there exist prime integral ideals $\mf{p}_{1},\ldots,\mf{p}_{k}$ such that $\mf{a}$ factors as
      \[
        \mf{a} = \mf{p}_{1} \cdots \mf{p}_{k}.
      \]
      Moreover, this factorization is unique up to reordering of the factors.
    \end{theorem}
    \begin{proof}
      We first prove existence and then uniqueness. For existence, let $\mc{S}$ be the set of integral ideals that are not a product of prime integral ideals. We will show that $\mc{S}$ is empty. Assuming otherwise and ordering $\mc{S}$ by inclusion, the fact that $\mc{O}$ is noetherian implies that there exists a maximal integral ideal $\mf{a} \in \mc{S}$. Necessarily $\mf{a}$ is not prime and since prime integral ideals are maximal in $\mc{O}$, there is some prime integral ideal $\mf{p}_{1}$ for which $\mf{a} \subset \mf{p}_{1}$. Then by \cref{lem:inverse_for_prime_ideals} (ii), we have $\mf{p}_{1}^{-1}\mf{a} \subset \mc{O}$ so that $\mf{p}_{1}^{-1}\mf{a}$ is also an integral ideal. Also, \cref{lem:inverse_for_prime_ideals} (i) implies that $\mf{a} \subset \mf{a}\mf{p}_{1}^{-1}$. By maximality of $\mf{a}$, $\mf{a}\mf{p}_{1}^{-1}$ factors into a product of prime integral ideals. That is, there exist prime integral ideals $\mf{p}_{2},\ldots,\mf{p}_{k}$ such that
      \[
        \mf{a}\mf{p}_{1}^{-1} = \mf{p}_{2},\ldots,\mf{p}_{k}.
      \]
      Hence
      \[
        \mf{a} = \mf{p}_{1},\ldots,\mf{p}_{k},
      \]
      so that $\mf{a}$ factors into a product of prime integral ideals which contradicts the fact that $\mf{a} \in \mc{S}$. Hence $\mc{S}$ is empty thus proving the existence of such a factorization. Now we prove uniqueness. Suppose that $\mf{a}$ admits factorizations
      \[
        \mf{a} = \mf{p}_{1},\ldots,\mf{p}_{k} \quad \text{and} \quad \mf{a} = \mf{q}_{1},\ldots,\mf{q}_{\ell},
      \]
      for prime integral ideals $\mf{p}_{i}$ and $\mf{q}_{j}$ with $1 \le i \le k$ and $1 \le j \le \ell$. Since $\mf{p}_{1}$ is prime, there is some $j$ for which $\mf{q}_{j} \subseteq \mf{p}_{1}$. Without loss of generality, we may assume $\mf{q}_{1} \subseteq \mf{p}_{1}$ and since prime integral ideals are maximal in $\mc{O}$ we have $\mf{q}_{1} = \mf{p}_{1}$. Then
      \[
        \mf{p}_{2},\ldots,\mf{p}_{k} = \mf{q}_{2},\ldots,\mf{q}_{\ell}.
      \]
      Repeating this process, we see that $k = \ell$ and $\mf{q}_{i} = \mf{p}_{i}$ for all $i$. This proves uniqueness of the factorization.
    \end{proof}

    As a near immediate corollary of \cref{thm:unique_product_prime_ideals}, all fractional ideal admits a factorization into a product of prime integral ideals and their inverses (up to reordering of the factors):

    \begin{corollary}\label{cor:fractional_ideal_prime_factorization}
      Let $\mc{O}$ be a Dedekind domain. Then for every fractional ideal $\mf{f}$ there exist prime integral ideals $\mf{p}_{1},\ldots,\mf{p}_{k}$ and $\mf{q}_{1},\ldots,\mf{q}_{\ell}$ such that $\mf{f}$ factors as
      \[
        \mf{f} = \mf{p}_{1} \cdots \mf{p}_{k}\mf{q}_{1}^{-1},\ldots,\mf{q}_{\ell}^{-1}.
      \]
      Moreover, this factorization is unique up to reordering of the factors.
    \end{corollary}
    \begin{proof}
      If $\mf{f}$ is a fractional ideal, then there exists a nonzero $\d \in \mc{O}$ and an integral ideal $\mf{a}$ such that $\mf{f} = \frac{1}{\d}\mf{a}$. In particular, $\mf{a}$ and $\d\mc{O}$ are integral ideals such that $\d\mc{O}\mf{f} = \mf{a}$. By \cref{thm:unique_product_prime_ideals}, $\mf{a}$ and $\d\mc{O}$ admit unique factorizations
      \[
        \mf{a} = \mf{p}_{1} \cdots \mf{p}_{k} \quad \text{and} \quad \d\mc{O} = \mf{q}_{1},\ldots,\mf{q}_{\ell},
      \]
      for some prime integral ideals $\mf{p}_{1},\ldots,\mf{p}_{k}$ and $\mf{q}_{1},\ldots,\mf{q}_{\ell}$ up to reordering of the factors. Hence
      \[
        \mf{q}_{1},\ldots,\mf{q}_{\ell}\mf{f} = \mf{p}_{1} \cdots \mf{p}_{k},
      \]
      which is equivalent to the factorization for $\mf{f}$.
    \end{proof}

    By \cref{thm:unique_product_prime_ideals}, for any integral ideal $\mf{a}$ there exist distinct prime integral ideal $\mf{p}_{1},\ldots,\mf{p}_{r}$ such that $\mf{a}$ admits a factorization
    \[
      \mf{a} = \mf{p}_{1}^{e_{1}} \cdots \mf{p}_{r}^{e_{r}},
    \]
    with $e_{i} \ge 1$ for all $i$, called the \textbf{prime factorization}\index{prime factorization} of $\mf{a}$ with \textbf{prime factors}\index{prime factors} $\mf{p}_{1},\ldots,\mf{p}_{r}$. Just as it is common to suppress the fundamental theorem of arithmetic and just state the prime factorization of an integer, we suppress \cref{thm:unique_product_prime_ideals} and simply state the prime factorization of an integral ideal. With the prime factorization in hand, we can discuss the group structure of the fractional ideals of $K$. Let $I_{K}$ denote the set of fractional ideals of $K$. We call $I_{K}$ the \textbf{ideal group}\index{ideal group} of $K$. The following theorem shows that $I_{K}$ is indeed a group:

    \begin{theorem}
      Let $\mc{O}$ be a Dedekind domain with field of fractions $K$. Then $I_{K}$ is an abelian group with identity $\mc{O}$.
    \end{theorem}
    \begin{proof}
      It is clear that the product of fractional ideals is a fractional ideal. Associativity and commutativity of $I_{K}$ are also obvious. The identity is $\mc{O}$ because every fractional ideal is a finitely generated $\mc{O}$-submodule of $K$. It follows that $\mf{p}^{-1}$ is the inverse of any prime integral ideal $\mf{p}$ by \cref{lem:inverse_for_prime_ideals} (ii). Therefore every prime integral ideal is invertible. If $\mf{a}$ is a integral ideal then it admits a prime factorization $\mf{a} = \mf{p}_{1}^{e_{1}} \cdots \mf{p}_{r}^{e_{r}}$ and then $\mf{b} = \mf{p}_{1}^{-e_{1}} \cdots \mf{p}_{r}^{-e_{r}}$ is its inverse. Hence every integral ideal is invertible. If $\mf{f}$ is a fractional ideal, then there exists a nonzero $\d \in \mc{O}$ and an integral ideal $\mf{a}$ such that $\mf{f} = \frac{1}{\d}\mf{a}$ and hence $\mf{f}$ is invertible because $\d$ and $\mf{a}$ are. It follows that every fractional ideal is invertible which completes the proof.
    \end{proof}

    Now that we have proved that the ideal group $I_{K}$ of $K$ is indeed a group, we can also deduce the explicit form for the inverse of any fractional ideal:

    \begin{proposition}\label{prop:explicit_inverse_ideal}
      Let $\mc{O}$ be a Dedekind domain with field of fractions $K$ and let $\mf{f}$ be a fractional ideal. Then
      \[
        \mf{f}^{-1} = \{\k \in K:\k\mf{f} \subseteq \mc{O}\}.
      \]
      In particular, $\mc{O} \subseteq \mf{f}$ if and only if $\mf{f}^{-1}$ is an integral ideal.
    \end{proposition}
    \begin{proof}
      For an integral ideal $\mf{a}$, we have
      \[
        \mf{a}^{-1} = \{\k \in K:\k\mf{a} \subseteq \mc{O}\},
      \]
      by the unique factorization of $\mf{a}$ and the definition of the inverse of a prime integral ideal. If $\mf{f}$ is a fractional ideal, then there exists a nonzero $\d \in \mc{O}$ and an integral ideal $\mf{a}$ such that $\mf{f} = \frac{1}{\d}\mf{a}$. But then $\d\mf{f} = \mf{a}$ so that
      \[
        \frac{1}{\d}\mf{f}^{-1} = \{\k \in K:\k\d\mf{f} \subseteq \mc{O}\},
      \]
      which is equivalent to the first statement. For the second statement, if $\mc{O} \subseteq \mf{f}$ then multiplying by $\mf{f}^{-1}$ shows $\mf{f}^{-1} \subseteq \mc{O}$ and hence $\mf{f}^{-1}$ is an integral ideal. Running this argument backwards by multiplying by $\mf{f}$ proves the converse.
    \end{proof}

    We will now discuss applications of the Chinese remainder theorem in the context of integral ideals. With it we can prove some interesting results. First, we recall a useful fact. Since prime integral ideals are maximal in a Dedekind domain $\mc{O}$ and distinct maximal ideals are relatively prime, we see that distinct prime integral ideals $\mf{p}$ and $\mf{q}$ are relatively prime. In particular, their powers are relatively prime as well (which follows by induction). So suppose $\mf{a}$ is an integral ideal with prime factorization
    \[
      \mf{a} = \mf{p}_{1}^{e_{1}} \cdots \mf{p}_{r}^{e_{r}}.
    \]
    Then the integral ideals $\mf{p}_{1}^{e_{1}},\ldots,\mf{p}_{r}^{e_{r}}$ are pairwise relatively prime so that the Chinese remainder theorem gives an isomorphism
    \[
      \mc{O}/\mf{a} \cong \bigop_{1 \le i \le r}\mc{O}/\mf{p}_{i}^{e_{i}}.
    \]
    In particular, for any $\a_{i} \in \mc{O}$ for all $i$, there exists a unique $\a \in \mc{O}$ such that
    \[
      \a \equiv \a_{i} \pmod{\mf{p}_{i}^{e_{i}}},
    \]
    for all $i$. We can now describe the quotient of a Dedekind domain by a prime integral ideal:

    \begin{proposition}\label{prop:isomorphism_of_quotient_by_prime_integral_ideals}
      Let $\mc{O}$ be a Dedekind domain. Then for any prime integral ideal $\mf{p}$ and $n \ge 0$, we have an isomorphism
      \[
        \mc{O}/\mf{p} \cong \mf{p}^{n}/\mf{p}^{n+1},
      \]
      as $\mc{O}$-modules.
    \end{proposition}
    \begin{proof}
      By the uniqueness of the factorization of integral ideals, there exists $\b \in \mf{p}^{n}-\mf{p}^{n+1}$. Now consider the homomorphism
      \[
        \phi:\mc{O} \to \mf{p}^{n}/\mf{p}^{n+1} \qquad \a \mapsto \a\b \pmod{\mf{p}^{n+1}}.
      \]
      By the first isomorphism theorem, it suffices to show $\ker\phi = \mf{p}$ and that $\phi$ is surjective. Let us first show $\ker\phi = \mf{p}$. As $\b \in \mf{p}^{n}$, it is obvious that $\mf{p} \subseteq \ker\phi$. Conversely, suppose $\a \in \mc{O}$ is such that $\phi(\a) = 0$. Then $\a\b \in \mf{p}^{n+1}$ and as $\b \in \mf{p}^{n}-\mf{p}^{n+1}$, we must have $\a \in \mf{p}$. It follows that $\ker\phi = \mf{p}$. We now show that $\phi$ is surjective. Let $\g \in \mf{p}^{n}$ be a representative of a class in $\mf{p}^{n}/\mf{p}^{n+1}$. As $\b \in \mf{p}^{n}$, we have $\b\mc{O} \subseteq \mf{p}^{n}$. But since $\b \notin \mf{p}^{n+1}$, we see that $\b\mc{O}\mf{p}^{-n}$ is necessarily an integral ideal relatively prime to $\mf{p}^{n+1}$. As $\mf{p}^{n+1}$ and $\b\mc{O}\mf{p}^{-n}$ are relatively prime, the Chinese remainder theorem implies that we can find a unique $\a \in \mc{O}$ such that
      \[
        \a \equiv \g \pmod{\mf{p}^{n+1}} \quad \text{and} \quad \a \equiv 0 \tmod{\b\mc{O}\mf{p}^{-n}}.
      \]
      The second condition implies $\a \in \b\mc{O}\mf{p}^{-n}$. As $\g \in \mf{p}^{n}$ and $\a$ and $\g$ differ by an element in $\mf{p}^{n+1} \subset \mf{p}^{n}$, we have that $\a \in \b\mc{O}\mf{p}^{-n} \cap \mf{p}^{n} = \b\mc{O}$ where the equality holds because the intersection of ideals is equal to their product provided the ideals are relatively prime. Thus $\a\b^{-1} \in \mc{O}$ and hence
      \[
        \phi(\a\b^{-1}) = \a \equiv \g \tmod{\mf{p}^{n+1}}.
      \]
      This shows $\phi$ is surjective completing the proof.
    \end{proof}
    
    We now show that any fractional ideal is generated by at most two elements:

    \begin{corollary}\label{cor:fractional_ideal_generated_by_two_elements}
      Let $\mc{O}$ be a Dedekind domain. Then any fractional ideal is generated by at most two elements.
    \end{corollary}
    \begin{proof}
      We first prove the claim for an integral ideal $\mf{a}$. Let $\a \in \mf{a}$ be nonzero and let $\mf{p}_{1},\ldots,\mf{p}_{r}$ be the prime factors of $\a\mc{O}$. As $\a\mc{O} \subseteq \mf{a}$, the prime factorization of $\mf{a}$ is
      \[
        \mf{a} = \mf{p}_{1}^{e_{1}} \cdots \mf{p}_{r}^{e_{r}},
      \]
      with $e_{i} \ge 0$ for $1 \le i \le r$. By uniqueness of the prime factorization of integral ideals, $\mf{p}_{i}^{e_{i}+1} \subset \mf{p}_{i}^{e_{i}}$ for all $i$. Thus there exist nonzero $\b_{i} \in \mf{p}_{i}^{e_{i}}-\mf{p}_{i}^{e_{i}+1}$ for all $i$. Since $\mf{p}_{1}^{e_{1}+1},\ldots,\mf{p}_{r}^{e_{r}+1}$ are pairwise relatively prime, the Chinese remainder theorem implies that there exists $\b \in \mc{O}$ with 
      \[
        b \equiv \b_{i} \tmod{\mf{p}_{i}^{e_{i}+1}},
      \]
      for all $i$. As $\b_{i} \in \mf{p}_{i}^{e_{i}}$ for all $i$, we have $\b \in \mf{a}$ and hence $\b\mc{O} \subseteq \mf{a}$. But as $\b \notin \mf{p}_{i}^{e_{i}+1}$ for all $i$, we see that $\b\mc{O}\mf{a}^{-1}$ is necessarily an integral ideal relatively prime to $\a\mc{O}$. This means
      \[
        \b\mc{O}\mf{a}^{-1}+\a\mc{O} = \mc{O},
      \]
      and hence
      \[
        \b\mc{O}+\a\mf{a} = \mf{a}.
      \]
      But as $\a,\b \in \mf{a}$, we have $\b\mc{O}+\a\mf{a} \subseteq \b\mc{O}+\a\mc{O} \subseteq \mf{a}$ and so
      \[
        \b\mc{O}+\a\mc{O} = \mf{a}.
      \]
      This shows that $\mf{a}$ is generated by at most two elements. Now suppose $\mf{f}$ is a fractional ideal. Then there exists a nonzero $\d \in \mc{O}$ and an integral ideal $\mf{a}$ such that $\mf{f} = \frac{1}{\d}\mf{a}$. Since $\mf{a}$ is generated by at most two elements, say $\a$ and $\b$, we have
      \[
        \mf{f} = \frac{\a}{\d}\mc{O}+\frac{\b}{\d}\mc{O},
      \]
      and so $\mf{f}$ is also generated by at most two elements as well.
    \end{proof}

    \cref{cor:fractional_ideal_generated_by_two_elements} shows that while a Dedekind domain $\mc{O}$ may not be a principal ideal domain, it is not far off from one since every integral ideal needs at most two generators. We can give a more refined interpretation of this using the ideal group $I_{K}$. Let $P_{K}$ denote the subgroup of $I_{K}$ of principal ideals $\a\mc{O}$ for nonzero $\a \in K$. Since $I_{K}$ is abelian, $P_{K}$ is normal. The \textbf{ideal class group}\index{ideal class group} $\Cl(K)$ of $K$ is defined to be the quotient group
    \[
      \Cl(K) = I_{K}/P_{K},
    \]
    of fractional ideals modulo principal ideals. We call an element of $\Cl(K)$ an \textbf{ideal class}\index{ideal class} of $K$. The \textbf{class number}\index{class number} $h_{K}$ of $K$ is defined by
    \[
      h_{K} = |\Cl(K)|.
    \]
    The ideal class group is an object which encodes how much $\mc{O}$ fails to be a principal ideal domain and the class number $h_{K}$ is a measure of the degree of failure. For example, $\mc{O}$ is a principal ideal domain if and only if $h_{K} = 1$. Indeed, if $\mc{O}$ is a principal ideal domain then every integral ideal is principal and hence every fractional ideal is too (because every fractional ideal is of the form $\frac{1}{\d}\mf{a}$ for some nonzero $\d \in \mc{O}$ and an integral ideal $\mf{a}$). But then $\Cl(K)$ is the trivial group an hence $h_{K} = 1$. Conversely, if $h_{K} = 1$ then every integral ideal is principal so that $\mc{O}_{K}$ is a principal ideal domain.

    \begin{remark}\label{rem:general_class_number_not_finite}
      The class number $h_{K}$ need not be finite for a general Dedekind domain $\mc{O}$ with field of fractions $K$.
    \end{remark}
    
    The \textbf{unit group}\index{unit group} of $K$ is defined to be $\mc{O}^{\ast}$. That is, the unit group is the group of units in the ring of integers of $K$. By abuse of language, we call any element $\mc{O}^{\ast}$ a \textbf{unit}\index{unit} of $K$. That is, a unit in $K$ is precisely an invertible element of $\mc{O}$. The ideal class group and the unit group of $K$ are related via the exact sequence

    \begin{center}
      \begin{tikzcd}
        1 \arrow{r} & \mc{O}^{\ast} \arrow{r} & K^{\ast} \arrow{r}{\cdot\mc{O}} & I_{K} \arrow{r} & \Cl(K) \arrow{r} & 1,
      \end{tikzcd}
    \end{center}

    where the middle map takes any $\k \in K^{\ast}$ to its associated principal ideal $\k\mc{O}$. Thinking of this map as passing from numbers in $K^{\ast}$ to fractional ideals in $I_{K}$, exactness means that unit group is measuring the contraction (how many numbers are annihilated) taking place during this process while the class group is measuring the expansion (how many fractional ideal are created).