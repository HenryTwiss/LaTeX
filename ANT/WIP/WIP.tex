\documentclass[12pt,oneside]{book}
\usepackage{import}
%===============================%
%  Packages and basic settings  %
%===============================%
\usepackage[headheight=15pt,rmargin=0.5in,lmargin=0.5in,tmargin=0.75in,bmargin=0.75in]{geometry}
\usepackage{imakeidx}
\usepackage{framed}
\usepackage{amssymb}
\usepackage{amsmath}
\usepackage{mathrsfs}
\usepackage{enumitem}
\usepackage{hyperref}
\usepackage{appendix}
\usepackage[capitalise,noabbrev]{cleveref}
\usepackage{tikz}
\usepackage{tikz-cd}
\usepackage{nomencl}\makenomenclature
\usetikzlibrary{braids,arrows,decorations.markings,calc}

%====================================%
%  Theorems, environments & cleveref  %
%====================================%
\newtheorem{theorem}{Theorem}[section]
\newtheorem{proposition}{Proposition}[section]
\newtheorem{corollary}{Corollary}[section]
\newtheorem{lemma}{Lemma}[section]
\newtheorem{conjecture}{Conjecture}[section]
\newtheorem{remark}{Remark}[section]

\newenvironment{stabular}[2][1]
  {\def\arraystretch{#1}\tabular{#2}}
  {\endtabular}

%==================================%
%  Custom commands & environments  %
%==================================%
\newcommand{\legendre}[2]{\left(\frac{#1}{#2}\right)}
\newcommand{\dlegendre}[2]{\displaystyle{\left(\frac{#1}{#2}\right)}}
\newcommand{\tlegendre}[2]{\textstyle{\left(\frac{#1}{#2}\right)}}
\newcommand{\psum}{\sideset{}{'}\sum}
\newcommand{\asum}{\sideset{}{^{\ast}}\sum}
\newcommand{\tmod}[1]{\ \left(\text{mod }#1\right)}
\newcommand{\xto}[1]{\xrightarrow{#1}}
\newcommand{\xfrom}[1]{\xleftarrow{#1}}
\newcommand{\normal}{\mathrel{\unlhd}}
\newcommand{\mf}{\mathfrak}
\newcommand{\mc}{\mathcal}
\newcommand{\ms}{\mathscr}

\newcommand{\Mat}{\mathrm{Mat}}
\newcommand{\GL}{\mathrm{GL}}
\newcommand{\SL}{\mathrm{SL}}
\newcommand{\PSL}{\mathrm{PSL}}
\renewcommand{\O}{\mathrm{O}}
\newcommand{\SO}{\mathrm{SO}}
\newcommand{\U}{\mathrm{U}}
\newcommand{\Sp}{\mathrm{Sp}}

\newcommand{\N}{\mathbb{N}}
\newcommand{\Z}{\mathbb{Z}}
\newcommand{\Q}{\mathbb{Q}}
\newcommand{\R}{\mathbb{R}}
\newcommand{\C}{\mathbb{C}}
\newcommand{\F}{\mathbb{F}}
\renewcommand{\H}{\mathbb{H}}
\renewcommand{\P}{\mathbb{P}}

\renewcommand{\a}{\alpha}
\renewcommand{\b}{\beta}
\newcommand{\g}{\gamma}
\renewcommand{\d}{\delta}
\newcommand{\z}{\zeta}
\renewcommand{\t}{\theta}
\renewcommand{\i}{\iota}
\renewcommand{\k}{\kappa}
\renewcommand{\l}{\lambda}
\newcommand{\s}{\sigma}
\newcommand{\w}{\omega}

\newcommand{\G}{\Gamma}
\newcommand{\D}{\Delta}
\renewcommand{\L}{\Lambda}
\newcommand{\W}{\Omega}

\newcommand{\e}{\varepsilon}
\newcommand{\vt}{\vartheta}
\newcommand{\vphi}{\varphi}
\newcommand{\emt}{\varnothing}

\newcommand{\x}{\times}
\newcommand{\ox}{\otimes}
\newcommand{\op}{\oplus}
\newcommand{\bigox}{\bigotimes}
\newcommand{\bigop}{\bigoplus}
\newcommand{\del}{\partial}
\newcommand{\<}{\langle}
\renewcommand{\>}{\rangle}
\newcommand{\lf}{\lfloor}
\newcommand{\rf}{\rfloor}
\newcommand{\wtilde}{\widetilde}
\newcommand{\what}{\widehat}
\newcommand{\conj}{\overline}
\newcommand{\cchi}{\conj{\chi}}

\DeclareMathOperator{\id}{\textrm{id}}
\DeclareMathOperator{\sgn}{\mathrm{sgn}}
\DeclareMathOperator{\im}{\mathrm{im}}
\DeclareMathOperator{\rk}{\mathrm{rk}}
\DeclareMathOperator{\tr}{\mathrm{trace}}
\DeclareMathOperator{\nm}{\mathrm{norm}}
\DeclareMathOperator{\ord}{\mathrm{ord}}
\DeclareMathOperator{\Hom}{\mathrm{Hom}}
\DeclareMathOperator{\End}{\mathrm{End}}
\DeclareMathOperator{\Aut}{\mathrm{Aut}}
\DeclareMathOperator{\Tor}{\mathrm{Tor}}
\DeclareMathOperator{\Ann}{\mathrm{Ann}}
\DeclareMathOperator{\Gal}{\mathrm{Gal}}
\DeclareMathOperator{\Trace}{\mathrm{Trace}}
\DeclareMathOperator{\Norm}{\mathrm{Norm}}
\DeclareMathOperator{\Span}{\mathrm{Span}}
\DeclareMathOperator*{\Res}{\mathrm{Res}}
\DeclareMathOperator{\Vol}{\mathrm{Vol}}
\DeclareMathOperator{\Li}{\mathrm{Li}}
\renewcommand{\Re}{\mathrm{Re}}
\renewcommand{\Im}{\mathrm{Im}}

\newcommand{\GH}{\G\backslash\H}
\newcommand{\GG}{\G_{\infty}\backslash\G}

\newenvironment{psmallmatrix}
  {\left(\begin{smallmatrix}}
  {\end{smallmatrix}\right)}

%============%
%  Comments  %
%============%
\newcommand{\todo}[1]{\textcolor{red}{\sf Todo: [#1]}}

%===================%
%  Label reminders  %
%===================%
% [label=(\roman*)]
% [label=(\alph*)]
% [label=(\arabic{enumi})]

%==================%
%  Other settings  %
%==================%
\pgfdeclarelayer{background}
\pgfsetlayers{background,main}
\tikzset{->-/.style={decoration={
  markings,
  mark=at position .5 with {\arrow{>}}},postaction={decorate}}}

%=================%
%  Title & Index  %
%=================%
\title{Analytic Number Theory}
\author{Henry Twiss}
\date{2024}
\makeindex

\begin{document}
  \section{\todo{The Kuznetsov Trace Formula}}
    The Kuznetsov trace formula is an analog of the Petersson trace formula for weight zero Maass forms. From \cref{thm:the_full_spectral_resolution}, $\mc{L}(N,\chi)$ admits an orthonormal basis of Maass forms for the point spectrum (these forms are generally not Hecke-Maass eigenforms because they need not be Hecke normalized or even cuspidal in the case of the discrete spectrum). However, by \cref{prop:residual_forms_weight_zero} and \cref{thm:newforms_characterization_Maass} we make take this orthonormal basis to consist of Hecke-Maass eigenforms and the constant function. Denote this basis by $\{u_{j}\}_{j \ge 0}$ with $u_{0}(z) = 1$ and let $u_{j}$ be of type $\nu_{j}$ for $j \ge 1$. In particular, $\{u_{j}\}_{j \ge 1}$ is an orthonormal basis of Hecke-Maass eigenforms and each such form admits a Fourier series at the $\mf{a}$ cusp given by
    \[
      (u_{j}|\s_{\mf{a}})(z) = \sum_{n \neq 0}a_{j,\mf{a}}(n)\sqrt{y}K_{\nu_{j}}(2\pi ny)e^{2\pi inx}.
    \]
    The Kuznetsov trace formula is an equation relating the Fourier coefficients $a_{j,\mf{a}}(n)$ and $a_{j,\mf{b}}(n)$ of the basis $\{u_{j}\}_{j \ge 1}$ for two cusps $\mf{a}$ and $\mf{b}$ of $\G_{0}(N)\backslash\H$ to a sum of integral transforms involving test functions and Sali\'e sums. Similar to the Petersson trace formula, we will compute the inner product of two Poincar\'e series $P_{n,\chi,\mf{a}}(z,\psi)(z)$ and $P_{m,\chi,\mf{b}}(z,\vphi)(z)$ in two different ways. The first will be geometric in nature while the second will be spectral. We first need to compute the Fourier series of such a Poincar\'e series. Although we will not need it explicitly, we will work over any congruence subgroup:

    \begin{proposition}
      Let $m \ge 1$, $\chi$ be Dirichlet character with conductor dividing the level, $\mf{a}$ and $\mf{b}$ be cusps of $\GH$, and $\psi(y)$ be a smooth function such that $\psi(y) \ll_{\e} y^{1+\e}$ as $y \to 0$. The Fourier series of $P_{m,\chi,\mf{a}}(z,\psi)$ on $\GH$ at the $\mf{b}$ cusp is given by
      \[
        (P_{m,\chi,\mf{a}}|\s_{\mf{b}})(z,\psi) = \sum_{t \in \Z}\left(\d_{\mf{a},\mf{b}}\d_{m,t}\psi(\Im(z))+\sum_{\substack{c \in \mc{C}_{\mf{a},\mf{b}}}}\psi(y,m,t,c)S_{\chi,\mf{a},\mf{b}}(m,t,c)\right)e^{2\pi itz},
      \]
      where $\psi(y,m,t,c)$ is the integral transform given by
      \[
        \psi(y,m,t,c) = \int_{\Im(z) = y}\psi\left(\frac{y}{|cz|^{2}}\right)e^{-\frac{2\pi im}{c^{2}z}-2\pi itz}\,dz.
      \]
    \end{proposition}
    \begin{proof}
      From the cocycle condition and \cref{rem:Bruhat_modulo_infity_exact}, we have
      \[
        (P_{m,\chi,\mf{a}}|\s_{\mf{b}})(z,\psi) = \d_{\mf{a},\mf{{b}}}\psi(\Im(z))e^{2\pi imz}+\sum_{\substack{c \in \mc{C}_{\mf{a},\mf{b}}, d \in \Z \\ d \tmod{c} \in \mc{D}_{\mf{a},\mf{b}}(c)}}\cchi(d)\psi\left(\frac{\Im(z)}{|cz+d|^{2}}\right)e^{2\pi im\left(\frac{a}{c}-\frac{1}{c^{2}z+cd}\right)},
      \]
      where $a$ and $b$ are chosen such that $\det\left(\begin{psmallmatrix} a & b \\ c & d \end{psmallmatrix}\right) = 1$ and we have used the fact that
      \[
        \frac{a}{c}-\frac{1}{c^{2}z+cd} = \frac{az+b}{cz+d}.
      \]
      Summing over all pairs $(c,d)$ with $c \in \mc{C}_{\mf{a},\mf{b}}$, $d \in \Z$, and $d \in \mc{D}_{\mf{a},\mf{b}}(c)$ is the same as summing over all triples $(c,\ell,r)$ with $c \in \mc{C}_{\mf{a},\mf{b}}$, $\ell \in \Z$, and $r$ taken modulo $c$ with $r \in \mc{D}_{\mf{a},\mf{b}}(c)$. Indeed, this is seen by writing $d = c\ell+r$. Moreover, since $ad-bc = 1$ we have $a(c\ell+r)-bc = 1$ which further implies that $ar \equiv 1 \tmod{c}$. So we may take $a$ to be the inverse for $r$ modulo $c$. Then
      \begin{align*}
        \sum_{\substack{c \in \mc{C}_{\mf{a},\mf{b}}, d \in \Z \\ d \tmod{c} \in \mc{D}_{\mf{a},\mf{b}}(c)}}\cchi(d)\psi\left(\frac{\Im(z)}{|cz+d|^{2}}\right)e^{2\pi im\left(\frac{a}{c}-\frac{1}{c^{2}z+cd}\right)} &= \sum_{(c,\ell,r)}\cchi(c\ell+r)\psi\left(\frac{\Im(z)}{|cz+c\ell+r|^{2}}\right)e^{2\pi im\left(\frac{a}{c}-\frac{1}{c^{2}z+c^{2}\ell+cr}\right)} \\
        &= \sum_{(c,\ell,r)}\cchi(r)\psi\left(\frac{\Im(z)}{|cz+c\ell+r|^{2}}\right)e^{2\pi im\left(\frac{a}{c}-\frac{1}{c^{2}z+c^{2}\ell+cr}\right)} \\
        &= \sum_{\substack{c \in \mc{C}_{\mf{a},\mf{b}} \\ r \in \mc{D}_{\mf{a},\mf{b}}(c)}}\sum_{\ell \in \Z}\cchi(r)\psi\left(\frac{\Im(z)}{|cz+c\ell+r|^{2}}\right)e^{2\pi im\left(\frac{a}{c}-\frac{1}{c^{2}z+c^{2}\ell+cr}\right)} \\
        &= \sum_{\substack{c \in \mc{C}_{\mf{a},\mf{b}} \\ r \in \mc{D}_{\mf{a},\mf{b}}(c)}}\cchi(r)\sum_{\ell \in \Z}\psi\left(\frac{\Im(z)}{|cz+c\ell+r|^{2}}\right)e^{2\pi im\left(\frac{a}{c}-\frac{1}{c^{2}z+c^{2}\ell+cr}\right)},
      \end{align*}
     where on the right-hand side it is understood that we are summing over all triples $(c,\ell,r)$ with the prescribed properties and the second line holds since $\chi$ has conductor diving the level and $d \in \mc{D}_{\mf{a},\mf{b}}(c)$ is determined modulo $c$. Now let
      \[
        I_{c,r}(z,\psi) = \sum_{\ell \in \Z}\psi\left(\frac{\Im(z)}{|cz+c\ell+r|^{2}}\right)e^{2\pi im\left(\frac{a}{c}-\frac{1}{c^{2}z+c^{2}\ell+cr}\right)}.
      \]
      We apply the Poisson summation formula to $I_{c,r}(z,\psi)$. This is allowed since the summands are absolutely integrable by \cref{prop:decay_unbounded_inteval_integral}, as they exhibit polynomial decay of order $\s > 1$ because $\psi(y) \ll_{\e} y^{1+\e}$ as $y \to 0$, and $I_{c,r}(z,\psi)$ is holomorphic because $(P_{m,\chi,\mf{a}}|\s_{\mf{b}})(z,\psi)$ is. By the identity theorem it suffices to apply the Poisson summation formula for $z = iy$ with $y > 0$. So let $f(x)$ be given by
      \[
        f(x) = \psi\left(\frac{y}{|cx+r+icy|^{2}}\right)e^{2\pi im\left(\frac{a}{c}-\frac{1}{c^{2}x+cr+ic^{2}y}\right)}.
      \]
      As we have just noted, $f(x)$ is absolutely integrable on $\R$. We compute the Fourier transform:
      \[
        \hat{f}(t) = \int_{-\infty}^{\infty}f(x)e^{-2\pi itx}\,dx = \int_{-\infty}^{\infty}\psi\left(\frac{y}{|cx+r+icy|^{2}}\right)e^{2\pi im\left(\frac{a}{c}-\frac{1}{c^{2}x+cr+ic^{2}y}\right)}e^{-2\pi itx}\,dx.
      \]
      Complexify the integral to get
      \[
        \int_{\Im(z) = 0}\psi\left(\frac{y}{|cz+r+icy|^{2}}\right)e^{2\pi im\left(\frac{a}{c}-\frac{1}{c^{2}z+cr+ic^{2}y}\right)}e^{-2\pi itz}\,dz.
      \]
      Now make the change of variables $z \to z-\frac{r}{c}-iy$ to obtain
      \[
        e^{2\pi im\frac{a}{c}+2\pi it\frac{r}{c}-2\pi ty}\int_{\Im(z) = y}\psi\left(\frac{y}{|cz|^{2}}\right)e^{-\frac{2\pi im}{c^{2}z}-2\pi itz}\,dz.
      \]
      As the remaining integral is $\psi(y,m,t,c)$, it follows that
      \[
        \hat{f}(t) = \psi(y,m,t,c)e^{2\pi im\frac{a}{c}+2\pi it\frac{r}{c}-2\pi ty}.
      \]
      By the Poisson summation formula and the identity theorem, we have
      \[
        I_{c,r}(z,\psi) = \sum_{t \in \Z}(\psi(y,m,t,c)e^{2\pi im\frac{a}{c}+2\pi it\frac{r}{c}})e^{2\pi itz},
      \]
      for all $z \in \H$. Substituting this back into the Eisenstein series gives a form of the Fourier series:
      \begin{align*}
        (P_{m,\chi,\mf{a}}|\s_{\mf{b}})(z,\psi) &= \d_{\mf{a},\mf{{b}}}\psi(\Im(z))e^{2\pi imz}+\sum_{\substack{c \in \mc{C}_{\mf{a},\mf{b}} \\ r \in \mc{D}_{\mf{a},\mf{b}}}}\cchi(r)\sum_{t \in \Z}\psi(y,m,t,c)e^{2\pi im\frac{a}{c}+2\pi it\frac{r}{c}}e^{2\pi itz} \\
        &= \sum_{t \in \Z}\left(\d_{\mf{a},\mf{b}}\d_{m,t}\psi(\Im(z))+\sum_{\substack{c \in \mc{C}_{\mf{a},\mf{b}} \\ r \in \mc{D}_{\mf{a},\mf{b}}}}\cchi(r)\psi(y,m,t,c)e^{2\pi im\frac{a}{c}+2\pi it\frac{r}{c}}\right)e^{2\pi itz} \\
        &= \sum_{t \in \Z}\left(\d_{\mf{a},\mf{b}}\d_{m,t}\psi(\Im(z))+\sum_{\substack{c \in \mc{C}_{\mf{a},\mf{b}}}}\psi(y,m,t,c)\sum_{r \in \mc{D}_{\mf{a},\mf{b}}}\cchi(r)e^{2\pi im\frac{a}{c}+2\pi it\frac{r}{c}}\right)e^{2\pi itz}.
      \end{align*}
      We will simplify the innermost sum. Since $a$ is the inverse for $r$ modulo $c$, the innermost sum above becomes
      \[
        \sum_{r \in \mc{D}_{\mf{a},\mf{b}}}\cchi(r)e^{2\pi im\frac{a}{c}+2\pi it\frac{r}{c}} = \sum_{r \in \mc{D}_{\mf{a},\mf{b}}}\cchi(\conj{a})e^{2\pi im\frac{a}{c}+2\pi it\frac{\conj{a}}{c}} = \sum_{r \in \mc{D}_{\mf{a},\mf{b}}}\chi(a)e^{\frac{2\pi i(am+\conj{a}t)}{c}} = S_{\chi,\mf{a},\mf{b}}(m,t,c).
      \]
      So at last, we obtain our desired Fourier series:
      \[
        (P_{m,\chi,\mf{a}}|\s_{\mf{b}})(z) = \sum_{t \in \Z}\left(\d_{\mf{a},\mf{b}}\d_{m,t}\psi(\Im(z))+\sum_{\substack{c \in \mc{C}_{\mf{a},\mf{b}}}}\psi(y,m,t,c)S_{\chi,\mf{a},\mf{b}}(m,t,c)\right)e^{2\pi itz}.
      \]
    \end{proof}

    We can now derive the first half of the Kuznetsov trace formula by computing the inner product between $P_{n,\chi,\mf{a}}(z,\psi)$ and $P_{m,\chi,\mf{b}}(z,\vphi)$: 
    \begin{align*}
      \<P_{n,\chi,\mf{a}}(\cdot,\psi),P_{m,\chi,\mf{b}}(\cdot,\vphi)\> &= \frac{1}{V_{\G_{0}(N)}}\int_{\mc{F}_{\G_{0}(N)}}P_{n,\chi,\mf{a}}(z,\psi)\conj{P_{m,\chi,\mf{b}}(z,\vphi)}\,d\mu \\
      &= \frac{1}{V_{\G_{0}(N)}}\int_{\mc{F}_{\G_{0}(N)}}\sum_{\g \in \G_{\mf{b}}\backslash\G_{0}(N)}\chi(\g)P_{n,\chi,\mf{a}}(z,\psi)\conj{\vphi(\Im(\s_{\mf{b}}^{-1}\g z))}e^{-2\pi im\conj{\s_{\mf{b}}^{-1}\g z}}\,d\mu \\
      &= \frac{1}{V_{\G_{0}(N)}}\int_{\mc{F}_{\G_{0}(N)}}\sum_{\g \in \G_{\mf{b}}\backslash\G_{0}(N)}P_{n,\chi,\mf{a}}(\g z,\psi)\conj{\vphi(\Im(\s_{\mf{b}}^{-1}\g z))}e^{-2\pi im\conj{\s_{\mf{b}}^{-1}\g z}}\,d\mu \\
      &= \frac{1}{V_{\G_{0}(N)}}\int_{\mc{F}_{\s_{\mf{b}}^{-1}\G_{0}(N)\s_{\mf{b}}}}\sum_{\g \in \G_{\mf{b}}\backslash\G_{0}(N)}P_{n,\chi,\mf{a}}(\g\s_{\mf{b}}z,\psi)\conj{\vphi(\Im(\s_{\mf{b}}^{-1}\g\s_{\mf{b}}z))}e^{-2\pi im\conj{\s_{\mf{b}}^{-1}\g\s_{\mf{b}}z}}\,d\mu \\
      &= \frac{1}{V_{\G_{0}(N)}}\int_{\mc{F}_{\s_{\mf{b}}^{-1}\G_{0}(N)\s_{\mf{b}}}}\sum_{\g \in \G_{\infty}\backslash\s_{\mf{b}}^{-1}\G_{0}(N)\s_{\mf{b}}^{-1}}P_{n,\chi,\mf{a}}(\s_{\mf{b}}\g z,\psi)\conj{\vphi(\Im(\g z))}e^{-2\pi im\conj{\g z}}\,d\mu \\
      &= \frac{1}{V_{\G_{0}(N)}}\int_{\G_{\infty}\backslash\H}(P_{n,\chi,\mf{a}}|\s_{\mf{b}})(z,\psi)\conj{\vphi(\Im(z))}e^{-2\pi im\conj{z}}\,d\mu,
    \end{align*}
    where in the third line we have used the automorphy of $P_{n,\chi,\mf{a}}(z,\psi)$, in the forth and fifth lines we have made the change of variables $z \to \s_{\mf{b}}z$ and $\g \to \s_{\mf{b}}\g\s_{\mf{b}}^{-1}$ respectively, and in the sixth line we have unfolded. Now substitute in the Fourier series of $P_{n,\chi,\mf{a}}(z,\psi)$ at the $\mf{b}$ cusp to obtain
    \[
      \frac{1}{V_{\G_{0}(N)}}\int_{\G_{\infty}\backslash\H}\sum_{t \in \Z}\left(\d_{\mf{a},\mf{b}}\d_{n,t}\psi(\Im(z))+\sum_{\substack{c \in \mc{C}_{\mf{a},\mf{b}}}}\psi(y,n,t,c)S_{\chi,\mf{a},\mf{b}}(n,t,c)\right)\conj{\vphi(\Im(z))}e^{2\pi itz-2\pi im\conj{z}}\,d\mu,
    \]
    which is equivalent to
    \[
      \frac{1}{V_{\G_{0}(N)}}\int_{0}^{\infty}\int_{0}^{1}\sum_{t \ge 1}\left(\d_{\mf{a},\mf{b}}\d_{n,t}\psi(y)+\sum_{\substack{c \in \mc{C}_{\mf{a},\mf{b}}}}\psi(y,n,t,c)S_{\chi,\mf{a},\mf{b}}(n,t,c)\right)\conj{\vphi(y)}e^{2\pi i(t-m)x}e^{-2\pi(t+m)y}\,\frac{dx\,dy}{y^{2}}.
    \]
    By the dominated convergence theorem, we can interchange the sum and the two integrals. Then \cref{equ:Dirac_integral_representation} implies that the inner integral cuts off all of the terms except the diagonal $t = m$. This leaves
    \[
      \frac{1}{V_{\G_{0}(N)}}\int_{0}^{\infty}\left(\d_{\mf{a},\mf{b}}\d_{n,m}\psi(y)+\sum_{\substack{c \in \mc{C}_{\mf{a},\mf{b}}}}\psi(y,n,m,c)S_{\chi,\mf{a},\mf{b}}(n,m,c)\right)\conj{\vphi(y)}e^{-4\pi my}\,\frac{dy}{y^{2}}.
    \]
    Interchanging the integral and the remaining sum by the dominated convergence theorem again, we arrive at
    \[
      \<P_{n,\chi,\mf{a}}(\cdot,\psi),P_{m,\chi,\mf{b}}(\cdot,\vphi)\> = \d_{\mf{a},\mf{b}}\d_{n,m}(\psi,\vphi)_{n,m}+\sum_{c \in \mc{C}_{\mf{a},\mf{b}}}S_{\chi,\mf{a},\mf{b}}(n,m,c)V(n,m,c,\psi,\vphi),
    \]
    where we have set
    \[
      (\psi,\vphi)_{n,m} = \frac{1}{V_{\G_{0}(N)}}\int_{0}^{\infty}\psi(y)\conj{\vphi(y)}e^{-2\pi(n+m)y}\,\frac{dy}{y^{2}},
    \]
    and
    \[
      V(n,m,c;\psi,\vphi) = \frac{1}{V_{\G_{0}(N)}}\int_{0}^{\infty}\int_{\Im(z) = y}\psi\left(\frac{y}{|cz|^{2}}\right)\conj{\vphi(y)}e^{-\frac{2\pi im}{c^{2}z}-2\pi inz-4\pi my}\,\frac{dz\,dy}{y^{2}}.
    \]
    This is the first half of the Kuznetsov trace formula. For the second half, \cref{thm:the_full_spectral_resolution} gives
    \[
      P_{n,\chi,\mf{a}}(\cdot,\psi) = \sum_{j \ge 0}\<P_{n,\chi,\mf{a}}(\cdot,\psi),u_{j}\>u_{j}(z)+\sum_{\mf{a}}\frac{1}{4\pi}\int_{-\infty}^{\infty}\left\<P_{n,\chi,\mf{a}}(\cdot,\psi),E_{\mf{a}}\left(\cdot,\frac{1}{2}+ir\right)\right\>E_{\mf{a}}\left(z,\frac{1}{2}+ir\right)\,dr,
    \]
    and
    \[
      P_{m,\chi,\mf{a}}(\cdot,\vphi) = \sum_{j \ge 0}\<P_{m,\chi,\mf{a}}(\cdot,\vphi),u_{j}\>u_{j}(z)+\sum_{\mf{a}}\frac{1}{4\pi}\int_{-\infty}^{\infty}\left\<P_{m,\chi,\mf{a}}(\cdot,\vphi),E_{\mf{a}}\left(\cdot,\frac{1}{2}+ir\right)\right\>E_{\mf{a}}\left(z,\frac{1}{2}+ir\right)\,dr.
    \]
    By orthonormality, it follows that
    \begin{align*}
      \<P_{n,\chi,\mf{a}}(\cdot,\psi),P_{m,\chi,\mf{a}}(\cdot,\vphi)\> &= \sum_{j}\<P_{n,\chi,\mf{a}}(\cdot,\psi),u_{j}\>\conj{\<P_{m,\chi,\mf{a}}(\cdot,\vphi),u_{j}\>} \\
      &+\sum_{\mf{a}}\frac{1}{4\pi}\int_{-\infty}^{\infty}\left\<P_{n,\chi,\mf{a}}(\cdot,\psi),E_{\mf{a}}\left(\cdot,\frac{1}{2}+ir\right)\right\>\conj{\left\<P_{m,\chi,\mf{a}}(\cdot,\vphi),E_{\mf{a}}\left(\cdot,\frac{1}{2}+ir\right)\right\>}\,dr.
    \end{align*}
    Now we must simplify the remaining inner products. Let $f \in \mc{L}(N,\chi)$ with Fourier series
    \[
      f(z) = a^{+}(0)y^{\frac{1}{2}+\nu}+a^{-}(0)y^{\frac{1}{2}-\nu}+\sum_{n \neq 0}a(n)\sqrt{y}K_{\nu}(2\pi|n|y)e^{2\pi inx}.
    \]
    By unfolding the integral in the Petersson inner product and cutting off everything except the diagonal using \cref{equ:Dirac_integral_representation} exactly as in the case for $\<P_{n,\chi,\mf{a}}(\cdot,\psi),P_{m,\chi,\mf{a}}(\cdot,\vphi)\>$, we see that
    \[
      \<P_{n,\chi,\mf{a}}(\cdot,\psi),f\> = \frac{1}{V_{\G}}\int_{0}^{\infty}\conj{a(n)\sqrt{y}K_{\nu}(2\pi ny)}\psi(y)e^{-4\pi my}\frac{dy}{y^{2}}.
    \]
    Now set
    \[
      \w_{\nu}(n,\psi) = \frac{1}{V_{\G}}\int_{0}^{\infty}\sqrt{y}K_{\nu}(2\pi|n|y)\conj{\psi(y)}e^{-4\pi my}\frac{dy}{y^{2}}.
    \]
    Then it follows from the Fourier series of cusp forms and Eisenstein series that
    \[
      \<P_{n,\chi,\mf{a}}(\cdot,\psi),u_{j}\> = \conj{a_{j}(n)\w_{\nu_{j}}(n,\psi)},
    \]
    for $j \ge 1$ and
    \[
      \left\<P_{n,\chi,\mf{a}}(\cdot,\psi),E_{\mf{a}}\left(\cdot,\frac{1}{2}+ir\right)\right\> = \conj{\tau_{\mf{a}}\left(n,\frac{1}{2}+ir\right)\w_{ir}(n,\psi)}.
    \]
    In particular, $\<P_{n,\chi,\mf{a}}(\cdot,\psi),u_{0}\> = 0$. So we obtain
    \begin{align*}
      \<P_{n,\chi,\mf{a}}(\cdot,\psi),P_{m,\chi,\mf{a}}(\cdot,\vphi)\> &= \sum_{j \ge 1}\conj{a_{j}(n)}a_{j}(m)\conj{\w(n,\psi)}\w(m,\vphi) \\
      &+\sum_{\mf{a}}\frac{1}{4\pi}\int_{-\infty}^{\infty}\conj{\tau_{\mf{a}}\left(n,\frac{1}{2}+ir\right)}\tau_{\mf{a}}\left(m,\frac{1}{2}+ir\right)\conj{\w(n,\psi)}\w(m,\vphi)\,dr.
    \end{align*}
    This is the second half of the Kuznetsov trace formula. Equating the first and second halves we get the \textbf{Kuznetsov trace formula}\index{Kuznetsov trace formula}:
    \begin{align*}
      \d_{n,m}(\psi,\vphi)+\sum_{\substack{c \ge 1 \\ c \equiv 0 \tmod{N}}}\frac{1}{c}S_{\chi}(n,m,c)&V(n,m,c,\psi,\vphi) = \sum_{j \ge 1}\conj{a_{j}(n)}a_{j}(m)\conj{\w(n,\psi)}\w(m,\vphi) \\
      &+\sum_{\mf{a}}\frac{1}{4\pi}\int_{-\infty}^{\infty}\conj{\tau_{\mf{a}}\left(n,\frac{1}{2}+ir\right)}\tau_{\mf{a}}\left(m,\frac{1}{2}+ir\right)\conj{\w(n,\psi)}\w(m,\vphi)\,dr.
    \end{align*}
    The left-hand side is called the \textbf{geometric side}\index{geometric side} and the right-hand side is called the \textbf{spectral side}\index{spectral side}. We collect our work as a theorem:

    \begin{theorem}[Kuznetsov trace formula]
      Let $\{u_{j}\}_{j \ge 1}$ be an orthonormal basis of Hecke-Maass \\ eigenforms for $\mc{L}(N,\chi)$ of types $\nu_{j}$ with Fourier coefficients $a_{j}(n)$. Then for any positive integers $n,m \ge 1$, we have
      \begin{align*}
        \d_{n,m}(\psi,\vphi)+\sum_{\substack{c \ge 1 \\ c \equiv 0 \tmod{N}}}\frac{1}{c}S_{\chi}(n,m,c)&V(n,m,c,\psi,\vphi) = \sum_{j \ge 1}\conj{a_{j}(n)}a_{j}(m)\conj{\w(n,\psi)}\w(m,\vphi) \\
        &+\sum_{\mf{a}}\frac{1}{4\pi}\int_{-\infty}^{\infty}\conj{\tau_{\mf{a}}\left(n,\frac{1}{2}+ir\right)}\tau_{\mf{a}}\left(m,\frac{1}{2}+ir\right)\conj{\w(n,\psi)}\w(m,\vphi)\,dr.
      \end{align*}
    \end{theorem}
  \section{Misc.}
    In particular, we also have the discriminant
    \[
      D_{\mc{O}_{K}/\Z} = d_{\mc{O}_{K}/\Z}(\a_{1},\ldots,\a_{n})\Z,
    \]
    for any integral basis $\a_{1},\ldots,\a_{n}$ of $K$. Since $(\Z^{\ast})^{2}$ is the trivial group, $d_{\mc{O}_{K}/\Z}(\a_{1},\ldots,\a_{n})$ is a well-defined nonzero integer by \cref{prop:norm_and_trace_AKBL,prop:discriminant_not_zero}. We then define the \textbf{discriminant}\index{discriminant} $\D_{K}$ of $K$ by
    \[
      \D_{K} = d_{\mc{O}_{K}/\Z}(\a_{1},\ldots,\a_{n}),
    \]
    which is well-defined. Moreover, $\D_{K}$ is nonzero by \cref{prop:discriminant_not_zero} and
    \[
      \D_{K} = \det(M(\a_{1},\ldots,\a_{n}))^{2},
    \]
    by \cref{disc_as_square_of_embedding_matrix}.

    We now discuss the factorization of prime integral ideals in extensions of number fields. First, we need to introduce the concept of prime integral ideals above primes. Let $K$ be a number field and let $\mf{p}$ be a prime integral ideal. Then $\mf{p} \cap \Z$ is a prime integral ideal of $\Q$. Indeed, it is clear that $\mf{p} \cap \Z$ is an integral ideal of $\Q$. It is proper because $1 \notin \mf{p} \cap \Z$ as $\mf{p}$ does not contain units. It is nonzero because any integral ideal contains its norm (as we have noted) and hence $\Norm(\mf{p}) \in \mf{p} \cap \Z$. To show that $\mf{p} \cap \Z$ is prime, suppose $a,b \in \Z$ are such that $ab \in \mf{p} \cap \Z$. Then $ab \in \mf{p}$ and since $\mf{p}$ is prime either $a \in \mf{p}$ or $b \in \mf{p}$. But then $a \in \mf{p} \cap \Z$ or $b \in \mf{p} \cap \Z$ as desired. We have now shown that $\mf{p} \cap \Z$ is a prime integral ideal of $\Q$. Hence
    \[
      \mf{p} \cap \Z = p\Z,
    \]
    for some prime integer $p$. Accordingly, we say that $\mf{p}$ is \textbf{above}\index{above} $p$, or equivalently, $p$ is \textbf{below}\index{below} $\mf{p}$. Moreover, if $\mf{p}$ is above $p$, then $\mf{p}$ must be a prime factor of $p\mc{O}_{K}$. Indeed, $p\Z \subseteq \mf{p}$ so that $p\mc{O}_{K} \subseteq \mf{p}$ and then the fact $\mf{p}$ is prime implies that some prime factor of $p\mc{O}_{K}$ is contained in $\mf{p}$. Since prime integral ideals are maximal, this prime factor must be $\mf{p}$ itself. We illustrate these relations by the extension

    \begin{center}
      \begin{tikzcd}
        \mf{p} \subset \mc{O}_{K} \subset K \arrow[dash]{d} \\
        p\Z \subset \Z \subset \Q.
      \end{tikzcd}
    \end{center}

    Since $\mf{p}$ and $p\Z$ are maximal in $\mc{O}_{K}$ and $\Z$ respectively, we have the residue fields $\mc{O}_{K}/\mf{p}$ and $\F_{p}$. It turns out that $\mc{O}_{K}/\mf{p}$ is a finite dimensional vector space over $\F_{p}$. To see this, consider the homomorphism
    \[
      \phi:\Z \to \mc{O}_{K}/\mf{p} \qquad a \mapsto a \tmod{\mf{p}}.
    \]
    Now $\ker\phi = \mf{p} \cap \Z$ and hence $\ker\phi = p\Z$ since $\mf{p}$ is above $p$. By the first isomorphism theorem, $\phi$ induces an injection $\phi:\F_{p} \to \mc{O}_{K}/\mf{p}$ and since $\mc{O}_{K}/\mf{p}$ is field (with $\Norm(\mf{p})$ elements), it must be a finite field containing $\F_{p}$. Necessarily $\mc{O}_{K}/\mf{p}$ is a finite dimensional vector space over $\F_{p}$. Accordingly, we define the \textbf{inertia degree}\index{inertia degree} $f_{p}(\mf{p})$ of $\mf{p}$ by
    \[
      f_{p}(\mf{p}) = [\mc{O}_{K}/\mf{p}:\F_{p}].
    \]
    That is, $f_{p}(\mf{p})$ is the dimension of the residue field $\mc{O}_{K}/\mf{p}$ as a vector space over $\F_{p}$. Then we have
    \[
      \Norm(\mf{p}) = |\mc{O}_{K}/\mf{p}| = |\F_{p}|^{f_{p}(\mf{p})} = p^{f_{p}(\mf{p})}.
    \]
    In particular, the norm of a prime integral ideal is a power of the prime below it. As we have already noted, $\mf{p}$ is a prime factor of $p\mc{O}_{K}$.

    Then it suffices to show $\l_{1},\ldots,\l_{m}$ is a basis for $L/K$ so that $m = n$. We claim $\l_{1},\ldots,\l_{m}$ are linearly independent over $K$. If not, as $K$ is the field of fractions of $\smc{O}$, we may multiply by a nonzero element of $\smc{O}$ to ensure they are linearly independent over $\smc{O}$ as well. Then there are $\a_{i} \in \smc{O}$, for $1 \le i \le m$ and not all zero, such that
    \[
      \a_{1}\l_{1}+\cdots+\a_{m}\l_{m} = 0.
    \]
    Let $\mf{a}$ be the integral ideal of $\smc{O}$ generated by the $\a_{i}$. By uniqueness of the prime factorization of integral ideals, $\mf{a}^{-1}\mf{p} \subset \mf{a}^{-1}$ so that there exists a nonzero $\a \in \mf{a}^{-1}-\mf{a}^{-1}\mf{p}$. Thus $a\mf{a} \not\subseteq \mf{p}$ and so the elements $\a\a_{1},\ldots,\a\a_{m}$ generating $a\mf{a}$ lie in $\smc{O}$ and at least one of them does not lie in $\mf{p}$. Therefore their reductions $\overline{\a\a_{1}},\ldots,\overline{\a\a_{m}}$ modulo $\mf{p}$ are not all zero and so the nontrivial linear dependence above implies a nontrivial linear dependence
    \[
      \overline{\a\a_{1}\l_{1}}+\cdots+\overline{\a\a_{m}\l_{m}} = 0.
    \]
    This contradicts the fact that $\conj{\l_{1}},\ldots,\conj{\l_{m}}$ is a basis. Therefore $\l_{1},\ldots,\l_{m}$ are linearly independent over $K$. To show that they span $L/K$, 

    Since $\mc{O}_{K}$ is a free abelian group of rank $n$ so is any fractional ideal by \cref{thm:integral_basis_AKBL}. Therefore fractional ideals are complete lattices in $K$ as a vector space over $\Q$. In particular, $\mc{O}_{K}$ is a complete lattice in $K$.
  \section{\todo{Lattices}}
    Let $K$ be a number field of degree $n$. By \cref{lem:trace_is_nondegenerate}, there is a nondegenerate symmetric bilinear form on $K$ given by
    \[
      \Trace:K \x K \to \Q \qquad (\k,\l) \mapsto \Trace(\k\l).
    \]
    We call this bilinear form the \textbf{trace form}\index{trace form} on $K$. The trace form makes $K$ into a nondegenerate inner product space over $\Q$. Since $\mc{O}_{K}$ is a free abelian group of rank $n$ so is any fractional ideal by \cref{thm:integral_basis_AKBL}. Therefore fractional ideals are complete lattices in $K$ as a vector space over $\Q$. In particular, $\mc{O}_{K}$ is a complete lattice in $K$. For a fractional ideal $\mf{f}$, note that the dual lattice $\mf{f}^{\vee}$ is
    \[
      \mf{f}^{\vee} = \{\k \in K:\text{$\Trace(\k \a) \in \Z$ for all $\a \in \mf{f}$}\}.
    \]
    We call $\mf{f}^{\vee}$ the \textbf{dual ideal}\index{dual ideal} of $\mf{f}$. The following proposition shows that the dual ideal $\mf{f}^{\vee}$ is indeed a fractional ideal:

    \begin{proposition}\label{prop:dual_lattice_is_fractional}
      Let $K$ be a number field and $\mf{f}$ be a fractional ideal. Then $\mf{f}^{\vee}$ is a fractional ideal and
      \[
        \mf{f}^{\vee} = \mf{f}^{-1}\mc{O}_{K}^{\vee}.
      \]
    \end{proposition}
    \begin{proof}
      By \cref{prop:dual_lattice_exists}, $\mf{f}^{\vee}$ is a finitely generated $\Z$-module. Therefore it is a finitely generated $\mc{O}_{K}$-submodule of $K$ if it is preserved under multiplication by $\mc{O}_{K}$. Let $\a \in \mc{O}_{K}$ and $\b \in \mf{f}^{\vee}$. Then we must show $\a\b \in \mf{f}^{\vee}$. To see this, observe that $\Trace(\a\b\mf{f}) \subseteq \Trace(\b\mf{f}) \subseteq \Z$ since $\a\mf{f} \subseteq \mf{f}$ and $\b \in \mf{f}^{\vee}$. Therefore $\a\b \in \mf{f}^{\vee}$ and hence $\mf{f}^{\vee}$ is a fractional ideal proving the first statement. To prove the second we will show containment in both directions. For the forward containment, first suppose $\a \in \mf{f}^{\vee}$ and $\b \in \mf{f}$. Then $\Trace(\a\b\mc{O}_{K}) \subseteq \Trace(\a\mf{f}) \subseteq \Z$ since $\b\mc{O}_{K} \subseteq \mf{f}$ and $\a \in \mf{f}^{\vee}$. Hence $\a\b \in \mc{O}_{K}^{\vee}$ so that $\mf{f}^{\vee}\mf{f} \subseteq \mc{O}_{K}^{\vee}$ and therefore $\mf{f}^{\vee} \subseteq \mf{f}^{-1}\mc{O}_{K}^{\vee}$. This proves the forward containment. For the reverse containment, suppose $\a \in \mf{f}^{-1}$ and $\b \in \mc{O}_{K}^{\vee}$. Then $\Trace(\a\b\mf{f}) \subseteq \Trace(\b\mc{O}_{K}) \subseteq \Z$ since $\a\mf{f} \subseteq \mc{O}_{K}$ and $\b \in \mc{O}_{K}^{\vee}$. This shows $\a\b \in \mf{f}^{\vee}$ and hence $\mf{f}^{-1}\mc{O}_{K}^{\vee} \subseteq \mf{f}^{\vee}$ proving the reverse containment and completing the proof.
    \end{proof}

    We define the \textbf{different}\index{different} $\mf{D}$ of $K$ by
    \[
      \mf{D}_{K} = (\mc{O}_{K}^{\vee})^{-1}.  
    \]
    This is an integral ideal. Indeed, first note that $\mc{O}_{K} \subseteq \mc{O}_{K}^{\vee}$. It follows from \cref{prop:explicit_inverse_ideal} that $\mc{D}_{K}$ is an integral ideal and
    \[
      \mf{D}_{K} = \{\k \in K: \k\mc{O}_{K}^{\vee} \subseteq \mc{O}_{K}\}.  
    \]
    Also, by \cref{prop:dual_lattice_is_fractional} we can express the dual ideal $\mf{f}^{\vee}$ of a fractional ideal $\mf{f}$ in terms of the different as
    \[
      \mf{f}^{\vee} = \mf{f}^{-1}\mf{D}_{K}^{-1}.
    \]
    It turns out that the norm of the different is the absolute value of the discriminant:

    \begin{proposition}\label{prop:norm_of_different}
      Let $K$ be an algebraic number field of degree $n$. Then we have an isomorphism
      \[
        \mc{O}_{K}/\mf{D}_{K} \cong \mc{O}_{K}^{\vee}/\mc{O}_{K},
      \]
      as $\mc{O}_{K}$-modules. In particular,
      \[
        \Norm(\mf{D}_{K}) = |\D_{K}|.
      \]
    \end{proposition}
    \begin{proof}
      By \cref{prop:explicit_inverse_ideal}, $\mc{O}_{K} \subseteq \mf{D}_{K}^{-1}$. Then the second isomorphism theorem implies
      \[
        \mc{O}_{K}/\mf{D}_{K} \cong \mf{D}^{-1}_{K}/\mc{O}_{K} \cong \mc{O}_{K}^{\vee}/\mc{O}_{K},
      \]
      which proves the first statement. For the second, this isomorphism shows that $\Norm(\mf{D}_{K}) = |\mc{O}_{K}^{\vee}/\mc{O}_{K}|$. Now let $\a_{1},\ldots,\a_{n}$ be an integral basis for $\mc{O}_{K}$. Then $\a_{1}^{\vee},\ldots,\a_{n}^{\vee}$ is a basis for $\mc{O}_{K}^{\vee}$ and by definition of the dual basis we have
      \[
        \a_{i}^{\vee} = \sum_{1 \le j \le n}\Trace(\a_{i}\a_{j})\a_{j}.
      \]
      But then the base change matrix from $\a_{1},\ldots,\a_{n}$ to $\a_{1}^{\vee},\ldots,\a_{n}^{\vee}$ is $(\Trace(\a_{i}\a_{j}))_{i,j}$. The claim follows by \cref{prop:base_change_quotient_determinant} and the definition of $\D_{K}$.
    \end{proof}

    As a corollary, we can compute the norm of dual ideals:

    \begin{corollary}\label{cor:norm_of_different}
      Let $K$ be a number field and $\mf{f}$ be a fractional ideal. Then
      \[
        \Norm(\mf{f}^{\vee}) = \frac{\Norm(\mf{f}^{-1})}{|\D_{K}|}.
      \]
    \end{corollary}
    \begin{proof}
      This follows immediately from the second statement of \cref{prop:norm_of_different}, complete multiplicativity of the norm, and that $\mf{f}^{\vee} = \mf{f}^{-1}\mf{D}_{K}^{-1}$.
    \end{proof}

    We have already remarked that the different is an integral ideal and that $\mc{O}_{K} \subseteq \mc{O}_{K}^{\vee}$. Therefore we have an inclusion of complete lattices
    \[
      \mc{D}_{K} \subseteq \mc{O}_{K} \subseteq \mc{O}_{K}^{\vee}.
    \]
    What \cref{prop:norm_of_different} shows is that each complete lattice in chain has index $|\D_{K}|$ in the next one. In particular, $\mc{O}_{K}^{\vee}$ is strictly larger than $\mc{O}_{K}$ if and only if $|\D_{K}| \ge 2$. So we can think of the different $\mc{D}_{K}$ as a measure of the failure of $\mc{O}_{K}$ to be self-dual since $\Norm(\mc{D}_{K}) = 1$ if and only if $\mc{O}_{K}^{\vee} = \mc{O}_{K}$.
  \section{The Ideal Norm}
    For a number field $K$, we can define a norm for integral ideals of $\mc{O}_{K}$ which will be immensely useful. Since $\mc{O}_{K}$ is a free abelian group of rank $n$ so is any integral ideal $\mf{a}$ by \cref{thm:integral_basis_AKBL}. Therefore $\mc{O}_{K}/\mf{a}$ is finite by \cref{prop:base_change_quotient_determinant}. Accordingly, we define the \textbf{norm}\index{norm} $\Norm(\mf{a})$ of $\mf{a}$ by
    \[
      \Norm(\mf{a}) = |\mc{O}_{K}/\mf{a}|.
    \]
    We also define the norm of the zero ideal to be zero. Moreover, Lagrange's theorem implies that $\Norm(\mf{a}) \in \mf{a}$ for any integral ideal $\mf{a}$. As we might hope, the norms of $\a$ and $\a\mc{O}_{K}$ are essentially the same for any $\a \in \mc{O}_{K}$:
    
    \begin{proposition}\label{prop:ideal_and_field_norms_are_identical}
      Let $K$ be a number field. Then for any $\a \in \mc{O}_{K}$, we have
      \[
        \Norm(\a\mc{O}_{K}) = |\Norm(\a)|.
      \]
    \end{proposition}
    \begin{proof}
      Let $\a_{1},\ldots,\a_{n}$ be an integral basis for $K$. Writing
      \[
        \a = \sum_{1 \le i \le n}a_{i}\a_{i},
      \]
      with $a_{i} \in \Z$, we see that $a_{1}\a_{1},\ldots,a_{n}\a_{n}$ is a basis for $\a\mc{O}_{K}$. In particular, the base change matrix from $\a_{1},\ldots,\a_{n}$  to this basis is a diagonal matrix with the $a_{i}$ on the diagonal. Then on the one hand, we have $\Norm(\a\mc{O}_{K}) = |a_{1} \cdots a_{n}|$ by \cref{prop:base_change_quotient_determinant}. On the other hand, in terms of the basis $a_{1}\a_{1},\ldots,a_{n}\a_{n}$ the map $T_{\a}$ is given by
      \[
        T_{\a} = \begin{pmatrix} a_{1} & & \\ & \ddots & \\ & & a_{n} \end{pmatrix},
      \]
      and so $\Norm(\a) = a_{1} \cdots a_{n}$. Hence
      \[
        \Norm(\a\mc{O}_{K}) = |\Norm(\a)|,
      \]
      as desired.
    \end{proof}

    The norm of an integral ideal is also completely multiplicative:

    \begin{proposition}\label{prop:ideal_norm_is_multiplicative}
      Let $K$ be a number field and let $\mf{a}$ and $\mf{b}$ be integral ideals. Then
      \[
        \Norm(\mf{a}\mf{b}) = \Norm(\mf{a})\Norm(\mf{b}).
      \]
    \end{proposition}
    \begin{proof}
      First suppose $\mf{a}$ and $\mf{b}$ are relatively prime. Then the Chinese remainder theorem implies
      \[
        \mc{O}_{K}/\mf{a}\mf{b} \cong \mc{O}_{K}/\mf{a} \op \mc{O}_{K}/\mf{b},
      \]
      and hence $|\mc{O}_{K}/\mf{a}\mf{b}| = |\mc{O}_{K}/\mf{a}||\mc{O}_{K}/\mf{b}|$ so that $\Norm(\mf{a}\mf{b}) = \Norm(\mf{a})\Norm(\mf{b})$. As distinct prime integral ideals are relatively prime (because they are maximal), it suffices to show $\Norm(\mf{p}^{n}) = \Norm(\mf{p})^{n}$ for all prime integral ideals $\mf{p}$ and $n \ge 0$. We will prove this by induction. The base case is clear so assume that the claim holds for $n-1$. By the third isomorphism theorem, we have
      \[
        \mc{O}_{K}/\mf{p}^{n-1} \cong (\mc{O}_{K}/\mf{p}^{n})/(\mf{p}^{n-1}/\mf{p}^{n}).
      \]
      Using \cref{prop:isomorphism_of_quotient_by_prime_integral_ideals}, it follows that
      \[
        |\mc{O}_{K}/\mf{p}^{n-1}| = \frac{|\mc{O}_{K}/\mf{p}^{n}|}{|\mf{p}^{n-1}/\mf{p}^{n}|} = \frac{|\mc{O}_{K}/\mf{p}^{n}|}{|\mc{O}_{K}/\mf{p}|}.
      \]
      Thus $\Norm(\mf{p}^{n}) = \Norm(\mf{p}^{n-1})\Norm(\mf{p})$ and our induction hypothesis implies $\Norm(\mf{p}^{n}) = \Norm(\mf{p})^{n}$ as desired.
    \end{proof}

    At last we can define the norm to fractional ideals. Let $\mf{f}$ be a fractional ideal. By \cref{cor:fractional_ideal_prime_factorization}, there exist unique integral ideals $\mf{a}$ and $\mf{b}$ such that $\mf{f} = \mf{a}\mf{b}^{-1}$. We define the \textbf{norm}\index{norm} $\Norm(\mf{f})$ of $\mf{f}$ by
    \[
      \Norm(\mf{f}) = \frac{\Norm(\mf{a})}{\Norm(\mf{b})}.
    \]
    From this definition and \cref{prop:ideal_norm_is_multiplicative} it follows that the norm of a fractional ideal is completely multiplicative. Then upon writing $\mf{f}\mf{b} = \mf{a}$, we have $\Norm(\mf{f})\Norm(\mf{b}) \in \mf{a}$ and hence $\Norm(\mf{f}) \in \mf{f}$ after multiplying by $\frac{1}{\Norm(\mf{b})}$ because $\frac{1}{\Norm(\mf{b})}\mf{a} \subseteq \mf{f}\mc{O} \subseteq \mf{f}$. That is, every fractional ideal contains its norm. We have now established a homomorphism
    \[
      \Norm:I_{K} \to \Q^{\ast} \qquad \mf{f} \mapsto \Norm(\mf{f}),
    \]
    which we call the \textbf{ideal norm}\index{ideal norm} of $K$.
  
\end{document}
