\chapter{Types of Sieves}
  \section{\todo{Brun's Sieve}}
    The modern development of sieves began with Brun (see \cite{brun1915uber}) in 1915. His idea was to apply a truncation of the inclusion-exclusion principle to the sifting function. To motivate this idea, recall that the sifting function $S(x,z;\mc{A})$ is obtained from $S(x;\mc{A})$ by removing the terms $a_{n}$ for which a sifting prime $p$ divides $n$ (maybe more than one). Removing those terms $a_{n}$ according to the smallest sifting prime $p$ that divides $n$, we immediately arrive at \textbf{Buchstab's identity}\index{Buchstab's identity}:
    \[
      S(x,z;\mc{A}) = S(x;\mc{A})-\sum_{p \mid P(z)}S(x,p;\mc{A}_{p}).
    \]
    Since $S(x,2;\mc{A}) = S(x;\mc{A})$, we may apply this formula iteratively $r$ times to obtain
    \begin{equation}\label{equ:iterative_Buchstab}
      S(x,z;\mc{A}) = \sum_{\substack{d \mid P(z) \\ \w(d) < r}}\mu(d)S(x;\mc{A}_{d})+(-1)^{r}\sum_{\substack{d \mid P(z) \\ \w(d) = r}}S(x,p(d);\mc{A}_{d}),
    \end{equation}
    where $p(d)$ is the smallest prime divisor of $d$. This identity can be thought of as an inclusion-exclusion principle for the sifting function. As $S(x,p(d);\mc{A}_{d})$ is nonnegative, \cref{equ:iterative_Buchstab} produces the upper bound
    \begin{equation}\label{equ:iterative_Buchstab_odd}
      S(x,z;\mc{A}) \le \sum_{\substack{d \mid P(z) \\ \w(d) < r}}\mu(d)S(x;\mc{A}_{d}),
    \end{equation}
    if $r$ is odd, and the lower bound
    \begin{equation}\label{equ:iterative_Buchstab_even}
      S(x,z;\mc{A}) \ge \sum_{\substack{d \mid P(z) \\ \w(d) < r}}\mu(d)S(x;\mc{A}_{d}),
    \end{equation}
    is $r$ is even. Thus \cref{equ:iterative_Buchstab} gives upper and lower bounds for the sifting function depending on the parity of $r$. Accordingly, for a fixed $r \ge 1$, \textbf{Brun's sieve}\index{Brun's sieve} $\L = (\l_{d})_{d \ge 1}$ is given by
    \[
      \l_{d} = \begin{cases} \mu(d) & \text{if $\w(d) < r$ and $d \mid P(z)$}, \\ 0 & \text{if $\w(d) \ge r$ or $d \nmid P(z)$}. \end{cases}
    \]
    Clearly the sieve level is $D = P(z)$. Moreover, \cref{equ:iterative_Buchstab_odd,equ:iterative_Buchstab_even} together imply
    \[
      S(x,z;\mc{A}) \le S^{\L}(x,z;\mc{A}) \quad \text{or} \quad S(x,z;\mc{A}) \ge S^{\L}(x,z;\mc{A}),
    \]
    according to if $r$ is odd or even. So Brun's sieve is an upper sieve if $r$ is odd and a lower sieve if $r$ is even. If we assume that there exists a density function $g$ for the sums $S(x;\mc{A}_{d})$, then
    \[
      V^{\L}(z) = \sum_{\substack{d \mid P(z) \\ \w(d) < r}}\mu(d)g(d) \quad \text{and} \quad R^{\L}(x,z;\mc{A}) = \sum_{\substack{d \mid P(z) \\ \w(d) < r}}\mu(d)r_{d}(x;\mc{A}).
    \]
    To see the relationship between $V(z)$ and $V^{\L}(z)$, we will derive an analog of Buchstab's identity. Expanding \cref{equ:V_product_formula} and grouping terms according to the largest sifting prime $p$ such that $g(p)$ divides them, we arrive at the formula
    \[
      V(z) = 1-\sum_{p \mid P(z)}g(p)V(p),
    \]
    which is clearly an analog to Buchstab's identity. Since $V(2) = 1$, we may apply this formula iteratively $r$ times and obtain
    \[
      V(z) = \sum_{\substack{d \mid P(z) \\ \w(d) < r}}\mu(d)g(d)+(-1)^{r}\sum_{\substack{d \mid P(z) \\ \w(d) = r}}g(d)V(p(d)).
    \]
    Note that the first sum on the right-hand side is exactly $V^{\L}(z)$ and so the difference between $V^{\L}(z)$ and $V(z)$ is exactly the second sum which alternates in sign according to the parity of $r$. We now aim to estimate $G^{\L}(z)$. To this end, let
    \[
      G(z) = \sum_{p \mid P(z)}g(p).
    \]
    Then $G(z) = -G^{\L}(z)$ when $r = 1$. Moreover, the Taylor series of the logarithm gives
    \[
      G(z) \le \sum_{p \mid P(z)}-\log(1-g(p)) = -\log V(z).
    \]
    Upon expanding the product $G(z)^{r}$, it is also clear that $G^{\L}(z) \le \frac{G(z)^{r}}{r!}$ for any $r \ge 1$. These facts with the inequality $r! \ge e\left(\frac{r}{e}\right)^{r}$ (which easily follows by induction), and the identity $-\log V(z) = |\log V(z)|$, together imply
    \begin{equation}\label{equ:Bruns_sieve_G_bound_1}
      G^{\L}(z) \le \frac{G(z)^{r}}{r!} \le \frac{1}{e}\left(\frac{eG(z)}{r}\right)^{r} \le \frac{1}{e}\left(\frac{e|\log V(z)|}{r}\right)^{r}.
    \end{equation}
    We now choose $r$ in order to make this last expression small. Clearly we at least need $r \ge c|\log V(z)|$, for some $c > e$, so that $\frac{e|\log V(z)|}{r} < 1$. To this end, let $c$ be the unique constant satisfying
    \[
      \left(\frac{c}{e}\right)^{c} = e.
    \]
    Note that $c > e$ because $e > 1$. Moreover, for any $b \ge c$ we have the inequality $b^{b} \ge e^{2b-c+1}$. Indeed, taking the logarithm and isolating $b$ gives $b(\log(b)-2) \ge 1-c$ which is equality when $b = c$ and the left-hand side is an increasing function of $b$. Letting $r \ge c|\log V(z)|$ and $b = \frac{r}{|\log V(z)|}$, the aforementioned inequality is equivalent to
    \begin{equation}\label{equ:Bruns_sieve_G_bound_2}
      \left(\frac{|\log V(z)|}{r}\right)^{r} \ge e^{-2r}V(z)^{1-c}.
    \end{equation}
    Combining \cref{equ:Bruns_sieve_G_bound_1,equ:Bruns_sieve_G_bound_2} results in the bound
    \[
      G^{\L}(z) \le e^{-(r+1)}V(z)^{1-c}.
    \]
    We choose $r$ to depend on the sifting variable $s$ by setting $r = \lfloor s \rfloor$ provided $s \ge 1-c\log V(z)$ (recall $-\log V(z) = |\log V(z)|$ so that $s \ge c|\log V(z)|$). Then our bound for $G^{\L}(z)$ becomes
    \[
      G^{\L}(z) \le e^{-s}V(z)^{1-c}.
    \]
    We now establish a relationship between the sifting function $S(x,z;\mc{A})$ and the sieving function $S^{\L}(x,z;\mc{A})$ that will allow us to make use of these estimates. \todo{xxx}
  \section{\todo{Beta Sieves}}
    We now wish to general Brun's sieve by replacing the associated sieving function with one that better approximates the sifting function. Precisely, we will replace the condition $\w(d) < r$ and $p \mid P(z)$ with the condition $d \in \mc{D}$ for set of square-free positive integers $\mc{D}$ with a small amount of small prime divisors coming from the sifting range $\mc{P}$. We will be concerned with upper and lower sieves $\L^{+} = (\l_{d}^{+})_{d \ge 1}$ and $\L^{-} = (\l_{d}^{+})_{d \ge 1}$ where $\l_{d}^{\pm} = \mu(d)$ only for $d$ belonging to the sets
    \[
      \mc{D}^{+} = \{d = p_{1}p_{2} \cdots p_{r} < D:\text{$p_{1},\ldots,p_{r} \mid P(z)$, $p_{r} < p_{r-1} < \cdots < p_{1}$, and $p_{m} < y_{m}$ for $m$ odd}\},
    \]
    and
    \[
      \mc{D}^{-} = \{d = p_{1}p_{2} \cdots p_{r} < D:\text{$p_{1},\ldots,p_{r} \mid P(z)$, $p_{r} < p_{r-1} < \cdots < p_{1}$, and $p_{m} < y_{m}$ for $m$ even}\},
    \]
    respectively, with $1 \in \mc{D}^{\pm}$, and for integer $D \ge 1$ and parameters $y_{m} > 0$ for all $m \ge 1$. The parameters $y_{m}$ are chosen to optimize the particular sifting sequence $\mc{A}$. For $\mc{D}^{\pm}$, the \textbf{beta sieve}\index{beta sieve} $\L^{\pm} = (\l_{d}^{\pm})_{d \ge 1}$ is defined by
    \[
      \l_{d}^{\pm} = \begin{cases} \mu(d) & \text{if $d \in \mc{D}^{\pm}$}, \\ 0 & \text{if $d \notin \mc{D}^{\pm}$}. \end{cases}
    \]
    Clearly the sieve level is $D$. \todo{xxx}
    
    \iffalse For a fixed $z$, index the sifting primes diving $P(z)$ as $q_{1} > q_{2} > \cdots > q_{m}$. Then we define the parameters $y_{m}$ by
    \[
      y_{m} = \left(\frac{D}{q_{1}q_{2} \cdots q_{m}}\right)^{\frac{1}{\b}},
    \]
    for some $\b \ge 1$. Thus if $d = p_{1}p_{2} \cdots p_{r} \in \mc{D}^{\pm}$ we have that $r \le m$. Moreover, $q_{1}\cdots q_{m}p_{m}^{\b} < D$ for $m$ odd or $m$ even depending on if $d \in \mc{D}^{+}$ or $d \in \mc{D}^{-}$ respectively.
    \fi
  \section{\todo{\texorpdfstring{$\L^{2}$}{L2} Sieves}}
  \section{\todo{The Large Sieve}}