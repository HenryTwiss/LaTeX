\documentclass[12pt,reqno,oneside]{amsart}
\usepackage{import}
\usepackage{hyperref}
%===============================%
%  Packages and basic settings  %
%===============================%
\usepackage[headheight=15pt,rmargin=0.5in,lmargin=0.5in,tmargin=0.75in,bmargin=0.75in]{geometry}
\usepackage{fancyhdr}
\usepackage{imakeidx}
\usepackage{framed}
\usepackage{amssymb}
\usepackage{amsmath}
\usepackage{mathrsfs}
\usepackage{enumitem}
\usepackage{multirow}
\usepackage{hyperref}
\usepackage[capitalise,noabbrev]{cleveref}
\usepackage{appendix}
\usepackage[hyperref,amsthm,amsmath,thref,framed,thmmarks]{ntheorem}
\usepackage{tikz}
\usepackage{tikz-cd}
\usepackage{nomencl}\makenomenclature
\usetikzlibrary{braids,arrows,decorations.markings,calc}

%=======================%
%  Book style settings  %
%=======================%
\pagestyle{fancy}
\fancyhf{}
\fancyhead[L]{\nouppercase{\leftmark}}
\fancyfoot[C]{\thepage}
\setlength\parindent{0pt}
\raggedbottom

%====================================%
%  Theorems, environments & cleveref  %
%====================================%
\theoremstyle{plain}\newtheorem{theorem}{Theorem}[section]
\theoremstyle{nonumberplain}\renewtheorem{theorem*}{Theorem}
\theoremstyle{plain}\newtheorem{proposition}[theorem]{Proposition}
\theoremstyle{nonumberplain}\renewtheorem{proposition*}{Proposition}
\theoremstyle{plain}\newtheorem{corollary}[theorem]{Corollary}
\theoremstyle{nonumberplain}\renewtheorem{corollary*}{Corollary}
\theoremstyle{plain}\newtheorem{lemma}[theorem]{Lemma}
\theoremstyle{nonumberplain}\renewtheorem{lemma*}{Lemma}
\theoremstyle{plain}\newtheorem{conjecture}[theorem]{Conjecture}
\theoremstyle{nonumberplain}\renewtheorem{conjecture*}{Conjecture}
\theoremstyle{plain}\newtheorem{remark}[theorem]{Remark}
\theoremstyle{nonumberplain}\renewtheorem{remark*}{Remark}
\theoremstyle{plain}\newtheorem{problem}[theorem]{Open Problem}
\theoremstyle{nonumberplain}\renewtheorem{problem*}{Open Problem}
\theoremstyle{plain}\newtheorem{heuristic}[theorem]{Heuristic}
\theoremstyle{nonumberplain}\renewtheorem{heuristic*}{Heuristic}
\crefname{conjecture}{Conjecture}{Conjectures}

\newenvironment{stabular}[2][1]
  {\def\arraystretch{#1}\tabular{#2}}
  {\endtabular}

%==================================%
%  Custom commands & environments  %
%==================================%
\newcommand{\legendre}[2]{\left(\frac{#1}{#2}\right)}
\newcommand{\dlegendre}[2]{\displaystyle{\left(\frac{#1}{#2}\right)}}
\newcommand{\tlegendre}[2]{\textstyle{\left(\frac{#1}{#2}\right)}}
\newcommand{\psum}{\sideset{}{'}\sum}
\newcommand{\asum}{\sideset{}{^{\ast}}\sum}
\newcommand{\tmod}[1]{\ (\mathrm{mod}\text{ }#1)}
\renewcommand{\bmod}[1]{\ \left(\mathrm{mod}\text{ }#1\right)}
\newcommand{\xto}[1]{\xrightarrow{#1}}
\newcommand{\xfrom}[1]{\xleftarrow{#1}}
\newcommand{\normal}{\mathrel{\unlhd}}
\newcommand{\mf}{\mathfrak}
\newcommand{\mc}{\mathcal}
\newcommand{\ms}{\mathscr}

\newcommand{\Mat}{\mathrm{Mat}}
\newcommand{\GL}{\mathrm{GL}}
\newcommand{\SL}{\mathrm{SL}}
\newcommand{\PSL}{\mathrm{PSL}}
\renewcommand{\O}{\mathrm{O}}
\newcommand{\SO}{\mathrm{SO}}
\newcommand{\U}{\mathrm{U}}
\newcommand{\Sp}{\mathrm{Sp}}

\newcommand{\N}{\mathbb{N}}
\newcommand{\Z}{\mathbb{Z}}
\newcommand{\Q}{\mathbb{Q}}
\newcommand{\R}{\mathbb{R}}
\newcommand{\C}{\mathbb{C}}
\newcommand{\F}{\mathbb{F}}
\renewcommand{\H}{\mathbb{H}}
\renewcommand{\P}{\mathbb{P}}

\renewcommand{\a}{\alpha}
\renewcommand{\b}{\beta}
\newcommand{\g}{\gamma}
\renewcommand{\d}{\delta}
\newcommand{\z}{\zeta}
\renewcommand{\t}{\theta}
\renewcommand{\i}{\iota}
\renewcommand{\k}{\kappa}
\renewcommand{\l}{\lambda}
\newcommand{\s}{\sigma}
\newcommand{\w}{\omega}

\newcommand{\G}{\Gamma}
\newcommand{\D}{\Delta}
\renewcommand{\L}{\Lambda}
\newcommand{\W}{\Omega}
\newcommand{\scL}{\mathscr{L}}

\newcommand{\e}{\varepsilon}
\newcommand{\vt}{\vartheta}
\newcommand{\vphi}{\varphi}
\newcommand{\emt}{\varnothing}

\newcommand{\x}{\times}
\newcommand{\ox}{\otimes}
\newcommand{\op}{\oplus}
\newcommand{\bigox}{\bigotimes}
\newcommand{\bigop}{\bigoplus}
\newcommand{\del}{\partial}
\newcommand{\<}{\langle}
\renewcommand{\>}{\rangle}
\newcommand{\lf}{\lfloor}
\newcommand{\rf}{\rfloor}
\newcommand{\wtilde}{\widetilde}
\newcommand{\what}{\widehat}
\newcommand{\conj}{\overline}
\newcommand{\cchi}{\conj{\chi}}

\DeclareMathOperator{\id}{\textrm{id}}
\DeclareMathOperator{\sgn}{\mathrm{sgn}}
\DeclareMathOperator{\im}{\mathrm{im}}
\DeclareMathOperator{\rk}{\mathrm{rk}}
\DeclareMathOperator{\adj}{\mathrm{adj}}
\DeclareMathOperator{\tr}{\mathrm{trace}}
\DeclareMathOperator{\nm}{\mathrm{norm}}
\DeclareMathOperator{\disc}{\mathrm{disc}}
\DeclareMathOperator{\ord}{\mathrm{ord}}
\DeclareMathOperator{\sym}{\mathrm{sym}}
\DeclareMathOperator{\ext}{\mathrm{ext}}
\DeclareMathOperator{\Hom}{\mathrm{Hom}}
\DeclareMathOperator{\End}{\mathrm{End}}
\DeclareMathOperator{\Aut}{\mathrm{Aut}}
\DeclareMathOperator{\Tor}{\mathrm{Tor}}
\DeclareMathOperator{\Ann}{\mathrm{Ann}}
\DeclareMathOperator{\Gal}{\mathrm{Gal}}
\DeclareMathOperator{\Trace}{\mathrm{Tr}}
\DeclareMathOperator{\Norm}{\mathrm{N}}
\DeclareMathOperator{\Cl}{\mathrm{Cl}}
\DeclareMathOperator{\Span}{\mathrm{Span}}
\DeclareMathOperator*{\Res}{\mathrm{Res}}
\DeclareMathOperator{\Vol}{\mathrm{Vol}}
\DeclareMathOperator{\Li}{\mathrm{Li}}
\DeclareMathOperator{\Supp}{\mathrm{Supp}}
\renewcommand{\Re}{\mathrm{Re}}
\renewcommand{\Im}{\mathrm{Im}}
\DeclareMathOperator{\Ph}{\mathrm{Ph}}
\DeclareMathOperator{\SC}{\mathrm{SC}}


\newcommand{\GH}{\G\backslash\H}
\newcommand{\GG}{\G_{\infty}\backslash\G}

\newenvironment{psmallmatrix}
  {\left(\begin{smallmatrix}}
  {\end{smallmatrix}\right)}

\newcommand{\smc}[1]{
    \mathchoice
    {{\scriptstyle\mathcal{#1}}}
    {{\scriptstyle\mathcal{#1}}}
    {{\scriptscriptstyle\mathcal{#1}}}
    {\scalebox{0.7}{$\scriptscriptstyle\mathcal{#1}$}}
}

%============%
%  Comments  %
%============%
\newcommand{\todo}[1]{\textcolor{red}{\sf Todo: [#1]}}

%===================%
%  Label reminders  %
%===================%
% [label=(\roman*)]
% [label=(\alph*)]
% [label=(\arabic{enumi})]

%==================%
%  Other settings  %
%==================%
\pgfdeclarelayer{background}
\pgfsetlayers{background,main}
\tikzset{->-/.style={decoration={
  markings,
  mark=at position .5 with {\arrow{>}}},postaction={decorate}}}

%=================%
%  Title & Index  %
%=================%
\title{Research Statement}
\author{Henry Twiss}
\date{\today}
\makeindex

\begin{document}

\maketitle 
I am a third year PhD candidate at Brown university under the advisement of Jeff Hoffstein. My academic background is in pure mathematics with a specialization in analytic number theory and a focus in Weyl group multiple Dirichlet series, modular and automorphic forms, and Siegel zeros. I also have some more applied interests in the rationality community and AI safety. Some recent projects I have completed and or are in preparation:

\textbf{Lower order terms for moments of quadratic Dirichlet L-functions:} In 2020, in conjunction with collaborator Adrian Diaconu, I exhibited secondary lower order terms in the asymptotic formula for the moments of quadratic Dirichlet L-functions over function fields under some mild hypotheses regarding the meromorphic continuation of certain Weyl group multiple Dirichlet series. We were also able to explicitly compute the coefficients occurring in our asymptotic formula. This work was published in JNT Prime in 2023 (see \cite{DT}).

\textbf{Converse theorems in half-integral weight:}
In my 2023 preprint \cite{CT}, in conjunction with collaborator Steven Creech, I was able to derive a converse theorem for L-functions attached to half-integral weight modular forms as a consequence of the half-integral weight Petersson trace formula. This work was an extension of a similar result due to Booker, Farmer, and Lee's result in \cite{BFL} to the half-integral weight setting. Our work is currently available on ArXiv until it is accepted for publication.

\textbf{Explicit zero-repulsion \& Linnik’s theorem:}
In the summer of 2023, I attended the PIMS CRG: Inclusive Paths in Explicit Number Theory summer school (see \href{https://sites.google.com/view/crgl-functions/summer-school-inclusive-paths-in-explicit-number-theory}{here} for details). There I developed an explicit version of zero-repulsion for Dirichlet L-functions using ideas of Heath-Brown, mollifiers, and the large sieve. This work is in collaboration with Kubra Benli, Shivani Goel, and Asif Zaman. Our result holds uniformly in the entire critical strip and for any size of conductor. It is currently the best uniformly result available.

We expect to submit for publication by the end of November 2023. We also have a follow-up project in the works with a few other members from the PIMS CRG to prove an explicit version of Linnik’s theorem. This follow-up work is expected to be complete in 2024.

\textbf{Shifted convolution sums of holomorphic cuspforms:}
Currently, I have been interested in shifted convolution sums of Fourier coefficients of holomorphic cuspforms. It is conjectured that such sums should exhibit square root cancellation for both positive and negative shifts. When the shift parameter is positive, only cube root saving has been obtained in the smooth setting. This was achieved in \cite{HH}. Using the analytic continuation of a certain shifted Dirichlet series introduced in \cite{HH}, I have been able to remove the smoothing condition and still obtain cube root cancellation. Essentialy, the shifted Dirichlet series admits a spectral expansion in terms of a basis of Maass cusp forms and one can use this spectral expansion along with standard complex integration techqunics and a truncated Perron's formua to obtain cube root cancellation of the shifted convolution sums. The major details of this work have been completed and a preprint is expected in early 2024.

\textbf{Special values of Rankin-Selberg convolutions in half integral weight:}
Another current project is regarding uniqueness theorems for modular forms. Multiplicity one theorems are standard ``uniqueness`` theorem results for modular forms. Such theorems are arithmetic in nature because they assume properties about the Fourier coefficients of the underlying modular forms. One would like to obtain uniqueness theorems for modular forms that have ``analytic flavor`` by assuming properties of the attached modular L-functions. I have been able to obtain a uniqueness result for half-integral weight modular forms that depends upon the Rankin-Selberg convolutions of such forms at the special value $s = \frac{1}{2}$. This result generalizes one of Luo that can be found in \cite{L}. Luo's result is obtained as an application of the Petersson trace formula and stnadard complex integrals to reduce his problem down to a multiplicity one theorem. In the half-integral weight setting, the Petersson trace formula that I devloped in \cite{CT} can be applied and the same argument will follow modulo some additional complications arising from gamma factors. A preprint is expected to appear in 2024.

\textbf{A Nonabelian Delta Method:}
In December of 2024 I will be visiting Duke University along with four other PhD candidates and postdocs to finish working out an nonabelian $\d$-symbol method that was outlined by J. R. Getz (of Duke) at the HCM summer school New Avenues in the Circle Method (see \href{https://www.hcm.uni-bonn.de/thecirclemethod2021/}{here} for more details). We will meet again in March of 2024 and a preprint is expected around that time. Essentially, a $\d$-symbol method is a way of obtaining an anaytic description of the Dirac delta function $\d$ and such meothds have applications to the Circle Method devloped from ideas of Ramanujan, Hardy, and Littlewood. We will be extending this method to nonabelian rings. 

\textbf{Miscellaneous:}
As for other miscellaneous mathematical projects, I am currently building a repository of notes for analytic number theory that can be found on GitHub through my \href{https://www.henrytwiss.com/writing-projects}{website}. This is purely to make the general methods of analytic number theory more available to a wider audience for free. On that page I also upload some more expository based work such as a fully detailed proof of the modularity of the quadratic theta function via Poisson summation (as far as I am aware, the only completely detailed proof is due to Hecke and is written in German). I am also currently working on fleshing out some ways to apply \href{https://www.lesswrong.com/tag/gears-level}{gears level models} to mathematical proof building. More specifically, I am interested in finding new ways of looking at mathematical machinery that help mathematicians discover the outlines for how to prove conjectures. Some essays on this work, such as how a gears level approach was used to figure out how to prove the explicit version of zero-repulsion for Dirichlet L-functions result mentioned above, are expected in early 2024.

\begin{thebibliography}{99}
  \bibitem{DT}
  Diaconu, Adrian, and Henry Twiss. "Secondary terms in the asymptotics of moments of L-functions." Journal of Number Theory 252 (2023): 243-297.

  \bibitem{BFL}
  Booker, Andrew R., Michael Farmer, and Min Lee. "An extension of Venkatesh’s converse theorem to the Selberg class." Forum of Mathematics, Sigma. Vol. 11. Cambridge University Press, 2023.

  \bibitem{CT}
  Creech, Steven, and Henry Twiss. "A converse theorem in half-integral weight." arXiv preprint arXiv:2306.02872 (2023).

  \bibitem{HH}
  Hoffstein, J., Hulse, T. A., \& Reznikov, A. (2016). Multiple Dirichlet series and shifted convolutions. Journal of Number Theory, 161, 457-533.

  \bibitem{L}
  Luo, Wenzhi. "Special L-values of Rankin-Selberg convolutions." Mathematische Annalen 314 (1999): 591-600.


  \bibitem{HB}
  Heath-Brown, D. R. (1979). The fourth power moment of the Riemann zeta function. Proceedings of the London Mathematical Society, 3(3), 385-422.

  \bibitem{JM}
  Deshouillers, J. M., \& Iwaniec, H. (1982). An additive divisor problem. Journal of the London Mathematical Society, 2(1), 1-14.

  \bibitem{NPR}
  Nordentoft, A. C., Petridis, Y. N., \& Risager, M. S. (2022). Bounds on shifted convolution sums for Hecke eigenforms. Research in Number Theory, 8(2), 26.
\end{thebibliography}

\end{document}