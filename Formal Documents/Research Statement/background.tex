\documentclass[12pt,reqno,oneside]{amsart}
\usepackage{import}
\usepackage{hyperref}
%===============================%
%  Packages and basic settings  %
%===============================%
\usepackage[headheight=15pt,rmargin=0.5in,lmargin=0.5in,tmargin=0.75in,bmargin=0.75in]{geometry}
\usepackage{imakeidx}
\usepackage{framed}
\usepackage{amssymb}
\usepackage{amsmath}
\usepackage{mathrsfs}
\usepackage{enumitem}
\usepackage{hyperref}
\usepackage{appendix}
\usepackage[capitalise,noabbrev]{cleveref}
\usepackage{tikz}
\usepackage{tikz-cd}
\usepackage{nomencl}\makenomenclature
\usetikzlibrary{braids,arrows,decorations.markings,calc}

%====================================%
%  Theorems, environments & cleveref  %
%====================================%
\newtheorem{theorem}{Theorem}[section]
\newtheorem{proposition}{Proposition}[section]
\newtheorem{corollary}{Corollary}[section]
\newtheorem{lemma}{Lemma}[section]
\newtheorem{conjecture}{Conjecture}[section]
\newtheorem{remark}{Remark}[section]

\newenvironment{stabular}[2][1]
  {\def\arraystretch{#1}\tabular{#2}}
  {\endtabular}

%==================================%
%  Custom commands & environments  %
%==================================%
\newcommand{\legendre}[2]{\left(\frac{#1}{#2}\right)}
\newcommand{\dlegendre}[2]{\displaystyle{\left(\frac{#1}{#2}\right)}}
\newcommand{\tlegendre}[2]{\textstyle{\left(\frac{#1}{#2}\right)}}
\newcommand{\psum}{\sideset{}{'}\sum}
\newcommand{\asum}{\sideset{}{^{\ast}}\sum}
\newcommand{\tmod}[1]{\ \left(\text{mod }#1\right)}
\newcommand{\xto}[1]{\xrightarrow{#1}}
\newcommand{\xfrom}[1]{\xleftarrow{#1}}
\newcommand{\normal}{\mathrel{\unlhd}}
\newcommand{\mf}{\mathfrak}
\newcommand{\mc}{\mathcal}
\newcommand{\ms}{\mathscr}

\newcommand{\Mat}{\mathrm{Mat}}
\newcommand{\GL}{\mathrm{GL}}
\newcommand{\SL}{\mathrm{SL}}
\newcommand{\PSL}{\mathrm{PSL}}
\renewcommand{\O}{\mathrm{O}}
\newcommand{\SO}{\mathrm{SO}}
\newcommand{\U}{\mathrm{U}}
\newcommand{\Sp}{\mathrm{Sp}}

\newcommand{\N}{\mathbb{N}}
\newcommand{\Z}{\mathbb{Z}}
\newcommand{\Q}{\mathbb{Q}}
\newcommand{\R}{\mathbb{R}}
\newcommand{\C}{\mathbb{C}}
\newcommand{\F}{\mathbb{F}}
\renewcommand{\H}{\mathbb{H}}
\renewcommand{\P}{\mathbb{P}}

\renewcommand{\a}{\alpha}
\renewcommand{\b}{\beta}
\newcommand{\g}{\gamma}
\renewcommand{\d}{\delta}
\newcommand{\z}{\zeta}
\renewcommand{\t}{\theta}
\renewcommand{\i}{\iota}
\renewcommand{\k}{\kappa}
\renewcommand{\l}{\lambda}
\newcommand{\s}{\sigma}
\newcommand{\w}{\omega}

\newcommand{\G}{\Gamma}
\newcommand{\D}{\Delta}
\renewcommand{\L}{\Lambda}
\newcommand{\W}{\Omega}

\newcommand{\e}{\varepsilon}
\newcommand{\vt}{\vartheta}
\newcommand{\vphi}{\varphi}
\newcommand{\emt}{\varnothing}

\newcommand{\x}{\times}
\newcommand{\ox}{\otimes}
\newcommand{\op}{\oplus}
\newcommand{\bigox}{\bigotimes}
\newcommand{\bigop}{\bigoplus}
\newcommand{\del}{\partial}
\newcommand{\<}{\langle}
\renewcommand{\>}{\rangle}
\newcommand{\lf}{\lfloor}
\newcommand{\rf}{\rfloor}
\newcommand{\wtilde}{\widetilde}
\newcommand{\what}{\widehat}
\newcommand{\conj}{\overline}
\newcommand{\cchi}{\conj{\chi}}

\DeclareMathOperator{\id}{\textrm{id}}
\DeclareMathOperator{\sgn}{\mathrm{sgn}}
\DeclareMathOperator{\im}{\mathrm{im}}
\DeclareMathOperator{\rk}{\mathrm{rk}}
\DeclareMathOperator{\tr}{\mathrm{trace}}
\DeclareMathOperator{\nm}{\mathrm{norm}}
\DeclareMathOperator{\ord}{\mathrm{ord}}
\DeclareMathOperator{\Hom}{\mathrm{Hom}}
\DeclareMathOperator{\End}{\mathrm{End}}
\DeclareMathOperator{\Aut}{\mathrm{Aut}}
\DeclareMathOperator{\Tor}{\mathrm{Tor}}
\DeclareMathOperator{\Ann}{\mathrm{Ann}}
\DeclareMathOperator{\Gal}{\mathrm{Gal}}
\DeclareMathOperator{\Trace}{\mathrm{Trace}}
\DeclareMathOperator{\Norm}{\mathrm{Norm}}
\DeclareMathOperator{\Span}{\mathrm{Span}}
\DeclareMathOperator*{\Res}{\mathrm{Res}}
\DeclareMathOperator{\Vol}{\mathrm{Vol}}
\DeclareMathOperator{\Li}{\mathrm{Li}}
\renewcommand{\Re}{\mathrm{Re}}
\renewcommand{\Im}{\mathrm{Im}}

\newcommand{\GH}{\G\backslash\H}
\newcommand{\GG}{\G_{\infty}\backslash\G}

\newenvironment{psmallmatrix}
  {\left(\begin{smallmatrix}}
  {\end{smallmatrix}\right)}

%============%
%  Comments  %
%============%
\newcommand{\todo}[1]{\textcolor{red}{\sf Todo: [#1]}}

%===================%
%  Label reminders  %
%===================%
% [label=(\roman*)]
% [label=(\alph*)]
% [label=(\arabic{enumi})]

%==================%
%  Other settings  %
%==================%
\pgfdeclarelayer{background}
\pgfsetlayers{background,main}
\tikzset{->-/.style={decoration={
  markings,
  mark=at position .5 with {\arrow{>}}},postaction={decorate}}}

%=================%
%  Title & Index  %
%=================%
\title{Research Statement}
\author{Henry Twiss}
\date{\today}
\makeindex

\begin{document}

\maketitle 

\textbf{Background}: I am intersted in studying the simultaneous non-vanshing of products of elliptic $L$-functions over function fields and twisted by quadratic Dirichlet characters, namely $L(s_{1},E_{1} \x \chi_{d})L(s_{2},E_{2} \x \chi_{d})$, at the central value $s = \frac{1}{2}$. The function field $\F_{q}(t)$ is the field of rational functions in the variable $t$ with coefficients in the finite field $\F_{q}$ of $q$ elements. An elliptic curve $E$ over $\F_{q}(t)$ is the set of solutions $(x,y)$ to the cubic equation
\[
  y^{2} = x^{3}+ax+b,
\]
where $a,b \in \F_{q}(t)$. A quadratic Dirichlet character $\chi_{d}$, with $d \in \F_{q}[t]$, is a $d$-periodic multiplicative function whose non-zero values are $\{\pm1\}$. The elliptic $L$-function $L(s,E \x \chi_{d})$ is a complex function that encodes the arithmetic information about the elliptic curve $E$ and quadratic Dirichlet character $\chi_{d}$ \textit{analytically} into itself. Studying the arithmetic information of $E$ is then amenable to understanding the analytic properties of $L(s,E \x \chi_{d})$. Similary, there are the $L$-functions $L(s,E)$ and $L(s, \chi_{d})$. This idea of passing from arithmetic to analytic investigations has its roots traced back to Dirichlet and the Riemann zeta function. In the setting of elliptic curves, there is the infamous Birch–Swinnerton-Dyer conjecture which claims that the rank of $E$ is equal to the order of vanishing of $L(s,E)$ at the special value $s = 1$. It is expected that any $L$-function only vanishes for either trivial or very good reasons at the special values $s = \frac{1}{2}$ and $s = 1$. Simultaneous non-vanshing results are those which state that many $L$-functions do not vanish at a special value. Complex analytic techniques combined with simultaneous non-vanshing results allow for the analytic non-vanishing information to be translated into arithmetic information leading to often astounding results. For example, non-vanshing at $s = 1$ well-known to be closely connected to primes in arithmetic progressions and prime number theorems (see \cite{M}) while non-vanshing at $s = \frac{1}{2}$ is related to the nonvanishing of theta lifting and the nonexistence of Landau–Siegel zeros (see \todo{cite}). These applications are responsible for simultaneous non-vanshing drawing much attention in recent years.

\textbf{Proposal}: An idea arose in the 1980's that studying an average over a family of $L$-functions would provide insight about each individal $L$-function in the family (see \todo{cite}). The first instance of this construction was to average the family of quadratic Dirichlet $L$-functions into a multiple Dirichlet series roughly of shape
\[
  Z(s,w) = \sum_{d}\frac{L(s,\chi_{d})}{|d|^{w}} = \sum_{m}\frac{L(w,\chi_{m})}{|m|^{s}}.
\]
Variations of this series were studied by \todo{cite}. The multiple Dirichlet series inherits functional equations as $s \to 1-s$ and $w \to 1-w$ from those of $L(s,\chi_{d})$ and $L(w,\chi_{m})$. Composing these functional equations, $Z(s,w)$ satisfies a group of functional equations that is isomorphic to the Weyl group of an $A_{2}$ type root system coming from representation theory. This group is isomorphic to the symmetric group $S_{3}$ (the symmetries of a triangle). As an almost immediate consquence, $Z(s,w)$ admits analytic continuation to $\C^{2}$ with a pole at $w = 1$. Taking the residue of $Z\left(\frac{1}{2},w\right)$ at $w = 1$ and applying a contour integral roughly yields an asymptotic of shape
\[
  \sum_{|d| \ll X}L\left(\frac{1}{2},\chi_{d}\right) = AX\log(X)+BX+o(X),
\]
for some nonzero constants $A$ and $B$. This implies simultaneous non-vanshing for infinitely many quadratic Dirichlet $L$-functions $L(s,\chi_{d})$ at the central value $s = \frac{1}{2}$. A similar procedure can be done for elliptic $L$-functions. \textbf{I will prove simultaneous non-vanshing for the product of two elliptic $L$-functions twisted by quadratic Dirichlet characters over function fields at the central value $s = \frac{1}{2}$. Precisely, for any two elliptic curves $E_{1}$ and $E_{2}$ over $\F_{q}(t)$ there exists infinitely many $d \in \F_{q}[t]$ such that
\[
  L\left(\frac{1}{2},E_{1} \x \chi_{d}\right)L\left(\frac{1}{2},E_{2} \x \chi_{d}\right) \neq 0.
\]}
The underlying idea is to construct a multiple Dirichlet series of shape
\[
  Z(s_{1},s_{3},s_{5}) = \sum_{d}\frac{L(s_{1},E_{1} \x \chi_{d})L(s_{3},E_{2} \x \chi_{d})}{|d|^{s_{5}}},
\]
and show that it admits analytic continuation to a region containing the point $\left(\frac{1}{2},\frac{1}{2},1\right)$ with a pole at $s_{3} = 1$. Similar analytic techniques to those for $Z(s,w)$ can then be applied to obtain the simultaneous non-vanishing. Therefore, the key insight is \textbf{the analytic continuation of $Z(s_{1},s_{2},s_{3})$ to a suitably large enough domain}. It can be shown that $Z(s_{1},s_{2},s_{3})$ exhibits a group of functional equations isomorphic to the symmetric group $S_{4}$. Using these functional equations, $Z(s_{1},s_{2},s_{3})$ exhibits analytic continuation to a tube domain inside $\C^{3}$ but this domain does not contain the point $\left(\frac{1}{2},\frac{1}{2},1\right)$. To overcome this issue, we must exploit the Euler product expressions for elliptic $L$-functions to write the four-fold product
\[
  L(s_{1},E_{1} \x \chi_{d})L(s_{3},E_{2} \x \chi_{d}) = \prod_{p}\prod_{i = 1,3}(1-\a_{i}(p)p^{-s_{i}})^{-1}(1-\b_{i}(p)p^{-s_{i}})^{-1},
\]
for some complex numbers $\a_{i}(p)$ and $\b_{i}(p)$ and over all irreducible moinic polynomials $p$. We now stratify the four-fold product into four different variables to write
\[
  Z(s_{1},\ldots,s_{5}) = \sum_{d}\frac{\prod_{p}\prod_{1 \le i \le 4}(1-\g_{i}(p)p^{-s_{i}})^{-1}}{|d|^{s_{5}}},
\]
where $\g_{i}(p) \in \{\a_{1}(p),\b_{1}(p),\a_{3}(p),\b_{3}(p)\}$. We have
\[
  Z(s_{1},s_{1},s_{3},s_{3},s_{5}) = Z(s_{1},s_{3},s_{5}), 
\]
and so we must continue $Z(s_{1},\ldots,s_{5})$ to a region containing $\left(\frac{1}{2},\frac{1}{2},\frac{1}{2},\frac{1}{2},1\right)$. It is this analytic continuation which can be acomplished. However, the continuation requires an extremely non-trivial argument. The continuation of $Z$ mimics that of the multiple Dirichlet series
\[
  Z_{\chi}(s_{1},\ldots,s_{5}) = \sum_{d}\frac{\prod_{1 \le i \le 4}L(s_{i},\chi_{d})}{|d|^{s_{5}}}.
\]
The group of functional equations posesed by $Z$ and $Z_{\chi}$ is an infinite Weyl group attached to the affine root system $\wtilde{D_{4}}$. As a consequence, the polar lines of these multiple Dirichlet series accumulate along the hyperplane $s_{1}+s_{2}+s_{3}+s_{4}+2s_{5} = -1$ and create a natural barrier to analytic continuation. The point $\left(\frac{1}{2},\frac{1}{2},\frac{1}{2},\frac{1}{2},1\right)$ occures before the natural barrier but after the region of continuation obtained by applying the functional equations. That is, it lies in a \textit{dead zone} where continuation should be possible but cannot be obtained by the functional equations directly. In a \textit{tour-de-force} paper, Diaconu-Pasol-Popa (see \todo{cite}) obtained the analytic continuation of $Z_{\chi}$ to the dead zone by discovering an extra functional equation...

\textbf{Broader Impacts}:

\begin{thebibliography}{}
  \bibitem{M}
  Murty, M. Ram, and V. Kumar Murty. Non-vanishing of L-functions and applications. Springer Science \& Business Media, 2012.
\end{thebibliography}



\end{document}